% !TeX spellcheck = cs_CZ
%================== Kapitola: Zemnění a stínění ====================================================
\setchaptertoc
\chapter{Zemnění a stínění}\label{emc:IchapIII}
  \section{Zemnění}\label{emc:IchapIIIsecI}    
    Zemněním nebo uzemněním se vytváří úmyslné, trvalé nebo dočasné vodivé spojení částí
    elektrických zařízení se zemí. Důvody uzemnění se dělí zejména na \textbf{provozní} a
    \textbf{ochranné}. Dalším dělícím faktorem je pásmo kmitočtů, pro které se zemnění vytváří.
    Podle tohoto faktoru se obvykle hovoří o
    \begin{itemize}[noitemsep]
      \item nízkofrekvenčním (nf) 
      \item vysokofrekvenčním (vf)
    \end{itemize}
    zemnění. NF zemnění je zejména vztahováno k výkonovým zařízením střídavého proudu normalizované
    soustavy 3 x \num{220}/\SI{380}{\V} s kmitočtem \SI{50}{\Hz}. Na tomto kmitočtu a bezprostředně
    následujících normovaných kmitočtech výkonových zařízení \SI{400}{\Hz}, \SI{500}{\Hz} a
    \SI{1}{\kHz}, se u uzemňovacích vodičů a zemničů (kterými je obvykle uzemnění realizováno)
    neprojeví podstatně efekty povrchového jevu a vlnového charakteru procesu vyrovnávání potenciálů
    mezi propojenými částmi. 
    
    U vf zemnění, které je spojováno obvykle s kmitočty nad \SI{10}{\kHz} se však stávají uvedené
    efekty významnými a dominantními. 

    \subsection{Nízkofrekvenční zemnění}\label{emc:IchapIIIsecIssecI}

      \textbf{Pracovní uzemnění} se ve výkonových zařízeních střídavého proudu, zejména u
      normalizované soustavy 3 x \num{220}/\SI{380}{\V}, zřizuje 
      \begin{enumerate}
        \item na ustálení napětí soustaVy proti zemi, 
        \item na ochranu před vniknutím napětí u sítě nad \SI{1000}{\V} do sítě s napětím do
              \SI{1000}{\V},
        \item na ochranu před atmosférickým přepětím.       
      \end{enumerate}

      Požadavky na zemnící soustavu pro nízkofrekvenční zemnění se stanovují obvykle mezními
      hodnotami ceikového odporu uzemnění. Například u bleskojistek nesmí být zemní odpor větší jak
      \SI{15}{\ohm}, při ochraně nulováním smí být celkový odpor uzemnění nejvýše \SI{2}{\ohm}. 

      \textbf{Ochranné uzemnění} zabezpečuje spolehlivé propojení všech neživých kovových částí
      strojů, přístrojů, plášťů, kabelů, armatur izolátorů, průchodek, odpojovačů apod., které sice
      při bezporuchovém stavu proud nevedou, ale na kterých se při poruše objeví napětí, které může
      být nebezpečné. Je nutné zabezpečit, aby odpor uzemnění byl opět co nejmenší. Například v
      sítích s uzemněným uzlem při dobrém uzemnění dojde k bezpečnému odpojení vadného zařízení při
      spojení jedné fáze se zemí. Z hlediska omezení dotykových krokových napětí je nejvýhodnější
      udělat společnou uzemňovací soustavu; pokud je možné dodržovat požadavky na velikost
      dotykových napětí menších než \SI{125}{\V} v elektrických provozovnách a \SI{65}{\V} v
      zařízeních spotřebitelských. Společná uzemňovací soustava zabezpečuje vyrovnání potenciálu a
      zmenšení nebezpečí z dotykového a krokového napětí. 

      V praxi se pro spojení se zemí používají jak strojené zemniče, tak náhodné zemniče. Jako
      nejhospodárnější strojené zemniče jsou kovové pásky, někdy trubky, desky apod. Pro posuzování
      účinnosti nízkofrekvenční zemnící soustavy je rozhodující dosažená hodnota přechodového odporu
      mezi zemničem a zemí (tzv. \emph{zemní odpor}). 

      \begin{figure}[ht!]
        \centering  
        \subcaptionbox{Impedanční charakteristika 
          \label{emc:fig001a}}{\luafigure[1]{emc_fig001a}} \\
        \subcaptionbox{Náhradní obvod propojovacího pásku 
          \label{emc:fig001b}}{\luafigure[0.8]{emc_fig001b}} 
        \caption{Impedanční charakteristiky náhradního obvodu propojovacího pásku}
        \label{emc:fig001}
      \end{figure}

      Zkušenosti s realizovanými zemnicími soustavami ukazují příznivý vliv obvodových zemničů a
      mřížových sítí; například zhušťováním sítě se podstatně snižují povrchová kroková a dotyková
      napětí. Bylo zjištěno, že optimální velikost ok sítě je 4x4 až 8x\SI{8}{\m}. 

      Náhodné zemniče (například ocelové konstrukce budov, uzemněné kovové pláště kabelů, kovová
      vodovodní potrubí vodivé spojená se zemí apod.) se smí používat pouze jako zemniče pomocné.
      
      Technologie nízkofrekvenčního zemnění musí zabezpečovat životnost a stálost zejména s ohledem
      na omezení vlivu korozního prostředí na kvalitní spojování zemničů a svodů a změny povrchu
      vodivých částí. Je třeba omezit vliv elektrochemické koroze přípustnou volbou kombinací
      spojovaích materiálů. 

      Pro nízkofrekvenční zemnící soustavu je tedy rozhodující hlavně rezistance a neuvádí se
      obvykle požadavky na kmitočtové závislé složky impedance (kapacitance a induktance),jejichž
      velikost lze pro nízké hodnoty pracovních kmitočtů většinou zanedbat. 

      Na obr. \ref{emc:fig001} je demonstrován zjednodušený náhradní obvod reálného propojovacího
      vodiče. Při nízkých kmitočtech lze vliv frekvenčně závislých složek na hodnotu impedance
      zanedbat. V oblasti - vyšších a vysokých kmitočtů toto zanedbání možné není, a proto i řešení
      vysokofrekvenčního zemnění je složitější. K potlačení vlivu kmitočtové závislých složek na
      přípustnou úroveň je obvykle třeba, aby nejvyšší hodnoty kruhové frekvence \(\omega\)
      uvažované části spektra byly podstatně menší než rezonanční \(\omega_0\). Tento požadavek lze
      splnit minimalizaci indukčnosti a kapacity propojovacích pásků, například zmenšováním jejich
      délky a zvětšováním jejich šířky, čímž se zvyšuje \(\omega_0\).

    \subsection{Vysokofrekvenční zemnění}\label{emc:IchapIIIsecIssecII}
      
      Účelem zemnícího (kostřícího) systému adekvátního požadavkům EMC je, populámě řečeno, vytvořit
      účinnou absorpční strukturu pro elektrický náboj měnící se v závislosti na hodnotách
      užitečných a rušivých proudů procházejících přes systém, aniž by přitom docházelo
      (nepřípustným změnám hodnoty opěrného (zemního) potenciálu. V obvodových schématech je tímto
      absorberem obvykle \emph{\uv{zemní bod}}, \emph{\uv{zemní linka}}, \emph{\uv{kostra}} apod.
      Praxe však ukazuje, že interpretace zařízení pomocí obvodových schémat i běžný způsob
      uvažování elcktrotechnických konstruktérů a inženýrů v obvodových schématech neumožňuje
      analýzu EMC. Obvyklý způsob uvažování je orientován na účelové funkce zařízení, zatímco rušivé
      signály se šíří po vodivých strukturách, které v obvodových schématech nejsou zakresleny.
      Proto je výhodné pro EMC analýzu a určení účinnosti koncepce opatření, například zemnění,
      vycházet ze skutečné geometrické struktury nebo z jejího modelu na topologickém prostoru,
      vhodně zobrazujícím podstatné lysy geometrické struktmy. 
      
%---------------------------------------------------------------------------------------------------