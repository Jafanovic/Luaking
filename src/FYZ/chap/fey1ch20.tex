% !TeX program = lualatex
% !TeX root = luaking.tex
% !TeX encoding = UTF-8
% !TeX spellcheck = cs_CZ
%---------------------------------------------------------------------------------------------------
% file fey1ch20.tex
%---------------------------------------------------------------------------------------------------
%=========================== Kapitola: Rotace v prostoru ===========================================
\setchaptertoc
\chapter{Rotace v prostoru}\label{fyz:IchapXX}
  \section{Momenty sil v prostoru}\label{fyz:IchapXXsecI}
    V této kapitole se budeme zabývat jedním z nejpozoruhodnějších a nejzábavnějších důsledků
    mechaniky - chováním rotujícího tělesa. Za tím účelem musíme nejdříve rozšířit matematickou
    formulaci rotačního pohybu, pojmy moment hybnosti, moment síly atd. na trojrozměrný prostor.
    Tyto rovnice nebudeme používat v jejich úplné obecnosti, ani nebudeme studovat všechny jejich
    důsledky, neboť by nám to zabralo mnoho let a my se brzy musíme věnovat jiným tématům. V úvodním
    kurzu můžeme uvést jen základní zákony a aplikovat je na několik mimořádně zajímavých příkladů.

    Nejdříve si všimněme, že máme-li rotaci ve třech rozměrech, ať už jde o tuhé těleso nebo o
    nějaký jiný systém, pak to, co jsme odvodili v dvojrozměrném případě, stále platí. Stále tedy
    platí, že \(xF_y – yF_x\) je moment síly „v rovině \(xy\)“ nebo moment síly „kolem osy \(z\)“.
    Ukáže se i to, že tento moment síly je roven rychlosti změny \(xp_y - yp_x\), neboť kdybychom se
    vrátili k odvození rovnice (\ref{fyz:eq663}) z Newtonových zákonů, viděli bychom, že nemusíme
    předpokládat rovinný pohyb; když diferencujeme \(xp_y – yp_x\), dostaneme \(xF_y – yF_x\), takže
    tato věta stále platí. Veličinu \(xp_y – yp_x\) nazýváme momentem hybnosti příslušejícím rovině
    \(xy\) nebo momentem hybnosti vzhledem k ose \(z\). Protože toto platí, můžeme si vzít jakoukoli
    jinou dvojici souřadnicových os a odvodit další rovnici. Vezměme si rovinu \(yz\) a ze symetrie
    je jasné, že když prostě dosadíme \(y\) za \(x\) a \(z\) za \(y\), dostaneme pro moment síly s
    \(yF_z –  zF_y\) a \(yp_z-zp_y\) bude moment hybnosti spojený s rovinou \(yz\). Samozřejmě,
    mohli bychom vzít ještě jinou rovinu, rovinu \(zx\), a pro ni bychom dostali
    \begin{equation*}
      zF_x - xF_z = \der{ }{dt}(zp_x - xp_z).
    \end{equation*}

    je zcela jasné, že tyto tři rovnice lze odvodit pro pohyb jedné částice. Navíc, kdybychom výrazy
    jako \(xp_y - yp_x\) sečetli pro mnoho částic a součet nazvali celkovým momentem hybnosti, měli
    bychom tři výrazy pro tři roviny \(xy\), \(yz\) a \(zx\). Kdybychom totéž provedli i se silami,
    mohli bychom hovořit o momentu síly v rovině \(xy\), \(yz\) i \(zx\). Dostali bychom poznatek,
    že vnější moment síly příslušející kterékoli rovině je roven rychlosti změny momentu hybnosti
    příslušejícímu této rovině. Toto je zobecnění našich poznatků o dvojrozměrném případu.

    Někdo by ale mohl říci: „Ale existuje více rovin! Nemůžeme vzít nějakou jinou rovinu pod nějakým
    jiným úhlem a vypočítat moment síly v této rovině? Protože bychom pro každou takovou rovinu
    museli napsat další sérii rovnic, měli bychom velmi mnoho rovnic!“ Kupodivu se ukazuje, že
    stačí, když v nějaké jiné rovině změříme \(x'\), \(F_{y'}\) atd. a vypočteme kombinaci
    \(x'F_{y'}-y'F_{x'}\), pak lze výsledek napsat jako \emph{kombinace} tří výrazů v rovinách
    \(xy\), \(yz\) a \(zx\). To není nic nového. Jìnými slovy známe-li tři momenty sil v rovinách
    \(xy\), \(yz\), \(zx\), potom lze moment sílyv libovolné rovině a příslušný moment hybnosti
    napsat jakojejich kombinace: 6 procent zjednoho, 92 procent z druhého atd. Tuto vlastnost si
    nyní rozebereme.

    Předpokládejme, že Petr určil všechny momenty síly a všechny momenty hybnosti v příslušných
    rovinách v souřadnìcích \(xyz\), ale Pavel má souřadnicové osy \(x'\), \(y'\), \(z'\), jež mají
    jiný směr. Abychom si to trochu zjednodušili, budeme předpokládat, že se pootočily jen osy \(x\)
    a \(y\). Pavlovy \(x'\) a \(y'\) jsou nové, ale \(z'\) je stejné jako \(z\). Má tedy nové roviny
    \(yz\) a \(zx\). Proto jeho momenty síly a momenty hybnosti jsou jiné. Například jeho moment
    síly v rovině \(x'y'\) bude \(x'F_{y'}-y'F_{x'}\), atd. Nyní musíme najít vztah mezi novými a
    starými momenty síly, tj. najít spojení mezi oběma souřadnicovými soustavami. Někdo možná
    poznamená: „To vypadá podobně, jako to, co jsme dělali s vektory.“ Ano, přesně to chceme
    provést. Ale pak může říci: „A není moment síly vlastně vektor?“ \emph{Ukáže se}, že je to
    vektor, ale to nemůžeme vědět dříve, než provedeme jeho analýzu. Každý krok nebudeme provádět
    podrobně, chceme jen naznačit, jak se do dělá. Momenty sil, které vypočítá Petr, jsou
    \begin{subequations}\label{fyz:eq710}
      \begin{align}
        N_{xy} &=  xF_y - yF_x, \label{fyz:eq710a}  \\
        N_{yz} &=  yF_z - zF_y, \label{fyz:eq710b}  \\
        N_{zx} &=  zF_x - xF_z. \label{fyz:eq710c}        
      \end{align}
    \end{subequations}

    Poznamenejme, že nedáme-li dostatečný pozor na souřadnice, můžeme v takovýchto výrazech dostat
    nesprávné znaménko. Proč nepíšeme \(N_{yz} = zF_y - yF_z\)? Souvisí to se skutečností, že
    souřadnicová soustava může být buď pravotočivá, nebo levotočivá. Zvolíme-li si (libovolně)
    znaménko pro \(N_{xy}\), můžeme správné vyjádření dalších dvou veličin vždy najít záměnou
    \(xyz\) v jednom z pořadí
    \begin{equation*}
      x \rightarrow y \rightarrow z \rightarrow x
    \end{equation*}
    nebo
    \begin{equation*}
      x \rightarrow z \rightarrow y \rightarrow x.
    \end{equation*}
    Pavel ve své soustavě vypočítá takovéto momenty sil
    \begin{subequations}\label{fyz:eq711}
      \begin{align}
        N_{x'y'} &=  x'F_{y'} - y'F_{x'}, \label{fyz:eq711a}  \\
        N_{y'z'} &=  y'F_{z'} - z'F_{y'}, \label{fyz:eq711b}  \\
        N_{z'x'} &=  z'F_{x'} - x'F_{z'}. \label{fyz:eq711c}        
      \end{align}
    \end{subequations} 
    Předpokládáme, že jedna souřadnicová soustava je vzhledem k druhé pootočena o pevný úhel
    \(\Theta\), přičemž osy \(z\) a \(z'\) jsou totožné. (Tento úhel \(\Theta\) nemá nic společného
    s rotací předmětů nebo s tím, co se děje v dané souřadnicové soustavě. Určuje jen vztah mezi
    souřadnicovými osamì jednoho a souřadnicovými osamì druhého pozorovatele, přičemž předpokládáme,
    že je konstantní.) Souřadnice v těchto dvou systémech souvisí navzájem takto:
    \begin{subequations}\label{fyz:eq712}
      \begin{align}
        x' &=  x\cos\Theta + y\sin\Theta, \label{fyz:eq712a}  \\
        y' &=  y\cos\Theta - x\sin\Theta, \label{fyz:eq712b}  \\
        z' &=  z.                         \label{fyz:eq712c}        
      \end{align}
    \end{subequations} 
    Podobně, protože síla je vektor, transformuje se do nového systému stejně jako \(x\), \(y\),
    \(z\), protože veličina je vektorem tehdy a jen tehdy, kdy její různé složky se transformují
    jako \(x\), \(y\), \(z\)
    \begin{subequations}\label{fyz:eq713}
      \begin{align}
        F_{x'} &=  F_x\cos\Theta + F_y\sin\Theta, \label{fyz:eq713a}  \\
        F_{y'} &=  F_y\cos\Theta - F_x\sin\Theta, \label{fyz:eq713b}  \\
        F_{z'} &=  F_z.                           \label{fyz:eq713c}        
      \end{align}
    \end{subequations} 
    Dosazením do (\ref{fyz:eq711}) za \(x'\), \(y'\), \(z'\) z (\ref{fyz:eq712}) a za \(F_{x'}\),
    \(F_{y'}\), \(F_{z'}\), z (\ref{fyz:eq713}) můžeme nyní snadno zjistit, jak se transformují
    momenty síly. Tak dostáváme dlouhý výraz pro \(N_{x'y'}\), v němž se ukáže na první pohled
    překvapující skutečnost, že se zredukuje na \(xF_y - yF_x\), v čemž poznáváme moment síly v
    rovině \(xy\)
    \begin{align}\label{fyz:eq714}
      N_{x′y′}  &=(x\cosθ+y\sinθ)(F_y\cosθ−F_x\sinθ)− (y\cosθ−x\sinθ)(F_x\cosθ+F_y\sinθ) \nonumber
                &=xF_y(\cos2θ+\sin2θ)−yF_x(\sin2θ+\cos2θ)+ xF_x(−\sinθ\cosθ+\sinθ\cosθ)+ \nonumber
                  yF_y(\sinθ\cosθ−\sinθ\cosθ)
                &=xF_y−yF_x=N_{xy}.
    \end{align}
    Tento výsledek je jasný, neboť pootočíme-li naše souřadnice jen v rovině, pak otáčení kolem osy
    \(z\) se nezmění, neboť ani rovina se nezměnila! Zajímavější bude sledování výrazu \(N_{y′z′}\)
    neboť tu jde o novou rovinu. To, co jsme dělali s rovinou \(x'y'\), proveďme nyní s rovinou
    \(y'z'\). Dostáváme
    \begin{align}\label{fyz:eq715}
      N_{y′z′}  &=(y\cosθ−x\sinθ)F_z− z(F_y\cosθ−F_x\sinθ)  \nonumber
                &=(yF_z−zF_y)\cosθ+(zF_x−xF_z)\sinθ         \nonumber
                &=N_{yz}\cosθ+N_{zx}\sinθ.
    \end{align}
    Nakonec si zopakujme totéž pro rovinu \(z'x'\)
    \begin{align}\label{fyz:eq716}
      N_{z′x′}  &=z(F_x\cosθ+F_y\sinθ)−(x\cosθ+y\sinθ)F_z
                &=(zF_x−xF_z)\cosθ−(yF_z−zF_y)\sinθ
                &=N_{zx}\cosθ−N_{yz}\sinθ.
    \end{align}
    Chtěli jsme odvodit pravidlo,jak najít momenty síly v nových souřadnicích a už ho máme. Jak si
    lze toto pravidlo zapamatovat? Když si pozorně prohlédneme vztahy (\ref{fyz:eq714}),
    (\ref{fyz:eq715}) a (\ref{fyz:eq716}), vidíme, že mezi nimi a vztahy pro \(x\), \(y\) a \(z\)
    existuje úzká souvislost. Kdybychom \(N_{xy}\) označili jako \(z\)-ovou složku \(N\), pak
    (\ref{fyz:eq715}) bychom mohli chápat jako vektorovou transformaci, neboť \(z\)-ová složka se
    nezmění a tak to má být. Dáme-li podobně do souvislosti rovinu \(yz\) a \(x\)-ovou složku našeho
    nového vektoru a rovinu \(zx\) s \(y\)-ovou složkou tohoto vektoru, pak tyto výrazy budou mít
    tvar
    \begin{subequations}\label{fyz:eq717}
      \begin{align}
        N_{z'} &=  N_z,                           \label{fyz:eq717a}  \\
        N_{x'} &=  N_x\cos\Theta + N_y\sin\Theta, \label{fyz:eq717b}  \\
        N_{y'} &=  N_y\cos\Theta - N_x\sin\Theta. \label{fyz:eq717c}        
      \end{align}
    \end{subequations} 
    což je právě transformační pravidlo vektorů!

    Dokázali jsme, že kombinaci \(xF_y - yF_x\) můžeme ztotožnit s tím, čemu běžně říkáme \(z\)-ová
    souřadnice určitého uměle zavedeného vektoru. Ačkoli moment síly způsobuje otáčení v rovině, a
    \emph{apriori} nemá vektorový charakter, matematicky se chová jako vektor. Tento vektor je kolmý
    k rovině otáčení a jeho délka je úměrná točivé síle. Tři složky takovéto veličiny se
    transformují jako skutečný vektor.

    Moment síly tedy reprezentujeme vektorem. Každé rovině, v níž moment síly působí, přiřadíme
    kolmici. Samotná kolmice ještě nespecifikuje znaménko směru kolmice. Proto musíme zavést
    pravidlo, které nám řekne, že působil-li moment síly v rovině \(xy\) v určitém smyslu, směr
    kolmice, kterou jí chceme přiřadit, je totožný s kladným směrem osy \(z\), tj. musíme definovat,
    co je \uv{pravé} a co \uv{levé}. Předpokládáme-li, že souřadnicová soustava \(x\), \(y\), \(z\)
    je pravotočivá, bude pravidlo následující: Představíme-li si otáčení tak, že otáčíme šroubem s
    pravotočìvým závitem, směr vektoru, který přiřazujeme tomuto otáčení, bude ve směru pohybu
    šroubu.

    Proč je moment síly vektor? Je to šťastná shoda okolností, že rovině můžeme přiřadit jednu osu,
    a tak momentu síly vektor. Je to zvláštnost trojrozměmého prostoru. V dvojrozměmém prostoru je
    moment síly obyčejný skalár a není třeba, abychom mu přiřazovali směr. V případě tří rozměrů je
    to vektor. Kdybychom měli čtyři rozměry, dostali bychom se do velkých problémů, neboť (kdybychom
    jako čtvrtý rozměr měli například čas) bychom neměli jen roviny \(xy\), \(yz\) a \(zx\), ale i
    \(tx\), \(ty\) a \(tz\). Bylo byjich šest a šest veličin nemůže reprezentovat jeden čtyřrozměrný
    vektor.

    Delší dobu se budeme zabývat trojrozměmým prostorem, a proto bude dobré, když si všimneme, že
    předcházející matematický popis nezávisel na tom, že \(x\) byla souřadnice polohy a \(F\) síla;
    závisel jen na transformačních zákonech platných pro vektory. Kdybychom tedy místo \(x\) použili
    \(x\)-ovou souřadnici nějakého jiného vektoru, nic by se nezměnilo. Jinými slovy, kdybychom
    vypočítali \(a_xb_y - a_yb_x\), kde \(\vec{a}\) a \(\vec{b}\) jsou vektory, a kdybychom tento
    výraz nazvali \(z\)-ovou složkou nějaké nové veličiny \(c\), pak tyto nové veličiny budou tvořit
    vektor \(\vec{c}\). K popisu vztahu tohoto nového vektoru k vektorům \(\vec{a}\) a \(\vec{b}\)
    potřebujeme matematické označení. K tomuto účelu se zavedlo označení \(\vec{c} = \vec{a} \times
    \vec{b}\). V teorii vektorové algebry máme tedy vedle známého skalárního součinu ještě nový druh
    součinu, tzv. \textbf{vektorový součin}. Mimochodem \(\vec{c} = \vec{a} \times \vec{b}\), je
    totéž, jako kdybychom napsali
    \begin{subequations}\label{fyz:eq718}
      \begin{align}
        cx =a_yb_z−a_zb_y,    \label{fyz:eq718a}  \\
        cy =a_zb_x−a_xb_z,    \label{fyz:eq718b}  \\
        cz =a_xb_y−a_yb_x.    \label{fyz:eq718c}
      \end{align}
    \end{subequations}
    Kdybychom zaměnili pořadí vektorů \(\vec{a}\) a \(\vec{b}\) (\(\vec{a}\) by byl \(\vec{b}\) a
    \(\vec{b}\) by byl \(\vec{a}\)), znaménko vektoru \(\vec{c}\) by se změnilo, neboť např. \(c_z\)
    by bylo rovno \(b_xa_y - b_ya_x\). Vektorový součin proto není podobný obyčejnému násobení \(ab=
    ba\), platí pro něj \(\vec{b}\times\vec{a}=−\vec{a}\times\vec{b}\). Odtud můžeme ihned dokázat,
    že když \(\vec{a} = \vec{b}\), pak vektorový součin je roven nule: \(\vec{a}\times\vec{a} = O\).

    Vektorový součin je velmi důležitý, neboť jím lze popsat vlastnosti rotací. Důležité je i to,
    abychom pochopili, jaký je geometrický vztah mezi vektory \(\vec{a}\), \(\vec{b}\) a
    \(\vec{c}\). Vztah mezi složkami těchto vektorů je dán rovnicemi (\ref{fyz:eq718}) a z toho lze
    určit i jejich geometrickou souvislost. Především zjistíme, že vektor \(\vec{c}\) je kolmý k
    oběma vektorům \(\vec{a}\) i \(\vec{b}\). (Zkusme vypočítat zda \(\vec{c}\cdot\vec{a}\) nebude
    nula.) Dále velikost vektoru \(\vec{c}\) je rovna velikosti \(\vec{a}\) krát velikosti
    \(\vec{b}\) krát sinus úhlu mezi nimi. Jaký směr má vektor \(\vec{c}\)? Představme si, že vektor
    \(\vec{a}\) otočíme do směru vektoru \(\vec{b}\) tak, aby příslušný úhel pootočení byl menší než
    \ang{180}. Pak šroub s pravotočìvým závitem otáčející se takovýmto způsobem bude směřovat ve
    směru \(\vec{c}\). Skutečnost, že hovoříme o pravotočivěm a ne o levotočivém šroubu, je věc
    konvence a neustále nám připomíná, že jsou-li \(\vec{a}\) a \(\vec{b}\) \uv{skutečné} vektory v
    běžném smyslu, pak nový vektor, který jsme sestrojili jako \(\vec{a}\times\vec{b}\), je umělý
    vektor, jenž se trochu liší od vektorů \( \vec{a}\) a \(\vec{b}\), neboť jsme ho zkonstruovali
    podle zvláštního pravidla. Pro obyčejné vektory \(\vec{a}\) a \(\vec{b}\) máme zvláštní název,
    říkáme jim \textbf{polární vektory}. Příkladem takových vektorů je polohový vektor \(\vec{r}\),
    síla \(\vec{F}\), hybnost \(\vec{p}\), rychlost \(\vec{v}\), intenzita elektrického pole
    \(\vec{E}\) atd. Vektory, v jejichž definici je jeden vektorový součin, se nazývají
    \textbf{axiálnímy vektory} nebo \textbf{pseudovektory}. Příkladem pseudovektorů jsou
    (samozřejmě) moment síly \(\vec{N}\) a moment hybnosti \(\vec{L}\). Úhlová rychlost
    \(\vec{\oemga}\) je rovněž pseudovektor, podobně jako intenzita magnetického pole \(\vec{B}\).

    Abychom dokončili studium matematických vlastností vektorů, uveďme pravidla jejich
    násobení pomocí skalárních a vektorových součinů. V této chvíli z toho pro naše aplikace
    budeme potřebovat velmi málo, ale z důvodu úplnosti uvedeme všechny vzorce násobení
    vektorů, takže později je budeme moci použít. Jsou to
    \begin{subequations}\label{fyz:eq719}
      \begin{align}
         \vec{a}\times(\vec{b}+\vec{c})&= %
            \vec{a}\times\vec{b}+\vec{a}\times\vec{c}                   \label{fyz:eq719a}  \\             
        (\alpha\vec{a})\times\vec{b}&= %
            \alpha(\vec{a}\times\vec{b})                               \label{fyz:eq719b}  \\     
         \vec{a}\cdot(\vec{b}\times\vec{c})&= %
            (\vec{a}\times\vec{b})\cdot\vec{c}                          \label{fyz:eq719c}  \\  
         \vec{a}\times(\vec{b}\times\vec{c})&= %
            \vec{b}(\vec{a}\cdot\vec{c})-\vec{c}(\vec{a}\cdot\vec{b})   \label{fyz:eq719d}  \\          
         \vec{a}\times\vec{a}&=0                                        \label{fyz:eq719e}  \\  
         \vec{a}\cdot(\vec{a}\times\vec{b})=0.                          \label{fyz:eq719f}
      \end{align}
    \end{subequations}

  \section{Rovnice rotace a vektorový součin}\label{fyz:IchapXXsecII}
    je vůbec možné napsat nějaké fyzikální rovnice pomocí vektorového součinu? Takových rovnic je
    velmi mnoho. Například, hned vidíme, že moment síly je roven vektorovému součinu polohového
    vektoru a vektoru síly
    \begin{equation}\label{fyz:eq720}
      \vec{N} = \vec{r}\times\vec{F}.
    \end{equation}
    To je vektorový zápis tří rovnic \(N_x = yF_z - zF_y\), atd. Podobně moment hybnosti jedné
    částice je roven vektorovému součinu vzdálenosti od počátku a hybnosti
    \begin{equation}\label{fyz:eq721}
      \vec{L} = \vec{r}\times\vec{p}.
    \end{equation}
    Pro rotace v trojrozměrném prostoru platí jako analog Newtonova zákona \(\vec{F} =
    \der{\vec{p}}{t}\) tvrzení, že vektor momentu síly je roven časové změně vektoru momentu hybnosti
    \begin{equation}\label{fyz:eq722}
      \vec{N} = \der{\vec{L}}{t}.
    \end{equation} 
    Sečteme-li (\ref{fyz:eq722}) pro mnoho částic, je moment vnějších sil působící na celý systém
    roven změně celkového momentu hybnosti
    \begin{equation}\label{fyz:eq723}
      \vec{N}_{ext} = \der{\vec{L}_{tot}}{t}.
    \end{equation}   
    Odtud plyne: Je-li celkový moment vnějších sil roven nule, je vektor celkového momentu hybnosti
    soustavy konstantní. Platí tedy \emph{zákon zachování momentu hybnosti}. Nepůsobí-li na daný
    systém moment síly, jeho moment hybnosti se nemůže změnit.

  \section{Setrvačník}\label{fyz:IchapXXsecIII}
    \begin{figure}[ht!] %\ref{fyz:fig406}
      \centering
      \subcaptionbox{\label{fyz:fig406a}}{\luafigure[0.45]{fyz_fig406a.pdf}} 
      \subcaptionbox{\label{fyz:fig406b}}{\luafigure[0.45]{fyz_fig406b.pdf}}
      \caption{a) osa má horizontální směr; moment hybnosti vzhledem k vertikální ose je roven nule 
              b) osa má vertikální směr; moment hybnosti vzhledem k vertikální ose je stále roven 
              nule; člověk a žídle se otáčejí v opačném směru než kolo
              (\cite[s.~278]{Feynman01}).}
      \label{fyz:fig406}
    \end{figure}

  \section{Moment hybnosti tuhého tělesa}\label{fyz:IchapXXsecIV}
  \section{Příklady a cvičení}\label{fyz:IchapXXsecV}  



  \begin{figure}[ht!] %\ref{fyz:fig407}
    \centering
    \includegraphics[width=0.3\linewidth]{fyz_fig407.pdf}
    \caption{Setrvačník
             (\cite[s.~279]{Feynman01})}
    \label{fyz:fig407}
  \end{figure}

  \begin{figure}[ht!] %\ref{fyz:fig408}
    \centering
    \includegraphics[width=0.3\linewidth]{fyz_fig408.pdf}
    \caption{Rychle se otáčející káča. Směr vektoru momentu síly je směrem precese 
             (\cite[s.~280]{Feynman01})}
    \label{fyz:fig408}
  \end{figure}

  \begin{figure}[ht!] %\ref{fyz:fig409}
    \centering
    \includegraphics[width=0.3\linewidth]{fyz_fig409.pdf}
    \caption{Při otáčení osy vykonávají částice rotujícího kola (obr. \ref{fyz:fig407}) pohyb, 
             který se děje po zakřivených trajektoriích 
             (\cite[s.~280]{Feynman01})}
    \label{fyz:fig409}
  \end{figure}

  \begin{figure}[ht!] %\ref{fyz:fig410}
    \centering
    \includegraphics[width=0.3\linewidth]{fyz_fig410.pdf}
    \caption{Skutečný pohyb konce setrvačníku pod vlivem tíhy ihned po uvolnění předtím uchycené osy
             (\cite[s.~281]{Feynman01})}
    \label{fyz:fig410}
  \end{figure}

  \begin{figure}[ht!] %\ref{fyz:fig411}
    \centering
    \includegraphics[width=0.3\linewidth]{fyz_fig411.pdf}
    \caption{Moment hybnosti rotujícího tělesa není nutně rovnoběný s úhlovou rychlostí
             (\cite[s.~282]{Feynman01})}
    \label{fyz:fig411}
  \end{figure}

  \begin{figure}[ht!] %\ref{fyz:fig412}
    \centering
    \includegraphics[width=0.3\linewidth]{fyz_fig412.pdf}
    \caption{Úhlová rychlost a moment hybnosti tuhého tělesa (\(A>B>C\))
             (\cite[s.~283]{Feynman01})}
    \label{fyz:fig412}
  \end{figure}

  \todo[inline]{Kapitola fey1ch20 je zcela prázdná, pouze obrázky}  
%} %tikzset
%---------------------------------------------------------------------------------------------------