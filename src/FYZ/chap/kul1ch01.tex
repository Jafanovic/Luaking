% !TeX program = lualatex
% !TeX root = luaking.tex
% !TeX encoding = UTF-8
% !TeX spellcheck = cs_CZ
%---------------------------------------------------------------------------------------------------
% file kul1ch01.tex
\graphicspath{{../src/FYZ/img/}}
%---------------------------------------------------------------------------------------------------
%=============================== Kapitola: Astrofyzika =============================================
\setchaptertoc
\chapter{Úvod}
  \wikiAstrofyzika  je vědní obor ležící na rozhraní \emph{fyziky} a \emph{astronomie}. Zabývá se 
  fyzikou vesmíru, včetně fyzikálních vlastností (svítivost, hustota, teplota, chemické složení) 
  astronomických objektů jako jsou hvězdy, galaxie a mezihvězdná hmota, jakož i jejich vzájemné 
  působení.

  \begin{figure*}
    \centering
    \luafigure[1]{fyz_fig0923a.jpg}
    \caption{Moderní barevného provedení Flammarionovy rytiny. Výklad rytiny byl použit pro
             animované sekvence o kosmologické vizi Giordana Bruna v premiéře televizního seriálu
             \uv{Cosmos: A Spacetime Odyssey}, pořádané astrofyzikem Neilem deGrasse Tysonem.
             Televizní seriál byl věnován popularizaci vědy a astronomii, stejně jako Flammarionova
             vlastní práce před 150 lety.}
    \label{fyz:fig0923}
  \end{figure*}
  
  Podle metod výzkumu těchto objektů se dělí na \emph{fotometrii}, \emph{spektroskopii},
  \emph{radioastronomii}, \emph{astrofyziku rentgenovou}, \emph{infračervenou}, 
  \emph{ultrafialovou} a \emph{neutrinovou}. Každý z těchto podoborů se dále dělí na praktickou a 
  teoretickou část. Praktická získává potřebná data. Teoretická s pomocí fyzikálních zákonů 
  vysvětluje pozorované cho\-vá\-ní vesmírných těles.
  
  \section{Historie astrofyziky}
  
  \section{Základní vztahy}
    \begin{itemize}
      \item \wikiAU - \emph{astronomická jednotka}: průměrná vzdálenost Země od Slunce, 
      $\qty{150e6}{\km}$. Vzájemné vzdálenosti planet či jiných objektů sluneční soustavy 
      vy\-já\-dře\-né v AU poskytují relativně názorné měřítko vzdáleností těchto objektů od sebe. 
      Přesná hodnota je
      \begin{equation*}
        1 AU = \SI[multi-part-units = single]{149597870691(6)}{\m}
      \end{equation*}
      Kvůli vyšší přesnosti \emph{Mezinárodní astronomická unie} (International Astronomical 
      Union, IAU) přijala novou de\-fi\-ni\-ci, podle které je AU délka poloměru nerušené oběžné 
      kruhové dráhy tělesa se zanedbatelnou hmotností, pohybujícího se okolo Slunce rychlostí 
      \newline \(\num{0,017202098950}\) radiánů za den (\(\qty{86400}{\second}\)). 
        \begin{itemize}
          \item Vzdálenost Země od Slunce je \qty{1.00(2)}{\AU}.
          \item Měsíc obíhá kolem Země ve vzdálenosti\newline \qty{0,0026(1)}{\AU}.
          \item Mars je od Slunce vzdálen \qty{1.52(14)}{\AU}.
          \item Jupiter je od Slunce vzdálen \qty{5.20(5)}{\AU}.
          \item Nejvzdálenější člověkem vyrobené těleso, sonda \newline Voyager 1, bylo 31.
                prosince 2007 ve vzdálenosti \qty{104.93}{\AU} od Slunce.
          \item Průměr sluneční soustavy bez \emph{Oortova oblaku} je přibližně \qty{105}{\AU}.
          \item Průměr sluneční soustavy s Oortovým oblakem se odhaduje na \qty{50000}{\AU} až    
                \qty{100000}{\AU}.
          \item Nejbližší hvězda (po Slunci), \emph{Proxima Centauri}, se nachází přibližně ve 
                vzdálenosti \qty{268000}{\AU}.
          \item Průměr hvězdy Betelgeuze je \qty{2.57}{\AU}.
          \item Vzdálenost Slunce od středu Galaxie je přibližně \qty{1.7e9}{\AU}.
          \item Velikost viditelného vesmíru je asi \qty{8.66e14}{\AU}.
        \end{itemize}
      \item \textbf{\unit{\lightyear}} - \emph{světelný rok}: vzdálenost, kterou světlo ulétna za    
            jeden rok, \qty{9.46e12}{\km},
      \item \textbf{\unit{\parsec}} - \emph{parsek, paralaktická sekunda}: vzdálenost, ze které by  
            poloměr oběžné dráhy Země byl kolmo k zornému paprsku vidět pod úhlem 
            \qty{1}{\arcsecond}, 
            \qty{30.9e12}{\km}. 
    \end{itemize}