% !TeX program = lualatex
% !TeX root = luaking.tex
% !TeX encoding = UTF-8
% !TeX spellcheck = cs_CZ
%---------------------------------------------------------------------------------------------------
% file fey1ch01_02_03.tex
%---------------------------------------------------------------------------------------------------
%===================== Kapitola: Základy fyziky ====================================================
\setchaptertoc
\chapter{Základy fyziky}\label{fyz:IchapI}
\epigraph{\emph{Fyzika je jako sex, může přinést praktické výsledky, ale to není důvod, proč to 
  děláme.}}{Richard P. Feynmann}

  \begin{figure}[ht!]  % \ref{fyz:fig067}
    \centering
    \luafigure[0.6]{fyz_fig067.jpg}
    \caption{\wikiFeynman \textasteriskcentered 11. května 1918 - \textdagger 15. února 1988, 
             americký fyzik, který patřil k největším fyzikům 20. století}
    \label{fyz:fig067}
  \end{figure} 

    Možná se zeptáte, zda není možné při vyučování fyziky na první straně uvést základní zákony a 
    potom ukázat, co z nich vyplývá v nejrůznějších situacích. Tak se postupuje v euklidovské 
    geometrii, kde se postulují axiomy, ze kterých se odvodí všechny možné závěry. (Protože se vám 
    nelíbí pětileté studium fyziky, chtěli byste seji naučit za pět minut?) Takovýmto způsobem však 
    nemůžeme postupovat ze dvou důvodů. Především, zatím neznáme všechny základní zákony - oblast 
    toho, co bychom ještě měli poznat, se nám stále zvětšuje. Dále, přesná formulace fyzikálních 
    zákonů zahrnuje mnoho neobvyklých myšlenek, jejichž vyjádření si vyžaduje vyšší matematiku. 
    Proto je nutná značná předběžná příprava jen k tomu, abychom rozuměli, co znamenají slova. Není 
    tedy možné postupovat tímto způsobem. \emph{Učit se můžeme pouze postupně, kousek po kousku}.

  \section{Jak studovat fyziku}\label{fyz:IchapIsecI}
    Každý kousek nebo část celku, který představuje příroda, je vždy jen přiblížením k úplné 
    pravdě; přesněji k úplné pravdě, pokud ji známe. Ve skutečnosti vše, co víme, je jen určitým 
    druhem aproximace, protože víme, ze ještě neznáme všechny zákony. Proto se věci musíme učit jen 
    proto, abychom se je znovu odnaučili, nebo, což je pravděpodobnější, abychom si naše znalosti o 
    nich opravovali.
    
    Princip vědy, téměř její definice, je následující: Prověrkou všech našich vědomostí je 
    experiment. Experiment je jediné kritérium vědecké „pravdy“. Jenže co je zdrojem našich 
    vědomostí? Odkud pocházejí zákony, které prověřujeme? Samotný experiment nám pomáhá odvozovat 
    zákony v tom smyslu, že nám poskytuje náznaky, pokyny. Navíc je však potřebná představivost, 
    aby z těchto náznaků mohla vzniknout velká zobecnění - abychom v nich odhadli nádherný, 
    jednoduchý, ale neobyčejný obraz a potom experimentem prověřili správnost našeho odhadu. Tento 
    proces představivosti je tak těžký, že si fyzici rozdělili práci - teoretičtí fyzici 
    představivostí, dedukcí a odhadem odvozují nové zákony, ale neexperimentují; experimentální 
    fyzici dělají pokusy a přitom také uplatňují představivost, dedukci a odhad.
    
    Řekli jsme, že přírodní zákony jsou přibližné: že nejdříve nacházíme „nesprávné“ a až potom 
    „správné“. Jak však může být experiment „nesprávný“? Především z velmi jednoduchého důvodu - 
    náš přístroj není v pořádku a my jsme to nezpozorovali. Takové chyby se však zjišťují lehce. 
    Odhlédneme-li od těchto drobností, jak může být výsledek experimentu nesprávný? Jen v důsledku 
    nepřesnosti. Například, hmotnost předmětu se zdá být neměnná; rotující káča má stejnou hmotnost 
    jako káča v klidu. Tak byl objeven „zákon“: hmotnost je konstantní, nezávislá na rychlosti. O 
    tomto „zákonu“ se zjistilo, že je nesprávný. Ukázalo se, že hmotnost roste s rychlostí, ale k 
    značnému růstu jsou potřebné rychlosti blízké rychlosti světla. Správný zákon zní: je-li 
    rychlost tělesa menší než \SI{100}{\km\per\second}, je hmotnost konstantní s přesností na jednu 
    milióntinu. V takové aproximativní podobě je tento zákon správný. Někdo by si mohl myslet, že 
    prakticky není rozdíl mezi starým a novým zákonem. To je i není pravda. Pro běžné rychlosti je 
    jistě možné zapomenout na to, o čem jsme mluvili a používat jednoduchý zákon konstantní 
    hmotnosti jako dobré přiblížení. Při velkých rychlostech se však dopustíme chyby, a to tím 
    větší, čím větší je rychlost

    \begin{figure}[ht!]  %\ref{fyz:fig891}
      \centering
      \luafigure[0.8]{fyz_fig891.jpg}
      \caption{Vodní kapka dopadající na hladinu. Kredit: Wikipedia}
      \label{fyz:fig891}
    \end{figure} 
    
    Ostatně, nejzajímavější je to, že z filozofického hlediska je tento aproximativní zákon zcela 
    nesprávný. Náš celkový obraz o světě musíme změnit, i kdyby se hmotnost měnila jen nepatrně. 
    Toto je svérázný znak filozofie nebo myšlenek stojících v pozadí zákonů. Někdy i velmi malý 
    efekt vyžaduje hlubokou změnu našich názorů.
    
    Čemu tedy máme dát přednost? Máme podat správné, ale nezvyklé zákony s jejich cizím a obtížným 
    pojetím jako je například teorie relativity, čtyřrozměrný prostoročas a podobně? Nebo máme 
    nejdříve vysvětlit jednoduchý zákon „konstantní hmotnosti“, který je pouze přibližný, ale 
    nevyžaduje náročné představy? První způsob je více vzrušující, nádhernější a zábavnější, ale s 
    druhým se jednodušeji začíná a představuje první krok ke skutečnému porozumění správného 
    zákona. Tento problém se vždy znovu objevuje při vyučování fyziky. V různých etapách ho musíme 
    řešit různými způsoby, ale vždy je vhodné se zajímat, do jaké míry je přesné to, co teď víme, 
    jak to souvisí s dalším a jak se to může změnit, budeme-li vědět víc.
    
    Nyní přejděme k náčrtu nebo k všeobecné mapě našeho chápání současné vědy (zejména fyziky, ale 
    i jiných věd, které s ní souvisejí). Když se později soustředíme na konkrétní problém, budeme 
    mít představu o jeho pozadí, o tom, proč je zajímavý a jak zapadá do celkové struktury. Jaký je 
    tedy náš celkový obraz světa? \cite[s.~16]{Feynman01}
    
    \subsection{Látka se stává z atomů}
      Kdyby při nějaké katastrofě zanikly všechny vědecké poznatky a dalším generacím by měla 
      zůstat jen jediná věta, které tvrzení by při nejmenším počtu slov obsahovalo nejbohatší 
      informaci? Takovým kandidátem je \textbf{atomová hypotéza} - tj. že \emph{všechny věci se 
      skládají z atomů malých částic, jež jsou v neustálém pohybu,  a vzájemně se přitahují, když 
      jsou od sebe trochu vzdálené, ale odpuzují se, když jsou těsně u sebe.} V této jediné větě, 
      jak uvidíme, je obsaženo nesmírné množství informací o světě. Je k tomu třeba jen trochu 
      představivosti a uvažování.

      \begin{figure}[ht!]  %\ref{fyz:fig007}
        \centering
        \luafigure[0.6]{fyz_fig007.pdf}
        \caption{Vodní kapka z obrázku \ref{fyz:fig891} zvětšená miliardkrát \cite[s.~17]{Feynman01}}
        \label{fyz:fig007}
      \end{figure} 

      Abychom ilustrovali sílu myšlenky o atomu, představme si podobnou kapku vody jako na obrázku
      \ref{fyz:fig891} o rozměru \SI{0.5}{\cm}. Podí\-váme-li se na ni zblízka, neuvidíme nic
      jiného, než vodu - klidnou, souvislou vodu. I když kapku zvětšíme tím nejlepším optickým
      mikroskopem, přibližně dvoutisíckrát, a kapka bude měřit deset metrů, tedy stejně jako velká
      místnost, i tehdy budeme stále vidět relativně klidnou vodu. Jen tu a tam v ní budou plavat
      jakési malé fotbalové míče. Tyto velmi zajímavé objekty na obrázku \ref{fyz:fig890} jsou
      trepky. Tady se můžeme zastavit a zajímat se o trepky, o jejich třepotající se řasičky, o
      jejich kroutící se těla a nepokračovat ve zvětšování. Nebo můžeme zvětšit trepky tak, abychom
      viděli i do nich). 

      \begin{figure}[ht!]  %\ref{fyz:fig890}
        % http://www.photomacrography.net/forum/viewtopic.php?t=18166
        \centering
        \luafigure[1]{fyz_fig890a.jpg}
        \caption{ Trepka velká \emph{(Paramecium caudatum)} je nálevník běžně se vyskytující v 
                  organicky znečištěných vodách po celém světě. Kredit: Wikipedia}
        \label{fyz:fig890}
      \end{figure} 

      Trepky jsou však předmětem biologie. Proto si jich teď nebudeme všímat, ale zahledíme se ještě
      pozorněji na vodu při dalším dvoutisícinásobném zvětšení. Teď měří kapka vody dvacet kilometrů
      a při pozorném sledování je vidět jakési hemžení - cosi, co už nevypadá klidně, ale připomíná
      dav na fotbalové tribuně při pohledu z velké vzdálenosti. Abychom zjistili, co je to za
      hemžení, zvětšíme kapku ještě 250krát a potom uvidíme něco podobného jako na obr.
      \ref{fyz:fig007}. Tento obrázek představuje vodu při zvětšení miliardkrát, je však v několika
      směrech idealizovaný. Především částice jsou zakreslené zjednodušeně - s ostrými okraji, což
      neodpovídá skutečnosti. Dále kvůli jednoduchosti jsou částice zakreslené v dvojrozměrném
      uspořádání, ačkoli se ve skutečnosti pohybují ve všech třech směrech. Všimněme si, že jsou tam
      dva druhy částic znázorněných kroužky, které představují atomy kyslíku (černé) a vodíku (bílé)
      a že na každý atom kyslíku se vážou dva atomy vodíku. Každá skupinka skládající se z atomu
      kyslíku a dvou atomů vodíku se nazývá \textbf{molekulou}. Obrázek je zjednodušený i v tom, že
      skutečné částice v přírodě se ustavičně kolébají a poskakují, obracejí se a točí jedna okolo
      druhé. Je třeba si to představit spíše jako \emph{dynamický a né jako statický obrázek}. Další
      věcí, kterou není možné vystihnout na obrázku, je skutečnost, že částice „drží pohromadě“ -
      přitahují se, jedna za sebou táhne druhou atd. Je možné říci, že jsou jakoby „slepené
      dohromady“. Na druhé straně se částice netlačí jedna přes druhou. Kdybychom se pokusili
      přitlačit dvě z nich příliš těsně k sobě, odpudily by se.

      Atomy mají poloměr \SI{1e-10}{\m} až \SI{2e-10}{\m}. Jejich velikost si můžeme pamatovat i 
      jinak: zvětšíme-li jablko na velikost Země, budou atomy v jablku tak velké, jak bylo původně 
      jablko.

      Představme si teď tuto velkou kapku vody s jejími hemžícími se částicemi, jež přilnuly k sobě 
      a honí jedna druhou. Voda udržuje svůj objem; nerozpadne se na části díky vzájemné 
      přitažlivosti molekul. Je-li tato kapka na šikmé ploše, kde se může hýbat z místa na místo, 
      voda poteče. Nestane se však, že by jednoduše zmizela. Věci se nerozpadají na části právě 
      díky přitažlivosti molekul. \emph{Hemživý pohyb částic je to, co chápeme jako teplo}: 
      zvýšíme-li teplotu, zvětšíme pohyb. Zahříváme-li vodu, pohyb roste a roste i vzdálenost mezi 
      částicemi, až nastane okamžik, kdy přitažlivost mezi molekulami je už nestačí udržet 
      pohromadě. Částice přestanou být vzájemně svázané a rozlétají se od sebe. Zvyšováním teploty 
      tak získáváme \emph{vodní páru}.  
      
      \begin{figure}[ht!]
        \centering
        \luafigure[0.6]{fyz_fig008.pdf}
        \caption{Pára \cite[s.~18]{Feynman01}}
        \label{fyz:fig008}
      \end{figure}

      Na obr. \ref{fyz:fig008} vidíme páru. V jednom směru tento obrázek páry selhává: při našem 
      zvětšení za normálního atmosférického tlaku připadá jen velmi málo molekul na celý pokoj, 
      takže na tak malém obrázku určitě nebudou tři molekuly. Většina plošek této velikosti nebude 
      obsahovat žádnou molekulu - na našem obrázku jsou náhodou dvě a část z třetí molekuly 
      (abychom tam neměli prázdné místo). V případě páry vidíme podobu molekul jasněji než v 
      případě vody. Pro jednoduchost jsou molekuly zakresleny tak, že atomy vodíku svírají úhel 
      \SI{120}{\degree}. Ve skutečnosti má tento úhel hodnotu \ang[arc-separator = \,]{105;3;} a 
      vzdálenost mezi středem vodíku a středem kyslíku je \SI{9.57e-11}{\m}. Tuto molekulu tedy 
      velmi dobře známe.

      Všimněme si jedné vlastnosti vodní páry nebo jiných plynů. Molekuly budou tím, že se vzdálily 
      jedna od druhé, narážet na stěny. Představme si místnost s určitým počtem (tak kolem sta) 
      neustále poskakujících tenisových míčků. Když míčky narážejí na stěnu, odtlačují ji a stěnu 
      proto musíme upevnit. Plyn působí přerušovanou silou, kterou naše nedokonalé smysly (jejich 
      citlivost nevzrostla miliardkrát) vnímají jako \emph{stálý tlak}. Abychom plyn udrželi, 
      musíme na něj působit tlakem z opačné strany. Obr. \ref{fyz:fig009} znázorňuje běžnou nádobu 
      na udržování plynu, kterou najdeme v každé učebnici: \textbf{válec s pístem}. Teď nám 
      nezáleží na tom, jaký je ve skutečností tvar molekul vody, a proto je kvůli jednoduchostí 
      znázorníme jako tenisové míčky nebo body. Jsou v neustálém pohybu a pohybují se na všechny 
      strany. Na spodek \emph{pístu} jich neustále naráží tolik, že na něj musíme působit určitou 
      silou dolů, aby ho molekuly nevytlačily z válce. Tuto \emph{sílu} nazýváme \textbf{tlakem} 
      (přesněji, \emph{tlak násobený plochou dává sílu}). Je jasné, že síla je úměrná ploše pístu, 
      protože zvětšíme-li plochu a přitom nezměníme počet molekul v kubickém centimetru, pak 
      vzroste počet srážek s pístem tolikrát, kolikrát se zvětšila jeho plocha.

      \begin{figure}[ht!]
        \centering
        \luafigure[0.4]{fyz_fig009.pdf}
        \caption{Píst \cite[s.~18]{Feynman01}}
        \label{fyz:fig009}
      \end{figure}

      Nyní \emph{zdvojnásobme} v této nádobě počet molekul, takže se zdvojnásobí jejich hustota, 
      ale ponechme jim stejnou rychlost, tj. \emph{stejnou teplotu}. Pak můžeme dost přesně říci, 
      že se zdvojnásobil počet srážek a jelikož je každá právě tak \uv{energická} jako dříve, tlak 
      je úměrný hustotě. Uvážíme-li skutečnou povahu meziatomových sil, můžeme očekávat mírný 
      pokles tlaku jako projev zvýšené přitažlivosti mezi atomy a mírný vzrůst související s 
      objemem, který zaujímají. Přesto však, pokud je hustota dostatečně nízká, tj. atomů není 
      příliš mnoho, můžeme s dostatečnou přesností říci, že \emph{tlak je úměrný hustotě}.
      
      Snadno pochopíme i něco jiného. \emph{Zvyšujeme-li teplotu} bez změny hustoty plynu, tj. když 
      zvětšujeme rychlost atomů, co se stane s tlakem? Atomy narážejí do pístu \emph{silněji}, 
      neboť se pohybují rychleji a navíc, narážejí častěji. Proto tlak vzrůstá. Vidíme, jak 
      jednoduché jsou myšlenky atomové teorie.
      
      Podívejme se na jinou situaci. Předpokládejme, že se píst \emph{pohybuje dovnitř}, takže 
      atomy jsou pomalu stlačovány do menšího prostoru. Co se stane, narazí-li atom do pohybujícího 
      se pístu? Je jasné, že při takové srážce \emph{získá rychlost}. Můžeme si to vyzkoušet na 
      ping-pongovém míčku: po úderu pálkou míček odletí od pálky rychleji, než k ní přiletěl. Ve 
      zvláštním případě, není-li atom v pohybu a píst na něj narazí, atom se začne určitě 
      pohybovat. Atomy jsou při návratu od pístu \uv{teplejší}, než byly před nárazem na píst. 
      Proto všechny atomy, které jsou v nádobě, získají na rychlosti. To znamená, že při pomalém 
      \emph{stlačení plynu jeho teplota vzrůstá}. Když plyn pomalu \emph{stlačujeme}, jeho teplota 
      \emph{vzrůstá} a když plyn pomalu \emph{rozpínáme}, jeho teplota \emph{klesá}.
      
      \begin{figure}[ht!]  % \ref{fyz:fig010}
        \centering
        \luafigure[0.6]{fyz_fig010.pdf}
        \caption{Led \cite[s.~19]{Feynman01}}
        \label{fyz:fig010}
      \end{figure}

      Vraťme se k naší kapce vody a podívejme se na ni z jiného pohledu. Snižme teplotu naší kapky.
      Předpokládejme, že hemžení molekul vody postupně slábne. Víme, že mezi atomy působí přitažlivé
      síly, které způsobí, že molekuly už nebudou moci tak snadno pohybovat. Obr. \ref{fyz:fig010}
      znázorňuje, co se stane při velmi nízkých teplotách: molekuly jsou vázány v nové struktuře,
      vytváří se \textbf{led}. Takové schematické znázornění ledu není správné, neboť je
      dvojrozměrné. Situaci však vystihuje kvalitativně. Je pozoruhodné, že každý atom má v této
      látce určité místo. Rozmístíme-li atomy na jednom konci kapky podle určitého pravidla, pak v
      důsledku pevné struktury meziatomových vazeb vznikne určité uspořádání atomů i na druhém konci
      kapky, vzdáleném (v našem měřítku) několik kilometrů. Proto, držíme-li ledový rampouch za
      jeden konec, jeho druhý konec bude při lámání klást odpor, chová se jinak než voda, ve které
      je pravidelná struktura rozrušena intenzívním pohybem atomů v rozličných směrech. Rozdíl mezi
      pevnými látkami a kapalinami spočívá v tom, že atomy pevné látky jsou \emph{uspořádány}
      zvláštním způsobem. Toto uspořádání se nazývá \textbf{krystalická struktura}. I tehdy, kdy jde
      o velmi vzdálené atomy, nepozorujeme nic náhodného v jejich polohách. Poloha atomu na jednom
      konci krystalu je určena polohou atomu na druhém konci, i když se mezi nimi nacházejí miliony
      jiných atomů. Obr. \ref{fyz:fig010} znázorňuje vymyšlené uspořádání ledu a ačkoli správně
      vystihuje mnohé vlastnosti ledu, neodpovídá skutečnému uspořádání. Jedním ze správných rysů je
      existence části \emph{hexagonální symetrie}. Můžeme se o tom přesvědčit: otočíme-li obrázek o
      \SI{120}{\degree}, dostaneme stejné seskupení. Taková symetrie ledu je příčinou šestihranného
      tvaru sněhových vloček. Další informací, kterou je možné vytušit z obrázku \ref{fyz:fig010},
      je \emph{zmenšování objemu ledu při tání}. Znázorněná struktura ledu, stejně tak jako
      skutečná, obsahuje \emph{mnoho dutin}. Když se struktura rozpadne, tyto dutiny mohou být
      \emph{zaplněny} molekulami. \emph{Většina jednoduchých látek, s výjimkou vody a
      liteřiny\footnote{Liteřina či písmovina je slitina používaná v písmolijectví. Její přibližné
      složení je: \SIrange{50}{86}{\percent} olova, \SIrange{3}{20}{\percent} cínu a
      \SIrange{11}{30}{\percent} antimonu. Liteřinu vyvinul v 15. století zakladatel knihtisku
      Johannes Gutenberg pro odlévání tiskařských liter.}, zvětšují při tání svůj objem, neboť atomy
      jsou v pevných krystalech těsně seskupeny a při tání potřebují více prostoru na kmitání}.
      Otevřené struktury se však při tání zhroutí - podobně jako led.
      
      I když má led pevnou krystalickou strukturu, jeho teplota se může měnit - v ledu je zásoba 
      tepla. Chceme-li, můžeme toto množství tepla změnit. Jaké je teplo, které se nachází v ledu? 
      Atomy ledu \emph{nejsou} v klidu, poskakují a kmitají. Ačkoli v krystalu existuje určité 
      uspořádání - struktura - všechny atomy kmitají, „na místě“. Zvyšujeme-li teplotu, budou 
      kmitat se stále větší amplitudou, až opustí svá místa. Tento jev nazýváme \textbf{táním}. 
      Snižujeme-li teplotu, kmity slábnou a při teplotě \emph{absolutní nuly} jsou 
      \emph{nejslabší}, ne však nulové. Toto nejmenší množství pohybu, který přísluší atomům, 
      nestačí na roztání látky - až na jednu výjimku: \emph{hélium}. V héliu se při ochlazování 
      také zpomaluje pohyb atomů na nejmenší možnou míru, ale i při teplotě absolutní nuly brání 
      tento pohyb zmrznutí hélia. Hélium nezmrzne, pokud nevytvoříme tak veliký tlak, abychom atomy 
      stlačili k sobě. Při velkém tlaku můžeme dosáhnout toho, že hélium ztuhne.
    
    \subsection{Atomové procesy}
      Dosud jsme si všímali stavby pevných látek, kapalin a plynů z atomového hlediska. Jenže 
      atomová hypotéza charakterizuje i procesy, a proto si všimněme některých procesů z atomového 
      hlediska. Nejdříve budeme hovořit o procesech, které se odehrávají na povrchu vody. Co se 
      vlastně děje na vodním povrchu? Úlohu si zkomplikujeme - bude tak blíže skutečnosti
      - předpokladem, že nad vodním povrchem se nachází vzduch. Obr. \ref{fyz:fig011} takovou 
      situaci znázorňuje. Tak jako předtím vidíme molekuly vody, které tvoří kapalinu, ale vidíme i 
      povrch vody. Nad povrchem vidíme různé molekuly. Jsou tam především \emph{molekuly vody} v 
      podobě vodní páry, kterou je možné pozorovat vždy nad kapalnou vodou (pára a voda jsou v 
      rovnováze, o které pohovoříme později). Dále tam nalezneme jiné molekuly, dvojice atomů 
      kyslíku tvořící \emph{molekulu kyslíku} a dvojice atomů dusíku tvořící \emph{molekulu 
      dusíku}. Vzduch se skládá téměř výhradně z dusíku, kyslíku, vodní páry a menšího množství 
      oxidu uhličitého, argonu a jiných příměsí. Nad povrchem vody se nachází vzduch - plyn 
      obsahující jisté množství vodní páry. Nyní si všimněme, co se odehrává na obrázku. Molekuly 
      vody se neustále pohybují. Občas některá z molekul, nacházejících se v blízkosti povrchu, 
      naráží na jinou molekulu trochu silněji než obvykle a vyskočí nad povrch. Na obrázku takovýto 
      děj \emph{přímo} neuvidíme, neboť vše je na něm nehybné. Můžeme si však představit, že jedna 
      molekula za druhou v důsledku srážek opouštějí vodu - voda mizí, \emph{vypařuje} se. Když 
      nádobu \emph{přikryjeme}, objevíme po nějakém čase velké množství molekul vody mezi 
      molekulami vzduchu. Čas od času některá z těchto molekul vody vletí zpět do vody a zůstává v 
      ní. To, co jsme považovali za mrtvé a nezajímavé - přikrytý pohár vody, který snad dvacet let 
      stál na jednom místě - v sobě skrývá stále probíhající zajímavý \textbf{dynamický proces}. 
      Náš nedokonalý zrak nepozoruje žádnou změnu, ale při miliardovém zvětšení bychom viděli, jak 
      se vše mění: jedny molekuly opouštějí povrch a druhé se vracejí.
      
      \begin{figure}[ht!]   % \ref{fyz:fig011}
        \centering
        \luafigure[0.7]{fyz_fig011.pdf}
        \caption{Voda vypařující se do vzduchu \cite[s.~21]{Feynman01}.}
        \label{fyz:fig011}
      \end{figure}

      Proč nepozorujeme tyto změny my? Protože do vody se vrací právě tolik molekul, kolik z ní 
      odešlo. Navenek se „nic neděje“. Když odkryjeme nádobu, odfoukneme vlhký vzduch pryč a 
      nahradíme ho suchým vzduchem, nezmění se počet z vody vylétajících molekul (neboť závisí 
      pouze na pohybu ve vodě), ale velmi se změní počet molekul do vody se vracejících, protože 
      nad vodou je mnohem méně molekul. Molekul, které opouštějí vodu, je víc než molekul, které se 
      do ní vracejí; voda se vypařuje. Chceme-li tedy, aby se voda vypařovala, zapneme ventilátor!
      
      Zůstává ještě otázka: Které molekuly opouštějí vodu? Molekula opustí vodu, když náhodně získá 
      malé množství dodatečné energie, kterou potřebuje na to, aby překonala přitažlivé působení 
      svých sousedů. Protože ty molekuly, které opouštějí vodu, mají větší než průměrnou energii, 
      budou se molekuly, které ve vodě zůstávají, v průměru pohybovat méně. Při vypařování se tedy 
      kapalina postupně \emph{ochlazuje}. Je samozřejmé, že když molekula páry sestoupí ze vzduchu 
      do vody, objeví se silné přitahování, když molekula dosahuje povrchu vody. Důsledkem toho je 
      zrychlení přicházející molekuly a s tím spojený vznik tepla. \emph{Můžeme tedy říci, že s 
      odchodem molekul odchází a s příchodem molekul přichází teplo.} Když jsou oba procesy 
      vyrovnány, voda svou teplotu nemění. Foukáme-li na vodu, aby odpařování převládalo nad 
      zkapalňováním, voda se ochlazuje. Proto, chcete-li ochladit polévku, foukejte na ni!
      
      \begin{figure}[ht!]    % \ref{fyz:fig012}
        \centering
        \luafigure[0.7]{fyz_fig012.pdf}
        \caption{Voda vypařující se do vzduchu \cite[s.~21]{Feynman01}.}
        \label{fyz:fig012}
      \end{figure}

      Musíme si však uvědomit, že procesy, o kterých jsme hovořili, probíhají ve skutečnosti 
      složitěji. Při unikání vody do vzduchu čas od času některá z molekul kyslíku nebo dusíku 
      vnikne do vody a \uv{ztratí se} mezi jejími molekulami. Vzduch se tedy rozpouští ve vodě. 
      Molekuly kyslíku a dusíku pronikají do vody, která pak obsahuje vzduch. Když z nádoby náhle 
      odstraníme vzduch, budou molekuly vzduchu unikat z vody rychleji, než do ní vnikají, což 
      způsobí vystupování bublinek. Tato skutečnost je velmi nepříjemná pro potápěče.
    
      Nyní si všimněme dalšího procesu. Obr. \ref{fyz:fig012} znázorňuje, jak se podle atomové 
      představy rozpouští pevná látka ve vodě. Co se stane, vložíme-li krystal soli do vody? Sůl je 
      pevná látka, krystal, organizované seskupení „atomů soli“. Na obr. \ref{fyz:fig013} je 
      znázorněna trojrozměrná struktura kuchyňské soli, chloridu sodného. Máme-li být přesní, 
      musíme říct, že krystal není tvořen atomy, ale ionty. Iont je atom, který má několik 
      elektronů navíc, nebo několik elektronů ztratil. V krystalu soli nalézáme \emph{ionty chlóru} 
      (atomy chlóru s přebytečným elektronem) a \emph{ionty sodíku} (atomy sodíku zbavené jednoho 
      elektronu). Ionty jsou v krystalu vzájemně vázány elektrickou přitažlivostí, ale ve vodě se 
      některé z nich pod vlivem přitažlivosti záporného kyslíku a kladného vodíku začnou uvolňovat. 
      Na obrázku \ref{fyz:fig012} vidíme uvolňující se iont chlóru a jiné atomy plavající ve vodě 
      ve formě iontů. Tento obrázek je pečlivě zakreslený. Všimněme si například, že vodíkové konce 
      molekul vody obvykle obklopují iont chlóru a u iontu sodíku zpravidla nalézáme kyslíkový 
      konec, neboť sodík je kladný a kyslíkový konec molekuly vody je záporný a tyto se elektricky 
      přitahují. Můžeme podle tohoto obrázku říci, jestli se sůl \emph{rozpouští} ve vodě, nebo 
      \emph{krystalizuje} z vody? Samozřejmě, že \emph{nemůžeme}, neboť zatím co jedny atomy 
      opouštějí krystal, jiné se k němu připojují. Takovýto proces je - podobně jako vypařování - 
      \emph{dynamický} všechno závisí na tom, je-li ve vodě více nebo méně soli, než je třeba k 
      rovnováze. Rovnováhou rozumíme takovou situaci, kdy počet atomů opouštějících krystal je 
      roven počtu atomů do krystalu se vracejících. Když sůl ve vodě téměř není, vstupuje do vody 
      více atomů, než vystupuje a sůl se rozpouští. Když je, naopak, „atomů soli“ příliš mnoho, do 
      krystalu se vrací více atomů, než ho opouští a sůl krystalizuje.
 
      \begin{figure}[hbt!]  % \ref{fyz:fig013}
        \centering
          \subcaptionbox{\label{fyz:fig013a}}{\luafigure[0.4]{fyz_fig013a.pdf}}   \newline                                   
          \subcaptionbox{\label{fyz:fig013b}}{\luafigure[0.8]{fyz_fig013b.pdf}}
        \caption{Vzdálenost nejbližších sousedů \(d = \dfrac{a}{2}\) \cite[s.~22]{Feynman01}}
        \label{fyz:fig013}
      \end{figure}
      
      Zmínili jsme se o tom, že představa \emph{molekuly} látky je pouze přibližná a je 
      opodstatněná jen pro určitou třídu látek. Je jasné, že v případě vody jsou její tři atomy 
      skutečně svázané, ale v případě pevného chloridu sodného už to tak jasné není. V takovém 
      případě jde o uspořádání sodíkových a chlorových iontů do krychlové mřížky a neexistuje 
      přirozený způsob jejich uspořádání do „molekul soli“.
      
      Vraťme se ještě k naší diskuzi o \emph{rozpouštění a srážení}. Zvýšíme-li teplotu roztoku 
      soli, vzroste počet atomů, které sůl opouštějí a vzroste i počet atomů, které se do soli 
      vracejí. Ukazuje se, že obecně je velmi těžké předpovědět, jak se ten proces realizuje, 
      proběhne-li rozpouštění rychleji nebo pomaleji. S rostoucí teplotou se většina látek 
      rozpouští lépe, ale některé látky se rozpouštějí hůře.
      
    \subsection{Chemické reakce}
      Ve všech procesech, o nichž jsem dosud hovořili, neměnily atomy a ionty své partnery. Za 
      určitých okolností však může dojít ke změně atomových kombinací, vytvoří se nové molekuly. 
      Taková situace je znázorněna na obr. \ref{fyz:fig014}.
      
      \begin{figure}[hbt!]   % \ref{fyz:fig014}
        \centering
        \luafigure[0.7]{fyz_fig014.pdf}
        \caption{Uhlík hořící v kyslíku \cite[s.~23]{Feynman01}}
        \label{fyz:fig014}
      \end{figure}

      Proces, ve kterém dochází k přeskupení atomových partnerů, nazýváme \textbf{chemickou 
      reakcí}. Ostatní dosud uvažované procesy nazýváme \textbf{fyzikálními procesy}. Mezi 
      uvedenými dvěma druhy, procesů však neexistuje ostrá hranice. Příroda se nestará o naše 
      názvosloví a pokračuje i nadále ve svém díle. Uvedený obrázek má znázornit hoření uhlíku v 
      kyslíku. Kyslík se vyznačuje tím, že jeho dva atomy jsou velmi pevně svázány. (Proč nejsou 
      svázány \emph{tři} nebo dokonce \emph{čtyři} atomy? Toto je jedna ze zvláštností atomových 
      procesů. Atomy jsou velmi svérázné: upřednostňují určité partnery, určité směry apod. Úlohou 
      fyziky je analyzovat, proč chtějí právě to, co chtějí. V každém případě dva atomy kyslíku, 
      nasycené a šťastné, tvoří molekulu.)
      
      Předpokládejme, že atomy uhlíku vytvářejí pevný krystal - grafit nebo diamant (diamant může 
      shořet ve vzduchu). Uvažujme situaci, kdy se molekula kyslíku dostane k uhlíku, Každý její 
      atom zachytí atom uhlíku a odletí v novém seskupení - „uhlík-kyslík“. Toto seskupení 
      představuje molekulu plynu nazývaného \emph{oxid uhelnatý}. Jeho chemické označení je 
      \ce{CO}. Je to velmi jednoduché: písmena „CO“ jsou vlastně obrazem jeho molekuly. Jenže 
      uhlík váže kyslík o mnoho silněji než kyslík váže kyslík nebo uhlík váže uhlík. Proto v tomto 
      procesu může kyslík přicházet s malou energií, ale kyslík a uhlík se spojí velmi „energicky“ 
      a uvolněnou energii pohltí okolní atomy. Tak se vytváří velké množství pohybové, kinetické 
      energie. Myslíme tím samozřejmě \textbf{hoření}; spojením uhlíku a kyslíku získáváme 
      \emph{teplo}. Teplo se obvykle projevuje formou pohybu molekul horkého plynu, ale za určitých 
      okolností ho může být tak mnoho, že způsobuje světlo. Tak vzniká \textbf{plamen}.
      
      \begin{figure}[hbt!]    % \ref{fyz:fig015}
        \centering
        \luafigure[0.7]{fyz_fig015.pdf}
        \caption{Vůně fialek \cite[s.~24]{Feynman01}}
        \label{fyz:fig015}
      \end{figure}

      Kromě toho, \emph{oxid uhelnatý} není zcela uspokojen. Je možné, aby k sobě připoutal další 
      atom kyslíku a tak dostaneme mnohem složitější reakci, ve které se kyslík spojuje s uhlíkem a 
      současně dochází ke srážce s molekulou oxidu uhelnatého. Kyslíkový atom se připojí k \ce{CO} 
      a v konečném důsledku vytvoří molekulu složenou z jednoho uhlíku a dvou kyslíků. Tato 
      molekula má označení \ce{CO2} a nazývá se \emph{oxid uhličitý}. Spalujeme-li uhlík ve velmi 
      malém množství kyslíku a reakce probíhá velmi rychle (např. v motoru automobilu, kde je 
      výbuch tak rychlý, že se nestačí vytvořit oxid uhličitý), vzniká velké množství oxidu 
      uhelnatého. V mnoha takových přeskupeních atomů se uvolňuje velké množství energie, vznikají 
      výbuchy, plamen apod., podle druhu reakce. Chemici studovali takové seskupení atomů a 
      zjistili, že každá látka představuje určitý druh \emph{uspořádání atomů}.

      K objasnění této myšlenky si zvolme jiný příklad. Ocitneme-li se na louce rozkvetlé fialkami, 
      víme, co je to za „vůni“. Je to určitý druh molekul nebo seskupení atomů, které se dostalo do 
      našeho nosu. Jak se nám to stalo? To je dost jednoduché! Jestliže vůně je jistý druh molekul, 
      tím nejrozmanitějším způsobem poletujících a srážejících se ve vzduchu, pak se může náhodou 
      dostat i do nosu. Tyto molekuly se určitě nesnažily dostat právě do našeho nosu. Jsou jen 
      bezmocnou částí strkajícího se zástupu molekul, jehož kousek se na svém bezcílném putování 
      dostal do našeho nosu.      
      
      Chemici mohou i takové zvláštní molekuly, jako je vůně fialek, podrobit analýze a říci nám 
      \emph{přesné uspořádání} jejich atomů v prostoru. Víme, že molekula oxidu uhličitého je 
      \emph{přímá a symetrická}: \ce{O\bond{-}C\bond{-}O} (lze to snadno zjistit i fyzikálními 
      metodami). I pro mnohem složitější seskupení atomů, jako jsou ty, se kterými pracuje chemie, 
      můžeme zdlouhavým, pozoruhodným procesem, připomínajícím práci detektiva, zjistit tvar 
      seskupení. Obr. \ref{fyz:fig015} znázorňuje vzduch v blízkosti fialky: ve vzduchu opět 
      nalézáme dusík, kyslík a vodní páru. (Odkud se vzala vodní pára? Fialka je vlhká, protože 
      všechny rostliny odpařují vodu.) Vidíme však i \uv{monstrum} složené z uhlíkových, vodíkových 
      a kyslíkových atomů, které vytvořily zcela určité, zvláštní seskupení. Je to mnohem 
      složitější seskupení než v případě oxidu uhličitého. Naneštěstí do obrázku nemůžeme zakreslit 
      všechno, co o něm po chemické stránce víme, neboť seskupení všech atomů je trojrozměrné, 
      zatímco náš obrázek je pouze dvojrozměrný. Šest uhlíků vytváří ne plochý, ale \uv{zvrásněný} 
      prstenec. Všechny úhly a vzdálenosti známe. Chemický vzorec je jen obrázkem takové molekuly. 
      Když chemik napíše vzorec na tabuli, snaží se \uv{nakreslit} dvojrozměrný obraz molekuly. 
      Například, vidíme \uv{prstenec} šesti uhlíků a na jednom konci visící \uv{řetěz} uhlíků, na 
      něm kyslík druhý od konce, tři vodíky vázané na tento uhlík, dva uhlíky a tři vodíky vázané 
      nahoře atd.

      \begin{figure}[hbt!]    % \ref{fyz:fig016}
        \centering
        \luafigure[0.9]{fyz_fig016.pdf}
        \caption{Strukturní vzorec vůně fialky (\(\alpha\)-iron) \cite[s.~24]{Feynman01}}
        \label{fyz:fig016}
      \end{figure}

      Jak chemik zjistí, o jaké uspořádání jde? Smíchá obsah dvou lahviček a když se směs zbarví 
      červeně, ví, že látka obsahuje jeden vodík a dva uhlíky vázané na určité místo molekuly. 
      Zbarví-li se směs modře, je to úplně jinak. To je organická chemie - jeden z 
      nejfantastičtějších kousků detektivní práce. Aby objevil uspořádání atomů v neobyčejně 
      komplikovaných útvarech, chemik sleduje, co se děje při smíchání dvou rozdílných látek. Fyzik 
      by nikdy zcela neuvěřil, že chemik ví, o čem mluví při popisu uspořádání atomů. Jenže asi 
      před dvaceti lety se objevila fyzikální metoda umožňující v některých případech pozorovat 
      molekuly (ne tak složité, jako je molekula vůně fialky, ale takové, které obsahují části této 
      molekuly). Touto metodou je možné lokalizovat každý atom, a to ne sledováním zbarvení směsi, 
      ale měřením skutečné polohy atomů. A světe, div se! Ukázalo se, že chemici měli téměř vždy 
      pravdu. Zjistilo se, že vůně fialky obsahuje tři málo se lišící molekuly, jejichž rozdílnost 
      spočívá pouze v jiném uspořádání vodíkových atomů. 
      
      Jedním z problémů chemie je tvorba chemického názvosloví. Každé molekule musíme najít jméno! 
      Toto jméno musí ukazovat nejen její tvar, ale musí vyjadřovat i to, že tu je kyslíkový atom, 
      tam vodíkový - musí říkat, kde přesně ten který atom je. Takto pochopíme, že chemické názvy 
      musí být složité, aby byly-úplné. Název fialkové vůně má v podobě prozrazující strukturu 
      následující znění: 4-(2,2,3,6 tetrametyl-5-cyklohexa\-nyl)-3-buten-2-on. Teď chápeme obtíže, 
      se kterými chemici zápolí a rovněž chápeme příčinu tak dlouhých názvů. Není to proto, že by 
      chemici chtěli být záhadnými, ale je to proto, že bojují s velmi obtížným problémem popisu 
      molekuly slovy.
      
      \emph{Jak víme, že atomy existují?} Používáme k tomu trik, o kterém jsme se již zmínili:
      \emph{předpokládáme} jejich existenci a všechny výsledky, jeden po druhém, vycházejí tak, jak
      by měly, kdyby se látka skládala z atomů. Existují i přímější důkazy. Příkladem takového
      důkazu je následující skutečnost. Atomy jsou tak malé, že je nemůžeme vidět pomocí
      \emph{světelného mikroskopu} - dokonce ani pomocí \emph{elektronového mikroskopu}. (Světelným
      mikroskopem je možné vidět jen věci mnohonásobně větší.) Atomy jsou však v neustálém pohybu a
      když vložíme do vody nějaký míček, který je mnohem větší než atomy, bude poskakovat. Bude se
      chovat podobně, jak se chová velký míč postrkovaný při hře velkého množství lidí. Lidé
      postrkují míč různými směry a ten se pohybuje po hřišti nepravidelně. Právě tak se bude
      pohybovat \uv{velký míč} ve vodě, neboť v různých okamžicích na něj budou z různých stran
      dopadat nestejné údery. Proto při sledování velmi malých částeček (koloidů) ve vodě pomocí
      výborného mikroskopu (obr. \ref{fyz:fig892}) pozorujeme jejich neustálé poskakování jako
      následek toho, že jsou bombardovány atomy. Tento jev se nazývá \textbf{Brounův pohyb}.

      \begin{figure}[hbt!]    % \ref{fyz:fig892}
        \centering
        \luafigure[0.9]{fyz_fig892.pdf}
        \caption{ V knize \emph{Les Atomes} z roku 1916, kterou napsal nobelista Jean Baptiste
                  Perrin, jsou publikovány tři stopy pohybu koloidních částic o poloměru
                  \SI{0.53}{\um}, pozorovaných pod mikroskopem. Postupné pozice jsou každých 30 sekund
                  spojeny přímými segmenty (velikost ok síťe je \SI{3.2}{\um}.
                  \cite[s.~115]{Perrin1914}}
        \label{fyz:fig892}
      \end{figure}
      
      Další důkaz existence atomů můžeme vidět ve \emph{struktuře krystalů}. V mnoha případech 
      souhlasí struktury odvozené na základě rentgenové analýzy svými prostorovými „tvary“ s 
      formami samotných přírodních krystalů. Úhly mezi různými krystalickými „stěnami“ souhlasí s 
      přesností na úhlové vteřiny s úhly určenými za předpokladu, že krystal je tvořen mnoha 
      „vrstvami“ atomů.
      
      \textbf{Vše se skládá z atomů}. To je klíčová hypotéza. Například v celé biologii je 
      nejdůležitější hypotézou to, že vše, co dělají živočichové, dělají atomy. Jinými slovy, v 
      živých věcech není nic, co by nemohlo být pochopeno z pohledu, že se skládají z atomů 
      podléhajících fyzikálním zákonům. To nebylo vždy známo: k formulování této hypotézy bylo 
      třeba mnoha experimentů i teoretických úvah. Dnes je tato hypotéza uznávána a je 
      nejužitečnější teorií pro vytváření nových myšlenek v oblasti biologie.
      
      Jestliže kousek oceli nebo kousek soli skládající se z uspořádaných atomů může mít tak 
      zajímavé vlastnosti, jestliže voda - která není ničím jiným než těmi malými kapkami stejnými 
      na celé Zemi - může tvořit vlny a pěnu, hučet příbojem a vytvářet podivné tvary omýváním 
      břehů, jestliže toto všechno, celý život vodního proudu nemůže být ničím jiným než hromada 
      atomů, co víc je ještě možné? Jestliže namísto uspořádání atomů podle určitého, stále 
      opakovaného vzoru, nebo jestliže namísto tvorby malých, ale složitých shluků, jako je vůně 
      fialky, seskupíme atomy v každém místě jinak, různé druhy atomů seskupíme různými způsoby 
      tak, aby se nic neopakovalo, o co úžasněji se může takováto věc chovat? Je možné, že „věci“, 
      které se před vámi procházejí a baví se s vámi, jsou velké shluky těchto atomů velmi složitým 
      způsobem seskupené, takže pouhá naše představivost nestačí předpovědět jejich chování? 
      Jestliže říkáme, že jsme shlukem atomů, nemyslíme tím, že jsme jen shlukem atomů, protože 
      takový shluk atomů, který se nikdy neopakuje, může vypadat právě tak jako to, co vidíme v 
      zrcadle.
            
  \section{Nejzákladnější myšlenky fyziky}\label{fyz:IchapIsecII}
    V této kapitole jsou zachyceny \emph{nejzákladnější myšlenky}, s nimiž se ve fyzice setkáváme - 
    bude pojednáváno o tom, jaká je v současnosti představa o povaze věcí. Nebude však hovořeno o 
    tom, jak se poznala správnost těchto představ - o těchto detailech bude pojednáváno později, až 
    přijde ten pravý čas.

    Věci, o něž se ve fyzice zajímáme, se ukazují množstvím projevů a atributů. Stojíme-li 
    například na břehu a hledíme na moře, vidíme vodu, na vodě pěnu, nad mořem oblaka, slunce, 
    modrou oblohu a vůbec světlo, slyšíme zvuk, nárazy vln, svištění větru, cítíme vzduch. Na břehu 
    je písek a skály, a každá má jinou tvrdost a pevnost, barvu a složení. Jsou tam zvířata a vodní 
    tráva, je tam hlad i nemoc a na břehu je pozorovatel se svými myšlenkami a snad i štěstím. 
    Každé jiné místo v přírodě se vyznačuje podobnou pestrostí věcí a vlivů, podobnou složitostí. 
    Naše zvědavost nás nutí klást otázky, hledat souvislosti a chápat mnohotvárnost věcí jako 
    následek snad relativně malého počtu nejjednodušších věcí a sil působících nekonečně rozmanitě.
    
    Klademe si otázku: Je písek jiný než skály? Není snad písek nic jiného, než velký počet velmi 
    malých kamínků? Je Měsíc velká skála? Kdybychom porozuměli tomu, co jsou skály, znamená to, že 
    bychom pochopili i podstatu písku a Měsíce? Co je to vítr? Jsou to nárazy vzduchu podobné 
    nárazům vody na břeh? Jaké společné rysy mají rozličné druhy pohybu? Co mají společného různé 
    druhy zvuku? Kolik různých barev existuje? A tak dále. Takovým způsobem se snažíme postupně 
    analyzovat všechny věci. Dáváme do souvislostí věci, které na první pohled vzájemně nesouvisí. 
    Děláme to s nadějí, že se nám podaří redukovat počet rozličných věcí a tak je lépe poznat.
    
    Před několika sty lety vznikla metoda hledání částečných odpovědí na uvedené otázky. 
    \emph{Pozorování, usuzování a experiment} vytvářejí to, co nazýváme \emph{vědeckou metodou}. 
    Budeme se muset omezit jen na holý popis našich představ o tom, co se nazývá \emph{základní 
    fyzikou} nebo základními myšlenkami, které vznikly aplikováním vědecké metody.
    
    Co to znamená něco „pochopit“? Můžeme si představit, že to složité nahromadění pohybujících se 
    věcí, které vytvářejí „svět“, je šachová hra bohů a my vystupujeme jako diváci, kteří neznají 
    pravidla hry, ale je jim dovoleno hru \emph{pozorovat}. Samozřejmě, pozorujeme-li dostatečně 
    dlouho, můžeme nakonec pochytit několik pravidel. \emph{Pravidla hry} představují to, co 
    chápeme jako \emph{základní fyziku}. I  kdybychom znali všechna pravidla, nemuseli bychom ještě 
    rozumět každému kroku hry, protože je příliš složitá a možnosti našeho rozumu omezené. 
    Hrajete-li šachy, jistě víte, že je jednoduché naučit se všechna pravidla, ale i tak je velmi 
    těžké zvolit ten správný tah nebo pochopit záměry protihráče. Stejné je to i s přírodou, jen 
    mnohem těžší. Máme však možnost najít alespoň všechna pravidla. Zatím je všechna neznáme. 
    (Každou chvíli se objevuje něco takového jako rošáda, kterou ještě neznáme.) Nejen, že neznáme 
    všechna pravidla, ale pomocí těch, která známe, umíme jen velmi málo vysvětlit. Je tomu tak 
    proto, že téměř všechny situace jsou ohromně složité a známá pravidla nám neumožní sledovat 
    všechny obraty hry, nemluvě o předvídání dalších kroků. Musíme se proto omezit na základnější 
    otázku pravidel hry. Naučíme-li se pravidla, budeme to považovat za „pochopení“ světa.
    
    Jak můžeme rozhodnout, zda pravidla, která vlastně jen „odhadujeme“, jsou skutečně správná, 
    když nemůžeme dokonale analyzovat hru? Existují zhruba tři způsoby. Především nám příroda může 
    poskytnout (nebo my si od přírody vynutíme) jednoduché situace skládající se z malého počtu 
    částí, umožňující přesnou předpověď budoucího dění, a tím i zkoušku pravidel. (V rohu 
    šachovnice zůstalo jen málo figurek, jejichž tahy již umíme přesně určit)
    
    Druhý způsob zkoušky pravidel spočívá v jejich použití k odvození obecnějších pravidel. 
    Například, střelec se na šachovnici pohybuje úhlopříčně. Odtud je možné usuzovat na skutečnost, 
    že určitý střelec bude vždy na bílém poli. Odhlédneme-li od podrobností, můžeme prověřovat naše 
    pravidlo o pohybu uvedeného střelce tak, že sledujeme, jestli se vždy nachází na bílém poli. Po 
    dlouhém čase se samozřejmě může stát, že se náhle objeví na černém poli (v průběhu hry byl 
    vzat, ale jeden pěšec došel na konec šachovnice a proměnil se na střelce na černém poli). Tak 
    to bývá i ve fyzice. Dlouho používáme pravidlo, které ve všech směrech dobře vyhovuje, ačkoliv 
    neznáme detaily, a potom najednou objevíme \emph{nové pravidlo}. Z hlediska základů fyziky 
    probíhají nejzajímavější jevy na nových místech, na místech, kde pravidla neplatí a ne tam, kde 
    pravidla \emph{platí}. To je způsob, jakým objevujeme nová pravidla.
    
    Třetí ze způsobů, kterými se můžeme přesvědčit o správnosti našich myšlenek, je poměrně hrubý, 
    ale snad nejúčinnější. Je to způsob přibližného odhadu. Ačkoliv nejsme schopni říci, proč 
    Aljechin \emph{táhl právě tou figurkou}, můžeme v \emph{hrubých rysech} chápat, že seskupuje 
    figurky okolo krále, aby ho chránil, protože za daných okolností je to nejrozumnější. Podobně 
    je to i s naším chápáním přírody. Často ji více či méně chápeme, aniž bychom byli schopni znát 
    význam tahu \emph{každé jednotlivé figurky}.
    
    Zpočátku se přírodní jevy hrubě rozdělovaly do tříd jako teplo, elektřina, mechanika, 
    magnetizmus, vlastnosti látek, chemické děje, světlo nebo optika, rentgenové paprsky, jaderná 
    fyzika, gravitace, mezonové jevy atd. Cílem je však pochopení \emph{celé přírody} jako různých 
    aspektů \emph{jednoho souboru} jevů. Úkolem základní teoretické fyziky dneška je \emph{nalezení 
    zákonů stojících za experimentem a sjednocení uvedených tříd}. Historicky se nám vždy podařilo 
    sloučit je, ale postupem času se objevovaly nové věci. Když jsme si již vytvořili ucelenou 
    představu, objevily se najednou rentgenové paprsky. Když se i tento jev dostal do jednotného 
    schématu, objevily se mezony. Proto v každém stádiu hry vypadá situace dost chaoticky. Mnohé se 
    objasnilo z jednotného hlediska, ale ještě stále je mnoho volných konců nitek, o nichž nevíme, 
    kam patří. Takový je dnes stav věcí a my se ho pokusíme popsat.
    
    Všimněme si v historii několika příkladů uvedeného sjednocování. Uvažujme nejdříve \emph{teplo 
    a mechaniku}. Jsou-li atomy v pohybu, obsahuje systém tím více tepla, čím více pohybu v něm je, 
    takže \emph{teplo a všechny tepelné efekty je možné vyjádřit pomocí zákonů mechaniky}. Dalším 
    úžasným sjednocením bylo objevení souvislosti mezi \emph{elektřinou, magnetizmem} a světlem, o 
    nichž se zjistilo, že jsou různými aspekty stejné věci, kterou dnes nazýváme 
    \emph{elektromagnetické pole}. Dále chemické děje, rozmanité vlastnosti různých látek a chování 
    atomových částic byly sjednoceny do \emph{kvantové chemie}.
    
    Zůstává zde však otázka, zda bude možné vše sjednotit tak, abychom mohli prohlásit, že svět 
    představuje rozmanité aspekty jediné věci? To nikdo neví. Víme pouze, že na naší cestě vpřed se 
    nám daří spojovat fragmenty, přičemž vždy nalézáme cosi, co nezapadá do obecného obrazu, a 
    proto se opět pokoušíme doplnit skládačku. Nevíme, zda tato skládačka má konečný počet částí a 
    zda má tato hra vůbec hranice. Dozvíme se to až tehdy, když složíme výsledný obraz, jestli ho 
    vůbec kdy složíme. Chtěli bychom však ukázat, kam až tento proces sjednocování pokročil a jaká 
    je dnešní situace při objasňování základních jevů pomocí co nejmenšího počtu principů. 
    Jednodušeji řečeno: \textbf{z čeho jsou složeny věci a kolik je těch stavebních prvků?} 
    \cite[s.~27]{Feynman02}
    
  \section{Hlavní etapy vývoje}\label{fyz:IchapIsecIII}
    Fyzika prošla dlouhým historickým vývojem a znalost tohoto vývoje pomáhá lépe pochopit logiku 
    soustavy fyzikálních poznatků a dokonce do\-cházet k poznatkům novým. V krátkosti dějiny 
    fyziky můžeme rozdělit na tři hlavní etapy:
    \begin{itemize}[noitemsep]
     	\item \textbf{Stará fyzika}: od starověku do počátku 17. století (orientačně do roku 1600).
      \item \textbf{Klasická fyzika}: 1600 - 1900.
      \item \textbf{Moderní fyzika}: 1900 - dosud.
    \end{itemize}
    Starou fyziku nemůžeme považovat za vědu ve vlastním smyslu, i když se dobrala celé řady 
    významných vědeckých poznatku. První z nich znali již staří Sumerové, Babyloňané, Egypťané a 
    Číňané. Šlo zejména o  poznatky astronomické a geometrické (Pythagorova veta) a také o metody 
    měření některých fyzikálních veličin (délka, hmotnost, čas). Fyzika ve starém Řecku byla jako 
    součást filosofie převážně spekulativní a tento charakter si pod vlivem aristotelismu udržela, 
    až do počátku novověku. Skutečný fyzikální výzkum prováděli až helenističtí Řekové, kdy se 
    centrem vědy a kultury antického světa stala Alexandrie. 
    
    \begin{figure}[ht!]  % \ref{fyz:fig894}
      \centering
      \luafigure[1]{fyz_fig894.jpg}
      \caption{ \wikiAlexLib byla největší a nejslavnější knihovna starověku. Byla součástí
                věhlasného múseia v Alexandrii, vybudovaného z podnětu Ptolemaia I. Byla považována
                za hlavní centrum vzdělanosti od 3. století př. n. l. až do roku 48 př. n. l., kdy
                za války mezi Caesarem a Pompeiem zčásti vyhořela. Starověké zdroje pojednávají o
                ničení knihovny, o tom, kdo je zodpovědný za ničení a kdy k němu došlo, se liší.}
      \label{fyz:fig894}
    \end{figure} 

    V Alexandrii studoval největší fyzik starověku Archimédes, který dospěl k důležitým poznatkům o
    statické rovnováze těles a plování těles a v matematice se těsně přiblížil objevu
    diferenciálního a integrálního počtu. Alexandrijští Řekové znali také zákon odrazu světla
    (nikoli lomu) a prováděli první měření teploty. Poznatky antiky byly středověké Evropě
    zprostředkovány Araby, kteří se též intenzivně zabývali optikou (Alhazen) a určováním měrné
    hmotnosti látek. Zatímco ve středověku byly hlavní přírodovědné poznatky čerpány z Euklidových ”
    Základu” (geometrie), ”Almagestu” Klaudia Ptolemaia (geocentrický výklad astronomie sluneční
    soustavy) a spisu Aristotelových (mj.”Fysika”), vešly práce Archimédovy v Evropě ve známost až
    teprve začátkem novověku. Ve starověku a středověku však fyzika neprováděla systematické
    experimenty, nevyužívala matematický aparát k popisu přírodních jevu a neměla ani přesně
    definovány základní pojmy (rychlost, zrychlení, síla apod.) Zrod fyziky jako vědy se datuje
    začátkem 17. století. Na základě astronomických výzkumu Keplerových (1571-1630) a pozemských
    mechanických experimentů Galileových (1564-1642) mohl Isaac Newton (1643-1727) vytvořit první
    fyzikální teorii, klasickou mechaniku, využívající matematický aparát diferenciálního a
    integrálního poctu. Newton přišel s koncepcí všeobecné gravitace a ukázal, že není přehrady mezi
    nebeskou a pozemskou fyzikou, že síla, která udržuje planety na jejich drahách kolem Slunce je
    táž jako síla, která nutí jablko padat k zemi. Základní Newtonovo dílo z r. l687 nese název ”
    Matematické základy přírodní filosofie” (”Philosophiae naturalis principia mathematica”) a
    představuje pravděpodobně nejvýznamnější vědeckou knihu, která byla kdy napsána. Newton se
    zabýval též optikou a rozpracoval teorii rozkladu bílého světla do spektra. V té době byl již
    zásluhou Snellovou a Descartovou znám i zákon lomu světla. Z roku 1600 pochází první vědecký
    spis o elektřině a magnetismu od anglického lékaře a fyzika Gilberta. Výzkumem  těchto jevu se v
    následujících stoletích zabývala celá řada fyziků (Coulomb, Volta, Oersted, Amp\`{e}re a další).
    Tento výzkum pak završil Faraday (1791-1867) svým objevem zákona elektromagnetické indukce a
    svou koncepcí siločár elektromagnetického pole. Úlohu Newtona elektromagnetismu pak sehrál James
    Clerk Maxwell (1831-1879), který ve svém ”Traktátě o elektřině a magnetismu” z r. 1873 sestavil
    slavné Maxwellovy rovnice popisující vlastnosti elektromagnetického pole. Maxwell zároveň
    teoreticky zdůvodnil elektromagnetickou povahu světla a ukázal, že jevy spojené s vlastnostmi
    elektrického náboje (”elektřina”), elektrického proudu (”galvanismus”), magnetického pole a
    světla (optika), jsou jedné a téže elektromagnetické povahy. V devatenáctém století byl tak
    dovršen výzkum mechanických jevů a elektromagnetismu a klasická fyzika tím za\-vršena. V přírodě
    tedy existovaly pouze dvě síly, dva způsoby vzájemné interakce mezi částicemi: gravitační a
    elektromagnetická. Mezi nimi se však projevoval určitý rozpor. Jak Newtonovy tak Maxwellovy
    rovnice platí v libovolné inerciální vztažné soustavě. Při přechodu od jedné inerciální soustavy
    k druhé se však Newtonovy rovnice transformují pomocí tzv. Galileiho transformací a Maxwellovy
    rovnice pomocí Lorentzových transformací. Fyzika se tak rozdvojila, mechanické a
    elektromagnetické děje se zdály být neslučitelné. Kromě toho existovaly některé experimenty,
    jejichž výsledek nedokázala klasická fyzika vysvětlit: průběh spektra rovnovážného
    elektromagnetického záření (tzv. záření absolutně černého tělesa) a pokus Michelsonův, který
    svědčil o neexistenci světelného éteru. Tyto zdánlivě nepodstatné rozpory vyústily ve 20.
    století ve vznik moderní fyziky, tj. fyziky kvantové a relativistické. Právě koncem roku 1900
    vyslovil Planck tzv. kvantovou hypotézu, jíž vysvětlil záření absolutně černého tělesa, a v r.
    1905 publikoval Einstein práci o speciální teorii relativity. V ní překlenul rozpor mezi
    Newtonovou a Maxwellovou fyzikou a fyziku opět sjednotil. Předpoklad o existenci světelného
    éteru se teorií relativity stal zbytečným. V roce 1916 vytvořil Einstein i obecnou teorii
    relativity jako moderní teorii gravitace. Gravitační síly podle této teorie souvisejí se
    zakřivením prostoročasu. Jak speciální, tak obecná teorie relativity přecházejí při rychlostech
    objektu podstatně menších než je rychlost světla ve vakuu a při slabých gravitačních polích v
    teorii Newtonovu. Přelom 19. a 20. století je též poznamenán objevem radioaktivity a vznikem
    jaderné fyziky, která tak významným způsobem zasáhla do života celého lidstva. V jaderné fyzice
    se uplatní další dvě přírodní síly - tzv. silná, která udržuje nukleony v atomových jádrech a
    slabá, která se projevuje při radioaktivní přeměně beta za vzniku neutrin. Moderní fyzika
    odhalila v kosmickém záření a pomocí urychlovačů obrovské množství částic, jejichž vlastnosti
    studuje a snaží se je utřídit a vysvětlit. Mezi všemi těmito částicemi působí čtyři základní
    síly přírody: gravitační, elektromagnetická, silná a slabá. V nedávné době se podařilo prokázat,
    že i elektromagnetická a slabá interakce jsou téže podstaty a tvoří jedinou sílu elektroslabou.
    V průběhu historie fyziky od Newtona a Maxwella k dnešku tak probíhá úsilí o sjednocování
    interakcí, které pokračuje i dnes. Fyzika se pokouší prokázat, že i silná a elektroslabá
    interakce jsou téže povahy, a že k nim konečně přistupuje i síla gravitační. Tím by vznikla idea
    jediné přírodní síly sjednocující všechny přírodní jevy a děje. Fyzika ovšem nemůže k takovému
    závěru dojít pouhým uvažováním, musí matematicky vypracovat a zdůvodnit příslušnou teorii a její
    závěry experimentálně ověřit. To vede ke snaze budovat stále větší a větší urychlovače a také k
    intenzivnímu výzkumu jevů v kosmu. Sjednocování interakcí má totiž těsnou návaznost na vývoj
    vesmíru podle hypotézy o tzv. ”velkém třesku”. Právě v počátcích vývoje vesmíru by se měly
    všechny čtyři (resp. tři) interakce uplatňovat rovnocenným způsobem a teprve v průběhu dalšího
    vývoje a rozpínání vesmíru se postupně oddělovat. Tak jako počátky vzniku vědecké fyziky v 17.
    století jsou spjaty s astronomickými pozorováními sluneční soustavy, je i dnes fyzika stále více
    propojena s astrofyzikou. Vesmír zůstává největší fyzikální laboratoří.
  
  \section{Fyzika před rokem 1920}\label{fyz:IchapIsecIV}
    Je dost těžké začít hned se současnými představami, a proto se podívejme, jak se jevil svět v 
    roce 1920 a potom na tomto obrázku něco změníme. Naše představa světa byla před rokem 
    \textbf{1920} následující: „Scénou“, na které vystupuje vesmír, je \emph{trojrozměrný 
    geometrický prostor} popsaný ještě Eukleidem a věci se mění v prostředí, které nazýváme časem. 
    Prvky vystupující na scéně jsou \emph{částice}, například atomy, které mají určité vlastnosti. 
    Především vlastnost setrvačnosti: pohybuje-li se částice, zachová si pohyb v původním směru, 
    pokud na ni nepůsobí \emph{síly}. Druhým prvkem jsou tedy síly, o nichž se tehdy  
    předpokládalo, že jsou dvojího druhu. K prvnímu, velmi složitému druhu, patřila síla vzájemného 
    působení, která udržovala atomy v jejich různých kombinacích komplikovaným způsobem a byla 
    zodpovědná za to, jestli se sůl při zvyšování teploty rozpouští rychleji nebo pomaleji. Druhou 
    známou silou byla interakce dalekého dosahu - hladké a klidné přitahování. Tato síla, měnící se 
    nepřímo úměrně čtverci vzdálenosti, byla nazvána \emph{gravitací}. Její zákon byl známý a byl 
    velmi jednoduchý. Proč věci zůstávají v pohybu, když se už začaly pohybovat, nebo proč existuje 
    gravitační zákon, bylo, samozřejmě, neznámé.
    
    Zabýváme se popisem přírody. Z tohoto hlediska je plyn a právě tak všechna hmota myriádou 
    pohybujících se částic. Takto se dostávají do souvislosti mnohé věci, které jsme viděli na 
    mořském břehu. \emph{Tlak} pochází od \emph{srážek atomů} se stěnami nebo s čímkoliv jiným; 
    atomy pohybující se převážně jedním směrem vytvářejí vítr; \emph{chaotické vnitřní pohyby} 
    představují \emph{teplo}. Známe vlny zvýšené hustoty, kde se shromáždilo příliš mnoho částic, 
    které při rozletu stlačují další shluky částic a pohyb se tak předává dál. Tyto vlny vyšší 
    hustoty představují \emph{zvuk}. Pochopení tolika věcí je možno považovat za úžasný úspěch. O 
    některých z těchto věcí jsme hovořili v předcházející kapitole.
    
    Jaké druhy částic existují? Tehdy předpokládali, že je jich 92. Nakonec bylo objeveno 92 
    různých druhů atomů. Měly různá jména podle svých chemických vlastností.
    
    Byl tu ještě problém \emph{povahy sil krátkého dosahu}. Proč uhlík přitahuje jeden kyslík, 
    případně dva, ale ne víc? Jaký je mechanizmus vzájemného působení mezi atomy? Je to gravitace? 
    Na tuto otázku musíme odpovědět záporně, protože gravitace je na to příliš slabá. Představme si 
    však sílu podobnou gravitaci, měnící se nepřímo úměrně čtverci vzdálenosti, ale mnohem silnější 
    a odlišnou ještě v jednom směru. V případě \emph{gravitace jde vždy o přitahování}. Představme 
    si však, že existují dva druhy „věcí“ a tato nová síla  (samozřejmě elektrické povahy) má tu 
    vlastnost, že věci stejného druhu se odpuzují a věci různého druhu se přitahují. „Předmět“, 
    jenž je nositelem tohoto silného vzájemného působení, se nazývá \emph{náboj}.  
    
    K čemu jsme došli? Předpokládejme, že máme dvě věci různého druhu, jež se vzájemně  
    přitahují (plus a minus) a které drží těsně u sebe. Předpokládejme, že v určité vzdálenosti od 
    uvedené dvojice máme další náboj. Bude tento náboj pociťovat přitažlivost? Mají-li první dva 
    náboje stejnou velikost, neměl by pocítit \emph{prakticky žádnou přitažlivost}, protože 
    přitahování jedním nábojem a odpuzování druhým nábojem se vykompenzují. Ve velkých 
    vzdálenostech je tedy síla velmi malá. Když třetí náboj \emph{hodně přiblížíme} k prvním dvěma, 
    objeví se přitahování, protože odpuzování stejných nábojů a přitahování různých se snaží 
    oddálit stejné náboje a přiblížit různé. Odpuzování bude nakonec \emph{slabší} než přitahování. 
    To je příčina, proč atomy, které se skládají z kladných a záporných elektrických nábojů, na 
    sebe téměř nepůsobí (zanedbáme-li gravitaci), jsou-li od sebe dost vzdáleny. Když se ale 
    přiblíží, mohou „\emph{vidět jeden do druhého}“, přeskupit své náboje a velmi silně vzájemně 
    působit. Podstatou interakce mezi atomy je \emph{elektrické} působení. Tato síla je tak veliká, 
    že všechny plusy a minusy se obvykle dostávají do tak těsné kombinace, jak je to jen možné. 
    Všechny věci, včetně nás samotných, se skládají z drobných, velmi silně interagujících kladných 
    a záporných částic, které jsou velmi přesně vyvážené. Na okamžik je možné náhodou odstranit 
    několik minusů nebo plusů (obvykle je jednodušší odstranit minusy), v tu chvíli jsou elektrické 
    síly \emph{nevyvážené} a můžeme pozorovat působení elektrické přitažlivosti.
    
    Abychom si vytvořili představu o tom, o kolik je elektrické působení silnější než gravitace, 
    představme si dvě zrnka písku, která mají jeden milimetr v průměru a jsou vzdálená třicet 
    metrů. Kdyby elektrické síly mezi nimi nebyly vyvážené, kdyby nebylo odpuzování a vše se 
    navzájem přitahovalo a nic se nekompenzovalo, jakou silou by se zrnka přitahovala? Byla by to 
    síla tří miliónů tun. Jistě chápete, že pro vytvoření značného elektrického působení stačí 
    velmi malý přebytek nebo nedostatek záporných nebo kladných nábojů. Proto není vidět rozdíl 
    mezi elektricky nabitým a nenabitým předmětem - pro nabití předmětu je třeba tak málo částic, 
    že se téměř neprojeví na jeho hmotnosti, či rozměru.
    
    S těmito poznatky bylo jednodušší pochopit atomy. Předpokládalo se, že mají uprostřed 
    „\emph{jádro}“, které je kladně elektricky nabité a velmi těžké, a toto jádro je obklopeno 
    určitým počtem „elektronů“, jež jsou velmi lehké a záporně nabité. Teď trochu pokročíme v našem 
    výkladu a poznamenáme, že v samotných jádrech byly objeveny dva druhy částic - \emph{protony} a 
    \emph{neutrony}, které mají téměř stejnou, velmi velkou hmotnost. Protony jsou elektricky 
    nabité a neutrony jsou neutrální. Máme-li atom se šesti protony v jádře, které je obklopeno 
    šesti elektrony (záporné částice obyčejného světa jsou všechno elektrony a ty jsou velmi lehké 
    v porovnání s protony a neutrony, které tvoří jádra), půjde o atom číslo šest v chemické 
    tabulce a tento atom se nazývá uhlík. Atom číslo osm se nazývá kyslík atd. Chemické     
    vlastnosti závisí na vnějších elektronech, ve skutečnosti jen na tom, kolik má atom elektronů. 
    \emph{Chemické vlastnosti} látek tedy závisí na jediném čísle, na \emph{počtu elektronů}. 
    (Seznam prvků sestavený chemiky by se mohl nahradit očíslováním 1, 2, 3, 4, 5 atd. Místo toho, 
    abychom říkali „uhlík“, stačilo by říci „prvek číslo šest“, což by znamenalo, že prvek má šest 
    elektronů. Při objevování prvků však tato skutečnost nebyla známa a dále, při číslování by vše 
    vypadalo velmi složitě. Proto je lepší ponechat prvkům názvy i symboly a nedožadovat se pouhého 
    očíslování.)

    \begin{figure*}[ht!] %\ref{fyz:fig006}
      \centering
      \luafigure[1]{fyz_fig006.pdf}
      \caption{Elektromagnetické spektrum (někdy zvané Maxwellova duha) zahrnuje elektromagnetické 
               záření všech možných vlnových délek. Srovnání délek elektromagnetických vln s 
               běžnými předměty a odpovídající teplotní stupnice umožňuje lépe získat představu o 
               jejich rozměrech a energiích.}
      \label{fyz:fig006}
    \end{figure*}
    
    O elektrické síle bylo získáno mnoho dalších poznatků. Bylo by přirozené předpokládat, že 
    elektrická interakce je jednoduché přitahování dvou předmětů: kladného a záporného. Zjistilo se 
    však, že toto není úplně vhodná představa. Situaci lépe vystihuje představa, že existence 
    kladného náboje v prostoru způsobuje jeho jisté \emph{zakřivení}, vytváří v něm určitou 
    „podmínku“, aby záporný náboj vložený do tohoto prostoru cítil působení síly. Tato možnost 
    vzniku síly se nazývá \emph{elektrické pole}. Dostane-li se elektron do elektrického pole, je 
    jakoby „tažen“. Přitom platí dvě pravidla: a) \emph{náboje vytvářejí pole}, b) \emph{v poli 
    působí na náboje síly a náboje se pohybují}. Příčina takového chování se stane jasnější, 
    jakmile rozebereme následující jev: Nabijeme-li těleso elektricky, například hřeben, a do 
    určité vzdálenosti položíme nabitý ústřižek papíru, přičemž začneme hřebenem pohybovat sem a 
    tam, bude se papír natáčet směrem k hřebenu. Zrychlíme-li pohyb hřebenu, zjistíme, že papír 
    zaostává, působení se opožďuje. (V prvním stádiu, když pohybujeme hřebenem poměrně 
    pomalu, zkomplikuje nám situaci \emph{magnetizmus}. Magnetické vlivy se projevují, když jsou 
    \emph{náboje v relativním pohybu}, takže magnetické a elektrické síly je možné skutečně připsat 
    jedinému poli jako dvě stránky jedné věci. Měnící se elektrické pole nemůže existovat bez 
    magnetizmu.) Oddálíme-li nabitý papír, zpoždění je větší. V tu chvíli pozorujeme zajímavou věc. 
    Ačkoliv se síly působící mezi dvěma nabitými předměty mění nepřímo úměrně čtverci vzdálenosti, 
    při kmitání náboje zjišťujeme, že jeho působení se rozprostírá mnohem dále, než by se dalo 
    očekávat. Pokles tohoto působení je mnohem pomalejší než při nepřímé úměrnosti čtverci 
    vzdálenosti.
    
    S analogickou situací se setkáváme, když na vodě plave splávek a my ho uvedeme do pohybu 
    „přímo“ tím, že způsobíme pohyb vody jiným splávkem. Kdybychom se dívali jen na dva splávky, 
    pozorovali bychom pouze to, že jeden se dává do pohybu jako odezva na pohyb druhého, že mezi 
    nimi existuje určitá „  interakce“. Ve skutečnosti jsme ale rozčeřili vodu a voda posunula 
    druhý splávek. Mohli bychom zformulovat „zákon“, že i při slabém zčeření vody se na vodě budou 
    pohybovat předměty nacházející se blízko zdroje zčeření. Kdyby byl druhý splávek dost daleko, 
    sotva by se dal do pohybu, neboť jsme uvedli vodu do pohybu jen v jednom místě. Bude-li však 
    druhý splávek pravidelně kmitat, vznikne nový úkaz, při kterém se pohyb vody přenáší dál, 
    vzniká \emph{vlnění} a vliv poskakujícího splávku již nemůžeme chápat jako přímé působení mezi 
    splávky. Myšlenku přímé interakce tedy musíme nahradit předpokladem o existenci vody nebo v 
    případě elektrických nábojů tím, co nazýváme \emph{elektromagnetickým polem}.
    
    Elektromagnetické pole může přenášet vlny. Některé z těchto vln jsou světlo jak je znázorněno 
    na obrázku \ref{fyz:fig006}, jiné se používají při rádiovém vysílání, ale obecně se 
    nazývají \emph{elektromagnetickými vlnami}. Tyto vlny mohou mít rozmanité \emph{frekvence}. 
    Jediné, čím se jedna vlna liší od druhé, je právě frekvence vlnění. Kdybychom pohybovali 
    nábojem sem a tam a dělali bychom to stále rychleji a rychleji, objevovala by se celá řada 
    různých jevů, které je možné systematizovat udáním čísla vyjadřujícího počet kmitů za sekundu. 
    Frekvence, s nimiž přicházíme do styku prostřednictvím běžných rozvodových elektrických sítí v 
    domech, jsou řádově sto kmitů za sekundu. Zvýšíme-li frekvenci na \SI{500}{\kHz} nebo 
    \SI{1000}{\kHz} (\SI{1}{\kHz} = 1000 kmitů za sekundu), dostáváme se z domů ven, „na 
    vzduch“, neboť máme co činit s frekvencemi používanými při rozhlasovém vysílání. (Se vzduchem 
    to ale nemá co dělat! Rádiové vlny se mohou šířit i v prostoru, v němž není vzduch.) 
    Zvyšujeme-li frekvenci, dostáváme se do oblasti \emph{VKV} a televizního vysílání. Při ještě 
    vyšších frekvencích máme velmi krátké vlny, které se využívají např. v \emph{radiolokaci}. 
    Kdybychom šli ještě výše, nepotřebovali bychom už zařízení na registraci takových vln, protože 
    bychom je viděli naším zrakem. Kdybychom dokázali pohybovat nabitým hřebenem tak rychle, aby 
    kmital s frekvencemi od \SI{5e14}{\Hz} do \SI{5e15}{\Hz}, viděli bychom toto kmitání jako 
    červené, modré nebo fialové světlo v závislosti na frekvenci. Frekvence pod touto oblastí 
    nazýváme \emph{infračervenými} a nad touto oblastí \emph{ultrafialovými}. Skutečnost, 
    že naše vidění je omezeno na určitou frekvenční oblast, nedělá tuto oblast elektromagnetického 
    spektra z fyzikálního hlediska důležitější než jiné oblasti, avšak z lidského hlediska je tato 
    oblast přece jen zajímavější. Kdybychom frekvenci ještě zvýšili, dostali bychom 
    \emph{rentgenové paprsky}. Tyto paprsky nejsou nic jiného, než světlo s velmi vysokou 
    frekvencí. Ještě vyšším frekvencím odpovídá \emph{záření gama}. Výrazy rentgenové paprsky a 
    záření gama jsou téměř synonyma. Zářením gama nazýváme obvykle elektromagnetické vlny 
    pocházející z jader a rentgenovými paprsky vlny pocházející z atomů; při shodě jejich frekvencí 
    jsou však fyzikálně nerozlišitelné, bez zřetele na jejich původ. Vlny ještě vyšších 
    frekvencí, řekněme \SI{10e24}{\Hz}, lze získat uměle, například na \emph{synchrotronu} v 
    Caltechu. Elektromagnetické vlny úžasně vysokých frekvencí (až tisíckrát vyšších) je možné 
    najít ve vlnách \emph{kosmického záření}. Tyto vlny však neumíme ovládat. 
    \cite[s.~29]{Feynman02}
  
  \section{Kvantová Fyzika}\label{fyz:IchapIsecV}
    Když jsme načrtli představu elektromagnetického pole, v němž se mohou šířit vlny, brzy 
    zjistíme, že tyto vlny se chovají nezvykle, jako kdyby to ani vlny nebyly. Při vyšších 
    frekvencích se více podobají \emph{částicím}! Jejich neobvyklé chování vysvětluje 
    \emph{kvantová mechanika}, jejíž vznik je spojován s obdobím těsně po roce 1920. Před rokem 
    1920 pozměnil Einstein obraz trojrozměrného prostoru a nezávislého času nejdříve na kombinaci, 
    kterou nazýváme \emph{prostoročasem} a potom na \emph{zakřivený} prostoročas, aby vystihl 
    gravitaci. „Scéna“ se změnila na prostoročas a o gravitaci předpokládáme, že je modifikací 
    prostoročasu. Zjistilo se dokonce, že zákony pro pohyb částic jsou nepřesné. Mechanické zákony 
    „setrvačnosti“ a „síly“ jsou \emph{nesprávné} - Newtonovy zákony neplatí ve světě atomů. 
    Zjistilo se, že věci se v malém měřítku chovají úplně jinak než věci ve velkém měřítku. To dělá 
    fyziku obtížnou, ale velmi zajímavou. Obtížnou proto, že chování věcí malých rozměrů je pro nás 
    „nepřirozené“, nemáme v tomto směru přímé zkušenosti. Věci se tu chovají úplně jinak, než jsme 
    zvyklí, a proto není možné popsat jejich chování jinak, než analyticky. Takový popis je těžký a 
    vyžaduje mnoho představivosti.
    
    Kvantová mechanika má mnoho zvláštností. Především vylučuje předpoklad, že částice má určitou 
    polohu a určitou rychlost. Abychom ukázali, do jaké míry je klasická fyzika správná, uvedeme 
    pravidlo kvantové mechaniky, které říká, že není možné současně vědět, kde se něco nachází a 
    jak rychle se to pohybuje. Neurčitost v hybnosti a neurčitost v poloze jsou 
    \emph{komplementární} a jejich součin je konstantní. Můžeme to zapsat následujícím způsobem: 
    \(\Delta x \Delta p \frac{\si{\planckbar}}{2\pi}\). Podrobněji bude o tomto principu mluveno 
    později. Vysvětluje se tím velmi záhadný paradox: jsou-li atomy složeny z kladných a záporných 
    nábojů, proč se záporný náboj prostě neusadí na kladném náboji (tyto náboje se přitahují) a to 
    tak těsně, že by ho úplně vyrušil? \emph{Proč jsou atomy tak velké}? Proč je jádro uprostřed a 
    elektrony okolo něho? Zpočátku se myslelo, že příčinou je velký rozměr jádra; jenže jádro je 
    velmi malé. Atom má průměr okolo \SI{10e-10}{\meter}. Jádro má průměr asi \SI{10e-15}{\meter}. 
    Kdybychom měli atom a chtěli bychom vidět jeho jádro, museli bychom ho zvětšit tak, aby dosáhl 
    velikosti místnosti a i potom by bylo jádro malé jako skvrnka, kterou sotva spatříte okem, ale 
    téměř \emph{všechna hmotnost} atomu připadá na toto nepatrné jádro. Co brání elektronu prostě 
    spadnout na jádro? Právě uvedený princip. Kdyby elektrony byly v jádru, znali bychom přesně 
    jejich polohu a princip neurčitosti by si potom vyžadoval, aby měly velmi velkou (ale 
    \emph{neurčitou}) hybnost, tj. velmi velkou \emph{kinetickou energii}. S takovou energií by se 
    odtrhly od jádra. Dochází proto ke kompromisu: elektrony si ponechají jakýsi prostor pro tuto 
    neurčitost a potom se ve shodě s tímto pravidlem pohybují s jistým minimálním množstvím pohybu. 
    (Vzpomeňte si, že atomy krystalu při ochlazení na absolutní nulu neustaly ve svém pohybu, ale 
    přece jen kmitaly. Proč? Kdyby se přestaly pohybovat, věděli bychom, kde se nacházejí a že mají 
    nulový pohyb a to by bylo v rozporu s principem neurčitosti. Nemůžeme vědět, kde jsou a jak 
    rychle se pohybují; proto atomy musí neustále kmitat!)
    
    Jinou, velmi zajímavou změnou v ideách a filozofii vědy, kterou přinesla kvantová mechanika, je 
    nemožnost přesně předpovědět, co se za jakýchkoli daných okolností odehraje. Například, je 
    možné připravit atom, který bude emitovat světlo, a můžeme zjistit, kdy k této emisi došlo tím, 
    že zachytíme foton (o tomto si brzy řekneme více). Nemůžeme však dopředu předpovědět, kdy se 
    uskuteční emise světla, nebo v případě více atomů, který z nich bude emitovat světlo. Možná se 
    domníváte, že je to proto, že v atomu se nacházejí jakási vnitřní „kolečka“, která jsme ještě 
    nerozeznali. Ne, taková vnitřní kolečka neexistují! Příroda, tak jak ji dnes chápeme, se chová 
    tak, že je principiálně nemožné přesně předpovědět, co se skutečně stane v daném experimentu. 
    
    Opět se vrátíme ke kvantové mechanice a základní fyzice, ale nebudeme zabíhat do podrobností 
    kvantově mechanických principů, protože jsou dost těžké k pochopení. Budeme prostě předpokládat 
    jejich existenci a ukážeme, k jakým následkům vedou. Jedním z následků je, že věci, které jsme 
    považovali za vlny, se chovají jako částice a částice zase jako vlny; ve skutečnosti se tedy 
    všechno chová stejně. Není rozdíl mezi vlnou a částicí. \textbf{Kvantová mechanika sjednocuje 
    myšlenku pole, jeho vln a částic vjedno.} Při nízkých frekvencích je aspekt pole více zřejmý, 
    resp. užitečnější pro přibližný popis vyjádřený řečí naší každodenní zkušenosti. Se vzrůstem 
    frekvence však zařízení, které obvykle používáme v experimentu, poskytuje spíše důkazy o 
    částicích. I když mluvíme o vysokých frekvencích, musíme přiznat, že v oblasti frekvencí nad 
    \SI{10e12}{\Hz} nebyl zatím zjištěn žádný jev přímo související s frekvencí. K existenci 
    vyšších frekvencí docházíme pouze úvahou vycházející z energie částic a předpokladu správnosti 
    \emph{vlnově-korpuskulární představy kvantové mechaniky}.
    
    Takto docházíme i k novému pohledu na \emph{elektromagnetickou interakci}. Kromě elektronu, 
    protonu a neutronu existuje nový druh částice. Tuto částici nazýváme foton. Nový pohled na 
    interakci elektronů a protonů, tj. \emph{elektromagnetickou teorii}, která zároveň 
    \emph{splňuje} zákonitosti \emph{kvantové mechaniky}, nazýváme \emph{kvantovou 
    elektrodynamikou}. Tato základní teorie \emph{interakce světla a hmoty}, nebo 
    \emph{elektrického pole a nábojů}, je dosud největším úspěchem fyziky. V této jediné teorii 
    máme základní zákony, jimiž se řídí všechny známé jevy s výjimkou gravitace a jaderných 
    procesů. Pomocí kvantové elektrodynamiky můžeme vysvětlit všechny známé zákony mechaniky, 
    elektřiny a chemie. Plynou, zní zákony srážek kulečníkových koulí, pohyb vodičů v magnetickém 
    poli i tepelná kapacita oxidu uhelnatého, barva neonových reklam, hustota soli, reakce vodíku a 
    kyslíku při vzniku vody - to vše jsou následky jediného zákona. Všechny tyto detaily je možné 
    získat, je-li situace dost jednoduchá na to, abychom ji mohli přibližně popsat. To sice není 
    splněno téměř nikdy, často však můžeme pochopit více či méně, co se vlastně děje. Dosud se 
    neobjevily žádné výjimky ze zákonů kvantové elektrodynamiky, až na atomová jádra. O jádrech 
    však nemůžeme říci, jestli jde v jejich případě o výjimku, protože vlastně nevíme, jaké procesy 
    v nich probíhají. Při budování teorie jádra musíme překonat tři hlavní problémy:
    \begin{enumerate}[noitemsep]
     \item Není znám přesný tvar sil působících mezi nukleony v jádře,
     \item rovnice popisující pohyb nukleonů v jádře jsou velmi komplikované - problém  
           matematického popisu,
     \item jádro má zároveň příliš mnoho nukleonů (nedá se popsat pohyb každé jeho částice) i    
           příliš málo (nedá se popsat jako makroskopické spojité prostředí).   
    \end{enumerate}
    Proto se musíme spokojit pouze s modely atomového jádra. 
    
    V podstatě je kvantová elektrodynamika teorií celé chemie a všech životních procesů, je-li 
    možné život v konečném důsledku redukovat na chemii, nebo vlastně na fyziku, protože chemie 
    vede k fyzice (a ta část fyziky, která se uplatňuje v chemii, je již dobře známá). Navíc, 
    kvantová elektrodynamika - ta úžasná vědní disciplína - předpověděla mnoho nových věcí. 
    Především mluví o vlastnostech fotonů velmi velkých energií, paprscích gama apod. Předpověděla 
    i jinou, velmi pozoruhodnou věc: kromě elektronu musí existovat jiná částice se stejnou 
    hmotností, ale s opačným nábojem, tzv. \emph{pozitron} a elektron s pozitronem mohou při srážce 
    anihilovat, přičemž se vyzáří světlo nebo paprsky gama (což je vlastně totéž, neboť světlo i 
    záření gama se liší polohou ve frekvenční škále elektromagnetických vln). Zobecnění poznatku, 
    že ke každé částici existuje antičástice, se ukazuje být pravdivým. V případě elektronů má 
    antičástice jiné jméno - nazývá se pozitronem, ale u většiny jiných částic mluvíme o anti-tom a 
    tom, např. o antiprotonu nebo antineutronu. Do kvantové elektrodynamiky se vkládají \emph{dvě 
    čísla} a o většině ostatních čísel ve světě se předpokládá, že jsou následkem těchto dvou. Tato 
    dvě vkládaná čísla nazýváme hmotností a nábojem elektronu. Ve skutečnosti to však není úplně 
    tak, neboť máme celý soubor chemických čísel, která hovoří o tom, jak těžká jsou jádra. To nás 
    přivádí k další kapitole.
  
  \section{Jádra a Částice}\label{fyz:IchapIsecVI}
    \emph{Z čeho jsou jádra a jak drží pohromadě}? Zjistilo se, že jádra jsou udržována obrovskými 
    silami. Při uvolnění těchto sil se uvolňuje energie, která je obrovská v porovnání s chemickou 
    energií, tak jak je obrovský výbuch atomové bomby v porovnání s výbuchem trinitrotoluenu. U 
    atomové bomby jde totiž o změny uvnitř jádra, zatímco výbuch trinitrotoluenu souvisí se změnami 
    elektronového obalu atomů. Proto si klademe otázku: co jsou to za síly, které udržují protony a 
    neutrony v jádře pohromadě? Tak, jako je možné elektrické působení přisoudit částici - fotonu, 
    předpokládal Yukawa, že i síly mezi neutrony a protony mají svá pole a kmity tohoto pole se 
    chovají jako částice. Kromě neutronů a protonů by proto měly existovat jiné částice a Yukawa 
    odvodil vlastnosti těchto částic z již známých charakteristik jaderných sil. Například, 
    předpověděl, že by měly mít hmotnost dvěstě až třistakrát větší než elektron; a div se 
    světe - v kosmickém záření byly objeveny částice s takovouto hmotností! Později se ukázalo, že 
    to nebyla ta správná částice. Tuto částici nazvali \(\mu\text{-mezon}\) neboli \emph{mion}.

    \begin{figure}[hbt!]  % \ref{fyz:fig895}
      \centering
      \luafigure[0.8]{fyz_fig895.pdf}
      \caption{ \wikiAtomJadro: Stylizovaný model atomu helia. Atomové jádro je vnitřní kladně
                nabitá část atomu a tvoří jeho hmotnostní i prostorové centrum. Atomové jádro
                představuje \SI{99.9}{\percent} hmotnosti atomu. Průměr jádra činí přibližně
                \SIrange{10}{15}{\m}, což je přibližně \(\num{100 000}\times\) méně než průměr
                celého atomu. Existence atomového jádra byla poprvé pozorována v Rutherfordově
                experimentu, na jehož základě vznikl tzv. planetární model atomu.}
      \label{fyz:fig895}
    \end{figure} 
    
    Trochu později, v roce 1947 nebo 1948, byla objevena jiná částice, \(\pi\text{-mezon}\) neboli 
    \emph{pion}, která vyhovovala Yukawovu kritériu. Abychom získali jaderné síly, musíme k protonu 
    a neutronu přidat pion. A teď si řeknete: „Och, jak velkolepé! - pomocí této teorie vybudujeme 
    nukleodynamiku, ve které budou mít piony takovou úlohu, jakou jim přisoudil Yukawa a všechno 
    bude vysvětleno“. Ta věc má však háček! Ukázalo se, že výpočty v této teorii jsou tak složité, 
    že se dodnes nikomu nepodařilo odvodit všechny důsledky této teorie, nebo ji porovnat s 
    experimentem; a to se už táhne spoustu let!
    
    Máme tedy teorii, ale nevíme, jestli je správná nebo nesprávná. Víme však už, že je trochu 
    chybná, nebo aspoň neúplná. Zatím co jsme marnili čas teorií a snažili se odvodit její 
    důsledky, experimentátoři některé věci objevili. Například, objevili \(\mu\text{-mezon}\) 
    neboli mion a my ani nevíme, jaká je jeho úloha. V kosmickém záření se našel velký počet 
    dalších „přebytečných“ částic. Dnes máme přibližně třista takových částic a je velmi těžké 
    porozumět vztahům mezi těmito částicemi a pochopit, na co je příroda potřebuje, nebo která z 
    nich na které závisí. Dnes tyto různé částice nechápeme jako různé aspekty téže věci a 
    skutečnost, že máme tak mnoho nesouvisejících částic, je odrazem toho, že máme tak mnoho 
    nesouvisejících informací bez dobré teorie. Po ohromném úspěchu kvantové elektrodynamiky máme 
    jisté znalosti z jaderné fyziky, ale jen hrubé znalosti, částečně experimentální a částečně 
    teoretické. Vycházíme přitom z charakteru sil působících mezi protony a neutrony a sledujeme, 
    co z toho vyplyne, ale v podstatě nechápeme, odkud ty síly pocházejí. Kromě toho nebylo 
    dosaženo téměř žádného pokroku. Objevili jsme velký počet chemických prvků. Mezi těmito prvky 
    se najednou objevila souvislost, neočekávaná souvislost zakotvená v Mendělejevově periodické 
    tabulce prvků. Například, sodík a draslík jsou téměř shodné ve svých chemických vlastnostech a 
    v Mendělejevově tabulce se nacházejí ve stejném sloupci. Hledala se tabulka Mendělejevova typu 
    pro nové částice. Taková tabulka nových částic byla sestavena nezávisle Gell-Mannem v USA a 
    Nishijimou v Japonsku. Základem jejich klasifikace je nové číslo, jež je možno, podobně jako 
    elektrický náboj, přiřadit každé částici a které se nazývá její „podivností“ S (od anglického 
    slova strangeness). Toto číslo se, podobně jako elektrický náboj, zachovává v reakcích 
    vyvolávaných jadernými silami.  
  
  \section{Vědecká revoluce 17. století}\label{fyz:IchapIsecVII}
    \textbf{Klasická fyzika}, jak ji popsal Richard Feynnman v předchozích kapitolách, tedy jako
    věda vycházející z měření a experimentů a opírající se o matematickou teorii, věda, která nám
    podává ucelený obraz přírody a světa a svými výsledky slouží technickému pokroku, vznikla v
    Evropě v průběhu sedmnáctého století. Tento dějinný převrat, který předznamenal naši dnešní
    civilizaci, nazýváme \textbf{obdobím vědecké revoluce}. Nebyla to ovšem nějaká náhlá událost a
    lidé si tehdy ani neuvědomili, jakou vlastně prožívají dobu a co přinese budoucím pokolením.
    Vědecká revoluce nastala za zvláštních podmínek evropského vývoje, které se v jiných částech
    světa nevytvořily.

    \begin{mdframed}[style=mdnote]
      \begin{note}
        \textbf{Kuhnovo pojetí vývoje vědy}: \textsc{Thomas Samuel Kuhn} přinesl argumenty o tom, že
        pokrok vědeckého poznání není přímočarý, ale že je čas od času přerušován zásadními
        zvraty-vědeckými revolucemi. Při těchto vědeckých revolucích dochází k přehodnocení
        samotných základů dosavadního vědění. Vědecké poznání tedy nesměřuje k nějaké jediné pravdě
        o světě, netýká se žádné „objektivní reality“ - nezávislé skutečnosti, všem společné, vždy
        zde již jsoucí. Věda, tak jako každá lidská činnost, má svůj kulturní, dějinný, instituční,
        sociální a psychologický rozměr. I vědecké poznatky jsou proto historicky podmíněné:
        vyjadřují ducha dané epochy, mění se s dobou i s okolnostmi.

        {\centering
        \captionsetup{type=figure} 
        \luafigure[0.5]{fyz_fig893.jpg}
        \captionof{figure}{\wikiKuhn (\textasteriskcentered 18. 7. 1922 - \textdagger 17. 6. 1996)
                  byl americký filosof, fyzik, teoretik vědy a vědeckého poznání, zabýval se
                  dějinami vědy, astronomií, kvantovou teorií a její prehistorií.}
        \label{fyz:fig893}
      \par}
      \end{note}
    \end{mdframed}

    Příčin, které vyvolaly tuto vědeckou revoluci, bylo mnoho a nemůžeme je zde podrobně zkoumat.
    Především to byly nové politické a hospodářské podmínky, potřeby výroby, obchodu a podnikání,
    které vyzvedly do popředí nové společenské síly, především měšťanské. Vzrůstající produktivita
    práce a vznik prvních kolektivních dílen, manufaktur, potřebovaly nové způsoby silového pohonu.
    Zásobování surovinami a vývoz hotových výrobků si vyžádal rozvoj mořeplavby a námořní navigace.
    Evropské války, jak už to bývá, také podnítily zdokonalování vojenské techniky a nepřímo i
    rozvoj přírodních věd \cite[s.~137]{Stoll2009}.

    Důležitou úlohu sehrála reformace, odklon řady zemí v západní a severní Evropě od katolické
    církve a papežství a vznik nových, protestantských církví. Protestantismus usiloval o bližší
    kontakt jednotlivého člověka s Bohem, bez prostřednictví církevní hierarchie, o návrat k podobě
    bible v jejich původních jazycích (hebrejském a řeckém) a podnítil vznik překladů biblických
    textů do národních evropských jazyků. Tím na jedné straně vyvolal potřebu studia klasických
    jazyků a umožnil také zpřístupnění výsledků vědy starověkého Řecka a na druhé straně podpořil
    rozvoj národních jazyků (připomeňme si jen krásnou češtinu naší Kralické bible). Latina, ve
    středověku univerzální jazyk vzdělanců, začala ztrácet své výsadní postavení.

    S rostoucím vědomím užitečnosti a nutnosti vědeckého poznání bez vměšování teologického
    dogmatismu přenášejí protestanti těžiště náboženského cítění do oblasti morální, jako vodítko
    při hledání smyslu lidského života, a ponechávají přírodním vědám zkoumání a využívání
    přírodních zákonů. Odmítají víru v Boží zázraky, která vlastně znemožňuje existenci vědy. Takový
    přístup, kdy Bůh je chápán jen jako stvořitel a první zákonodárce, který se však do dalšího
    chodu přírody už nevměšuje, nazýváme \textbf{deismus}, na rozdíl od katolického teismu, podle
    něhož Bůh do běhu světa stále zasahuje a bez jehož vůle, ani vlas z hlavy nespadne“. Protože
    nositeli idejí protestantismu byly především měšťanské a hospodářsky aktivní vrstvy společností,
    rozvíjí se věda a vědecká revoluce zejména v protestantských zemích západní Evropy v Holandsku,
    Anglii, Švýcarsku, Dánsku, částečně ve Francii a Německu. Také příznačný podnikatelský duch
    Ameriky má své kořeny v anglosaském protestantismu prvních přistěhovalců. Katolická Itálie,
    která renesanci vědy zahájila, nakonec odsoudila svého Galilea, katolické Španělsko a
    Portugalsko, které zbohatly při zámořské kolonizaci, postupně svou moc ztrácejí a k vědecké
    revoluci v Evropě nepřispívají.

    Evropa nebyla nikdy soběstačná v některých druzích výrobků, ať už šlo o tropické plody, rostliny
    (bavlna, cukrová třtina) a koření, drahé kovy, ale třeba i hedvábí, vzácné kožešiny a jiné
    výnosné luxusní předměty. Obchodní cesty k jejich získávání vedly odedávna přes Středozemí,
    Blízký a Střední Východ a na tomto obchodu bohatly zejména italské městské státy jako Benátky
    nebo Janov. Když postupující turecká expanze tyto přístupové cesty znesnadnila a ohrozila,
    hledaly státy západní a jihozápadní Evropy přístup na východní trhy obeplutím Afriky a po
    úspěšných výpravach Kolumbovych západním směrem do Ameriky.

    Španělsko a Portugalsko začaly z těchto nových cest a výbojů těžit jako první, jejich karavely a
    galeony, obtížené kořením, stříbrem a zlatem, přivážely toto zboží na evropské trhy, pokud
    neskončilo na mořském dně nebo v rukou pirátů. Obě tyto námořní mocnosti si známými smlouvami z
    Tordesillas (1494) a Zaragozy (1529) dokonce rozdělily celý svět na dvě poloviny a uskutečnily
    tak první globalizaci světového obchodu a kolonizace ve znamení katolicizmu.

    Nedokázaly však své nové hospodářské zdroje produktivně využít. Jejich pozice zaujala postupně
    Anglie, Francie, a zejména malé protestantské Holandsko, které se začátkem 17. století
    osvobodilo od španělské nadvlády a vytvořilo republiku pod vládou místodržitelů z rodu
    Oranžskeho je téměř neuvěřitelné, že Holandsko, počtem obyvatel srovnatelné s tehdejším českým
    královstvím, vytvořilo jeden čas největší koloniální říši světa a disponovalo flotilou 16 000
    lodi, počtem trojnásobně převyšujícím flotilu všech ostatních západoevropských států dohromady.
    Hospodářsky se postupně vzmáhala i Anglie, kde společenské napětí vyvrcholilo občanskou válkou a
    revoluci, která přivedla v roce 1649 krále Karla I. na popraviště. Všechny tyto společenské
    otřesy a změny v západní Evropě postupně vytvářely nové impulzy k rychlému vědeckému a
    technickému pokroku.

    Koloniální výboje vyvolaly potřebu mapovat nová území, dokonce mapovat zeměkouli jako celek,
    především přesně měřit zeměpisnou šířku a délku, ale i hloubku moří, teplotu a slanost mořské
    vody, rychlost a směr mořských proudů a magnetickou deklinaci, odchylku směru udávaného kompasem
    od pravého severu. To ovšem vyžadovalo prozkoumat přesný geometrický tvar zeměkoule a vytvořit
    nové fyzikální a astronomické měřicí metody a přístroje.

    Největší problém činilo určování zeměpisné délky. Dokud se Evropané ve starověku a středověku
    plavili v útulném Středomoří, kde bylo možno z každého místa doplout za jeden den k nejbližšímu
    pobřeží, nebo když Vikingové provozovali pobřežní plavbu podél západoevropských břehů, nebyla
    tato otázka příliš naléhavá. Jakmile se ovšem Kolumbus vydal na neprobádanou cestu na západ
    Atlantickým oceánem a začal překračovat další a další poledníky, mohl určovat svou polohu jen
    podle rychlosti lodi, měřené nedokonalým plavboměrem, a porovnávat místní čas s časem ve
    výchozím přístavu, odměřovaným přesýpacími hodinami. Ty měl plavčík za úkol každou čtvrthodinu
    převracet, a záleželo tak i na jeho problematické svědomitosti. Kolumbus ostatně údaje o
    zeměpisné délce sám upravoval, aby posádka neměla představu, jak daleko na západ už dopluli.
    Dost na tom, že námořníci byli vyděšeni tím, že jim střelka kompasu přestala ukazovat na
    Polárku.

    Když si však někdo chce dělit zeměkouli napůl, musí být schopen určovat zeměpisnou délku přesně.
    Potřebuje k tomu dalekohled, sextant, astronomické znalosti a přesné lodní hodiny - chronometr.
    To si uvědomil dokonce i anglický král Karel II., když se na něj v roce 1675 obrátil astronom
    \textsc{John Flamsteed} (1646-1719) s návrhem na zřízení státní, tedy královské hvězdárny. V
    královském rozhodnutí se založení hvězdárny výslovně zdůvodňuje \emph{„aby bylo možno zjišťovat
    zeměpisnou délku míst ke zdokonalení navigace a astronomie."} Král se dokonce vzdal svého
    honebního revíru na stráni v Greenwichi na pravém břehu Temže (byla stejně holá a málo
    zvěřinatá) a souhlasil s tím, aby tam byla z použitého stavebního materiálu vybudována
    observatoř. Zároveň zavedl novou funkci a jmenoval Flamsteeda ,,královským astronomem". Ten
    musel investovat do vybavení hvězdárny své vlastní finanční prostředky a v podstatě živořil.
    Současně vznikla ve Francii i královská pařížská observatoř, kam byl z Itálie povolán astronom
    \textsc{Giovanni Domenico Cassini} (1625-1712), jehož potomci ho následovali v této funkci v
    několika generacích.

    Vědecká revoluce v Evropě byla tedy vyvolána naléhavými praktickými potřebami, ale měla
    připraveno i myšlenkové, filozofické zázemí. Postupně se prosazoval světový názor založený na
    Koperníkově modelu sluneční soustavy a astronomická měření ho stále přesvědčivěji potvrzovala.
    Vědecká metoda zkoumání se mohla opřít o výsledky práce myslitelů, kteří stoji u počátků
    novověké evropské filozofie. věku evropské filozofie. V Anglii to byl \textsc{Francis Bacon}
    (1561-l626) \textsc{René Descartes} (1596-1650). Oba představuji poněkud odlišné, ale vzájemně
    se doplňující přístupy ke zkoumání přírody a charakterizují různé směry, jimiž se ubírala
    vzájemně soupeřící anglická a francouzská fyzika té doby. 
    
    Bacon zastával v Anglii vysoké státní funkce. Zdůrazňoval význam vědění, které dává člověku
    obrovskou moc, a zabýval se myšlenkami \uv{velkého obnovení věd}, které by přinášelo lidem
    užitek a přispělo i k lepší organizaci lidské společnosti. Ve svém spise \uv{Nové organon} z
    roku 1620 reaguje na Aristotelovo dílo ,,Organon", odmítá čistě spekulativní, scholastickou
    aristotelovskou logiku a vychází z empirického, smyslového poznání, pozorování a pokusů. Je
    zakladatelem vědecké indukce, tedy metody, která logicky analyzuje a třídí zkušenosti, fakta a
    dospívá k obecným zákonitostem. Přitom se vědec musí oprostit od předsudků a vžitých představ,
    které Bacon nazývá  \uv{idoly}. Ve svém zaujetí pro pokusy šel Bacon tak daleko, že zemřel na
    zápal plic právě když zkoumal dlouhodobý vliv chladu na živý organismus. Bacon je představitelem
    anglického empirismu, který zapůsobil i na anglické fyziky včetně Newtona.  
    
    Ve Francii ovlivnil filozofické myšlení především Descartes (latinsky Kartesius). Pocházel z
    aristokratického katolického rodu, od dětství byl chabého zdraví a prošel složitým myšlenkovým
    vývojem. Navštěvoval jezuitskou kolej, studium ho však neuspokojilo, a naopak v něm rozvířilo
    mnoho pochyb. Studoval práva i medicínu, jako dobrovolník v holandském a pak v bavorském vojsku
    prošel Evropou i Čechami a někdy se uvádí, že se účastnil i bitvy na Bílé hoře.  Na dlouhých
    dvacet let pak zakotvil v Holandsku, kde se v červenci 1642 sešel i s Janem Amosem Komenským, i
    když se s ním filozoficky nepohodl. Descartovy názory narážely na odpor a vyvolávaly útoky ze
    strany jak katolických, tak protestantských kruhů a tyto útoky poněkud plachého Descarta
    deprimovaly. Descartes byl zastáncem Koperníkova názoru na sluneční soustavu, ale po Galileově
    odsouzení se zalekl a byl ve formulaci svých názorů vysloveně opatrný. Aby si zajistil větší
    klid k práci, často dokonce měnil místo svého pobytu. Jeho vědecké dílo mělo i řadu stoupenců a
    vzbudilo nakonec zájem švédské královny Kristýny. Pozvala Descarta do Stockholmu a ten ji musel
    vyučovat filozofii třikrát týdně od pěti hodin ráno. Descartes, který byl zvyklý vstávat až k
    poledni, takový režim, znásobený drsným severským podnebím, ovšem dlouho nepřežil. V únoru 1650
    zemřel na zápal plic a v r. 1666 byly jeho ostatky převezeny do Paříže. Dnes je pohřben ve
    starobylém kostele Saint Germain-des-Prés, jeho lebka, která byla při převozu ostatků zcizena,
    odděleně v Museu člověka v Paříži. 
    
    Descartes je zakladatelem francouzského racionalismu. Je znám jeho výrok \uv{Cogito erg sum},
    \uv{Myslím, tedy jsem} a na základě rozumových úvah také založil svou vědeckou metodu. Ve svém
    slavném spise \uv{Rozprava o metodě} stanoví pravidla správného vědeckého uvažování. Jako první
    krok požaduje zpochybnit všechny dosavadní názory a tvrzení, pokud nejsou nade vší pochybnosti
    Descartes je zakladatelem francouzského \emph{racionalizmu}. Je znám jeho v rok \uv{Cogito ergo
    sum}, \uv{Myslím, tedy jsem} a na základě rozumových úvah také založil svou vědeckou metodu. Ve
    svém slavném spise \uv{Rozprava o metodě} stanoví pravidla správného vědeckého uvažování. Jako
    první krok požaduje zpochybnit všechny dosavadní názory a tvrzení, pokud nejsou nade vší
    pochybnost dokázány. Jeho \uv{De omnibus dubitandum}, \uv{O všem pochybovat}, znamená začínat
    zkoumání s čistou a nepředpojatou myslí. Dále požaduje rozdělit každou zkoumanou otázku na
    části, které by bylo možno lépe řešit. Při zkoumání je třeba postupovat od předmětů
    jednodušších, které lze snáze poznávat, ke složitějším. A konečně za čtvrté je třeba uspořádávat
    zjištěná fakta do výčtů a přehledů, aby nic nebylo opomenuto. Tato Descartova doporučení jsou
    jakýmsi základem vědecké metody rozumového zkoumání; týmž způsobem musí ostatně postupovat i
    detektiv při řešení složitého kriminálního případu. Descartes je tak zakladatelem analytické
    deduktivní metody, která vychází z několika málo obecných principů a zákonů a postupuje podle
    pravidel rozumového uvažování.

    Descartova filozofie, karteziánství, ovlivnila celou řadu pozdějších filozofů a myslitelů.
    Patřil k nim např. \textsc{Benedikt (Baruch) Spinoza} (1632-1677), holandský filozof
    portugalsko-židovského původu, ale i Leibniz, Pascal a další. Spinoza se pokusil pomocí
    Descartovy racionalistické filozofie a axiomatické metody geometrie vyložit i taková témata,
    jako je politika, etika nebo teologie. Dochází k závěru, že existuje jen jedna jediná substance,
    jíž je Bůh ztotožněný s přírodou. Takový názor, podle něhož se nic a nikdo do přírody zvnějšku
    nevměšuje se nazývá \emph{panteizmem}. Zmiňujeme se o něm proto, že je blízký chápání velkých
    fyziků. Ti byli uchváceni krásou a řádem přírody, a ta jim splývala s božstvím v jedno. Ke
    Spinozově panteizmu se hlásil např. i Einstein \cite[s.~141]{Stoll2009}.    
    
  \section{Integrační tendence ve fyzice}\label{fyz:IchapIsecVIII}
    Není to tak dávno, co se fyzikové dělili na dvě velké skupiny – experimentátory a teoretiky.
    Příslušník každé skupiny věděl, že se bez členů druhé skupiny neobejde. Výsledkem byla plodná
    spolupráce plná zdánlivé řevnivosti a úsměvných historek. S nástupem výpočetní techniky se vše
    změnilo. Postupně vznikala skupina třetí, která se zabývá numerickými simulacemi. Bez nich si
    dnes fyziku nedovedeme představit. Numerické simulace umožňují první ověření výsledků nových
    teorií bez nákladných experimentů. Při zpracování experimentálních dat pomáhají hledat procesy,
    které se za naměřenými údaji skrývají. V současnosti má fyzika tři nedílné celky: teorii,
    experiment a numerické simulace. Tato učebnice je věnována, jak její název říká, vybraným
    kapitolám z teoretické fyziky. Je úvodem do teoretické mechaniky, kvantové teorie a statistické
    fyziky. 
    
    Fyzika zaznamenává v průběhu staletí dvě základní tendence. První z nich je postupné členění na
    další a další podobory. Tento vývoj souvisí s prohlubujícím se poznáním a je přirozenou cestou v
    každé vědní disciplíně. Postupně vznikají specialisté na stále užší a užší obory, vytvářejí si
    svůj vlastní vědecký jazyk a schopnost komunikace odborníků z dříve blízkých oblastí fyziky se
    stále zhoršuje. Na druhé straně dochází k hlubšímu pochopení souvislostí mezi jednotlivými
    částmi fyziky a k jejich postupnému sjednocování do univerzálnějších teorií. Možná se jednou
    podaří sjednotit fyzikální pohled na všechny základní přírodní interakce do jedné jediné teorie,
    kterou dnes nazýváme Teorie všeho (anglicky TOE, Theory Of Everything). Tyto integrační tendence
    ve fyzice jsou znázorněny na obrázku 1. 

    \luagraphic[1]{fyz_fig924.pdf}{Integrační tendence ve fyzice.
    (\cite[s.~12]{Kulhanek2019})}{fyz:fig924}
    
    Mechanika jakožto vědecká fyzikální disciplína vznikala od 17. století. První známější vědecké
    experimenty prováděl \textsc{Galileo Galilei} (1564–1642). Teoretickou konstrukci klasické
    mechaniky, jakožto nástroje pro předpověď pohybu těles v daném silovém poli, navrhnul
    \textsc{Isaac Newton} (1642–1727) ve svých \emph{Principiích (Philosophiæ Naturalis Principia
    Mathematica)} z roku 1687. V 18. století dovršil konstrukci klasické mechaniky \textsc{Joseph
    Louis Lagrange} (1736–1813), který mechanické úlohy formuloval nezávisle na volbě souřadnicové
    soustavy za pomoci variačního počtu.
    
    V 19. století se úspěšně dařilo poznávat a postupně chápat elektrické a magnetické děje. Na
    experimentech se podílela celá řada významných fyziků, například \textsc{Hans Oersted}
    (1777–1881), \textsc{André Ampère} (1775–1836), \textsc{ichael Faraday} (1791–1867),
    \textsc{Heinrich Hertz} (1857–1894), \textsc{Oliver Heaviside} (1850–1925) a další. Celé toto
    údobí vyvrcholilo poznáním, že jevy elektrické a magnetické mají shodnou povahu a společný
    původ. V roce 1873 publikoval \textsc{James Clerk Maxwell} (1831–1879) pojednání \uv{A Treatise
    on Electricity and Magnetism}, které obsahovalo rovnice, jež završily klasickou elektrodynamiku
    do jednoho jediného celku obsahujícího jak děje elektrické, tak magnetické. 
    
    Na konci 19. století podlehlo mnoho fyziků iluzi, že fyzika jako věda je dokončena. Byly známy
    zákony mechaniky na jedné straně a zákony elektřiny a magnetizmu na straně druhé. Na první
    pohled se zdálo, že veškeré přírodní děje jsou důsledkem těchto dvou vědních disciplin a
    budoucnost fyziky je pouze v aplikaci známých zákonů na neznámé situace. Šlo samozřejmě o krutý
    omyl, který se rychle projevil na počátku dvacátého století, kdy nebylo možné tehdejšími
    znalostmi vysvětlit řadu fyzikálních dějů. 

    Ukázalo se, že jak klasická mechanika, tak klasická elektrodynamika nedokáží uspokojivě popsat
    svět na úrovni atomů. Důsledkem toho byla neschopnost objasnit chování elektronu v atomárním
    obalu, vysvětlit záření absolutně černého tělesa, pochopit fotoelektrický jev a smířit se s
    projevy objektů mikrosvěta, které vykazovaly někdy částicové a jindy vlnové vlastnosti. Zrodila
    se kvantová mechanika, ve které neplatí \(ab = ba\), a nekomutativnost se stala nově objeveným
    rysem přírody na mikroskopické úrovni. Kvantová mechanika s sebou přinesla celou řadu těžko
    představitelných jevů – kvantování energie a momentu hybnosti, dualismus vln a částic, relace
    neurčitosti, nejednoznačnost aktu měření a pravděpodobnostní interpretaci výsledků vedoucí na
    nedeterminizmus kvantové fyziky.

    A to byl teprve začátek. Spin elementárních částic objevený v roce 1925 znamenal další výrazný
    posun lidstva v chápání přírody. Je důsledkem relativistické fyziky, která se od počátku 20.
    století rozvíjela paralelně s kvantovou mechanikou. Spojení kvantové mechaniky se speciální
    relativitou vedlo na Diracovu rovnici, která se stala základem kvantového popisu pohybu
    elektronu. \textsc{Paul Adrien Maurice Dirac} (1902–1984) navrhnul svou rovnici v roce 1928 a
    téhož roku z ní odvodil existenci pozitronu, antičástice k elektronu. Pozitron byl
    experimentálně objeven až o 4 roky později \textsc{Carlem Andersonem} (1905–1991). Za svou práci
    získal Dirac Nobelovu cenu za fyziku pro rok 1933. V letech 1946 až 1949 byla dokončena první
    kvantově polní teorie – \emph{kvantová teorie elektromagnetického pole}, které dnes říkáme
    \textbf{kvantová elektrodynamika} (\emph{QED, Quantum Electro-Dynamics}). Za její formulaci
    získali Nobelovu cenu za fyziku pro rok 1965 \textsc{Richard Feynman} (1918–1988),
    \textsc{Shin-Itiro Tomonaga} (1906–1979) a \textsc{Julian Schwinger} (1918–1994). Kvantová
    elektrodynamika je kvantovou analogií Maxwellových rovnic. Elektromagnetická interakce je
    způsobena polními částicemi, v tomto případě fotony, které si mezi sebou posílají nabité
    částice. Klasický pojem síly ztrácí svůj smysl. Feynmanovi se podařilo složité rovnice
    interpretovat za pomoci názorných grafů, kterým dnes říkáme \emph{Feynmanovy diagramy}. Na
    obdobném základě byla později vytvořena také současná \textbf{kvantová teorie slabé a silné
    interakce}. Základním rysem těchto teorií jsou tzv. \emph{kalibrační symetrie}, které předurčují
    způsob působení dané interakce na elementární částice.

    Od počátku 60. let probíhaly snahy o spojení elektromagnetické a slabé interakce do jednoho
    jediného celku. Podařilo se to \textsc{Stevenu Weinbergovi} (1933), \textsc{Abdusu Salamovi}
    (1926–1996) a \textsc{Sheldonu Glashowovi} (1932). Za svou práci získali Nobelovu cenu za fyziku
    pro rok 1979. Jimi předpovězené polní částice slabé interakce \(W^+\), \(W^–\) a \(Z^0\) byly
    objeveny na přelomu let 1983 a 1984 v evropském středisku jaderného výzkumu CERN. Jejich
    objevitelé, \textsc{Carlo Rubbia} (1934) a \textsc{Simon van der Meer} (1925–2011) získali
    Nobelovu cenu ještě téhož roku (1984).

    K pochopení silné interakce přispěl již ve 30. letech japonský fyzik \textsc{Hideki Yuakawa}
    (1907–1981). Za svou práci získal Nobelovu cenu za fyziku pro rok 1949. Současná kvantově polní
    teorie silné interakce se nazývá \textbf{kvantová chromodynamika} (\emph{QCD, Quantum
    Chromo-Dynamics}) a za její formulaci a zejména za objev asymptotické volnosti silné interakce
    kvarků a gluonů získali Nobelovu cenu za fyziku pro rok 2004 \textsc{Frank Wilczek} (1951),
    \textsc{David Gross} (1941) a \textsc{David Politzer} (1949).
    
    Kvantová mechanika slavila v průběhu 20. století mimořádné úspěchy. Jednoduchá teorie popisující
    mechanické děje postupně přerostla v polní kvantovou teorii schopnou úspěšně popsat hned tři ze
    čtyř základních přírodních interakcí. Tato cesta se samozřejmě neobešla bez potíží a problémů,
    nicméně vyústila v dnešní \textbf{standardní model elementárních částic a interakcí}. Bez
    kvantové teorie a hlubokého pochopení zákonitostí mikrosvěta bychom dnes neměli ani počítače ani
    jinou elektroniku. 

    Na počátku 20. století ale vznikala ještě jedna, neméně úspěšná teorie – obecná relativita. Z
    Maxwellovy elektrodynamiky plynulo, že rychlost světla by ve vakuu měla být univerzální
    konstantou a že by se neměla sčítat s rychlostí zdroje elektromagnetického vlnění. Tento
    výsledek byl na první pohled v rozporu s klasickou mechanikou, ve které se rychlost zdroje s
    rychlostí signálu sčítá. Řada experimentů potvrdila správnost elektrodynamiky. Bylo tedy třeba
    přeformulovat mechaniku tak, aby byla v souladu s Maxwellovou elektrodynamikou. To se v roce
    1905 podařilo Albertu Einsteinovi v rámci tzv. \textbf{speciální teorie relativity}. Daň za
    sjednocení obou teorií byla veliká. Čas spolu s prostorem přestaly být absolutní. Délka letící
    tyče a časový úsek mezi dvěma událostmi ve skutečnosti závisejí na volbě souřadnicové soustavy
    pozorovatele.
    
    Einsteinovy snahy o zobecnění speciální relativity na neinerciální souřadnicové soustavy vedly v
    roce 1915 ke vzniku obecné relativity – zcela nové teorie gravitace, která popisuje tuto
    interakci za pomoci zakřiveného času a prostoru. Za základ nové teorie lze chápat dvě myšlenky:
    \begin{itemize}[noitemsep]
      \item každé těleso svou přítomností zakřivuje časoprostor kolem sebe;
      \item každé těleso se v tomto zakřiveném časoprostoru pohybuje po nejrovnějších možných
            drahách – tzv. \emph{geodetikách}.
    \end{itemize}

    Nové chápání času a prostoru bylo zcela revoluční. Samotná tělesa se podílejí na vytváření času
    a prostoru, bez nich by čas a prostor neexistoval. Otázka, jak by vypadal vesmír bez přítomnosti
    těles, přestává mít smysl.
    
    Fyzika dvacátého století se tak stala v jistém smyslu poněkud schizofrenní. Tři ze čtyř
    interakcí jsou popsány za pomoci výměnných (polních) částic v rámci kvantové teorie pole. A
    jedna interakce, gravitační, je popsána za pomoci pokřiveného světa obecné teorie relativity.
    Vyřešení mnoha fyzikálních hádanek s sebou přineslo ještě větší záhady. Existuje jednotná teorie
    všech čtyř interakcí? Je možné spojit kvantovou teorii a obecnou relativitu do jedné jediné
    teorie? Odpověď na tyto otázky zatím neznáme. Velké úspěchy slaví různé strunové teorie, ve
    kterých jsou částice chápány jako jednorozměrné kmitající útvary ve vícerozměrném světě, ale zda
    jde o krok správným směrem či nikoli, není v tuto chvíli jasné. V roce 2010 se objevila hypotéza
    holandského fyzika Erika Verlindeho, podle které by gravitace nemusela být skutečnou silou, ale
    jen statistickým projevem růstu entropie v mikrosvětě. Těžko odhadnout, zda tato odvážná
    myšlenka najde podporu v dalších experimentech, nebo jde o slepou uličku.
    
    Pokud vás zajímají základní vlastnosti přírody a jejich teoretický popis, je třeba v první řadě
    začít se studiem klasické mechaniky, na kterou úzce navazuje mechanika kvantová. Další studium
    polních problémů zase není možné bez znalosti statistické fyziky \cite[s.~14]{Kulhanek2019}. 