% !TeX program = lualatex
% !TeX root = luaking.tex
% !TeX encoding = UTF-8
% !TeX spellcheck = cs_CZ
%---------------------------------------------------------------------------------------------------
% file fey1ch01.tex
%---------------------------------------------------------------------------------------------------
%===================== Kapitola: Dějiny fyziky ====================================================
\setchaptertoc
\chapter{Dějiny fyziky}\label{fyz:IchapII}
  \epigraph{\emph{Fyzika je jako sex, může přinést praktické výsledky, ale to není důvod, proč to 
    děláme.}}{Richard P. Feynmann}

  \begin{figure}[ht!]  % \ref{fyz:fig0067}
    \centering
    \luafigure[0.6]{fyz_fig0067.jpg}
    \caption{\wikiFeynman \textasteriskcentered 11. května 1918 - \textdagger 15. února 1988, 
             americký fyzik, který patřil k největším fyzikům 20. století}
    \label{fyz:fig0067}
  \end{figure} 

  Možná nás napadne, zda není možné při studiu fyziky začít tak, že si nejdříve vypíšeme základní
  zákony a potom se budeme zabývat tím, co z nich vyplývá v nejrůznějších situacích. Tak se
  postupuje v euklidovské geometrii, kde se postulují axiomy, ze kterých se odvodí všechny možné
  závěry. Takovýmto způsobem však nemůžeme postupovat ze dvou důvodů. Především, zatím neznáme
  všechny základní zákony - oblast toho, co bychom ještě měli poznat, se nám stále zvětšuje. Dále,
  přesná formulace fyzikálních zákonů zahrnuje mnoho neobvyklých myšlenek, jejichž vyjádření si
  vyžaduje vyšší matematiku. Proto je nutná značná předběžná příprava jen k tomu, abychom rozuměli,
  co znamenají slova. Není tedy možné postupovat tímto způsobem. \emph{Učit se můžeme pouze
  postupně, kousek po kousku}.

  \section{Stěžejní myšlenky fyziky}\label{fyz:IchapIsecII}
    V této kapitole jsou zachyceny \emph{základní myšlenky}, s nimiž se ve fyzice setkáváme - 
    bude pojednáváno o tom, jaká je v současnosti představa o povaze věcí. Nebude však hovořeno o 
    tom, jak se poznala správnost těchto představ - o těchto detailech bude pojednáváno později, až 
    přijde ten pravý čas.

    Věci, o něž se ve fyzice zajímáme, se ukazují množstvím projevů a atributů. Stojíme-li 
    například na břehu a hledíme na moře, vidíme vodu, na vodě pěnu, nad mořem oblaka, slunce, 
    modrou oblohu a vůbec světlo, slyšíme zvuk, nárazy vln, svištění větru, cítíme vzduch. Na břehu 
    je písek a skály, a každá má jinou tvrdost a pevnost, barvu a složení. Jsou tam zvířata a vodní 
    tráva, je tam hlad i nemoc a na břehu je pozorovatel se svými myšlenkami a snad i štěstím. 
    Každé jiné místo v přírodě se vyznačuje podobnou pestrostí věcí a vlivů, podobnou složitostí. 
    Naše zvědavost nás nutí klást otázky, hledat souvislosti a chápat mnohotvárnost věcí jako 
    následek snad relativně malého počtu nejjednodušších věcí a sil působících nekonečně rozmanitě.
    
    Zadívejme se na obr. \ref{fyz:fig0652} a položme si pár zdánlivě jednoduchých otázek: Je písek
    jiný než skály? Není snad písek nic jiného, než velký počet velmi malých kamínků? Je Měsíc velká
    skála? Kdybychom porozuměli tomu, co jsou skály, znamená to, že bychom pochopili i podstatu
    písku a Měsíce? Co je to vítr? Jsou to nárazy vzduchu podobné nárazům vody na břeh? Jaké
    společné rysy mají rozličné druhy pohybu? Co mají společného různé druhy zvuku? Kolik různých
    barev existuje? A tak dále. Takovým způsobem se snažíme postupně analyzovat všechny věci. Dáváme
    do souvislostí věci, které na první pohled vzájemně nesouvisí. Děláme to s nadějí, že se nám
    podaří redukovat počet rozličných věcí a tak je lépe poznat.

    \luagraphic[1]{fyz_fig0652.jpg}{Je písek jiný než skály? Není snad písek nic jiného, než velký
      počet velmi malých kamínků? Jak se pohybuje mořská vlna? Proč loď plave? Fyzika nám umožňuje
      chápat materiální svět, v němž žijeme, a porozumět zákonům, jimiž se řídí.}{fyz:fig0652}
    
    Před několika sty lety vznikla metoda hledání částečných odpovědí na uvedené otázky. 
    \emph{Pozorování, usuzování a experiment} vytvářejí to, co nazýváme \textbf{vědeckou metodou}. 
    Budeme se muset omezit jen na holý popis našich představ o tom, co se nazývá \emph{základní 
    fyzikou} nebo základními myšlenkami, které vznikly aplikováním vědecké metody.
    
    Co to znamená něco „pochopit“? Můžeme si představit, že to složité nahromadění pohybujících se 
    věcí, které vytvářejí „svět“, je šachová hra bohů a my vystupujeme jako diváci, kteří neznají 
    pravidla hry, ale je jim dovoleno hru \emph{pozorovat}. Samozřejmě, pozorujeme-li dostatečně 
    dlouho, můžeme nakonec pochytit několik pravidel. \emph{Pravidla hry} představují to, co 
    chápeme jako \emph{základní fyziku}. I  kdybychom znali všechna pravidla, nemuseli bychom ještě 
    rozumět každému kroku hry, protože je příliš složitá a možnosti našeho rozumu omezené. 
    Ten, kdo hraje šachy, jistě ví, že je jednoduché naučit se všechna pravidla, ale i tak je velmi 
    těžké zvolit ten správný tah nebo pochopit záměry protihráče. Stejné je to i s přírodou, jen 
    mnohem těžší. Máme však možnost najít alespoň všechna pravidla. Zatím je všechna neznáme. 
    (Každou chvíli se objevuje něco takového jako rošáda, kterou ještě neznáme.) Nejen, že neznáme 
    všechna pravidla, ale pomocí těch, která známe, umíme jen velmi málo vysvětlit. Je tomu tak 
    proto, že téměř všechny situace jsou ohromně složité a známá pravidla nám neumožní sledovat 
    všechny obraty hry, nemluvě o předvídání dalších kroků. Musíme se proto omezit na základnější 
    otázku pravidel hry. Naučíme-li se pravidla, budeme to považovat za „pochopení“ světa.
    
    Jak můžeme rozhodnout, zda pravidla, která vlastně jen „odhadujeme“, jsou skutečně správná, 
    když nemůžeme dokonale analyzovat hru? Existují zhruba tři způsoby. Především nám příroda může 
    poskytnout (nebo my si od přírody vynutíme) jednoduché situace skládající se z malého počtu 
    částí, umožňující přesnou předpověď budoucího dění, a tím i zkoušku pravidel. (V rohu 
    šachovnice zůstalo jen málo figurek, jejichž tahy již umíme přesně určit)

    \begin{tcnote}
      \textbf{Vědecká metoda} je posloupnost nebo sada procesů, používaných při vědeckém výzkumu.
      Cílem je získat znalosti a vědomosti pomocí pozorování a dedukce na základě dosud známých
      poznatků. Přijímání nových vědeckých poznatků je založeno na konkrétních důkazech. Vědecká
      metoda je založena na předpokladu, že kritériem pravdivosti vědecké hypotézy je souhlas
      předpovědí s výsledky výzkumu. Tento přístup udržuje vědecké \emph{hypotézy} v neustálém
      kontaktu s realitou a umožňuje jejich \emph{falzifikaci}, neboť hypotéza, jejíž důsledky jsou
      v rozporu s výzkumnými zjištěními, bude falzifikována. Mnohokrát ověřená hypotéza, kterou se
      zatím nepovedlo vyvrátit, se stává vědeckou teorií. Důsledkem je omezení vědy na otázky a
      hypotézy, jež jsou alespoň v principu rozhodnutelné pozorováním. Větší vědecký tým je však
      spíše konzervativní, což brzdí výzkum a vývoj.

      {\centering
      \captionsetup{type=figure}
      \luafigure[0.7]{fyz_fig0924.jpg}
      \captionof{figure}{\wikiPopper \textasteriskcentered	28. července 1902, Vídeň \textdagger 17.
        září 1994 (ve věku 92 let). Byl významným představitelem moderního liberalismu, teorie vědy
        a filosofie.Kredit: Wikipedia}
      \label{fyz:fig0924}
      \par}

      \textbf{Falzifikovatelnost} (vyvratitelnost nebo zpochybnitelnost) je ve filosofii vědy
      vlastnost takového vědeckého tvrzení, hypotézy nebo teorie, kterou je principiálně možné
      vyvrátit, například experimentem. Jinak řečeno, tvrzení nebo teorie je falzifikovatelná tehdy,
      pokud víme, jak by se dala vyvrátit čili negovat. Teorie, kterou experimenty potvrzují, sice
      platí, ale jen dokud ji nějaký experiment nevyvrátí. Problémem falzifikovatelnosti se
      především zabýval rakouský filozof kritického racionalismu \textsc{Karl R. Popper}, který
      zdůraznil, že žádný počet pokusných potvrzení nemůže vědeckou teorii definitivně dokázat. Na
      rozdíl od verifikace, která je vždy jen částečná, pouze falzifikace může být definitivní.

      Ve skutečnosti se jedná o podmínku testovatelnosti a vyvratitelnosti hypotéz a teorií.
      Princip, který takto formuloval, se podle něj nazývá \textbf{Popperova břitva} – „Vědecké
      teorie jsou ověřitelné. Ověřitelné teorie je možné na základě ověřovacího postupu zamítnout (a
      nahradit teoriemi jinými).“

      \tcblower
      Pravdivost vědecké teorie podle něj nelze dokazovat, ale jen empiricky testovat. Základem
      vědeckého poznání tedy není verifikace (potvrzení), ale falsifikace. Pouze ta teorie, kterou
      je možné podrobit falsifikaci, tedy vystavit ji možnosti vyvrácení, je vědecká, tím větší
      hodnotu má pro vědu. Netřeba hromadit důkazy, které teorii potvrzují; spíše hlídat to, co by
      ji mohlo vyvrátit. Konečnou, definitivní jistotu naší přesné znalosti pravdy nemůžeme mít
      nikdy, k pravdě se můžeme pouze přibližovat neustálým vylučováním falsifikovaných teorií,
      hovoříme o evoluci vědy. K evoluci, vývoji vědy dochází právě díky falzifikaci: tím, že něco
      popřeme, získáváme nový prostor pro otevření dalších, nových otázek.

      Důsledkem jeho pojetí vědeckého poznání je obrana otevřeného myšlení a otevřené společnosti.
      Síla vědy netkví tolik v tom, že se její tvrzení dají dokázat, nýbrž v tom, že musí být
      formulována tak, aby se dala vyvrátit. Právě tak síla demokracie nespočívá v tom, že by
      vybírala ty nejlepší k vládě, nýbrž že každou vládu lze běžnými prostředky (volbami) odvolat
    \end{tcnote}

    Druhý způsob zkoušky pravidel spočívá v jejich použití k odvození obecnějších pravidel. 
    Například, střelec se na šachovnici pohybuje úhlopříčně. Odtud je možné usuzovat na skutečnost, 
    že určitý střelec bude vždy na bílém poli. Odhlédneme-li od podrobností, můžeme prověřovat naše 
    pravidlo o pohybu uvedeného střelce tak, že sledujeme, jestli se vždy nachází na bílém poli. Po 
    dlouhém čase se samozřejmě může stát, že se náhle objeví na černém poli (v průběhu hry byl 
    vzat, ale jeden pěšec došel na konec šachovnice a proměnil se na střelce na černém poli). Tak 
    to bývá i ve fyzice. Dlouho používáme pravidlo, které ve všech směrech dobře vyhovuje, ačkoliv 
    neznáme detaily, a potom najednou objevíme \emph{nové pravidlo}. Z hlediska základů fyziky 
    probíhají nejzajímavější jevy na nových místech, na místech, kde pravidla neplatí a ne tam, kde 
    pravidla \emph{platí}. To je způsob, jakým objevujeme nová pravidla.

    Třetí ze způsobů, kterými se můžeme přesvědčit o správnosti našich myšlenek, je poměrně hrubý, 
    ale snad nejúčinnější. Je to způsob přibližného odhadu. Ačkoliv nejsme schopni říci, proč 
    Aljechin \emph{táhl právě tou figurkou}, můžeme v \emph{hrubých rysech} chápat, že seskupuje 
    figurky okolo krále, aby ho chránil, protože za daných okolností je to nejrozumnější. Podobně 
    je to i s naším chápáním přírody. Často ji více či méně chápeme, aniž bychom byli schopni znát 
    význam tahu \emph{každé jednotlivé figurky}.
    
    Zpočátku se přírodní jevy hrubě rozdělovaly do tříd jako teplo, elektřina, mechanika, 
    magnetizmus, vlastnosti látek, chemické děje, světlo nebo optika, rentgenové paprsky, jaderná 
    fyzika, gravitace, mezonové jevy atd. Cílem je však pochopení \emph{celé přírody} jako různých 
    aspektů \emph{jednoho souboru} jevů. Úkolem základní teoretické fyziky dneška je \emph{nalezení 
    zákonů stojících za experimentem a sjednocení uvedených tříd}. Historicky se nám vždy podařilo 
    sloučit je, ale postupem času se objevovaly nové věci. Když jsme si již vytvořili ucelenou 
    představu, objevily se najednou rentgenové paprsky. Když se i tento jev dostal do jednotného 
    schématu, objevily se mezony. Proto v každém stádiu hry vypadá situace dost chaoticky. Mnohé se 
    objasnilo z jednotného hlediska, ale ještě stále je mnoho volných konců nitek, o nichž nevíme, 
    kam patří. Takový je dnes stav věcí a my se ho pokusíme popsat.
    
    Všimněme si v historii několika příkladů uvedeného sjednocování. Uvažujme nejdříve \emph{teplo 
    a mechaniku}. Jsou-li atomy v pohybu, obsahuje systém tím více tepla, čím více pohybu v něm je, 
    takže \emph{teplo a všechny tepelné efekty je možné vyjádřit pomocí zákonů mechaniky}. Dalším 
    úžasným sjednocením bylo objevení souvislosti mezi \emph{elektřinou, magnetizmem} a světlem, o 
    nichž se zjistilo, že jsou různými aspekty stejné věci, kterou dnes nazýváme 
    \emph{elektromagnetické pole}. Dále chemické děje, rozmanité vlastnosti různých látek a chování 
    atomových částic byly sjednoceny do \emph{kvantové chemie}.
    
    Zůstává zde však otázka, zda bude možné vše sjednotit tak, abychom mohli prohlásit, že svět 
    představuje rozmanité aspekty jediné věci? To nikdo neví. Víme pouze, že na naší cestě vpřed se 
    nám daří spojovat fragmenty, přičemž vždy nalézáme cosi, co nezapadá do obecného obrazu, a 
    proto se opět pokoušíme doplnit skládačku. Nevíme, zda tato skládačka má konečný počet částí a 
    zda má tato hra vůbec hranice. Dozvíme se to až tehdy, když složíme výsledný obraz, jestli ho 
    vůbec kdy složíme. Chtěli bychom však ukázat, kam až tento proces sjednocování pokročil a jaká 
    je dnešní situace při objasňování základních jevů pomocí co nejmenšího počtu principů. 
    Jednodušeji řečeno: \textbf{z čeho jsou složeny věci a kolik je těch stavebních prvků?} 
    \cite[s.~27]{Feynman02}    


  \twocolumn[\section{Jak studovat fyziku podle Feynmana}\label{fyz:IchapIsecI}]
    Každý kousek nebo část celku, který představuje příroda, je vždy jen přiblížením k úplné pravdě;
    přesněji k úplné pravdě, pokud ji známe. Ve skutečnosti vše, co víme, je jen určitým druhem
    aproximace, protože víme, že ještě neznáme všechny zákony. Proto se věci musíme učit jen proto,
    abychom se je znovu odnaučili, nebo, což je pravděpodobnější, abychom si naše znalosti o nich
    opravovali.
    
    Princip vědy, téměř její definice, je následující: Prověrkou všech našich vědomostí je 
    experiment. Experiment je jediné kritérium vědecké „pravdy“. Jenže co je zdrojem našich 
    vědomostí? Odkud pocházejí zákony, které prověřujeme? Samotný experiment nám pomáhá odvozovat 
    zákony v tom smyslu, že nám poskytuje náznaky, pokyny. Navíc je však potřebná představivost, 
    aby z těchto náznaků mohla vzniknout velká zobecnění - abychom v nich odhadli nádherný, 
    jednoduchý, ale neobyčejný obraz a potom experimentem prověřili správnost našeho odhadu. Tento
    proces představivosti je tak těžký, že si fyzici rozdělili práci - teoretičtí fyzici 
    představivostí, dedukcí a odhadem odvozují nové zákony, ale neexperimentují; experimentální 
    fyzici dělají pokusy a přitom také uplatňují představivost, dedukci a odhad.
    
    Řekli jsme, že přírodní zákony jsou přibližné: že nejdříve nacházíme „nesprávné“ a až potom
    „správné“. Jak však může být experiment „nesprávný“? Především z velmi jednoduchého důvodu - náš
    přístroj není v pořádku a my jsme to nezpozorovali. Takové chyby se však zjišťují lehce.
    Odhlédneme-li od těchto drobností, jak může být výsledek experimentu nesprávný? Jen v důsledku
    nepřesnosti. Například, hmotnost předmětu se zdá být neměnná; rotující káča má stejnou hmotnost
    jako káča v klidu. Tak byl objeven „zákon“: hmotnost je konstantní, nezávislá na rychlosti. O
    tomto „zákonu“ se zjistilo, že je nesprávný. Ukázalo se, že hmotnost roste s rychlostí, ale k
    značnému růstu jsou potřebné rychlosti blízké rychlosti světla. Správný zákon zní: je-li
    rychlost tělesa menší než \SI{100}{\km\per\second}, je hmotnost konstantní s přesností na jednu
    milióntinu. V takové aproximativní podobě je tento zákon správný. Někdo by si mohl myslet, že
    prakticky není rozdíl mezi starým a novým zákonem. To je i není pravda. Pro běžné rychlosti je
    jistě možné zapomenout na to, o čem jsme mluvili a používat jednoduchý zákon konstantní
    hmotnosti jako dobré přiblížení. Při velkých rychlostech se však dopustíme chyby, a to tím
    větší, čím větší je rychlost

    \begin{figure}[ht!]  %\ref{fyz:fig0891}
      \centering
      \luafigure[0.9]{fyz_fig0891.jpg}
      \caption{Vodní kapka dopadající na hladinu. Kredit: Wikipedia}
      \label{fyz:fig0891}
    \end{figure} 
    
    Ostatně, nejzajímavější je to, že z filozofického hlediska je tento aproximativní zákon zcela
    nesprávný. Náš celkový obraz o světě musíme změnit, i kdyby se hmotnost měnila jen nepatrně.
    Toto je svérázný znak filozofie nebo myšlenek stojících v pozadí zákonů. Někdy i velmi malý
    efekt vyžaduje hlubokou změnu našich názorů.
    
    Čemu tedy máme dát přednost? Máme podat správné, ale nezvyklé zákony s jejich cizím a obtížným
    pojetím jako je například teorie relativity, čtyřrozměrný prostoročas a podobně? Nebo máme
    nejdříve vysvětlit jednoduchý zákon „konstantní hmotnosti“, který je pouze přibližný, ale
    nevyžaduje náročné představy? První způsob je více vzrušující, nádhernější a zábavnější, ale s
    druhým se jednodušeji začíná a představuje první krok ke skutečnému porozumění správného zákona.
    Tento problém se vždy znovu objevuje při vyučování fyziky. V různých etapách ho musíme řešit
    různými způsoby, ale vždy je vhodné se zajímat, do jaké míry je přesné to, co teď víme, jak to
    souvisí s dalším a jak se to může změnit, budeme-li vědět víc.
    
    Nyní přejděme k náčrtu nebo k všeobecné mapě našeho chápání současné vědy (zejména fyziky, ale i
    jiných věd, které s ní souvisejí). Když se později soustředíme na konkrétní problém, budeme mít
    představu o jeho pozadí, o tom, proč je zajímavý a jak zapadá do celkové struktury. Jaký je tedy
    náš celkový obraz světa? \cite[s.~16]{Feynman01}
    
  \section{Látka se stává z atomů}
    Kdyby při nějaké katastrofě zanikly všechny vědecké poznatky a dalším generacím by měla zůstat
    jen jediná věta, které tvrzení by při nejmenším počtu slov obsahovalo nejbohatší informaci?
    Takovým kandidátem je \textbf{atomová hypotéza} - tj. že \emph{všechny věci se skládají z atomů
    malých částic, jež jsou v neustálém pohybu,  a vzájemně se přitahují, když jsou od sebe trochu
    vzdálené, ale odpuzují se, když jsou těsně u sebe.} V této jediné větě, jak uvidíme, je obsaženo
    nesmírné množství informací o světě. Je k tomu třeba jen trochu představivosti a uvažování.

    \begin{figure}[ht!]  %\ref{fyz:fig0007}
      \centering
      \luafigure[0.6]{fyz_fig0007.pdf}
      \caption{Vodní kapka z obrázku \ref{fyz:fig0891} zvětšená miliardkrát \cite[s.~17]{Feynman01}}
      \label{fyz:fig0007}
    \end{figure} 

    Abychom ilustrovali sílu myšlenky o atomu, představme si podobnou kapku vody jako na obrázku
    \ref{fyz:fig0891} o rozměru \SI{0.5}{\cm}. Podí\-váme-li se na ni zblízka, neuvidíme nic jiného,
    než vodu - klidnou, souvislou vodu. I když kapku zvětšíme tím nejlepším optickým mikroskopem,
    přibližně dvoutisíckrát, a kapka bude měřit deset metrů, tedy stejně jako velká místnost, i
    tehdy budeme stále vidět relativně klidnou vodu. Jen tu a tam v ní budou plavat jakési malé
    fotbalové míče. Tyto velmi zajímavé objekty na obrázku \ref{fyz:fig0890} jsou trepky. Tady se
    můžeme zastavit a zajímat se o trepky, o jejich třepotající se řasičky, o jejich kroutící se
    těla a nepokračovat ve zvětšování. Nebo můžeme zvětšit trepky tak, abychom viděli i do nich). 

    \begin{figure}[ht!]  %\ref{fyz:fig0890}
      % http://www.photomacrography.net/forum/viewtopic.php?t=18166
      \centering
      \luafigure[1]{fyz_fig0890a.jpg}
      \caption{ Trepka velká \emph{(Paramecium caudatum)} je nálevník běžně se vyskytující v 
                organicky znečištěných vodách po celém světě. Kredit: Wikipedia}
      \label{fyz:fig0890}
    \end{figure} 

    Trepky jsou však předmětem biologie. Proto si jich teď nebudeme všímat, ale zahledíme se ještě
    pozorněji na vodu při dalším dvoutisícinásobném zvětšení. Teď měří kapka vody dvacet kilometrů a
    při pozorném sledování je vidět jakési hemžení - cosi, co už nevypadá klidně, ale připomíná dav
    na fotbalové tribuně při pohledu z velké vzdálenosti. Abychom zjistili, co je to za hemžení,
    zvětšíme kapku ještě 250krát a potom uvidíme něco podobného jako na obr. \ref{fyz:fig0007}. Tento
    obrázek představuje vodu při zvětšení miliardkrát, je však v několika směrech idealizovaný.
    Především částice jsou zakreslené zjednodušeně - s ostrými okraji, což neodpovídá skutečnosti.
    Dále kvůli jednoduchosti jsou částice zakreslené v dvojrozměrném uspořádání, ačkoli se ve
    skutečnosti pohybují ve všech třech směrech. Všimněme si, že jsou tam dva druhy částic
    znázorněných kroužky, které představují atomy kyslíku (černé) a vodíku (bílé) a že na každý atom
    kyslíku se vážou dva atomy vodíku. Každá skupinka skládající se z atomu kyslíku a dvou atomů
    vodíku se nazývá \textbf{molekulou}. Obrázek je zjednodušený i v tom, že skutečné částice v
    přírodě se ustavičně kolébají a poskakují, obracejí se a točí jedna okolo druhé. Je třeba si to
    představit spíše jako \emph{dynamický a né jako statický obrázek}. Další věcí, kterou není možné
    vystihnout na obrázku, je skutečnost, že částice „drží pohromadě“ - přitahují se, jedna za sebou
    táhne druhou atd. Je možné říci, že jsou jakoby „slepené dohromady“. Na druhé straně se částice
    netlačí jedna přes druhou. Kdybychom se pokusili přitlačit dvě z nich příliš těsně k sobě,
    odpudily by se.

    Atomy mají poloměr \SI{1e-10}{\m} až \SI{2e-10}{\m}. Jejich velikost si můžeme pamatovat i
    jinak: zvětšíme-li jablko na velikost Země, budou atomy v jablku tak velké, jak bylo původně
    jablko.

    Představme si teď tuto velkou kapku vody s jejími hemžícími se částicemi, jež přilnuly k sobě a
    honí jedna druhou. Voda udržuje svůj objem; nerozpadne se na části díky vzájemné přitažlivosti
    molekul. Je-li tato kapka na šikmé ploše, kde se může hýbat z místa na místo, voda poteče.
    Nestane se však, že by jednoduše zmizela. Věci se nerozpadají na části právě díky přitažlivosti
    molekul. \emph{Hemživý pohyb částic je to, co chápeme jako teplo}: zvýšíme-li teplotu, zvětšíme
    pohyb. Zahříváme-li vodu, pohyb roste a roste i vzdálenost mezi částicemi, až nastane okamžik,
    kdy přitažlivost mezi molekulami je už nestačí udržet pohromadě. Částice přestanou být vzájemně
    svázané a rozlétají se od sebe. Zvyšováním teploty tak získáváme \emph{vodní páru}.  
    
    \begin{figure}[ht!]
      \centering
      \luafigure[0.6]{fyz_fig0008.pdf}
      \caption{Pára \cite[s.~18]{Feynman01}}
      \label{fyz:fig0008}
    \end{figure}

    Na obr. \ref{fyz:fig0008} vidíme páru. V jednom směru tento obrázek páry selhává: při našem
    zvětšení za normálního atmosférického tlaku připadá jen velmi málo molekul na celý pokoj, takže
    na tak malém obrázku určitě nebudou tři molekuly. Většina plošek této velikosti nebude obsahovat
    žádnou molekulu - na našem obrázku jsou náhodou dvě a část z třetí molekuly (abychom tam neměli
    prázdné místo). V případě páry vidíme podobu molekul jasněji než v případě vody. Pro
    jednoduchost jsou molekuly zakresleny tak, že atomy vodíku svírají úhel \SI{120}{\degree}. Ve
    skutečnosti má tento úhel hodnotu \ang[arc-separator = \,]{105;3;} a vzdálenost mezi středem
    vodíku a středem kyslíku je \SI{9.57e-11}{\m}. Tuto molekulu tedy velmi dobře známe.

    Všimněme si jedné vlastnosti vodní páry nebo jiných plynů. Molekuly budou tím, že se vzdálily
    jedna od druhé, narážet na stěny. Představme si místnost s určitým počtem (tak kolem sta)
    neustále poskakujících tenisových míčků. Když míčky narážejí na stěnu, odtlačují ji a stěnu
    proto musíme upevnit. Plyn působí přerušovanou silou, kterou naše nedokonalé smysly (jejich
    citlivost nevzrostla miliardkrát) vnímají jako \emph{stálý tlak}. Abychom plyn udrželi, musíme
    na něj působit tlakem z opačné strany. Obr. \ref{fyz:fig0009} znázorňuje běžnou nádobu na
    udržování plynu, kterou najdeme v každé učebnici: \textbf{válec s pístem}. Teď nám nezáleží na
    tom, jaký je ve skutečností tvar molekul vody, a proto je kvůli jednoduchostí znázorníme jako
    tenisové míčky nebo body. Jsou v neustálém pohybu a pohybují se na všechny strany. Na spodek
    \emph{pístu} jich neustále naráží tolik, že na něj musíme působit určitou silou dolů, aby ho
    molekuly nevytlačily z válce. Tuto \emph{sílu} nazýváme \textbf{tlakem} (přesněji, \emph{tlak
    násobený plochou dává sílu}). Je jasné, že síla je úměrná ploše pístu, protože zvětšíme-li
    plochu a přitom nezměníme počet molekul v kubickém centimetru, pak vzroste počet srážek s pístem
    tolikrát, kolikrát se zvětšila jeho plocha.

    \begin{figure}[ht!]
      \centering
      \luafigure[0.4]{fyz_fig0009.pdf}
      \caption{Píst \cite[s.~18]{Feynman01}}
      \label{fyz:fig0009}
    \end{figure}

    Nyní \emph{zdvojnásobme} v této nádobě počet molekul, takže se zdvojnásobí jejich hustota, ale
    ponechme jim stejnou rychlost, tj. \emph{stejnou teplotu}. Pak můžeme dost přesně říci, že se
    zdvojnásobil počet srážek a jelikož je každá právě tak \uv{energická} jako dříve, tlak je úměrný
    hustotě. Uvážíme-li skutečnou povahu meziatomových sil, můžeme očekávat mírný pokles tlaku jako
    projev zvýšené přitažlivosti mezi atomy a mírný vzrůst související s objemem, který zaujímají.
    Přesto však, pokud je hustota dostatečně nízká, tj. atomů není příliš mnoho, můžeme s
    dostatečnou přesností říci, že \emph{tlak je úměrný hustotě}.
    
    Snadno pochopíme i něco jiného. \emph{Zvyšujeme-li teplotu} bez změny hustoty plynu, tj. když
    zvětšujeme rychlost atomů, co se stane s tlakem? Atomy narážejí do pístu \emph{silněji}, neboť
    se pohybují rychleji a navíc, narážejí častěji. Proto tlak vzrůstá. Vidíme, jak jednoduché jsou
    myšlenky atomové teorie.
    
    Podívejme se na jinou situaci. Předpokládejme, že se píst \emph{pohybuje dovnitř}, takže atomy
    jsou pomalu stlačovány do menšího prostoru. Co se stane, narazí-li atom do pohybujícího se
    pístu? Je jasné, že při takové srážce \emph{získá rychlost}. Můžeme si to vyzkoušet na
    ping-pongovém míčku: po úderu pálkou míček odletí od pálky rychleji, než k ní přiletěl. Ve
    zvláštním případě, není-li atom v pohybu a píst na něj narazí, atom se začne určitě pohybovat.
    Atomy jsou při návratu od pístu \uv{teplejší}, než byly před nárazem na píst. Proto všechny
    atomy, které jsou v nádobě, získají na rychlosti. To znamená, že při pomalém \emph{stlačení
    plynu jeho teplota vzrůstá}. Když plyn pomalu \emph{stlačujeme}, jeho teplota \emph{vzrůstá} a
    když plyn pomalu \emph{rozpínáme}, jeho teplota \emph{klesá}.
    
    \begin{figure}[ht!]  % \ref{fyz:fig0010}
      \centering
      \luafigure[0.6]{fyz_fig0010.pdf}
      \caption{Led \cite[s.~19]{Feynman01}}
      \label{fyz:fig0010}
    \end{figure}

    Vraťme se k naší kapce vody a podívejme se na ni z jiného pohledu. Snižme teplotu naší kapky.
    Předpokládejme, že hemžení molekul vody postupně slábne. Víme, že mezi atomy působí přitažlivé
    síly, které způsobí, že molekuly už nebudou moci tak snadno pohybovat. Obr. \ref{fyz:fig0010}
    znázorňuje, co se stane při velmi nízkých teplotách: molekuly jsou vázány v nové struktuře,
    vytváří se \textbf{led}. Takové schematické znázornění ledu není správné, neboť je dvojrozměrné.
    Situaci však vystihuje kvalitativně. Je pozoruhodné, že každý atom má v této látce určité místo.
    Rozmístíme-li atomy na jednom konci kapky podle určitého pravidla, pak v důsledku pevné
    struktury meziatomových vazeb vznikne určité uspořádání atomů i na druhém konci kapky, vzdáleném
    (v našem měřítku) několik kilometrů. Proto, držíme-li ledový rampouch za jeden konec, jeho druhý
    konec bude při lámání klást odpor, chová se jinak než voda, ve které je pravidelná struktura
    rozrušena intenzívním pohybem atomů v rozličných směrech. Rozdíl mezi pevnými látkami a
    kapalinami spočívá v tom, že atomy pevné látky jsou \emph{uspořádány} zvláštním způsobem. Toto
    uspořádání se nazývá \textbf{krystalická struktura}. I tehdy, kdy jde o velmi vzdálené atomy,
    nepozorujeme nic náhodného v jejich polohách. Poloha atomu na jednom konci krystalu je určena
    polohou atomu na druhém konci, i když se mezi nimi nacházejí miliony jiných atomů. Obr.
    \ref{fyz:fig0010} znázorňuje vymyšlené uspořádání ledu a ačkoli správně vystihuje mnohé
    vlastnosti ledu, neodpovídá skutečnému uspořádání. Jedním ze správných rysů je existence části
    \emph{hexagonální symetrie}. Můžeme se o tom přesvědčit: otočíme-li obrázek o \SI{120}{\degree},
    dostaneme stejné seskupení. Taková symetrie ledu je příčinou šestihranného tvaru sněhových
    vloček. Další informací, kterou je možné vytušit z obrázku \ref{fyz:fig0010}, je \emph{zmenšování
    objemu ledu při tání}. Znázorněná struktura ledu, stejně tak jako skutečná, obsahuje \emph{mnoho
    dutin}. Když se struktura rozpadne, tyto dutiny mohou být \emph{zaplněny} molekulami.
    \emph{Většina jednoduchých látek, s výjimkou vody a liteřiny\footnote{Liteřina či písmovina je
    slitina používaná v písmolijectví. Její přibližné složení je: \SIrange{50}{86}{\percent} olova,
    \SIrange{3}{20}{\percent} cínu a \SIrange{11}{30}{\percent} antimonu. Liteřinu vyvinul v 15.
    století zakladatel knihtisku Johannes Gutenberg pro odlévání tiskařských liter.}, zvětšují při
    tání svůj objem, neboť atomy jsou v pevných krystalech těsně seskupeny a při tání potřebují více
    prostoru na kmitání}. Otevřené struktury se však při tání zhroutí - podobně jako led.
    
    I když má led pevnou krystalickou strukturu, jeho teplota se může měnit - v ledu je zásoba 
    tepla. Chceme-li, můžeme toto množství tepla změnit. Jaké je teplo, které se nachází v ledu? 
    Atomy ledu \emph{nejsou} v klidu, poskakují a kmitají. Ačkoli v krystalu existuje určité 
    uspořádání - struktura - všechny atomy kmitají, „na místě“. Zvyšujeme-li teplotu, budou 
    kmitat se stále větší amplitudou, až opustí svá místa. Tento jev nazýváme \textbf{táním}. 
    Snižujeme-li teplotu, kmity slábnou a při teplotě \emph{absolutní nuly} jsou 
    \emph{nejslabší}, ne však nulové. Toto nejmenší množství pohybu, který přísluší atomům, 
    nestačí na roztání látky - až na jednu výjimku: \emph{hélium}. V héliu se při ochlazování 
    také zpomaluje pohyb atomů na nejmenší možnou míru, ale i při teplotě absolutní nuly brání 
    tento pohyb zmrznutí hélia. Hélium nezmrzne, pokud nevytvoříme tak veliký tlak, abychom atomy 
    stlačili k sobě. Při velkém tlaku můžeme dosáhnout toho, že hélium ztuhne.
  
  \subsection{Atomové procesy}
    Dosud jsme si všímali stavby pevných látek, kapalin a plynů z atomového hlediska. Jenže atomová
    hypotéza charakterizuje i procesy, a proto si všimněme některých procesů z atomového hlediska.
    Nejdříve budeme hovořit o procesech, které se odehrávají na povrchu vody. Co se vlastně děje na
    vodním povrchu? Úlohu si zkomplikujeme - bude tak blíže skutečnosti - předpokladem, že nad
    vodním povrchem se nachází vzduch. Obr. \ref{fyz:fig0011} takovou situaci znázorňuje. Tak jako
    předtím vidíme molekuly vody, které tvoří kapalinu, ale vidíme i povrch vody. Nad povrchem
    vidíme různé molekuly. Jsou tam především \emph{molekuly vody} v podobě vodní páry, kterou je
    možné pozorovat vždy nad kapalnou vodou (pára a voda jsou v rovnováze, o které pohovoříme
    později). Dále tam nalezneme jiné molekuly, dvojice atomů kyslíku tvořící \emph{molekulu
    kyslíku} a dvojice atomů dusíku tvořící \emph{molekulu dusíku}. Vzduch se skládá téměř výhradně
    z dusíku, kyslíku, vodní páry a menšího množství oxidu uhličitého, argonu a jiných příměsí. Nad
    povrchem vody se nachází vzduch - plyn obsahující jisté množství vodní páry. Nyní si všimněme,
    co se odehrává na obrázku. Molekuly vody se neustále pohybují. Občas některá z molekul,
    nacházejících se v blízkosti povrchu, naráží na jinou molekulu trochu silněji než obvykle a
    vyskočí nad povrch. Na obrázku takovýto děj \emph{přímo} neuvidíme, neboť vše je na něm nehybné.
    Můžeme si však představit, že jedna molekula za druhou v důsledku srážek opouštějí vodu - voda
    mizí, \emph{vypařuje} se. Když nádobu \emph{přikryjeme}, objevíme po nějakém čase velké množství
    molekul vody mezi molekulami vzduchu. Čas od času některá z těchto molekul vody vletí zpět do
    vody a zůstává v ní. To, co jsme považovali za mrtvé a nezajímavé - přikrytý pohár vody, který
    snad dvacet let stál na jednom místě - v sobě skrývá stále probíhající zajímavý
    \textbf{dynamický proces}. Náš nedokonalý zrak nepozoruje žádnou změnu, ale při miliardovém
    zvětšení bychom viděli, jak se vše mění: jedny molekuly opouštějí povrch a druhé se vracejí.
    
    \begin{figure}[ht!]   % \ref{fyz:fig0011}
      \centering
      \luafigure[0.7]{fyz_fig0011.pdf}
      \caption{Voda vypařující se do vzduchu \cite[s.~21]{Feynman01}.}
      \label{fyz:fig0011}
    \end{figure}

    Proč nepozorujeme tyto změny my? Protože do vody se vrací právě tolik molekul, kolik z ní
    odešlo. Navenek se „nic neděje“. Když odkryjeme nádobu, odfoukneme vlhký vzduch pryč a nahradíme
    ho suchým vzduchem, nezmění se počet z vody vylétajících molekul (neboť závisí pouze na pohybu
    ve vodě), ale velmi se změní počet molekul do vody se vracejících, protože nad vodou je mnohem
    méně molekul. Molekul, které opouštějí vodu, je víc než molekul, které se do ní vracejí; voda se
    vypařuje. Chceme-li tedy, aby se voda vypařovala, zapneme ventilátor!
    
    Zůstává ještě otázka: Které molekuly opouštějí vodu? Molekula opustí vodu, když náhodně získá
    malé množství dodatečné energie, kterou potřebuje na to, aby překonala přitažlivé působení svých
    sousedů. Protože ty molekuly, které opouštějí vodu, mají větší než průměrnou energii, budou se
    molekuly, které ve vodě zůstávají, v průměru pohybovat méně. Při vypařování se tedy kapalina
    postupně \emph{ochlazuje}. Je samozřejmé, že když molekula páry sestoupí ze vzduchu do vody,
    objeví se silné přitahování, když molekula dosahuje povrchu vody. Důsledkem toho je zrychlení
    přicházející molekuly a s tím spojený vznik tepla. \emph{Můžeme tedy říci, že s odchodem molekul
    odchází a s příchodem molekul přichází teplo.} Když jsou oba procesy vyrovnány, voda svou
    teplotu nemění. Foukáme-li na vodu, aby odpařování převládalo nad zkapalňováním, voda se
    ochlazuje. Proto, chcete-li ochladit polévku, foukejte na ni!
    
    \begin{figure}[ht!]    % \ref{fyz:fig0012}
      \centering
      \luafigure[0.7]{fyz_fig0012.pdf}
      \caption{Sůl rozpouštějící se ve vodě \cite[s.~21]{Feynman01}.}
      \label{fyz:fig0012}
    \end{figure}

    Musíme si však uvědomit, že procesy, o kterých jsme hovořili, probíhají ve skutečnosti
    složitěji. Při unikání vody do vzduchu čas od času některá z molekul kyslíku nebo dusíku vnikne
    do vody a \uv{ztratí se} mezi jejími molekulami. Vzduch se tedy rozpouští ve vodě. Molekuly
    kyslíku a dusíku pronikají do vody, která pak obsahuje vzduch. Když z nádoby náhle odstraníme
    vzduch, budou molekuly vzduchu unikat z vody rychleji, než do ní vnikají, což způsobí
    vystupování bublinek. Tato skutečnost je velmi nepříjemná pro potápěče.
  
    Nyní si všimněme dalšího procesu. Obr. \ref{fyz:fig0012} znázorňuje, jak se podle atomové
    představy rozpouští pevná látka ve vodě. Co se stane, vložíme-li krystal soli do vody? Sůl je
    pevná látka, krystal, organizované seskupení „atomů soli“. Na obr. \ref{fyz:fig0013} je
    znázorněna trojrozměrná struktura kuchyňské soli, chloridu sodného. Máme-li být přesní, musíme
    říct, že krystal není tvořen atomy, ale ionty. Iont je atom, který má několik elektronů navíc,
    nebo několik elektronů ztratil. V krystalu soli nalézáme \emph{ionty chlóru} (atomy chlóru s
    přebytečným elektronem) a \emph{ionty sodíku} (atomy sodíku zbavené jednoho elektronu). Ionty
    jsou v krystalu vzájemně vázány elektrickou přitažlivostí, ale ve vodě se některé z nich pod
    vlivem přitažlivosti záporného kyslíku a kladného vodíku začnou uvolňovat. Na obrázku
    \ref{fyz:fig0012} vidíme uvolňující se iont chlóru a jiné atomy plavající ve vodě ve formě iontů.
    Tento obrázek je pečlivě zakreslený. Všimněme si například, že vodíkové konce molekul vody
    obvykle obklopují iont chlóru a u iontu sodíku zpravidla nalézáme kyslíkový konec, neboť sodík
    je kladný a kyslíkový konec molekuly vody je záporný a tyto se elektricky přitahují. Můžeme
    podle tohoto obrázku říci, jestli se sůl \emph{rozpouští} ve vodě, nebo \emph{krystalizuje} z
    vody? Samozřejmě, že \emph{nemůžeme}, neboť zatím co jedny atomy opouštějí krystal, jiné se k
    němu připojují. Takovýto proces je - podobně jako vypařování - \emph{dynamický} všechno závisí
    na tom, je-li ve vodě více nebo méně soli, než je třeba k rovnováze. Rovnováhou rozumíme takovou
    situaci, kdy počet atomů opouštějících krystal je roven počtu atomů do krystalu se vracejících.
    Když sůl ve vodě téměř není, vstupuje do vody více atomů, než vystupuje a sůl se rozpouští. Když
    je, naopak, „atomů soli“ příliš mnoho, do krystalu se vrací více atomů, než ho opouští a sůl
    krystalizuje.

    \begin{figure}[ht!]  % \ref{fyz:fig0013}
      \centering
        \subcaptionbox{Tabulka vzdáleností. \label{fyz:fig0013a}}
          {\luafigure[0.4]{fyz_fig0013a.pdf}}                                                      \\                                   
        \subcaptionbox{Vzdálenost nejbližších sousedů \(d = \dfrac{a}{2}\). \label{fyz:fig0013b}}
          {\luafigure[0.7]{fyz_fig0013b.pdf}}
      \caption{Trojrozměrná struktura kuchyňské soli, chloridu sodného\cite[s.~22]{Feynman01}}
      \label{fyz:fig0013}
    \end{figure}
    
    Zmínili jsme se o tom, že představa \emph{molekuly} látky je pouze přibližná a je opodstatněná
    jen pro určitou třídu látek. Je jasné, že v případě vody jsou její tři atomy skutečně svázané,
    ale v případě pevného chloridu sodného už to tak jasné není. V takovém případě jde o uspořádání
    sodíkových a chlorových iontů do krychlové mřížky a neexistuje přirozený způsob jejich
    uspořádání do „molekul soli“.
    
    Vraťme se ještě k naší diskuzi o \emph{rozpouštění a srážení}. Zvýšíme-li teplotu roztoku soli,
    vzroste počet atomů, které sůl opouštějí a vzroste i počet atomů, které se do soli vracejí.
    Ukazuje se, že obecně je velmi těžké předpovědět, jak se ten proces realizuje, proběhne-li
    rozpouštění rychleji nebo pomaleji. S rostoucí teplotou se většina látek rozpouští lépe, ale
    některé látky se rozpouštějí hůře.
    
  \subsection{Chemické reakce}
    Ve všech procesech, o nichž jsem dosud hovořili, neměnily atomy a ionty své partnery. Za
    určitých okolností však může dojít ke změně atomových kombinací, vytvoří se nové molekuly.
    Taková situace je znázorněna na obr. \ref{fyz:fig0014}.
    
    \begin{figure}[ht!]   % \ref{fyz:fig0014}
      \centering
      \luafigure[0.7]{fyz_fig0014.pdf}
      \caption{Uhlík hořící v kyslíku \cite[s.~23]{Feynman01}}
      \label{fyz:fig0014}
    \end{figure}

    Proces, ve kterém dochází k přeskupení atomových partnerů, nazýváme \textbf{chemickou reakcí}.
    Ostatní dosud uvažované procesy nazýváme \textbf{fyzikálními procesy}. Mezi uvedenými dvěma
    druhy, procesů však neexistuje ostrá hranice. Příroda se nestará o naše názvosloví a pokračuje i
    nadále ve svém díle. Uvedený obrázek má znázornit hoření uhlíku v kyslíku. Kyslík se vyznačuje
    tím, že jeho dva atomy jsou velmi pevně svázány. (Proč nejsou svázány \emph{tři} nebo dokonce
    \emph{čtyři} atomy? Toto je jedna ze zvláštností atomových procesů. Atomy jsou velmi svérázné:
    upřednostňují určité partnery, určité směry apod. Úlohou fyziky je analyzovat, proč chtějí právě
    to, co chtějí. V každém případě dva atomy kyslíku, nasycené a šťastné, tvoří molekulu.)
    
    Předpokládejme, že atomy uhlíku vytvářejí pevný krystal - grafit nebo diamant (diamant může
    shořet ve vzduchu). Uvažujme situaci, kdy se molekula kyslíku dostane k uhlíku, Každý její atom
    zachytí atom uhlíku a odletí v novém seskupení - „uhlík-kyslík“. Toto seskupení představuje
    molekulu plynu nazývaného \emph{oxid uhelnatý}. Jeho chemické označení je \ce{CO}. Je to velmi
    jednoduché: písmena „CO“ jsou vlastně obrazem jeho molekuly. Jenže uhlík váže kyslík o mnoho
    silněji než kyslík váže kyslík nebo uhlík váže uhlík. Proto v tomto procesu může kyslík
    přicházet s malou energií, ale kyslík a uhlík se spojí velmi „energicky“a uvolněnou energii
    pohltí okolní atomy. Tak se vytváří velké množství pohybové, kinetické energie. Myslíme tím
    samozřejmě \textbf{hoření}; spojením uhlíku a kyslíku získáváme \emph{teplo}. Teplo se obvykle
    projevuje formou pohybu molekul horkého plynu, ale za určitých okolností ho může být tak mnoho,
    že způsobuje světlo. Tak vzniká \textbf{plamen}.
    
    \begin{figure}[hbt!]    % \ref{fyz:fig0015}
      \centering
      \luafigure[0.7]{fyz_fig0015.pdf}
      \caption{Vůně fialek \cite[s.~24]{Feynman01}}
      \label{fyz:fig0015}
    \end{figure}

    Kromě toho, \emph{oxid uhelnatý} není zcela uspokojen. Je možné, aby k sobě připoutal další atom
    kyslíku a tak dostaneme mnohem složitější reakci, ve které se kyslík spojuje s uhlíkem a
    současně dochází ke srážce s molekulou oxidu uhelnatého. Kyslíkový atom se připojí k \ce{CO} a v
    konečném důsledku vytvoří molekulu složenou z jednoho uhlíku a dvou kyslíků. Tato molekula má
    označení \ce{CO2} a nazývá se \emph{oxid uhličitý}. Spalujeme-li uhlík ve velmi malém množství
    kyslíku a reakce probíhá velmi rychle (např. v motoru automobilu, kde je výbuch tak rychlý, že
    se nestačí vytvořit oxid uhličitý), vzniká velké množství oxidu uhelnatého. V mnoha takových
    přeskupeních atomů se uvolňuje velké množství energie, vznikají výbuchy, plamen apod., podle
    druhu reakce. Chemici studovali takové seskupení atomů a zjistili, že každá látka představuje
    určitý druh \emph{uspořádání atomů}.

    K objasnění této myšlenky si zvolme jiný příklad. Ocitneme-li se na louce rozkvetlé fialkami,
    víme, co je to za „vůni“. Je to určitý druh molekul nebo seskupení atomů, které se dostalo do
    našeho nosu. Jak se nám to stalo? To je dost jednoduché! Jestliže vůně je jistý druh molekul,
    tím nejrozmanitějším způsobem poletujících a srážejících se ve vzduchu, pak se může náhodou
    dostat i do nosu. Tyto molekuly se určitě nesnažily dostat právě do našeho nosu. Jsou jen
    bezmocnou částí strkajícího se zástupu molekul, jehož kousek se na svém bezcílném putování
    dostal do našeho nosu.      
    
    Chemici mohou i takové zvláštní molekuly, jako je vůně fialek, podrobit analýze a říci nám
    \emph{přesné uspořádání} jejich atomů v prostoru. Víme, že molekula oxidu uhličitého je
    \emph{přímá a symetrická}: \ce{O\bond{-}C\bond{-}O} (lze to snadno zjistit i fyzikálními
    metodami). I pro mnohem složitější seskupení atomů, jako jsou ty, se kterými pracuje chemie,
    můžeme zdlouhavým, pozoruhodným procesem, připomínajícím práci detektiva, zjistit tvar
    seskupení. Obr. \ref{fyz:fig0015} znázorňuje vzduch v blízkosti fialky: ve vzduchu opět nalézáme
    dusík, kyslík a vodní páru. (Odkud se vzala vodní pára? Fialka je vlhká, protože všechny
    rostliny odpařují vodu.) Vidíme však i \uv{monstrum} složené z uhlíkových, vodíkových a
    kyslíkových atomů, které vytvořily zcela určité, zvláštní seskupení. Je to mnohem složitější
    seskupení než v případě oxidu uhličitého. Naneštěstí do obrázku nemůžeme zakreslit všechno, co o
    něm po chemické stránce víme, neboť seskupení všech atomů je trojrozměrné, zatímco náš obrázek
    je pouze dvojrozměrný. Šest uhlíků vytváří ne plochý, ale \uv{zvrásněný} prstenec. Všechny úhly
    a vzdálenosti známe. Chemický vzorec je jen obrázkem takové molekuly. Když chemik napíše vzorec
    na tabuli, snaží se \uv{nakreslit} dvojrozměrný obraz molekuly. Například, vidíme \uv{prstenec}
    šesti uhlíků a na jednom konci visící \uv{řetěz} uhlíků, na něm kyslík druhý od konce, tři
    vodíky vázané na tento uhlík, dva uhlíky a tři vodíky vázané nahoře atd.

    \begin{figure}[hbt!]    % \ref{fyz:fig0016}
      \centering
      \luafigure[0.9]{fyz_fig0016.pdf}
      \caption{Strukturní vzorec vůně fialky (\(\alpha\)-iron) \cite[s.~24]{Feynman01}}
      \label{fyz:fig0016}
    \end{figure}

    Jak chemik zjistí, o jaké uspořádání jde? Smíchá obsah dvou lahviček a když se směs zbarví
    červeně, ví, že látka obsahuje jeden vodík a dva uhlíky vázané na určité místo molekuly.
    Zbarví-li se směs modře, je to úplně jinak. To je organická chemie - jeden z nejfantastičtějších
    kousků detektivní práce. Aby objevil uspořádání atomů v neobyčejně komplikovaných útvarech,
    chemik sleduje, co se děje při smíchání dvou rozdílných látek. Fyzik by nikdy zcela neuvěřil, že
    chemik ví, o čem mluví při popisu uspořádání atomů. Jenže asi před dvaceti lety se objevila
    fyzikální metoda umožňující v některých případech pozorovat molekuly (ne tak složité, jako je
    molekula vůně fialky, ale takové, které obsahují části této molekuly). Touto metodou je možné
    lokalizovat každý atom, a to ne sledováním zbarvení směsi, ale měřením skutečné polohy atomů. A
    světe, div se! Ukázalo se, že chemici měli téměř vždy pravdu. Zjistilo se, že vůně fialky
    obsahuje tři málo se lišící molekuly, jejichž rozdílnost spočívá pouze v jiném uspořádání
    vodíkových atomů. 
    
    Jedním z problémů chemie je tvorba chemického názvosloví. Každé molekule musíme najít jméno!
    Toto jméno musí ukazovat nejen její tvar, ale musí vyjadřovat i to, že tu je kyslíkový atom, tam
    vodíkový - musí říkat, kde přesně ten který atom je. Takto pochopíme, že chemické názvy musí být
    složité, aby byly-úplné. Název fialkové vůně má v podobě prozrazující strukturu následující
    znění: 4-(2,2,3,6 tetrametyl-5-cyklohexa\-nyl)-3-buten-2-on. Teď chápeme obtíže, se kterými
    chemici zápolí a rovněž chápeme příčinu tak dlouhých názvů. Není to proto, že by chemici chtěli
    být záhadnými, ale je to proto, že bojují s velmi obtížným problémem popisu molekuly slovy.
    
    \emph{Jak víme, že atomy existují?} Používáme k tomu trik, o kterém jsme se již zmínili:
    \emph{předpokládáme} jejich existenci a všechny výsledky, jeden po druhém, vycházejí tak, jak by
    měly, kdyby se látka skládala z atomů. Existují i přímější důkazy. Příkladem takového důkazu je
    následující skutečnost. Atomy jsou tak malé, že je nemůžeme vidět pomocí \emph{světelného
    mikroskopu} - dokonce ani pomocí \emph{elektronového mikroskopu}. (Světelným mikroskopem je
    možné vidět jen věci mnohonásobně větší.) Atomy jsou však v neustálém pohybu a když vložíme do
    vody nějaký míček, který je mnohem větší než atomy, bude poskakovat. Bude se chovat podobně, jak
    se chová velký míč postrkovaný při hře velkého množství lidí. Lidé postrkují míč různými směry a
    ten se pohybuje po hřišti nepravidelně. Právě tak se bude pohybovat \uv{velký míč} ve vodě,
    neboť v různých okamžicích na něj budou z různých stran dopadat nestejné údery. Proto při
    sledování velmi malých částeček (koloidů) ve vodě pomocí výborného mikroskopu (obr.
    \ref{fyz:fig0892}) pozorujeme jejich neustálé poskakování jako následek toho, že jsou
    bombardovány atomy. Tento jev se nazývá \textbf{Brounův pohyb}.

    \begin{figure}[hbt!]    % \ref{fyz:fig0892}
      \centering
      \luafigure[0.9]{fyz_fig0892.pdf}
      \caption{ V knize \emph{Les Atomes} z roku 1916, kterou napsal nobelista Jean Baptiste
                Perrin, jsou publikovány tři stopy pohybu koloidních částic o poloměru
                \SI{0.53}{\um}, pozorovaných pod mikroskopem. Postupné pozice jsou každých 30 sekund
                spojeny přímými segmenty (velikost ok síťe je \SI{3.2}{\um}.
                \cite[s.~115]{Perrin1914}}
      \label{fyz:fig0892}
    \end{figure}
    
    Další důkaz existence atomů můžeme vidět ve \emph{struktuře krystalů}. V mnoha případech
    souhlasí struktury odvozené na základě rentgenové analýzy svými prostorovými „tvary“ s formami
    samotných přírodních krystalů. Úhly mezi různými krystalickými „stěnami“ souhlasí s přesností na
    úhlové vteřiny s úhly určenými za předpokladu, že krystal je tvořen mnoha „vrstvami“ atomů.
    
    \textbf{Vše se skládá z atomů}. To je klíčová hypotéza. Například v celé biologii je
    nejdůležitější hypotézou to, že vše, co dělají živočichové, dělají atomy. Jinými slovy, v živých
    věcech není nic, co by nemohlo být pochopeno z pohledu, že se skládají z atomů podléhajících
    fyzikálním zákonům. To nebylo vždy známo: k formulování této hypotézy bylo třeba mnoha
    experimentů i teoretických úvah. Dnes je tato hypotéza uznávána a je nejužitečnější teorií pro
    vytváření nových myšlenek v oblasti biologie.
    
    Jestliže kousek oceli nebo kousek soli skládající se z uspořádaných atomů může mít tak zajímavé
    vlastnosti, jestliže voda - která není ničím jiným než těmi malými kapkami stejnými na celé Zemi
    - může tvořit vlny a pěnu, hučet příbojem a vytvářet podivné tvary omýváním břehů, jestliže toto
    všechno, celý život vodního proudu nemůže být ničím jiným než hromada atomů, co víc je ještě
    možné? Jestliže namísto uspořádání atomů podle určitého, stále opakovaného vzoru, nebo jestliže
    namísto tvorby malých, ale složitých shluků, jako je vůně fialky, seskupíme atomy v každém místě
    jinak, různé druhy atomů seskupíme různými způsoby tak, aby se nic neopakovalo, o co úžasněji se
    může takováto věc chovat? Je možné, že „věci“, které se před vámi procházejí a baví se s vámi,
    jsou velké shluky těchto atomů velmi složitým způsobem seskupené, takže pouhá naše představivost
    nestačí předpovědět jejich chování? Jestliže říkáme, že jsme shlukem atomů, nemyslíme tím, že
    jsme jen shlukem atomů, protože takový shluk atomů, který se nikdy neopakuje, může vypadat právě
    tak jako to, co vidíme v zrcadle.

  \section{Fyzikální veličiny a jejich měření}\label{fyz:IchapIsecIX}  
    Základem fyziky je \textbf{měření}. Objevovat fyziku znamená také poznávat možnosti měření
    veličin, které jsou s ní spjaty. Nazýváme je \textbf{fyzikálními veličinami}. Patří k nim
    například délka, čas, hmotnost, teplota, tlak nebo elektrický odpor. Každá fyzikální veličina se
    skládá z \emph{číselné hodnoty} a \emph{rozměru}. 
    
    Jinými slovy, abychom mohli fyzikální veličinu popsat, musíme zavést nejprve její
    \textbf{jednotku}, tj. takovou \emph{míru} (rozměr) této veličiny, které přisoudíme číselnou
    hodnotu přesně \num{1.0}. Poté vytvoříme \textbf{standard}, s nímž budeme všechny ostatní
    hodnoty dané fyzikální veličiny porovnávat. Tak například jednotkou délky je \emph{metr}. Jeho
    standard je definován jako vzdálenost, kterou urazí světlo ve vakuu za přesně definovaný zlomek
    sekundy. (K definici metru se ještě vrátíme.) Jednotku fyzikální veličiny i její standard můžeme
    definovat naprosto libovolným způsobem. 
    
    Jednotky fyzikálních veličin jsou nesmírně důležité a neměli bychom na ně zapomínat. Pokud v
    zápisu veličiny jednotku (rozměr) zapomeneme, může dojít k zajímavým absurditám, jak dokládá
    následující příklad:  

    %---- Zdánlivě nevinný výpočet rychlosti -----------------------
    % !TeX spellcheck = cs_CZ
\begin{mdframed}[style=mdexam]
  \begin{example}\label{fyz:exam023}
    Pokud v zápise rozměr zapomeneme, může dojít k zajímavým absurditám:
    \begin{equation*}
      v = \dfrac{s}{t} = \textcolor{red}{\dfrac{6}{3}} = \SI{2}{\m\per\s}
    \end{equation*}
    Zdánlivě nevinný výpočet rychlosti jako podílu dráhy a času je nesmyslný. U červeně označených
    hodnot chybí rozměry. Pokud bychom vzali v úvahu poslední rovnost, máme
    \begin{align*}
      \dfrac{6}{3} &= \SI{2}{\m\per\s}  \rightarrow \\
                 2 &= \SI{2}{\m\per\s}  \rightarrow \\
                 1 &= \si{\m\per\s}     \rightarrow \\
         \si{\s} &= \si{\m} 
    \end{align*}
    Docházíme tak ke zcela jistě nepravdivému tvrzení, že sekunda je totéž co metr. Dejme si proto
    pozor a nezapomeňme psát všude jednotky \cite[s.~1]{Kulhanek2020}. 
  \end{example}
\end{mdframed}
    %---------------------------------------------------------------

    Standardy základních veličin musí být dostupné a při opakovaném měření neproměnné. Kdybychom
    třeba definovali jako standard délky starý \emph{český sáh}, tedy vzdálenost mezi prsty
    rozpažených rukou (cca \SI{190}{\cm}), získali bychom bezpochyby standard snadno dostupný, avšak
    pro každého člověka jiný. Věda a technika však vyžadují přesnost, a proto je neproměnnost
    standardu mnohem důležitější než jeho snadná dosažitelnost. Je tedy třeba mít k dispozici
    dostatečný počet jeho přesných kopií i za cenu náročnosti jejich zhotovení.

    V klasické fyzice (včetně teorie relativity) mlčky předpokládáme, že měření můžeme provádět tak,
    abychom při něm měřenou hodnotu neovlivnili. (Nebo - realističtěji - tak, že vliv měření je
    zanedbatelně malý.) Tak např. při měření průměru šroubu mikrometrem stiskneme šroub čelistmi
    měřidla přesně definovanou silou. Tím jej nepatrně stlačíme a naměříme údaj menší. Tento rozdíl
    je pro šroub jistě zanedbatelný. Kdybychom však měřili gumový špalík, už by byl vliv patrný. Lze
    však jistě najít jiný, vhodnější způsob měření, který průměr špalku znatelně neovlivní. 
    
    V kvantové fyzice je problém měření mnohem složitější.

    \begin{tcnote}      
      Zapamatujme si:
      \begin{itemize}[leftmargin=10pt, noitemsep]
        \item Každá fyzikální veličina se skládá z hodnoty a jenotky. Jednotky nikdy nesmíme
              vynechávat, jsou stejně důležité jako číslo samotné.
        \item Jednotky fyzikálních veličin v sobě nesou důležité informace a mnohdy určují možný tvar
              fyzikálních zákonů (příklad \ref{fyz:exam024}).
        \item Veškeré matematické funkce musí mít bezrozměrné argumenty, například \(\sin(ωt)\),
              \(\log(I/I_0)\) atd.
      \end{itemize}
    \end{tcnote}

    \subsection{Mezinárodní soustava jednotek}\label{fyz:IchapIsecIXssecII}

      %---- Rozměrová analýza matematického kyvadla-------------------
      \begin{fyzexam}{Odhadněte na základě rozměrové analýzy tvar vztahu pro úhlovou frekvenci kmitů
  matematického kyvadla \cite[s.~1]{Kulhanek2020}.}{exam024}  

  Předpokládejme, že frekvence kmitů bude záviset na délce závěsu \(l\), na hmotnosti zavěšené
  kuličky \(m\) a na tíhovém zrychlení \(g\), tj.
  \begin{equation*}
    ω= ω(l, m, g).
  \end{equation*}

  {\centering
  \luafigure[0.6]{fyz_fig0925.pdf}
  \captionsetup{type=figure}  
  \captionof{figure}{Matematické kyvadlo}           
  \label{fyz:fig0925}
  \par}

  Dále předpokládejme, že vztah pro úhlovou frekvenci je jednoduchý a lze ho zapsat jako většinu
  fyzikálních vztahů za pomoci mocninných závislostí:
  \begin{equation*}
    ω= l^\alpha m^\beta g^\gamma.     
  \end{equation*}
  Na první pohled se zdá úloha neřešitelná. Máme totiž jedinou rovnici pro tři neznámé \(α\), \(β\),
  \(γ\). Ve fyzice je každá rovnice nejen rovností číselných hodnot, ale i rovností rozměrů. Pokud
  zapíšeme rozměry všech veličin na levé a pravé straně rovnosti, dostaneme
  \begin{equation}\label{fyz:eq748}
    \dfrac{1}{\si{\s}} = \si{\m}^α\si{\kg}^β\left(\dfrac{\si{\m}}{\si{\square\s}}\right)^γ.
  \end{equation}
  V posledních dvou vztazích si opět povšimněte, že proměnné jsou sázeny šikmým a jednotky svislým
  řezem písma. Nyní porovnejme mocninné koeficienty u sekundy, kilogramu a metru na obou stranách
  rovnosti:
  \begin{align*}
    \si{m}:  \qquad 0 &= α + γ, \\
    \si{kg}: \qquad 0 &= β,     \\
    \si{s}:  \qquad-1 &= -2γ.
  \end{align*}
  Rovnice (\ref{fyz:eq748}) je skutečně řešitelná. Snadno zjistíme, že \(β = 0\), \(γ = ½\), \(α =
  −½\). Úhlová frekvence kyvadla tedy je: 
  \begin{equation}\label{fyz:eq749}
    ω=\sqrt{\dfrac{g}{l}}.
  \end{equation}
\end{fyzexam}
      %---------------------------------------------------------------
      \begin{tcnote}
        K odvození vztahu (\ref{fyz:eq749}) jsme nepotřebovali znát žádné fyzikální mechanizmy:
        \begin{itemize}[leftmargin=10pt, noitemsep]
          \item Pouhá rozměrová analýza určila jediný možný tvar fyzikálního zákona.
          \item Odvodili jsme pouze tvar zákona, nikoli číselný koeficient před ním. Před odmocninou
                by mohla být jakákoli bezrozměrná konstanta, například \num{2}, \num{3}, \(π\). V
                našem případě je koeficient před odmocninou skutečně roven jedné. K určení
                koeficientu by postačil jediný experiment. Kdybychom ale chtěli experimentálně
                odvodit celý vztah, museli bychom provádět sady měření s různými délkami závěsů,
                různými hmotnostmi těles a v různých tíhových zrychleních.
          \item Výsledný vztah nezávisí na hmotnosti tělesa. Tělesa všech hmotností kývají na
                konkrétním závěsu se stejnou frekvencí. To není náhoda. Jde o velmi důležitou
                vlastnost gravitace. Všechna tělesa se v gravitaci pohybují stejným způsobem.
                Například malá kulička a cihla dopadnou na zem při volném pádu za stejný čas. K této
                vlastnosti gravitačního pole se ještě vrátíme. 
        \end{itemize}
      \end{tcnote}

    \subsection{Co jsou to Planckovy škály?}\label{fyz:IchapIsecIXssecI}
      Na počátku 20. století ukázal \textsc{Max Planck}, že tři fundamentální konstanty \(c\), \(G\)
      a \(\hslash\) lze jednoznačným způsobem (až na násobící číselný faktor) zkombinovat tak,
      abychom získali veličinu, která má rozměr času. Obdobně lze vytvořit jednoznačné kombinace,
      které mají rozměr délky, hmotnosti a energie. Těmto veličinám se říká \textbf{Planckovy
      škály}. Výsledné hodnoty jsou více než zarážející. Planckova délka, Planckův čas, Planckova
      hmotnost a energie by měly být jakýmisi přirozenými jednotkami v našem vesmíru. Pak se ale
      musíme ptát: "Proč je náš vesmír tak veliký, tak starý a tak hmotný? Jaký je význam
      Planckových škál?"

      Zdá se, že na některé otázky dávají odpověď dnešní kosmologické modely založené na
      sjednocovacích teoriích gravitace s ostatními interakcemi. \textbf{Planckův čas} zde
      koresponduje s okamžikem oddělení gravitační interakce od ostatních interakcí. Teprve od doby
      \SI{e-43}{\s} zde začíná fungovat samostatná gravitační interakce a pro popis vesmíru je možné
      použít obecnou relativitu. V časech dřívějších musíme uvažovat i ostatní interakce.
      \textbf{Planckova energie} je potom typickou energií částic v Planckově čase, tedy v době
      oddělení gravitační interakce. \textbf{Planckova hmotnost} je jen hmotnostní ekvivalent
      Planckovy energie (hmotnost a energie souvisí vztahem \(E = mc^2\)).
      
      \emph{Kvantová teorie} popisuje úspěšně slabou, silnou a elektromagnetickou interakci. K
      základním axiomům patří \emph{nekomutativnost teorie} (k popisu přírody využívá nekomutující
      operace). \emph{Gravitační interakce} je popsána obecnou relativitou, kde k základním axiomům
      patří zakřivení časoprostoru hmotnými objekty. Spojení obou teorií je těžko řešitelný oříšek.
      Nejnadějnější se zdá \emph{teorie strun}, ve které jsou elementární částice jednorozměrné
      útvary ve vícerozměrném vesmíru. Nejčastěji používaný počet dimenzí vesmíru je v těchto
      teoriích deset. Desetirozměrný svět, ve kterém jsou čtyři pozorovatelné dimenze (čas a
      prostor) a šest je nějakým způsobem zavinutých (tzv. \emph{kompaktifikovaných}) tak, že nejsou
      viditelné. Představme si chomáč vaty, na který se díváme z dálky – vypadá jako třírozměrné
      těleso. Když se podíváme zblízka, uvidíme jednotlivá vlákénka. I náš svět vypadá jinak při
      našem pohledu a jinak na úrovni malých rozměrů. Někdy se používá pojem \emph{kvantová pěna}. V
      každém případě by Planckova délka měla odpovídat nejmenším strukturám na této úrovni, ať už
      jakýmsi vlákénkům či pěně.
      
      Přicházíme tak k zajímavému poznání. V hodnotách fundamentálních konstant jsou zakódovány
      informace z nejranějších fází existence tohoto světa. A při poznání, že měřením rychlosti
      světla, gravitační a Planckovy konstanty se dozvídáme poselství 14 miliard let staré, až
      zamrazí.

      %---- Planckovy škály-------------------------------------------
      % !TeX spellcheck = cs_CZ
\begin{fyzexam}{Nalezněte takové kombinace konstant \(c\), \(G\), \(\hslash\) (rychlosti světla,
  gravitační konstanty a Planckovy konstanty), které dají přirozenou jednotku pro délku, čas,
  hmotnost a energii.}{exam025}  
    \begin{subequations}\label{fyz:eq750} 
      \begin{align}
        c       &= \SI{3e8}{\m\per\s}                              \label{fyz:eq750a}  \\
        G       &= \SI{6,67e-11}{\per\kg\cubic\m\per\square\s},    \label{fyz:eq750b}  \\
        \hslash &= \SI{1,05e-34}{\kg\square\m\per\s} .             \label{fyz:eq750c}
      \end{align}
    \end{subequations}

    Pokusíme se vytvořit výraz pro délku \(l_0\), čas \(t_0\), hmotnost \(m_0\) a energii \(E_0\).
    Začneme délkou tak, že napíšeme součin výše uvedených tří konstant, s neznámými exponenty \(α\),
    \(β\), \(γ\): 
    \begin{equation*}
      l_0 = c^αG^β\hslash^γ.
    \end{equation*}
    Tato rovnice ve skutečnosti představuje čtyřnásobnou rovnost: rovnost číselnou a rovnost
    rozměrovou v metrech, kilogramech a sekundách. Napíšeme nyní rozměrové části vytvořeného výrazu:
    \begin{equation*}
      \mathrm{m^1kg^0s^0} = \si{\m}^α\si{\s}^{-α}                   %c
                            \si{\kg}^{-β}\si{\m}^{3β}\si{\s}^{-2β}   %G
                            \si{\kg}^γ\si{\m}^{2γ}\si{\s}^{-γ}.    %hslash
    \end{equation*}
    Nyní zapíšeme soustavu rovnic pro exponenty u metru, kilogramu a sekundy:
    \begin{alignat*}{9}
      &\;1 &=&   &α &+ &3β &\;+ &2γ    \\
      &\;0 &=&   &  &- & β &\;+ & γ    \\
      &\;0 &=& - &α &- &2β &\;- & γ    
    \end{alignat*}
    Řešením této soustavy získáme jednoznačné řešení pro exponenty
    \begin{equation*}
      α =−3/2;\quad β=1/2;\quad γ=1/2. 
    \end{equation*}
    Tyto exponenty jednoznačně až na násobící číselný faktor určují velikost Planckovy délky. Zcela
    analogickým způsobem můžeme odvodit vztahy pro ostatní Planckovy veličiny. Výsledky jsou:
    \begin{subequations}\label{fyz:eq751} 
      \begin{align}
        l_0&=\sqrt{\dfrac{\hslash G}{c^3}}\approx \SI{e-35}{\m},               \label{fyz:eq751a}\\
        t_0&=\sqrt{\dfrac{\hslash G}{c^5}}\approx \SI{e-43}{\s},               \label{fyz:eq751b}\\
        m_0&=\sqrt{\dfrac{\hslash c}{G}}  \approx \SI{e-8}{\kg},               \label{fyz:eq751c}\\
        E_0&=\sqrt{\dfrac{\hslash c^5}{G}}\approx \SI{e19}{\giga\electronvolt},\label{fyz:eq751d}
      \end{align}
    \end{subequations} 
\end{fyzexam}
      %---------------------------------------------------------------