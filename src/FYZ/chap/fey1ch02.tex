% !TeX program = lualatex
% !TeX root = luaking.tex
% !TeX encoding = UTF-8
% !TeX spellcheck = cs_CZ
%---------------------------------------------------------------------------------------------------
% file fey1ch02.tex
%---------------------------------------------------------------------------------------------------
%===================== Kapitola: Dějiny fyziky =====================================================
\setchaptertoc
\chapter{Dějiny fyziky}\label{fyz:IchapII} 
  \section{Člověk a příroda}
    \subsection{Co je fyzika}
      Řecké slovo \emph{„fýsis“} znamená \emph{„příroda“}, takže fyzika je vlastně věda o přírodě.
      Toto označení vzniklo v době, kdy vědy o přírodě nebyly ještě rozlišeny. Dnes se zkoumáním
      přírody zabývá i chemie, biologie, ale třeba i mineralogie, geologie, botanika, zoologie a
      další. Tyto vědy jsou ovšem zaměřeny vždy na určitou oblast přírodních jevů. Chemii zajímá,
      jak se atomy prvků spojují v molekuly sloučenin, biologii zas projevy živé hmoty, která je,
      pokud zatím víme, omezena jen na naši planetu. Někdy je těžké rozhodnout, zda nějaký jev patří
      do fyziky, nebo do jiné přírodní vědy - a tak máme i fyzikální chemii, biofyziku, geofyziku,
      astrofyziku a jiné. 
      
      Název fyzika v dnešním smyslu se začal používat až v 19. století. Dříve byla: fyzika
      považována za součást filozofie a používal se pro ni název „přírodní filozofie“. Ten ostatně
      dosud přežívá v některých západních jazycích. Slovem „fysik“, „fysikus“ se zpravidla rozuměl
      lékař v úřední funkci hygienika. ještě dnes někteří anglicky hovořící mluvčí zaměňují ve svém
      jazyce výraz \emph{„physicist“} (fyzik) a \emph{„physician“} (lékař), což někdy vede k
      humorným situacím. 
      
      Fyzika zkoumá obecné vlastnosti přírody a zákonitosti společné všem oblastem přírodních jevů.
      Zabývá se strukturou a vývojem vesmíru, vlastnostmi částic, z nichž se skládají tělesa,
      planety a hvězdy, vlastnostmi polí, jimiž tyto částice mezi sebou vzájemně působí. Gravitační
      pole zprostředkuje přitažlivou sílu mezi Sluncem a planetami a vlastně vládne celým vesmírem,
      ale ovlivňuje i růst rostlin a krevní oběh člověka. Podobně elektrické pole vyvolává blesky,
      ale je i základem síly chemických výbuchů a umožňuje přenos iontů mezibuněčnými membránami a
      podmiňuje i sílu našich svalů. 

      Jiné přírodní vědy jako chemie, biologie nebo lékařství využívají fyzikální zákony, fyzikální
      přístroje a měřicí či diagnostické metody. Moderní technika a technologie, průmyslová výroba,
      doprava a komunikace jsou založeny na využití fyzikálních principů. A to už nemluvíme o
      uplatnění počítačů, ]eiichž pavoučí sítě stále více opřádají zeměkouli a blízký kosmos. V
      tomto smyslu je tedy fyzika základní, nechceme-li říci přímo nejdůležitější, přírodní vědou. 
      
      To nás přivádí k otázce, co je to vlastně \emph{věda}. Věda je jistě určitý \emph{způsob
      poznávání světa}. Vychází z přesvědčení, že tento svět existuje nezávisle na nás, že
      existoval, když jsme tu ještě nebyli, a bude existovat, až tu nebudeme, a že je poznatelný. I
      když není poznatelný úplně a do všech podrobností a i když naše poznání je vlastně nikdy
      nekončící proces doprovázený úspěchy i omyly - v tom je i určitá romantika vědy.  
      
      Poznaný svět, to je jakási kopie, odraz vnějšího světa v naší mysli. Abychom mohli s tímto
      odrazem pracovat, musíme vnější svět nejen poznat, ale i pochopit a naučit se události v něm
      předvídat. Krásným příkladem je skutečnost, že se astronomie naučila přesné určovat postavení
      planet, příchody komet nebo zatmění Slunce a Měsíce v minulosti i v budoucnosti. To ovšem
      předpokládá že ve světě platí \emph{přírodní zákony}, které můžeme odhalit rozumově postihnout
      a matematicky vyjádřit. Věda nemůže připustit zázraky: kdyby nějaká vyšší moc měnila zákony
      volného pádu podle své libovůle, nemohla by existovat fyzika ani technika. Věda je tedy
      rozumové poznávání světa a jeho zákonů pomocí logického a matematického uvažování. Správnost
      svých předpovědí musí věda experimentálně prakticky ověřovat.

      Neznamená to ovšem, že všechno se dá předem vypočítat, že vše je předurčeno, že do hry nemůže
      vstoupit náhoda. Především nemůžeme nikdy znát hodnoty změřených fyzikálních veličin s
      naprostou přesností. Pověstná astronomická přesnost nám sice umožňuje vypočítat okamžik
      zatmění Slunce na deset milionů let dopředu, ale už ne na sto milionů let, ať použijeme
      sebevýkonnější počítač. Víme, jak usilovně se dnes astrofyzikové snaží určit dráhy malých
      nebeských těles, planetek a komet, aby mohli včas varovat před jejich srážkou se Zemí.

      Stejně tak nemůžeme sledovat náhodné pohyby každé jednotlivé molekuly plynu a musíme se naučit
      zkoumat i statistické zákonitosti a vlastnosti chaosu. A konečně moderní kvantová fyzika
      dokázala matematicky popsat podivuhodné vlastnosti částic mikrosvěta, které se chovají jako
      vlny a řídí se zákony pravděpodobnosti. Fyzika tady není jen suchá, matematicky strohá věda,
      která umí všechno předvídat, ale počítá i s náhodou. Náhoda je ostatně i kořením života -
      můžeme najít poklad, nebo nám ale také může na hlavu spadnout meteorit

      Vedle vědy mohou existovat i jiné způsoby poznávání světa. Především je to umění, které je
      charakteristickým projevem vyspělého člověka našeho druhu a jedním z pilířů kultury. Poznání
      může nabývat i podobu mytologie a náboženství: lidé si tak vysvětlovali záhadné a
      nepochopitelné přírodní jevy. Není to poznání rozumové, ale spíše duchovní a emotivní,
      připadnė založené na víře, a je třeba ho od vědeckého poznání odlišovat. Ale mytologické
      představy starých národů vycházely z tisíciletých zkušenosti a pozorování, představovaly
      přírodní děje ve fantastické a poetické podobě, a maji nám i dnes co říci. Ani fyzika se
      neobejde bez intuice a fantazie.

      \luagraphic[1]{fyz_fig0939.jpg}{\wikiDavidTeniersYounger, Interiér laboratoře s alchymistou
        17. století. Téma lékařů a alchymistů bylo ve vlámském umění v 17. století velmi populární.
        David Teniers mladší byl hlavním přispěvatelem k tomuto žánru ve Vlámsku. Kredit:
        Wikipedia}{fyz:fig0939}

      Existují ovšem i vysloveně falešné přístupy k světu, které využívají a zneužívají lidskou
      pověrčivost a důvěřivost. Označujeme je jako \emph{pseudovědy}. Jako příklad za mnohé můžeme
      uvést astrologii, přesvědčení o tom, že postavení hvězd v okamžiku narození má vliv na lidské
      osudy. Pseudovědy existovaly od pradávna, často odvozují svůj původ od mystických učení
      východních národů a v jistém smyslu ovlivnily i vývoj vědy. Tak astrologie šla po staletí ruku
      v ruce s astronomii (stará čeština rozlišovala \emph{„hvězdář"} astronom, a \emph{„hvězdník"}
      astrolog), a protože byla mocnými finančně podporována, přispěla tak i k rozvoji pozorovací
      astronomie. Úsilí alchymistů o nalezeni „kamene mudrců a výrobu zlata z obyčejných kovů
      přispělo ke zdokonalení chemických laboratorních operací, a dokonce k objevu některých nových
      prvků. Ostatně výrobu zlata z jiných kovů zvládla až moderní jaderná fyzika; taková výroba se
      ovšem nevyplatí.

      Výsledky moderní vědy, zejména fyziky, jsou bez dlouhého soustavného studia a bez zvládnutí
      matematiky málo srozumitelné, podobně jako pohled nezasvěceného do notové partitury symfonie.
      Je jistě lákavější a více vzrušující číst o záhadných, nevysvětlitelných a nadpřirozených
      jevech. Víru těch, kdo jsou o nevědeckých představách přesvědčeni, nelze vyvrátit rozumovými
      důvody. Tito lidé jsou dokonce svým způsobem šťastni, zejména v případech, kdy věda ještě
      nemůže nabídnout racionální vysvětlení nebo pomoc. Odmyslíme-li si zištné šarlatány a
      podvodníky, ohrožující majetek nebo zdraví svých klientů, mohou být některé pověry docela
      neškodné. Věří-li někdo ve svůj horoskop nebo šťastné dny, nic proti tomu, pokud ovšem
      neobětuje v takový den všechny své úspory do loterie v přesvědčení, že musí vyhrát.

      Známý britský filozof rakouského původu \textsc{Karl Popper} navrhl k rozlišení věd od
      nevědeckosti \emph{„princip falzifikace"}. Vědecké poznání může být dalším vývojem, novými
      experimenty nebo výpočty opraveno a nahrazeno novým, přesnějším. Aristoteles se domníval, že
      světlo se šiří okamžitě, nekonečnou rychlosti Galileo se pokusil tento názor vyvrátit,
      „falzifikovat", a dnes víme přesně, jakou rychlosti se světlo šiří, Pro vědu je
      charakteristické, že si je vědoma své ohraničenosti, že zná hranici mezi věděním a nevěděním.
      Byl to právě Newton, jeden z největších vědců, který jako jeden z prvních dokázal říci
      \emph{„nevím"}, nepodložené domněnky si nevymýšlím, odpověď na nevyřešené otázky přenechávám
      dalším pokolením.

      \textbf{Věda je tedy historický proces soustavného rozumového poznávání světa, vytváření jeho
      vědeckého obrazu. Přitom věda ověřuje své poznatky, sama sebe opravuje a je vždy připravena
      připustit omyl nebo nepřesnost svého poznání. \cite[s.~15]{Stoll2009}}

      \begin{tcnote}
        \textbf{Vědecká metoda} je posloupnost nebo sada procesů, používaných při vědeckém výzkumu.
        Cílem je získat znalosti a vědomosti pomocí pozorování a dedukce na základě dosud známých
        poznatků. Přijímání nových vědeckých poznatků je založeno na konkrétních důkazech. Vědecká
        metoda je založena na předpokladu, že kritériem pravdivosti vědecké hypotézy je souhlas
        předpovědí s výsledky výzkumu. Tento přístup udržuje vědecké \emph{hypotézy} v neustálém
        kontaktu s realitou a umožňuje jejich \emph{falzifikaci}, neboť hypotéza, jejíž důsledky
        jsou v rozporu s výzkumnými zjištěními, bude falzifikována. Mnohokrát ověřená hypotéza,
        kterou se zatím nepovedlo vyvrátit, se stává vědeckou teorií. Důsledkem je omezení vědy na
        otázky a hypotézy, jež jsou alespoň v principu rozhodnutelné pozorováním. Větší vědecký tým
        je však spíše konzervativní, což brzdí výzkum a vývoj.

        {\centering
        \captionsetup{type=figure}
        \luafigure[0.7]{fyz_fig0924.jpg}
        \captionof{figure}{\wikiPopper \textasteriskcentered	28. července 1902, Vídeň \textdagger
          17. září 1994 (ve věku 92 let). Byl významným představitelem moderního liberalismu, teorie
          vědy a filosofie.Kredit: Wikipedia}
        \label{fyz:fig0924}
        \par}

        \textbf{Falzifikovatelnost} (vyvratitelnost nebo zpochybnitelnost) je ve filosofii vědy
        vlastnost takového vědeckého tvrzení, hypotézy nebo teorie, kterou je principiálně možné
        vyvrátit, například experimentem. Jinak řečeno, tvrzení nebo teorie je falzifikovatelná
        tehdy, pokud víme, jak by se dala vyvrátit čili negovat. Teorie, kterou experimenty
        potvrzují, sice platí, ale jen dokud ji nějaký experiment nevyvrátí. Problémem
        falzifikovatelnosti se především zabýval rakouský filozof kritického racionalismu
        \textsc{Karl R. Popper}, který zdůraznil, že žádný počet pokusných potvrzení nemůže vědeckou
        teorii definitivně dokázat. Na rozdíl od verifikace, která je vždy jen částečná, pouze
        falzifikace může být definitivní.

        Ve skutečnosti se jedná o podmínku testovatelnosti a vyvratitelnosti hypotéz a teorií.
        Princip, který takto formuloval, se podle něj nazývá \textbf{Popperova břitva} – „Vědecké
        teorie jsou ověřitelné. Ověřitelné teorie je možné na základě ověřovacího postupu zamítnout
        (a nahradit teoriemi jinými).“

        \tcblower
        Pravdivost vědecké teorie podle něj nelze dokazovat, ale jen empiricky testovat. Základem
        vědeckého poznání tedy není verifikace (potvrzení), ale falsifikace. Pouze ta teorie, kterou
        je možné podrobit falsifikaci, tedy vystavit ji možnosti vyvrácení, je vědecká, tím větší
        hodnotu má pro vědu. Netřeba hromadit důkazy, které teorii potvrzují; spíše hlídat to, co by
        ji mohlo vyvrátit. Konečnou, definitivní jistotu naší přesné znalosti pravdy nemůžeme mít
        nikdy, k pravdě se můžeme pouze přibližovat neustálým vylučováním falsifikovaných teorií,
        hovoříme o evoluci vědy. K evoluci, vývoji vědy dochází právě díky falzifikaci: tím, že něco
        popřeme, získáváme nový prostor pro otevření dalších, nových otázek.

        Důsledkem jeho pojetí vědeckého poznání je obrana otevřeného myšlení a otevřené společnosti.
        Síla vědy netkví tolik v tom, že se její tvrzení dají dokázat, nýbrž v tom, že musí být
        formulována tak, aby se dala vyvrátit. Právě tak síla demokracie nespočívá v tom, že by
        vybírala ty nejlepší k vládě, nýbrž že každou vládu lze běžnými prostředky (volbami) odvolat
      \end{tcnote}

      Jakými způsoby ale fyzika přírodu poznává a jak ověřuje správnost svého poznání? Poznání
      začalo praktickou zkušenosti. Tisíce let předtím, než Galileo odhalil zákony volného pádu a
      šikmého vrhu těles, věděl pravěký člověk, jak má hodit kámen, oštěp nebo vystřelit šíp, aby
      zasáhl kořist. Používal úspěšně bumerang, jehož let dnes vyžaduje náročný matematický popis.
      Aniž by znal zákon lomu světelného paprsku vycházejícího z vody do vzduchu, věděl, kam má
      namířit harpunu, aby zasáhl rybu. Při výrobě nástrojů z kamene používal znalosti o jeho
      mechanických vlastnostech, tvrdosti, křehkosti, štěpitelnosti. Všechny tyto znalosti a
      poznatky byly vyvolány životní nutnosti, předávaly se od pokolení k pokolení a člověk k jejich
      získávání měl miliony let času. Můžeme si o nich udělat dobrou představu ještě dnes studiem
      života izolovaných primitivních národů.

      Takové poznáváni přírody bylo od počátku spojeno s praxí, dalo by se říci s technikou. Člověk
      nebyl pasivním pozorovatelem přírody, ale aktivně na ni působil, i když ještě nemohl přírodní
      rovnováhu nijak vážněji ovlivnit. Tříbil si přitom důvtip a občas se mu podařilo přírodu nějak
      obelstít, usnadnit si práci. Poznal účinnost páky, odstředivé sily, pružnost lučiště, strojil
      zvěři důmyslné pasti. Řecké slovo „techné" znamená řemeslo, dílo lidských rukou, ale také lest
      nebo úskok. I dnes se snažíme pomocí techniky nad přírodou s větším nebo menším úspěchem
      vyzrát. „Fýsis" a „techné" tak vždy kráčely ruku v ruce. 

      U zdrojů fyziky stojí však i další lidská vlastnost, která se zrodila někdy během dlouhého
      vývoje - přirozená zvídavost. Člověk si dokázal přírodního jevu všimnout, podivit se mu a
      zvědavě ho pozorovat. Nedovedl ho ještě vysvětlit, a tak zapojoval svou fantazii, vytvářel
      antropomorfní představy, mýty a náboženské rituály. Údiv a zvídavost jsou tedy vedle
      zkušenosti prvním ze zdrojů poznání. 

      Systematickým, dlouhodobým pozorováním můžeme odhalit určité pravidelnosti, zákonitosti,
      například sled fázi Měsíce, slunečních nebo měsíčních zatmění. Pozorované pravidelnosti se
      můžeme pokusit vyjádřit v podobě přírodního zákona, a dokonce ho popsat matematicky. To je
      ovšem jen první etapa fyzikálního poznání; vytváříme vědeckou \textbf{hypotézu}, domněnku. Z
      ní pak můžeme vyvodit předpověď, která ovšem musí být teprve ověřena. Abychom ji ověřili,
      musíme provést experiment, jak se říká „položit přírodě otázku". Kdysi to bývalo vyjádřeno i
      domýšlivě drasticky - položit přírody na skřipec a vypáčit z ní odpověď! 

      K fyzikálnímu experimentu je ovšem dlouhá cesta. Musíme zavést pojem fyzikální veličiny,
      kterou chceme měřit, zkonstruovat měřici přístroje a metody Těžko bychom mohli odhalit zákony
      tepelných jevů, kdybychom neměli k dis pozici teploměry, tlakoměry a kalorimetry Jediný
      experiment ovšem nestačí, musíme si být jisti, že námi objevený zákon bude platit za stejných
      podmínek vždy opakovaně a stejně. 

      Nejpřesvědčivějším důkazem správnosti našeho poznání je, podaří-li se ho uplatnit v technické
      praxi. Dnes už například nikdo soudný nemůže pochybovat o správnosti Einsteinovy speciální
      teorie relativity, protože na jejím základě fungují nesčetné urychlovače, elektronické
      přístroje a v podstatě celá elektrotechnika, nechceme-li už mluvit o jaderné energetice.
      Naproti tomu obecná teorie relativity popisuje gravitační pole ve vesmíru a tam jsme ovšem
      odkázáni jen na přesná a důmyslná pozorovaní. S vesmírem zatím ještě experimentovat
      nedovedeme, gravitační sily ovládat neumíme.

      \begin{tcnote}
        Můžeme tedy říci, že fyzika je \emph{základní věda} o nejobecnějších vlastnostech přírodních
        objektů a zákonitostech přírodních jevů, která vychází z pozorování, zkušenosti a
        experimentů, jejich výsledky zpracovává matematicky a své výpočty a teorie systematicky
        experimentálně ověřuje. Výsledky fyzikálního poznání slouží lidstvu v jeho technické a
        společenské praxi a z této praxe čerpá fyzika opět nové podněty a prostředky ke svému
        výzkumu. \cite[s.~16]{Stoll2009} 
      \end{tcnote}
    
    \newpage  
    \subsection{Vývoj fyziky}\label{fyz:IchapIIsecIII}     
      Věda v obecnosti, a fyzika zejména, se v průběhu dějin vyvíjí a mění. Proces fyzikálního
      poznávání, jakmile byl jednou spuštěn a stal se nedílnou součástí lidské kultury, pokračuje
      bez ustáni vpřed a nemůže být nikdy vyčerpán, zastaven nebo ukončen. Přitom neprobíhá stále
      stejně rychle, zná období trpělivého shromažďování poznatků, dokonce období stagnace a úpadku,
      a pak zase rychlého bouřlivého postupu vpřed, kdy mluvíme o vědecké revoluci. V průběhu dějin
      lidské společnosti vzniká čas od času naléhavá potřeba a poptávka po výsledcích a uplatnění
      vědeckého výzkumu. Takovými podněty byly např. rozvoj mořeplavby a zámořských cest, rozvoj
      dopravy, komunikaci a průmyslu, aeronautiky a kosmonautiky a bohužel příliš často i války.

      Fyzika má mezinárodní charakter, dochází k přejímání a zprostředkovávání nových poznatků mezi
      vzdálenými národy a kulturami. Vnějším projevem této skutečnosti je i to, že si fyzika vždy
      vyžadovala mezinárodní jazyk bez ohledu na hranice států. Tak na přelomu našeho letopočtu byla
      takovým jazykem ve Středomoří helénistická řečtina, ve středověké Evropě latina a v dnešní
      globální době angličtina. Důležitým jednotícím prostředkem mezinárodního porozumění fyziků je
      ovšem jazyk matematických symbolů, který hraje podobnou úlohu jako notový záznam u hudebníků.

      Zkoumáním průběhu vývoje vědy a vědeckých revoluci se zabýval americký historik, původně
      vystudovaný fyzik, \textsc{Thomas Kuhn}, autor knihy \emph{„Struktura vědeckých revolucí"}.
      Podle něho se ve vývoji fyziky střídají období tzv. „normální vědy", kdy dochází k hromadění
      poznatků v rámci určitého výkladu světa (Kuhn si pro takový výklad vymyslel dnes módní termín
      „paradigma", z řeckého vzor, předloha), a pak kritická období, vědecké revoluce, kdy je staré
      „paradigma" zavrženo a nastupuje nové. S takovým zjednodušeným a do jisté míry zavádějícím
      přístupem k vývoji vědy však někteří přední světoví fyzikové nesouhlasí. Patři k nim např.
      historik fyziky, profesor Harvardovy univerzity \textsc{Gerald Holton} nebo laureát Nobelovy
      ceny \textsc{Steven Weinberg}, kteří Kuhnovu koncepci ostře kritizuji. Je to pojetí, které
      nahrává některým obecným názorům o tom, že fyzika vlastně nemá pravdivou, objektivní představu
      o světě, že její zákony jsou věci jakési „dohody" nebo „povinné víry", kterou fyzikové čas od
      času popřou a vytvoří novou.

      \begin{tcnote}
        \textbf{Kuhnovo pojetí vývoje vědy}: \textsc{Thomas Samuel Kuhn} přinesl argumenty o tom,
        že pokrok vědeckého poznání není přímočarý, ale že je čas od času přerušován zásadními
        zvraty-vědeckými revolucemi. Při těchto vědeckých revolucích dochází k přehodnocení
        samotných základů dosavadního vědění. Vědecké poznání tedy nesměřuje k nějaké jediné
        pravdě o světě, netýká se žádné „objektivní reality“ - nezávislé skutečnosti, všem
        společné, vždy zde již jsoucí. Věda, tak jako každá lidská činnost, má svůj kulturní,
        dějinný, instituční, sociální a psychologický rozměr. I vědecké poznatky jsou proto
        historicky podmíněné: vyjadřují ducha dané epochy, mění se s dobou i s okolnostmi.

        {\centering
        \captionsetup{type=figure} 
        \luafigure[0.5]{fyz_fig0893.jpg}
        \captionof{figure}{\wikiKuhn (\textasteriskcentered 18. 7. 1922 - \textdagger 17. 6. 1996)
                  byl americký filosof, fyzik, teoretik vědy a vědeckého poznání, zabýval se
                  dějinami vědy, astronomií, kvantovou teorií a její prehistorií.}
        \label{fyz:fig0893}
      \par}
      \end{tcnote}

      Situace je složitější. Je známou vlastnosti lidského poznání, že čím více poznáváme, tím vice
      zjišťujeme, kolik toho ještě nevíme. Každá zodpovězená otázka navozuje desítky nových. Při
      hromadění poznatků se mezi nimi mohou objevovat rozpory, které se nedaří vysvětlit. V rámci
      daného přijatého výkladu přírody se objevují myšlenky, které připravují půdu novému pohledu na
      svět, \textbf{vědecké revoluci}. Ta však neznamená, že by předchozí vědění bylo odvrženo.
      znehodnoceno a popřeno, ale dochází k jeho zobecnění, prohloubení a vzniká pohled z nové,
      vyšší úrovně poznání.

      Ani nástup koperníkovské revoluce neznamenal, že by pozorování a práce předchozích generací
      astronomů byla znehodnocena, nehledě k tomu, že už ve starověku se vedle geocentrické soustavy
      uplatňovaly i názory, že se Země pohybuje as ostatními planetami obíhá kolem Slunce. Podobně
      Einsteinova teorie relativity neznamenala odvržení, ale zobecnění Newtonovy fyziky. Ostatně
      celá astronautika a kosmická technika počítá stále se zákony Newtonovy mechaniky a až na
      některé výjimky nemusí brát v úvahu jemné efekty obecné teorie relativity. Kvantová fyzika ve
      20. století nepopřela předchozí fyziku klasickou, ale naopak objasnila rozpory v chápání
      dvojaké podstaty světla a otevřela fyzice cestu k dalším oblastem výzkumu, do mikrosvěta.
      Změny fyzikálních teorii a skoky ve vývoji fyziky nám nesmějí zakrýt pohled na vnitřní
      souvislost historického poznávání, k němuž přispívalo mnoho a mnoho známých i neznámých
      badatelů, kteří obohacovali naše znalosti o přírodních zákonech, vytvářeli mozaiku našeho
      obrazu světa.

      \luagraphic[1]{fyz_fig0940.jpg}{Eduard Ender - Kopernikus, die Bahnen der Gestirne
      aufzeichnend - 3764 - Kunsthistorisches Museum Kredit: Wikipedia}{fyz:fig0940}

      V dějinách fyziky si musíme ujasnit, kdy vlastně fyzika začíná a jak její historické etapy
      rozčlenit. Kdybychom vycházeli z dnešního stavu, kdy na světě pracuji ve vědeckých ústavech,
      na vysokých školách a v průmyslu tisíce fyziků ve spolupráci s inženýry a dalšími odborníky,
      kteří se scházejí každý rok na stovkách mezinárodních konferenci, kdy vyspělé země ve vzájemné
      spolupráci investují miliardy dolarů do fyzikálního výzkumu, pak fyzika v této podobě je ovšem
      produktem teprve posledních desetiletí. Kdybychom chápali fyziku jako vědu, která vychází ze
      systematicky prováděných experimentů a matematicky formulované teorie opět experimentálně
      ověřit, mohli bychom počátek \textbf{klasické fyziky} klást na rozhraní 16. a 17. století.

      Lidé však poznávali přírodu, pozorovali přírodní jevy a snažili si je rozumově vysvětlovat už
      od starověku, a dokonce ještě dříve. Toto nejstarší období bychom mohli označovat jako
      „předvědeckou fyziku", „předhistorii fyziky" nebo období fyzikálního vědění před vznikem
      fyziky jako samostatné experimentální vědy", jak to někteří historikové vědy činí. Taková
      vymezení jsou však příliš šroubovaná a není důvodu považovat úsilí a výsledky starověkých
      badatelů za „nevědecké". Zjednodušíme proto periodizaci dějin fyziky a vyčleníme jen dva
      mezníky - nástup vědecké revoluce symbolickým rokem 1600 a vznik kvantové a relativistické
      fyziky symbolickým rokem 1900. Jak jsme již uvedli. každý takový převrat ve fyzice se
      připravoval během předcházejícího vývoje.

      \luagraphic[1]{fyz_fig0941.jpg}{\wikiAristoteles - Rembrandtova olejomalba: Aristotelés s 
      Homérovou bystou Kredit: Wikipedia}{fyz:fig0941}

      Označíme období od starověku do roku 1600 jako \textbf{stará fyzika}. Tato fyzika se omezovala
      v podstatě jen na mechaniku (především statiku, včetně hydrostatiky) a optiku, která byla pod
      názvem „perspektiva" často vnímána jako součást geometrie. K fyzice musíme přiřadit i
      astronomii, snad nejstarší z věd, která měla ovšem pozorovací charakter, ale přispívala k
      rozvoji matematiky a později i fyziky. Stará fyzika neprováděla systematický experimentální
      výzkum a neměla k dispozici potřebné měřicí přístroje s výjimkou astronomických přístrojů
      úhloměrných a nedokonalých hodin k měření času. Nejpřesnějším měřením, které mohla provádět,
      bylo vážení na rovnoramenných váhách. Přesto ojedinělé experimenty prováděla a již během
      středověku narůstalo přesvědčeni, že přírodu je třeba měřit, vážit a matematicky popisovat.
      Popis fyzikálních jevů byl ovšem jen kvalitativní a byl založen na přírodní filozofii
      antických autorů, především Aristotela. Podobná situace byla ostatně i v lékařství, kde vládla
      autorita Galenova.

      Období tří set let mezi rokem 1600 a 1900 označíme jako období \textbf{klasické fyziky}.
      Fyzika v této době má už plně charakter přírodní vědy, používá desítek nových přístrojů a
      měřicích metod a jako matematický aparát ji vedle geometrie slouží metody matematické analýzy.
      Zabývá se oblasti makroskopických jevů, tedy takových, které se odehrávají v prostorových a
      časových rozměrech naší každodenní zkušenosti. Zásluhou Galilea, Keplera a Newtona byly
      objeveny zákony dynamiky, pozemské i nebeské. Vedle mechaniky a optiky se součásti klasické
      fyziky stává v této době i nauka o elektřině a magnetizmu, termika a později termodynamika a
      statistická fyzika. Začíná se postupně vytvářet i vědecká chemie, která nakonec přispěla k
      objevu Mendělejevova periodického zákona a která s fyzikou úzce souvisí.

      Koncem 19. století se dokonce zdálo, že fyzika už všechny otázky položené přírodě zodpověděla,
      že je v podstatě vědou uzavřenou a zbývá snad už jen upřesnit několik drobných nejasností.
      Nástup kvantové a relativistické fyziky ve 20. století spojený se jmény \textsc{Maxe Plancka}
      a \textsc{Alberta Einsteina} znamenal myšlenkový zvrat, s nimž se fyzikové dlouho
      vypořádávali. Teorie relativity přinesla zásadní zobecnění Newtonovy klasické mechaniky, nové
      pojmy a představy, které jsou zdánlivě v rozporu s běžným tzv. \emph{„zdravým lidským
      rozumem"}. Kvantová fyzika přinesla nové chápání fyzikálních zákonitosti založené na pojmu
      pravděpodobnosti a umožnila matematicky a experimentálně zkoumat svět molekul, atomů a částic,
      který nemůžeme přímo pozorovat našimi smysly a který leží mimo dosah naší zkušenosti.

      \begin{tcnote}  
        Fyzika prošla dlouhým historickým vývojem a znalost tohoto vývoje pomáhá lépe pochopit
        logiku soustavy fyzikálních poznatků a dokonce docházet k poznatkům novým. V krátkosti
        dějiny fyziky můžeme rozdělit na tři hlavní etapy:
        \begin{itemize}[noitemsep]
          \item \textbf{Stará fyzika}: od starověku do počátku 17. století (orientačně do roku
                1600).
          \item \textbf{Klasická fyzika}: 1600 - 1900.
          \item \textbf{Moderní fyzika}: 1900 - dosud.
        \end{itemize}
      \end{tcnote}

      V obou případech však nešlo o znehodnocení výsledků předchozího vývoje fyziky, ale uplatnil se
      tzv. \textbf{princip korespondence}. Tento pojem zavedl původně \textsc{Niels Bohr}, aby
      vysvětlil, jak spolu souvisí kmitočty vyzařování spektrálních čar atomu vodíku z hlediska
      klasické a kvantové fyziky. Relativistická mechanika se zabývá pohybem částic nebo těles s
      rychlostmi blízkými rychlosti světla ve vakuu \(c=\SI{300000}{\km\per\s}\). Klesne-li jejich
      rychlost na hodnoty, s nimiž se setkáváme v běžném životě, přejdou zákony teorie relativity na
      tvar známý z Newtonovy mechaniky a budou korespondovat s fyzikou klasickou. Podobně kvantová
      fyzika obsahuje velmi malou konstantu zvanou Planckova, rovnou \(h =
      \SI{6.626e-34}{\joule\s}\). Pokud můžeme tuto konstantu považovat za zanedbatelné malou,
      jako je tomu ve světě našich makroskopických rozměrů, budou zákony kvantové fyziky
      korespondovat s fyzikou klasickou.   

      
      Pro jednoduchost budeme období od r. 1900 do dneška nazývat \textbf{moderní fyzikou}. Není to
      asi název ideální, stejně tak bychom mohli použit označení \uv{současná fyzika} (kdy začíná
      současnost?), \uv{relativistická a kvantová fyzika}, ale ani to by nebylo úplně výstižné.
      Moderní fyzika má i řadu dalších rysů, studuje např. i jevy nelineární, chaotické a jiné.
      Někdy bývá kvantová fyzika chápána jako protiklad k fyzice klasické a nekvantová
      relativistická fyzika řazena do fyziky klasické. Existuje ovšem i kvantová nerelativistická
      fyzika a kvantová relativistická fyzika a jejich vývoj se časově prolíná. Nebudeme se proto
      pokoušet periodizaci tohoto vývoje komplikovat.

      Moderní fyzika vyústila do dnešní vědeckotechnické revoluce a podstatně změnila život celé
      lidské společnosti. Sama byla zpětně ovlivněna především uplatněním počítačů, ale i nových
      materiálů a měřicích metod neuvěřitelné přesnosti. Velké urychlovače sloužící moderní fyzice
      patří k nejnáročnějším dílům, které kdy lidská technika vytvořila. Možnosti kosmických letů
      ovlivnily charakter astronomie a astrofyziky. Také výchova a vzdělávání fyziků a komunikace
      mezi nimi se změnily a dále mění. V dnešní době pracuje na světě nesrovnatelně více fyziků,
      než kolik jich žilo během celé lidské historie. Ještě před několika desítkami let byl svět
      zaplavován stovkami fyzikálních časopisů, stále více specializovaných, ale dnes už fyzikové
      nemohou všechny práce v nich obsažené sledovat. Nastupují nové komunikační databáze a
      vyhledávače potřebných informací. Tvořivou silou ve fyzice jsou především mladí lidé - není
      náhodou, že i v historii většinu největších objevů učinili fyzikové ve stáří od 20 do 30
      let...

    \subsection{Člověk}
      Mluvíme-li o fyzice jako o procesu poznávání přírody, musíme se také krátce zamyslet nad tím,
      kdo poznává, tedy nad člověkem. Existence inteligentních a civilizovaných bytostí na naší
      planetě patří k nejpodivuhodnějším přírodním úkazům i vzhledem k tomu, že o mimozemských
      civilizacích, ba ani o jiných formách života ve vesmíru zatím stále nic nevíme. Zrození
      člověka předcházel miliardy let trvající vývoj vesmíru a Země, který nakonec vytvořil takové
      podmínky, v nichž člověk může žít.

      Bylo k tomu třeba neuvěřitelné souhry náhod, shody fyzikálních, chemických a biologických
      podmínek. Stačilo by, kdyby planeta Země měla větší nebo menší gravitaci, než má, kdyby byla
      blíž k Slunci nebo dál od něho, neměla aspoň částečně pevný a suchý povrch, postrádala
      ochrannou atmosféru obsahující kyslík a ozón, magnetosféru chránící před nabitými částicemi z
      kosmu, a člověk v dnešní podobě by se na ní nemohl vyvinout. Důležité je, aby teplotní rozmezí
      na povrchu Země umožňovalo přítomnost vody v tekutém stavu. Existence Měsíce vhodné hmotnosti
      a vzdálenosti od Země přispívá ke stabilizaci směru osy zemské rotace a tím i klimatických
      podmínek.

      \luagraphicx[1]{fyz_fig0942.jpg}{Východ Země (anglicky \emph{Earthrise}) je fotografie,
      kterou pořídil 24. prosince 1968 americký astronaut William Anders z paluby kosmické lodi
      Apollo 8, nacházející se na oběžné dráze Měsíce. Snímek zachycuje výřez měsíčního povrchu a
      v pozadí osvětlenou část planety Země (pootočenou oproti severojižní orientaci o 135
      stupňů). Kredit: NASA/Bill Anders}{fyz:fig0924}

      Na oběžné dráze Měsíce při čtvrtém obletu \textbf{Apolla 8}\footnote{první lidé u měsíce} měli
      astronauti možnost poprvé pozorovat něco neskutečného: východ Země nad Měsícem. William Anders
      zachoval duchapřítomnost a událost vyfotografoval – nejprve na černobílý film, vzápětí i na
      barevný. Fotografie se pak stala jedním ze symbolů programu Apollo a kosmonautiky vůbec.
      Snímek se stal ikonickým a jedním ze symbolů 20. století. Jak uvedl William Anders:
      \emph{„Urazili jsme tuto dlouhou cestu, abychom zkoumali Měsíc. Ovšem to nejdůležitější je, že
      jsme nakonec objevili Zemi.“} \cite[s.~69]{Prikryl2019}.

      Protože se planety sluneční soustavy pohybuji po eliptických trajektoriích málo odlišných od
      kruhových, je tím prakticky vyloučena možnost jejich srážek. Pokud jde o srážky Země s malými
      tělesy sluneční soustavy, kdysi hojnými. bylo již okolí Země od těchto těles vyčištěno
      především vlivem velkých planet, zejména Jupiteru. Také v rámci naší Galaxie je sluneční
      soustava, a tedy i naše Země umístěna mimořádně výhodně. Je dostatečně daleko od centra
      Galaxie. takže není ohrožena pronikavým galaktickým zářením, ale zase dosti hluboko uvnitř
      Galaxie, aby nás výbuchy supernov stačily zásobit těžkými prvky potřebnými pro život. Dokonce
      by stačilo, aby molekula vody měla jiné geometrické uspořádání svých atomů nebo atom uhlíku
      jiné chemické vazební vlastnosti, a život, jak ho známe na Zemi, by nemohl existovat.

      Zda byl vznik dnešního člověka za těchto okolnosti nutný a zákonitý, nevíme. Kdyby však k němu
      nedošlo a lidstvo by neexistovalo, nebyli by ani fyzikové zkoumající přírodu a kladoucí
      všetečné otázky. Tento poněkud krkolomný logický závěr se někdy nazývá \emph{„antropický
      princip"}. V každém případě si dnes stále více uvědomujeme, jak křehká je světová ekologická
      rovnováha a jak malé změny teploty, klimatu či složeni atmosféry mohou přivést lidstvo do
      potíží, nehledě na ohrožení pocházející z hlubin Země nebo z kosmu.

      \subsubsection{Evoluce člověka}
        Pomineme-li předchozí geologický vývoj Země a života na ni, můžeme hledat předky člověka
        někde mezi třetihorními savci, kteří vystřídali nadvládu dnes tak oblíbených dinosaurů.
        Příslušníci řádu primátů, vcelku nevzhledných a nenápadných živočichů, se postupně
        rozčlenili do řady čeledi, z nichž lidoopi přežili do dneška a jsou našimi nejbližšími
        příbuznými v živočišné říši. Jedna z těchto čeledi, tzv. \textbf{hominidi}, se však vydala
        na dlouhou cestu vývoje, který dospěl až k dnešnímu člověku.

        Někteří antropologové považuji za první známé zástupce čeledi hominidů příslušníky rodu
        \emph{ramapitéků}, jejichž pozůstatky nalezené v Indii a v Africe jsou staré asi 15 milionů
        let. Novější výzkum využívající metody molekulární biologie (srovnávaní genetické výbavy
        dnešního člověka a afrických opic) však naznačuje, že první hominidi se objevují v Africe
        později, asi před sedmi miliony lety. Z kosterních nálezů známe dnes dva rody čeledi
        hominidů - je to rod \emph{australopiteků}, členící se na mnoho druhů, a rod \emph{Homo},
        člověk.

        Čas od času se dozvídáme o nových nálezech kosterních pozůstatků australopiteku; tyto
        pozůstatky pocházejí z doby před 3-4 miliony let. Australopitekové byli menší než dnešní
        lidé, vyznačovali se základní vlastnosti hominidů, vzpřímenou chůzí po dvou zadních
        končetinách, a používali nástroje, které ovšem sami ještě nevyráběli. Nějakou dobu zřejmě
        existovali souběžně s populací rodu Homo, ale později zanikli bez následovníků, možná pro
        nedostatek přizpůsobivosti změněným přírodním podmínkám. Mnoho o nich nevíme ani o tom, jaké
        byly vzájemné vztahy mezi nimi a příslušníky rodu Homo. Australopitekové jsou tak našimi
        nejbližšími příbuznými a máme s nimi společné, dosud neznámé předky.

        Kolébkou dnešního člověka je Afrika, kde se vytvořily nejvhodnější klimatické a zeměpisné
        podmínky pro vznik jeho rodu. Během vývoje prošel člověk třemi stádii, která bychom mohli
        nazvat \emph{hominizace}, \emph{sapientizace} a \emph{civilizace}. Člověk se postupně
        vyčleňoval z živočišné říše, uvědomoval si sám sebe a stále více vystupoval jako poznávající
        a aktivní subjekt vůči přírodě. První lidé vytvářeli velké rodiny, skupiny kolem třiceti
        jedinců, v nichž docházelo k dělbě činností a zdokonalovala se vzájemná komunikace. Ta
        nakonec dovedla ke vzniku artikulované řeči, lidského jazyka. Vyžádalo si to ovšem některé
        anatomické změny ve stavbě hrtanu, ústní dutiny i mozku. Přitom jazyk a myšlení se podmiňují
        a jejich rozvoj se vzájemně urychluje. Bohužel dosud neznáme odpověď na vzrušující otázku,
        jak spolu komunikovali příslušníci prvních druhů člověka, co si sdělovali a vyprávěli.
        Mluvené slovo nezanechává v historii hmotné stopy.

        Za počátek hominizačního procesu, tedy odlišení člověka od ostatních živočichů, považují
        antropologové napřímení postavy a stabilní \emph{bipedální chůzi}\footnote{(z lat. bi-pes,
        bipedis, dvounohý) znamená pohyb po dvou končetinách.}. Jakmile se člověk ve vlnící se
        africké trávě postavil, mohl se rozhlédnout po krajině, zpozorovat blížící se nebezpečí, ale
        také pozvednout zrak k noční hvězdné obloze a konat první astronomická pozorování. Chůze po
        dvou nohách je ovšem méně bezpečná než po čtyřech a od této chvíle hrozila člověku i možnost
        pádů. Vzpřímená chůze, nový způsob života a nutnost přežití, rostoucí všestrannost při
        výběru potravy vedly postupně k dalším anatomickým změnám. Měnila se kostra lebky a stavba
        chrupu, zvětšoval se objem mozkovny, zdokonalovala stavba nohou a rukou, nyní uvolněných k
        rozmanitým, často jemným mechanickým činnostem. Zároveň s tím postupovali rozvoj rozumových
        schopností a myšlení. Člověk začal používat nástroje a zbraně, zprvu jen předměty nalezené v
        přírodě a mírně upravené, později vyráběné ze dřeva, kostí a kamene podle předchozího
        promyšleného záměru.

        Uvědomíme-li si, že člověk na rozdíl od jiných živočichů není vybaven ani mimořádnou silou,
        schopností rychlého pohybu, ani zvláštní bystrostí smyslů, zdá se téměř neuvěřitelné, že
        mohl při nepočetné populaci přežít miliony let, překonat nástrahy přírody, nepřízeň počasí a
        ledových dob, nemocí a hladu. Existují známky toho, že v některých obdobích poklesla
        početnost lidské populace na světě tak hluboko, že byl vlastně „ohroženým druhem". Člověk se
        dožíval nízkého věku a nevyhnutelnou dětskou úmrtnost musely zřejmě pravěké ženy kompenzovat
        vysokou porodností. Při nedostatku vhodné dětské potravy v nepříznivých ročních obdobích
        musely také zřejmě velmi dlouhou dobu děti kojit. Zvětšování objemu lebky novorozenců vedlo
        vzhledem k omezené průchodnosti porodních cest k tomu, že lidská mláďata se rodila ještě ne
        úplně vyvinutá, byla bezbranná a vyžadovala dlouhodobou péči rodičů i širší rodiny.

        Na druhé straně si pravěký člověk vyvinul mnoho schopnosti a dovedností, které jsme my v
        průběhu civilizačního vývoje už ztratili. Žil v těsném kontaktu s přírodou, znal a využíval
        vlastnosti rostlin a zvířat, s nimiž přicházel do styku. Ovládal zřejmě i podvědomé signály,
        které oznamovaly nebezpečí, aby se mu mohl vyhnout, všímal si v přírodě detailů, které nám
        dnes unikají. Jeho myšlení bylo velmi konkrétní, a tím i blízké uměleckému. Je např. známo,
        že jazyky některých indiánských kmenů mají nesmírně bohatý slovník, pokud jde o označováni
        přírodních jevů - desítky výrazů označuji různý stav oblohy, barvu trávy apod.

        Pokud jde o fyzikální poznání, snadno si představíme, že se pravěký člověk naučil používat
        páku k manipulaci s břemeny, znal chování primitivních plavidel na vodě, využíval vlastnosti
        pohybu těles při šikmém vrhu. Uměl házet oštěpem i harpunou, později střílet z luku a někde
        využívat k usmrcení oběti i rostlinné jedy. Své praktické zkušenosti hromadil a předával po
        tisíce generací. Setkával se ovšem i se záhadnými impozantními přírodními jevy, které byly
        mimo dosah jeho chápání. Sledoval pohyby hvězdné oblohy, krásu tropických hvězdných nocí a
        těžko vědět, co si přitom myslel. Bázeň, ale i zvědavost v něm musel vyvolávat blesk a hrom.
        Vnímal střídání dní a noci, jejich měnící se délku, střídání ročních dob a periodické změny
        v přírodě. To všechno muselo podněcovat jeho fantazii.

        Nejstarší nalezené kosterní pozůstatky a kamenné nástroje příslušníka rodu Homo pocházejí z
        doby asi před dvěma miliony let. Našli je v roce 1960 známi angličtí paleontologové a
        antropologové z rodiny manželů Leakeyovych v Olduvajské rokli v Tanzánii a dali mu název
        \textbf{Homo habilis}, \emph{„člověk zručný“}. Homo habilis zahájil \textbf{paleolit},
        starší dobu kamennou, a vytvořil první lidskou hmotnou kulturu, kterou nazýváme
        \emph{olduvajská}. Jsou pro ni typické hrubé kamenné sekáče opracované jen několika údery
        jiným kamenem 

        Velmi úspěšným následovníkem člověka zručného byl \textbf{Homo erectus}, \emph{„člověk
        vzpřímený"}. Ten se již rozšířil z Afriky po celém Starém světě a jeho pozůstatky nacházíme
        v Indonésii (na Jávě ho objevil E Dubois v r. 1892), v Číně, v Evropě i dalších místech. Pro
        přesnost poznamenejme, že podle posledních, ještě nedostatečně ověřených nálezů se zdá, že
        se Homo erectus „nevyvinul" přímo z Homo habilis, ale že oba lidské druhy žily téměř půl
        milionu let vedle sebe v týchž lokalitách, a musely tedy mít společného, dosud neobjeveného
        předka. Homo erectus byl zdatným lovcem, vynikal fyzickou silou a dovedl rychle běhat. Jeho
        typickým produktem jsou štípané kamenné pěstní klíny: nedávné experimenty ukázaly, jak je
        pro dnešního člověka obtížné takový klín v přírodě vyrobit. Někteří antropologové mají za
        to, že ovládal určitou formu mluvené řeči, byť ještě málo artikulované. Homo erectus přežil
        na Zemi zřejmě jako jediný představitel svého rodu v období asi před 1,5 milionem až 250 000
        lety, a překonal útrapy několika ledových dob.

        Z hlediska dějin fyziky mu přísluší prvenství z nejvýznamnějších - objev rozdělávání a
        využívání ohně. Znamená to vlastně uvolnění části energie vázané v látce podle známého
        Einsteinova vztahu \(E = mc^2\) a význam tohoto objevu je v dějinách srovnatelný pouze s
        uvolněním jaderné energie ve 20. století. Dnes by měl objevitel způsobu rozdělávání ohně
        nesporný nárok na Nobelovu cenu. K objevu došlo někdy před půl milionem let, samozřejmě za
        neznámých okolnosti. Zprvu člověk zřejmě jen udržoval oheň získaný z náhodného přírodního
        požáru. Teprve později našel různé způsoby, jak ho rozdělávat, založené vesměs na zahřívání
        třením, případně křesáním. Prometheovský vynález ohně změnil život člověka, jeho schopnost
        bránit se chladu i šelmám, tepelně upravená strava ovlivnila způsob jeho výživy a mozkovou
        činnost. Z hlediska psychologického je ovšem důležité, že člověk jako jediný živočich našel
        odvahu překonat strach z ohně a manipulovat s ním. Také odvaha riskovat, patří k základním
        předpokladům poznávání.

        Pozdní formy druhu Homo erectus daly vzniknout novému druhu, který označujeme jako
        \textbf{Homo sapiens}. \emph{,,člověk moudrý"}. Proces sapientizace, který dále odlišil
        člověka od ostatních živočichů, znamená v podstatě vznik duševního života, který vytváří
        nehmotnou kulturu - rituály, náboženství, mytologii a na konec umění a vědy. Jeden z mála
        projevů charakterizujících raná stadia Homo sapiens, které mohou archeologové odhalit, je
        pohřbívání mrtvých. Nálezy hrobů a svědectví o pohřbech spojených s nejrůznějšími rituály
        svědčí o tom, že Homo sapiens si opět jako jediný ze živočichů začal uvědomovat konečnost
        svého pozemského bytí a začal se zamýšlet nad posmrtným životem, vztahem mezi lidským tělem
        a duší. Je pravděpodobné, že pohřební rituály byly doprovázeny i hudebními projevy a
        rytmickým tancem a zpěvem.

        Člověk začal ve své fantazii zabydlovat přírodu nadpřirozenými bytostmi, dobrými a zlými
        duchy, zlidšťovat si přírodní jevy. Vzniká \emph{šamanismus}, představa o tom, že zlé duchy
        způsobující nemoci a neštěstí je možno zažehnat a dobré duchy si naklonit. K nejstarším
        rituálům patří obřady mající zajistit úspěch při lovu, později za mladší doby kamenné
        úrodnost polí. Důležitou úlohu hrály i kulty mateřství a plodnosti jako záruky pokračování
        lidského rodu a s tím související ženské sošky a artefakty známé jako pravěké venuše.

        Homo sapiens vytvořil \emph{dvě větve} - \textbf{člověka neandertálského} a člověka našeho
        poddruhu, který je též nazýván \textbf{člověkem kromaňonským} nebo člověkem předvěkým.
        Nevíme přesně kdy a jak se to stalo, geneticky však bylo ustanoveno, že člověk předvěký není
        potomkem neandertálců. V Evropě obývali neandertálci nehostinné chladné kraje např. oblast
        Alp, prokázali značnou fyzickou zdatnost a odolnost, používali oheň, odívali se do zvířecích
        kůží, pohřbívali své mrtvé, vyznávali zvířecí kulty, především medvěda, a dorozumívali se
        již artikulovanou řečí. Nějakou dobu žili v Evropě souběžně s člověkem předvěkým a asi před
        35 000 lety vyhynuli bez následovníků. 

        Tím se dostáváme k našemu vlastnímu poddruhu, který se již od nás anatomicky, tělesnými ani
        duševními schopnostmi ničím neliší. Říkáme mu možná s přehnanou hrdosti \textbf{Homo sapiens
        sapiens}, \emph{„člověk moudrý, moudrý“}. Tento náš přímý předek se objevil v Evropě na
        sklonku starší doby kamenné, asi před 40 000 lety, v podstatě nevíme odkud. V té době se
        pomalu schylovala ke konci \emph{poslední doba ledová}, Evropou putovala stáda mamutů, a tak
        máme tyto lidi v povědomí jako „lovce mamutů". Četná naleziště ohnišť, hrobů, jemně
        opracovaných kamenných nástrojů, ozdob a sošek (některé z nich z hlíny vypalované v peci
        nesou ještě otisky prstů svého tvůrce) svědčí o vyspělé kultuře. 

        Největší překvapení však přinesly objevy jeskynních obrazáren s nástěnnými malbami zvířat v
        pohybu a lovců, které jsou postupně nacházeny především v předhůří Pyrenejí. Nejstaršími z
        těchto objevů jsou španělská jeskyně Altamira známá od r. 1879 (stáří 13 000 let), na
        francouzské straně jeskyně Lascaux (objevena 1940, stáři 15 000 let), v poslední době
        jeskyně Chauvet v údolí řeky Ardèche (1994, stáří 20 000 let) a rada dalších. Podobné
        obrazárny a skalní malby přibližně z téhož období najdeme i v jiných částech světa,
        především v Africe.

        Nástěnné malby zobrazují zvířata, bizony, pratury, divoké koně, jeleny, soby, nosorožce, lvy
        a medvědy, některá z nich už v Evropě dávno vyhynulá. Estetická hodnota obrazů je plné
        srovnatelná s uměleckými výkony dnešního člověka a svědčí o plně vyvinutém duševním životě.
        Musí se nás zmocnit vzrušení, zamyslíme-li se nad tím, co pohánělo pravěkého umělce, aby
        vstupoval do temného prostoru a za blikavého světla olejových lampiček vytvářel svá umělecká
        díla. Po desítky tisíc let se v jeskyních uchovalo i nářadí umělců, zbytky barev, lampičky,
        otisky malířových rukou a stopy bosých nohou včetně dětských. Účel maleb mohl souviset s
        loveckými kulty, ale mohl mít také jiný, hlubší, nám neznámý symbolický smysl. Zároveň ale
        jsou projevem talentu, touhy po tvůrčí seberealizaci, která je vlastní člověku stejně jako
        schopnost údivu, zvídavost, fantazie a odvaha, Patří i k předpokladům vzniku vědy.

        Konečně třetím stupněm, který přiblížil vznik naši dnešní společnosti, byl \emph{proces
        civilizace}: jeho název souvisí s latinským \emph{„civitas"}, obec, a poukazuje na vznik
        strukturované společenské organizace, prvních státních celků. Odehrál se v mladší době
        kamenné, neolitu, a na počátku doby bronzové, a to poměrně nedávno, v prvních tisíciletích
        před našim letopočtem. Souvisí s přechodem k novým způsobům obživy lidi, k pastevectví a
        zemědělství. Přinesl významné zásahy člověka do přírody, vypásání stepi, vypalováni lesů,
        zavlažováni pudy a počátky znehodnocování životního prostředí. Někdy mluvíme o
        \emph{neolitické revoluci}. Člověk začal zakládat města, rozvíjet obchod a dopravu, začal se
        cítit pánem přírody. Dostatek potravy umožnil rychlý růst lidské populace. Odhaduje se, že
        kolem r. 15 000 př. n. l. žily na světě asi 3 miliony lidi, r. 2 000 př. n. l. už asi 50
        milionů a na přelomu letopočtu kolem 250 milionů obyvatel.

%---------------------------------------------------------------------------------------------------    
  \section{Stará fyzika}
    Starou fyziku nemůžeme považovat za vědu ve vlastním smyslu, i když se dobrala celé řady
    významných vědeckých poznatku. První z nich znali již staří Sumerové, Babyloňané, Egypťané a
    Číňané. Šlo zejména o  poznatky astronomické a geometrické (Pythagorova veta) a také o metody
    měření některých fyzikálních veličin (délka, hmotnost, čas). Fyzika ve starém Řecku byla jako
    součást filosofie převážně spekulativní a tento charakter si pod vlivem aristotelismu udržela,
    až do počátku novověku. Skutečný fyzikální výzkum prováděli až helenističtí Řekové, kdy se
    centrem vědy a kultury antického světa stala Alexandrie. 
    
    \begin{figure}[ht!]  % \ref{fyz:fig0894}
      \centering
      \luafigure[1]{fyz_fig0894.jpg}
      \caption{ \wikiAlexLib byla největší a nejslavnější knihovna starověku. Byla součástí
                věhlasného múseia v Alexandrii, vybudovaného z podnětu Ptolemaia I. Byla považována
                za hlavní centrum vzdělanosti od 3. století př. n. l. až do roku 48 př. n. l., kdy
                za války mezi Caesarem a Pompeiem zčásti vyhořela. Starověké zdroje pojednávají o
                ničení knihovny, o tom, kdo je zodpovědný za ničení a kdy k němu došlo, se liší.
                \cite[s.~76]{Stoll2009}}
      \label{fyz:fig0894}
    \end{figure} 

    V Alexandrii studoval největší fyzik starověku \textsc{Archimédes}, který dospěl k důležitým
    poznatkům o statické rovnováze těles a plování těles a v matematice se těsně přiblížil objevu
    diferenciálního a integrálního počtu. Alexandrijští Řekové znali také zákon odrazu světla
    (nikoli lomu) a prováděli první měření teploty. Poznatky antiky byly středověké Evropě
    zprostředkovány Araby, kteří se též intenzivně zabývali optikou (\textsc{Alhazen}) a určováním
    měrné hmotnosti látek. Zatímco ve středověku byly hlavní přírodovědné poznatky čerpány z
    Euklidových ”Základu” (geometrie), ”Almagestu” Klaudia Ptolemaia (geocentrický výklad astronomie
    sluneční soustavy) a spisu Aristotelových (mj.”Fysika”), vešly práce Archimédovy v Evropě ve
    známost až teprve začátkem novověku. Ve starověku a středověku však fyzika neprováděla
    systematické experimenty, nevyužívala matematický aparát k popisu přírodních jevu a neměla ani
    přesně definovány základní pojmy (rychlost, zrychlení, síla apod.) Zrod fyziky jako vědy se
    datuje začátkem 17. století. Na základě astronomických výzkumu \textsc{Keplerových} (1571-1630)
    a pozemských mechanických experimentů\textsc{ Galileových} (1564-1642) mohl Isaac Newton
    (1643-1727) vytvořit první fyzikální teorii, klasickou mechaniku, využívající matematický aparát
    diferenciálního a integrálního poctu. Newton přišel s koncepcí všeobecné gravitace a ukázal, že
    není přehrady mezi nebeskou a pozemskou fyzikou, že síla, která udržuje planety na jejich
    drahách kolem Slunce je táž jako síla, která nutí jablko padat k zemi. Základní Newtonovo dílo z
    r. l687 nese název ”Matematické základy přírodní filosofie” (”Philosophiae naturalis principia
    mathematica”) a představuje pravděpodobně nejvýznamnější vědeckou knihu, která byla kdy napsána.
    Newton se zabýval též optikou a rozpracoval teorii rozkladu bílého světla do spektra. V té době
    byl již zásluhou Snellovou a Descartovou znám i zákon lomu světla. Z roku 1600 pochází první
    vědecký spis o elektřině a magnetismu od anglického lékaře a fyzika Gilberta. Výzkumem  těchto
    jevu se v následujících stoletích zabývala celá řada fyziků (Coulomb, Volta, Oersted, Amp\`{e}re
    a další). Tento výzkum pak završil \textsc{Faraday} (1791-1867) svým objevem zákona
    elektromagnetické indukce a svou koncepcí siločár elektromagnetického pole. Úlohu Newtona
    elektromagnetismu pak sehrál \textsc{James Clerk Maxwell} (1831-1879), který ve svém ”Traktátě o
    elektřině a magnetismu” z r. 1873 sestavil slavné Maxwellovy rovnice popisující vlastnosti
    elektromagnetického pole. Maxwell zároveň teoreticky zdůvodnil elektromagnetickou povahu světla
    a ukázal, že jevy spojené s vlastnostmi elektrického náboje (”elektřina”), elektrického proudu
    (”galvanismus”), magnetického pole a světla (optika), jsou jedné a téže elektromagnetické
    povahy. V devatenáctém století byl tak dovršen výzkum mechanických jevů a elektromagnetismu a
    klasická fyzika tím završena. V přírodě tedy existovaly pouze dvě síly, dva způsoby vzájemné
    interakce mezi částicemi: gravitační a elektromagnetická. Mezi nimi se však projevoval určitý
    rozpor. Jak Newtonovy tak Maxwellovy rovnice platí v libovolné inerciální vztažné soustavě. Při
    přechodu od jedné inerciální soustavy k druhé se však Newtonovy rovnice transformují pomocí tzv.
    Galileiho transformací a Maxwellovy rovnice pomocí Lorentzových transformací. Fyzika se tak
    rozdvojila, mechanické a elektromagnetické děje se zdály být neslučitelné. Kromě toho existovaly
    některé experimenty, jejichž výsledek nedokázala klasická fyzika vysvětlit: průběh spektra
    rovnovážného elektromagnetického záření (tzv. záření absolutně černého tělesa) a pokus
    Michelsonův, který svědčil o neexistenci světelného éteru. Tyto zdánlivě nepodstatné rozpory
    vyústily ve 20. století ve vznik moderní fyziky, tj. fyziky kvantové a relativistické. Právě
    koncem roku 1900 vyslovil Planck tzv. kvantovou hypotézu, jíž vysvětlil záření absolutně černého
    tělesa, a v r. 1905 publikoval Einstein práci o speciální teorii relativity. V ní překlenul
    rozpor mezi Newtonovou a Maxwellovou fyzikou a fyziku opět sjednotil. Předpoklad o existenci
    světelného éteru se teorií relativity stal zbytečným. V roce 1916 vytvořil Einstein i obecnou
    teorii relativity jako moderní teorii gravitace. Gravitační síly podle této teorie souvisejí se
    zakřivením prostoročasu. Jak speciální, tak obecná teorie relativity přecházejí při rychlostech
    objektu podstatně menších než je rychlost světla ve vakuu a při slabých gravitačních polích v
    teorii Newtonovu. Přelom 19. a 20. století je též poznamenán objevem radioaktivity a vznikem
    jaderné fyziky, která tak významným způsobem zasáhla do života celého lidstva. V jaderné fyzice
    se uplatní další dvě přírodní síly - tzv. silná, která udržuje nukleony v atomových jádrech a
    slabá, která se projevuje při radioaktivní přeměně beta za vzniku neutrin. Moderní fyzika
    odhalila v kosmickém záření a pomocí urychlovačů obrovské množství částic, jejichž vlastnosti
    studuje a snaží se je utřídit a vysvětlit. Mezi všemi těmito částicemi působí čtyři základní
    síly přírody: gravitační, elektromagnetická, silná a slabá. V nedávné době se podařilo prokázat,
    že i elektromagnetická a slabá interakce jsou téže podstaty a tvoří jedinou sílu elektroslabou.
    V průběhu historie fyziky od Newtona a Maxwella k dnešku tak probíhá úsilí o sjednocování
    interakcí, které pokračuje i dnes. Fyzika se pokouší prokázat, že i silná a elektroslabá
    interakce jsou téže povahy, a že k nim konečně přistupuje i síla gravitační. Tím by vznikla idea
    jediné přírodní síly sjednocující všechny přírodní jevy a děje. Fyzika ovšem nemůže k takovému
    závěru dojít pouhým uvažováním, musí matematicky vypracovat a zdůvodnit příslušnou teorii a její
    závěry experimentálně ověřit. To vede ke snaze budovat stále větší a větší urychlovače a také k
    intenzivnímu výzkumu jevů v kosmu. Sjednocování interakcí má totiž těsnou návaznost na vývoj
    vesmíru podle hypotézy o tzv. ”velkém třesku”. Právě v počátcích vývoje vesmíru by se měly
    všechny čtyři (resp. tři) interakce uplatňovat rovnocenným způsobem a teprve v průběhu dalšího
    vývoje a rozpínání vesmíru se postupně oddělovat. Tak jako počátky vzniku vědecké fyziky v 17.
    století jsou spjaty s astronomickými pozorováními sluneční soustavy, je i dnes fyzika stále více
    propojena s astrofyzikou. Vesmír zůstává největší fyzikální laboratoří.

%--------------------------------------------------------------------------------------------------- 
  \section{Klasická fyzika}\label{fyz:IchapIIsecIV}
    Je dost těžké začít hned se současnými představami, a proto se podívejme, jak se jevil svět v
    roce 1920 a potom na tomto obrázku něco změníme. Naše představa světa byla před rokem
    \textbf{1920} následující: „Scénou“, na které vystupuje vesmír, je \emph{trojrozměrný
    geometrický prostor} popsaný ještě Eukleidem a věci se mění v prostředí, které nazýváme časem.
    Prvky vystupující na scéně jsou \emph{částice}, například atomy, které mají určité vlastnosti.
    Především vlastnost setrvačnosti: pohybuje-li se částice, zachová si pohyb v původním směru,
    pokud na ni nepůsobí \emph{síly}. Druhým prvkem jsou tedy síly, o nichž se tehdy  
    předpokládalo, že jsou dvojího druhu. K prvnímu, velmi složitému druhu, patřila síla vzájemného
    působení, která udržovala atomy v jejich různých kombinacích komplikovaným způsobem a byla
    zodpovědná za to, jestli se sůl při zvyšování teploty rozpouští rychleji nebo pomaleji. Druhou
    známou silou byla interakce dalekého dosahu - hladké a klidné přitahování. Tato síla, měnící se
    nepřímo úměrně čtverci vzdálenosti, byla nazvána \emph{gravitací}. Její zákon byl známý a byl
    velmi jednoduchý. Proč věci zůstávají v pohybu, když se už začaly pohybovat, nebo proč existuje
    gravitační zákon, bylo, samozřejmě, neznámé.
    
    Zabýváme se popisem přírody. Z tohoto hlediska je plyn a právě tak všechna hmota myriádou
    pohybujících se částic. Takto se dostávají do souvislosti mnohé věci, které jsme viděli na
    mořském břehu. \emph{Tlak} pochází od \emph{srážek atomů} se stěnami nebo s čímkoliv jiným;
    atomy pohybující se převážně jedním směrem vytvářejí vítr; \emph{chaotické vnitřní pohyby}
    představují \emph{teplo}. Známe vlny zvýšené hustoty, kde se shromáždilo příliš mnoho částic,
    které při rozletu stlačují další shluky částic a pohyb se tak předává dál. Tyto vlny vyšší
    hustoty představují \emph{zvuk}. Pochopení tolika věcí je možno považovat za úžasný úspěch. O
    některých z těchto věcí jsme hovořili v předcházející kapitole.
    
    Jaké druhy částic existují? Tehdy předpokládali, že je jich 92. Nakonec bylo objeveno 92 různých
    druhů atomů. Měly různá jména podle svých chemických vlastností.
    
    Byl tu ještě problém \emph{povahy sil krátkého dosahu}. Proč uhlík přitahuje jeden kyslík,
    případně dva, ale ne víc? Jaký je mechanizmus vzájemného působení mezi atomy? Je to gravitace?
    Na tuto otázku musíme odpovědět záporně, protože gravitace je na to příliš slabá. Představme si
    však sílu podobnou gravitaci, měnící se nepřímo úměrně čtverci vzdálenosti, ale mnohem silnější
    a odlišnou ještě v jednom směru. V případě \emph{gravitace jde vždy o přitahování}. Představme
    si však, že existují dva druhy „věcí“ a tato nová síla  (samozřejmě elektrické povahy) má tu
    vlastnost, že věci stejného druhu se odpuzují a věci různého druhu se přitahují. „Předmět“, jenž
    je nositelem tohoto silného vzájemného působení, se nazývá \emph{náboj}.  
    
    K čemu jsme došli? Předpokládejme, že máme dvě věci různého druhu, jež se vzájemně  
    přitahují (plus a minus) a které drží těsně u sebe. Předpokládejme, že v určité vzdálenosti od
    uvedené dvojice máme další náboj. Bude tento náboj pociťovat přitažlivost? Mají-li první dva
    náboje stejnou velikost, neměl by pocítit \emph{prakticky žádnou přitažlivost}, protože
    přitahování jedním nábojem a odpuzování druhým nábojem se vykompenzují. Ve velkých vzdálenostech
    je tedy síla velmi malá. Když třetí náboj \emph{hodně přiblížíme} k prvním dvěma, objeví se
    přitahování, protože odpuzování stejných nábojů a přitahování různých se snaží oddálit stejné
    náboje a přiblížit různé. Odpuzování bude nakonec \emph{slabší} než přitahování. To je příčina,
    proč atomy, které se skládají z kladných a záporných elektrických nábojů, na sebe téměř nepůsobí
    (zanedbáme-li gravitaci), jsou-li od sebe dost vzdáleny. Když se ale přiblíží, mohou
    „\emph{vidět jeden do druhého}“, přeskupit své náboje a velmi silně vzájemně působit. Podstatou
    interakce mezi atomy je \emph{elektrické} působení. Tato síla je tak veliká, že všechny plusy a
    minusy se obvykle dostávají do tak těsné kombinace, jak je to jen možné. Všechny věci, včetně
    nás samotných, se skládají z drobných, velmi silně interagujících kladných a záporných částic,
    které jsou velmi přesně vyvážené. Na okamžik je možné náhodou odstranit několik minusů nebo
    plusů (obvykle je jednodušší odstranit minusy), v tu chvíli jsou elektrické síly
    \emph{nevyvážené} a můžeme pozorovat působení elektrické přitažlivosti.
    
    Abychom si vytvořili představu o tom, o kolik je elektrické působení silnější než gravitace,
    představme si dvě zrnka písku, která mají jeden milimetr v průměru a jsou vzdálená třicet metrů.
    Kdyby elektrické síly mezi nimi nebyly vyvážené, kdyby nebylo odpuzování a vše se navzájem
    přitahovalo a nic se nekompenzovalo, jakou silou by se zrnka přitahovala? Byla by to síla tří
    miliónů tun. Jistě chápete, že pro vytvoření značného elektrického působení stačí velmi malý
    přebytek nebo nedostatek záporných nebo kladných nábojů. Proto není vidět rozdíl mezi elektricky
    nabitým a nenabitým předmětem - pro nabití předmětu je třeba tak málo částic, že se téměř
    neprojeví na jeho hmotnosti, či rozměru.
    
    S těmito poznatky bylo jednodušší pochopit atomy. Předpokládalo se, že mají uprostřed
    „\emph{jádro}“, které je kladně elektricky nabité a velmi těžké, a toto jádro je obklopeno
    určitým počtem „elektronů“, jež jsou velmi lehké a záporně nabité. Teď trochu pokročíme v našem
    výkladu a poznamenáme, že v samotných jádrech byly objeveny dva druhy částic - \emph{protony} a
    \emph{neutrony}, které mají téměř stejnou, velmi velkou hmotnost. Protony jsou elektricky nabité
    a neutrony jsou neutrální. Máme-li atom se šesti protony v jádře, které je obklopeno šesti
    elektrony (záporné částice obyčejného světa jsou všechno elektrony a ty jsou velmi lehké v
    porovnání s protony a neutrony, které tvoří jádra), půjde o atom číslo šest v chemické tabulce a
    tento atom se nazývá uhlík. Atom číslo osm se nazývá kyslík atd. Chemické     
    vlastnosti závisí na vnějších elektronech, ve skutečnosti jen na tom, kolik má atom elektronů.
    \emph{Chemické vlastnosti} látek tedy závisí na jediném čísle, na \emph{počtu elektronů}.
    (Seznam prvků sestavený chemiky by se mohl nahradit očíslováním 1, 2, 3, 4, 5 atd. Místo toho,
    abychom říkali „uhlík“, stačilo by říci „prvek číslo šest“, což by znamenalo, že prvek má šest
    elektronů. Při objevování prvků však tato skutečnost nebyla známa a dále, při číslování by vše
    vypadalo velmi složitě. Proto je lepší ponechat prvkům názvy i symboly a nedožadovat se pouhého
    očíslování.)

    \begin{figure*}[ht!] %\ref{fyz:fig0006}
      \centering
      \luafigure[1]{fyz_fig0006.pdf}
      \caption{Elektromagnetické spektrum (někdy zvané Maxwellova duha) zahrnuje elektromagnetické
              záření všech možných vlnových délek. Srovnání délek elektromagnetických vln s běžnými
              předměty a odpovídající teplotní stupnice umožňuje lépe získat představu o jejich
              rozměrech a energiích.}
      \label{fyz:fig0006}
    \end{figure*}
    
    O elektrické síle bylo získáno mnoho dalších poznatků. Bylo by přirozené předpokládat, že
    elektrická interakce je jednoduché přitahování dvou předmětů: kladného a záporného. Zjistilo se
    však, že toto není úplně vhodná představa. Situaci lépe vystihuje představa, že existence
    kladného náboje v prostoru způsobuje jeho jisté \emph{zakřivení}, vytváří v něm určitou
    „podmínku“, aby záporný náboj vložený do tohoto prostoru cítil působení síly. Tato možnost
    vzniku síly se nazývá \emph{elektrické pole}. Dostane-li se elektron do elektrického pole, je
    jakoby „tažen“. Přitom platí dvě pravidla: a) \emph{náboje vytvářejí pole}, b) \emph{v poli
    působí na náboje síly a náboje se pohybují}. Příčina takového chování se stane jasnější, jakmile
    rozebereme následující jev: Nabijeme-li těleso elektricky, například hřeben, a do určité
    vzdálenosti položíme nabitý ústřižek papíru, přičemž začneme hřebenem pohybovat sem a tam, bude
    se papír natáčet směrem k hřebenu. Zrychlíme-li pohyb hřebenu, zjistíme, že papír zaostává,
    působení se opožďuje. (V prvním stádiu, když pohybujeme hřebenem poměrně pomalu, zkomplikuje nám
    situaci \emph{magnetizmus}. Magnetické vlivy se projevují, když jsou \emph{náboje v relativním
    pohybu}, takže magnetické a elektrické síly je možné skutečně připsat jedinému poli jako dvě
    stránky jedné věci. Měnící se elektrické pole nemůže existovat bez magnetizmu.) Oddálíme-li
    nabitý papír, zpoždění je větší. V tu chvíli pozorujeme zajímavou věc. Ačkoliv se síly působící
    mezi dvěma nabitými předměty mění nepřímo úměrně čtverci vzdálenosti, při kmitání náboje
    zjišťujeme, že jeho působení se rozprostírá mnohem dále, než by se dalo očekávat. Pokles tohoto
    působení je mnohem pomalejší než při nepřímé úměrnosti čtverci vzdálenosti.
    
    S analogickou situací se setkáváme, když na vodě plave splávek a my ho uvedeme do pohybu „přímo
    “ tím, že způsobíme pohyb vody jiným splávkem. Kdybychom se dívali jen na dva splávky,
    pozorovali bychom pouze to, že jeden se dává do pohybu jako odezva na pohyb druhého, že mezi
    nimi existuje určitá „  interakce“. Ve skutečnosti jsme ale rozčeřili vodu a voda posunula druhý
    splávek. Mohli bychom zformulovat „zákon“, že i při slabém zčeření vody se na vodě budou
    pohybovat předměty nacházející se blízko zdroje zčeření. Kdyby byl druhý splávek dost daleko,
    sotva by se dal do pohybu, neboť jsme uvedli vodu do pohybu jen v jednom místě. Bude-li však
    druhý splávek pravidelně kmitat, vznikne nový úkaz, při kterém se pohyb vody přenáší dál, vzniká
    \emph{vlnění} a vliv poskakujícího splávku již nemůžeme chápat jako přímé působení mezi splávky.
    Myšlenku přímé interakce tedy musíme nahradit předpokladem o existenci vody nebo v případě
    elektrických nábojů tím, co nazýváme \emph{elektromagnetickým polem}.
    
    Elektromagnetické pole může přenášet vlny. Některé z těchto vln jsou světlo jak je znázorněno na
    obrázku \ref{fyz:fig0006}, jiné se používají při rádiovém vysílání, ale obecně se nazývají
    \emph{elektromagnetickými vlnami}. Tyto vlny mohou mít rozmanité \emph{frekvence}. Jediné, čím
    se jedna vlna liší od druhé, je právě frekvence vlnění. Kdybychom pohybovali nábojem sem a tam a
    dělali bychom to stále rychleji a rychleji, objevovala by se celá řada různých jevů, které je
    možné systematizovat udáním čísla vyjadřujícího počet kmitů za sekundu. Frekvence, s nimiž
    přicházíme do styku prostřednictvím běžných rozvodových elektrických sítí v domech, jsou řádově
    sto kmitů za sekundu. Zvýšíme-li frekvenci na \SI{500}{\kHz} nebo \SI{1000}{\kHz} (\SI{1}{\kHz}
    = 1000 kmitů za sekundu), dostáváme se z domů ven, „na vzduch“, neboť máme co činit s
    frekvencemi používanými při rozhlasovém vysílání. (Se vzduchem to ale nemá co dělat! Rádiové
    vlny se mohou šířit i v prostoru, v němž není vzduch.) Zvyšujeme-li frekvenci, dostáváme se do
    oblasti \emph{VKV} a televizního vysílání. Při ještě vyšších frekvencích máme velmi krátké vlny,
    které se využívají např. v \emph{radiolokaci}. Kdybychom šli ještě výše, nepotřebovali bychom už
    zařízení na registraci takových vln, protože bychom je viděli naším zrakem. Kdybychom dokázali
    pohybovat nabitým hřebenem tak rychle, aby kmital s frekvencemi od \SI{5e14}{\Hz} do
    \SI{5e15}{\Hz}, viděli bychom toto kmitání jako červené, modré nebo fialové světlo v závislosti
    na frekvenci. Frekvence pod touto oblastí nazýváme \emph{infračervenými} a nad touto oblastí
    \emph{ultrafialovými}. Skutečnost, že naše vidění je omezeno na určitou frekvenční oblast,
    nedělá tuto oblast elektromagnetického spektra z fyzikálního hlediska důležitější než jiné
    oblasti, avšak z lidského hlediska je tato oblast přece jen zajímavější. Kdybychom frekvenci
    ještě zvýšili, dostali bychom \emph{rentgenové paprsky}. Tyto paprsky nejsou nic jiného, než
    světlo s velmi vysokou frekvencí. Ještě vyšším frekvencím odpovídá \emph{záření gama}. Výrazy
    rentgenové paprsky a záření gama jsou téměř synonyma. Zářením gama nazýváme obvykle
    elektromagnetické vlny pocházející z jader a rentgenovými paprsky vlny pocházející z atomů; při
    shodě jejich frekvencí jsou však fyzikálně nerozlišitelné, bez zřetele na jejich původ. Vlny
    ještě vyšších frekvencí, řekněme \SI{10e24}{\Hz}, lze získat uměle, například na
    \emph{synchrotronu} v Caltechu. Elektromagnetické vlny úžasně vysokých frekvencí (až tisíckrát
    vyšších) je možné najít ve vlnách \emph{kosmického záření}. Tyto vlny však neumíme ovládat. 
    \cite[s.~29]{Feynman02}

    \subsection{Vědecká revoluce 17. století}\label{fyz:IchapIIsecVII}
      \textbf{Klasická fyzika}, jak ji popsal Richard Feynnman v předchozích kapitolách, tedy jako
      věda vycházející z měření a experimentů a opírající se o matematickou teorii, věda, která nám
      podává ucelený obraz přírody a světa a svými výsledky slouží technickému pokroku, vznikla v
      Evropě v průběhu sedmnáctého století. Tento dějinný převrat, který předznamenal naši dnešní
      civilizaci, nazýváme \textbf{obdobím vědecké revoluce}. Nebyla to ovšem nějaká náhlá událost a
      lidé si tehdy ani neuvědomili, jakou vlastně prožívají dobu a co přinese budoucím pokolením.
      Vědecká revoluce nastala za zvláštních podmínek evropského vývoje, které se v jiných částech
      světa nevytvořily.

      Příčin, které vyvolaly tuto vědeckou revoluci, bylo mnoho a nemůžeme je zde podrobně zkoumat.
      Především to byly nové politické a hospodářské podmínky, potřeby výroby, obchodu a podnikání,
      které vyzvedly do popředí nové společenské síly, především měšťanské. Vzrůstající produktivita
      práce a vznik prvních kolektivních dílen, manufaktur, potřebovaly nové způsoby silového
      pohonu. Zásobování surovinami a vývoz hotových výrobků si vyžádal rozvoj mořeplavby a námořní
      navigace. Evropské války, jak už to bývá, také podnítily zdokonalování vojenské techniky a
      nepřímo i rozvoj přírodních věd \cite[s.~137]{Stoll2009}.

      Důležitou úlohu sehrála reformace, odklon řady zemí v západní a severní Evropě od katolické
      církve a papežství a vznik nových, protestantských církví. Protestantismus usiloval o bližší
      kontakt jednotlivého člověka s Bohem, bez prostřednictví církevní hierarchie, o návrat k
      podobě bible v jejich původních jazycích (hebrejském a řeckém) a podnítil vznik překladů
      biblických textů do národních evropských jazyků. Tím na jedné straně vyvolal potřebu studia
      klasických jazyků a umožnil také zpřístupnění výsledků vědy starověkého Řecka a na druhé
      straně podpořil rozvoj národních jazyků (připomeňme si jen krásnou češtinu naší Kralické
      bible). Latina, ve středověku univerzální jazyk vzdělanců, začala ztrácet své výsadní
      postavení.

      S rostoucím vědomím užitečnosti a nutnosti vědeckého poznání bez vměšování teologického
      dogmatismu přenášejí protestanti těžiště náboženského cítění do oblasti morální, jako vodítko
      při hledání smyslu lidského života, a ponechávají přírodním vědám zkoumání a využívání
      přírodních zákonů. Odmítají víru v Boží zázraky, která vlastně znemožňuje existenci vědy.
      Takový přístup, kdy Bůh je chápán jen jako stvořitel a první zákonodárce, který se však do
      dalšího chodu přírody už nevměšuje, nazýváme \textbf{deismus}, na rozdíl od katolického
      teismu, podle něhož Bůh do běhu světa stále zasahuje a bez jehož vůle, ani vlas z hlavy
      nespadne“. Protože nositeli idejí protestantismu byly především měšťanské a hospodářsky
      aktivní vrstvy společností, rozvíjí se věda a vědecká revoluce zejména v protestantských
      zemích západní Evropy v Holandsku, Anglii, Švýcarsku, Dánsku, částečně ve Francii a Německu.
      Také příznačný podnikatelský duch Ameriky má své kořeny v anglosaském protestantismu prvních
      přistěhovalců. Katolická Itálie, která renesanci vědy zahájila, nakonec odsoudila svého
      Galilea, katolické Španělsko a Portugalsko, které zbohatly při zámořské kolonizaci, postupně
      svou moc ztrácejí a k vědecké revoluci v Evropě nepřispívají.

      Evropa nebyla nikdy soběstačná v některých druzích výrobků, ať už šlo o tropické plody,
      rostliny (bavlna, cukrová třtina) a koření, drahé kovy, ale třeba i hedvábí, vzácné kožešiny a
      jiné výnosné luxusní předměty. Obchodní cesty k jejich získávání vedly odedávna přes
      Středozemí, Blízký a Střední Východ a na tomto obchodu bohatly zejména italské městské státy
      jako Benátky nebo Janov. Když postupující turecká expanze tyto přístupové cesty znesnadnila a
      ohrozila, hledaly státy západní a jihozápadní Evropy přístup na východní trhy obeplutím Afriky
      a po úspěšných výpravach Kolumbovych západním směrem do Ameriky.

      Španělsko a Portugalsko začaly z těchto nových cest a výbojů těžit jako první, jejich karavely
      a galeony, obtížené kořením, stříbrem a zlatem, přivážely toto zboží na evropské trhy, pokud
      neskončilo na mořském dně nebo v rukou pirátů. Obě tyto námořní mocnosti si známými smlouvami
      z Tordesillas (1494) a Zaragozy (1529) dokonce rozdělily celý svět na dvě poloviny a
      uskutečnily tak první globalizaci světového obchodu a kolonizace ve znamení katolicizmu.

      Nedokázaly však své nové hospodářské zdroje produktivně využít. Jejich pozice zaujala postupně
      Anglie, Francie, a zejména malé protestantské Holandsko, které se začátkem 17. století
      osvobodilo od španělské nadvlády a vytvořilo republiku pod vládou místodržitelů z rodu
      Oranžskeho je téměř neuvěřitelné, že Holandsko, počtem obyvatel srovnatelné s tehdejším českým
      královstvím, vytvořilo jeden čas největší koloniální říši světa a disponovalo flotilou 16 000
      lodi, počtem trojnásobně převyšujícím flotilu všech ostatních západoevropských států
      dohromady. Hospodářsky se postupně vzmáhala i Anglie, kde společenské napětí vyvrcholilo
      občanskou válkou a revoluci, která přivedla v roce 1649 krále Karla I. na popraviště. Všechny
      tyto společenské otřesy a změny v západní Evropě postupně vytvářely nové impulzy k rychlému
      vědeckému a technickému pokroku.

      Koloniální výboje vyvolaly potřebu mapovat nová území, dokonce mapovat zeměkouli jako celek,
      především přesně měřit zeměpisnou šířku a délku, ale i hloubku moří, teplotu a slanost mořské
      vody, rychlost a směr mořských proudů a magnetickou deklinaci, odchylku směru udávaného
      kompasem od pravého severu. To ovšem vyžadovalo prozkoumat přesný geometrický tvar zeměkoule a
      vytvořit nové fyzikální a astronomické měřicí metody a přístroje.

      Největší problém činilo určování zeměpisné délky. Dokud se Evropané ve starověku a středověku
      plavili v útulném Středomoří, kde bylo možno z každého místa doplout za jeden den k
      nejbližšímu pobřeží, nebo když Vikingové provozovali pobřežní plavbu podél západoevropských
      břehů, nebyla tato otázka příliš naléhavá. Jakmile se ovšem Kolumbus vydal na neprobádanou
      cestu na západ Atlantickým oceánem a začal překračovat další a další poledníky, mohl určovat
      svou polohu jen podle rychlosti lodi, měřené nedokonalým plavboměrem, a porovnávat místní čas
      s časem ve výchozím přístavu, odměřovaným přesýpacími hodinami. Ty měl plavčík za úkol každou
      čtvrthodinu převracet, a záleželo tak i na jeho problematické svědomitosti. Kolumbus ostatně
      údaje o zeměpisné délce sám upravoval, aby posádka neměla představu, jak daleko na západ už
      dopluli. Dost na tom, že námořníci byli vyděšeni tím, že jim střelka kompasu přestala ukazovat
      na Polárku.

      Když si však někdo chce dělit zeměkouli napůl, musí být schopen určovat zeměpisnou délku
      přesně. Potřebuje k tomu dalekohled, sextant, astronomické znalosti a přesné lodní hodiny -
      chronometr. To si uvědomil dokonce i anglický král Karel II., když se na něj v roce 1675
      obrátil astronom \textsc{John Flamsteed} (1646-1719) s návrhem na zřízení státní, tedy
      královské hvězdárny. V královském rozhodnutí se založení hvězdárny výslovně zdůvodňuje
      \emph{„aby bylo možno zjišťovat zeměpisnou délku míst ke zdokonalení navigace a astronomie."}
      Král se dokonce vzdal svého honebního revíru na stráni v Greenwichi na pravém břehu Temže
      (byla stejně holá a málo zvěřinatá) a souhlasil s tím, aby tam byla z použitého stavebního
      materiálu vybudována observatoř. Zároveň zavedl novou funkci a jmenoval Flamsteeda
      ,,královským astronomem". Ten musel investovat do vybavení hvězdárny své vlastní finanční
      prostředky a v podstatě živořil. Současně vznikla ve Francii i královská pařížská observatoř,
      kam byl z Itálie povolán astronom \textsc{Giovanni Domenico Cassini} (1625-1712), jehož
      potomci ho následovali v této funkci v několika generacích.

      Vědecká revoluce v Evropě byla tedy vyvolána naléhavými praktickými potřebami, ale měla
      připraveno i myšlenkové, filozofické zázemí. Postupně se prosazoval světový názor založený na
      Koperníkově modelu sluneční soustavy a astronomická měření ho stále přesvědčivěji potvrzovala.
      Vědecká metoda zkoumání se mohla opřít o výsledky práce myslitelů, kteří stoji u počátků
      novověké evropské filozofie. věku evropské filozofie. V Anglii to byl \textsc{Francis Bacon}
      (1561-l626) \textsc{René Descartes} (1596-1650). Oba představuji poněkud odlišné, ale vzájemně
      se doplňující přístupy ke zkoumání přírody a charakterizují různé směry, jimiž se ubírala
      vzájemně soupeřící anglická a francouzská fyzika té doby. 
      
      Bacon zastával v Anglii vysoké státní funkce. Zdůrazňoval význam vědění, které dává člověku
      obrovskou moc, a zabýval se myšlenkami \uv{velkého obnovení věd}, které by přinášelo lidem
      užitek a přispělo i k lepší organizaci lidské společnosti. Ve svém spise \uv{Nové organon} z
      roku 1620 reaguje na Aristotelovo dílo ,,Organon", odmítá čistě spekulativní, scholastickou
      aristotelovskou logiku a vychází z empirického, smyslového poznání, pozorování a pokusů. Je
      zakladatelem vědecké indukce, tedy metody, která logicky analyzuje a třídí zkušenosti, fakta a
      dospívá k obecným zákonitostem. Přitom se vědec musí oprostit od předsudků a vžitých představ,
      které Bacon nazývá  \uv{idoly}. Ve svém zaujetí pro pokusy šel Bacon tak daleko, že zemřel na
      zápal plic právě když zkoumal dlouhodobý vliv chladu na živý organismus. Bacon je
      představitelem anglického empirismu, který zapůsobil i na anglické fyziky včetně Newtona.  
      
      Ve Francii ovlivnil filozofické myšlení především Descartes (latinsky Kartesius). Pocházel z
      aristokratického katolického rodu, od dětství byl chabého zdraví a prošel složitým myšlenkovým
      vývojem. Navštěvoval jezuitskou kolej, studium ho však neuspokojilo, a naopak v něm rozvířilo
      mnoho pochyb. Studoval práva i medicínu, jako dobrovolník v holandském a pak v bavorském
      vojsku prošel Evropou i Čechami a někdy se uvádí, že se účastnil i bitvy na Bílé hoře.  Na
      dlouhých dvacet let pak zakotvil v Holandsku, kde se v červenci 1642 sešel i s Janem Amosem
      Komenským, i když se s ním filozoficky nepohodl. Descartovy názory narážely na odpor a
      vyvolávaly útoky ze strany jak katolických, tak protestantských kruhů a tyto útoky poněkud
      plachého Descarta deprimovaly. Descartes byl zastáncem Koperníkova názoru na sluneční
      soustavu, ale po Galileově odsouzení se zalekl a byl ve formulaci svých názorů vysloveně
      opatrný. Aby si zajistil větší klid k práci, často dokonce měnil místo svého pobytu. Jeho
      vědecké dílo mělo i řadu stoupenců a vzbudilo nakonec zájem švédské královny Kristýny. Pozvala
      Descarta do Stockholmu a ten ji musel vyučovat filozofii třikrát týdně od pěti hodin ráno.
      Descartes, který byl zvyklý vstávat až k poledni, takový režim, znásobený drsným severským
      podnebím, ovšem dlouho nepřežil. V únoru 1650 zemřel na zápal plic a v r. 1666 byly jeho
      ostatky převezeny do Paříže. Dnes je pohřben ve starobylém kostele Saint Germain-des-Prés,
      jeho lebka, která byla při převozu ostatků zcizena, odděleně v Museu člověka v Paříži. 
      
      Descartes je zakladatelem francouzského racionalismu. Je znám jeho výrok \uv{Cogito erg sum},
      \uv{Myslím, tedy jsem} a na základě rozumových úvah také založil svou vědeckou metodu. Ve svém
      slavném spise \uv{Rozprava o metodě} stanoví pravidla správného vědeckého uvažování. Jako
      první krok požaduje zpochybnit všechny dosavadní názory a tvrzení, pokud nejsou nade vší
      pochybnosti Descartes je zakladatelem francouzského \emph{racionalizmu}. Je znám jeho v rok
      \uv{Cogito ergo sum}, \uv{Myslím, tedy jsem} a na základě rozumových úvah také založil svou
      vědeckou metodu. Ve svém slavném spise \uv{Rozprava o metodě} stanoví pravidla správného
      vědeckého uvažování. Jako první krok požaduje zpochybnit všechny dosavadní názory a tvrzení,
      pokud nejsou nade vší pochybnost dokázány. Jeho \uv{De omnibus dubitandum}, \uv{O všem
      pochybovat}, znamená začínat zkoumání s čistou a nepředpojatou myslí. Dále požaduje rozdělit
      každou zkoumanou otázku na části, které by bylo možno lépe řešit. Při zkoumání je třeba
      postupovat od předmětů jednodušších, které lze snáze poznávat, ke složitějším. A konečně za
      čtvrté je třeba uspořádávat zjištěná fakta do výčtů a přehledů, aby nic nebylo opomenuto. Tato
      Descartova doporučení jsou jakýmsi základem vědecké metody rozumového zkoumání; týmž způsobem
      musí ostatně postupovat i detektiv při řešení složitého kriminálního případu. Descartes je tak
      zakladatelem analytické deduktivní metody, která vychází z několika málo obecných principů a
      zákonů a postupuje podle pravidel rozumového uvažování.

      Descartova filozofie, karteziánství, ovlivnila celou řadu pozdějších filozofů a myslitelů.
      Patřil k nim např. \textsc{Benedikt (Baruch) Spinoza} (1632-1677), holandský filozof
      portugalsko-židovského původu, ale i Leibniz, Pascal a další. Spinoza se pokusil pomocí
      Descartovy racionalistické filozofie a axiomatické metody geometrie vyložit i taková témata,
      jako je politika, etika nebo teologie. Dochází k závěru, že existuje jen jedna jediná
      substance, jíž je Bůh ztotožněný s přírodou. Takový názor, podle něhož se nic a nikdo do
      přírody zvnějšku nevměšuje se nazývá \emph{panteizmem}. Zmiňujeme se o něm proto, že je blízký
      chápání velkých fyziků. Ti byli uchváceni krásou a řádem přírody, a ta jim splývala s božstvím
      v jedno. Ke Spinozově panteizmu se hlásil např. i Einstein \cite[s.~141]{Stoll2009}.    
%---------------------------------------------------------------------------------------------------  
  \section{Moderní fyzika}
    \subsection{Kvantová Fyzika}\label{fyz:IchapIIsecV}
      Když jsme načrtli představu elektromagnetického pole, v němž se mohou šířit vlny, brzy
      zjistíme, že tyto vlny se chovají nezvykle, jako kdyby to ani vlny nebyly. Při vyšších
      frekvencích se více podobají \emph{částicím}! Jejich neobvyklé chování vysvětluje
      \emph{kvantová mechanika}, jejíž vznik je spojován s obdobím těsně po roce 1920. Před rokem
      1920 pozměnil Einstein obraz trojrozměrného prostoru a nezávislého času nejdříve na kombinaci,
      kterou nazýváme \emph{prostoročasem} a potom na \emph{zakřivený} prostoročas, aby vystihl
      gravitaci. „Scéna“ se změnila na prostoročas a o gravitaci předpokládáme, že je modifikací
      prostoročasu. Zjistilo se dokonce, že zákony pro pohyb částic jsou nepřesné. Mechanické zákony
      „setrvačnosti“ a „síly“ jsou \emph{nesprávné} - Newtonovy zákony neplatí ve světě atomů.
      Zjistilo se, že věci se v malém měřítku chovají úplně jinak než věci ve velkém měřítku. To
      dělá fyziku obtížnou, ale velmi zajímavou. Obtížnou proto, že chování věcí malých rozměrů je
      pro nás „nepřirozené“, nemáme v tomto směru přímé zkušenosti. Věci se tu chovají úplně jinak,
      než jsme zvyklí, a proto není možné popsat jejich chování jinak, než analyticky. Takový popis
      je těžký a vyžaduje mnoho představivosti.
      
      Kvantová mechanika má mnoho zvláštností. Především vylučuje předpoklad, že částice má určitou
      polohu a určitou rychlost. Abychom ukázali, do jaké míry je klasická fyzika správná, uvedeme
      pravidlo kvantové mechaniky, které říká, že není možné současně vědět, kde se něco nachází a
      jak rychle se to pohybuje. Neurčitost v hybnosti a neurčitost v poloze jsou
      \emph{komplementární} a jejich součin je konstantní. Můžeme to zapsat následujícím způsobem:
      \(\Delta x \Delta p \frac{\si{\planckbar}}{2\pi}\). Podrobněji bude o tomto principu mluveno
      později. Vysvětluje se tím velmi záhadný paradox: jsou-li atomy složeny z kladných a záporných
      nábojů, proč se záporný náboj prostě neusadí na kladném náboji (tyto náboje se přitahují) a to
      tak těsně, že by ho úplně vyrušil? \emph{Proč jsou atomy tak velké}? Proč je jádro uprostřed a
      elektrony okolo něho? Zpočátku se myslelo, že příčinou je velký rozměr jádra; jenže jádro je
      velmi malé. Atom má průměr okolo \SI{10e-10}{\meter}. Jádro má průměr asi \SI{10e-15}{\meter}.
      Kdybychom měli atom a chtěli bychom vidět jeho jádro, museli bychom ho zvětšit tak, aby dosáhl
      velikosti místnosti a i potom by bylo jádro malé jako skvrnka, kterou sotva spatříte okem, ale
      téměř \emph{všechna hmotnost} atomu připadá na toto nepatrné jádro. Co brání elektronu prostě
      spadnout na jádro? Právě uvedený princip. Kdyby elektrony byly v jádru, znali bychom přesně
      jejich polohu a princip neurčitosti by si potom vyžadoval, aby měly velmi velkou (ale
      \emph{neurčitou}) hybnost, tj. velmi velkou \emph{kinetickou energii}. S takovou energií by se
      odtrhly od jádra. Dochází proto ke kompromisu: elektrony si ponechají jakýsi prostor pro tuto
      neurčitost a potom se ve shodě s tímto pravidlem pohybují s jistým minimálním množstvím
      pohybu. (Vzpomeňte si, že atomy krystalu při ochlazení na absolutní nulu neustaly ve svém
      pohybu, ale přece jen kmitaly. Proč? Kdyby se přestaly pohybovat, věděli bychom, kde se
      nacházejí a že mají nulový pohyb a to by bylo v rozporu s principem neurčitosti. Nemůžeme
      vědět, kde jsou a jak rychle se pohybují; proto atomy musí neustále kmitat!)
      
      Jinou, velmi zajímavou změnou v ideách a filozofii vědy, kterou přinesla kvantová mechanika,
      je nemožnost přesně předpovědět, co se za jakýchkoli daných okolností odehraje. Například, je
      možné připravit atom, který bude emitovat světlo, a můžeme zjistit, kdy k této emisi došlo
      tím, že zachytíme foton (o tomto si brzy řekneme více). Nemůžeme však dopředu předpovědět, kdy
      se uskuteční emise světla, nebo v případě více atomů, který z nich bude emitovat světlo. Možná
      se domníváte, že je to proto, že v atomu se nacházejí jakási vnitřní „kolečka“, která jsme
      ještě nerozeznali. Ne, taková vnitřní kolečka neexistují! Příroda, tak jak ji dnes chápeme, se
      chová tak, že je principiálně nemožné přesně předpovědět, co se skutečně stane v daném
      experimentu. 
      
      Opět se vrátíme ke kvantové mechanice a základní fyzice, ale nebudeme zabíhat do podrobností
      kvantově mechanických principů, protože jsou dost těžké k pochopení. Budeme prostě
      předpokládat jejich existenci a ukážeme, k jakým následkům vedou. Jedním z následků je, že
      věci, které jsme považovali za vlny, se chovají jako částice a částice zase jako vlny; ve
      skutečnosti se tedy všechno chová stejně. Není rozdíl mezi vlnou a částicí. \textbf{Kvantová
      mechanika sjednocuje myšlenku pole, jeho vln a částic vjedno.} Při nízkých frekvencích je
      aspekt pole více zřejmý, resp. užitečnější pro přibližný popis vyjádřený řečí naší každodenní
      zkušenosti. Se vzrůstem frekvence však zařízení, které obvykle používáme v experimentu,
      poskytuje spíše důkazy o částicích. I když mluvíme o vysokých frekvencích, musíme přiznat, že
      v oblasti frekvencí nad \SI{10e12}{\Hz} nebyl zatím zjištěn žádný jev přímo související s
      frekvencí. K existenci vyšších frekvencí docházíme pouze úvahou vycházející z energie částic a
      předpokladu správnosti \emph{vlnově-korpuskulární představy kvantové mechaniky}.
      
      Takto docházíme i k novému pohledu na \emph{elektromagnetickou interakci}. Kromě elektronu,
      protonu a neutronu existuje nový druh částice. Tuto částici nazýváme foton. Nový pohled na
      interakci elektronů a protonů, tj. \emph{elektromagnetickou teorii}, která zároveň
      \emph{splňuje} zákonitosti \emph{kvantové mechaniky}, nazýváme \emph{kvantovou
      elektrodynamikou}. Tato základní teorie \emph{interakce světla a hmoty}, nebo
      \emph{elektrického pole a nábojů}, je dosud největším úspěchem fyziky. V této jediné teorii
      máme základní zákony, jimiž se řídí všechny známé jevy s výjimkou gravitace a jaderných
      procesů. Pomocí kvantové elektrodynamiky můžeme vysvětlit všechny známé zákony mechaniky,
      elektřiny a chemie. Plynou, zní zákony srážek kulečníkových koulí, pohyb vodičů v magnetickém
      poli i tepelná kapacita oxidu uhelnatého, barva neonových reklam, hustota soli, reakce vodíku
      a kyslíku při vzniku vody - to vše jsou následky jediného zákona. Všechny tyto detaily je
      možné získat, je-li situace dost jednoduchá na to, abychom ji mohli přibližně popsat. To sice
      není splněno téměř nikdy, často však můžeme pochopit více či méně, co se vlastně děje. Dosud
      se neobjevily žádné výjimky ze zákonů kvantové elektrodynamiky, až na atomová jádra. O jádrech
      však nemůžeme říci, jestli jde v jejich případě o výjimku, protože vlastně nevíme, jaké
      procesy v nich probíhají. Při budování teorie jádra musíme překonat tři hlavní problémy:
      \begin{enumerate}[noitemsep]
      \item Není znám přesný tvar sil působících mezi nukleony v jádře,
      \item rovnice popisující pohyb nukleonů v jádře jsou velmi komplikované - problém  
            matematického popisu,
      \item jádro má zároveň příliš mnoho nukleonů (nedá se popsat pohyb každé jeho částice) i    
            příliš málo (nedá se popsat jako makroskopické spojité prostředí).   
      \end{enumerate}
      Proto se musíme spokojit pouze s modely atomového jádra. 
      
      V podstatě je kvantová elektrodynamika teorií celé chemie a všech životních procesů, je-li
      možné život v konečném důsledku redukovat na chemii, nebo vlastně na fyziku, protože chemie
      vede k fyzice (a ta část fyziky, která se uplatňuje v chemii, je již dobře známá). Navíc,
      kvantová elektrodynamika - ta úžasná vědní disciplína - předpověděla mnoho nových věcí.
      Především mluví o vlastnostech fotonů velmi velkých energií, paprscích gama apod. Předpověděla
      i jinou, velmi pozoruhodnou věc: kromě elektronu musí existovat jiná částice se stejnou
      hmotností, ale s opačným nábojem, tzv. \emph{pozitron} a elektron s pozitronem mohou při
      srážce anihilovat, přičemž se vyzáří světlo nebo paprsky gama (což je vlastně totéž, neboť
      světlo i záření gama se liší polohou ve frekvenční škále elektromagnetických vln). Zobecnění
      poznatku, že ke každé částici existuje antičástice, se ukazuje být pravdivým. V případě
      elektronů má antičástice jiné jméno - nazývá se pozitronem, ale u většiny jiných částic
      mluvíme o anti-tom a tom, např. o antiprotonu nebo antineutronu. Do kvantové elektrodynamiky
      se vkládají \emph{dvě čísla} a o většině ostatních čísel ve světě se předpokládá, že jsou
      následkem těchto dvou. Tato dvě vkládaná čísla nazýváme hmotností a nábojem elektronu. Ve
      skutečnosti to však není úplně tak, neboť máme celý soubor chemických čísel, která hovoří o
      tom, jak těžká jsou jádra. To nás přivádí k další kapitole.
    
    \subsection{Atom}  
      \luagraphic[0.8]{fyz_fig0895.pdf}{\wikiAtomJadro: Stylizovaný model atomu helia s atomovým
      poloměrem \SI{30}{\pm}. \uv{Mlha} znároňuje  elektronový obal, sestávající z orbitalu 1s,
      přičemž odstín vyjadřuje hustotu pravděpodobnosti výskytu 2 elektronů (integrovanou podél
      přímky pohledu). Oblast atomového jádra, je vyznačena růžově; jeho zvětšenina, na které jsou
      červeně zobrazeny 2 protony a fialově 2 neutrony, je však jen schematická. Ve skutečnosti je i
      jádro helia (a vlnové funkce jednotlivých nukleonů) kulově symetrické. Jádro je tedy kladně
      nabitou částí atomu, která tvoří jeho hmotnostní i prostorové centrum (jádro představuje
      \SI{99.9}{\percent} hmotnosti atomu). Průměr jádra činí přibližně \SI{10e-15}{\m}, což je
      přibližně \(\num{100 000}\times\) méně než průměr celého atomu. Existence atomového jádra byla
      poprvé pozorována v Rutherfordově experimentu, na jehož základě vznikl tzv. planetární model
      atomu.}{fyz:fig0895} 
    %-----------------------------------------------------------------------------------------------
    \subsection{Jádra a Částice}\label{fyz:IchapIIsecVI}     
      \emph{Z čeho jsou jádra a jak drží pohromadě}? Zjistilo se, že jádra jsou udržována obrovskými
      silami. Při uvolnění těchto sil se uvolňuje energie, která je obrovská v porovnání s chemickou
      energií, tak jak je obrovský výbuch atomové bomby v porovnání s výbuchem trinitrotoluenu. U
      atomové bomby jde totiž o změny uvnitř jádra, zatímco výbuch trinitrotoluenu souvisí se
      změnami elektronového obalu atomů. Proto si klademe otázku: co jsou to za síly, které udržují
      protony a neutrony v jádře pohromadě? Tak, jako je možné elektrické působení přisoudit částici
      - fotonu, předpokládal Yukawa, že i síly mezi neutrony a protony mají svá pole a kmity tohoto
      pole se chovají jako částice. Kromě neutronů a protonů by proto měly existovat jiné částice a
      Yukawa odvodil vlastnosti těchto částic z již známých charakteristik jaderných sil. Například,
      předpověděl, že by měly mít hmotnost dvěstě až třistakrát větší než elektron; a div se světe -
      v kosmickém záření byly objeveny částice s takovouto hmotností! Později se ukázalo, že to
      nebyla ta správná částice. Tuto částici nazvali \(\mu\text{-mezon}\) neboli \emph{mion}.

      Trochu později, v roce 1947 nebo 1948, byla objevena jiná částice, \(\pi\text{-mezon}\) neboli
      \emph{pion}, která vyhovovala Yukawovu kritériu. Abychom získali jaderné síly, musíme k
      protonu a neutronu přidat pion. A teď si řeknete: „Och, jak velkolepé! - pomocí této teorie
      vybudujeme nukleodynamiku, ve které budou mít piony takovou úlohu, jakou jim přisoudil Yukawa
      a všechno bude vysvětleno“. Ta věc má však háček! Ukázalo se, že výpočty v této teorii jsou
      tak složité, že se dodnes nikomu nepodařilo odvodit všechny důsledky této teorie, nebo ji
      porovnat s experimentem; a to se už táhne spoustu let!
      
      Máme tedy teorii, ale nevíme, jestli je správná nebo nesprávná. Víme však už, že je trochu
      chybná, nebo aspoň neúplná. Zatím co jsme marnili čas teorií a snažili se odvodit její
      důsledky, experimentátoři některé věci objevili. Například, objevili \(\mu\text{-mezon}\)
      neboli mion a my ani nevíme, jaká je jeho úloha. V kosmickém záření se našel velký počet
      dalších „přebytečných“ částic. Dnes máme přibližně třista takových částic a je velmi těžké
      porozumět vztahům mezi těmito částicemi a pochopit, na co je příroda potřebuje, nebo která z
      nich na které závisí. Dnes tyto různé částice nechápeme jako různé aspekty téže věci a
      skutečnost, že máme tak mnoho nesouvisejících částic, je odrazem toho, že máme tak mnoho
      nesouvisejících informací bez dobré teorie. Po ohromném úspěchu kvantové elektrodynamiky máme
      jisté znalosti z jaderné fyziky, ale jen hrubé znalosti, částečně experimentální a částečně
      teoretické. Vycházíme přitom z charakteru sil působících mezi protony a neutrony a sledujeme,
      co z toho vyplyne, ale v podstatě nechápeme, odkud ty síly pocházejí. Kromě toho nebylo
      dosaženo téměř žádného pokroku. Objevili jsme velký počet chemických prvků. Mezi těmito prvky
      se najednou objevila souvislost, neočekávaná souvislost zakotvená v Mendělejevově periodické
      tabulce prvků. Například, sodík a draslík jsou téměř shodné ve svých chemických vlastnostech a
      v Mendělejevově tabulce se nacházejí ve stejném sloupci. Hledala se tabulka Mendělejevova typu
      pro nové částice. Taková tabulka nových částic byla sestavena nezávisle Gell-Mannem v USA a
      Nishijimou v Japonsku. Základem jejich klasifikace je nové číslo, jež je možno, podobně jako
      elektrický náboj, přiřadit každé částici a které se nazývá její „podivností“ S (od anglického
      slova strangeness). Toto číslo se, podobně jako elektrický náboj, zachovává v reakcích
      vyvolávaných jadernými silami.  
    
    \subsection{Kosmonautika}
      \subsubsection{Pohled na Zemi z vesmíru}  
      \subsubsection{Projekt Apollo}  
%---------------------------------------------------------------------------------------------------
  \section{Integrační tendence ve fyzice}\label{fyz:IchapIIsecVIII}
    Není to tak dávno, co se fyzikové dělili na dvě velké skupiny – experimentátory a teoretiky.
    Příslušník každé skupiny věděl, že se bez členů druhé skupiny neobejde. Výsledkem byla plodná
    spolupráce plná zdánlivé řevnivosti a úsměvných historek. S nástupem výpočetní techniky se vše
    změnilo. Postupně vznikala skupina třetí, která se zabývá numerickými simulacemi. Bez nich si
    dnes fyziku nedovedeme představit. Numerické simulace umožňují první ověření výsledků nových
    teorií bez nákladných experimentů. Při zpracování experimentálních dat pomáhají hledat procesy,
    které se za naměřenými údaji skrývají. V současnosti má fyzika tři nedílné celky: teorii,
    experiment a numerické simulace. 
    
    Fyzika zaznamenává v průběhu staletí dvě základní tendence. První z nich je postupné členění na
    další a další podobory. Tento vývoj souvisí s prohlubujícím se poznáním a je přirozenou cestou v
    každé vědní disciplíně. Postupně vznikají specialisté na stále užší a užší obory, vytvářejí si
    svůj vlastní vědecký jazyk a schopnost komunikace odborníků z dříve blízkých oblastí fyziky se
    stále zhoršuje. Na druhé straně dochází k hlubšímu pochopení souvislostí mezi jednotlivými
    částmi fyziky a k jejich postupnému sjednocování do univerzálnějších teorií. Možná se jednou
    podaří sjednotit fyzikální pohled na všechny základní přírodní interakce do jedné jediné teorie,
    kterou dnes nazýváme Teorie všeho (anglicky TOE, Theory Of Everything). Tyto integrační tendence
    ve fyzice jsou znázorněny na obrázku 1. 

    \luagraphic[1]{fyz_fig0924.pdf}{Integrační tendence ve fyzice.
      \cite[s.~12]{Kulhanek2019}}{fyz:fig0924}
    
    Mechanika jakožto vědecká fyzikální disciplína vznikala od 17. století. První známější vědecké
    experimenty prováděl \textsc{Galileo Galilei} (1564–1642). Teoretickou konstrukci klasické
    mechaniky, jakožto nástroje pro předpověď pohybu těles v daném silovém poli, navrhnul
    \textsc{Isaac Newton} (1642–1727) ve svých \emph{Principiích (Philosophiæ Naturalis Principia
    Mathematica)} z roku 1687. V 18. století dovršil konstrukci klasické mechaniky \textsc{Joseph
    Louis Lagrange} (1736–1813), který mechanické úlohy formuloval nezávisle na volbě souřadnicové
    soustavy za pomoci variačního počtu.
    
    V 19. století se úspěšně dařilo poznávat a postupně chápat elektrické a magnetické děje. Na
    experimentech se podílela celá řada významných fyziků, například \textsc{Hans Oersted}
    (1777–1881), \textsc{André Ampère} (1775–1836), \textsc{ichael Faraday} (1791–1867),
    \textsc{Heinrich Hertz} (1857–1894), \textsc{Oliver Heaviside} (1850–1925) a další. Celé toto
    údobí vyvrcholilo poznáním, že jevy elektrické a magnetické mají shodnou povahu a společný
    původ. V roce 1873 publikoval \textsc{James Clerk Maxwell} (1831–1879) pojednání \uv{A Treatise
    on Electricity and Magnetism}, které obsahovalo rovnice, jež završily klasickou elektrodynamiku
    do jednoho jediného celku obsahujícího jak děje elektrické, tak magnetické. 
    
    Na konci 19. století podlehlo mnoho fyziků iluzi, že fyzika jako věda je dokončena. Byly známy
    zákony mechaniky na jedné straně a zákony elektřiny a magnetizmu na straně druhé. Na první
    pohled se zdálo, že veškeré přírodní děje jsou důsledkem těchto dvou vědních disciplin a
    budoucnost fyziky je pouze v aplikaci známých zákonů na neznámé situace. Šlo samozřejmě o krutý
    omyl, který se rychle projevil na počátku dvacátého století, kdy nebylo možné tehdejšími
    znalostmi vysvětlit řadu fyzikálních dějů. 

    Ukázalo se, že jak klasická mechanika, tak klasická elektrodynamika nedokáží uspokojivě popsat
    svět na úrovni atomů. Důsledkem toho byla neschopnost objasnit chování elektronu v atomárním
    obalu, vysvětlit záření absolutně černého tělesa, pochopit fotoelektrický jev a smířit se s
    projevy objektů mikrosvěta, které vykazovaly někdy částicové a jindy vlnové vlastnosti. Zrodila
    se kvantová mechanika, ve které neplatí \(ab = ba\), a nekomutativnost se stala nově objeveným
    rysem přírody na mikroskopické úrovni. Kvantová mechanika s sebou přinesla celou řadu těžko
    představitelných jevů – kvantování energie a momentu hybnosti, dualismus vln a částic, relace
    neurčitosti, nejednoznačnost aktu měření a pravděpodobnostní interpretaci výsledků vedoucí na
    nedeterminizmus kvantové fyziky.

    A to byl teprve začátek. Spin elementárních částic objevený v roce 1925 znamenal další výrazný
    posun lidstva v chápání přírody. Je důsledkem relativistické fyziky, která se od počátku 20.
    století rozvíjela paralelně s kvantovou mechanikou. Spojení kvantové mechaniky se speciální
    relativitou vedlo na Diracovu rovnici, která se stala základem kvantového popisu pohybu
    elektronu. \textsc{Paul Adrien Maurice Dirac} (1902–1984) navrhnul svou rovnici v roce 1928 a
    téhož roku z ní odvodil existenci pozitronu, antičástice k elektronu. Pozitron byl
    experimentálně objeven až o 4 roky později \textsc{Carlem Andersonem} (1905–1991). Za svou práci
    získal Dirac Nobelovu cenu za fyziku pro rok 1933. V letech 1946 až 1949 byla dokončena první
    kvantově polní teorie – \emph{kvantová teorie elektromagnetického pole}, které dnes říkáme
    \textbf{kvantová elektrodynamika} (\emph{QED, Quantum Electro-Dynamics}). Za její formulaci
    získali Nobelovu cenu za fyziku pro rok 1965 \textsc{Richard Feynman} (1918–1988),
    \textsc{Shin-Itiro Tomonaga} (1906–1979) a \textsc{Julian Schwinger} (1918–1994). Kvantová
    elektrodynamika je kvantovou analogií Maxwellových rovnic. Elektromagnetická interakce je
    způsobena polními částicemi, v tomto případě fotony, které si mezi sebou posílají nabité
    částice. Klasický pojem síly ztrácí svůj smysl. Feynmanovi se podařilo složité rovnice
    interpretovat za pomoci názorných grafů, kterým dnes říkáme \emph{Feynmanovy diagramy}. Na
    obdobném základě byla později vytvořena také současná \textbf{kvantová teorie slabé a silné
    interakce}. Základním rysem těchto teorií jsou tzv. \emph{kalibrační symetrie}, které předurčují
    způsob působení dané interakce na elementární částice.

    Od počátku 60. let probíhaly snahy o spojení elektromagnetické a slabé interakce do jednoho
    jediného celku. Podařilo se to \textsc{Stevenu Weinbergovi} (1933), \textsc{Abdusu Salamovi}
    (1926–1996) a \textsc{Sheldonu Glashowovi} (1932). Za svou práci získali Nobelovu cenu za fyziku
    pro rok 1979. Jimi předpovězené polní částice slabé interakce \(W^+\), \(W^–\) a \(Z^0\) byly
    objeveny na přelomu let 1983 a 1984 v evropském středisku jaderného výzkumu CERN. Jejich
    objevitelé, \textsc{Carlo Rubbia} (1934) a \textsc{Simon van der Meer} (1925–2011) získali
    Nobelovu cenu ještě téhož roku (1984).

    K pochopení silné interakce přispěl již ve 30. letech japonský fyzik \textsc{Hideki Yuakawa}
    (1907–1981). Za svou práci získal Nobelovu cenu za fyziku pro rok 1949. Současná kvantově polní
    teorie silné interakce se nazývá \textbf{kvantová chromodynamika} (\emph{QCD, Quantum
    Chromo-Dynamics}) a za její formulaci a zejména za objev asymptotické volnosti silné interakce
    kvarků a gluonů získali Nobelovu cenu za fyziku pro rok 2004 \textsc{Frank Wilczek} (1951),
    \textsc{David Gross} (1941) a \textsc{David Politzer} (1949).
    
    Kvantová mechanika slavila v průběhu 20. století mimořádné úspěchy. Jednoduchá teorie popisující
    mechanické děje postupně přerostla v polní kvantovou teorii schopnou úspěšně popsat hned tři ze
    čtyř základních přírodních interakcí. Tato cesta se samozřejmě neobešla bez potíží a problémů,
    nicméně vyústila v dnešní \textbf{standardní model elementárních částic a interakcí}. Bez
    kvantové teorie a hlubokého pochopení zákonitostí mikrosvěta bychom dnes neměli ani počítače ani
    jinou elektroniku. 

    Na počátku 20. století ale vznikala ještě jedna, neméně úspěšná teorie – obecná relativita. Z
    Maxwellovy elektrodynamiky plynulo, že rychlost světla by ve vakuu měla být univerzální
    konstantou a že by se neměla sčítat s rychlostí zdroje elektromagnetického vlnění. Tento
    výsledek byl na první pohled v rozporu s klasickou mechanikou, ve které se rychlost zdroje s
    rychlostí signálu sčítá. Řada experimentů potvrdila správnost elektrodynamiky. Bylo tedy třeba
    přeformulovat mechaniku tak, aby byla v souladu s Maxwellovou elektrodynamikou. To se v roce
    1905 podařilo Albertu Einsteinovi v rámci tzv. \textbf{speciální teorie relativity}. Daň za
    sjednocení obou teorií byla veliká. Čas spolu s prostorem přestaly být absolutní. Délka letící
    tyče a časový úsek mezi dvěma událostmi ve skutečnosti závisejí na volbě souřadnicové soustavy
    pozorovatele.
    
    Einsteinovy snahy o zobecnění speciální relativity na neinerciální souřadnicové soustavy vedly v
    roce 1915 ke vzniku obecné relativity – zcela nové teorie gravitace, která popisuje tuto
    interakci za pomoci zakřiveného času a prostoru. Za základ nové teorie lze chápat dvě myšlenky:
    \begin{itemize}[noitemsep]
      \item každé těleso svou přítomností zakřivuje časoprostor kolem sebe;
      \item každé těleso se v tomto zakřiveném časoprostoru pohybuje po nejrovnějších možných
            drahách – tzv. \emph{geodetikách}.
    \end{itemize}

    Nové chápání času a prostoru bylo zcela revoluční. Samotná tělesa se podílejí na vytváření času
    a prostoru, bez nich by čas a prostor neexistoval. Otázka, jak by vypadal vesmír bez přítomnosti
    těles, přestává mít smysl.
    
    Fyzika dvacátého století se tak stala v jistém smyslu poněkud schizofrenní. Tři ze čtyř
    interakcí jsou popsány za pomoci výměnných (polních) částic v rámci kvantové teorie pole. A
    jedna interakce, gravitační, je popsána za pomoci pokřiveného světa obecné teorie relativity.
    Vyřešení mnoha fyzikálních hádanek s sebou přineslo ještě větší záhady. Existuje jednotná teorie
    všech čtyř interakcí? Je možné spojit kvantovou teorii a obecnou relativitu do jedné jediné
    teorie? Odpověď na tyto otázky zatím neznáme. Velké úspěchy slaví různé strunové teorie, ve
    kterých jsou částice chápány jako jednorozměrné kmitající útvary ve vícerozměrném světě, ale zda
    jde o krok správným směrem či nikoli, není v tuto chvíli jasné. V roce 2010 se objevila hypotéza
    holandského fyzika Erika Verlindeho, podle které by gravitace nemusela být skutečnou silou, ale
    jen statistickým projevem růstu entropie v mikrosvětě. Těžko odhadnout, zda tato odvážná
    myšlenka najde podporu v dalších experimentech, nebo jde o slepou uličku.
    
    Pokud vás zajímají základní vlastnosti přírody a jejich teoretický popis, je třeba v první řadě
    začít se studiem klasické mechaniky, na kterou úzce navazuje mechanika kvantová. Další studium
    polních problémů zase není možné bez znalosti statistické fyziky \cite[s.~14]{Kulhanek2019}. 