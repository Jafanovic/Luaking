% !TeX program = lualatex
% !TeX root = luaking.tex
% !TeX encoding = UTF-8
% !TeX spellcheck = cs_CZ
%---------------------------------------------------------------------------------------------------
% file fey1ch02.tex
%---------------------------------------------------------------------------------------------------
%===================== Kapitola: Dějiny fyziky ====================================================
\setchaptertoc
\chapter{Dějiny fyziky}\label{fyz:IchapI} 
  \section{Hlavní etapy vývoje}\label{fyz:IchapIsecIII}
    Fyzika prošla dlouhým historickým vývojem a znalost tohoto vývoje pomáhá lépe pochopit logiku 
    soustavy fyzikálních poznatků a dokonce do\-cházet k poznatkům novým. V krátkosti dějiny 
    fyziky můžeme rozdělit na tři hlavní etapy:
    \begin{itemize}[noitemsep]
     	\item \textbf{Stará fyzika}: od starověku do počátku 17. století (orientačně do roku 1600).
      \item \textbf{Klasická fyzika}: 1600 - 1900.
      \item \textbf{Moderní fyzika}: 1900 - dosud.
    \end{itemize}
    Starou fyziku nemůžeme považovat za vědu ve vlastním smyslu, i když se dobrala celé řady 
    významných vědeckých poznatku. První z nich znali již staří Sumerové, Babyloňané, Egypťané a 
    Číňané. Šlo zejména o  poznatky astronomické a geometrické (Pythagorova veta) a také o metody 
    měření některých fyzikálních veličin (délka, hmotnost, čas). Fyzika ve starém Řecku byla jako 
    součást filosofie převážně spekulativní a tento charakter si pod vlivem aristotelismu udržela, 
    až do počátku novověku. Skutečný fyzikální výzkum prováděli až helenističtí Řekové, kdy se 
    centrem vědy a kultury antického světa stala Alexandrie. 
    
    \begin{figure}[ht!]  % \ref{fyz:fig0894}
      \centering
      \luafigure[1]{fyz_fig0894.jpg}
      \caption{ \wikiAlexLib byla největší a nejslavnější knihovna starověku. Byla součástí
                věhlasného múseia v Alexandrii, vybudovaného z podnětu Ptolemaia I. Byla považována
                za hlavní centrum vzdělanosti od 3. století př. n. l. až do roku 48 př. n. l., kdy
                za války mezi Caesarem a Pompeiem zčásti vyhořela. Starověké zdroje pojednávají o
                ničení knihovny, o tom, kdo je zodpovědný za ničení a kdy k němu došlo, se liší.}
      \label{fyz:fig0894}
    \end{figure} 

    V Alexandrii studoval největší fyzik starověku Archimédes, který dospěl k důležitým poznatkům o
    statické rovnováze těles a plování těles a v matematice se těsně přiblížil objevu
    diferenciálního a integrálního počtu. Alexandrijští Řekové znali také zákon odrazu světla
    (nikoli lomu) a prováděli první měření teploty. Poznatky antiky byly středověké Evropě
    zprostředkovány Araby, kteří se též intenzivně zabývali optikou (Alhazen) a určováním měrné
    hmotnosti látek. Zatímco ve středověku byly hlavní přírodovědné poznatky čerpány z Euklidových ”
    Základu” (geometrie), ”Almagestu” Klaudia Ptolemaia (geocentrický výklad astronomie sluneční
    soustavy) a spisu Aristotelových (mj.”Fysika”), vešly práce Archimédovy v Evropě ve známost až
    teprve začátkem novověku. Ve starověku a středověku však fyzika neprováděla systematické
    experimenty, nevyužívala matematický aparát k popisu přírodních jevu a neměla ani přesně
    definovány základní pojmy (rychlost, zrychlení, síla apod.) Zrod fyziky jako vědy se datuje
    začátkem 17. století. Na základě astronomických výzkumu Keplerových (1571-1630) a pozemských
    mechanických experimentů Galileových (1564-1642) mohl Isaac Newton (1643-1727) vytvořit první
    fyzikální teorii, klasickou mechaniku, využívající matematický aparát diferenciálního a
    integrálního poctu. Newton přišel s koncepcí všeobecné gravitace a ukázal, že není přehrady mezi
    nebeskou a pozemskou fyzikou, že síla, která udržuje planety na jejich drahách kolem Slunce je
    táž jako síla, která nutí jablko padat k zemi. Základní Newtonovo dílo z r. l687 nese název ”
    Matematické základy přírodní filosofie” (”Philosophiae naturalis principia mathematica”) a
    představuje pravděpodobně nejvýznamnější vědeckou knihu, která byla kdy napsána. Newton se
    zabýval též optikou a rozpracoval teorii rozkladu bílého světla do spektra. V té době byl již
    zásluhou Snellovou a Descartovou znám i zákon lomu světla. Z roku 1600 pochází první vědecký
    spis o elektřině a magnetismu od anglického lékaře a fyzika Gilberta. Výzkumem  těchto jevu se v
    následujících stoletích zabývala celá řada fyziků (Coulomb, Volta, Oersted, Amp\`{e}re a další).
    Tento výzkum pak završil Faraday (1791-1867) svým objevem zákona elektromagnetické indukce a
    svou koncepcí siločár elektromagnetického pole. Úlohu Newtona elektromagnetismu pak sehrál James
    Clerk Maxwell (1831-1879), který ve svém ”Traktátě o elektřině a magnetismu” z r. 1873 sestavil
    slavné Maxwellovy rovnice popisující vlastnosti elektromagnetického pole. Maxwell zároveň
    teoreticky zdůvodnil elektromagnetickou povahu světla a ukázal, že jevy spojené s vlastnostmi
    elektrického náboje (”elektřina”), elektrického proudu (”galvanismus”), magnetického pole a
    světla (optika), jsou jedné a téže elektromagnetické povahy. V devatenáctém století byl tak
    dovršen výzkum mechanických jevů a elektromagnetismu a klasická fyzika tím za\-vršena. V přírodě
    tedy existovaly pouze dvě síly, dva způsoby vzájemné interakce mezi částicemi: gravitační a
    elektromagnetická. Mezi nimi se však projevoval určitý rozpor. Jak Newtonovy tak Maxwellovy
    rovnice platí v libovolné inerciální vztažné soustavě. Při přechodu od jedné inerciální soustavy
    k druhé se však Newtonovy rovnice transformují pomocí tzv. Galileiho transformací a Maxwellovy
    rovnice pomocí Lorentzových transformací. Fyzika se tak rozdvojila, mechanické a
    elektromagnetické děje se zdály být neslučitelné. Kromě toho existovaly některé experimenty,
    jejichž výsledek nedokázala klasická fyzika vysvětlit: průběh spektra rovnovážného
    elektromagnetického záření (tzv. záření absolutně černého tělesa) a pokus Michelsonův, který
    svědčil o neexistenci světelného éteru. Tyto zdánlivě nepodstatné rozpory vyústily ve 20.
    století ve vznik moderní fyziky, tj. fyziky kvantové a relativistické. Právě koncem roku 1900
    vyslovil Planck tzv. kvantovou hypotézu, jíž vysvětlil záření absolutně černého tělesa, a v r.
    1905 publikoval Einstein práci o speciální teorii relativity. V ní překlenul rozpor mezi
    Newtonovou a Maxwellovou fyzikou a fyziku opět sjednotil. Předpoklad o existenci světelného
    éteru se teorií relativity stal zbytečným. V roce 1916 vytvořil Einstein i obecnou teorii
    relativity jako moderní teorii gravitace. Gravitační síly podle této teorie souvisejí se
    zakřivením prostoročasu. Jak speciální, tak obecná teorie relativity přecházejí při rychlostech
    objektu podstatně menších než je rychlost světla ve vakuu a při slabých gravitačních polích v
    teorii Newtonovu. Přelom 19. a 20. století je též poznamenán objevem radioaktivity a vznikem
    jaderné fyziky, která tak významným způsobem zasáhla do života celého lidstva. V jaderné fyzice
    se uplatní další dvě přírodní síly - tzv. silná, která udržuje nukleony v atomových jádrech a
    slabá, která se projevuje při radioaktivní přeměně beta za vzniku neutrin. Moderní fyzika
    odhalila v kosmickém záření a pomocí urychlovačů obrovské množství částic, jejichž vlastnosti
    studuje a snaží se je utřídit a vysvětlit. Mezi všemi těmito částicemi působí čtyři základní
    síly přírody: gravitační, elektromagnetická, silná a slabá. V nedávné době se podařilo prokázat,
    že i elektromagnetická a slabá interakce jsou téže podstaty a tvoří jedinou sílu elektroslabou.
    V průběhu historie fyziky od Newtona a Maxwella k dnešku tak probíhá úsilí o sjednocování
    interakcí, které pokračuje i dnes. Fyzika se pokouší prokázat, že i silná a elektroslabá
    interakce jsou téže povahy, a že k nim konečně přistupuje i síla gravitační. Tím by vznikla idea
    jediné přírodní síly sjednocující všechny přírodní jevy a děje. Fyzika ovšem nemůže k takovému
    závěru dojít pouhým uvažováním, musí matematicky vypracovat a zdůvodnit příslušnou teorii a její
    závěry experimentálně ověřit. To vede ke snaze budovat stále větší a větší urychlovače a také k
    intenzivnímu výzkumu jevů v kosmu. Sjednocování interakcí má totiž těsnou návaznost na vývoj
    vesmíru podle hypotézy o tzv. ”velkém třesku”. Právě v počátcích vývoje vesmíru by se měly
    všechny čtyři (resp. tři) interakce uplatňovat rovnocenným způsobem a teprve v průběhu dalšího
    vývoje a rozpínání vesmíru se postupně oddělovat. Tak jako počátky vzniku vědecké fyziky v 17.
    století jsou spjaty s astronomickými pozorováními sluneční soustavy, je i dnes fyzika stále více
    propojena s astrofyzikou. Vesmír zůstává největší fyzikální laboratoří.
  
  \section{Fyzika před rokem 1920}\label{fyz:IchapIsecIV}
    Je dost těžké začít hned se současnými představami, a proto se podívejme, jak se jevil svět v 
    roce 1920 a potom na tomto obrázku něco změníme. Naše představa světa byla před rokem 
    \textbf{1920} následující: „Scénou“, na které vystupuje vesmír, je \emph{trojrozměrný 
    geometrický prostor} popsaný ještě Eukleidem a věci se mění v prostředí, které nazýváme časem. 
    Prvky vystupující na scéně jsou \emph{částice}, například atomy, které mají určité vlastnosti. 
    Především vlastnost setrvačnosti: pohybuje-li se částice, zachová si pohyb v původním směru, 
    pokud na ni nepůsobí \emph{síly}. Druhým prvkem jsou tedy síly, o nichž se tehdy  
    předpokládalo, že jsou dvojího druhu. K prvnímu, velmi složitému druhu, patřila síla vzájemného 
    působení, která udržovala atomy v jejich různých kombinacích komplikovaným způsobem a byla 
    zodpovědná za to, jestli se sůl při zvyšování teploty rozpouští rychleji nebo pomaleji. Druhou 
    známou silou byla interakce dalekého dosahu - hladké a klidné přitahování. Tato síla, měnící se 
    nepřímo úměrně čtverci vzdálenosti, byla nazvána \emph{gravitací}. Její zákon byl známý a byl 
    velmi jednoduchý. Proč věci zůstávají v pohybu, když se už začaly pohybovat, nebo proč existuje 
    gravitační zákon, bylo, samozřejmě, neznámé.
    
    Zabýváme se popisem přírody. Z tohoto hlediska je plyn a právě tak všechna hmota myriádou 
    pohybujících se částic. Takto se dostávají do souvislosti mnohé věci, které jsme viděli na 
    mořském břehu. \emph{Tlak} pochází od \emph{srážek atomů} se stěnami nebo s čímkoliv jiným; 
    atomy pohybující se převážně jedním směrem vytvářejí vítr; \emph{chaotické vnitřní pohyby} 
    představují \emph{teplo}. Známe vlny zvýšené hustoty, kde se shromáždilo příliš mnoho částic, 
    které při rozletu stlačují další shluky částic a pohyb se tak předává dál. Tyto vlny vyšší 
    hustoty představují \emph{zvuk}. Pochopení tolika věcí je možno považovat za úžasný úspěch. O 
    některých z těchto věcí jsme hovořili v předcházející kapitole.
    
    Jaké druhy částic existují? Tehdy předpokládali, že je jich 92. Nakonec bylo objeveno 92 
    různých druhů atomů. Měly různá jména podle svých chemických vlastností.
    
    Byl tu ještě problém \emph{povahy sil krátkého dosahu}. Proč uhlík přitahuje jeden kyslík, 
    případně dva, ale ne víc? Jaký je mechanizmus vzájemného působení mezi atomy? Je to gravitace? 
    Na tuto otázku musíme odpovědět záporně, protože gravitace je na to příliš slabá. Představme si 
    však sílu podobnou gravitaci, měnící se nepřímo úměrně čtverci vzdálenosti, ale mnohem silnější 
    a odlišnou ještě v jednom směru. V případě \emph{gravitace jde vždy o přitahování}. Představme 
    si však, že existují dva druhy „věcí“ a tato nová síla  (samozřejmě elektrické povahy) má tu 
    vlastnost, že věci stejného druhu se odpuzují a věci různého druhu se přitahují. „Předmět“, 
    jenž je nositelem tohoto silného vzájemného působení, se nazývá \emph{náboj}.  
    
    K čemu jsme došli? Předpokládejme, že máme dvě věci různého druhu, jež se vzájemně  
    přitahují (plus a minus) a které drží těsně u sebe. Předpokládejme, že v určité vzdálenosti od 
    uvedené dvojice máme další náboj. Bude tento náboj pociťovat přitažlivost? Mají-li první dva 
    náboje stejnou velikost, neměl by pocítit \emph{prakticky žádnou přitažlivost}, protože 
    přitahování jedním nábojem a odpuzování druhým nábojem se vykompenzují. Ve velkých 
    vzdálenostech je tedy síla velmi malá. Když třetí náboj \emph{hodně přiblížíme} k prvním dvěma, 
    objeví se přitahování, protože odpuzování stejných nábojů a přitahování různých se snaží 
    oddálit stejné náboje a přiblížit různé. Odpuzování bude nakonec \emph{slabší} než přitahování. 
    To je příčina, proč atomy, které se skládají z kladných a záporných elektrických nábojů, na 
    sebe téměř nepůsobí (zanedbáme-li gravitaci), jsou-li od sebe dost vzdáleny. Když se ale 
    přiblíží, mohou „\emph{vidět jeden do druhého}“, přeskupit své náboje a velmi silně vzájemně 
    působit. Podstatou interakce mezi atomy je \emph{elektrické} působení. Tato síla je tak veliká, 
    že všechny plusy a minusy se obvykle dostávají do tak těsné kombinace, jak je to jen možné. 
    Všechny věci, včetně nás samotných, se skládají z drobných, velmi silně interagujících kladných 
    a záporných částic, které jsou velmi přesně vyvážené. Na okamžik je možné náhodou odstranit 
    několik minusů nebo plusů (obvykle je jednodušší odstranit minusy), v tu chvíli jsou elektrické 
    síly \emph{nevyvážené} a můžeme pozorovat působení elektrické přitažlivosti.
    
    Abychom si vytvořili představu o tom, o kolik je elektrické působení silnější než gravitace, 
    představme si dvě zrnka písku, která mají jeden milimetr v průměru a jsou vzdálená třicet 
    metrů. Kdyby elektrické síly mezi nimi nebyly vyvážené, kdyby nebylo odpuzování a vše se 
    navzájem přitahovalo a nic se nekompenzovalo, jakou silou by se zrnka přitahovala? Byla by to 
    síla tří miliónů tun. Jistě chápete, že pro vytvoření značného elektrického působení stačí 
    velmi malý přebytek nebo nedostatek záporných nebo kladných nábojů. Proto není vidět rozdíl 
    mezi elektricky nabitým a nenabitým předmětem - pro nabití předmětu je třeba tak málo částic, 
    že se téměř neprojeví na jeho hmotnosti, či rozměru.
    
    S těmito poznatky bylo jednodušší pochopit atomy. Předpokládalo se, že mají uprostřed 
    „\emph{jádro}“, které je kladně elektricky nabité a velmi těžké, a toto jádro je obklopeno 
    určitým počtem „elektronů“, jež jsou velmi lehké a záporně nabité. Teď trochu pokročíme v našem 
    výkladu a poznamenáme, že v samotných jádrech byly objeveny dva druhy částic - \emph{protony} a 
    \emph{neutrony}, které mají téměř stejnou, velmi velkou hmotnost. Protony jsou elektricky 
    nabité a neutrony jsou neutrální. Máme-li atom se šesti protony v jádře, které je obklopeno 
    šesti elektrony (záporné částice obyčejného světa jsou všechno elektrony a ty jsou velmi lehké 
    v porovnání s protony a neutrony, které tvoří jádra), půjde o atom číslo šest v chemické 
    tabulce a tento atom se nazývá uhlík. Atom číslo osm se nazývá kyslík atd. Chemické     
    vlastnosti závisí na vnějších elektronech, ve skutečnosti jen na tom, kolik má atom elektronů. 
    \emph{Chemické vlastnosti} látek tedy závisí na jediném čísle, na \emph{počtu elektronů}. 
    (Seznam prvků sestavený chemiky by se mohl nahradit očíslováním 1, 2, 3, 4, 5 atd. Místo toho, 
    abychom říkali „uhlík“, stačilo by říci „prvek číslo šest“, což by znamenalo, že prvek má šest 
    elektronů. Při objevování prvků však tato skutečnost nebyla známa a dále, při číslování by vše 
    vypadalo velmi složitě. Proto je lepší ponechat prvkům názvy i symboly a nedožadovat se pouhého 
    očíslování.)

    \begin{figure*}[ht!] %\ref{fyz:fig0006}
      \centering
      \luafigure[1]{fyz_fig0006.pdf}
      \caption{Elektromagnetické spektrum (někdy zvané Maxwellova duha) zahrnuje elektromagnetické 
               záření všech možných vlnových délek. Srovnání délek elektromagnetických vln s 
               běžnými předměty a odpovídající teplotní stupnice umožňuje lépe získat představu o 
               jejich rozměrech a energiích.}
      \label{fyz:fig0006}
    \end{figure*}
    
    O elektrické síle bylo získáno mnoho dalších poznatků. Bylo by přirozené předpokládat, že 
    elektrická interakce je jednoduché přitahování dvou předmětů: kladného a záporného. Zjistilo se 
    však, že toto není úplně vhodná představa. Situaci lépe vystihuje představa, že existence 
    kladného náboje v prostoru způsobuje jeho jisté \emph{zakřivení}, vytváří v něm určitou 
    „podmínku“, aby záporný náboj vložený do tohoto prostoru cítil působení síly. Tato možnost 
    vzniku síly se nazývá \emph{elektrické pole}. Dostane-li se elektron do elektrického pole, je 
    jakoby „tažen“. Přitom platí dvě pravidla: a) \emph{náboje vytvářejí pole}, b) \emph{v poli 
    působí na náboje síly a náboje se pohybují}. Příčina takového chování se stane jasnější, 
    jakmile rozebereme následující jev: Nabijeme-li těleso elektricky, například hřeben, a do 
    určité vzdálenosti položíme nabitý ústřižek papíru, přičemž začneme hřebenem pohybovat sem a 
    tam, bude se papír natáčet směrem k hřebenu. Zrychlíme-li pohyb hřebenu, zjistíme, že papír 
    zaostává, působení se opožďuje. (V prvním stádiu, když pohybujeme hřebenem poměrně 
    pomalu, zkomplikuje nám situaci \emph{magnetizmus}. Magnetické vlivy se projevují, když jsou 
    \emph{náboje v relativním pohybu}, takže magnetické a elektrické síly je možné skutečně připsat 
    jedinému poli jako dvě stránky jedné věci. Měnící se elektrické pole nemůže existovat bez 
    magnetizmu.) Oddálíme-li nabitý papír, zpoždění je větší. V tu chvíli pozorujeme zajímavou věc. 
    Ačkoliv se síly působící mezi dvěma nabitými předměty mění nepřímo úměrně čtverci vzdálenosti, 
    při kmitání náboje zjišťujeme, že jeho působení se rozprostírá mnohem dále, než by se dalo 
    očekávat. Pokles tohoto působení je mnohem pomalejší než při nepřímé úměrnosti čtverci 
    vzdálenosti.
    
    S analogickou situací se setkáváme, když na vodě plave splávek a my ho uvedeme do pohybu 
    „přímo“ tím, že způsobíme pohyb vody jiným splávkem. Kdybychom se dívali jen na dva splávky, 
    pozorovali bychom pouze to, že jeden se dává do pohybu jako odezva na pohyb druhého, že mezi 
    nimi existuje určitá „  interakce“. Ve skutečnosti jsme ale rozčeřili vodu a voda posunula 
    druhý splávek. Mohli bychom zformulovat „zákon“, že i při slabém zčeření vody se na vodě budou 
    pohybovat předměty nacházející se blízko zdroje zčeření. Kdyby byl druhý splávek dost daleko, 
    sotva by se dal do pohybu, neboť jsme uvedli vodu do pohybu jen v jednom místě. Bude-li však 
    druhý splávek pravidelně kmitat, vznikne nový úkaz, při kterém se pohyb vody přenáší dál, 
    vzniká \emph{vlnění} a vliv poskakujícího splávku již nemůžeme chápat jako přímé působení mezi 
    splávky. Myšlenku přímé interakce tedy musíme nahradit předpokladem o existenci vody nebo v 
    případě elektrických nábojů tím, co nazýváme \emph{elektromagnetickým polem}.
    
    Elektromagnetické pole může přenášet vlny. Některé z těchto vln jsou světlo jak je znázorněno 
    na obrázku \ref{fyz:fig0006}, jiné se používají při rádiovém vysílání, ale obecně se 
    nazývají \emph{elektromagnetickými vlnami}. Tyto vlny mohou mít rozmanité \emph{frekvence}. 
    Jediné, čím se jedna vlna liší od druhé, je právě frekvence vlnění. Kdybychom pohybovali 
    nábojem sem a tam a dělali bychom to stále rychleji a rychleji, objevovala by se celá řada 
    různých jevů, které je možné systematizovat udáním čísla vyjadřujícího počet kmitů za sekundu. 
    Frekvence, s nimiž přicházíme do styku prostřednictvím běžných rozvodových elektrických sítí v 
    domech, jsou řádově sto kmitů za sekundu. Zvýšíme-li frekvenci na \SI{500}{\kHz} nebo 
    \SI{1000}{\kHz} (\SI{1}{\kHz} = 1000 kmitů za sekundu), dostáváme se z domů ven, „na 
    vzduch“, neboť máme co činit s frekvencemi používanými při rozhlasovém vysílání. (Se vzduchem 
    to ale nemá co dělat! Rádiové vlny se mohou šířit i v prostoru, v němž není vzduch.) 
    Zvyšujeme-li frekvenci, dostáváme se do oblasti \emph{VKV} a televizního vysílání. Při ještě 
    vyšších frekvencích máme velmi krátké vlny, které se využívají např. v \emph{radiolokaci}. 
    Kdybychom šli ještě výše, nepotřebovali bychom už zařízení na registraci takových vln, protože 
    bychom je viděli naším zrakem. Kdybychom dokázali pohybovat nabitým hřebenem tak rychle, aby 
    kmital s frekvencemi od \SI{5e14}{\Hz} do \SI{5e15}{\Hz}, viděli bychom toto kmitání jako 
    červené, modré nebo fialové světlo v závislosti na frekvenci. Frekvence pod touto oblastí 
    nazýváme \emph{infračervenými} a nad touto oblastí \emph{ultrafialovými}. Skutečnost, 
    že naše vidění je omezeno na určitou frekvenční oblast, nedělá tuto oblast elektromagnetického 
    spektra z fyzikálního hlediska důležitější než jiné oblasti, avšak z lidského hlediska je tato 
    oblast přece jen zajímavější. Kdybychom frekvenci ještě zvýšili, dostali bychom 
    \emph{rentgenové paprsky}. Tyto paprsky nejsou nic jiného, než světlo s velmi vysokou 
    frekvencí. Ještě vyšším frekvencím odpovídá \emph{záření gama}. Výrazy rentgenové paprsky a 
    záření gama jsou téměř synonyma. Zářením gama nazýváme obvykle elektromagnetické vlny 
    pocházející z jader a rentgenovými paprsky vlny pocházející z atomů; při shodě jejich frekvencí 
    jsou však fyzikálně nerozlišitelné, bez zřetele na jejich původ. Vlny ještě vyšších 
    frekvencí, řekněme \SI{10e24}{\Hz}, lze získat uměle, například na \emph{synchrotronu} v 
    Caltechu. Elektromagnetické vlny úžasně vysokých frekvencí (až tisíckrát vyšších) je možné 
    najít ve vlnách \emph{kosmického záření}. Tyto vlny však neumíme ovládat. 
    \cite[s.~29]{Feynman02}
  
  \section{Kvantová Fyzika}\label{fyz:IchapIsecV}
    Když jsme načrtli představu elektromagnetického pole, v němž se mohou šířit vlny, brzy 
    zjistíme, že tyto vlny se chovají nezvykle, jako kdyby to ani vlny nebyly. Při vyšších 
    frekvencích se více podobají \emph{částicím}! Jejich neobvyklé chování vysvětluje 
    \emph{kvantová mechanika}, jejíž vznik je spojován s obdobím těsně po roce 1920. Před rokem 
    1920 pozměnil Einstein obraz trojrozměrného prostoru a nezávislého času nejdříve na kombinaci, 
    kterou nazýváme \emph{prostoročasem} a potom na \emph{zakřivený} prostoročas, aby vystihl 
    gravitaci. „Scéna“ se změnila na prostoročas a o gravitaci předpokládáme, že je modifikací 
    prostoročasu. Zjistilo se dokonce, že zákony pro pohyb částic jsou nepřesné. Mechanické zákony 
    „setrvačnosti“ a „síly“ jsou \emph{nesprávné} - Newtonovy zákony neplatí ve světě atomů. 
    Zjistilo se, že věci se v malém měřítku chovají úplně jinak než věci ve velkém měřítku. To dělá 
    fyziku obtížnou, ale velmi zajímavou. Obtížnou proto, že chování věcí malých rozměrů je pro nás 
    „nepřirozené“, nemáme v tomto směru přímé zkušenosti. Věci se tu chovají úplně jinak, než jsme 
    zvyklí, a proto není možné popsat jejich chování jinak, než analyticky. Takový popis je těžký a 
    vyžaduje mnoho představivosti.
    
    Kvantová mechanika má mnoho zvláštností. Především vylučuje předpoklad, že částice má určitou 
    polohu a určitou rychlost. Abychom ukázali, do jaké míry je klasická fyzika správná, uvedeme 
    pravidlo kvantové mechaniky, které říká, že není možné současně vědět, kde se něco nachází a 
    jak rychle se to pohybuje. Neurčitost v hybnosti a neurčitost v poloze jsou 
    \emph{komplementární} a jejich součin je konstantní. Můžeme to zapsat následujícím způsobem: 
    \(\Delta x \Delta p \frac{\si{\planckbar}}{2\pi}\). Podrobněji bude o tomto principu mluveno 
    později. Vysvětluje se tím velmi záhadný paradox: jsou-li atomy složeny z kladných a záporných 
    nábojů, proč se záporný náboj prostě neusadí na kladném náboji (tyto náboje se přitahují) a to 
    tak těsně, že by ho úplně vyrušil? \emph{Proč jsou atomy tak velké}? Proč je jádro uprostřed a 
    elektrony okolo něho? Zpočátku se myslelo, že příčinou je velký rozměr jádra; jenže jádro je 
    velmi malé. Atom má průměr okolo \SI{10e-10}{\meter}. Jádro má průměr asi \SI{10e-15}{\meter}. 
    Kdybychom měli atom a chtěli bychom vidět jeho jádro, museli bychom ho zvětšit tak, aby dosáhl 
    velikosti místnosti a i potom by bylo jádro malé jako skvrnka, kterou sotva spatříte okem, ale 
    téměř \emph{všechna hmotnost} atomu připadá na toto nepatrné jádro. Co brání elektronu prostě 
    spadnout na jádro? Právě uvedený princip. Kdyby elektrony byly v jádru, znali bychom přesně 
    jejich polohu a princip neurčitosti by si potom vyžadoval, aby měly velmi velkou (ale 
    \emph{neurčitou}) hybnost, tj. velmi velkou \emph{kinetickou energii}. S takovou energií by se 
    odtrhly od jádra. Dochází proto ke kompromisu: elektrony si ponechají jakýsi prostor pro tuto 
    neurčitost a potom se ve shodě s tímto pravidlem pohybují s jistým minimálním množstvím pohybu. 
    (Vzpomeňte si, že atomy krystalu při ochlazení na absolutní nulu neustaly ve svém pohybu, ale 
    přece jen kmitaly. Proč? Kdyby se přestaly pohybovat, věděli bychom, kde se nacházejí a že mají 
    nulový pohyb a to by bylo v rozporu s principem neurčitosti. Nemůžeme vědět, kde jsou a jak 
    rychle se pohybují; proto atomy musí neustále kmitat!)
    
    Jinou, velmi zajímavou změnou v ideách a filozofii vědy, kterou přinesla kvantová mechanika, je 
    nemožnost přesně předpovědět, co se za jakýchkoli daných okolností odehraje. Například, je 
    možné připravit atom, který bude emitovat světlo, a můžeme zjistit, kdy k této emisi došlo tím, 
    že zachytíme foton (o tomto si brzy řekneme více). Nemůžeme však dopředu předpovědět, kdy se 
    uskuteční emise světla, nebo v případě více atomů, který z nich bude emitovat světlo. Možná se 
    domníváte, že je to proto, že v atomu se nacházejí jakási vnitřní „kolečka“, která jsme ještě 
    nerozeznali. Ne, taková vnitřní kolečka neexistují! Příroda, tak jak ji dnes chápeme, se chová 
    tak, že je principiálně nemožné přesně předpovědět, co se skutečně stane v daném experimentu. 
    
    Opět se vrátíme ke kvantové mechanice a základní fyzice, ale nebudeme zabíhat do podrobností 
    kvantově mechanických principů, protože jsou dost těžké k pochopení. Budeme prostě předpokládat 
    jejich existenci a ukážeme, k jakým následkům vedou. Jedním z následků je, že věci, které jsme 
    považovali za vlny, se chovají jako částice a částice zase jako vlny; ve skutečnosti se tedy 
    všechno chová stejně. Není rozdíl mezi vlnou a částicí. \textbf{Kvantová mechanika sjednocuje 
    myšlenku pole, jeho vln a částic vjedno.} Při nízkých frekvencích je aspekt pole více zřejmý, 
    resp. užitečnější pro přibližný popis vyjádřený řečí naší každodenní zkušenosti. Se vzrůstem 
    frekvence však zařízení, které obvykle používáme v experimentu, poskytuje spíše důkazy o 
    částicích. I když mluvíme o vysokých frekvencích, musíme přiznat, že v oblasti frekvencí nad 
    \SI{10e12}{\Hz} nebyl zatím zjištěn žádný jev přímo související s frekvencí. K existenci 
    vyšších frekvencí docházíme pouze úvahou vycházející z energie částic a předpokladu správnosti 
    \emph{vlnově-korpuskulární představy kvantové mechaniky}.
    
    Takto docházíme i k novému pohledu na \emph{elektromagnetickou interakci}. Kromě elektronu, 
    protonu a neutronu existuje nový druh částice. Tuto částici nazýváme foton. Nový pohled na 
    interakci elektronů a protonů, tj. \emph{elektromagnetickou teorii}, která zároveň 
    \emph{splňuje} zákonitosti \emph{kvantové mechaniky}, nazýváme \emph{kvantovou 
    elektrodynamikou}. Tato základní teorie \emph{interakce světla a hmoty}, nebo 
    \emph{elektrického pole a nábojů}, je dosud největším úspěchem fyziky. V této jediné teorii 
    máme základní zákony, jimiž se řídí všechny známé jevy s výjimkou gravitace a jaderných 
    procesů. Pomocí kvantové elektrodynamiky můžeme vysvětlit všechny známé zákony mechaniky, 
    elektřiny a chemie. Plynou, zní zákony srážek kulečníkových koulí, pohyb vodičů v magnetickém 
    poli i tepelná kapacita oxidu uhelnatého, barva neonových reklam, hustota soli, reakce vodíku a 
    kyslíku při vzniku vody - to vše jsou následky jediného zákona. Všechny tyto detaily je možné 
    získat, je-li situace dost jednoduchá na to, abychom ji mohli přibližně popsat. To sice není 
    splněno téměř nikdy, často však můžeme pochopit více či méně, co se vlastně děje. Dosud se 
    neobjevily žádné výjimky ze zákonů kvantové elektrodynamiky, až na atomová jádra. O jádrech 
    však nemůžeme říci, jestli jde v jejich případě o výjimku, protože vlastně nevíme, jaké procesy 
    v nich probíhají. Při budování teorie jádra musíme překonat tři hlavní problémy:
    \begin{enumerate}[noitemsep]
     \item Není znám přesný tvar sil působících mezi nukleony v jádře,
     \item rovnice popisující pohyb nukleonů v jádře jsou velmi komplikované - problém  
           matematického popisu,
     \item jádro má zároveň příliš mnoho nukleonů (nedá se popsat pohyb každé jeho částice) i    
           příliš málo (nedá se popsat jako makroskopické spojité prostředí).   
    \end{enumerate}
    Proto se musíme spokojit pouze s modely atomového jádra. 
    
    V podstatě je kvantová elektrodynamika teorií celé chemie a všech životních procesů, je-li 
    možné život v konečném důsledku redukovat na chemii, nebo vlastně na fyziku, protože chemie 
    vede k fyzice (a ta část fyziky, která se uplatňuje v chemii, je již dobře známá). Navíc, 
    kvantová elektrodynamika - ta úžasná vědní disciplína - předpověděla mnoho nových věcí. 
    Především mluví o vlastnostech fotonů velmi velkých energií, paprscích gama apod. Předpověděla 
    i jinou, velmi pozoruhodnou věc: kromě elektronu musí existovat jiná částice se stejnou 
    hmotností, ale s opačným nábojem, tzv. \emph{pozitron} a elektron s pozitronem mohou při srážce 
    anihilovat, přičemž se vyzáří světlo nebo paprsky gama (což je vlastně totéž, neboť světlo i 
    záření gama se liší polohou ve frekvenční škále elektromagnetických vln). Zobecnění poznatku, 
    že ke každé částici existuje antičástice, se ukazuje být pravdivým. V případě elektronů má 
    antičástice jiné jméno - nazývá se pozitronem, ale u většiny jiných částic mluvíme o anti-tom a 
    tom, např. o antiprotonu nebo antineutronu. Do kvantové elektrodynamiky se vkládají \emph{dvě 
    čísla} a o většině ostatních čísel ve světě se předpokládá, že jsou následkem těchto dvou. Tato 
    dvě vkládaná čísla nazýváme hmotností a nábojem elektronu. Ve skutečnosti to však není úplně 
    tak, neboť máme celý soubor chemických čísel, která hovoří o tom, jak těžká jsou jádra. To nás 
    přivádí k další kapitole.
  
  \section{Jádra a Částice}\label{fyz:IchapIsecVI}
    \emph{Z čeho jsou jádra a jak drží pohromadě}? Zjistilo se, že jádra jsou udržována obrovskými 
    silami. Při uvolnění těchto sil se uvolňuje energie, která je obrovská v porovnání s chemickou 
    energií, tak jak je obrovský výbuch atomové bomby v porovnání s výbuchem trinitrotoluenu. U 
    atomové bomby jde totiž o změny uvnitř jádra, zatímco výbuch trinitrotoluenu souvisí se změnami 
    elektronového obalu atomů. Proto si klademe otázku: co jsou to za síly, které udržují protony a 
    neutrony v jádře pohromadě? Tak, jako je možné elektrické působení přisoudit částici - fotonu, 
    předpokládal Yukawa, že i síly mezi neutrony a protony mají svá pole a kmity tohoto pole se 
    chovají jako částice. Kromě neutronů a protonů by proto měly existovat jiné částice a Yukawa 
    odvodil vlastnosti těchto částic z již známých charakteristik jaderných sil. Například, 
    předpověděl, že by měly mít hmotnost dvěstě až třistakrát větší než elektron; a div se 
    světe - v kosmickém záření byly objeveny částice s takovouto hmotností! Později se ukázalo, že 
    to nebyla ta správná částice. Tuto částici nazvali \(\mu\text{-mezon}\) neboli \emph{mion}.

    \begin{figure}[hbt!]  % \ref{fyz:fig0895}
      \centering
      \luafigure[0.8]{fyz_fig0895.pdf}
      \caption{ \wikiAtomJadro: Stylizovaný model atomu helia s atomovým poloměrem \SI{30}{\pm}.
                \uv{Mlha} znároňuje  elektronový obal, sestávající z orbitalu 1s, přičemž odstín
                vyjadřuje hustotu pravděpodobnosti výskytu 2 elektronů (integrovanou podél přímky
                pohledu). Oblast atomového jádra, je vyznačena růžově; jeho zvětšenina, na které
                jsou červeně zobrazeny 2 protony a fialově 2 neutrony, je však jen schematická. Ve
                skutečnosti je i jádro helia (a vlnové funkce jednotlivých nukleonů) kulově
                symetrické. Jádro je tedy kladně nabitou částí atomu, která tvoří jeho hmotnostní i
                prostorové centrum (jádro představuje \SI{99.9}{\percent} hmotnosti atomu). Průměr
                jádra činí přibližně \SIrange{10}{-15}{\m}, což je přibližně \(\num{100 000}\times\)
                méně než průměr celého atomu. Existence atomového jádra byla poprvé pozorována v
                Rutherfordově experimentu, na jehož základě vznikl tzv. planetární model atomu.}
      \label{fyz:fig0895}
    \end{figure} 
    
    Trochu později, v roce 1947 nebo 1948, byla objevena jiná částice, \(\pi\text{-mezon}\) neboli 
    \emph{pion}, která vyhovovala Yukawovu kritériu. Abychom získali jaderné síly, musíme k protonu 
    a neutronu přidat pion. A teď si řeknete: „Och, jak velkolepé! - pomocí této teorie vybudujeme 
    nukleodynamiku, ve které budou mít piony takovou úlohu, jakou jim přisoudil Yukawa a všechno 
    bude vysvětleno“. Ta věc má však háček! Ukázalo se, že výpočty v této teorii jsou tak složité, 
    že se dodnes nikomu nepodařilo odvodit všechny důsledky této teorie, nebo ji porovnat s 
    experimentem; a to se už táhne spoustu let!
    
    Máme tedy teorii, ale nevíme, jestli je správná nebo nesprávná. Víme však už, že je trochu 
    chybná, nebo aspoň neúplná. Zatím co jsme marnili čas teorií a snažili se odvodit její 
    důsledky, experimentátoři některé věci objevili. Například, objevili \(\mu\text{-mezon}\) 
    neboli mion a my ani nevíme, jaká je jeho úloha. V kosmickém záření se našel velký počet 
    dalších „přebytečných“ částic. Dnes máme přibližně třista takových částic a je velmi těžké 
    porozumět vztahům mezi těmito částicemi a pochopit, na co je příroda potřebuje, nebo která z 
    nich na které závisí. Dnes tyto různé částice nechápeme jako různé aspekty téže věci a 
    skutečnost, že máme tak mnoho nesouvisejících částic, je odrazem toho, že máme tak mnoho 
    nesouvisejících informací bez dobré teorie. Po ohromném úspěchu kvantové elektrodynamiky máme 
    jisté znalosti z jaderné fyziky, ale jen hrubé znalosti, částečně experimentální a částečně 
    teoretické. Vycházíme přitom z charakteru sil působících mezi protony a neutrony a sledujeme, 
    co z toho vyplyne, ale v podstatě nechápeme, odkud ty síly pocházejí. Kromě toho nebylo 
    dosaženo téměř žádného pokroku. Objevili jsme velký počet chemických prvků. Mezi těmito prvky 
    se najednou objevila souvislost, neočekávaná souvislost zakotvená v Mendělejevově periodické 
    tabulce prvků. Například, sodík a draslík jsou téměř shodné ve svých chemických vlastnostech a 
    v Mendělejevově tabulce se nacházejí ve stejném sloupci. Hledala se tabulka Mendělejevova typu 
    pro nové částice. Taková tabulka nových částic byla sestavena nezávisle Gell-Mannem v USA a 
    Nishijimou v Japonsku. Základem jejich klasifikace je nové číslo, jež je možno, podobně jako 
    elektrický náboj, přiřadit každé částici a které se nazývá její „podivností“ S (od anglického 
    slova strangeness). Toto číslo se, podobně jako elektrický náboj, zachovává v reakcích 
    vyvolávaných jadernými silami.  
  
  \section{Vědecká revoluce 17. století}\label{fyz:IchapIsecVII}
    \textbf{Klasická fyzika}, jak ji popsal Richard Feynnman v předchozích kapitolách, tedy jako
    věda vycházející z měření a experimentů a opírající se o matematickou teorii, věda, která nám
    podává ucelený obraz přírody a světa a svými výsledky slouží technickému pokroku, vznikla v
    Evropě v průběhu sedmnáctého století. Tento dějinný převrat, který předznamenal naši dnešní
    civilizaci, nazýváme \textbf{obdobím vědecké revoluce}. Nebyla to ovšem nějaká náhlá událost a
    lidé si tehdy ani neuvědomili, jakou vlastně prožívají dobu a co přinese budoucím pokolením.
    Vědecká revoluce nastala za zvláštních podmínek evropského vývoje, které se v jiných částech
    světa nevytvořily.

    \begin{mdframed}[style=mdnote]
      \begin{note}
        \textbf{Kuhnovo pojetí vývoje vědy}: \textsc{Thomas Samuel Kuhn} přinesl argumenty o tom, že
        pokrok vědeckého poznání není přímočarý, ale že je čas od času přerušován zásadními
        zvraty-vědeckými revolucemi. Při těchto vědeckých revolucích dochází k přehodnocení
        samotných základů dosavadního vědění. Vědecké poznání tedy nesměřuje k nějaké jediné pravdě
        o světě, netýká se žádné „objektivní reality“ - nezávislé skutečnosti, všem společné, vždy
        zde již jsoucí. Věda, tak jako každá lidská činnost, má svůj kulturní, dějinný, instituční,
        sociální a psychologický rozměr. I vědecké poznatky jsou proto historicky podmíněné:
        vyjadřují ducha dané epochy, mění se s dobou i s okolnostmi.

        {\centering
        \captionsetup{type=figure} 
        \luafigure[0.5]{fyz_fig0893.jpg}
        \captionof{figure}{\wikiKuhn (\textasteriskcentered 18. 7. 1922 - \textdagger 17. 6. 1996)
                  byl americký filosof, fyzik, teoretik vědy a vědeckého poznání, zabýval se
                  dějinami vědy, astronomií, kvantovou teorií a její prehistorií.}
        \label{fyz:fig0893}
      \par}
      \end{note}
    \end{mdframed}

    Příčin, které vyvolaly tuto vědeckou revoluci, bylo mnoho a nemůžeme je zde podrobně zkoumat.
    Především to byly nové politické a hospodářské podmínky, potřeby výroby, obchodu a podnikání,
    které vyzvedly do popředí nové společenské síly, především měšťanské. Vzrůstající produktivita
    práce a vznik prvních kolektivních dílen, manufaktur, potřebovaly nové způsoby silového pohonu.
    Zásobování surovinami a vývoz hotových výrobků si vyžádal rozvoj mořeplavby a námořní navigace.
    Evropské války, jak už to bývá, také podnítily zdokonalování vojenské techniky a nepřímo i
    rozvoj přírodních věd \cite[s.~137]{Stoll2009}.

    Důležitou úlohu sehrála reformace, odklon řady zemí v západní a severní Evropě od katolické
    církve a papežství a vznik nových, protestantských církví. Protestantismus usiloval o bližší
    kontakt jednotlivého člověka s Bohem, bez prostřednictví církevní hierarchie, o návrat k podobě
    bible v jejich původních jazycích (hebrejském a řeckém) a podnítil vznik překladů biblických
    textů do národních evropských jazyků. Tím na jedné straně vyvolal potřebu studia klasických
    jazyků a umožnil také zpřístupnění výsledků vědy starověkého Řecka a na druhé straně podpořil
    rozvoj národních jazyků (připomeňme si jen krásnou češtinu naší Kralické bible). Latina, ve
    středověku univerzální jazyk vzdělanců, začala ztrácet své výsadní postavení.

    S rostoucím vědomím užitečnosti a nutnosti vědeckého poznání bez vměšování teologického
    dogmatismu přenášejí protestanti těžiště náboženského cítění do oblasti morální, jako vodítko
    při hledání smyslu lidského života, a ponechávají přírodním vědám zkoumání a využívání
    přírodních zákonů. Odmítají víru v Boží zázraky, která vlastně znemožňuje existenci vědy. Takový
    přístup, kdy Bůh je chápán jen jako stvořitel a první zákonodárce, který se však do dalšího
    chodu přírody už nevměšuje, nazýváme \textbf{deismus}, na rozdíl od katolického teismu, podle
    něhož Bůh do běhu světa stále zasahuje a bez jehož vůle, ani vlas z hlavy nespadne“. Protože
    nositeli idejí protestantismu byly především měšťanské a hospodářsky aktivní vrstvy společností,
    rozvíjí se věda a vědecká revoluce zejména v protestantských zemích západní Evropy v Holandsku,
    Anglii, Švýcarsku, Dánsku, částečně ve Francii a Německu. Také příznačný podnikatelský duch
    Ameriky má své kořeny v anglosaském protestantismu prvních přistěhovalců. Katolická Itálie,
    která renesanci vědy zahájila, nakonec odsoudila svého Galilea, katolické Španělsko a
    Portugalsko, které zbohatly při zámořské kolonizaci, postupně svou moc ztrácejí a k vědecké
    revoluci v Evropě nepřispívají.

    Evropa nebyla nikdy soběstačná v některých druzích výrobků, ať už šlo o tropické plody, rostliny
    (bavlna, cukrová třtina) a koření, drahé kovy, ale třeba i hedvábí, vzácné kožešiny a jiné
    výnosné luxusní předměty. Obchodní cesty k jejich získávání vedly odedávna přes Středozemí,
    Blízký a Střední Východ a na tomto obchodu bohatly zejména italské městské státy jako Benátky
    nebo Janov. Když postupující turecká expanze tyto přístupové cesty znesnadnila a ohrozila,
    hledaly státy západní a jihozápadní Evropy přístup na východní trhy obeplutím Afriky a po
    úspěšných výpravach Kolumbovych západním směrem do Ameriky.

    Španělsko a Portugalsko začaly z těchto nových cest a výbojů těžit jako první, jejich karavely a
    galeony, obtížené kořením, stříbrem a zlatem, přivážely toto zboží na evropské trhy, pokud
    neskončilo na mořském dně nebo v rukou pirátů. Obě tyto námořní mocnosti si známými smlouvami z
    Tordesillas (1494) a Zaragozy (1529) dokonce rozdělily celý svět na dvě poloviny a uskutečnily
    tak první globalizaci světového obchodu a kolonizace ve znamení katolicizmu.

    Nedokázaly však své nové hospodářské zdroje produktivně využít. Jejich pozice zaujala postupně
    Anglie, Francie, a zejména malé protestantské Holandsko, které se začátkem 17. století
    osvobodilo od španělské nadvlády a vytvořilo republiku pod vládou místodržitelů z rodu
    Oranžskeho je téměř neuvěřitelné, že Holandsko, počtem obyvatel srovnatelné s tehdejším českým
    královstvím, vytvořilo jeden čas největší koloniální říši světa a disponovalo flotilou 16 000
    lodi, počtem trojnásobně převyšujícím flotilu všech ostatních západoevropských států dohromady.
    Hospodářsky se postupně vzmáhala i Anglie, kde společenské napětí vyvrcholilo občanskou válkou a
    revoluci, která přivedla v roce 1649 krále Karla I. na popraviště. Všechny tyto společenské
    otřesy a změny v západní Evropě postupně vytvářely nové impulzy k rychlému vědeckému a
    technickému pokroku.

    Koloniální výboje vyvolaly potřebu mapovat nová území, dokonce mapovat zeměkouli jako celek,
    především přesně měřit zeměpisnou šířku a délku, ale i hloubku moří, teplotu a slanost mořské
    vody, rychlost a směr mořských proudů a magnetickou deklinaci, odchylku směru udávaného kompasem
    od pravého severu. To ovšem vyžadovalo prozkoumat přesný geometrický tvar zeměkoule a vytvořit
    nové fyzikální a astronomické měřicí metody a přístroje.

    Největší problém činilo určování zeměpisné délky. Dokud se Evropané ve starověku a středověku
    plavili v útulném Středomoří, kde bylo možno z každého místa doplout za jeden den k nejbližšímu
    pobřeží, nebo když Vikingové provozovali pobřežní plavbu podél západoevropských břehů, nebyla
    tato otázka příliš naléhavá. Jakmile se ovšem Kolumbus vydal na neprobádanou cestu na západ
    Atlantickým oceánem a začal překračovat další a další poledníky, mohl určovat svou polohu jen
    podle rychlosti lodi, měřené nedokonalým plavboměrem, a porovnávat místní čas s časem ve
    výchozím přístavu, odměřovaným přesýpacími hodinami. Ty měl plavčík za úkol každou čtvrthodinu
    převracet, a záleželo tak i na jeho problematické svědomitosti. Kolumbus ostatně údaje o
    zeměpisné délce sám upravoval, aby posádka neměla představu, jak daleko na západ už dopluli.
    Dost na tom, že námořníci byli vyděšeni tím, že jim střelka kompasu přestala ukazovat na
    Polárku.

    Když si však někdo chce dělit zeměkouli napůl, musí být schopen určovat zeměpisnou délku přesně.
    Potřebuje k tomu dalekohled, sextant, astronomické znalosti a přesné lodní hodiny - chronometr.
    To si uvědomil dokonce i anglický král Karel II., když se na něj v roce 1675 obrátil astronom
    \textsc{John Flamsteed} (1646-1719) s návrhem na zřízení státní, tedy královské hvězdárny. V
    královském rozhodnutí se založení hvězdárny výslovně zdůvodňuje \emph{„aby bylo možno zjišťovat
    zeměpisnou délku míst ke zdokonalení navigace a astronomie."} Král se dokonce vzdal svého
    honebního revíru na stráni v Greenwichi na pravém břehu Temže (byla stejně holá a málo
    zvěřinatá) a souhlasil s tím, aby tam byla z použitého stavebního materiálu vybudována
    observatoř. Zároveň zavedl novou funkci a jmenoval Flamsteeda ,,královským astronomem". Ten
    musel investovat do vybavení hvězdárny své vlastní finanční prostředky a v podstatě živořil.
    Současně vznikla ve Francii i královská pařížská observatoř, kam byl z Itálie povolán astronom
    \textsc{Giovanni Domenico Cassini} (1625-1712), jehož potomci ho následovali v této funkci v
    několika generacích.

    Vědecká revoluce v Evropě byla tedy vyvolána naléhavými praktickými potřebami, ale měla
    připraveno i myšlenkové, filozofické zázemí. Postupně se prosazoval světový názor založený na
    Koperníkově modelu sluneční soustavy a astronomická měření ho stále přesvědčivěji potvrzovala.
    Vědecká metoda zkoumání se mohla opřít o výsledky práce myslitelů, kteří stoji u počátků
    novověké evropské filozofie. věku evropské filozofie. V Anglii to byl \textsc{Francis Bacon}
    (1561-l626) \textsc{René Descartes} (1596-1650). Oba představuji poněkud odlišné, ale vzájemně
    se doplňující přístupy ke zkoumání přírody a charakterizují různé směry, jimiž se ubírala
    vzájemně soupeřící anglická a francouzská fyzika té doby. 
    
    Bacon zastával v Anglii vysoké státní funkce. Zdůrazňoval význam vědění, které dává člověku
    obrovskou moc, a zabýval se myšlenkami \uv{velkého obnovení věd}, které by přinášelo lidem
    užitek a přispělo i k lepší organizaci lidské společnosti. Ve svém spise \uv{Nové organon} z
    roku 1620 reaguje na Aristotelovo dílo ,,Organon", odmítá čistě spekulativní, scholastickou
    aristotelovskou logiku a vychází z empirického, smyslového poznání, pozorování a pokusů. Je
    zakladatelem vědecké indukce, tedy metody, která logicky analyzuje a třídí zkušenosti, fakta a
    dospívá k obecným zákonitostem. Přitom se vědec musí oprostit od předsudků a vžitých představ,
    které Bacon nazývá  \uv{idoly}. Ve svém zaujetí pro pokusy šel Bacon tak daleko, že zemřel na
    zápal plic právě když zkoumal dlouhodobý vliv chladu na živý organismus. Bacon je představitelem
    anglického empirismu, který zapůsobil i na anglické fyziky včetně Newtona.  
    
    Ve Francii ovlivnil filozofické myšlení především Descartes (latinsky Kartesius). Pocházel z
    aristokratického katolického rodu, od dětství byl chabého zdraví a prošel složitým myšlenkovým
    vývojem. Navštěvoval jezuitskou kolej, studium ho však neuspokojilo, a naopak v něm rozvířilo
    mnoho pochyb. Studoval práva i medicínu, jako dobrovolník v holandském a pak v bavorském vojsku
    prošel Evropou i Čechami a někdy se uvádí, že se účastnil i bitvy na Bílé hoře.  Na dlouhých
    dvacet let pak zakotvil v Holandsku, kde se v červenci 1642 sešel i s Janem Amosem Komenským, i
    když se s ním filozoficky nepohodl. Descartovy názory narážely na odpor a vyvolávaly útoky ze
    strany jak katolických, tak protestantských kruhů a tyto útoky poněkud plachého Descarta
    deprimovaly. Descartes byl zastáncem Koperníkova názoru na sluneční soustavu, ale po Galileově
    odsouzení se zalekl a byl ve formulaci svých názorů vysloveně opatrný. Aby si zajistil větší
    klid k práci, často dokonce měnil místo svého pobytu. Jeho vědecké dílo mělo i řadu stoupenců a
    vzbudilo nakonec zájem švédské královny Kristýny. Pozvala Descarta do Stockholmu a ten ji musel
    vyučovat filozofii třikrát týdně od pěti hodin ráno. Descartes, který byl zvyklý vstávat až k
    poledni, takový režim, znásobený drsným severským podnebím, ovšem dlouho nepřežil. V únoru 1650
    zemřel na zápal plic a v r. 1666 byly jeho ostatky převezeny do Paříže. Dnes je pohřben ve
    starobylém kostele Saint Germain-des-Prés, jeho lebka, která byla při převozu ostatků zcizena,
    odděleně v Museu člověka v Paříži. 
    
    Descartes je zakladatelem francouzského racionalismu. Je znám jeho výrok \uv{Cogito erg sum},
    \uv{Myslím, tedy jsem} a na základě rozumových úvah také založil svou vědeckou metodu. Ve svém
    slavném spise \uv{Rozprava o metodě} stanoví pravidla správného vědeckého uvažování. Jako první
    krok požaduje zpochybnit všechny dosavadní názory a tvrzení, pokud nejsou nade vší pochybnosti
    Descartes je zakladatelem francouzského \emph{racionalizmu}. Je znám jeho v rok \uv{Cogito ergo
    sum}, \uv{Myslím, tedy jsem} a na základě rozumových úvah také založil svou vědeckou metodu. Ve
    svém slavném spise \uv{Rozprava o metodě} stanoví pravidla správného vědeckého uvažování. Jako
    první krok požaduje zpochybnit všechny dosavadní názory a tvrzení, pokud nejsou nade vší
    pochybnost dokázány. Jeho \uv{De omnibus dubitandum}, \uv{O všem pochybovat}, znamená začínat
    zkoumání s čistou a nepředpojatou myslí. Dále požaduje rozdělit každou zkoumanou otázku na
    části, které by bylo možno lépe řešit. Při zkoumání je třeba postupovat od předmětů
    jednodušších, které lze snáze poznávat, ke složitějším. A konečně za čtvrté je třeba uspořádávat
    zjištěná fakta do výčtů a přehledů, aby nic nebylo opomenuto. Tato Descartova doporučení jsou
    jakýmsi základem vědecké metody rozumového zkoumání; týmž způsobem musí ostatně postupovat i
    detektiv při řešení složitého kriminálního případu. Descartes je tak zakladatelem analytické
    deduktivní metody, která vychází z několika málo obecných principů a zákonů a postupuje podle
    pravidel rozumového uvažování.

    Descartova filozofie, karteziánství, ovlivnila celou řadu pozdějších filozofů a myslitelů.
    Patřil k nim např. \textsc{Benedikt (Baruch) Spinoza} (1632-1677), holandský filozof
    portugalsko-židovského původu, ale i Leibniz, Pascal a další. Spinoza se pokusil pomocí
    Descartovy racionalistické filozofie a axiomatické metody geometrie vyložit i taková témata,
    jako je politika, etika nebo teologie. Dochází k závěru, že existuje jen jedna jediná substance,
    jíž je Bůh ztotožněný s přírodou. Takový názor, podle něhož se nic a nikdo do přírody zvnějšku
    nevměšuje se nazývá \emph{panteizmem}. Zmiňujeme se o něm proto, že je blízký chápání velkých
    fyziků. Ti byli uchváceni krásou a řádem přírody, a ta jim splývala s božstvím v jedno. Ke
    Spinozově panteizmu se hlásil např. i Einstein \cite[s.~141]{Stoll2009}.    
    
  \section{Integrační tendence ve fyzice}\label{fyz:IchapIsecVIII}
    Není to tak dávno, co se fyzikové dělili na dvě velké skupiny – experimentátory a teoretiky.
    Příslušník každé skupiny věděl, že se bez členů druhé skupiny neobejde. Výsledkem byla plodná
    spolupráce plná zdánlivé řevnivosti a úsměvných historek. S nástupem výpočetní techniky se vše
    změnilo. Postupně vznikala skupina třetí, která se zabývá numerickými simulacemi. Bez nich si
    dnes fyziku nedovedeme představit. Numerické simulace umožňují první ověření výsledků nových
    teorií bez nákladných experimentů. Při zpracování experimentálních dat pomáhají hledat procesy,
    které se za naměřenými údaji skrývají. V současnosti má fyzika tři nedílné celky: teorii,
    experiment a numerické simulace. 
    
    Fyzika zaznamenává v průběhu staletí dvě základní tendence. První z nich je postupné členění na
    další a další podobory. Tento vývoj souvisí s prohlubujícím se poznáním a je přirozenou cestou v
    každé vědní disciplíně. Postupně vznikají specialisté na stále užší a užší obory, vytvářejí si
    svůj vlastní vědecký jazyk a schopnost komunikace odborníků z dříve blízkých oblastí fyziky se
    stále zhoršuje. Na druhé straně dochází k hlubšímu pochopení souvislostí mezi jednotlivými
    částmi fyziky a k jejich postupnému sjednocování do univerzálnějších teorií. Možná se jednou
    podaří sjednotit fyzikální pohled na všechny základní přírodní interakce do jedné jediné teorie,
    kterou dnes nazýváme Teorie všeho (anglicky TOE, Theory Of Everything). Tyto integrační tendence
    ve fyzice jsou znázorněny na obrázku 1. 

    \luagraphic[1]{fyz_fig0924.pdf}{Integrační tendence ve fyzice.
    (\cite[s.~12]{Kulhanek2019})}{fyz:fig0924}
    
    Mechanika jakožto vědecká fyzikální disciplína vznikala od 17. století. První známější vědecké
    experimenty prováděl \textsc{Galileo Galilei} (1564–1642). Teoretickou konstrukci klasické
    mechaniky, jakožto nástroje pro předpověď pohybu těles v daném silovém poli, navrhnul
    \textsc{Isaac Newton} (1642–1727) ve svých \emph{Principiích (Philosophiæ Naturalis Principia
    Mathematica)} z roku 1687. V 18. století dovršil konstrukci klasické mechaniky \textsc{Joseph
    Louis Lagrange} (1736–1813), který mechanické úlohy formuloval nezávisle na volbě souřadnicové
    soustavy za pomoci variačního počtu.
    
    V 19. století se úspěšně dařilo poznávat a postupně chápat elektrické a magnetické děje. Na
    experimentech se podílela celá řada významných fyziků, například \textsc{Hans Oersted}
    (1777–1881), \textsc{André Ampère} (1775–1836), \textsc{ichael Faraday} (1791–1867),
    \textsc{Heinrich Hertz} (1857–1894), \textsc{Oliver Heaviside} (1850–1925) a další. Celé toto
    údobí vyvrcholilo poznáním, že jevy elektrické a magnetické mají shodnou povahu a společný
    původ. V roce 1873 publikoval \textsc{James Clerk Maxwell} (1831–1879) pojednání \uv{A Treatise
    on Electricity and Magnetism}, které obsahovalo rovnice, jež završily klasickou elektrodynamiku
    do jednoho jediného celku obsahujícího jak děje elektrické, tak magnetické. 
    
    Na konci 19. století podlehlo mnoho fyziků iluzi, že fyzika jako věda je dokončena. Byly známy
    zákony mechaniky na jedné straně a zákony elektřiny a magnetizmu na straně druhé. Na první
    pohled se zdálo, že veškeré přírodní děje jsou důsledkem těchto dvou vědních disciplin a
    budoucnost fyziky je pouze v aplikaci známých zákonů na neznámé situace. Šlo samozřejmě o krutý
    omyl, který se rychle projevil na počátku dvacátého století, kdy nebylo možné tehdejšími
    znalostmi vysvětlit řadu fyzikálních dějů. 

    Ukázalo se, že jak klasická mechanika, tak klasická elektrodynamika nedokáží uspokojivě popsat
    svět na úrovni atomů. Důsledkem toho byla neschopnost objasnit chování elektronu v atomárním
    obalu, vysvětlit záření absolutně černého tělesa, pochopit fotoelektrický jev a smířit se s
    projevy objektů mikrosvěta, které vykazovaly někdy částicové a jindy vlnové vlastnosti. Zrodila
    se kvantová mechanika, ve které neplatí \(ab = ba\), a nekomutativnost se stala nově objeveným
    rysem přírody na mikroskopické úrovni. Kvantová mechanika s sebou přinesla celou řadu těžko
    představitelných jevů – kvantování energie a momentu hybnosti, dualismus vln a částic, relace
    neurčitosti, nejednoznačnost aktu měření a pravděpodobnostní interpretaci výsledků vedoucí na
    nedeterminizmus kvantové fyziky.

    A to byl teprve začátek. Spin elementárních částic objevený v roce 1925 znamenal další výrazný
    posun lidstva v chápání přírody. Je důsledkem relativistické fyziky, která se od počátku 20.
    století rozvíjela paralelně s kvantovou mechanikou. Spojení kvantové mechaniky se speciální
    relativitou vedlo na Diracovu rovnici, která se stala základem kvantového popisu pohybu
    elektronu. \textsc{Paul Adrien Maurice Dirac} (1902–1984) navrhnul svou rovnici v roce 1928 a
    téhož roku z ní odvodil existenci pozitronu, antičástice k elektronu. Pozitron byl
    experimentálně objeven až o 4 roky později \textsc{Carlem Andersonem} (1905–1991). Za svou práci
    získal Dirac Nobelovu cenu za fyziku pro rok 1933. V letech 1946 až 1949 byla dokončena první
    kvantově polní teorie – \emph{kvantová teorie elektromagnetického pole}, které dnes říkáme
    \textbf{kvantová elektrodynamika} (\emph{QED, Quantum Electro-Dynamics}). Za její formulaci
    získali Nobelovu cenu za fyziku pro rok 1965 \textsc{Richard Feynman} (1918–1988),
    \textsc{Shin-Itiro Tomonaga} (1906–1979) a \textsc{Julian Schwinger} (1918–1994). Kvantová
    elektrodynamika je kvantovou analogií Maxwellových rovnic. Elektromagnetická interakce je
    způsobena polními částicemi, v tomto případě fotony, které si mezi sebou posílají nabité
    částice. Klasický pojem síly ztrácí svůj smysl. Feynmanovi se podařilo složité rovnice
    interpretovat za pomoci názorných grafů, kterým dnes říkáme \emph{Feynmanovy diagramy}. Na
    obdobném základě byla později vytvořena také současná \textbf{kvantová teorie slabé a silné
    interakce}. Základním rysem těchto teorií jsou tzv. \emph{kalibrační symetrie}, které předurčují
    způsob působení dané interakce na elementární částice.

    Od počátku 60. let probíhaly snahy o spojení elektromagnetické a slabé interakce do jednoho
    jediného celku. Podařilo se to \textsc{Stevenu Weinbergovi} (1933), \textsc{Abdusu Salamovi}
    (1926–1996) a \textsc{Sheldonu Glashowovi} (1932). Za svou práci získali Nobelovu cenu za fyziku
    pro rok 1979. Jimi předpovězené polní částice slabé interakce \(W^+\), \(W^–\) a \(Z^0\) byly
    objeveny na přelomu let 1983 a 1984 v evropském středisku jaderného výzkumu CERN. Jejich
    objevitelé, \textsc{Carlo Rubbia} (1934) a \textsc{Simon van der Meer} (1925–2011) získali
    Nobelovu cenu ještě téhož roku (1984).

    K pochopení silné interakce přispěl již ve 30. letech japonský fyzik \textsc{Hideki Yuakawa}
    (1907–1981). Za svou práci získal Nobelovu cenu za fyziku pro rok 1949. Současná kvantově polní
    teorie silné interakce se nazývá \textbf{kvantová chromodynamika} (\emph{QCD, Quantum
    Chromo-Dynamics}) a za její formulaci a zejména za objev asymptotické volnosti silné interakce
    kvarků a gluonů získali Nobelovu cenu za fyziku pro rok 2004 \textsc{Frank Wilczek} (1951),
    \textsc{David Gross} (1941) a \textsc{David Politzer} (1949).
    
    Kvantová mechanika slavila v průběhu 20. století mimořádné úspěchy. Jednoduchá teorie popisující
    mechanické děje postupně přerostla v polní kvantovou teorii schopnou úspěšně popsat hned tři ze
    čtyř základních přírodních interakcí. Tato cesta se samozřejmě neobešla bez potíží a problémů,
    nicméně vyústila v dnešní \textbf{standardní model elementárních částic a interakcí}. Bez
    kvantové teorie a hlubokého pochopení zákonitostí mikrosvěta bychom dnes neměli ani počítače ani
    jinou elektroniku. 

    Na počátku 20. století ale vznikala ještě jedna, neméně úspěšná teorie – obecná relativita. Z
    Maxwellovy elektrodynamiky plynulo, že rychlost světla by ve vakuu měla být univerzální
    konstantou a že by se neměla sčítat s rychlostí zdroje elektromagnetického vlnění. Tento
    výsledek byl na první pohled v rozporu s klasickou mechanikou, ve které se rychlost zdroje s
    rychlostí signálu sčítá. Řada experimentů potvrdila správnost elektrodynamiky. Bylo tedy třeba
    přeformulovat mechaniku tak, aby byla v souladu s Maxwellovou elektrodynamikou. To se v roce
    1905 podařilo Albertu Einsteinovi v rámci tzv. \textbf{speciální teorie relativity}. Daň za
    sjednocení obou teorií byla veliká. Čas spolu s prostorem přestaly být absolutní. Délka letící
    tyče a časový úsek mezi dvěma událostmi ve skutečnosti závisejí na volbě souřadnicové soustavy
    pozorovatele.
    
    Einsteinovy snahy o zobecnění speciální relativity na neinerciální souřadnicové soustavy vedly v
    roce 1915 ke vzniku obecné relativity – zcela nové teorie gravitace, která popisuje tuto
    interakci za pomoci zakřiveného času a prostoru. Za základ nové teorie lze chápat dvě myšlenky:
    \begin{itemize}[noitemsep]
      \item každé těleso svou přítomností zakřivuje časoprostor kolem sebe;
      \item každé těleso se v tomto zakřiveném časoprostoru pohybuje po nejrovnějších možných
            drahách – tzv. \emph{geodetikách}.
    \end{itemize}

    Nové chápání času a prostoru bylo zcela revoluční. Samotná tělesa se podílejí na vytváření času
    a prostoru, bez nich by čas a prostor neexistoval. Otázka, jak by vypadal vesmír bez přítomnosti
    těles, přestává mít smysl.
    
    Fyzika dvacátého století se tak stala v jistém smyslu poněkud schizofrenní. Tři ze čtyř
    interakcí jsou popsány za pomoci výměnných (polních) částic v rámci kvantové teorie pole. A
    jedna interakce, gravitační, je popsána za pomoci pokřiveného světa obecné teorie relativity.
    Vyřešení mnoha fyzikálních hádanek s sebou přineslo ještě větší záhady. Existuje jednotná teorie
    všech čtyř interakcí? Je možné spojit kvantovou teorii a obecnou relativitu do jedné jediné
    teorie? Odpověď na tyto otázky zatím neznáme. Velké úspěchy slaví různé strunové teorie, ve
    kterých jsou částice chápány jako jednorozměrné kmitající útvary ve vícerozměrném světě, ale zda
    jde o krok správným směrem či nikoli, není v tuto chvíli jasné. V roce 2010 se objevila hypotéza
    holandského fyzika Erika Verlindeho, podle které by gravitace nemusela být skutečnou silou, ale
    jen statistickým projevem růstu entropie v mikrosvětě. Těžko odhadnout, zda tato odvážná
    myšlenka najde podporu v dalších experimentech, nebo jde o slepou uličku.
    
    Pokud vás zajímají základní vlastnosti přírody a jejich teoretický popis, je třeba v první řadě
    začít se studiem klasické mechaniky, na kterou úzce navazuje mechanika kvantová. Další studium
    polních problémů zase není možné bez znalosti statistické fyziky \cite[s.~14]{Kulhanek2019}. 