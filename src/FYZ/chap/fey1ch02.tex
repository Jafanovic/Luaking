% !TeX program = lualatex
% !TeX root = luaking.tex
% !TeX encoding = UTF-8
% !TeX spellcheck = cs_CZ
%---------------------------------------------------------------------------------------------------
% file fey1ch02.tex
%---------------------------------------------------------------------------------------------------
%===================== Kapitola: Dějiny fyziky =====================================================
\setchaptertoc
\chapter{Dějiny fyziky}\label{fyz:IchapII} 
  \section{Člověk a příroda}\label{fyz:IchapIIsecI}
    \subsection{Co je fyzika}\label{fyz:IchapIIsecIssecI}
      Řecké slovo \emph{„fýsis“} znamená \emph{„příroda“}, takže fyzika je vlastně věda o přírodě.
      Toto označení vzniklo v době, kdy vědy o přírodě nebyly ještě rozlišeny. Dnes se zkoumáním
      přírody zabývá i chemie, biologie, ale třeba i mineralogie, geologie, botanika, zoologie a
      další. Tyto vědy jsou ovšem zaměřeny vždy na určitou oblast přírodních jevů. Chemii zajímá,
      jak se atomy prvků spojují v molekuly sloučenin, biologii zas projevy živé hmoty, která je,
      pokud zatím víme, omezena jen na naši planetu. Někdy je těžké rozhodnout, zda nějaký jev patří
      do fyziky, nebo do jiné přírodní vědy - a tak máme i fyzikální chemii, biofyziku, geofyziku,
      astrofyziku a jiné. 
      
      Název fyzika v dnešním smyslu se začal používat až v 19. století. Dříve byla: fyzika
      považována za součást filozofie a používal se pro ni název „přírodní filozofie“. Ten ostatně
      dosud přežívá v některých západních jazycích. Slovem „fysik“, „fysikus“ se zpravidla rozuměl
      lékař v úřední funkci hygienika. ještě dnes někteří anglicky hovořící mluvčí zaměňují ve svém
      jazyce výraz \emph{„physicist“} (fyzik) a \emph{„physician“} (lékař), což někdy vede k
      humorným situacím. 
      
      Fyzika zkoumá obecné vlastnosti přírody a zákonitosti společné všem oblastem přírodních jevů.
      Zabývá se strukturou a vývojem vesmíru, vlastnostmi částic, z nichž se skládají tělesa,
      planety a hvězdy, vlastnostmi polí, jimiž tyto částice mezi sebou vzájemně působí. Gravitační
      pole zprostředkuje přitažlivou sílu mezi Sluncem a planetami a vlastně vládne celým vesmírem,
      ale ovlivňuje i růst rostlin a krevní oběh člověka. Podobně elektrické pole vyvolává blesky,
      ale je i základem síly chemických výbuchů a umožňuje přenos iontů mezibuněčnými membránami a
      podmiňuje i sílu našich svalů. 

      Jiné přírodní vědy jako chemie, biologie nebo lékařství využívají fyzikální zákony, fyzikální
      přístroje a měřicí či diagnostické metody. Moderní technika a technologie, průmyslová výroba,
      doprava a komunikace jsou založeny na využití fyzikálních principů. A to už nemluvíme o
      uplatnění počítačů, jejichž pavoučí sítě stále více opřádají zeměkouli a blízký kosmos. V
      tomto smyslu je tedy fyzika základní, nechceme-li říci přímo nejdůležitější, přírodní vědou. 
      
      To nás přivádí k otázce, co je to vlastně \emph{věda}. Věda je jistě určitý \emph{způsob
      poznávání světa}. Vychází z přesvědčení, že tento svět existuje nezávisle na nás, že
      existoval, když jsme tu ještě nebyli, a bude existovat, až tu nebudeme, a že je poznatelný. I
      když není poznatelný úplně a do všech podrobností a i když naše poznání je vlastně nikdy
      nekončící proces doprovázený úspěchy i omyly - v tom je i určitá romantika vědy.  
      
      Poznaný svět, to je jakási kopie, odraz vnějšího světa v naší mysli. Abychom mohli s tímto
      odrazem pracovat, musíme vnější svět nejen poznat, ale i pochopit a naučit se události v něm
      předvídat. Krásným příkladem je skutečnost, že se astronomie naučila přesné určovat postavení
      planet, příchody komet nebo zatmění Slunce a Měsíce v minulosti i v budoucnosti. To ovšem
      předpokládá že ve světě platí \emph{přírodní zákony}, které můžeme odhalit rozumově postihnout
      a matematicky vyjádřit. Věda nemůže připustit zázraky: kdyby nějaká vyšší moc měnila zákony
      volného pádu podle své libovůle, nemohla by existovat fyzika ani technika. Věda je tedy
      rozumové poznávání světa a jeho zákonů pomocí logického a matematického uvažování. Správnost
      svých předpovědí musí věda experimentálně prakticky ověřovat.

      Neznamená to ovšem, že všechno se dá předem vypočítat, že vše je předurčeno, že do hry nemůže
      vstoupit náhoda. Především nemůžeme nikdy znát hodnoty změřených fyzikálních veličin s
      naprostou přesností. Pověstná astronomická přesnost nám sice umožňuje vypočítat okamžik
      zatmění Slunce na deset milionů let dopředu, ale už ne na sto milionů let, ať použijeme
      sebevýkonnější počítač. Víme, jak usilovně se dnes astrofyzikové snaží určit dráhy malých
      nebeských těles, planetek a komet, aby mohli včas varovat před jejich srážkou se Zemí.

      Stejně tak nemůžeme sledovat náhodné pohyby každé jednotlivé molekuly plynu a musíme se naučit
      zkoumat i statistické zákonitosti a vlastnosti chaosu. A konečně moderní kvantová fyzika
      dokázala matematicky popsat podivuhodné vlastnosti částic mikrosvěta, které se chovají jako
      vlny a řídí se zákony pravděpodobnosti. Fyzika tady není jen suchá, matematicky strohá věda,
      která umí všechno předvídat, ale počítá i s náhodou. Náhoda je ostatně i kořením života -
      můžeme najít poklad, nebo nám ale také může na hlavu spadnout meteorit

      Vedle vědy mohou existovat i jiné způsoby poznávání světa. Především je to umění, které je
      charakteristickým projevem vyspělého člověka našeho druhu a jedním z pilířů kultury. Poznání
      může nabývat i podobu mytologie a náboženství: lidé si tak vysvětlovali záhadné a
      nepochopitelné přírodní jevy. Není to poznání rozumové, ale spíše duchovní a emotivní,
      případně založené na víře, a je třeba ho od vědeckého poznání odlišovat. Ale mytologické
      představy starých národů vycházely z tisíciletých zkušeností a pozorování, představovaly
      přírodní děje ve fantastické a poetické podobě, a maji nám i dnes co říci. Ani fyzika se
      neobejde bez intuice a fantazie.

      \luagraphic[1]{fyz_fig0939.jpg}{\wikiDavidTeniersYounger, Interiér laboratoře s alchymistou
        17. století. Téma lékařů a alchymistů bylo ve vlámském umění v 17. století velmi populární.
        David Teniers mladší byl hlavním přispěvatelem k tomuto žánru ve Vlámsku. Kredit:
        Wikipedia}{fyz:fig0939}

      Existují ovšem i vysloveně falešné přístupy k světu, které využívají a zneužívají lidskou
      pověrčivost a důvěřivost. Označujeme je jako \emph{pseudovědy}. Jako příklad za mnohé můžeme
      uvést astrologii, přesvědčení o tom, že postavení hvězd v okamžiku narození má vliv na lidské
      osudy. Pseudovědy existovaly od pradávna, často odvozují svůj původ od mystických učení
      východních národů a v jistém smyslu ovlivnily i vývoj vědy. Tak astrologie šla po staletí ruku
      v ruce s astronomii (stará čeština rozlišovala \emph{„hvězdář"} astronom, a \emph{„hvězdník"}
      astrolog), a protože byla mocnými finančně podporována, přispěla tak i k rozvoji pozorovací
      astronomie. Úsilí alchymistů o nalezení \emph{\uv{kamene mudrců}} (obr. \ref{fyz:fig0939}) a
      výrobu zlata z obyčejných kovů přispělo ke zdokonalení chemických laboratorních operací, a
      dokonce k objevu některých nových prvků. Ostatně výrobu zlata z jiných kovů zvládla až moderní
      jaderná fyzika; taková výroba se ovšem nevyplatí.

      Výsledky moderní vědy, zejména fyziky, jsou bez dlouhého soustavného studia a bez zvládnutí
      matematiky málo srozumitelné, podobně jako pohled nezasvěceného do notové partitury symfonie.
      Je jistě lákavější a více vzrušující číst o záhadných, nevysvětlitelných a nadpřirozených
      jevech. Víru těch, kdo jsou o nevědeckých představách přesvědčeni, nelze vyvrátit rozumovými
      důvody. Tito lidé jsou dokonce svým způsobem šťastni, zejména v případech, kdy věda ještě
      nemůže nabídnout racionální vysvětlení nebo pomoc. Odmyslíme-li si zištné šarlatány a
      podvodníky, ohrožující majetek nebo zdraví svých klientů, mohou být některé pověry docela
      neškodné. Věří-li někdo ve svůj horoskop nebo šťastné dny, nic proti tomu, pokud ovšem
      neobětuje v takový den všechny své úspory do loterie v přesvědčení, že musí vyhrát.

      Známý britský filozof rakouského původu \textsc{Karl Popper} navrhl k rozlišení věd od
      nevědeckosti \emph{„princip falzifikace"}. Vědecké poznání může být dalším vývojem, novými
      experimenty nebo výpočty opraveno a nahrazeno novým, přesnějším. Aristoteles se domníval, že
      světlo se šiří okamžitě, nekonečnou rychlosti Galileo se pokusil tento názor vyvrátit,
      „falzifikovat", a dnes víme přesně, jakou rychlostí se světlo šiří. Pro vědu je
      charakteristické, že si je vědoma své ohraničenosti, že zná hranici mezi věděním a nevěděním.
      Byl to právě Newton, jeden z největších vědců, který jako jeden z prvních dokázal říci
      \emph{„nevím"}, nepodložené domněnky si nevymýšlím, odpověď na nevyřešené otázky přenechávám
      dalším pokolením.

      \textbf{Věda je tedy historický proces soustavného rozumového poznávání světa, vytváření jeho
      vědeckého obrazu. Přitom věda ověřuje své poznatky, sama sebe opravuje a je vždy připravena
      připustit omyl nebo nepřesnost svého poznání. \cite[s.~15]{Stoll2009}}

      \begin{tcnote}
        \textbf{Vědecká metoda} je posloupnost nebo sada procesů, používaných při vědeckém výzkumu.
        Cílem je získat znalosti a vědomosti pomocí pozorování a dedukce na základě dosud známých
        poznatků. Přijímání nových vědeckých poznatků je založeno na konkrétních důkazech. Vědecká
        metoda je založena na předpokladu, že kritériem pravdivosti vědecké hypotézy je souhlas
        předpovědí s výsledky výzkumu. Tento přístup udržuje vědecké \emph{hypotézy} v neustálém
        kontaktu s realitou a umožňuje jejich \emph{falzifikaci}, neboť hypotéza, jejíž důsledky
        jsou v rozporu s výzkumnými zjištěními, bude falzifikována. Mnohokrát ověřená hypotéza,
        kterou se zatím nepovedlo vyvrátit, se stává vědeckou teorií. Důsledkem je omezení vědy na
        otázky a hypotézy, jež jsou alespoň v principu rozhodnutelné pozorováním. Větší vědecký tým
        je však spíše konzervativní, což brzdí výzkum a vývoj.

        {\centering
        \captionsetup{type=figure}
        \luafigure[0.7]{fyz_fig0950.jpg}
        \captionof{figure}{\wikiPopper \textasteriskcentered	28. července 1902, Vídeň \textdagger
          17. září 1994 (ve věku 92 let). Byl významným představitelem moderního liberalismu, teorie
          vědy a filosofie.Kredit: Wikipedia}
        \label{fyz:fig0950}
        \par}

        \textbf{Falzifikovatelnost} (vyvratitelnost nebo zpochybnitelnost) je ve filosofii vědy
        vlastnost takového vědeckého tvrzení, hypotézy nebo teorie, kterou je principiálně možné
        vyvrátit, například experimentem. Jinak řečeno, tvrzení nebo teorie je falzifikovatelná
        tehdy, pokud víme, jak by se dala vyvrátit čili negovat. Teorie, kterou experimenty
        potvrzují, sice platí, ale jen dokud ji nějaký experiment nevyvrátí. Problémem
        falzifikovatelnosti se především zabýval rakouský filozof kritického racionalismu
        \textsc{Karl R. Popper}, který zdůraznil, že žádný počet pokusných potvrzení nemůže vědeckou
        teorii definitivně dokázat. Na rozdíl od verifikace, která je vždy jen částečná, pouze
        falzifikace může být definitivní.

        Ve skutečnosti se jedná o podmínku testovatelnosti a vyvratitelnosti hypotéz a teorií.
        Princip, který takto formuloval, se podle něj nazývá \textbf{Popperova břitva} - „Vědecké
        teorie jsou ověřitelné. Ověřitelné teorie je možné na základě ověřovacího postupu zamítnout
        (a nahradit teoriemi jinými).“

        \tcblower
        Pravdivost vědecké teorie podle něj nelze dokazovat, ale jen empiricky testovat. Základem
        vědeckého poznání tedy není verifikace (potvrzení), ale falsifikace. Pouze ta teorie, kterou
        je možné podrobit falsifikaci, tedy vystavit ji možnosti vyvrácení, je vědecká, tím větší
        hodnotu má pro vědu. Netřeba hromadit důkazy, které teorii potvrzují; spíše hlídat to, co by
        ji mohlo vyvrátit. Konečnou, definitivní jistotu naší přesné znalosti pravdy nemůžeme mít
        nikdy, k pravdě se můžeme pouze přibližovat neustálým vylučováním falsifikovaných teorií,
        hovoříme o evoluci vědy. K evoluci, vývoji vědy dochází právě díky falzifikaci: tím, že něco
        popřeme, získáváme nový prostor pro otevření dalších, nových otázek.

        Důsledkem jeho pojetí vědeckého poznání je obrana otevřeného myšlení a otevřené společnosti.
        Síla vědy netkví tolik v tom, že se její tvrzení dají dokázat, nýbrž v tom, že musí být
        formulována tak, aby se dala vyvrátit. Právě tak síla demokracie nespočívá v tom, že by
        vybírala ty nejlepší k vládě, nýbrž že každou vládu lze běžnými prostředky (volbami) odvolat
      \end{tcnote}

      Jakými způsoby ale fyzika přírodu poznává a jak ověřuje správnost svého poznání? Poznání
      začalo praktickou zkušenosti. Tisíce let předtím, než Galileo odhalil zákony volného pádu a
      šikmého vrhu těles, věděl pravěký člověk, jak má hodit kámen, oštěp nebo vystřelit šíp, aby
      zasáhl kořist. Používal úspěšně bumerang, jehož let dnes vyžaduje náročný matematický popis.
      Aniž by znal zákon lomu světelného paprsku vycházejícího z vody do vzduchu, věděl, kam má
      namířit harpunu, aby zasáhl rybu. Při výrobě nástrojů z kamene používal znalosti o jeho
      mechanických vlastnostech, tvrdosti, křehkosti, štěpitelnosti. Všechny tyto znalosti a
      poznatky byly vyvolány životní nutností, předávaly se od pokolení k pokolení a člověk k jejich
      získávání měl miliony let času. Můžeme si o nich udělat dobrou představu ještě dnes studiem
      života izolovaných primitivních národů.

      \luagraphicx[1]{fyz_fig0943c.jpg}{Zdeněk Burian: Tábor pozdně paleolitických lovců:
        \textbf{Paleolit} (z řeckého \foreignlanguage{greek}{πάλαιος} palaios - „starý“ a
        \foreignlanguage{greek}{λίθος} lithos - kámen) neboli starší doba kamenná je označení
        nejstaršího a nejdelšího (\protect\SI{99}{\protect\percent}) období lidských dějin. Paleolit začal v době,
        kdy se člověk zručný poprvé naučil užívat nástrojů, a končí s poslední dobou ledovou, kdy na
        něj navazuje mezolit, který se odlišuje adaptací člověka na klimatické podmínky v podstatě
        současného rázu. Geologicky toto období spadá do \emph{pleistocénu}, a proto je pro něj
        typické střídání dob ledových a meziledových (pleistocénní klimatický cyklus) a adaptibilita
        rodu homo (člověk) na měnící se klimatické podmínky. V Evropě bylo tou dobou velmi chladné
        podnebí. Hlavním způsobem získávání potravy byl tedy lov a sběr plodin.}{fyz:fig0943c}

      Takové poznáváni přírody bylo od počátku spojeno s praxí, dalo by se říci s technikou. Člověk
      nebyl pasivním pozorovatelem přírody, ale aktivně na ni působil, i když ještě nemohl přírodní
      rovnováhu nijak vážněji ovlivnit. Tříbil si přitom důvtip a občas se mu podařilo přírodu nějak
      obelstít, usnadnit si práci. Poznal účinnost páky, odstředivé sily, pružnost lučiště, strojil
      zvěři důmyslné pasti. Řecké slovo „techné" znamená řemeslo, dílo lidských rukou, ale také lest
      nebo úskok. I dnes se snažíme pomocí techniky nad přírodou s větším nebo menším úspěchem
      vyzrát. „Fýsis" a „techné" tak vždy kráčely ruku v ruce. 

      U zdrojů fyziky stojí však i další lidská vlastnost, která se zrodila někdy během dlouhého
      vývoje - přirozená zvídavost. Člověk si dokázal přírodního jevu všimnout, podivit se mu a
      zvědavě ho pozorovat. Nedovedl ho ještě vysvětlit, a tak zapojoval svou fantazii, vytvářel
      antropomorfní představy, mýty a náboženské rituály. Údiv a zvídavost jsou tedy vedle
      zkušenosti prvním ze zdrojů poznání. 

      Systematickým, dlouhodobým pozorováním můžeme odhalit určité pravidelnosti, zákonitosti,
      například sled fázi Měsíce, slunečních nebo měsíčních zatmění. Pozorované pravidelnosti se
      můžeme pokusit vyjádřit v podobě přírodního zákona, a dokonce ho popsat matematicky. To je
      ovšem jen první etapa fyzikálního poznání; vytváříme vědeckou \textbf{hypotézu}, domněnku. Z
      ní pak můžeme vyvodit předpověď, která ovšem musí být teprve ověřena. Abychom ji ověřili,
      musíme provést experiment, jak se říká \emph{„položit přírodě otázku"}. Kdysi to bývalo
      vyjádřeno i domýšlivě drasticky - položit přírodu na skřipec a vypáčit z ní odpověď! 

      K fyzikálnímu experimentu je ovšem dlouhá cesta. Musíme zavést pojem fyzikální veličiny,
      kterou chceme měřit, zkonstruovat měřici přístroje a metody. Těžko bychom mohli odhalit zákony
      tepelných jevů, kdybychom neměli k dispozici teploměry, tlakoměry a kalorimetry. Jediný
      experiment ovšem nestačí, musíme si být jisti, že námi objevený zákon bude platit za stejných
      podmínek vždy opakovaně a stejně. 

      Nejpřesvědčivějším důkazem správnosti našeho poznání je, podaří-li se ho uplatnit v technické
      praxi. Dnes už například nikdo soudný nemůže pochybovat o správnosti Einsteinovy speciální
      teorie relativity, protože na jejím základě fungují nesčetné urychlovače, elektronické
      přístroje a v podstatě celá elektrotechnika, nechceme-li už mluvit o jaderné energetice.
      Naproti tomu obecná teorie relativity popisuje gravitační pole ve vesmíru a tam jsme ovšem
      odkázáni jen na přesná a důmyslná pozorovaní. S vesmírem zatím ještě experimentovat
      nedovedeme, gravitační sily ovládat neumíme.

      \begin{tcnote}
        Můžeme tedy říci, že fyzika je \emph{základní věda} o nejobecnějších vlastnostech přírodních
        objektů a zákonitostech přírodních jevů, která vychází z pozorování, zkušenosti a
        experimentů, jejich výsledky zpracovává matematicky a své výpočty a teorie systematicky
        experimentálně ověřuje. Výsledky fyzikálního poznání slouží lidstvu v jeho technické a
        společenské praxi a z této praxe čerpá fyzika opět nové podněty a prostředky ke svému
        výzkumu. \cite[s.~16]{Stoll2009} 
      \end{tcnote}
    
    \subsection{Vývoj fyziky}\label{fyz:IchapIIsecIssecII}     
      Věda v obecnosti, a fyzika zejména, se v průběhu dějin vyvíjí a mění. Proces fyzikálního
      poznávání, jakmile byl jednou spuštěn a stal se nedílnou součástí lidské kultury, pokračuje
      bez ustání vpřed a nemůže být nikdy vyčerpán, zastaven nebo ukončen. Přitom neprobíhá stále
      stejně rychle, zná období trpělivého shromažďování poznatků, dokonce období stagnace a úpadku,
      a pak zase rychlého bouřlivého postupu vpřed, kdy mluvíme o vědecké revoluci. V průběhu dějin
      lidské společnosti vzniká čas od času naléhavá potřeba a poptávka po výsledcích a uplatnění
      vědeckého výzkumu. Takovými podněty byly např. rozvoj mořeplavby a zámořských cest, rozvoj
      dopravy, komunikaci a průmyslu, aeronautiky a kosmonautiky a bohužel příliš často i války.

      Fyzika má mezinárodní charakter, dochází k přejímání a zprostředkovávání nových poznatků mezi
      vzdálenými národy a kulturami. Vnějším projevem této skutečnosti je i to, že si fyzika vždy
      vyžadovala mezinárodní jazyk bez ohledu na hranice států. Tak na přelomu našeho letopočtu byla
      takovým jazykem ve Středomoří helénistická řečtina, ve středověké Evropě latina a v dnešní
      globální době angličtina. Důležitým jednotícím prostředkem mezinárodního porozumění fyziků je
      ovšem jazyk matematických symbolů, který hraje podobnou úlohu jako notový záznam u hudebníků.

      Zkoumáním průběhu vývoje vědy a vědeckých revoluci se zabýval americký historik, původně
      vystudovaný fyzik, \textsc{Thomas Kuhn}, autor knihy \emph{„Struktura vědeckých revolucí"}.
      Podle něho se ve vývoji fyziky střídají období tzv. „normální vědy", kdy dochází k hromadění
      poznatků v rámci určitého výkladu světa (Kuhn si pro takový výklad vymyslel dnes módní termín
      „paradigma", z řeckého vzor, předloha), a pak kritická období, vědecké revoluce, kdy je staré
      „paradigma" zavrženo a nastupuje nové. S takovým zjednodušeným a do jisté míry zavádějícím
      přístupem k vývoji vědy však někteří přední světoví fyzikové nesouhlasí. Patři k nim např.
      historik fyziky, profesor Harvardovy univerzity \textsc{Gerald Holton} nebo laureát Nobelovy
      ceny \textsc{Steven Weinberg}, kteří Kuhnovu koncepci ostře kritizuji. Je to pojetí, které
      nahrává některým obecným názorům o tom, že fyzika vlastně nemá pravdivou, objektivní představu
      o světě, že její zákony jsou věci jakési „dohody" nebo „povinné víry", kterou fyzikové čas od
      času popřou a vytvoří novou.

      \begin{tcnote}
        \textbf{Kuhnovo pojetí vývoje vědy}: \textsc{Thomas Samuel Kuhn} přinesl argumenty o tom,
        že pokrok vědeckého poznání není přímočarý, ale že je čas od času přerušován zásadními
        zvraty-vědeckými revolucemi. Při těchto vědeckých revolucích dochází k přehodnocení
        samotných základů dosavadního vědění. Vědecké poznání tedy nesměřuje k nějaké jediné
        pravdě o světě, netýká se žádné „objektivní reality“ - nezávislé skutečnosti, všem
        společné, vždy zde již jsoucí. Věda, tak jako každá lidská činnost, má svůj kulturní,
        dějinný, instituční, sociální a psychologický rozměr. I vědecké poznatky jsou proto
        historicky podmíněné: vyjadřují ducha dané epochy, mění se s dobou i s okolnostmi.
        
        \vspace{1em}
        {\centering
        \captionsetup{type=figure} 
        \luafigure[0.5]{fyz_fig0893.jpg}
        \captionof{figure}{\wikiKuhn (\textasteriskcentered 18. 7. 1922 - \textdagger 17. 6. 1996)
                  byl americký filosof, fyzik, teoretik vědy a vědeckého poznání, zabýval se
                  dějinami vědy, astronomií, kvantovou teorií a její prehistorií.}
        \label{fyz:fig0893}
      \par}
      \end{tcnote}

      Situace je složitější. Je známou vlastnosti lidského poznání, že čím více poznáváme, tím vice
      zjišťujeme, kolik toho ještě nevíme. Každá zodpovězená otázka navozuje desítky nových. Při
      hromadění poznatků se mezi nimi mohou objevovat rozpory, které se nedaří vysvětlit. V rámci
      daného přijatého výkladu přírody se objevují myšlenky, které připravují půdu novému pohledu na
      svět, \textbf{vědecké revoluci}. Ta však neznamená, že by předchozí vědění bylo odvrženo,
      znehodnoceno a popřeno, ale dochází k jeho zobecnění, prohloubení a vzniká pohled z nové,
      vyšší úrovně poznání.

      Ani nástup koperníkovské revoluce neznamenal, že by pozorování a práce předchozích generací
      astronomů byla znehodnocena, nehledě k tomu, že už ve starověku se vedle geocentrické soustavy
      uplatňovaly i názory, že se Země pohybuje a s ostatními planetami obíhá kolem Slunce. Podobně
      Einsteinova teorie relativity neznamenala odvržení, ale zobecnění Newtonovy fyziky. Ostatně
      celá astronautika a kosmická technika počítá stále se zákony Newtonovy mechaniky a až na
      některé výjimky nemusí brát v úvahu jemné efekty obecné teorie relativity. Kvantová fyzika ve
      20. století nepopřela předchozí fyziku klasickou, ale naopak objasnila rozpory v chápání
      dvojaké podstaty světla a otevřela fyzice cestu k dalším oblastem výzkumu, do mikrosvěta.
      Změny fyzikálních teorií a skoky ve vývoji fyziky nám nesmějí zakrýt pohled na vnitřní
      souvislost historického poznávání, k němuž přispívalo mnoho a mnoho známých i neznámých
      badatelů, kteří obohacovali naše znalosti o přírodních zákonech, vytvářeli mozaiku našeho
      obrazu světa.

      V dějinách fyziky si musíme ujasnit, kdy vlastně fyzika začíná a jak její historické etapy
      rozčlenit. Kdybychom vycházeli z dnešního stavu, kdy na světě pracuji ve vědeckých ústavech,
      na vysokých školách a v průmyslu tisíce fyziků ve spolupráci s inženýry a dalšími odborníky,
      kteří se scházejí každý rok na stovkách mezinárodních konferencí, kdy vyspělé země ve vzájemné
      spolupráci investují miliardy dolarů do fyzikálního výzkumu, pak fyzika v této podobě je ovšem
      produktem teprve posledních desetiletí. Kdybychom chápali fyziku jako vědu, která vychází ze
      systematicky prováděných experimentů a matematicky formulované teorie opět experimentálně
      ověřuje, mohli bychom počátek \textbf{klasické fyziky} klást na rozhraní 16. a 17. století.

      \luagraphic[1]{fyz_fig0940.jpg}{Eduard Ender - Kopernikus, die Bahnen der Gestirne
        aufzeichnend - 3764 - Kunsthistorisches Museum Kredit: Wikipedia}{fyz:fig0940}

      Lidé však poznávali přírodu, pozorovali přírodní jevy a snažili si je rozumově vysvětlovat už
      od starověku, a dokonce ještě dříve. Toto nejstarší období bychom mohli označovat jako
      „předvědeckou fyziku", „předhistorii fyziky" nebo \emph{\uv{období fyzikálního vědění před
      vznikem fyziky jako samostatné experimentální vědy}}, jak to někteří historikové vědy činí.
      Taková vymezení jsou však příliš šroubovaná a není důvodu považovat úsilí a výsledky
      starověkých badatelů za „nevědecké". Zjednodušíme proto periodizaci dějin fyziky a vyčleníme
      jen dva mezníky - nástup vědecké revoluce symbolickým rokem 1600 a vznik kvantové a
      relativistické fyziky symbolickým rokem 1900. Jak jsme již uvedli, každý takový převrat ve
      fyzice se připravoval během předcházejícího vývoje.

      Označíme období od starověku do roku 1600 jako \textbf{stará fyzika}. Tato fyzika se omezovala
      v podstatě jen na mechaniku (především statiku, včetně hydrostatiky) a optiku, která byla pod
      názvem „perspektiva" často vnímána jako součást geometrie. K fyzice musíme přiřadit i
      astronomii, snad nejstarší z věd, která měla ovšem pozorovací charakter, ale přispívala k
      rozvoji matematiky a později i fyziky. Stará fyzika neprováděla systematický experimentální
      výzkum a neměla k dispozici potřebné měřicí přístroje s výjimkou astronomických přístrojů
      úhloměrných a nedokonalých hodin k měření času. Nejpřesnějším měřením, které mohla provádět,
      bylo vážení na rovnoramenných váhách. Přesto ojedinělé experimenty prováděla a již během
      středověku narůstalo přesvědčeni, že přírodu je třeba měřit, vážit a matematicky popisovat.
      Popis fyzikálních jevů byl ovšem jen kvalitativní a byl založen na přírodní filozofii
      antických autorů, především \textsc{Aristotela}. Podobná situace byla ostatně i v lékařství,
      kde vládla autorita \textsc{Galenova}.

      \luagraphic[1]{fyz_fig0941.jpg}{\wikiAristoteles - Rembrandtova olejomalba: Aristotelés s 
      Homérovou bystou Kredit: Wikipedia}{fyz:fig0941}

      Období tří set let mezi rokem 1600 a 1900 označíme jako období \textbf{klasické fyziky}.
      Fyzika v této době má už plně charakter přírodní vědy, používá desítek nových přístrojů a
      měřicích metod a jako matematický aparát ji vedle geometrie slouží metody matematické analýzy.
      Zabývá se oblastí makroskopických jevů, tedy takových, které se odehrávají v prostorových a
      časových rozměrech naší každodenní zkušenosti. Zásluhou \textsc{Galilea}, \textsc{Keplera} a
      \textsc{Newtona} byly objeveny zákony dynamiky, pozemské i nebeské. Vedle mechaniky a optiky
      se součástí klasické fyziky stává v této době i nauka o elektřině a magnetizmu, termika a
      později termodynamika a statistická fyzika. Začíná se postupně vytvářet i vědecká chemie,
      která nakonec přispěla k objevu Mendělejevova periodického zákona a která s fyzikou úzce
      souvisí.

      Koncem 19. století se dokonce zdálo, že fyzika už všechny otázky položené přírodě zodpověděla,
      že je v podstatě vědou uzavřenou a zbývá snad už jen upřesnit několik drobných nejasností.
      Nástup kvantové a relativistické fyziky ve 20. století spojený se jmény \textsc{Maxe Plancka}
      a \textsc{Alberta Einsteina} znamenal myšlenkový zvrat, s nimž se fyzikové dlouho
      vypořádávali. Teorie relativity přinesla zásadní zobecnění Newtonovy klasické mechaniky, nové
      pojmy a představy, které jsou zdánlivě v rozporu s běžným tzv. \emph{„zdravým lidským
      rozumem"}. Kvantová fyzika přinesla nové chápání fyzikálních zákonitosti založené na pojmu
      pravděpodobnosti a umožnila matematicky a experimentálně zkoumat svět molekul, atomů a částic,
      který nemůžeme přímo pozorovat našimi smysly a který leží mimo dosah naší zkušenosti.

      \begin{tcnote}  
        Fyzika prošla dlouhým historickým vývojem a znalost tohoto vývoje pomáhá lépe pochopit
        logiku soustavy fyzikálních poznatků a dokonce docházet k poznatkům novým. V krátkosti
        dějiny fyziky můžeme rozdělit na tři hlavní etapy:
        \begin{itemize}[noitemsep]
          \item \textbf{Stará fyzika}: od starověku do počátku 17. století (orientačně do roku
                1600).
          \item \textbf{Klasická fyzika}: 1600 - 1900.
          \item \textbf{Moderní fyzika}: 1900 - dosud.
        \end{itemize}
      \end{tcnote}

      V obou případech však nešlo o znehodnocení výsledků předchozího vývoje fyziky, ale uplatnil se
      tzv. \textbf{princip korespondence}. Tento pojem zavedl původně \textsc{Niels Bohr}, aby
      vysvětlil, jak spolu souvisí kmitočty vyzařování spektrálních čar atomu vodíku z hlediska
      klasické a kvantové fyziky. Relativistická mechanika se zabývá pohybem částic nebo těles s
      rychlostmi blízkými rychlosti světla ve vakuu \(c=\SI{300000}{\km\per\s}\). Klesne-li jejich
      rychlost na hodnoty, s nimiž se setkáváme v běžném životě, přejdou zákony teorie relativity na
      tvar známý z Newtonovy mechaniky a budou korespondovat s fyzikou klasickou. Podobně kvantová
      fyzika obsahuje velmi malou konstantu zvanou Planckova, rovnou \(h =
      \SI{6.626e-34}{\joule\s}\). Pokud můžeme tuto konstantu považovat za zanedbatelně malou,
      jako je tomu ve světě našich makroskopických rozměrů, budou zákony kvantové fyziky
      korespondovat s fyzikou klasickou.   

      
      Pro jednoduchost budeme období od r. 1900 do dneška nazývat \textbf{moderní fyzikou}. Není to
      asi název ideální, stejně tak bychom mohli použit označení \uv{současná fyzika} (kdy začíná
      současnost?), \uv{relativistická a kvantová fyzika}, ale ani to by nebylo úplně výstižné.
      Moderní fyzika má i řadu dalších rysů, studuje např. i jevy nelineární, chaotické a jiné.
      Někdy bývá kvantová fyzika chápána jako protiklad k fyzice klasické a nekvantová
      relativistická fyzika řazena do fyziky klasické. Existuje ovšem i kvantová nerelativistická
      fyzika a kvantová relativistická fyzika a jejich vývoj se časově prolíná. Nebudeme se proto
      pokoušet periodizaci tohoto vývoje komplikovat.

      Moderní fyzika vyústila do dnešní vědeckotechnické revoluce a podstatně změnila život celé
      lidské společnosti. Sama byla zpětně ovlivněna především uplatněním počítačů, ale i nových
      materiálů a měřicích metod neuvěřitelné přesnosti. Velké urychlovače sloužící moderní fyzice
      patří k nejnáročnějším dílům, které kdy lidská technika vytvořila. Možnosti kosmických letů
      ovlivnily charakter astronomie a astrofyziky. Také výchova a vzdělávání fyziků a komunikace
      mezi nimi se změnily a dále mění. V dnešní době pracuje na světě nesrovnatelně více fyziků,
      než kolik jich žilo během celé lidské historie. Ještě před několika desítkami let byl svět
      zaplavován stovkami fyzikálních časopisů, stále více specializovaných, ale dnes už fyzikové
      nemohou všechny práce v nich obsažené sledovat. Nastupují nové komunikační databáze a
      vyhledávače potřebných informací. Tvořivou silou ve fyzice jsou především mladí lidé - není
      náhodou, že i v historii většinu největších objevů učinili fyzikové ve stáří od 20 do 30
      let...

    \subsection{Člověk}\label{fyz:IchapIIsecIssecIII}
      Mluvíme-li o fyzice jako o procesu poznávání přírody, musíme se také krátce zamyslet nad tím,
      kdo poznává, tedy nad člověkem. Existence inteligentních a civilizovaných bytostí na naší
      planetě patří k nejpodivuhodnějším přírodním úkazům i vzhledem k tomu, že o mimozemských
      civilizacích, ba ani o jiných formách života ve vesmíru zatím stále nic nevíme. Zrození
      člověka předcházel miliardy let trvající vývoj vesmíru a Země, který nakonec vytvořil takové
      podmínky, v nichž člověk může žít.

      Bylo k tomu třeba neuvěřitelné souhry náhod, shody fyzikálních, chemických a biologických
      podmínek. Stačilo by, kdyby planeta Země měla větší nebo menší gravitaci, než má, kdyby byla
      blíž k Slunci nebo dál od něho, neměla aspoň částečně pevný a suchý povrch, postrádala
      ochrannou atmosféru obsahující kyslík a ozón, magnetosféru chránící před nabitými částicemi z
      kosmu, a člověk v dnešní podobě by se na ní nemohl vyvinout. Důležité je, aby teplotní rozmezí
      na povrchu Země umožňovalo přítomnost vody v tekutém stavu. Existence Měsíce vhodné hmotnosti
      a vzdálenosti od Země přispívá ke stabilizaci směru osy zemské rotace a tím i klimatických
      podmínek.

      \luagraphicx[1]{fyz_fig0942.jpg}{Východ Země (anglicky \emph{Earthrise}) je fotografie,
      kterou pořídil 24. prosince 1968 americký astronaut William Anders z paluby kosmické lodi
      Apollo 8, nacházející se na oběžné dráze Měsíce. Snímek zachycuje výřez měsíčního povrchu a
      v pozadí osvětlenou část planety Země (pootočenou oproti severojižní orientaci o 135
      stupňů). Kredit: NASA/Bill Anders}{fyz:fig0942}

      Na oběžné dráze Měsíce při čtvrtém obletu \textbf{Apolla 8}\footnote{první lidé u měsíce} měli
      astronauti možnost poprvé pozorovat něco neskutečného: východ Země nad Měsícem (obr.
      \ref{fyz:fig0942}). William Anders zachoval duchapřítomnost a událost vyfotografoval - nejprve
      na černobílý film, vzápětí i na barevný. Fotografie se pak stala jedním ze symbolů programu
      Apollo a kosmonautiky vůbec. Snímek se stal ikonickým a jedním ze symbolů 20. století. Jak
      uvedl William Anders: \emph{„Urazili jsme tuto dlouhou cestu, abychom zkoumali Měsíc. Ovšem to
      nejdůležitější je, že jsme nakonec objevili Zemi.“} \cite[s.~69]{Prikryl2019}.

      Protože se planety sluneční soustavy pohybuji po eliptických trajektoriích málo odlišných od
      kruhových, je tím prakticky vyloučena možnost jejich srážek. Pokud jde o srážky Země s malými
      tělesy sluneční soustavy, kdysi hojnými, bylo již okolí Země od těchto těles vyčištěno
      především vlivem velkých planet, zejména Jupiteru. Také v rámci naší Galaxie (viz obr.
      \ref{fyz:fig0916} v kapitole \ref{kulIchIIsecI} \uv{Naše Galaxie}) je sluneční soustava, a
      tedy i naše Země umístěna mimořádně výhodně. Je dostatečně daleko od centra Galaxie. takže
      není ohrožena pronikavým galaktickým zářením, ale zase dosti hluboko uvnitř Galaxie, aby nás
      výbuchy supernov stačily zásobit těžkými prvky potřebnými pro život. Dokonce by stačilo, aby
      molekula vody měla jiné geometrické uspořádání svých atomů nebo atom uhlíku jiné chemické
      vazební vlastnosti, a život, jak ho známe na Zemi, by nemohl existovat.

      Zda byl vznik dnešního člověka za těchto okolností nutný a zákonitý, nevíme. Kdyby však k němu
      nedošlo a lidstvo by neexistovalo, nebyli by ani fyzikové zkoumající přírodu a kladoucí
      všetečné otázky. Tento poněkud krkolomný logický závěr se někdy nazývá \emph{„antropický
      princip"}. V každém případě si dnes stále více uvědomujeme, jak křehká je světová ekologická
      rovnováha a jak malé změny teploty, klimatu či složení atmosféry mohou přivést lidstvo do
      potíží, nehledě na ohrožení pocházející z hlubin Země nebo z kosmu.

      \subsubsection{Evoluce člověka}
        Pomineme-li předchozí geologický vývoj Země a života na ni, můžeme hledat předky člověka
        někde mezi třetihorními savci, kteří vystřídali nadvládu dnes tak oblíbených dinosaurů.
        Příslušníci řádu primátů, vcelku nevzhledných a nenápadných živočichů, se postupně
        rozčlenili do řady čeledí, z nichž lidoopi přežili do dneška a jsou našimi nejbližšími
        příbuznými v živočišné říši. Jedna z těchto čeledi, tzv. \textbf{hominidé}, se však vydala
        na dlouhou cestu vývoje, který dospěl až k dnešnímu člověku.

        Někteří antropologové považuji za první známé zástupce čeledi hominidů příslušníky rodu
        \emph{ramapitéků}, jejichž pozůstatky nalezené v Indii a v Africe jsou staré asi 15 milionů
        let. Novější výzkum využívající metody molekulární biologie (srovnávaní genetické výbavy
        dnešního člověka a afrických opic) však naznačuje, že první hominidi se objevují v Africe
        později, asi před sedmi miliony lety. Z kosterních nálezů známe dnes dva rody čeledi
        hominidů - je to rod \emph{australopiteků}, členící se na mnoho druhů, a rod \emph{Homo},
        člověk.

        Čas od času se dozvídáme o nových nálezech kosterních pozůstatků australopiteku; tyto
        pozůstatky pocházejí z doby před 3-4 miliony let. Australopitekové byli menší než dnešní
        lidé, vyznačovali se základní vlastnosti hominidů, vzpřímenou chůzí po dvou zadních
        končetinách, a používali nástroje, které ovšem sami ještě nevyráběli. Nějakou dobu zřejmě
        existovali souběžně s populací rodu Homo, ale později zanikli bez následovníků, možná pro
        nedostatek přizpůsobivosti změněným přírodním podmínkám. Mnoho o nich nevíme ani o tom, jaké
        byly vzájemné vztahy mezi nimi a příslušníky rodu Homo. Australopitekové jsou tak našimi
        nejbližšími příbuznými a máme s nimi společné, dosud neznámé předky.

        Kolébkou dnešního člověka je Afrika, kde se vytvořily nejvhodnější klimatické a zeměpisné
        podmínky pro vznik jeho rodu. Během vývoje prošel člověk třemi stádii, která bychom mohli
        nazvat \emph{hominizace}, \emph{sapientizace} a \emph{civilizace}. Člověk se postupně
        vyčleňoval z živočišné říše, uvědomoval si sám sebe a stále více vystupoval jako poznávající
        a aktivní subjekt vůči přírodě. První lidé vytvářeli velké rodiny, skupiny kolem třiceti
        jedinců, v nichž docházelo k dělbě činností a zdokonalovala se vzájemná komunikace. Ta
        nakonec dovedla ke vzniku artikulované řeči, lidského jazyka. Vyžádalo si to ovšem některé
        anatomické změny ve stavbě hrtanu, ústní dutiny i mozku. Přitom jazyk a myšlení se podmiňují
        a jejich rozvoj se vzájemně urychluje. Bohužel dosud neznáme odpověď na vzrušující otázku,
        jak spolu komunikovali příslušníci prvních druhů člověka, co si sdělovali a vyprávěli.
        Mluvené slovo nezanechává v historii hmotné stopy.

        \begin{tcnote}
          \textbf{Hominidé} \emph{(Hominidae)} je čeleď primátů z nadčeledi \textbf{hominoidi}
          \emph{(Hominoidea)}. Někdy se uvádí český název čeledi Hominidae jako „lidem podobní“ nebo
          „lidoopi“. Podle českého názvosloví „Člověkovití“.Je třeba odlišovat:
          \begin{description}[leftmargin=8em,labelindent=1em, style=nextline]
            \item[\textbf{Hominoidi}] (Hominoidea) - skupina spojující gibony a hominidy.
            \item[\textbf{Hominidé}] (Hominidae) - skupina spojující orangutany, gorily, šimpanze a
              lidi.
          \end{description}

          \noindent\textbf{Ramapithecus} (česky též Ramapiték) je neplatné taxonomické označení,
          užívané zhruba do 80. let 20. století pro několik druhů vyhynulých hominoidů. V současné
          době převládá názor, že Ramapitek je totožný se Sivapitekem. To jen potvrzuje předpoklad,
          že Ramapithecus nebyl předkem moderního člověka, neboť rod Sivapithecus je příbuzný spíše
          současným orangutanům. Ramapithecus byl zejména v 60. a 70. letech 20. století považován
          za nejstaršího předka moderního člověka a přímého předchůdce rodu Australopithecus. Stáří
          fosilií, určené na 10 - 14 milionů let, dobře odpovídalo tehdy uvažované době oddělení
          samostatné vývojové linie rodu Homo. V té době bylo toto oddělení předpokládáno asi před
          15 miliony let, zatímco v současnosti se klade do doby asi před 6 - 8 miliony let.
        \end{tcnote}

        Za počátek hominizačního procesu, tedy odlišení člověka od ostatních živočichů, považují
        antropologové napřímení postavy a stabilní \emph{bipedální chůzi}\footnote{(z lat. bi-pes,
        bipedis, dvounohý) znamená pohyb po dvou končetinách.}. Jakmile se člověk ve vlnící se
        africké trávě postavil, mohl se rozhlédnout po krajině, zpozorovat blížící se nebezpečí, ale
        také pozvednout zrak k noční hvězdné obloze a konat první astronomická pozorování. Chůze po
        dvou nohách je ovšem méně bezpečná než po čtyřech a od této chvíle hrozila člověku i možnost
        pádů. Vzpřímená chůze, nový způsob života a nutnost přežití, rostoucí všestrannost při
        výběru potravy vedly postupně k dalším anatomickým změnám. Měnila se kostra lebky a stavba
        chrupu, zvětšoval se objem mozkovny, zdokonalovala stavba nohou a rukou, nyní uvolněných k
        rozmanitým, často jemným mechanickým činnostem. Zároveň s tím postupovali rozvoj rozumových
        schopností a myšlení. Člověk začal používat nástroje a zbraně, zprvu jen předměty nalezené v
        přírodě a mírně upravené, později vyráběné ze dřeva, kostí a kamene podle předchozího
        promyšleného záměru.

        Uvědomíme-li si, že člověk na rozdíl od jiných živočichů není vybaven ani mimořádnou silou,
        schopností rychlého pohybu, ani zvláštní bystrostí smyslů, zdá se téměř neuvěřitelné, že
        mohl při nepočetné populaci přežít miliony let, překonat nástrahy přírody, nepřízeň počasí a
        ledových dob, nemocí a hladu. Existují známky toho, že v některých obdobích poklesla
        početnost lidské populace na světě tak hluboko, že byl vlastně „ohroženým druhem". Člověk se
        dožíval nízkého věku a nevyhnutelnou dětskou úmrtnost musely zřejmě pravěké ženy kompenzovat
        vysokou porodností. Při nedostatku vhodné dětské potravy v nepříznivých ročních obdobích
        musely také zřejmě velmi dlouhou dobu děti kojit. Zvětšování objemu lebky novorozenců vedlo
        vzhledem k omezené průchodnosti porodních cest k tomu, že lidská mláďata se rodila ještě ne
        úplně vyvinutá, byla bezbranná a vyžadovala dlouhodobou péči rodičů i širší rodiny.

        \begin{tcnote}
          Příklad \wikiKlasCloveka:
          \begin{itemize}[noitemsep]
            \item \textbf{Doména}:Eukaryota
            \item \textbf{Říše}:živočichové (Animalia)
            \item \textbf{Kmen}:strunatci (Chordata)
            \item \textbf{Podkmen}:	obratlovci (Vertebrata)
            \item \textbf{Nadtřída}: čtyřnožci (Tetrapoda)
            \item \textbf{Třída}:	savci (Mammalia)
            \item \textbf{Podtřída}: živorodí (Theria)
            \item \textbf{Nadřád}: placentálové (Placentalia)
            \item \textbf{Řád}:	primáti (Primates)
            \item \textbf{Podřád}: Haplorhini
            \item \textbf{Infrařád}: opice (Simiiformes)
            \item \textbf{Oddělení}: úzkonosí (Catarrhini)
            \item \textbf{Nadčeleď}: hominoidi (Hominoidea)
            \item \textbf{Čeleď}:	hominidi (Hominidae)
            \item \textbf{Podčeleď}: Homininae
            \item \textbf{Tribus}: Hominini
            \item \textbf{Rod}:	člověk (Homo)
            \item \textbf{druh}: Homo sapiens - člověk moudrý
            \item \textbf{poddruh}: Homo sapiens sapiens - člověk moudrý vyspělý
          \end{itemize}
          \tcblower
          Postupný vznik a vývoj jednotlivých zástupců rodu Homo označujeme termínem antropogeneze a
          začíná asi před 8 miliony let v Africe, kdy žil společný předek lidí a šimpanzů, což jsou
          naši nejbližší žijící příbuzní. Potomci tohoto předka se rozdělili do dvou větví - jedné
          směřující k současným lidem a druhé vedoucí k dvěma dnes žijícím druhům šimpanzů.

          \vspace{1em}
          {\centering
          \captionsetup{type=figure} 
          \luafigure[1]{fyz_fig0953.pdf}
          \captionof{figure}{Historie rozšíření člověka (údaje jsou v letech před současností). 
                    Kredit: Wikipedia}
          \label{fyz:fig0953}
        \par}

        \end{tcnote}

        Na druhé straně si pravěký člověk vyvinul mnoho schopnosti a dovedností, které jsme my v
        průběhu civilizačního vývoje už ztratili. Žil v těsném kontaktu s přírodou, znal a využíval
        vlastnosti rostlin a zvířat, s nimiž přicházel do styku. Ovládal zřejmě i podvědomé signály,
        které oznamovaly nebezpečí, aby se mu mohl vyhnout, všímal si v přírodě detailů, které nám
        dnes unikají. Jeho myšlení bylo velmi konkrétní, a tím i blízké uměleckému. Je např. známo,
        že jazyky některých indiánských kmenů mají nesmírně bohatý slovník, pokud jde o označováni
        přírodních jevů - desítky výrazů označují různý stav oblohy, barvu trávy apod.

        \luagraphic[0.9]{fyz_fig0954.jpg}{Slavné jeskynní malby jeskynního komplexu Lascaux v
          (výslovnost: [lasko]) jihozápadní Francii. Vyobrazení představují především býložravá
          zvířata (jeleni, bizoni, koně, býci), tedy především lovná zvěř.
          Kredit:Wikipedia}{fyz:fig0954}

        Pokud jde o fyzikální poznání, snadno si představíme, že se pravěký člověk naučil používat
        páku k manipulaci s břemeny, znal chování primitivních plavidel na vodě, využíval vlastnosti
        pohybu těles při šikmém vrhu. Uměl házet oštěpem i harpunou, později střílet z luku a někde
        využívat k usmrcení oběti i rostlinné jedy. Své praktické zkušenosti hromadil a předával po
        tisíce generací. Setkával se ovšem i se záhadnými impozantními přírodními jevy, které byly
        mimo dosah jeho chápání. Sledoval pohyby hvězdné oblohy, krásu tropických hvězdných nocí a
        těžko vědět, co si přitom myslel. Bázeň, ale i zvědavost v něm musel vyvolávat blesk a hrom.
        Vnímal střídání dní a noci, jejich měnící se délku, střídání ročních dob a periodické změny
        v přírodě. To všechno muselo podněcovat jeho fantazii.

        Nejstarší nalezené kosterní pozůstatky a kamenné nástroje příslušníka rodu Homo pocházejí z
        doby asi před dvěma miliony let. Našli je v roce 1960 známi angličtí paleontologové a
        antropologové z rodiny manželů Leakeyovych v Olduvajské rokli\footnote{Kaňon má hloubku
        přibližně 100 metrů a na délku měří přes 40 km. Dříve se zde nacházelo jezero a v okolí byla
        hojná vegetace, která lákala spoustu živočichů a také hominidy. Sloužila jim k obživě a také
        jako útočiště.} v Tanzánii a dali mu název \textbf{Homo habilis}, \emph{„člověk zručný“}.
        Homo habilis zahájil \textbf{paleolit}, starší dobu kamennou, a vytvořil první lidskou
        hmotnou kulturu, kterou nazýváme \emph{olduvajská}. Jsou pro ni typické hrubé kamenné sekáče
        opracované jen několika údery jiným kamenem 

        Velmi úspěšným následovníkem člověka zručného byl \textbf{Homo erectus}, \emph{„člověk
        vzpřímený"}. Ten se již rozšířil z Afriky po celém Starém světě a jeho pozůstatky nacházíme
        v Indonésii (na Jávě ho objevil E Dubois v r. 1892), v Číně, v Evropě i dalších místech. Pro
        přesnost poznamenejme, že podle posledních, ještě nedostatečně ověřených nálezů se zdá, že
        se Homo erectus „nevyvinul" přímo z Homo habilis, ale že oba lidské druhy žily téměř půl
        milionu let vedle sebe v týchž lokalitách, a musely tedy mít společného, dosud neobjeveného
        předka. Homo erectus byl zdatným lovcem, vynikal fyzickou silou a dovedl rychle běhat. Jeho
        typickým produktem jsou štípané kamenné pěstní klíny: nedávné experimenty ukázaly, jak je
        pro dnešního člověka obtížné takový klín v přírodě vyrobit. Někteří antropologové mají za
        to, že ovládal určitou formu mluvené řeči, byť ještě málo artikulované. Homo erectus přežil
        na Zemi zřejmě jako jediný představitel svého rodu v období asi před 1,5 milionem až 250 000
        lety, a překonal útrapy několika ledových dob.

        Z hlediska dějin fyziky mu přísluší prvenství z nejvýznamnějších - objev rozdělávání a
        využívání ohně. Znamená to vlastně uvolnění části energie vázané v látce podle známého
        Einsteinova vztahu \(E = mc^2\) a význam tohoto objevu je v dějinách srovnatelný pouze s
        uvolněním jaderné energie ve 20. století. Dnes by měl objevitel způsobu rozdělávání ohně
        nesporný nárok na Nobelovu cenu. K objevu došlo někdy před půl milionem let, samozřejmě za
        neznámých okolnosti. Zprvu člověk zřejmě jen udržoval oheň získaný z náhodného přírodního
        požáru. Teprve později našel různé způsoby, jak ho rozdělávat, založené vesměs na zahřívání
        třením, případně křesáním. Prometheovský vynález ohně změnil život člověka, jeho schopnost
        bránit se chladu i šelmám, tepelně upravená strava ovlivnila způsob jeho výživy a mozkovou
        činnost. Z hlediska psychologického je ovšem důležité, že člověk jako jediný živočich našel
        odvahu překonat strach z ohně a manipulovat s ním. Také odvaha riskovat, patří k základním
        předpokladům poznávání.

        Pozdní formy druhu Homo erectus daly vzniknout novému druhu, který označujeme jako
        \textbf{Homo sapiens}. \emph{,,člověk moudrý"}. Proces sapientizace, který dále odlišil
        člověka od ostatních živočichů, znamená v podstatě vznik duševního života, který vytváří
        nehmotnou kulturu - rituály, náboženství, mytologii a na konec umění a vědy. Jeden z mála
        projevů charakterizujících raná stadia Homo sapiens, které mohou archeologové odhalit, je
        pohřbívání mrtvých. Nálezy hrobů a svědectví o pohřbech spojených s nejrůznějšími rituály
        svědčí o tom, že Homo sapiens si opět jako jediný ze živočichů začal uvědomovat konečnost
        svého pozemského bytí a začal se zamýšlet nad posmrtným životem, vztahem mezi lidským tělem
        a duší. Je pravděpodobné, že pohřební rituály byly doprovázeny i hudebními projevy a
        rytmickým tancem a zpěvem.

        Člověk začal ve své fantazii zabydlovat přírodu nadpřirozenými bytostmi, dobrými a zlými
        duchy, zlidšťovat si přírodní jevy. Vzniká \emph{šamanismus}, představa o tom, že zlé duchy
        způsobující nemoci a neštěstí je možno zažehnat a dobré duchy si naklonit. K nejstarším
        rituálům patří obřady mající zajistit úspěch při lovu, později za mladší doby kamenné
        úrodnost polí. Důležitou úlohu hrály i kulty mateřství a plodnosti jako záruky pokračování
        lidského rodu a s tím související ženské sošky a artefakty známé jako pravěké venuše (obr.
        \ref{fyz:fig0943}).

        \luagraphic[1]{fyz_fig0943b.jpg}{\wikiVestonickaVenuse: podle fantazie ilustrátora
        \wikiBurian. \uv{Srdce se mu svírá bolestí. Jeho ženu zabil bizon a on bez ní neumí žít.
        Lovec Njan bezmyšlenkovitě z hroudy hlíny vymodeluje její podobu. Opatrně ji obejme dlaněmi,
        zadívá se na ni a se slzami v očích ji hodí do ohně …}. Tak nějak podle fantazie českého
        spisovatele a archeologa Eduarda Štorcha (1878-1956) v knize Lovci mamutů přijde na
        svět legendární soška známá jako Věstonická venuše. Zhruba 25 tisíc let a možná ještě déle
        odpočívala netknutě tajemná soška, rozlomená na dva kusy, ve vrstvě popela v pozůstatcích
        pradávného ohniště mezi dnešními obcemi Dolní Věstonice a Pavlov v Jihomoravském kraji. Až v
        období mezi světovými válkami přiláká tato oblast věhlasného českého archeologa
        \wikiKarelAbsolon (1877-1960). Kredit:časopis Epocha}{fyz:fig0943}

        Homo sapiens vytvořil \emph{dvě větve} - \textbf{člověka neandertálského} a člověka našeho
        poddruhu, který je též nazýván \textbf{člověkem kromaňonským} nebo člověkem předvěkým.
        Nevíme přesně kdy a jak se to stalo, geneticky však bylo ustanoveno, že člověk předvěký není
        potomkem neandertálců. V Evropě obývali neandertálci nehostinné chladné kraje např. oblast
        Alp, prokázali značnou fyzickou zdatnost a odolnost, používali oheň, odívali se do zvířecích
        kůží, pohřbívali své mrtvé, vyznávali zvířecí kulty, především medvěda, a dorozumívali se
        již artikulovanou řečí. Nějakou dobu žili v Evropě souběžně s člověkem předvěkým a asi před
        35 000 lety vyhynuli bez následovníků. 

        Tím se dostáváme k našemu vlastnímu poddruhu, který se již od nás anatomicky, tělesnými ani
        duševními schopnostmi ničím neliší. Říkáme mu možná s přehnanou hrdostí \textbf{Homo sapiens
        sapiens}, \emph{„člověk moudrý, moudrý“}. Tento náš přímý předek se objevil v Evropě na
        sklonku starší doby kamenné, asi před 40 000 lety, v podstatě nevíme odkud. V té době se
        pomalu schylovala ke konci \emph{poslední doba ledová}, Evropou putovala stáda mamutů, a tak
        máme tyto lidi v povědomí jako „lovce mamutů". Četná naleziště ohnišť, hrobů, jemně
        opracovaných kamenných nástrojů, ozdob a sošek (některé z nich z hlíny vypalované v peci
        nesou ještě otisky prstů svého tvůrce) svědčí o vyspělé kultuře. 

        Největší překvapení však přinesly objevy jeskynních obrazáren s nástěnnými malbami zvířat v
        pohybu a lovců, které jsou postupně nacházeny především v předhůří Pyrenejí. Nejstaršími z
        těchto objevů jsou španělská jeskyně Altamira známá od r. 1879 (stáří 13 000 let), na
        francouzské straně jeskyně Lascaux (objevena 1940, stáři 15 000 let), v poslední době
        jeskyně Chauvet v údolí řeky Ardèche (1994, stáří 20 000 let) a rada dalších. Podobné
        obrazárny a skalní malby přibližně z téhož období najdeme i v jiných částech světa,
        především v Africe.

        Nástěnné malby zobrazují zvířata, bizony, pratury, divoké koně, jeleny, soby, nosorožce, lvy
        a medvědy, některá z nich už v Evropě dávno vyhynulá. Estetická hodnota obrazů je plné
        srovnatelná s uměleckými výkony dnešního člověka a svědčí o plně vyvinutém duševním životě.
        Musí se nás zmocnit vzrušení, zamyslíme-li se nad tím, co pohánělo pravěkého umělce, aby
        vstupoval do temného prostoru a za blikavého světla olejových lampiček vytvářel svá umělecká
        díla. Po desítky tisíc let se v jeskyních uchovalo i nářadí umělců, zbytky barev, lampičky,
        otisky malířových rukou a stopy bosých nohou včetně dětských. Účel maleb mohl souviset s
        loveckými kulty, ale mohl mít také jiný, hlubší, nám neznámý symbolický smysl. Zároveň ale
        jsou projevem talentu, touhy po tvůrčí seberealizaci, která je vlastní člověku stejně jako
        schopnost údivu, zvídavost, fantazie a odvaha, Patří i k předpokladům vzniku vědy.

        Konečně třetím stupněm, který přiblížil vznik naši dnešní společnosti, byl \emph{proces
        civilizace}: jeho název souvisí s latinským \emph{„civitas"}, obec, a poukazuje na vznik
        strukturované společenské organizace, prvních státních celků. Odehrál se v mladší době
        kamenné, neolitu, a na počátku doby bronzové, a to poměrně nedávno, v prvních tisíciletích
        před našim letopočtem. Souvisí s přechodem k novým způsobům obživy lidi, k pastevectví a
        zemědělství. Přinesl významné zásahy člověka do přírody, vypásání stepi, vypalováni lesů,
        zavlažováni pudy a počátky znehodnocování životního prostředí. Někdy mluvíme o
        \emph{neolitické revoluci}. Člověk začal zakládat města, rozvíjet obchod a dopravu, začal se
        cítit pánem přírody. Dostatek potravy umožnil rychlý růst lidské populace. Odhaduje se, že
        kolem r. 15 000 př. n. l. žily na světě asi 3 miliony lidi, r. 2 000 př. n. l. už asi 50
        milionů a na přelomu letopočtu kolem 250 milionů obyvatel.

%---------------------------------------------------------------------------------------------------    
  \section{Stará fyzika}\label{fyz:IchapIIsecII}
    Starou fyziku nemůžeme považovat za vědu ve vlastním smyslu, i když se dobrala celé řady
    významných vědeckých poznatku. První z nich znali již staří Sumerové, Babyloňané, Egypťané a
    Číňané. Šlo zejména o  poznatky astronomické a geometrické (Pythagorova veta) a také o metody
    měření některých fyzikálních veličin (délka, hmotnost, čas). Fyzika ve starém Řecku byla jako
    součást filosofie převážně spekulativní a tento charakter si pod vlivem aristotelismu udržela,
    až do počátku novověku. Skutečný fyzikální výzkum prováděli až helenističtí Řekové, kdy se
    centrem vědy a kultury antického světa stala Alexandrie. 
    
    \begin{figure}[ht!]  % \ref{fyz:fig0894}
      \centering
      \luafigure[1]{fyz_fig0894.jpg}
      \caption{ \wikiAlexLib byla největší a nejslavnější knihovna starověku. Byla součástí
                věhlasného múseia v Alexandrii, vybudovaného z podnětu Ptolemaia I. Byla považována
                za hlavní centrum vzdělanosti od 3. století př. n. l. až do roku 48 př. n. l., kdy
                za války mezi Caesarem a Pompeiem zčásti vyhořela. Starověké zdroje pojednávají o
                ničení knihovny, o tom, kdo je zodpovědný za ničení a kdy k němu došlo, se liší.
                \cite[s.~76]{Stoll2009}}
      \label{fyz:fig0894}
    \end{figure} 

    V Alexandrii studoval největší fyzik starověku \textsc{Archimédes}, který dospěl k důležitým
    poznatkům o statické rovnováze těles a plování těles a v matematice se těsně přiblížil objevu
    diferenciálního a integrálního počtu. Alexandrijští Řekové znali také zákon odrazu světla
    (nikoli lomu) a prováděli první měření teploty. Poznatky antiky byly středověké Evropě
    zprostředkovány Araby, kteří se též intenzivně zabývali optikou (\textsc{Alhazen}) a určováním
    měrné hmotnosti látek. Zatímco ve středověku byly hlavní přírodovědné poznatky čerpány z
    Euklidových \uv{Základu} (geometrie), \uv{Almagestu} Klaudia Ptolemaia (geocentrický výklad
    astronomie sluneční soustavy) a spisu Aristotelových (mj.\uv{Fysika}), vešly práce Archimédovy v
    Evropě ve známost až teprve začátkem novověku. Ve starověku a středověku však fyzika neprováděla
    systematické experimenty, nevyužívala matematický aparát k popisu přírodních jevu a neměla ani
    přesně definovány základní pojmy (rychlost, zrychlení, síla apod.) Zrod fyziky jako vědy se
    datuje začátkem 17. století. Na základě astronomických výzkumu \textsc{Keplerových} (1571-1630)
    a pozemských mechanických experimentů\textsc{ Galileových} (1564-1642) mohl Isaac Newton
    (1643-1727) vytvořit první fyzikální teorii, klasickou mechaniku, využívající matematický aparát
    diferenciálního a integrálního poctu. Newton přišel s koncepcí všeobecné gravitace a ukázal, že
    není přehrady mezi nebeskou a pozemskou fyzikou, že síla, která udržuje planety na jejich
    drahách kolem Slunce je táž jako síla, která nutí jablko padat k zemi. Základní Newtonovo dílo z
    r. l687 nese název \uv{Matematické základy přírodní filosofie} (\uv{Philosophiae naturalis
    principia mathematica}) a představuje pravděpodobně nejvýznamnější vědeckou knihu, která byla
    kdy napsána. Newton se zabýval též optikou a rozpracoval teorii rozkladu bílého světla do
    spektra. V té době byl již zásluhou Snellovou a Descartovou znám i zákon lomu světla. Z roku
    1600 pochází první vědecký spis o elektřině a magnetismu od anglického lékaře a fyzika Gilberta.
    Výzkumem  těchto jevu se v následujících stoletích zabývala celá řada fyziků (Coulomb, Volta,
    Oersted, Amp\`{e}re a další). Tento výzkum pak završil \textsc{Faraday} (1791-1867) svým objevem
    zákona elektromagnetické indukce a svou koncepcí siločár elektromagnetického pole. Úlohu Newtona
    elektromagnetismu pak sehrál \textsc{James Clerk Maxwell} (1831-1879), který ve svém
    \uv{Traktátě o elektřině a magnetismu} z r. 1873 sestavil slavné Maxwellovy rovnice popisující
    vlastnosti elektromagnetického pole. Maxwell zároveň teoreticky zdůvodnil elektromagnetickou
    povahu světla a ukázal, že jevy spojené s vlastnostmi elektrického náboje (\uv{elektřina}),
    elektrického proudu (\uv{galvanismus}), magnetického pole a světla (optika), jsou jedné a téže
    elektromagnetické povahy. V devatenáctém století byl tak dovršen výzkum mechanických jevů a
    elektromagnetismu a klasická fyzika tím završena. V přírodě tedy existovaly pouze dvě síly, dva
    způsoby vzájemné interakce mezi částicemi: gravitační a elektromagnetická. Mezi nimi se však
    projevoval určitý rozpor. Jak Newtonovy tak Maxwellovy rovnice platí v libovolné inerciální
    vztažné soustavě. Při přechodu od jedné inerciální soustavy k druhé se však Newtonovy rovnice
    transformují pomocí tzv. Galileiho transformací a Maxwellovy rovnice pomocí Lorentzových
    transformací. Fyzika se tak rozdvojila, mechanické a elektromagnetické děje se zdály být
    neslučitelné. Kromě toho existovaly některé experimenty, jejichž výsledek nedokázala klasická
    fyzika vysvětlit: průběh spektra rovnovážného elektromagnetického záření (tzv. záření absolutně
    černého tělesa) a pokus Michelsonův, který svědčil o neexistenci světelného éteru. Tyto zdánlivě
    nepodstatné rozpory vyústily ve 20. století ve vznik moderní fyziky, tj. fyziky kvantové a
    relativistické. Právě koncem roku 1900 vyslovil Planck tzv. kvantovou hypotézu, jíž vysvětlil
    záření absolutně černého tělesa, a v r. 1905 publikoval Einstein práci o speciální teorii
    relativity. V ní překlenul rozpor mezi Newtonovou a Maxwellovou fyzikou a fyziku opět sjednotil.
    Předpoklad o existenci světelného éteru se teorií relativity stal zbytečným. V roce 1916
    vytvořil Einstein i obecnou teorii relativity jako moderní teorii gravitace. Gravitační síly
    podle této teorie souvisejí se zakřivením prostoročasu. Jak speciální, tak obecná teorie
    relativity přecházejí při rychlostech objektu podstatně menších než je rychlost světla ve vakuu
    a při slabých gravitačních polích v teorii Newtonovu. Přelom 19. a 20. století je též poznamenán
    objevem radioaktivity a vznikem jaderné fyziky, která tak významným způsobem zasáhla do života
    celého lidstva. V jaderné fyzice se uplatní další dvě přírodní síly - tzv. silná, která udržuje
    nukleony v atomových jádrech a slabá, která se projevuje při radioaktivní přeměně beta za vzniku
    neutrin. Moderní fyzika odhalila v kosmickém záření a pomocí urychlovačů obrovské množství
    částic, jejichž vlastnosti studuje a snaží se je utřídit a vysvětlit. Mezi všemi těmito
    částicemi působí čtyři základní síly přírody: gravitační, elektromagnetická, silná a slabá. V
    nedávné době se podařilo prokázat, že i elektromagnetická a slabá interakce jsou téže podstaty a
    tvoří jedinou sílu elektroslabou. V průběhu historie fyziky od Newtona a Maxwella k dnešku tak
    probíhá úsilí o sjednocování interakcí, které pokračuje i dnes. Fyzika se pokouší prokázat, že i
    silná a elektroslabá interakce jsou téže povahy, a že k nim konečně přistupuje i síla
    gravitační. Tím by vznikla idea jediné přírodní síly sjednocující všechny přírodní jevy a děje.
    Fyzika ovšem nemůže k takovému závěru dojít pouhým uvažováním, musí matematicky vypracovat a
    zdůvodnit příslušnou teorii a její závěry experimentálně ověřit. To vede ke snaze budovat stále
    větší a větší urychlovače a také k intenzivnímu výzkumu jevů v kosmu. Sjednocování interakcí má
    totiž těsnou návaznost na vývoj vesmíru podle hypotézy o tzv. \uv{velkém třesku}. Právě v
    počátcích vývoje vesmíru by se měly všechny čtyři (resp. tři) interakce uplatňovat rovnocenným
    způsobem a teprve v průběhu dalšího vývoje a rozpínání vesmíru se postupně oddělovat. Tak jako
    počátky vzniku vědecké fyziky v 17. století jsou spjaty s astronomickými pozorováními sluneční
    soustavy, je i dnes fyzika stále více propojena s astrofyzikou. Vesmír zůstává největší
    fyzikální laboratoří.

%--------------------------------------------------------------------------------------------------- 
  \section{Klasická fyzika}\label{fyz:IchapIIsecIII}
    Je dost těžké začít hned se současnými představami, a proto se podívejme, jak se jevil svět v
    roce 1920 a potom na tomto obrázku něco změníme. Naše představa světa byla před rokem
    \textbf{1920} následující: „Scénou“, na které vystupuje vesmír, je \emph{trojrozměrný
    geometrický prostor} popsaný ještě Eukleidem a věci se mění v prostředí, které nazýváme časem.
    Prvky vystupující na scéně jsou \emph{částice}, například atomy, které mají určité vlastnosti.
    Především vlastnost setrvačnosti: pohybuje-li se částice, zachová si pohyb v původním směru,
    pokud na ni nepůsobí \emph{síly}. Druhým prvkem jsou tedy síly, o nichž se tehdy  
    předpokládalo, že jsou dvojího druhu. K prvnímu, velmi složitému druhu, patřila síla vzájemného
    působení, která udržovala atomy v jejich různých kombinacích komplikovaným způsobem a byla
    zodpovědná za to, jestli se sůl při zvyšování teploty rozpouští rychleji nebo pomaleji. Druhou
    známou silou byla interakce dalekého dosahu - hladké a klidné přitahování. Tato síla, měnící se
    nepřímo úměrně čtverci vzdálenosti, byla nazvána \emph{gravitací}. Její zákon byl známý a byl
    velmi jednoduchý. Proč věci zůstávají v pohybu, když se už začaly pohybovat, nebo proč existuje
    gravitační zákon, bylo, samozřejmě, neznámé.
    
    Zabýváme se popisem přírody. Z tohoto hlediska je plyn a právě tak všechna hmota myriádou
    pohybujících se částic. Takto se dostávají do souvislosti mnohé věci, které jsme viděli na
    mořském břehu. \emph{Tlak} pochází od \emph{srážek atomů} se stěnami nebo s čímkoliv jiným;
    atomy pohybující se převážně jedním směrem vytvářejí vítr; \emph{chaotické vnitřní pohyby}
    představují \emph{teplo}. Známe vlny zvýšené hustoty, kde se shromáždilo příliš mnoho částic,
    které při rozletu stlačují další shluky částic a pohyb se tak předává dál. Tyto vlny vyšší
    hustoty představují \emph{zvuk}. Pochopení tolika věcí je možno považovat za úžasný úspěch. O
    některých z těchto věcí jsme hovořili v předcházející kapitole.
    
    Jaké druhy částic existují? Tehdy předpokládali, že je jich 92. Nakonec bylo objeveno 92 různých
    druhů atomů. Měly různá jména podle svých chemických vlastností.
    
    Byl tu ještě problém \emph{povahy sil krátkého dosahu}. Proč uhlík přitahuje jeden kyslík,
    případně dva, ale ne víc? Jaký je mechanizmus vzájemného působení mezi atomy? Je to gravitace?
    Na tuto otázku musíme odpovědět záporně, protože gravitace je na to příliš slabá. Představme si
    však sílu podobnou gravitaci, měnící se nepřímo úměrně čtverci vzdálenosti, ale mnohem silnější
    a odlišnou ještě v jednom směru. V případě \emph{gravitace jde vždy o přitahování}. Představme
    si však, že existují dva druhy „věcí“ a tato nová síla  (samozřejmě elektrické povahy) má tu
    vlastnost, že věci stejného druhu se odpuzují a věci různého druhu se přitahují. „Předmět“, jenž
    je nositelem tohoto silného vzájemného působení, se nazývá \emph{náboj}.  
    
    K čemu jsme došli? Předpokládejme, že máme dvě věci různého druhu, jež se vzájemně  
    přitahují (plus a minus) a které drží těsně u sebe. Předpokládejme, že v určité vzdálenosti od
    uvedené dvojice máme další náboj. Bude tento náboj pociťovat přitažlivost? Mají-li první dva
    náboje stejnou velikost, neměl by pocítit \emph{prakticky žádnou přitažlivost}, protože
    přitahování jedním nábojem a odpuzování druhým nábojem se vykompenzují. Ve velkých vzdálenostech
    je tedy síla velmi malá. Když třetí náboj \emph{hodně přiblížíme} k prvním dvěma, objeví se
    přitahování, protože odpuzování stejných nábojů a přitahování různých se snaží oddálit stejné
    náboje a přiblížit různé. Odpuzování bude nakonec \emph{slabší} než přitahování. To je příčina,
    proč atomy, které se skládají z kladných a záporných elektrických nábojů, na sebe téměř nepůsobí
    (zanedbáme-li gravitaci), jsou-li od sebe dost vzdáleny. Když se ale přiblíží, mohou
    „\emph{vidět jeden do druhého}“, přeskupit své náboje a velmi silně vzájemně působit. Podstatou
    interakce mezi atomy je \emph{elektrické} působení. Tato síla je tak veliká, že všechny plusy a
    minusy se obvykle dostávají do tak těsné kombinace, jak je to jen možné. Všechny věci, včetně
    nás samotných, se skládají z drobných, velmi silně interagujících kladných a záporných částic,
    které jsou velmi přesně vyvážené. Na okamžik je možné náhodou odstranit několik minusů nebo
    plusů (obvykle je jednodušší odstranit minusy), v tu chvíli jsou elektrické síly
    \emph{nevyvážené} a můžeme pozorovat působení elektrické přitažlivosti.
    
    Abychom si vytvořili představu o tom, o kolik je elektrické působení silnější než gravitace,
    představme si dvě zrnka písku, která mají jeden milimetr v průměru a jsou vzdálená třicet metrů.
    Kdyby elektrické síly mezi nimi nebyly vyvážené, kdyby nebylo odpuzování a vše se navzájem
    přitahovalo a nic se nekompenzovalo, jakou silou by se zrnka přitahovala? Byla by to síla tří
    miliónů tun. Jistě chápete, že pro vytvoření značného elektrického působení stačí velmi malý
    přebytek nebo nedostatek záporných nebo kladných nábojů. Proto není vidět rozdíl mezi elektricky
    nabitým a nenabitým předmětem - pro nabití předmětu je třeba tak málo částic, že se téměř
    neprojeví na jeho hmotnosti, či rozměru.
    
    S těmito poznatky bylo jednodušší pochopit atomy. Předpokládalo se, že mají uprostřed
    „\emph{jádro}“, které je kladně elektricky nabité a velmi těžké, a toto jádro je obklopeno
    určitým počtem „elektronů“, jež jsou velmi lehké a záporně nabité. Teď trochu pokročíme v našem
    výkladu a poznamenáme, že v samotných jádrech byly objeveny dva druhy částic - \emph{protony} a
    \emph{neutrony}, které mají téměř stejnou, velmi velkou hmotnost. Protony jsou elektricky nabité
    a neutrony jsou neutrální. Máme-li atom se šesti protony v jádře, které je obklopeno šesti
    elektrony (záporné částice obyčejného světa jsou všechno elektrony a ty jsou velmi lehké v
    porovnání s protony a neutrony, které tvoří jádra), půjde o atom číslo šest v chemické tabulce a
    tento atom se nazývá uhlík. Atom číslo osm se nazývá kyslík atd. Chemické     
    vlastnosti závisí na vnějších elektronech, ve skutečnosti jen na tom, kolik má atom elektronů.
    \emph{Chemické vlastnosti} látek tedy závisí na jediném čísle, na \emph{počtu elektronů}.
    (Seznam prvků sestavený chemiky by se mohl nahradit očíslováním 1, 2, 3, 4, 5 atd. Místo toho,
    abychom říkali „uhlík“, stačilo by říci „prvek číslo šest“, což by znamenalo, že prvek má šest
    elektronů. Při objevování prvků však tato skutečnost nebyla známa a dále, při číslování by vše
    vypadalo velmi složitě. Proto je lepší ponechat prvkům názvy i symboly a nedožadovat se pouhého
    očíslování.)

    \begin{figure*}[ht!] %\ref{fyz:fig0006}
      \centering
      \luafigure[1]{fyz_fig0006.pdf}
      \caption{Elektromagnetické spektrum (někdy zvané Maxwellova duha) zahrnuje elektromagnetické
              záření všech možných vlnových délek. Srovnání délek elektromagnetických vln s běžnými
              předměty a odpovídající teplotní stupnice umožňuje lépe získat představu o jejich
              rozměrech a energiích.}
      \label{fyz:fig0006}
    \end{figure*}
    
    O elektrické síle bylo získáno mnoho dalších poznatků. Bylo by přirozené předpokládat, že
    elektrická interakce je jednoduché přitahování dvou předmětů: kladného a záporného. Zjistilo se
    však, že toto není úplně vhodná představa. Situaci lépe vystihuje představa, že existence
    kladného náboje v prostoru způsobuje jeho jisté \emph{zakřivení}, vytváří v něm určitou
    „podmínku“, aby záporný náboj vložený do tohoto prostoru cítil působení síly. Tato možnost
    vzniku síly se nazývá \emph{elektrické pole}. Dostane-li se elektron do elektrického pole, je
    jakoby „tažen“. Přitom platí dvě pravidla: a) \emph{náboje vytvářejí pole}, b) \emph{v poli
    působí na náboje síly a náboje se pohybují}. Příčina takového chování se stane jasnější, jakmile
    rozebereme následující jev: Nabijeme-li těleso elektricky, například hřeben, a do určité
    vzdálenosti položíme nabitý ústřižek papíru, přičemž začneme hřebenem pohybovat sem a tam, bude
    se papír natáčet směrem k hřebenu. Zrychlíme-li pohyb hřebenu, zjistíme, že papír zaostává,
    působení se opožďuje. (V prvním stádiu, když pohybujeme hřebenem poměrně pomalu, zkomplikuje nám
    situaci \emph{magnetizmus}. Magnetické vlivy se projevují, když jsou \emph{náboje v relativním
    pohybu}, takže magnetické a elektrické síly je možné skutečně připsat jedinému poli jako dvě
    stránky jedné věci. Měnící se elektrické pole nemůže existovat bez magnetizmu.) Oddálíme-li
    nabitý papír, zpoždění je větší. V tu chvíli pozorujeme zajímavou věc. Ačkoliv se síly působící
    mezi dvěma nabitými předměty mění nepřímo úměrně čtverci vzdálenosti, při kmitání náboje
    zjišťujeme, že jeho působení se rozprostírá mnohem dále, než by se dalo očekávat. Pokles tohoto
    působení je mnohem pomalejší než při nepřímé úměrnosti čtverci vzdálenosti.
    
    S analogickou situací se setkáváme, když na vodě plave splávek a my ho uvedeme do pohybu „přímo
    “ tím, že způsobíme pohyb vody jiným splávkem. Kdybychom se dívali jen na dva splávky,
    pozorovali bychom pouze to, že jeden se dává do pohybu jako odezva na pohyb druhého, že mezi
    nimi existuje určitá „  interakce“. Ve skutečnosti jsme ale rozčeřili vodu a voda posunula druhý
    splávek. Mohli bychom zformulovat „zákon“, že i při slabém zčeření vody se na vodě budou
    pohybovat předměty nacházející se blízko zdroje zčeření. Kdyby byl druhý splávek dost daleko,
    sotva by se dal do pohybu, neboť jsme uvedli vodu do pohybu jen v jednom místě. Bude-li však
    druhý splávek pravidelně kmitat, vznikne nový úkaz, při kterém se pohyb vody přenáší dál, vzniká
    \emph{vlnění} a vliv poskakujícího splávku již nemůžeme chápat jako přímé působení mezi splávky.
    Myšlenku přímé interakce tedy musíme nahradit předpokladem o existenci vody nebo v případě
    elektrických nábojů tím, co nazýváme \emph{elektromagnetickým polem}.
    
    Elektromagnetické pole může přenášet vlny. Některé z těchto vln jsou světlo jak je znázorněno na
    obrázku \ref{fyz:fig0006}, jiné se používají při rádiovém vysílání, ale obecně se nazývají
    \emph{elektromagnetickými vlnami}. Tyto vlny mohou mít rozmanité \emph{frekvence}. Jediné, čím
    se jedna vlna liší od druhé, je právě frekvence vlnění. Kdybychom pohybovali nábojem sem a tam a
    dělali bychom to stále rychleji a rychleji, objevovala by se celá řada různých jevů, které je
    možné systematizovat udáním čísla vyjadřujícího počet kmitů za sekundu. Frekvence, s nimiž
    přicházíme do styku prostřednictvím běžných rozvodových elektrických sítí v domech, jsou řádově
    sto kmitů za sekundu. Zvýšíme-li frekvenci na \SI{500}{\kHz} nebo \SI{1000}{\kHz} (\SI{1}{\kHz}
    = 1000 kmitů za sekundu), dostáváme se z domů ven, „na vzduch“, neboť máme co činit s
    frekvencemi používanými při rozhlasovém vysílání. (Se vzduchem to ale nemá co dělat! Rádiové
    vlny se mohou šířit i v prostoru, v němž není vzduch.) Zvyšujeme-li frekvenci, dostáváme se do
    oblasti \emph{VKV} a televizního vysílání. Při ještě vyšších frekvencích máme velmi krátké vlny,
    které se využívají např. v \emph{radiolokaci}. Kdybychom šli ještě výše, nepotřebovali bychom už
    zařízení na registraci takových vln, protože bychom je viděli naším zrakem. Kdybychom dokázali
    pohybovat nabitým hřebenem tak rychle, aby kmital s frekvencemi od \SI{5e14}{\Hz} do
    \SI{5e15}{\Hz}, viděli bychom toto kmitání jako červené, modré nebo fialové světlo v závislosti
    na frekvenci. Frekvence pod touto oblastí nazýváme \emph{infračervenými} a nad touto oblastí
    \emph{ultrafialovými}. Skutečnost, že naše vidění je omezeno na určitou frekvenční oblast,
    nedělá tuto oblast elektromagnetického spektra z fyzikálního hlediska důležitější než jiné
    oblasti, avšak z lidského hlediska je tato oblast přece jen zajímavější. Kdybychom frekvenci
    ještě zvýšili, dostali bychom \emph{rentgenové paprsky}. Tyto paprsky nejsou nic jiného, než
    světlo s velmi vysokou frekvencí. Ještě vyšším frekvencím odpovídá \emph{záření gama}. Výrazy
    rentgenové paprsky a záření gama jsou téměř synonyma. Zářením gama nazýváme obvykle
    elektromagnetické vlny pocházející z jader a rentgenovými paprsky vlny pocházející z atomů; při
    shodě jejich frekvencí jsou však fyzikálně nerozlišitelné, bez zřetele na jejich původ. Vlny
    ještě vyšších frekvencí, řekněme \SI{10e24}{\Hz}, lze získat uměle, například na
    \emph{synchrotronu} v Caltechu. Elektromagnetické vlny úžasně vysokých frekvencí (až tisíckrát
    vyšších) je možné najít ve vlnách \emph{kosmického záření}. Tyto vlny však neumíme ovládat. 
    \cite[s.~29]{Feynman02}

    \subsection{Vědecká revoluce 17. století}\label{fyz:IchapIIsecIIIssecI}
      \textbf{Klasická fyzika}, jak ji popsal Richard Feynnman v předchozích kapitolách, tedy jako
      věda vycházející z měření a experimentů a opírající se o matematickou teorii, věda, která nám
      podává ucelený obraz přírody a světa a svými výsledky slouží technickému pokroku, vznikla v
      Evropě v průběhu sedmnáctého století. Tento dějinný převrat, který předznamenal naši dnešní
      civilizaci, nazýváme \textbf{obdobím vědecké revoluce}. Nebyla to ovšem nějaká náhlá událost a
      lidé si tehdy ani neuvědomili, jakou vlastně prožívají dobu a co přinese budoucím pokolením.
      Vědecká revoluce nastala za zvláštních podmínek evropského vývoje, které se v jiných částech
      světa nevytvořily.

      Příčin, které vyvolaly tuto vědeckou revoluci, bylo mnoho a nemůžeme je zde podrobně zkoumat.
      Především to byly nové politické a hospodářské podmínky, potřeby výroby, obchodu a podnikání,
      které vyzvedly do popředí nové společenské síly, především měšťanské. Vzrůstající produktivita
      práce a vznik prvních kolektivních dílen, manufaktur, potřebovaly nové způsoby silového
      pohonu. Zásobování surovinami a vývoz hotových výrobků si vyžádal rozvoj mořeplavby a námořní
      navigace. Evropské války, jak už to bývá, také podnítily zdokonalování vojenské techniky a
      nepřímo i rozvoj přírodních věd \cite[s.~137]{Stoll2009}.

      Důležitou úlohu sehrála reformace, odklon řady zemí v západní a severní Evropě od katolické
      církve a papežství a vznik nových, protestantských církví. Protestantismus usiloval o bližší
      kontakt jednotlivého člověka s Bohem, bez prostřednictví církevní hierarchie, o návrat k
      podobě bible v jejich původních jazycích (hebrejském a řeckém) a podnítil vznik překladů
      biblických textů do národních evropských jazyků. Tím na jedné straně vyvolal potřebu studia
      klasických jazyků a umožnil také zpřístupnění výsledků vědy starověkého Řecka a na druhé
      straně podpořil rozvoj národních jazyků (připomeňme si jen krásnou češtinu naší Kralické
      bible). Latina, ve středověku univerzální jazyk vzdělanců, začala ztrácet své výsadní
      postavení.

      S rostoucím vědomím užitečnosti a nutnosti vědeckého poznání bez vměšování teologického
      dogmatismu přenášejí protestanti těžiště náboženského cítění do oblasti morální, jako vodítko
      při hledání smyslu lidského života, a ponechávají přírodním vědám zkoumání a využívání
      přírodních zákonů. Odmítají víru v Boží zázraky, která vlastně znemožňuje existenci vědy.
      Takový přístup, kdy Bůh je chápán jen jako stvořitel a první zákonodárce, který se však do
      dalšího chodu přírody už nevměšuje, nazýváme \textbf{deismus}, na rozdíl od katolického
      teismu, podle něhož Bůh do běhu světa stále zasahuje a bez jehož vůle, ani vlas z hlavy
      nespadne“. Protože nositeli idejí protestantismu byly především měšťanské a hospodářsky
      aktivní vrstvy společností, rozvíjí se věda a vědecká revoluce zejména v protestantských
      zemích západní Evropy v Holandsku, Anglii, Švýcarsku, Dánsku, částečně ve Francii a Německu.
      Také příznačný podnikatelský duch Ameriky má své kořeny v anglosaském protestantismu prvních
      přistěhovalců. Katolická Itálie, která renesanci vědy zahájila, nakonec odsoudila svého
      Galilea, katolické Španělsko a Portugalsko, které zbohatly při zámořské kolonizaci, postupně
      svou moc ztrácejí a k vědecké revoluci v Evropě nepřispívají.

      Evropa nebyla nikdy soběstačná v některých druzích výrobků, ať už šlo o tropické plody,
      rostliny (bavlna, cukrová třtina) a koření, drahé kovy, ale třeba i hedvábí, vzácné kožešiny a
      jiné výnosné luxusní předměty. Obchodní cesty k jejich získávání vedly odedávna přes
      Středozemí, Blízký a Střední Východ a na tomto obchodu bohatly zejména italské městské státy
      jako Benátky nebo Janov. Když postupující turecká expanze tyto přístupové cesty znesnadnila a
      ohrozila, hledaly státy západní a jihozápadní Evropy přístup na východní trhy obeplutím Afriky
      a po úspěšných výpravach Kolumbovych západním směrem do Ameriky.

      Španělsko a Portugalsko začaly z těchto nových cest a výbojů těžit jako první, jejich karavely
      a galeony, obtížené kořením, stříbrem a zlatem, přivážely toto zboží na evropské trhy, pokud
      neskončilo na mořském dně nebo v rukou pirátů. Obě tyto námořní mocnosti si známými smlouvami
      z Tordesillas (1494) a Zaragozy (1529) dokonce rozdělily celý svět na dvě poloviny a
      uskutečnily tak první globalizaci světového obchodu a kolonizace ve znamení katolicizmu.

      Nedokázaly však své nové hospodářské zdroje produktivně využít. Jejich pozice zaujala postupně
      Anglie, Francie, a zejména malé protestantské Holandsko, které se začátkem 17. století
      osvobodilo od španělské nadvlády a vytvořilo republiku pod vládou místodržitelů z rodu
      Oranžskeho je téměř neuvěřitelné, že Holandsko, počtem obyvatel srovnatelné s tehdejším českým
      královstvím, vytvořilo jeden čas největší koloniální říši světa a disponovalo flotilou 16 000
      lodi, počtem trojnásobně převyšujícím flotilu všech ostatních západoevropských států
      dohromady. Hospodářsky se postupně vzmáhala i Anglie, kde společenské napětí vyvrcholilo
      občanskou válkou a revoluci, která přivedla v roce 1649 krále Karla I. na popraviště. Všechny
      tyto společenské otřesy a změny v západní Evropě postupně vytvářely nové impulzy k rychlému
      vědeckému a technickému pokroku.

      Koloniální výboje vyvolaly potřebu mapovat nová území, dokonce mapovat zeměkouli jako celek,
      především přesně měřit zeměpisnou šířku a délku, ale i hloubku moří, teplotu a slanost mořské
      vody, rychlost a směr mořských proudů a magnetickou deklinaci, odchylku směru udávaného
      kompasem od pravého severu. To ovšem vyžadovalo prozkoumat přesný geometrický tvar zeměkoule a
      vytvořit nové fyzikální a astronomické měřicí metody a přístroje.

      Největší problém činilo určování zeměpisné délky. Dokud se Evropané ve starověku a středověku
      plavili v útulném Středomoří, kde bylo možno z každého místa doplout za jeden den k
      nejbližšímu pobřeží, nebo když Vikingové provozovali pobřežní plavbu podél západoevropských
      břehů, nebyla tato otázka příliš naléhavá. Jakmile se ovšem Kolumbus vydal na neprobádanou
      cestu na západ Atlantickým oceánem a začal překračovat další a další poledníky, mohl určovat
      svou polohu jen podle rychlosti lodi, měřené nedokonalým plavboměrem, a porovnávat místní čas
      s časem ve výchozím přístavu, odměřovaným přesýpacími hodinami. Ty měl plavčík za úkol každou
      čtvrthodinu převracet, a záleželo tak i na jeho problematické svědomitosti. Kolumbus ostatně
      údaje o zeměpisné délce sám upravoval, aby posádka neměla představu, jak daleko na západ už
      dopluli. Dost na tom, že námořníci byli vyděšeni tím, že jim střelka kompasu přestala ukazovat
      na Polárku.

      Když si však někdo chce dělit zeměkouli napůl, musí být schopen určovat zeměpisnou délku
      přesně. Potřebuje k tomu dalekohled, sextant, astronomické znalosti a přesné lodní hodiny -
      chronometr. To si uvědomil dokonce i anglický král Karel II., když se na něj v roce 1675
      obrátil astronom \textsc{John Flamsteed} (1646-1719) s návrhem na zřízení státní, tedy
      královské hvězdárny. V královském rozhodnutí se založení hvězdárny výslovně zdůvodňuje
      \emph{„aby bylo možno zjišťovat zeměpisnou délku míst ke zdokonalení navigace a astronomie."}
      Král se dokonce vzdal svého honebního revíru na stráni v Greenwichi na pravém břehu Temže
      (byla stejně holá a málo zvěřinatá) a souhlasil s tím, aby tam byla z použitého stavebního
      materiálu vybudována observatoř. Zároveň zavedl novou funkci a jmenoval Flamsteeda
      ,,královským astronomem". Ten musel investovat do vybavení hvězdárny své vlastní finanční
      prostředky a v podstatě živořil. Současně vznikla ve Francii i královská pařížská observatoř,
      kam byl z Itálie povolán astronom \textsc{Giovanni Domenico Cassini} (1625-1712), jehož
      potomci ho následovali v této funkci v několika generacích.

      Vědecká revoluce v Evropě byla tedy vyvolána naléhavými praktickými potřebami, ale měla
      připraveno i myšlenkové, filozofické zázemí. Postupně se prosazoval světový názor založený na
      Koperníkově modelu sluneční soustavy a astronomická měření ho stále přesvědčivěji potvrzovala.
      Vědecká metoda zkoumání se mohla opřít o výsledky práce myslitelů, kteří stoji u počátků
      novověké evropské filozofie. věku evropské filozofie. V Anglii to byl \textsc{Francis Bacon}
      (1561-l626) \textsc{René Descartes} (1596-1650). Oba představuji poněkud odlišné, ale vzájemně
      se doplňující přístupy ke zkoumání přírody a charakterizují různé směry, jimiž se ubírala
      vzájemně soupeřící anglická a francouzská fyzika té doby. 
      
      Bacon zastával v Anglii vysoké státní funkce. Zdůrazňoval význam vědění, které dává člověku
      obrovskou moc, a zabýval se myšlenkami \uv{velkého obnovení věd}, které by přinášelo lidem
      užitek a přispělo i k lepší organizaci lidské společnosti. Ve svém spise \uv{Nové organon} z
      roku 1620 reaguje na Aristotelovo dílo ,,Organon", odmítá čistě spekulativní, scholastickou
      aristotelovskou logiku a vychází z empirického, smyslového poznání, pozorování a pokusů. Je
      zakladatelem vědecké indukce, tedy metody, která logicky analyzuje a třídí zkušenosti, fakta a
      dospívá k obecným zákonitostem. Přitom se vědec musí oprostit od předsudků a vžitých představ,
      které Bacon nazývá  \uv{idoly}. Ve svém zaujetí pro pokusy šel Bacon tak daleko, že zemřel na
      zápal plic právě když zkoumal dlouhodobý vliv chladu na živý organismus. Bacon je
      představitelem anglického empirismu, který zapůsobil i na anglické fyziky včetně Newtona.  
      
      Ve Francii ovlivnil filozofické myšlení především Descartes (latinsky Kartesius). Pocházel z
      aristokratického katolického rodu, od dětství byl chabého zdraví a prošel složitým myšlenkovým
      vývojem. Navštěvoval jezuitskou kolej, studium ho však neuspokojilo, a naopak v něm rozvířilo
      mnoho pochyb. Studoval práva i medicínu, jako dobrovolník v holandském a pak v bavorském
      vojsku prošel Evropou i Čechami a někdy se uvádí, že se účastnil i bitvy na Bílé hoře.  Na
      dlouhých dvacet let pak zakotvil v Holandsku, kde se v červenci 1642 sešel i s Janem Amosem
      Komenským, i když se s ním filozoficky nepohodl. Descartovy názory narážely na odpor a
      vyvolávaly útoky ze strany jak katolických, tak protestantských kruhů a tyto útoky poněkud
      plachého Descarta deprimovaly. Descartes byl zastáncem Koperníkova názoru na sluneční
      soustavu, ale po Galileově odsouzení se zalekl a byl ve formulaci svých názorů vysloveně
      opatrný. Aby si zajistil větší klid k práci, často dokonce měnil místo svého pobytu. Jeho
      vědecké dílo mělo i řadu stoupenců a vzbudilo nakonec zájem švédské královny Kristýny. Pozvala
      Descarta do Stockholmu a ten ji musel vyučovat filozofii třikrát týdně od pěti hodin ráno.
      Descartes, který byl zvyklý vstávat až k poledni, takový režim, znásobený drsným severským
      podnebím, ovšem dlouho nepřežil. V únoru 1650 zemřel na zápal plic a v r. 1666 byly jeho
      ostatky převezeny do Paříže. Dnes je pohřben ve starobylém kostele Saint Germain-des-Prés,
      jeho lebka, která byla při převozu ostatků zcizena, odděleně v Museu člověka v Paříži. 
      
      Descartes je zakladatelem francouzského racionalismu. Je znám jeho výrok \uv{Cogito erg sum},
      \uv{Myslím, tedy jsem} a na základě rozumových úvah také založil svou vědeckou metodu. Ve svém
      slavném spise \uv{Rozprava o metodě} stanoví pravidla správného vědeckého uvažování. Jako
      první krok požaduje zpochybnit všechny dosavadní názory a tvrzení, pokud nejsou nade vší
      pochybnosti Descartes je zakladatelem francouzského \emph{racionalizmu}. Je znám jeho v rok
      \uv{Cogito ergo sum}, \uv{Myslím, tedy jsem} a na základě rozumových úvah také založil svou
      vědeckou metodu. Ve svém slavném spise \uv{Rozprava o metodě} stanoví pravidla správného
      vědeckého uvažování. Jako první krok požaduje zpochybnit všechny dosavadní názory a tvrzení,
      pokud nejsou nade vší pochybnost dokázány. Jeho \uv{De omnibus dubitandum}, \uv{O všem
      pochybovat}, znamená začínat zkoumání s čistou a nepředpojatou myslí. Dále požaduje rozdělit
      každou zkoumanou otázku na části, které by bylo možno lépe řešit. Při zkoumání je třeba
      postupovat od předmětů jednodušších, které lze snáze poznávat, ke složitějším. A konečně za
      čtvrté je třeba uspořádávat zjištěná fakta do výčtů a přehledů, aby nic nebylo opomenuto. Tato
      Descartova doporučení jsou jakýmsi základem vědecké metody rozumového zkoumání; týmž způsobem
      musí ostatně postupovat i detektiv při řešení složitého kriminálního případu. Descartes je tak
      zakladatelem analytické deduktivní metody, která vychází z několika málo obecných principů a
      zákonů a postupuje podle pravidel rozumového uvažování.

      Descartova filozofie, karteziánství, ovlivnila celou řadu pozdějších filozofů a myslitelů.
      Patřil k nim např. \textsc{Benedikt (Baruch) Spinoza} (1632-1677), holandský filozof
      portugalsko-židovského původu, ale i Leibniz, Pascal a další. Spinoza se pokusil pomocí
      Descartovy racionalistické filozofie a axiomatické metody geometrie vyložit i taková témata,
      jako je politika, etika nebo teologie. Dochází k závěru, že existuje jen jedna jediná
      substance, jíž je Bůh ztotožněný s přírodou. Takový názor, podle něhož se nic a nikdo do
      přírody zvnějšku nevměšuje se nazývá \emph{panteizmem}. Zmiňujeme se o něm proto, že je blízký
      chápání velkých fyziků. Ti byli uchváceni krásou a řádem přírody, a ta jim splývala s božstvím
      v jedno. Ke Spinozově panteizmu se hlásil např. i Einstein \cite[s.~141]{Stoll2009}.    
%---------------------------------------------------------------------------------------------------  
  \section{Moderní fyzika}\label{fyz:IchapIIsecIV}
    \subsection{Kořeny nové fyziky}\label{fyz:IchapIIsecIVssecI}
      Na konci 19. století se jevilo, že klasická fyzika opírající se o Newtonovu mechaniku,
      Maxwellovu teorii elektromagnetizmu a Maxwellovo-Boltzmannovo statistické pojetí termodynamiky
      je v podstatě uzavřenou vědou, která umožňuje vysvětlit všechny fyzikální přírodní jevy a
      vytvořit ucelený obraz světa. \uv{Klasičtí} experimentální fyzici usilovně a svědomitě
      prováděli svá měření, aby upřesnili vlastnosti látek a jejich chování v elektrických a
      magnetických polích, vlastnosti světla a nově objevených neviditelných paprsků a dlouho
      ověřovali získané poznatky, než se je odhodlali publikovat. Podle Newtonova hesla \uv{hypotézy
      si nevymýšlím} odmítali neobvyklé spekulace a nové matematické teorie a v duchu pozitivistické
      filozofie nechtěli za zjištěnými fakty vidět nějaké další hlubší jevy a příčiny nedostupné
      přímému pozorování. Pokud jde o smělé matematické modely, většina experimentálních fyziků
      neměla k jejich pochopení a přijetí ani dostatečné znalosti a předpoklady. Kdyby Hertz nebyl
      býval včas objevil elektromagnetické vlny, zůstávala by zřejmě i elegantní Maxwellova teorie
      elektromagnetizmu jen intelektuálním výtvorem pro vyvolené.

      Přitom si mnozí klasičtí experimentátoři ani neuvědomovali, že právě svými přesnými a
      svědomitě prováděnými výzkumy vlastně začínají klasickou fyzikou otřásat a připravují půdu
      fyzice nové, jejíž rodící se koncepce nechtěli brát na vědomí. Krásným příkladem může být
      postoj klasického fyzika \textsc{Wilhelm Conrad Röntgena} (1845 - 1923), který se v dopise
      svému spolupracovníku \textsc{Ludwigu Zehnderovi} (1854 - 1949) z r. 1896 přiznává, že ho
      podstata rentgenových paprsků vlastně nezajímá. \emph{\uv{Jaké povahy ty paprsky jsou, je mi
      zcela nejasné, pátráni po tom, jde-li opravdu o podélné světelné paprsky, přichází pro mě v
      úvahu až v druhé fázi. Hlavní věcí jsou skutečnosti}}. Typický je i výrok stařičkého Röntgena
      z období kolem roku 1920, kdy si už teorie relativity a kvantová fyzika razily cestu moderní
      fyzikou: \emph{\uv{Stále mi nechce jít do hlavy, že musí člověk používat tak zcela
      abstraktních úvah a pojmů, aby vysvětlil přírodní jevy}}
      
      K ilustraci toho, jak viděli situaci na přelomu dvou století velcí klasičtí fyzici, bývá
      nejčastěji citován výrok \textsc{Williama Thomsona} (1824 - 1907) alias lorda Kelvina o dvou
      obláčcích z 27. dubna 1900. Tehdy tento uznávaný koryfej fyziky na přednášce v Královském
      institutu prohlásil:
      
      
      \emph{\uv{Krása a jasnost dynamické teorie, která vysvětluje teplo a světlo jako druhy pohybu,
      je v současné době zahalena dvěma obláčky. První se týká otázky, jak se pohybuje Země v
      pružném prostředí, které představuje v podstatě světelný éter. Druhý je
      Maxwellovo-Boltzmannovo pojetí rozdělení energie.}}

      \subsubsection{Problém spektra záření absolutně černého tělesa}\label{fyz:IchapIIsecIVssecIsssecI}

        Pod druhým obláčkem měl Thomson na mysli problém spektra záření absolutně černého tělesa,
        které se nedařilo teoreticky vysvětlit v souhlase s experimentálními měřeními. Thomson zde
        ukázal lví dráp své fyzikální intuice, když vytypoval právě dva zdánlivě nenápadné problémy,
        jejichž řešení přivedlo brzy poté k hlubokému přezkoumání celé klasické fyziky a vzniku
        fyziky relativistické a kvantové. Byl však přílišným optimistou - obláčků nad klasickou
        fyzikou bylo víc, ba obloha byla téměř zatažena.
    
      \subsubsection{Problém stáčení perihelia Merkuru}\label{fyz:IchapIIsecIVssecIsssecII}  

        O třetím obláčku, který souvisel se \textbf{stáčením perihelia Merkuru} a který klasická
        Newtonova mechanika nedokázala vysvětlit, jsme se už zmiňovali. O tom, že tato zdánlivá
        maličkost fyziky v té době opravdu znepokojovala, svědčí stať o Merkuru v Ottově slovníků
        naučném z roku 1901 od \textsc{Gustava Grusse}\footnote{český astronom} (1854-1922),
        asistenta \textsc{Ernsta Macha} (1838 - 1916). Uvádí se v ni: \emph{\uv{Merkur (Dobropán)
        jest planeta Slunci nejbližší a nejmenší …Leverrier shledal, že přísluní dráhy Merkurovy se
        ve století o 40" rychleji pohybuje, než by se mělo pohybovati dle gravitačního zákona.
        Odchylku tu hleděl Leverrier vysvětliti působením jedné aneb celé skupiny malých
        intramerkuriálních planet. Pozorovaný pohyb přísluní nedal se však posud vysvětliti ani
        působením a objevením jediné domnělé oběžnice (Vulcana"), ani skupiny malých planetoid,
        nalézajícich se uvnitř dráhy Merkurovy, ani působením světla zodiakálního.}} 

        \luagraphic[0.7]{fyz_fig0944.png}{Schematický nákres záhadného fenoménu. Uvedený obrázek je
        pro názornost extrémně zkreslený. Reálné stáčení Merkurova perihelia je řádově menší (na
        obrázku by nebylo rozeznatelné) a navíc jeho převážnou část způsobuje gravitace ostatních
        planet Sluneční soustavy. Přesto zůstává obloukových 43“ za století, jež klasická mechanika
        své doby neuměla vysvětlit.}{fyz:fig0944}

      Z hlediska klasické mechaniky nevysvětlitelnou zůstávala otázka vzniku záření, ať už šlo o
      \textbf{vyzařování zahřátých těles}, chladné světélkování při \textbf{luminiscenci krystalů}
      nebo vyzařováni jednotlivých \textbf{spektrálních čar} atomy. To byl tedy skutečný mrak nad
      klasickou fyzikou. Zatímco sama existence atomu nebyla ještě s konečnou platnosti prokázána,
      existovaly už dosti dobré odhady jejich velikosti hmotnosti a bylo známo, že atomy různých
      prvků vydávají po zahřátí charakteristická spektra záření v oblasti infračervené, ultrafialové
      a viditelného světla.

      \subsubsection{Problém spektrálních čar atomů}\label{fyz:IchapIIsecIVssecIsssecIII}

        Již v roce 1870 si irský fyzik \textsc{Johnstone Stoney}\footnote{Zavedl termín elektron
        jako „základní jednotku množství elektřiny“.} (1826 - 1911) všiml, že některým
        Fraunhoferovým čarám slunečního spektra odpovídají frekvence, které jsou v poměru celých
        čísel podobně jako u harmonických hudebních intervalů. Jako by se ve světě atomů opět
        ozývala pythagorejská a keplerovská \uv{hudba sfér}. Pozornost se soustředila na spektrum
        vodíku jako na nejjednodušší. Jeho spektrální čáry vytvářejí skupiny, série podle vlnových
        délek, rozestupy mezi čarami v sérii se postupně zmenšují a čáry se zhušťuji až k tzv. hraně
        série. Tak např. pro jednu sérii čar vodíkového spektra byly naměřeny vlnové délky
        \SI{656.2}{\nm} (červená), \SI{486.1}{\nm} (modrá), \SI{434.0}{\nm} (modrofialová),
        \SI{410.1}{\nm} (fialová) a dále se čáry zhušťovaly až k hraně \SI{364.6}{\nm}.

        Bylo zřejmé, že v takovýchto posloupnostech čísel se musí tajit nějaká matematická
        zákonitost, přijít na ni nebylo ovšem snadné. Problém zaujal švýcarského matematika
        \textsc{Johanna Jacoba Balmera} (1825-1898), jehož koníčkem byla numerologie a který ve
        volném čase počítal ovečky a schody. Balmer studoval matematiku v Basileji, Karlsruhe a v
        Berlíně, pak učil v Basileji na pedagogickém ústavu a v dívčí škole a na univerzitě
        přednášel jako soukromý docent deskriptivní geometrii. Po značné námaze dospěl k formuli,
        která zvěčnila jeho jméno - zjistil, že vlnové délky jsou úměrné výrazu
        \begin{equation*}
          \lambda = B\left(\dfrac{n^2}{n^2-m^2}\right) = B\left(\dfrac{n^2}{n^2-2^2}\right)
        \end{equation*} 
        kde
        \begin{description}[leftmargin=3em,labelindent=1em, style=nextline]
          \item \(\lambda\) je vlnová délka
          \item \(B\) je konstanta o hodnotě \SI{364.506802}{\nm}
          \item \(m\) je rovno \num{2}
          \item \(n\) je přirozené číslo větší než \num{2}. 
        \end{description}

        Později, ve 20. století, byly objeveny další série spektrálních čar vodíku. V infračervené
        oblasti spektra jednu z nich našel r. 1908 \textsc{Fridrich Paschen} (1865 až 1947),
        spolupracovník Hittorfův a profesor fyziky na univerzitě v Tübingenu v letech 1901-1919,
        později prezident Říšského fyzikálně technického ústavu v Berlíně (1924-1933). Paschenova
        série odpovídá Balmerově formuli pro \(n = 3\) a \(m = 4, 5,\ldots\). Paschen byl vynikající
        odborník na atomovou spektroskopii a náleží mu také objev (spolu s \textsc{Ernstem Backem}
        (1881-1959)) chování spektrálních čar v silných magnetických polích.
    
        \luagraphicx[1]{fyz_fig0945.jpg}{„Viditelné“ čáry spektra emise vodíku v sérii Balmer.
          \(H-\alpha\) je červená čára vpravo. Čtyři čáry (počítané zprava) jsou formálně ve
          viditelném rozsahu. Čáry pět a šest lze vidět pouhým okem, ale jsou považovány za
          ultrafialové, protože mají vlnové délky menší než \protect\SI{400}{\protect\nm}. Kredit:
          Wikipedia}{fyz:fig0945}
        
        Další série spektrálních čar vodíku našli postupně američtí spektroskopici -v infračervené
        oblasti \textsc{Frederic Summer Brackett} (1896-1953) v roce 1922 pro \(n = 4\),
        \textsc{August Herman Pfund} (1879-1949) v roce 1924 pro \(n = 5\), \textsc{Curtis J.
        Humphreys} (1898-1986) v roce 1953 pro \(n = 6\) a v ultrafialové oblasti \textsc{Theodore
        Lyman} (1874-1954) v letech 1906-1914 pro \(n =1\). Zákonitosti rozložení čar ve spektrech
        jiných prvků zkoumali švédský fyzik \textsc{Johannes Robert Rydberg} (1854-1919) a
        \textsc{Walther Ritz} (1878-1909) původem ze Švýcarska, později Paschenův spolupracovník v
        Tübingenu a docent v Göttingenu. Rydberg a Ritz našli v r. 1908 tzv. \textbf{kombinační
        princip}, podle něhož lze vlnové délky (resp. vlnočty) spektrálních čar určovat jako rozdíly
        dvou veličin z množiny tzv. \textbf{termů} charakteristických pro daný atom.

        Balmerovi se tedy podařilo rozluštit číselnou šarádu, která udávala velmi jednoduchou
        zákonitost rozložení spektrálních čar, ale její smysl klasické fyzice unikal - mechanika ani
        elektrodynamika ji nemohla objasnit. Přitom bylo zajímavé a pro dění v přírodě příznačné, že
        charakteristiky spektrálních čar všech prvků dostaneme tak, že si budeme pohrávat s malými
        celými čísly.
      
      Velký francouzský matematik, fyzik a astronom \textsc{Henri Poincare} (1854 až 1912), profesor
      Sorbonny a ředitel pařížské hvězdárny, jeden z těch, kteří stáli u vzniku teorie relativity, v
      roce 1908 přímo prorocky napsal: \emph{\uv{Výzkum rozložení u spektrálních čar nám hned
      připomene harmonické vztahy v akustice, ale přesto je tu obrovský rozdíl. Jsou to zákony zcela
      jiné povahy. Zatím se v nich ještě dostatečně nevyznáme a myslím, že se v nich skrývá jedno z
      nejdůležitějších tajemství přírody}} (\uv{La valeur de la science}).

      \begin{tcnote}
        \textbf{Henri Poincaré} se narodil do vlivné rodiny. Jeho otec byl profesorem lékařství na
        univerzitě v Nancy (Université de Nancy). Velmi významným členem rodiny byl jeho bratranec
        Raymond Poincaré, který se stal v roce 1913 francouzským prezidentem a zůstal jím po celou
        dobu první světové války až do roku 1920. Raymond Poincaré se stal také členem Francouzské
        akademie.
        
        Roku 1905 dospěl současně s Einsteinem k základním principům speciální teorie relativity.
        Poincaré je významným představitelem kritiky vědy jako sebeuvědomění postupů vědy: zajímala
        ho především otázka původu základních vědeckých přesvědčení. Pohledem dnešních matematiků se
        Poincaré řadí mezi největší matematiky všech dob. Vytvořil důležité odvětví matematiky známé
        jako algebraická topologie. Díky značné šířce vědomostí v matematických oborech, mechanice
        vesmírných těles, fyzice a psychologii, je Poincaré nazýván posledním velkým univerzalistou
        vědy. Těžko říci zda to bylo výhodou, ale Poincaré raději vyvozoval výsledky přímo ze
        základních principů, spíše než by navazoval na dřívější práce jiných vědců nebo dokonce na
        své vlastní.

        {\centering
          \captionsetup{type=figure} 
          \luafigure[0.7]{fyz_fig0946.jpg}
          \captionof{figure}{\wikiPoincare (\textasteriskcentered 	29. 4. 1854 - \textdagger 17. 7.
                    1912)}
          \label{fyz:fig0946}
        \par}

        Poincarého výzkumné zájmy zahrnovaly mnoho oborů matematiky, fyziky a filozofie vědy. Byl
        také jedinou osobou všech dob, která byla zvolena do všech pěti sekcí Francouzské akademie
        věd: geometrie, mechaniky, fyziky, geografie a navigace. V roce 1906 byl navíc prezidentem
        celé Akademie. Široký obzor jeho znalostí a jeho schopnost vidět souvislosti mezi zdánlivě
        velice vzdálenými oblastmi mu umožňovaly nahlížet na problémy z mnoha různých a často nových
        úhlů. Jeho práce ve fyzice obsahují závažné příspěvky k optice, elektřině, telegrafii,
        elasticitě, kosmologii, mechanice tekutin, kvantové teorii a speciální teorii relativity.

        Poincaré dosáhl později během své kariéry dalších výsledků o komplexních číslech a je
        považován za zakladatele nesmírně důležitého oboru analytických funkcí několika komplexních
        proměnných. V různých obdobích svého života také využíval svůj talent ke studiu teorie čísel
        a geometrie.

        Poincarého práce v matematickém oboru zvaném topologie. Právě v ní se zrodil pátý z problémů
        milénia, Poincarého domněnka. Ačkoli počátky topologie sahají až ke Gaussovi a dalším
        matematikům poloviny devatenáctého století, skutečně závažnou disciplínou se stala až v roce
        1895, kdy Poincaré publikoval rozsáhlou studii Analysis Situs, k níž v letech 1899 až 1904
        připsal pět vysvětlujících dodatků (včetně Poincarého domněnky). V této jediné publikaci
        zavedl Poincaré prakticky všechny koncepty a klíčové metody, které se pak staly hnací silou
        oboru po následujících padesát let.
      \end{tcnote}

      \subsubsection{Problém existence éteru}\label{fyz:IchapIIsecIVssecIsssecIV}
        Další velkou nerozřešenou otázkou, s níž si klasická fyzika nevěděla rady, chmurou na čistém
        nebi fyzikálního poznání či přímo balvanem, který bylo třeba odvalit, byla otázka
        \textbf{éteru}, jak správně vytušil William Thomson. Éter, původně pátý, nadměsíční živel
        aristotelovské fyziky, vstoupil do klasické fyziky především jako prostředí zaplňující celý
        vesmír a pronikající všechna průzračná tělesa se šiří světlo podobně jako zvuk ve vzduchu.
        Pokud by tento éter byl nehybný (vzhledem k čemu?), mohl by sloužit jako absolutní vztažná
        soustava podobně jako Newtonův absolutní prostor. Otázka existence, či neexistence éteru
        tedy souvisí i se samotnými základy mechaniky a jako taková se občas vynořuje, zejména v
        kosmické fyzice, dodnes.

        Postupně získávané experimentální výsledky týkající se vlastnosti světla představu o éteru
        komplikovaly. Na jedné straně musel být éter nehybný, nevažitelný a na druhé straně by měl
        mít vlastnosti pružných látek. Přitom musel být schopen pronikat mezi atomy plynů, kapalin a
        pevných látek, např. skla nebo krystalu. Experiment, který by přímo dokazoval jeho
        existenci, samozřejmě chyběl, takže šlo o látku hypotetickou. Newton, který odmítal
        „hypotézy" a světlo si představoval jako proud částic, se sice o éteru občas zmiňoval, ale
        vlastně o jeho existenci pochyboval. Stejně tak odmítal spekulovat o tom, čím je zaplněn
        prostor, v němž se přenášejí gravitační sily. Pro Huygense a Eulera, zastánce vlnové teorie
        světla, představoval éter světlonosné prostředí, i když se onem Euler vyjadřoval dosti
        neurčitě. 

        Důležitý poznatek přinesl Bradleyův objev hvězdné aberace, změny směru světelných paprsků
        přicházejících od hvězd v závislosti na ročním, resp. denním pohybu Země. \textsc{Thomas
        Young} ukázal, že tento výsledek svědčí o tom, že světelný éter je nehybný a volně prochází
        pohybujícími se tělesy. Protože rychlost světla v látkách s větší hustotou klesá, pokusil se
        \textsc{Jean Fresnel} (1788 - 1827) a po něm \textsc{Gabriel Stokes} (1819 - 1903) vysvětlit
        tuto skutečnost tím, že se éter při vnikání do látkového prostředí zhušťuje a při opouštění
        tohoto prostředí se opět \emph{zřeďuje}. Jeho pružné vlastnosti se přitom nemění. Objev
        polarizace světla přivedl Fresnela k závěru, že světelné vlny jsou příčné. Takové vlny se
        ovšem šiří v pružných tělesech nebo na povrchu vody a je obtížné si představit jemnou látku,
        která nic neváži, přitom se chová jako ocelová tyč. Tento rozpor se pokusil vysvětlit Stokes
        a pomoci velmi umělých předpokladů.

      S nástupem Maxwellovy teorie elektromagnetismu se ukázala souvislost mezi elektromagnetickými
      jevy a světlem a éter by se tak vlastně měl stát i nositelem elektromagnetických vln. Znovu se
      vynořila otázka, zda éter zůstává při pohybu těles nehybný nebo je jimi částečně či úplně
      strháván: šlo tedy o elektrodynamiku pohybujících se těles. Touto otázkou se zabýval, jak
      víme, francouzský fyzik Fizeau, když měnil rychlost světla v proudící vodě Faraday a Maxwell
      jako tvůrci koncepce elektromagnetického pole byli vůči éteru dost skeptičtí. Maxwell soudil,
      že otázku strhávání éteru by musel vyřešit jen nový, přesný experiment, a sám se o něj
      nepokoušel, neboť pochyboval, že takové přesnost měřeni lze dosáhnout. Nejrychleji se
      pohybující těleso, které bylo dispozici, byla sama Země, která se řití na oběžné dráze kolem
      Slunce, tedy v kosmickém éteru, rychlosti přibližně \SI{30}{\km\per\s}, desettisíckrát
      pomaleji než světlo. Vyšleme-li světelný paprsek ve směru pohybu Země a ve směru kolmém a
      necháme-li pak tyto paprsky interferovat, objeví se světlé a tmavé interferenční proužky.
      Je-li dráhový rozdíl obou paprsků způsoben pohybem Země vzhledem k éteru, mělo by při
      pootočeni přístroje o pravý úhel kolem svislé osy dojít k posunu těchto proužků. Jde ovšem o
      efekt řádu \num{e-8}, který vyžaduje přesnost měření na miliontinu procenta. 

      Tento úkol se stal výzvou pro \textsc{Alberta Abrahama Michelsona} (1852 až 1931), původem z
      malého městečka Strzelna v tehdejší pruské části Polska. Ve dvou letech Michelson emigroval s
      rodiči do Kalifornie, kde tenkrát právě zuřila zlatá horečka. V Americe vystudoval námořní
      akademii v Annapolisu a na této škole pak několik let působil. V letech 1880-1882 odjel na
      vědeckou stáž do Francie a Německa, kde pracoval u Helmholtze na astronomické observatoři v
      Postupimi\footnote{ S postupimskou hvězdárnou byli spjati takoví astronomové a fyzikové jako
      Leibniz, Alexander von Humboldt. Euler, Bode nebo Galle v roce 1874 u ní Helmholtz zřídil také
      astrofyzikální observatoř, kde později působil známý matematik a astronom Schwarzschild.}
        
      Po návratu do Ameriky učil Michelson na Vysoké škole aplikovaných věd v Clevelandu, na
      Clarkově univerzitě ve Worcesteru a nakonec od r. 1892 až do konce života na Chicagské
      univerzitě. Michelson byl znám jako fanatik přesných měření, konstruktér optických a
      spektrálních přístrojů a tvůrce metrologických metod (porovnával etalon metru s vlnovou délkou
      světla). Za tyto své práce také dostal v roce 1907 Nobelovu cenu.

      \begin{tcnote}
        Michelson si vytkl za cíl změřit co nejpřesněji rychlost světla ve vakuu, resp. ve vzduchu.
        K tomu účelu zkombinoval Fizeauovu metodu ozubeného kola s Foucaultovou metodou rotujícího
        zrcadla a snažil se použit co nejdelší základnu, kterou světelný paprsek po odrazu od
        vzdáleného zrcadla zpět k pozorovateli prochází. První měření provedl v letech 1878-1882 s
        rotujícím osmibokým skleněným hranolem a chyba měřeni činila \SI{60}{\km\per\s}. Po 42
        letech se k měření vrátil a nechal světlo procházet po dráze \SI{35}{\km} mezi vrcholy hor
        Mount Wilson a Mount San Antonio v Kalifornii. Tentokrát činila chyba jen \SI{4}{\km\per\s}.
        Aby vyloučil nepravidelnosti dané atmosférickými podmínkami, zkonstruoval půldruhého
        kilometru dlouhou trubici, z níž vyčerpal vzduch, kde se světlo mnohonásobné odráželo od
        soustavy zrcadel a rotující hranol měl 32 stěn. S touto aparaturou bylo prováděno na 3 000
        měřeni, ale to už byl Michelson nemocen a nových, přesnějších výsledků se nedožil. Dá se
        říct, že ještě na smrtelném lůžku myslel na to, jak zpřesnit měření alespoň o jedno další
        desetinné místo. Byl vlastně dovršitelem klasických mechanických metod měření rychlosti
        světla, než nastoupily metody moderní, které využívali jako přerušovače světelného paprsku
        Kerrův elektroopticky článek nebo jsou založeny na měření rezonance mikrovln v dutině a na
        vlastnostech plynových laserů. Michelsona by jistě potěšilo, že dnes známe rychlost světla
        ve vakuu s přesnosti \SI{1.2}{\ms}, tedy na \num{4} desetimiliontiny procenta. To je větší
        přesnost, než s jakou umíme měřit délky, a proto klademe rychlost světla rovnou přesně \(c =
        \SI{299792458}{\m\per\s}\). Kdyby nějaký novodobý Michelson změřil rychlost světla ještě
        přesněji, raději tuto hodnotu ponecháme a změníme délku metru!
      \end{tcnote}

      Za svého pobytu v Evropě se Michelson rozhodl, že navzdory pochybnostem velkých fyziků
      uskuteční experiment, který umožní pozorovat pohyb Země vzhledem k éteru. V Helmholtzově
      laboratoři v Berlíně k tomu účelu zkonstruoval zrcadlový interferometr a pokusil se zjistit
      posun interferenčních proužků vyvolaný pohybem Země. Předpokládaný jev se ale neprokázal. Po
      návratu do Ameriky se Michelson spojil s profesorem chemie na Clevelandské univerzitě
      \textsc{Edwardem Morleyem} (1838-1923) a v roce 1887 spolu provedli známy
      \textbf{Michelsonův-Morleyův experiment} s největší možnou přesností. Jejich interferometr byl
      umístěn na mramorové desce, která plavala a natáčela se v bazénu se rtutí, světelné paprsky se
      vícenásobně odrážely od zrcadel, aby prošly co nejdelší dráhu ve směru pohybu Země a kolmo k
      němu, a nakonec bylo možno porovnat jejich vzájemné zpoždění. Ani tento experiment, později
      ještě dále zdokonalovaný, nedal žádaný výsledek - interferenční proužky se při pootočení
      přístroje neposunuly. Přesnost měření byla přitom taková, že pohyb Země vzhledem k éteru by se
      projevit musel. Michelsonův experiment byl jedním z mála v historii, který se proslavil,
      přestože se vlastně nepodařil, ukázal, že hledaný jev nenastává. Takový negativní výsledek
      bývá přijímán s rozpaky a zanechává pocit určitého neuspokojení.

      Situace s éterem se dále povážlivé zkomplikovala. Pokus bylo ovšem možno vysvětlit tak, že
      Země na své pouti sluneční soustavou strhává okolní éter a unáší ho s sebou (do jaké asi
      výšky?), takže se vůči němu nepohybuje. Tak nějak si to představoval \textsc{Gabriel Stokes}
      (1819 - 1903), totéž stanovisko zastával i Heinrich Hertz (1857 - 1894). Bradleyův objev
      hvězdné aberace ale svědčil o tom, že éter zůstává v celém prostoru nehybný a Země se v něm
      pohybuje. Fizeauův pokus demonstroval, že pohybující se prostředí strhává s sebou éter jen
      částečně, pomaleji, jak se domníval Fresnel.

      \luagraphic[1]{fyz_fig0947.jpg}{Albert Einstein a Hendrik Antoon Lorentz v Leidenu roku 1921
        Kredit: Wikipedia}{fyz:fig0947}

      V této chvíli vstoupil do diskuze jeden z předních teoretických fyziků na přelomu od klasické
      k moderní fyzice, Holanďan \textsc{Hendrik Antoon Lorentz} (1853-1928). Lorentz se narodil v
      Arnhemu a studoval na starobylé univerzitě v Leidenu, kde se stal v r. 1878 profesorem
      teoretické fyziky. V r. 1913 se tohoto postavení nezištně vzdal, aby uvolnil místo mladšímu
      fyzikovi, a do konce života působil jako ředitel Teylerova fyzikálního kabinetu muzea v
      Haarlemu. Přitom každé pondělí dopoledne konal pravidelné přednášky na leidenské univerzitě
      jako čestný profesor. Lorentz byl znám svou mimořádnou skromností, vyzařoval osobní kouzlo.
      Vládl hlubokou fyzikální intuicí a u mezinárodní fyzikální obce se těšil velké autoritě. Byl
      doživotním předsedou mezinárodních tzv. \textbf{Solvayovských kongresů} a přitom zásadně
      odmítal lukrativní nabídky na profesorská místa v zahraničí. Když Lorentz zemřel, odmlčely se
      v celém Holandsku na několik minut všechny telegrafy a telefony. Nad Lorentzovým hrobem
      promluvil prezident londýnské Královské společnosti \textsc{Ernest Rutherford} (1871 - 1931).

      \begin{tcnote}
        Ve své disertační práci na leidenské univerzitě odvodil Lorentz z Maxwellových rovnic zákony
        odrazu a lomu světla, později objasnil vzájemnou závislost elektrické a tepelné vodivosti
        látek, vypracoval teorii disperze světla. Nezávisle na dánském fyzikovi Lorenzovi odvodil
        závislost permitivity dielektrik a indexu lomu látek na jejich hustotě (známá formule
        Lorentzova-Lorenzova). \textsc{Ludwig Valentin Lorenz} (1829. 1891), který působil na
        univerzitě v Kodani, dospěl také nezávisle na Maxwellovi k poznatku, že světlo je
        elektromagnetické vlnění.
      \end{tcnote}

      \textsc{Lorentz} vycházel z Maxwellovy teorie elektromagnetismu, kterou prozkoumal do hloubky,
      a na jejím základě objevil a předpověděl řadu nových elektromagnetických a optických jevů.
      Našel sílu, kterou elektromagnetické pole působí na pohybující se elektrický náboj. Tato tzv.
      Lorentzova síla je základem činnosti přístrojů moderní elektroniky, elektronové a iontové
      optiky a techniky urychlovačů. Stejně jako \textsc{Henri Poincaré} dospěl \textsc{Lorentz} k
      závěru, že elektromagnetická vlna je nositelem hybnosti.

      \textsc{Lorentz} se nechtěl smířit s Michelsonovým výkladem, jak pohybující se tělesa s se
      sebou unášejí éter, a v r. 1892 navrhl výklad  svůj, i když trochu fantastický. Podle
      \textsc{Lorentze} tělesa, která se pohybují v nehybném éteru rychlostí v blízkou rychlosti
      světla, zkracují svůj podélný rozměr v poměru \((1- \frac{v^2}{c^2})^\frac{1}{2}\).
      Vysvětloval to tak, že síly mezi molekulami či atomy tvořícími těleso jsou zprostředkovány
      nehybným éterem a jak se těleso tímto éterem prodírá, vzdálenosti mezi molekulami se zkracují.
      Je-li tomu tak, zkrátí se i dráha světelného paprsku ve směru pohybu Země v Michelsonově
      experimentu a ve stejném poměru se zkracují ovšem i délky všech pohybujících se měřidel v
      tomto směru. Sama zeměkoule se přitom zploští o neuvěřitelně malou veličinu asi \SI{6}{\cm}.
      Ke stejnému výkladu dospěl i irský fyzik \textsc{George Franancis Fitzgerald} (1851-1901) v
      témž roce jako \textsc{Lorentz}, mluvíme proto o \textbf{Lorentzově-Fitzgeraldově kontrakci
      délek}.

      V letech 1892-95 si \textsc{Lorentz} uvědomil, že Maxwellovy rovnice nevyhovují Galileovým
      transformacím newtonovské fyziky při přechodu od jedné inerciální vztažné soustavy k druhé,
      pohybující se vůči ní konstantní rychlostí \(v\). Rovnoprávnost takových inerciálních soustav
      by bylo možno obnovit, kdybychom připustili, že čas není absolutní a že v každé z těchto
      soustav plyne jiným způsobem. \textsc{Lorentz} tak zavedl jakýsi „místní čas“ v každé
      pohybující se vztažné soustavě. Domníval se ale, že je to vlastně jen vhodný matematický
      „trik" a že nenahrazuje skutečný ,, absolutní“ čas fyzikální. 

      V roce 1895 tak \textsc{Lorentz} dospěl k novým transformacím mezi vztažnými soustavami, které
      představovaly zobecnění Galileovy a Newtonovy mechaniky a vedly k paradoxním závěrům.
      Publikoval je v práci \emph{„Pokus o jednu teorii elektrických a optických jevů v pohybujících
      se tělesech"}\footnote{Versuch einer Theorie der elektrischen und optischen Erscheinungen in
      bewegten Körpern, Leiden 1895}. Teprve v roce 1904 však zapsal tyto vztahy v dnes obvyklé
      podobě a publikoval je ve Zprávách Amsterodamské akademie; \textsc{Poincaré} je pak nazval
      \textbf{Lorentzovy transformace}. \textsc{Lorentz} sám však stále považoval svůj výklad za
      jednu z možných teorií, jeden z možných výkladů. Je třeba uvést, že v téže době se těmito
      otázkami zabýval i anglický fyzik \textsc{Joseph Larmor} (1857-1942), který působil střídavě v
      Belfastu a v Cambridgi, a ve své vynikající práci \emph{„Éter a látka"}\footnote{Aether and
      matter, Cambridge 1900} dospěl k týmž výsledkům jako \textsc{Lorentz}. Došlo k jakémusi takřka
      schizofrennímu rozdvojení klasické fyziky. Mechanické jevy se podřizovaly Galileovým
      transformacím, zatímco elektromagnetické jevy transformacím Lorentzovým. Obě tyto oblasti
      klasické fyziky byly přitom bohatě experimentálně potvrzovány.

      Záhadné rozložení čar ve spektrech atomů, protichůdné vlastnosti předpokládaného světelného
      éteru, nesoulad klasické termodynamiky a statistické fyziky s pozorovaným spektrem záření
      černého tělesa a neujasněnost některých základních pojmů klasické mechaniky budily ke konci
      19. století pochyby a neuspokojenost se stavem dosaženého fyzikálního poznání. K tomu
      přistupovaly překvapivé objevy \textbf{neviditelných paprsků}, katodových, rentgenových a
      radioaktivních, přímo výronů energie ze samé podstaty hmoty, které fyzika také nedovedla
      vysvětlit. Velcí klasičtí fyzikové a chemikové, jako např. \textsc{William Thomson} a
      \textsc{Dmitrij Mendělejev}, nebyli už s to tyto nové výzvy přírody přijímat a o samém objevu
      radioaktivity s možností přeměny jednoho prvku v druhý pochybovali.

      Někdy se mluví o \emph{\uv{krizi fyziky}} nebo \emph{,,krizi myšlení"} na přelomu století.
      Myslím, že to není správné, fyzika neprocházela žádnou krizí. Krize znamená zhroucení, úpadek,
      bezvýchodnou situaci. Místo toho se fyzikální výzkum a jeho aplikace rozvíjely plným tempem a
      v mezinárodní spolupráci, vznikala nová fyzikální pracoviště a ústavy, hromadily se nové
      poznatky. Některé z nich nebylo možné vysvětlit v rámci klasické teorie, podle dosavadních
      představ. Bylo tedy zřejmé, že tento rámec bude třeba rozšířit, zobecnit a prohloubit, najít
      nové matematické přístupy. Zatímco klasická fyzika dobře vysvětlovala jevy a pohyby známé z
      naší denní zkušenosti (ostatně i dnes létáme do vesmíru na základě Newtonových zákonů),
      ukazovalo se, že ve světě atomů a molekul, v mikrosvětě, a také v případě pohybů rychlostmi
      blízkými rychlosti světla, bude třeba najít nové, obecnější fyzikální zákonitosti. Nikdo ovšem
      netušil, jak hluboký a zásadní převrat fyziku čeká, že se schyluje k nové vědecké revoluci a
      že bude trvat několik desetiletí, než se s ní fyzikové vyrovnají.

      Nový pohled na svět vyžadoval mimořádnou fyzikální intuici a duševní odvahu, schopnost
      odpoutat se od zažitých představ tzv. \emph{„zdravého lidského rozumu"} a pracovat s velmi
      abstraktním matematickým aparátem. Jen málo fyziků bylo schopno se v této nové situaci
      orientovat. Přesně to vyjádřil \textsc{Max Planck} slovy: \emph{\uv{Nová vědecká pravda se
      neprosazuje tím způsobem, že se její odpůrci dají přesvědčit, ale spíše tak, že odpůrci pomalu
      vymřou a dorůstající generace se s pravdou seznamuje hned na počátku}.}

      \subsubsection{Mach}\label{fyz:IchapIIsecIVssecIsssecV}
        Maxwellova teorie, která nastolila nový, elektromagnetický obraz světa, skutečně otřásla
        Newtonovou mechanikou. Nezávisle na tom se ve druhé polovině 19. století začaly objevovat
        práce, které kriticky přehodnocovaly základní pojmy Newtonovy mechaniky, jako jsou hmotnost,
        síla, absolutní prostor a absolutní čas. Vedly se diskuse o tom, zda druhý Newtonův pohybový
        zákon není vlastně definici síly, zda síla není příliš antropomorfní pojem, který odkazuje
        na naše subjektivní pocity při svalové námaze. Vždyť konec konců i sám pojem energie se
        nakonec převádí na schopnost konat práci, např. při zvedání břemen v tíhovém poli, a práce
        je typicky lidská činnost s velmi obecným významem. Tak \textsc{Gustav Kirchhoff} se pokusil
        v r. 1876 definovat silu jako čistě matematický výraz bez odkazu na chování fyzikálních
        těles. \textsc{Heinrich Hertz} se ve své Mechanice" pokusil odstranit pojem sily z fyziky
        vůbec a vycházet z principu, v němž vystupují pouze prostorové souřadnice, čas a hmotnosti
        těles

        Tyto diskuse směřovaly k revizi Newtonovy mechaniky, jejich základních pojmů a východisek,
        ale neusilovaly o její nahrazení mechanikou novou. Byly vedeny snahami učinit fyziku
        exaktnější, přiblížit ji struktuře matematiky, která dosáhla v 19. století obrovského
        pokroku. Vedlo to k tomu, že teoretická fyzika byla považována víceméně za odvětví
        matematiky, a tak došlo k určitému odtržení teoretické a experimentální fyziky, narušeni
        komunikace mezi fyziky teoretickými a experimentálními. Mnozí svědomití a pracovití
        experimentální fyzici neměli dost času, erudice ani matematické přípravy, aby pochopili nové
        abstraktní teorie matematické fyziky a jejich důsledky. Teoretičtí fyzikové zase žili v
        iluzi, že fyziku je možno vybudovat jako uzavřený matematický systém bez odkazu na
        experimentální fakta a lidskou zkušenost.

        Jedním z nejvýznamnějších a nejznámějších zástupců klasické fyziky, který podrobil Newtonovu
        mechaniku, při vší úctě k jejímu tvůrci, zevrubné kritice, byl teoretik, experimentátor,
        didaktik, fyziolog a filozof \textsc{Ernst Mach} (1838-1916). Svými myšlenkami, a zejména
        svým stěžejním dílem v oboru fyziky \emph{„Mechanika, jak se vyvíjela. Historicko-kritické
        pojednání.“}\footnote{Die Mechanik in ihrer Entwicklung. Historisch-kritisch dargestellt.} z
        roku 1883, inspiroval celou řadu představitelů nové éry ve fyzice, včetně \textsc{Alberta
        Einsteina}, kteří si ho vážili, považovali ho za svého učitele, a dokonce ho navrhovali na
        udělení Nobelovy ceny. Přitom je v jeho spisech pozoruhodně málo matematiky, v podstatě na
        elementární úrovni. \textsc{Mach} však zřejmě vyslovil některé myšlenky, které druzi zatím
        jen tušili a které s tímto jejich tušením rezonovaly.

        \textsc{Mach} jako experimentátor byl nucen věnovat značnou pozornost fyziologii zraku a
        sluchu, už jenom proto, že tenkrát ještě neexistovaly spolehlivé detektory, čidla a
        registrační zařízení pro měření kmitočtu, intenzity zvuku a světla a dalších veličin. Tak
        např. při Dopplerových pokusech se změnou výšky tónu parních píšťal lokomotiv museli být
        angažováni hudebníci s absolutním hudebním sluchem. Jako staří filozofové si byl Mach vědom
        toho, že svět poznáváme svými smysly, které nás ovšem mohou klamat, a že máme k dispozici
        vlastně jen soubory našich počitků a vjemů.

        V duchu \wikiPozitivismus pak odmítal hledat za těmito vjemy a experimentálními údaji
        nějaké nepozorovatelné jevy a objekty, jako jsou např. atomy. Odmítal proto i Newtonův
        absolutní prostor a absolutní čas jako metafyzické fikce. Prostor je podle Macha dán jen
        rozložením a uspořádáním těles, čas sledem a četností událostí, např. kyvů kyvadla. S
        nepolapitelným éterem, který by měl zaplňovat nekonečný prostor, si ovšem \textsc{Mach}
        neporadil, stejně jako \textsc{Newton}. Také hmotnost chápal \textsc{Mach} jako relaci,
        vztah. Nemá smysl hovořit o hmotnosti jednoho izolovaného tělesa. Teprve začne-li na toto
        těleso působit těleso jiné, vyjeví se v tomto vztahu v souladu se zákonem akce a reakce
        jejich hmotnosti.

        Snad nejdůležitější Machova myšlenka se týkala původu setrvačných sil. které se objevují v
        neinerciálních vztažných soustavách (síla vznikající při zrychlování nebo zpomalování
        vztažné soustavy, sila odstředivá nebo Coriolisova) a které \textsc{Newton} přičítal pohybu
        vzhledem k absolutnímu prostoru. Podle \textsc{Macha} není rozdíl mezi pravými a setrvačnými
        silami, všechny jsou dány vzájemným, relativním působením těles a rozložením hmoty ve
        vesmíru. Odlehčeně řečeno, poraníme-li se při prudkém zabrzdění dopravního prostředku, kdy
        nás setrvačná síla vrhne kupředu, není to vina řidiče, ale působení vzdálených galaxii.
        Relační výklad setrvačných sil naložených celostně z vesmírného hlediska označil
        \textsc{Einstein} v r. 1918 jako \textbf{Machův princip}. I když se tento princip v Machově
        formulaci přímo nestal součásti obecné teorie relativity, vedl Einsteina k zamyšlení, je
        stále diskutován a považován za inspirující.

        Machův důraz na relativitu pohybu, kdy můžeme určit vždy jen vzájemný pohyb vztažných
        soustav, a nikoli jejich pohyb vůči absolutnímu prostoru, Einsteina ovlivnil a přímo nadchl.
        Einstein si Macha vážil, jak vyplývá z jeho korespondence, a domníval se, že Machovy
        myšlenky jsou v plném souladu s jeho teorií relativity. Tím více ho mrzelo a překvapilo, že
        se Mach ostře proti teorii relativity postavil, odmítal být s ní jakkoli spojován, a dokonce
        podporoval úsilí o její experimentální vyvrácení, které přímo vložil do odkazu svému synu
        Ludwigovi. Při vší kritičnosti svého myšlení zůstával Mach hluboce zakořeněn v klasické
        fyzice 19. století a odmítal překročit práh století nového. Odmítl i přesvědčivé důkazy o
        existenci atomů na začátku 20. století.

        Mach se narodil v Chrlicích, moravské obci, která je dnes součásti Brna, a byl pokřtěn v
        nedalekých Tuřanech jako \textsc{Ernst Walfried Joseph Wenzl}. Jeho otec \textsc{Johann
        Nepomuk Mach} měl rozsáhlé vzdělání a byl tak trochu podivín. Svého jediného syna (Ernst měl
        ještě dvě sestry) vychovával podle svých představ, otužoval ho ledovou vodou a sám ho
        vzdělával v klasických i živých jazycích, v matematice a přírodních vědách. Mach se učil na
        arcibiskupském gymnáziu Kroměříži a pak na vídeňské univerzitě, kde v roce 1860 obhájil
        doktorskou práci o elektrickém výboji.

        V téže době vystoupil na obhajobu myšlenek zesnulého \textsc{Christiana Dopplera},
        experimentálně prokázal existenci akustického \textbf{Doppleova jevu} a jako mladý doktorand
        neváhal ostře oponovat autoritativnímu profesoru Petzvalovi. Možná i tím si zkomplikoval své
        postavení na vídeňské univerzitě. Byl sice přijat za asistenta, působil jako (neplacený)
        soukromý docent a byl pověřen přednáškami z fyziky pro mediky, ale musel si přivydělávat
        doučováním a konáním populárních přednášek pro veřejnost. Publikovali řadu článků o
        fyziologických souvislostech fyziky. Po krátkém působení ve Štýrském Hradci (kde se také
        oženil) přijal vr. 1867 nabídku na místo profesora experimentální fyziky na pražské
        univerzitě.

        V Praze strávil Mach 28 let, narodili se mu tam čtyři synové a dcera. Brzy byl zvolen i
        děkanem filozofické fakulty, ve školním roce 1879-80 pak rektorem univerzity. Podruhé byl
        zvolen rektorem již oddělené, německé univerzity roku 1883, ale brzy se funkce vzdal. Mach s
        rozdělením pražské univerzity na českou a německou nesouhlasil, odmítal národnostní spory
        jako nevědecké a nepodstatné a varoval, že rozdělením se materiální podmínky obou univerzit
        zhorši. I když Mach vyrůstal v německém prostředí, pohyboval se v mládí na Moravě v českém
        živlu, obstojně česky hovořil, podporoval české studenty, z nichž vychoval řadu úspěšných
        žáků, a podporoval Jednotu českých matematiků a fyziků. Přesto byl nakonec po rozdělení
        univerzity v době vypjatých národnostních sporů v Praze vnímán jako představitel německé
        strany. 

        Mach byl uznáván jako vynikající pedagog, konstruktér experimentálních zařízení a učebních
        pomůcek, z nichž mnohé slouží na vysokých školách při výuce fyziky dodnes. Jeho odborný
        zájem se soustřeďoval na akustiku a optiku. Dřívější studie o Dopplerově jevu ho přivedly k
        otázkám nadzvukových pohybů střel ve vzduchu a nadzvukového obtékání těles. Použil
        \emph{šlírovou metodu fotografování} prudce letících těles, vynalezenou jak už víme
        \textsc{Toeplerem} ve Štýrském Hradci, vymyslel důvtipný spouštěcí mechanizmus fotografické
        závěrky a jako první pořídil fotografie střel pohybujících se nadzvukovou rychlosti. První
        zprávu o tom podal na zasedání Vídeňské akademie v červnu 1886. Zpočátku měl problém v tom,
        že střely rakouských vojenských pušek rychlosti zvuku nedosahovaly, a tak se musel obrátit
        na německou zbrojařskou firmu Krupp. Pokusy s fotografováním střel prováděl na rakouské
        námořní střelnici v Pule. Mach tak objevil \textbf{rázovou vlnu}. Její charakteristiky,
        \textbf{Machův kužel} a \textbf{Machovo číslo}, jsou pojmy, které se dnes používají v
        nadzvukové letecké a raketové technice.

        Poslední léta Machova pobytu v Praze byla poznamenána osobními konflikty ve Fyzikálním
        ústavu, depresemi a smutnými událostmi. V roce 1894 spáchal sebevraždu jeho dvacetiletý syn
        Heinrich, nadaný chemik. Machem to hluboce otřáslo a uspíšilo to jeho rozhodnuti opustit
        Prahu. Přešel na vídeňskou univerzitu, kde získal profesuru \uv{filozofie, zejména historie
        a teorie induktivní vědy}. V poslední fázi svého života se už Mach fyzikou nezabýval, její
        nová snažení a objevy nepřijal a věnoval se spíše filozofii a teorii poznání. Byl i
        politicky činný - jako člen panské sněmovny podporoval sociální zákonodárství, přičemž
        odmítl nabízené povýšení do šlechtického stavu. V roce 1898 ho při cestě vlakem ranila
        mrtvice a Mach ochrnul na pravou polovinu těla. Jeho zdravotní stav se postupně zhoršoval, a
        nakonec byl Mach nucen se přestěhovat k synovi Ludwigovi do Vaterstettenu u Mnichova, kde
        také zemřel.

        Machův objev rázových vln vznikajících při nadzvukových pohybech těles našel nečekanou
        analogii v elektromagnetismu ve 30. letech 20. století a do konce nesmírně důležité
        uplatnění v jaderné a částicové fyzice. Jde o tzv. \textbf{záření Čerenkovovo}. Už v roce
        1889, kdy ještě nebylo nic známo o elektronu, publikoval předvídavý fyzik \textsc{O.
        Heaviside} ve Filozofickém magazínu článek, v němž uváděl, že při pohybu elektřiny v
        dielektriku rychlostí větší, než je v rychlost světla v tomto prostředí, by měly být
        vyzařovány směrované elektromagnetické vlny na způsob Machova nadzvukového kužele.
        Experimentálně to prokázal ruský fyzik \textsc{Pavel Alexejevič Čerenkov} (1904-1990), který
        pracoval ve Fyzikálním ústavu akademie věd SSSR pod vedením významného odborníka v oblasti
        luminiscence \textsc{Sergěje Ivanoviče Vavilova} (1891-1951), Čerenkovův příspěvek ve
        Zprávách AV SSSR z roku 1934 měl název \emph{\uv{Viditelné světélkování čistých kapalin pod
        vlivem záření gama}}. V témž svazku publikoval Vavilov i první nástin teorie tohoto jevu, v
        němž zdůraznil, že podstata pozorovaného světélkování není luminiscence molekul či atomu
        kapaliny.

        Toto zvláštní namodralé světélkováni pozorovala už paní Curieová, když pracovala s preparáty
        radia, můžeme ho vidět, nahlížíme-li do jaderného reaktoru přes několikametrovou stínicí
        vrstvu vody, a zdá se, že je i podstatou záblesků, které vznikají v oční kapalině kosmonautů
        pod vlivem kosmických paprsků. Ukázalo se, že toto záření vyvolávají rychlé elektrony, které
        vznikají pod vlivem záření gama v důsledku Comptonova jevu a které se pohybuji nadsvětelnými
        rychlostmi vzhledem k rychlosti světla v dané látce. Tato rychlost je ovšem menší než
        rychlost světla ve vakuu. Přitom nejde o tzv. brzdné záření, které známe z rentgenek nebo
        elektronových synchrotronů, elektrony se pohybuji prakticky konstantní rychlosti a na své
        dráze budí vyzařování průhledného optického prostředí. Jde o zcela nový jev, který umožnil
        konstrukci Čerenkovových detektorů částic umožňujících určovat jejich energie, hmotnosti a
        směr pohybu. Poměrně komplikovanou teorii Čerenkovova jevu podali později ruští fyzikové
        \textsc{Ilja Michajlovič Frank} (1908-1990) a \textsc{Igor Jevgenjevič Tamm} (1895-1971). V
        roce 1958 dostali Čerenkov, Tamm a Frank Nobelovu cenu.

      \subsubsection{Planck}\label{fyz:IchapIIsecIVssecIsssecVI}
        Vědecká revoluce, v níž se zrodila moderní fyzika, vypukla nenápadně, aniž si to v
        předvánočním ruchu někdo uvědomil, přesně 14. prosince roku 1900. Toho dne vystoupil na
        zasedání Německé fyzikální společnosti v Berlíně \textsc{Max Planck} a přednesl teoretické
        odvození závislosti hustoty energie záření černého tělesa na frekvenci, která přesně
        odpovídala experimentálně naměřeným hodnotám. Rozehnal tak druhý „obláček" na fyzikálním
        nebi, který \textsc{William Thomson} půl roku předtím označil jako problém
        \emph{„Maxwellova-Boltzmannova rozdělení energie".}

        Z obláčku se ale pořádně zablýsklo. Planck byl totiž nucen, v podstatě proti svému
        přesvědčení, udělat „drobný" matematický předpoklad. A to, že energie se nevyzařuje spojitě,
        ale jen v malých porcích, kvantech \(h\nu\) úměrných frekvenci. Tak se zrodila
        \textbf{kvantová fyzika}, i když zatím jen v zárodečné podobě (hovořilo se o kvantové
        hypotéze"), která otevřela cestu do mikrosvěta a úplně změnila život lidstva v následujících
        staletích. Planck uveřejnil svou práci v Pojednáních Společnosti pod názvem \emph{„K teorii
        zákona rozdělení energie v normálním spektru"}\footnote{\foreignlanguage{ngerman}{Zur
        Theorie des Gesetzes der Energieverteilung im Normalspektrum}} a brzy na to i v Annalen der
        Physik.

        \textsc{Max Karl Ernst Ludwig Planck} (1858-1947) patřil svým životem a dílem rovnou měrou
        devatenáctému i dvacátému století, byl „klasikem", který dokázal překročit práh moderní
        fyziky, a dokonce se stát jejím spolutvůrcem. Byl snad nejuctívanějším německým fyzikem nové
        doby a historické tragédie dvojího válečného běsnění, které z Německa vzešly, se staly i
        jeho osobními tragédiemi. Jako vyznavač ideálů absolutna dokázal dokonce jako jeden z
        prvních pochopit i převratný význam teorie relativity a stal se jejím celoživotním
        zastáncem.

        Planck se narodil v Kielu, severoněmeckém přístavním městě, a po celý život k tomuto svému
        domovskému městu tíhnul. Jeho otec byl právník a univerzitní profesor, který svých sedm dětí
        vedl ke skromnosti a pracovitosti, pěstoval v nich logické myšlení, ale také náboženskou
        víru, patriotismus a oddanost k pruskému státu. Když bylo Planckovi devět let, rodina se
        odstěhovala do Mnichova. Malý Planck tam navštěvoval gymnázium, kde maturoval jako
        šestnáctiletý s výtečným prospěchem. Dostalo se mu dobré humanitní a jazykové výchovy,
        rozvíjel ale také své hudební sklony, zpíval v chlapeckém sboru, hrál na klavír a varhany,
        komponoval, a stál dokonce před rozhodnutím, zda se dát cestou klavírního virtuóza. V každém
        případě mu hudba stejné jako náboženská víra byla v těžkých chvílích života utěšitelkou.

        Na gymnáziu nebyla fyzika předmětem výuky. Přesto ale nadšený učitel matematiky našel
        způsob, jak seznámit studenty se zákonem zachování energie, který byl v té době vnímán jako
        vyvrcholení celé fyziky. Právě obecnost tohoto zákona to, že množství energie nezávisí na
        jejích konkrétních podobách a zákon má univerzální platnost nezávislou na vůli člověka,
        mladého Plancka nadchla. Spatřil v něm projev absolutna, nebo chceme-li Boha, k němuž se
        usiloval přiblížit. Ostatně Planckovi, který v životě publikoval desítky fyzikálních prací,
        obvykle ani nešlo o konkrétní vlastnosti látek a těles, ale spíše o obecné, absolutní
        principy. Když se Planck v roce 1874 rozhodoval mezi drahou klavírního virtuóza, k níž měl
        všechny předpoklady, a studiem fyziky, profesor fyziky na mnichovské univerzitě
        \textsc{Phillip von Jolly} (1809 - 1884) ho od studia fyziky odrazoval. Poukazoval na to, že
        fyzika je už v podstatě ukončenou vědou, v níž se nedá objevit nic nového.

        Planck se ale naštěstí přece rozhodl pro dráhu fyzika a dal se zapsat na mnichovskou
        univerzitu. Přitom se ovšem nevzdával svých dalších aktivit, žil intenzivním hudebním
        životem, dokonce složil operetu, a pěstoval horskou turistiku. Zimní semestr 1877-1878
        strávil studiem na univerzitě v Berlíně, kde poslouchal \textsc{Kirchhoffa} a
        \textsc{Helmholtze}. Jak uvádí ve své autobiografii, přednášky těchto dvou velikánů fyziky
        ho příliš nenadchly - Helmholtz předčítal ze svých chaotických zápisků, Kirchhoff měl
        přednášky formálně vybroušené, ale suché. Ani v Mnichově nenašel Planck učitele, který by ho
        uvedl do fyzikálního výzkumu, a tak si jako téma své doktorské práce sám vybral Clausiův
        druhý zákon termodynamiky o tom, že teplo nemůže samovolně přecházet z chladnějšího tělesa
        na teplejší. Přeformuloval ho v tom smyslu, že směr přenosu tepla se nedá žádným fyzikálním
        procesem úplně obrátit, a pokusil se dát zákonu absolutní platnost. Doktorskou práci sice
        obhájil a závěrečné zkoušky složil, ale ohlasu na svou práci se nedočkal.

        V roce 1880 se Planck habilitoval, ale neměl pracovní uplatnění a k tomu se chtěl ženit se
        sestrou spolužáka z gymnázia \textsc{Marii Merckovou}. Nakonec se mu podařilo získat nově
        zřízenou pozici mimořádného profesora matematické fyziky na univerzitě v Kielu, kam odjel v
        roce 1885. Brzy se oženil a narodil se mu syn Karl. Za působení v Kielu se Planck dále
        zabýval termodynamikou a snažil se podle teploty tuhnutí roztoků dokázat, že molekuly
        rozpuštěných látek jsou v nich disociovány. Narazil však na nesouhlas fyzikálních chemiků.
        Když v Berlíně zemřel \textsc{Gustav Kirchhoff} (1824 - 1887) a hledal se jeho nástupce,
        Planck se ocitl ve výběru uchazečů na třetím místě. Protože \textsc{Heinrich Hertz} dal
        přednost místu učitele Fyzikálního ústavu v Bonnu a \textsc{Ludwig Boltzmann} Helmholtzovu
        nabídku nepřijal, byl vybrán Planck.

        Do Berlína se Planck přestěhoval v roce 1889 a v hlavním městě říše zůstal téměř do konce
        života. Planckovi se usadili ve vilové čtvrti Grunewald, kde žila i řada dalších
        univerzitních profesorů, a časem si zde postavili vlastní dům. V roce 1889 se jim narodila
        dvojčata Emmy a Grete, 1893 druhý syn Erwin. Planck se stal nejprve mimořádným, od r. 1892
        řádným profesorem berlínské univerzity, r. 1894 členem a později předsedou Pruské akademie
        věd. Ihned po příchodu do Berlína byl přijat za člena Berlínské fyzikální společnosti, která
        se v roce 1899 rozšířila na Německou, a Planck ji pak také dlouhá léta předsedal. Byl i
        spolupracovníkem časopisu Annalen der Physik.

        Planckův denní rozvrh se postupně ustaloval, dalo by se říci přímo s německou pedantičností.
        Vedle plnění pracovních, pedagogických a společenských povinnosti zahrnoval čas vyměřený
        odpočinku, odpolední siestě, vyřizování korespondence, klavíru, a ovšem i rodině. Při práci
        Planck zpravidla neseděl u stolu, ale stál u pracovního pultu; lépe se soustředil a práce mu
        šla rychleji od ruky. Známe ostatně i fotografii Einsteina u pracovního pultíku. V domě
        Planckových se pořádaly hudební večery, dětské slavnosti a diskuze s přáteli. První léta měl
        Planck štěstí, že mohl spolupracovat s \textsc{Helmholtzem}, jehož si hluboce vážil. Ten
        však v r. 1894 zemřel a v témž roce ho následovali i Hertz a Kundt. To znamenalo pro Plancka
        další povinnosti při řešení organizačních a personálních záležitostí německé vědy, zejména
        při jednání se státními úřady. V některých případech se musel postavit na obhajobu fyziků,
        kteří byli v nemilosti úřadů pro svůj židovský původ nebo politické přesvědčení.

        Je zřejmé, že Planckovy společenské a organizační aktivity mu zabraly mnoho času, a to na
        úkor vlastní vědecké práce. Byly jistě ku prospěchu německé fyzice a svědčí o důvěře a úctě,
        jíž se Planck těšil, a jeho organizačních schopnostech. Byl uznávaným odborníkem v oboru
        termodynamiky; to také uváděl Helmholtz v návrhu na přijetí Plancka do Akademie věd. Na
        druhou stranu někteří fyzikové Planckovy výsledky nepřijímali nebo poukazovali na to, že
        byly již známy, např. z prací Gibbsových. Také Boltzmann s Planckem ostře polemizoval,
        zejména v otázce statistického výkladu entropie. V každém případě Planck tehdy ještě stále
        nenašel v termodynamice své \uv{velké téma}. Tímto tématem se mu stalo \textbf{záření
        černého tělesa}.

        Těleso nazýváme černým, resp. \emph{\uv{absolutně černým}}, jestliže pohlcuje všechno
        dopadající záření. Zmiňovali jsme se, že už Kirchhoff navrhl realizovat takové těleso v
        podobě dutiny, do níž lze nahlížet malým otvorem a jejíž stěny jsou zahřáty na
        termodynamickou teplotu \(T\). Záření uvnitř dutiny je pak v termodynamické rovnováze s
        látkou a ve stejné míře, jak je stěnami vyzařováno, je jimi také pohlcováno. Kirchhoff také
        dokázal, že spektrum záření černého tělesa závisí pouze na teplotě \(T\) jako parametru a
        nikoli na druhu látky z níž je dutina utvořena. Závislost hustoty záření černého tělesa na
        frekvenci představuje tedy jakousi \emph{\uv{univerzální funkci}}, \emph{\uv{absolutní
        zákon}}, který bylo třeba najít. Není proto divu, že tento problém Plancka mimořádně zaujal.
        Kromě toho, otázka záření černého tělesa nabývala i praktického významu pro tvorbu své
        tepelných normálů a rozvoj osvětlovací techniky.

        \luagraphic[1]{fyz_fig0948.jpg}{Termografickým měřicím systémem lze zobrazit teplotní pole
        měřeného objektu (hlavy), ale pouze na jeho povrchu. Obor termografie se v širším měřítku
        rozvinul společně s rozšířením infračervených kamer, pro které se obecně vžilo slovo
        termovizní kamera, resp. termovize. Kredit: Wikipedia}{fyz:fig0948}

        Dílčí poznatky o spektru záření černého tělesa byly objevovány postupně. Mimořádně
        jednoduchým, elegantním a překvapivým se ukázal \textbf{Stefanův -Boltzmannův zákon}, podle
        něhož je celková energie vyzařovaná černým tělesem přímo úměrná čtvrté mocnině
        termodynamické teploty. V roce 1893 \textsc{Wilhelm Wien} (1864-1928) odvodil tzv.
        \textbf{posunovací zákon}, podle něhož černé těleso vysílá maximum energie na vlnové délce,
        která je nepřímo úměrná teplotě. Tak lidské tělo o teplotě \SI{300}{\K} vyzařuje s maximální
        intenzitou na vlnové délce \SI{10}{\mm} v daleké infračervené oblasti (a může být detekováno
        termovizí). Slunce o povrchové teplotě \SI{6000}{\K} září na vlnové délce kolem
        \SI{500}{\nm}, což přibližně odpovídá jeho žlutému zbarvení. V 90. létech začali Wien,
        \textsc{Otto Lummer} (1860-1925) a další berlínští fyzikové intenzivně experimentovat se
        zářením černého tělesa, aby proměřili vlastnosti jeho spektra a teoreticky je odůvodnili.

        \textsc{Max Wien} (1866 - 1938) vystudoval berlínskou univerzitu, byl asistentem
        Helmholtzovým a kolegou Planckovým. Od roku 1900 působil jako profesor v Würzburgu, po 1920
        v Mnichově. Byl uznávaným experimentátorem, vedle záření černého tělesa zkoumal i
        odchylování katodových, rentgenových a anodových paprsků v elektrických a magnetických
        polích. V roce 1898 při studiu chování proudu ionizovaných plynů identifikoval kladně
        nabitou částici o hmotnosti rovné hmotnosti vodíkového atomu. Tato částice byla později
        rozpoznána jako \emph{proton} a stala se součástí Rutherfordova modelu atomu. Wien svými
        experimenty položil také základy hmotové spektroskopie částic. Podobné jako J. J. Thomson
        měřil poměr náboje a hmotnosti částic tvořících katodové paprsky a byl tak na stopě objevu
        elektronu.

        \textsc{Wien} se usilovně snažil najít závislost energie záření černého tělesa na frekvenci
        na základě zákonů termodynamiky. Úkol byl tím lákavější, protože se očekávalo, že tato
        univerzální přírodní závislost musí mít jednoduchý tvar. Wien a stejně tak Paschen uvažovali
        tuto závislost ve tvaru součinu záporné třetí mocniny a exponenciální funkce frekvence. V
        roce 1896 se Wienovi podařilo najít takovou závislost, která dobře souhlasila s
        experimentálními údaji. Obsahovala dvě neznámé konstanty, které bylo třeba určit měřením.
        Wienův \textbf{rozdělovací zákon} připomínal Maxwellovo-Boltzmannovo rozdělení energie,
        ovšem nebyl dostatečně teoreticky zdůvodněn.

        V té době se o problém záření černého tělesa začal zajímat Planck, ale zdálo se, že přišel
        pozdě a že úloha je už v podstatě vyřešena. Zbývalo jen blíže zdůvodnit Wienův zákon a určit
        neznámé konstanty. To Planck učinil na základě teoretických úvah o vztahu mezi entropii,
        energii a termodynamickou teplotou. 

        Čertovo kopýtko se skrývalo ve dvou Wienových konstantách. V roce 1900 fyzikové v Říšském
        fyzikálně-technickém ústavu v Berlíně-Charlottenburgu, kde probíhal výzkum nových,
        účinnějších světelných zdrojů, rozšířili měření záření černého tělesa do oblasti
        infračervených vlnových délek a vyšších teplot a zjistili, že tyto konstanty přestávají být
        konstantní. S kritikou Wienova zákona vystoupil i anglický fyzik Rayleigh, Maxwellův
        nástupce ve vedení Cavendishovy laboratoře v létech 1879-1884.

        \begin{tcnote}
          Lord Rayleigh, vlastním jménem \textsc{John William Strutt} (1842-1919), se narodil ve
          šlechtické rodině, která získala svůj titul za zásluhy o království v napoleonských
          válkách; Rayleigh zdědil titul po smrti otce v roce 1873. Sňatkem s \textsc{Evelyn
          Balfourovou} se pak spříznil se šlechtickým rodem Salisbury z něhož pocházel i anglický
          ministerský předseda, a tak se Rayleigh stále pohyboval v nejvyšších společenských
          kruzích. Svého postavení však využíval výhradně ve prospěch anglické fyziky a fyziků a své
          zásluhy opírá jen o vlastní vědecké výsledky. Po absolvování Trinity College Cambridgeské
          univerzity byl přijat za jejího člena a později, r. 1908 se stal rektorem univerzity. Byl
          také členem (od r. 1873), sekretářem a později prezidentem londýnské Královské
          společnosti. V roce 1884 se stal předsedou Britské asociace pro pokrok vědy a r. 1887
          vystřídal \textsc{Johna Tyndalla} (1820 - 1893) v postavení profesora Královského
          institutu, takže zastával vlastně všechny nejprestižnější funkce, jichž mohl fyzik v
          Anglii dosáhnout. Udržoval rozsáhlé mezinárodní styky, při několika cestách do Spojených
          států se sešel s Rowlandem, Michelsonem, Edisonem a dalšími americkými fyziky a byl to
          vlastně on, kdo přimel Michelsona k preciznímu opakování jeho slavného pokusu. Ve své
          rezidenci v Terling Place v Essexu přijímal přední evropské fyziky včetně Helmholtze a
          také si tam zřídil soukromou fyzikální laboratoř, kde jednoduchými prostředky dosahoval
          mimořádných vědeckých výsledků.
        \end{tcnote}
        
        Rayleigh byl znám především svými pracemi z teorie vln a kmitů, akustiky a optiky. Přímo
        klasickým se stalo jeho dílo \emph{\uv{Teorie zvuku}}. Rayleigh rozlišil fázovou a grupovou
        rychlost vlny a našel \textbf{Rayleighův zákon rozptylu světla na malých částicích}. Zatímco
        \textsc{Tyndall} experimentoval s rozptylem světla v zakaleném prostředí, uvažoval Rayleigh
        o rozptylu přímo na molekulách čistého vzduchu. Našel zákon, podle něhož je intenzita
        rozptýleného světla nepřímo úměrná čtvrté (!) mocnině vlnové délky dopadajícího světla, a
        tím vysvětlil \emph{modrou barvu oblohy}. (Einstein později ukázal, že jde o rozptyl na
        malých fluktuacích hustoty vzduchu.) Rayleighova práce \emph{„O září oblohy, její polarizaci
        a barvě"} vyšla v r. 1871. Nejvíce se však Rayleigh proslavil svými mířeními hustot
        atmosférických plynů, které vedly k objevu nového prvku, argonu (spolu s chemikem
        \textsc{William Ramsayem} (1852 - 1916)). Rayleigh za něj obdržel Nobelovu cenu za fyziku v
        roce 1904 (Ramsay za chemii) a získanou peněžní částku věnoval Cambridžské univerzitě.

        V červnu 1900 publikoval Rayleigh v časopise Philosophical Magazine jiný zákon záření
        černého tělesa, odlišný od Wienova. Podle něho je hustota energie úměrná druhé mocnině
        frekvence a termodynamické teplotě. Tento zákon vycházel z poznatku klasické statistické
        fyziky, podle něhož střední energie kmitající částice (oscilátoru) je úměrná teplotě a je
        rovnoměrně rozdělena mezi všechny její stupně volnosti. Jeho podrobnější teoretické
        zdůvodnění podal v r. 1905 mladý anglický matematik, fyzik a astrofyzik \textsc{James
        Hopwood Jeans} (1877-1946).

        \begin{tcnote}
          \textsc{Jeans} byl absolventem Cambridžské univerzity a velkou část života působil v USA,
          na Princetonské univerzitě a na astronomické observatoři Mount Wilson v Kalifornii. Byl
          znám jako autor monografie Dynamická teorie plynů a také hypotézy o vzniku sluneční
          soustavy při blízké kolizi Slunce s procházející hvězdou. Napsal řadu svého času velmi
          populárních knih o astronomii; některé z jeho astrofyzikálních představ se ovšem časem
          ukázaly jako poněkud dogmatické a zkreslené.
        \end{tcnote}
      
        \textbf{Rayleighův-Jeansův zákon} záření černého tělesa byl ovšem plně ve shodě s klasickou
        termodynamikou záření a experiment ho potvrzoval v infračervené části světla, kde právě
        selhával zákon Wienův. Naopak v ultrafialové části spektra dával tento zákon nekonečné
        hodnoty energie a hovořilo se o tzv. \emph{ultrafialové katastrofě}, jak tento rozpor
        později v r. 1911 označil \textsc{Paul Ehrenfest} (1880 - 1933). Rayleigh si byl této
        skutečnosti vědom a pokoušel se nějak zkombinovat svůj zákon se zákonem Wienovým, ovšem bez
        úspěchu. 

        Na podzim r. 1900 se situace pro Plancka stala dramatickou. Berlínští experimentátoři
        \textsc{Ferdinand Kurlbaum} (1857-1927) a \textsc{Heinrich Rubens} (1865 až 1922) pomocí
        vysoce citlivého bolometru (odporového teploměru) měřili intenzitu tepelného záření a
        zjistili, že Wienův zákon, který Planck teoreticky odvodil a zdůvodnil, přestává v
        infračervené oblasti spektra platit. Sdělili Planckovi, že se o tom chystají referovat 19.
        října na zasedání Fyzikální společnosti. Byl k dispozici Rayleighův zákon, který byl
        zaručeně správně z klasické fyziky odvozen, ale nesouhlasil s experimentem pro ultrafialovou
        oblast spektra. To bylo vážné - experiment tak vlastně poprvé ukázal, že klasická fyzika
        alespoň v tomto případě selhává.

        Planck měl k dispozici jen několik dní, aby našel chybu ve svých výpočtech. Pokusil se najít
        nový zákon tak, aby platil v celém rozsahu spektra, splňoval zákon Stefanův-Boltzmannův,
        Wienův zákon posunovací a v limitních případech přecházel na rozdělovací zákony Wienův a
        Rayleighův. Měl ještě k dispozici dvě konstanty, které bylo třeba určit experimentálně tak,
        aby nový zákon odpoví dal výsledkům měření. Planck nakonec tento svůj zákon odhadl
        (nechceme-li říci, že uhádl), na zasedání Společnosti ho v diskuzi ke Kurlbaumově přednášce
        předložil a požádal experimentátory, aby ho ověřili. Ukázalo se, že Planck byl šťastnější
        než Rayleigh - jeho zákon přesně souhlasil s experimentem v celém rozsahu spektra.
        \textbf{Rubens} ověřil Planckův vzorec ještě v noci po zasedání, \textsc{Lummer} a
        \textsc{Ernst Pringsheim} (1859-1917) došli po přepočítání svých výsledků během několika dní
        také k souhlasu.

        Experimentátoři mohli být spokojeni, pro teoretika Plancka hledajícího absolutní principy to
        ovšem byla další pohroma. Zákon sice krásně odpovídal přírodě, ale Planck nevěděl proč.
        Nebylo vůbec jasné, odkud se vlastně bere, z jakých obecných fyzikálních principů pramení.
        Následovaly dva měsíce horečného úsilí, v nichž se zkoncentroval vrcholný Planckův tvůrčí
        přínos fyzice.

        Planck opět vyšel z druhého principu termodynamiky, ale tentokrát z jeho statistické
        formulace, jak ji navrhl Boltzmann. Použil Boltzmannův vztah mezi entropií a logaritmem
        pravděpodobnosti odpovídající rovnovážnému stavu, zavedl konstantu \(k\), která v tomto
        vztahu vystupuje, a nazval ji Boltzmannova konstanta. Byl to důležitý krok na cestě k cíli a
        není vyloučeno, že by Boltzmann, kdyby se tímto problémem dále a hlouběji zabýval, dospěl ke
        kvantové hypotéze sám a dříve. Planck musel překonat svůj počáteční odpor k Boltzmannovu
        chápání entropie, ale nakonec se s jeho pojetím ztotožnil. Co mu vlastně zbývalo!

        Planck předpokládal, že černé elektromagnetické záření vydávají malé oscilátory, jakési
        anténky ve stěnách zahřáté dutiny, tedy vlastně atomy, v nichž kmitají elektrické náboje a
        které vyzařují a pohlcují záření všech možných frekvencí. Bylo třeba zjistit, kolik je
        takových oscilátorů (kmitavých modů) v jednotce objemu a jaká je jejich celková střední
        energie. Budeme-li při výpočtu této střední energie vycházet z toho, že oscilátory mohou
        nabývat jakékoli energie, tj. že jejich energie je rozdělena spojitě, dostaneme jako
        výsledek klasický Rayleighův-Jeansův zákon a jsme opět na počátku.

        Planck se proto uchýlil k matematické hypotéze, která se stala esencí jeho postupu, totiž že
        energie oscilátoru se nemůže měnit spojité, ale jen po jakýchsi malých skocích,
        elementárních dávkách \(\varepsilon\). Celkovou střední energii pak nebudeme hledat
        integrováním, ale sčítáním nekonečné řady. Tím dostaneme výsledek, který přesné souhlasí s
        \textbf{Planckovým vyzařovacím zákonem}. V tomto zákoně pak vystupují dvě konstanty,
        \textbf{Boltzmannova konstanta} \( k = \SI{1.380649e-23}{\joule\per\kelvin}\) (přesně) a
        \textbf{Planckova konstanta} \(h = \SI{6.62607015e-34}{\joule\s} \) (přesně). Elementární
        kvantum energie je pak dáno jako \(\varepsilon = h\nu\), součin Planckovy konstanty a
        kmitočtu. Tyto dvě konstanty spolu s konstantou gravitační a konstantou \(c\), vyjadřující
        rychlost světla ve vakuu, tvoři jakýsi opěrný rámec teoretické fyziky a drží náš obraz světa
        pohromadě. Planckova konstanta je klíčem do mikrosvěta a její nepatrná hodnota ukazuje, proč
        kvantování, zrnitost" energie a dalších fyzikálních veličin, v běžném životě nepozorujeme.
        Svět se tak trochu podobá rastru fotografie tvořeném tmavými a světlými body.

        Své odvození přednesl tedy Planck v závěru roku 1900 na zasedání Německé fyzikální
        společnosti. Hypotézu o elementárním kvantu energie prezentoval s určitými rozpaky,
        považoval ji za jakýsi šikovný matematický trik a v duchu doufal, že se toto kvantum podaří
        nějakým limitním přechodem z fyzikální teorie opět vyloučit. Nepodařilo se. Vždy platilo
        dávné pořekadlo přírodních filozofů, že \emph{„Natura non facit salta"}\footnote{Příroda
        nedělá skoky}. A přesto se najednou zdálo, že příroda přece jen skoky dělá!

        V prvních letech 20. století Planck svou kvantovou hypotézu nijak dále nerozvíjel a
        nepublikoval o ni. Mnoho fyziků se k ní stavělo s pochybnostmi. Zatímco např. \textsc{J.
        Jeans} ji stroze odmítal, přijímal ji \textsc{Rayleigh} s větším nadhledem. V r. 1911 píše
        Nernstovi: \emph{„Musím přiznat, že se mi nelíbí takové vyřešeni problému. Samozřejmě nemám
        nic proti závěrům vyplývajícím z kvantové teorie, je to postup, který v rukou schopných lidi
        vede k zajímavým výsledkům. Přesto je pro mne obtížné ji přijmout jako představu odrážející
        realitu."} 

        Někteří fyzikové, a zejména chemici, si význam Planckova objevu uvědomovali, viděli v něm
        potvrzení atomismu a v letech 1907 a 1908 navrhovali Plancka na udělení Nobelovy ceny. Tyto
        návrhy nebyly ale vyslyšeny, také s poukazem na stanovisko Lorentzovo, který ukázal, že
        Planckova hypotéza je v rozporu s klasickou mechanikou. Paradoxně právě za to by si Planck
        býval tehdy Nobelovu cenu zasloužil. Nobelovu cenu za objevy týkající se zákonů tepelného
        záření dostal v r. 1911 \textsc{Wien}.

        Je zajímavé zkoumat, jak se vyvíjel názor samotného Plancka na jeho objev a kdy si uvědomil
        jeho převratný význam. Ve své Nobelovské přednášce z 2. června 1920 uvádí: \emph{„Buď bylo
        kvantum akce jen fiktivní veličinou a přitom celé odvození vyzařovacího zákona bylo v
        podstatě jen iluzí představující prázdnou hru se vzorci bez jakékoli významu, anebo se
        odvození vyzařovacího zákona opíralo o zdravou fyzikální koncepci. V tom případě muselo
        kvantum akce hrát fundamentální úlohu ve fyzice, a to bylo něco úplně nového, nikdy předtím
        neslýchaného, co se zdálo vyžadovat od nás základní revizi celého našeho fyzikálního
        myšlení, které je od doby, kdy Leibniz a Newton zavedli infinitezimální počet, založeno na
        přijetí spojitosti všech kauzálních vztahů. Experiment rozhodl, že správná byla druhá
        možnost."}

        Planckovi trvalo zřejmě velmi dlouho, možná dvacet let, než si to plně uvědomil, než se
        vzdal pokusů smířit svá kvanta s klasickou fyzikou. Ve své „Vědecké autobiografii“, kterou
        napsal ke konci života, říká, že někteří jeho kolegové v tom spatřovali určitou osobní
        tragiku. Planck to ovšem popírá a říká, že si při tomto úsilí věci důkladně ujasnil. Přesto
        však, jaké asi musel prožívat pocity, když jako stárnoucí profesor sledoval bouřlivý rozvoj
        kvantové fyziky, objevy nových zákonů, rovnic a vztahů, kde se to Planckovou konstantou jen
        hemžilo, rozvoj, do něhož už nemohl zasahovat ani se s ním plně vyrovnávat!

        V roce 1905 vstoupil do Planckova života \textsc{Albert Einstein}, a ovšem i naopak. Vztah
        těchto dvou velkých vědců tak rozdílného původu, povah, založení i věku, kteří si v mnoha
        otázkách rozuměli i nerozuměli, vzájemně se doplňovali a přitom si po celý život zachovali
        vzájemnou úctu, by zasloužil zvláštní psychologickou studii. Ukázalo by se, že to, co je
        spojovalo, byla skromnost, smysl pro spravedlnost a zaujetí pro vědeckou pravdu. Oba
        zůstávali často dlouho se svými názory a objevy mezi fyziky osamoceni, oba museli dlouho
        čekat na svou Nobelovu cenu, Planck 18 a Einstein 16 let. S jejich jmény jsou spjaty dva
        hlavní myšlenkové proudy moderní fyziky, kvantová teorie a teorie relativity. Přitom se
        ukazuje, že se Einstein svou intuicí, myšlenkovou odvahou a schopností nazvat věci pravými
        jmény zasloužil o uznání kvantové fyziky vlastně ještě více než Planck.

        Jako redaktor časopisu Annalen der Physik byl Planck jeden z prvních, komu se dostaly do
        rukou Einsteinovy práce o teorii relativity. Zatímco většina soudobých fyziků Einsteinovy
        závěry nepřijala a ti nejčestnější z nich se alespoň přiznali, že jim nerozumí, byl Planck
        jeden z mála fyziků, který si okamžitě uvědomil hloubku a převratný význam této teorie. Hned
        v roce 1906 poslal do Švýcarska za neznámým mladým fyzikem Einsteinem svého oddaného žáka
        \textsc{Maxe von Laueho}. Ve své \uv{Vědecké autobiografii} považuje za potřebné vysvětlit,
        proč právě on, hledač absolutních ideálů, se stal přesvědčeným \uv{relativistou}. Vycházel z
        toho, že absolutní můžeme nacházet jen prostřednictvím relativního, a navíc právě v teorii
        relativity našel absolutní smysl v principu stálosti rychlosti světla ve vakuu \(c\). Stejné
        jako Planck i Einstein měl svou konstantu. Planck Einsteina podpořil, zprostředkoval mu
        později pozvání na Berlínskou univerzitu a stal se vlastně jeho celoživotním ochráncem.

        Na druhou stranu Einstein jako snad jediný dovedl ocenit Planckovu kvantovou hypotézu a dát
        ji úplně nový, hlubší smysl, jehož se Planck neodvážil. Už ve své práci z roku 1904
        \emph{\uv{K obecné molekulární teorii tepla}}\footnote{Zur allgemeinen molekularen Theorie
        der Warme} se Einstein zabýval podmínkami rovnováhy záření černého tělesa a vycházel přitom
        ze Stefanova-Boltzmannova zákona a Wienova zákona posunovacího. Jako přesvědčený atomista se
        řídil Boltzmannovým statistickým pojetím entropie a nesnažil se ho křížit s klasickou
        Maxwellovou elektrodynamikou, která popisuje chování spojitých elektromagnetických vln.
        Zamýšlel se především nad malými fluktuacemi (náhodnými změnami) energie záření a nad tím,
        že při vzájemném působení s jednotlivými atomy by i záření mělo mít diskrétní, částicovou
        povahu. Dospěl tedy k Planckovým kvantům úplně jinými cestami, neopírajícími se o
        experiment, a měl jasnou představu záření v dutině v podobě jakéhosi ideálního plynu. Stal
        se tak vlastně teoretickým objevitelem nové částice, která později dostala název
        \textbf{foton}. Nazval ji tak v roce 1926 americký fyzikální chemik \textsc{Gilbert Newton
        Lewis} (1875-1946), objevitel těžké vody. 

        Představa světla jako souboru částic byla pro tehdejší fyziky nepřijatelná: Einstein se tím
        jakoby vracel k Newtonově částicové teorii. Einsteinova světelná kvanta by musela navíc
        vyhovovat i všem známým a ověřeným optickým jevům a chovat se i jako elektromagnetická vlna.
        Spojení částicových a vlnových vlastnosti v jednom objektu nazýváme někdy
        \textbf{korpuskulárně vlnový dualismus} a je to jeden z těch takřka schizofrenních pojmů,
        jimiž moderní fyzika oplývá. Einstein byl zřejmé první, kdo to pochopil.

        Svou kvantovou teorii světla formuloval Einstein ve slavném „březnovém článku" z roku 1905,
        \emph{\uv{O jednom heuristickém přístupu ke vzniku a přeměnám světla}}\footnote{Über einen
        die Erzeugung und Verwandlung des Lichtes be treffenden heuristischen Gesichtspunkt}.
        Zatímco Planck a další byli ještě ochotni připustit, že světelná energie je vyzařována a
        pohlcována v malých porcích, kvantech, Einstein směle dokazoval, že ani energie šířících se
        světelných paprsků není rozdělena spojité, ale že tyto paprsky jsou tvořeny konečným počtem
        částic lokalizovaných v prostoru, dále nedělitelných, které mohou být vyzařovány a
        pohlcovány jen jako celek. Teprve vystředováním, které je jakýmsi opakem kvantování,
        dospějeme k představě elektromagnetických vln, jak je krásně popisuje klasická Maxwellova
        elektrodynamika. Bylo to Kolumbovo vejce postavené na špičku, a přitom tak těžko stravitelné
        pro většinu tehdejších fyziků. Einstein nazval své stanovisko „heuristickým", tedy jakýmsi
        intuitivním výsledkem teoretických úvah, a nikoli účelovým vysvětlením určitého experimentu.

        \begin{tcnote}
          V posledních paragrafech svého článku podává Einstein vysvětlení tří důležitých
          fyzikálních jevů pomocí své teorie světelných kvant - ionizace plynů ultrafialovým
          zářením, fotoluminiscence a vnějšího fotoelektrického jevu, který byl objeven Hertzem,
          částečně prostudován \textsc{Lenardem}, ale stále ještě nedostatečně experimentálně
          prozkoumán. Jak zjistil \textsc{George Gabriel Stokes}, při fotoluminiscenci je látka
          ozařována zářením vyšší frekvence a pak vydává studené jednobarevné světlo o nižší
          frekvenci, tedy vyzařuje fotony o menší energii, jak se to dnes děje v zářivkách. Pokud
          jde o fotoelektrický jev, při dopadu světla na fotokatodu jsou z látky uvolňovány
          elektrony. Aby k tomu však mohlo dojít, nestačí, aby dopadající světlo mělo dostatečnou
          intenzitu. Frekvence záření musí převyšovat určitou mezní hodnotu, neboli každý jednotlivý
          foton musí mít energii dostatečnou k tomu, aby uvolnil z látky „svůj" elektron a dodal mu
          energii k překonání jeho \emph{výstupní práce}.
        \end{tcnote}

        V následujících letech objasnil Einstein pomoci teorie kvant ještě další jevy. V r. 1907
        upozornil na to, že nejen světelná, ale tepelná energie je pohlcována látkou, tedy
        oscilujícími atomy po kvantech, a vysvětil tím teplotní závislost molární tepelné kapacity
        nekovových krystalů při nízkých teplotách, s níž si klasická fyzika také nevěděla rady. Jak
        známo, při vyšších teplotách přestává tato kapacita na teplotě záviset (\textbf{zákon
        Dulongův-Petitův} z roku 1819). Na tyto Einsteinovy práce navázal všestranný holandsky fyzik
        \textsc{Peter Joseph William Debye} (1884-1966), jeden ze zakladatelů kvantové fyziky
        pevných látek. \textsc{Debye} působil na mnoha německých a dalších evropských univerzitách.
        Ve. 1940 emigroval do Spojených států a stal se profesorem chemie na Cornellově univerzitě v
        Ithace.

        Roku 1909 se v Salzburku konal 81. sjezd Společnosti německých přírodovědců a lékařů a při
        této příležitosti se Planck s Einsteinem poprvé sešli. Einstein tu přednesl hlubokou analýzu
        fluktuací světelného tlaku působícího na odrazovou destičku. Ukázal, že světelná kvanta mají
        nejen energii, ale i hmotnost a hybnost, a jsou to tedy plnoprávné částice. Plancka, který
        zasedání předsedal, to však stále nepřesvědčilo, ve svém diskusním příspěvku Einsteinovu
        koncepci zpochybňoval a vyslovil názor, že světelná kvanta není třeba zavádět. Když byl v r.
        1913 na Planckův návrh Einstein přijímán za člena Pruské akademie věd v Berlíně, omlouval ho
        Planck tím, že Einstein občas „přestřelí". O existenci fotonů, které se dnes uplatňují v
        mnoha oblastech tzv. fotoniky, řada fyziků pochybovala ještě ve 20. letech 20. století. 

        Ve světle těchto skutečností zastávají někteří historikové fyziky názor, že Planck vlastně
        nebyl objevitelem kvantové fyziky, že tato čest náleží Einsteinovi. Ze moderní vědeckou
        revoluci ve fyzice bychom neměli datovat rokem 1900, ale rokem 1905. Domnívá se to např. i
        známý odborník na „vědecké revoluce" \textsc{T. S. Kuhn}, ale i řada dalších. S tímto
        názorem je jistě možno souhlasit nebo nesouhlasit, ale Planckovu zásluhu na „kvantové
        hypotéze" a jeho přínos pominout nelze. Fyzikální objevy, nebo dokonce revoluce nejsou dílem
        jednoho člověka.

        Uveďme ještě krátce, jak se odvíjel Planckův život v jeho druhé fázi, v neklidných vodách
        historie první poloviny 20. století. Jako renomovaný a oblíbený učitel věnoval Planck mnoho
        času právě pedagogické činnosti. Jeho šestisemestrální kurz zahrnoval v podstatě celou
        fyziku včetně speciálních otázek teoretické fyziky a v knižní podobě byl dlouho klasickou
        vysokoškolskou učebnicí. Planck byl sice ochoten diskutovat se svými studenty na jakékoli
        téma, zachovává si však určitou uzavřenost, až plachost a požadoval, aby si studenti našli
        téma své vědecké práce a způsob zpracováni sami. Jemu kdysi s vědeckými počátky také nikdo
        nepomáhal. Planck vlastně ani nezaložil vědeckou školu s mnoha následovníky. Mezi jeho
        nemnohé žáky patřili např. \textsc{Max von Laue} nebo budoucí jaderný fyzik a nositel
        Nobelovy ceny \textsc{Walther Bothe} či \textsc{Lise Meitnerová}.

        První těžká rána v osobním životě postihla Plancka, když mu v r. 1909 zemřela manželka Marie
        a zanechala ho se čtyřmi dětmi. Planck se brzy oženil podruhé s neteři své první ženy Margou
        a měl s ní třetího syna Hermanna. V období před první světovou válkou fyzikální výzkum v
        Německu vzkvétal, byl podporován státem i průmyslem. Planck zastával významné funkce v
        Akademii i na univerzitě, v roce 1914 získal pro práci v berlínské Akademii i Einsteina. Na
        domácích hudebních večírcích pak hrával Einstein na housle a Planck na klavír. Idylu však
        brzy vystřídala tvrdá realita. Rozpoutání války bylo v Německu provázeno vlnou
        nacionalistické euforie. Planck, i když se snažil držet akademickou půdu stranou politických
        vášní, byl natolik loajální k pruskému císařství, že podobně jako \textsc{Röntgen},
        \textsc{Nernst} a další podepsal Provoláni 93 německých akademiků ke kulturnímu světu, které
        obhajovalo počínání německé armády vůči civilistům při dobývání Belgie. Ještě v průběhu
        války vysvětloval Planck v dopise Lorentzovi do Holandska, že inteligence a vědci musí
        podporovat armádu svého státu.

        Dva Planckovi synové narukovali do armády a odešli na frontu. Nejstarší Karl se z ní už
        nevrátil, padl v bitvě u Verdunu. Planck to nesl velmi těžce a vyčítal si, že si se svým
        synem příliš nerozuměl. V r. 1917 zemřela při porodu Planckova dcera Grete. Krátce po válce
        její sestra, dvojče Emma, která se vdala za vdovce po sestře, zemřela za přesně stejných
        okolností. Tíhu těchto osobních tragédií mohla stěží vyvážit radost z udělení Nobelovy ceny
        za rok 1918 \emph{„za objev kvant energie"}. Planck převzal cenu ve Stockholmu v létě 1920.
        Spolu s ním ji převzali další dva němečtí fyzikové, jimž byla cena udělena ve válečných
        letech - Planckův žák \textsc{Max von Laue} (za rok 1914) a \textsc{Johannes Stark} (za rok
        1919).

        V období mezi dvěma válkami se Planckovi blížila sedmdesátka, a i když stále publikoval,
        nestačil už prudkému rozvoji teoretické fyziky a často se obracel k otázkám filozofickým a
        náboženským. Přesto usiloval o rozvoj fyziky Německu a obnovu její mezinárodní prestiže a
        zůstával jejím symbolem v zahraničí. V roce 1930 se stal prezidentem Společnosti císaře
        Viléma, která zastřešovala německý vědecký výzkum a jeho napojení na sponzory z průmyslových
        kruhů. Přitom se znepokojením a nepochopením sledoval nástup antisemitských nálad v německé
        společnosti a útoky na práci svých židovských kolegů, především Einsteina. 

        Po nástupu Hitlera k moci byli židovští vědci postupné vylučováni z Akademie, ústavů a škol,
        donuceni emigrovat a Planck, stále loajální státní moci, zůstával v politické naivitě. Jejím
        výrazem je historka publikovaná po válce časopise Physikalische Blätter (3. 1947, s. 143) o
        tom, v jak se Planck při oficiální audienci u Hitlera snažil hájit alespoň některé
        „kvalitní" a zasloužilé židovské vědce. Audience skončila Hitlerovým hysterickým záchvatem a
        otřesený Planck se odpotácel. Také vztah mezi Einsteinem, který emigroval z Německa jako
        jeden z prvních a na Německo vlastně zanevřel, se narušil. Einstein zejména vytýkal
        Planckovi, že i vzhledem ke svému věku nerezignoval na oficiální funkce.

        Planck byl vystaven útokům, zejména ze strany stoupenců tzv. „německé fyziky" Lenarda a
        Starka, jako ochránce Židů. Zatímco teorie relativity a kvantová fyzika byly ze škol
        vymycovány a zejména Einsteinovo jméno ze všech učebnic vymazáno, měl Planck odvahu i v době
        války v souvislosti se vzpomínkami na počátky kvantové fyziky Einsteinovo jméno uvádět. Míra
        jeho osobní tragédie však ještě nebyla dovršena. V průběhu války sílilo i bombardování
        německých měst, včetně Berlína, a Planck se musel uchýlit na venkov k Labi. Jeho dům byl pak
        skutečně vybombardován, knihovna, korespondence a vědecké deníky zničeny. V roce 1944 došlo
        k pokusu o atentát na Hitlera a Planckův syn Erwin byl obviněn z účasti na spiknutí. Přes
        všechnu otcovu zoufalou snahu vyvrátit toto obviněni byl Erwin popraven. Na samém konci
        války se Planck trpící zdravotními potížemi ocitl přímo na frontové linii, musel se skrývat
        v lese, a nakonec se dostal do Göttingenu, kde žila jeho neteř. 

        V roce 1946 byl Planck pozván do Anglie na oslavy výročí Newtonova narození jako jediný
        Němec a nejstarší člen londýnské Královské společnosti. Při oficiálním uvádění byl označen
        jako host bez státní příslušnosti. Muselo to být pro něho kruté. Také název Společnost
        císaře Viléma byl pro Spojence nepřijatelný a hrozilo, že Společnost bude rozpuštěna.
        Nakonec ji zachránil návrh, aby byla přejmenována na Společnost Maxe Plancka pro podporu
        vědy. Tak Planck poskytl ještě poslední službu německé fyzice. V říjnu 1947, ve věku
        nedožitých 90 let Planck umírá. V poselství Americké akademie věd se s ním na dálku
        rozloučil i Albert Einstein.

      \subsubsection{Einstein}\label{fyz:IchapIIsecIVssecIsssecVII}
        \epigraph{\emph{\uv{Jednu věc jsem během svého dlouhého života pochopil: že ve srovnání s
        realitou je veškerá naše věda primitivní a dětinská - a přesto je nejcennější věcí, kterou
        máme".}}}{Albert Einstein}

        Jméno \textsc{Alberta Einsteina} (1879-1955) zaujímá v historii fyziky a celé naší moderní
        kultury výjimečné postavení. Stalo se symbolem geniality, ať už to znamená cokoliv, a z
        vlastního jména dokonce vstoupilo do obecného jazyka. V civilizovaném světě není snad nikdo,
        kdo by toto jméno neznal, a každý si s ním spojuje své vlastní, převážnou většinou zkreslené
        představy. Tyto představy se obyčejně soustřeďují do sousloví \emph{„teorie relativity"}
        označujícího něco sice matematicky složitého, ale zároveň jednoduše názorného v tom smyslu,
        že všechno je relativní.

        Filozofický a vulgární relativismus je starý jako myšlení samo; někomu může být zima a
        jinému zároveň horko, věci se jistě jeví jinak člověku mladému a starci nad hrobem. Dvacáté
        století přivedlo k dalšímu rozrůznění názorů, ideových uměleckých přístupů a v dnešním
        multikulturním světě může relativita názorů dokonce vést k vyhrocení neřešitelných
        konfliktů. Často se setkáváme s úvahami o tom, jak moderní umění ve svém chápání prostoru a
        času bylo ovlivněno, nebo dokonce předcházelo Einsteina s jeho teorii relativity. S tím vším
        nemá Einstein nic společného. Jeho princip relativity říká jen, že v různých vztažných
        soustavách mají fyzikální zákony stejný matematický tvar, a že proto dvě takové různé
        soustavy nelze experimentálně od sebe odlišit. Výraz \uv{teorie relativity} dokonce ani
        nepochází od Einsteina, setkáváme se s ním u \textsc{Hamiltona}, \textsc{Maxwella},
        \textsc{Poincarého} a Einstein ho prostě převzal, neboť se vžil.

        Einsteinovo postavení dobře charakterizuje výměna bonmotů, která proběhla mezi ním a jeho
        dobrým přítelem Charliem Chaplinem. \emph{\uv{Jste velký člověk}}, napsal jednou Einstein
        Chaplinovi, \emph{\uv{protože vašemu uměni porozumí každý}}. \emph{\uv{Jste také velký
        člověk}}, reagoval Chaplin, \emph{\uv{protože vaši teorii relativity nerozumí nikdo}}. V
        podobném duchu vyznívá i známé dvojí čtyřverší na oslavu Newtona a úlohu Einsteina. Anglický
        básník \textsc{Alexander Pope} svého času oslavil výkon zbožňovaného Newtona verši:

        \begin{verse}
          Řád přírody byl dlouho  \\*
          tmou noci obestřen.     \\*
          Bůh řekl: Budiž Newton! \\*
          A jasný září den.
        \end{verse}

        Ve 20. století doplnil britský novinář a vtipálek \textsc{John Collins Squire} toto
        čtyřverší dalším

        \begin{verse}
          Však ďábel brzy dočkal se.    \\*
          ten pomstychtivý brach.       \\*
          Hle - Einstein světlo zakrývá \\*
          a vše je opět v tmach
        \end{verse}

        Je to samozřejmé humorná nadsázka, ale na mnohé současníky zapůsobila situace skutečně tak,
        že Newton odhalil lidstvu božský záměr přírody a moderní fyzika ho opět ďábelsky
        zkomplikovala a učinila nesrozumitelným. Ve skutečnosti ovšem moderní fyzika naše poznání
        řádu přírody prohloubila, ukázala jeho úchvatnou vnitřní krásu a umožnila nám ji pochopit a
        uchopit. Teorii relativity, ale i moderní kvantovou fyziku nelze ovládnout bez matematiky,
        stejně tak jako nemůžeme ocenit partituru symfonie bez znalosti notového záznamu. To ovšem
        nebrání tomu, abychom mohli symfonii s potěšením naslouchat nebo abychom se těšili z plodů
        moderní fyziky a jejich technických vymoženosti, s nimiž se denně setkáváme a používáme je.

        Teorie relativity je souhrnný název pro dvě různé oblasti fyzikálního bádání, které mají
        ovšem společné východisko. To, co nazýváme speciální teorii relativity je vlastně zobecněním
        klasické fyziky na pohyby těles a částic rychlostmi blízkými rychlosti světla.
        \textbf{Speciální teorie relativity} překlenula rozpor mezi Newtonovou a Maxwellovou
        fyzikou. Stalo se to ovšem za cenu nových paradoxů z hlediska tzv. \emph{„zdravého lidského
        rozumu"}. Teorie relativity přinese nový pohled na čas, který může plynout různě v různých
        vztažných soustavách. Ovšem vždy tak, že události, které jsou spojeny příčinným vztahem
        zachovávají zásadu, že příčina musí předcházet následek. Speciální teorie relativity se dnes
        uplatňuje při konstrukci fyzikálních přístrojů a zařízeni a je experimentálně nade vší
        pochybnost ověřena. \textbf{Obecná teorie relativity} je vlastně moderní teorií gravitace a
        uplatňuje se zejména u fyzikálních jevů v kosmu; musíme ji brát v úvahu při velmi jemných
        měřeních, kdy i gravitace může ovlivňovat chod času.

        Při hodnoceni Alberta Einsteina musíme rozlišovat dvě stránky. Jednou je jeho vědecký přínos
        a podstata jeho geniality, druhou jsou zdroje jeho všeobecné popularity, kterou on sám dobře
        nechápal a přispěl k ní jen částečně svým chováním. V dopise \textsc{Maxu Bornovi} z 12. 4.
        1949 napsal: \emph{„Nemohu pochopit, proč ze mne udělali jakýsi idol. Je to stejně obtížné
        pochopit jako to, proč lavina začíná z jednoho zrnka a řítí se po určené dráze"}. Desítky
        životopisů a tisíce statí a článků o Einsteinovi odrážejí značnou rozporuplnost jeho povahy
        a chování. Některé z nich ho nekriticky adoruji, jiné se až s bulvární nechutnosti
        rozebírají v jeho nectnostech a soukromém živote. Einstein tomu nahrával tím, že choval
        často nevyzpytatelně a provokativně, popíral veškerou etiketu a společenské zvyklosti,
        odmítal jakékoli autority a svým soukromím se netajil. Další sousto pro Einsteinovy
        životopisce přinesl rok 2006, kdy Hebrejská univerzita v Jeruzalémě po uplynuti 50 let od
        Einsteinovy smrti dala k dispozici více než 3 500 stránek jeho korespondence, včetně
        intimní. 

        Na rozdíl od některých jiných velikánů fyziky nebyl Einstein v dětství \emph{„zázračným
        dítětem"} (ostatně ani Newton). Dokonce začal mluvit opožděně. Sám odmítal názor, že zdědil
        nějaké mimořádné nadání, a pro svůj myšlenkový vývoj měl dokonce zajímavé vysvětlení tím, že
        začínal chápat později než ostatní děti, měl více času se podivovat a zamýšlet se nad věcmi,
        které jeho vrstevnici považovali dávno za samozřejmé. Není pravda, jak se někdy uvádí, že se
        Einstein ve škole špatně učil. Prospíval ale především v těch předmětech, které ho zajímaly,
        a nesnášel jakékoliv donucování. Jeho slabou stránkou byly cizí jazyky, nejhorší známky měl
        z francouzštiny a jeho angličtina byla dosti otřesná i po mnoha letech strávených ve
        Spojených státech. Bytostný odpor měl také k tělocviku a utužování tělesného zdraví. Nebyl
        ani geniálním matematikem, ani experimentátorem, ani vynálezcem. Sám často zdůrazňoval, že
        měl s matematikou velké potíže, i na vysoké škole ji ostatně zanedbával. Hudební nadání a
        zájmy získal po matce Pauline, rozené Kochové, ale jeho pověstná hra na housle měla daleko
        do virtuozity a preciznosti Planckova klavíru. Byla ale zase procítěnější.

        Zamýšlíme-li se nad smyslem našeho života na Zemi, citujeme často krásná a hluboká
        Einsteinova slova z jeho osobního vyznání \emph{\uv{Jak vidim svět}}\footnote{Mein Weltbild}
        z roku 1930: \emph{\uv{Jak pozoruhodná je situace nás, dětí země! Každý sem přichází jen na
        krátkou návštěvu. Neví proč, ale občas si myslí, že to tuší. Z hlediska každodenního života,
        bez hlubšího uvažování však víme: jsme tu pro druhé lidi. Především pro ty, na jejichž
        úsměvu a blahu zcela závisí naše vlastní štěstí, ale i pro ty mnohé nepoznané, s jejichž
        osudy jsme spjati pouty soucítění. Denně nesčetněkrát myslívám na to, že můj vnější i
        vnitřní život spočívá v tom, co vykonali a co konají lidé dnešní, i ti, co už zemřeli, že se
        musím přičiňovat, abych stejnou měrou dával, jako jsem přijímal a ještě přijímám. Toužím
        stačit sám sobě, ale často mívám tísnivé vědomí, že žádám z práce svých bližních víc, než je
        třeba.}}

        Einstein se skutečné celý život snažil pomáhat svým bližním, kteří se ocitli tísni nebo
        ohrožení a obraceli se na něho ve stovkách dopisů. Statečně vystupoval na obranu demokracie,
        proti fašismu, za uchování míru, byl znepokojen možným nebezpečím, které by pro lidstvo
        mohlo znamenat zneužití výsledků vědy. Většinu života byl přesvědčeným pacifistou a teprve
        agresivní činy hitlerovského Německa ho přiměly tento svůj postoj korigovat. V politických
        otázkách projevoval někdy neuvěřitelnou naivitu; války chtěl odstranit tím, že všichni lidé
        odmítnou nastoupit vojenskou službu, a konflikty mezi státy vytvořením jediné světové vlády,
        která už nebude mít proti komu útočit.

        Na druhé straně, pokud jde o Einsteinovy nejbližší, nepodařilo se mu své krásné krédo
        naplnit. Svou první, nemanželskou dcerku vůbec ani nespatřil, s první manželkou Milevou se
        rozvedl za dosti pro ni ponižujících okolností, druhou ženu, svou sestřenici Elsu, v
        podstatě jen využíval jako udržovatelku domácnosti, staršímu synu Hansi Albertovi až
        hystericky bránil v uzavření sňatku, druhého syna Eduarda, citlivého a nadaného chlapce,
        který byl postižen schizofrenií, nechal dožívat ve švýcarském sanatoriu, aniž se o jeho osud
        příliš zajímal. Jeho vztah k ženám byl velmi spontánní a v životě vystřídal několik milenek,
        tyto své avantýry před manželkami neskrýval. Spíše naopak. Takto zhuštěně řečeno může
        Einstein působit dojmem cynika. Do jisté míry jim byl, i když jeho citový život, mezilidské
        vztahy a hnutí svědomí byly jistě mnohotvárnější a nikoli černobílé. Sám si byl vědom tohoto
        rozporu ve svých ideálech a chování. Na jednu stranu uvádí svou vášnivou touhu po sociální
        spravedlnosti a na druhé straně chybějící touhu družit se s lidmi a lidským kolektivem.
        Uznává, že jako manžel a otec zklamal. Jeho ideálem bylo povolání strážce majáku, kde by byl
        sám a nezávislý a mohl přemýšlet o teoretické fyzice.

        A přece přes všechno negativní a problematické, co jsme o Einsteinovi uvedli, zůstává vedle
        Newtona zřejmě největším fyzikem nové doby a my se oprávněně skláníme před jeho velikostí,
        hloubkou jeho myšlenek a chápání přírody. Byl spolutvůrcem nového fyzikálního myšlení, měl
        odvahu domyslet důsledky nových poznatků a teorii, pojmenovat pravým jménem věci, které
        ostatní fyzikové jen opatrně obcházeli.

        Einsteinovi rodiče pocházeli z židovských rodin usedlých ve Švábsku. Jeho otec Hermann se
        narodil v Buchau, kde náhrobky na židovském hřbitově dosud připomínají desítky nositelů
        příjmení Einstein (původně Ainstein). Uvádí se, že Alberta Einsteina po celý život
        prozrazoval měkký švábský dialekt, jimž se v rodině mluvilo. I když rodina přísluší k
        židovské obci, liberální otec nikdy na dodržování židovských obyčejů nelpěl, cítil
        přináležitost ke klasické německé kultuře, miloval zejména \textsc{Schillera} a
        \textsc{Heina}. Hermann byl spolumajitelem malého elektrotechnického závodu, který střídavě
        prosperoval a bankrotoval a rodina se proto ta také často stěhovala.

        Albert se narodil v Ulmu v třípatrovém rohovém domě nedaleko nádraží. Za 2. světové války
        byl dům zcela zničen bombardováním a na jeho místě dnes stojí modernistická struktura s
        nápisem: \emph{\uv{Zde stál dům, v němž 14. března 1879 přišel no svět Albert Einstein}}.
        Brzy nato se rodina odstěhovala do Mnichova, kde se Einsteinovým v r. 1881 narodila dcera
        Maja (Marie), s niž si Albert po celý život snad nejvíce rozuměl. V Mnichově navštěvoval
        Albert základní školu, pak známé Luitpoldovo gymnázium, kde mu nejvíce vadil prosazovaný
        školní prušácký dril. Setkal se také s prvními projevy antisemitismu.

        Z dětství se mu vybavily dvě charakteristické vzpomínky, které jako by předznamenaly jeho
        vědeckou dráhu. V pěti letech se mu dostal do ruky kompas a fascinovalo ho jeho stálé
        směřování k severu. Zamýšlel se také nad otázkou, zda by bylo možno cestovat rychlostí
        světla, osedlat si světelný paprsek. V jedenácti letech dostal do ruky Eukleidovy základy,
        které se mu staly knihou knih Fascinoval ho zejména důkaz, že se výšky v trojúhelníku
        protínají v jednom bode (proč by měly?). I když to lze chápat jen symbolicky, právě
        magnetizmus je důsledkem speciální teorie relativity a přechod od eukleidovské geometrie ke
        geometrii zakřivených prostorů je základem obecné teorie relativity. V roce 1894 nebo 1895
        (Einstein sám si už rok nepamatoval) sepsal dokonce svou první chlapeckou vědeckou práci s
        příznačným názvem \emph{\uv{O výzkumu stavu éteru v magnetickém poli}} a poslal ji na
        posouzení svému strýci Casaru Kochovi.

        Když mu bylo 15 let, odjeli rodiče za podnikáním do Itálie a mladého Alberta nechali v
        Mnichové samotného ve školním internátu. Chlapec prožíval období zmatků spojených s
        dospíváním, pod vlivem četby hledal svůj světový názor a nakonec se rozhodl učinit tři
        závažné kroky: zanechal studia na gymnáziu ještě před maturitou, požádal o vyvázání z
        německého občanství (svou úlohu zřejmě sehrál i odpor k hrozící vojenské službě) a vystoupil
        z židovské církve. Považoval se pak za občana bez vyznání, i když svou přináležitost k
        židovství pociťoval po celý další život více či méně intenzivně. Takto nakonec stanul v
        Itálii před ohromenými rodiči.

        Bylo rozhodnuto, aby se 16letý Albert přihlásil na prestižní curyšskou
        Polytechniku\footnote{ETH - Eidgenössische Technische Hochschule}, kde u nadaných uchazečů
        nebyla vyžadována maturita. Einstein zároveň požádal o udělení švýcarské státní
        příslušnosti. U přijímací zkoušky na Polytechniku tenkrát neprošel, byl shledán ještě
        nedostatečně zralým a bylo mu doporučeno doplnit si středoškolské vzdělání na kantonální
        průmyslové škole v Aarau. Tam Einstein v následujícím roce maturoval a byl přijat na
        Polytechniku, obor odborného učitelství. Na Polytechnice strávil 5 let a v roce 1900 složil
        závěrečné zkoušky s průměrným prospěchem. Diplom obhájil na základě předložené práce o
        vedení tepla. Věřil, že bude moci pokračovat na Polytechnice jako učitel, podobně jako
        někteří jeho spolužáci, ale ke svému zklamání vybrán nebyl a stál před úkolem najít si
        zaměstnání. Jako švýcarský občan se musel dokonce hlásit k nástupu vojenské služby, ale byl
        od ní ze zdravotních důvodů osvobozen (ploché nohy, křečové žily).

        Léta strávená na Polytechnice měla pro jeho další život hluboký význam. Nebyl příliš
        svědomitým posluchačem a k nelibosti svých profesorů se na některých přednáškách objevoval
        zřídka. Je paradoxní, že nenavštěvoval ani přednášky vynikajícího matematika
        \textsc{Hermanna Minkowskiho}, který později rozpracoval matematickou teorii
        relativistického čtyřprostoru. Minkowski později prohlásil, že takový objev, jakým je teorie
        relativity, od svého záškoláka rozhodně nečekal. Einstein byl ovšem s obsahem jeho přednášek
        dobře obeznámen. Jeho oddaný spolužák \textsc{Marcel Grossmann} je všechny pilně navštěvoval
        a Einstein z jeho zápisků čerpal. Poznamenejme pro zajímavost, že v době Einsteinových
        studii působil na polytechnice i věhlasný teoretik a konstruktér parnich strojů a turbín
        slovenského původu \textsc{Aurel Stodola}, kterého si Einstein velmi vážil. Koncem 90. let
        tam přijel studovat budoucí významný český fyzik a inspirátor polarografické metody
        \textsc{Bohumil Kučera} a je pravděpodobné, že mohl s Einsteinem přijít do kontaktu.

        Einstein navázal za studií četná přátelství, některá z nich na celý život. Vedle zmíněného
        Grossmanna byl mu nejblíže italský technik \textsc{Michele Besso}, jeho pozdější kolega z
        patentového úřadu v Bernu, s nimž si celý život dopisoval a svěřoval se mu, k dalším patřil
        i student matematiky \textsc{Konrad Habicht}, židovský student filozofie z Rumunska
        \textsc{Maurice Solovine} a další. Přátelé diskutovali o fyzice a filozofii a Einstein si v
        diskusích tříbil a upevňoval své myšlenky. Za studií se sblížil i se spolužačkou
        \textsc{Milevou Marićovou}, Srbkou z Vojvodiny v tehdejším Uhersku. Ženy studující na
        technice byly v té době výjimkou.

        Vztah Einsteina a Milevy bývá často předmětem zájmu životopisců a novinářů. Oběma partnerům
        přinesl v životě mnoho hořkosti a utrpení. Einsteinovi rodiče jejich svazku nepřáli a
        odmítali dát jim souhlas k sňatku. Krátce po ukončení studii, v době finanční nejistoty a
        nejasné perspektivy zaměstnání, se Einstein dozvěděl, že je otcem nemanželské dcery Lieserl.
        Mileva ji porodila doma u rodičů a o osudu děvčátka není nic známo, neexistují ani úřední
        záznamy. Uvažuje se o tom, že zemřela v útlém věku, možná za epidemie záškrtu, nebo že byla
        dána k adopci cizím lidem.

        Einsteinův otec se na smrtelném loži dal konečně obměkčit a svolil ke sňatku mladých lidí.
        Ten se konal v lednu 1903 a v květnu 1904 se Einsteinovým narodil syn Hans Albert. Druhy
        Einsteinův syn Eduard se narodil v r. 1910. Manželství, mateřství a starost o domácnost
        uzavřely Milevě případnou odbornou kariéru. Občas se spekuluje o tom, jaký podíl měla Mileva
        na vytvoření speciální teorie relativity a proč ji Einstein neuvedl alespoň jako spoluautora
        svých prací. Tyto úvahy jsou liché. Einstein se ženou jistě o svých vědeckých pracích a
        plánech diskutoval, Mileva mohla jinak dost nepořádnému Einsteinovi pomáhat s některými
        výpočty a opravovat případná matematická opomenutí, ale rozhodně nemohla proniknout do
        hloubky a novátorství jeho fyzikálního myšlení.

        Einstein byl samotář a promýšlel věci svým vlastním způsobem. Byl ovšem přesvědčeným
        atomistou v Boltzmannově duchu, zajímal se o otázky elektromagnetického záření, éteru a
        magnetismu. První práce, kterou mu uveřejnil časopis Annalen der Physik v roce 1901, se
        týkala kapilárních jevů. Nevzbudila žádný zvláštní rozruch, ostatně o mezimolekulárních
        silách v kapalinách nebylo ještě skoro nic známo. Bylo to ale v době, kdy Einstein usilovně
        sháněl místo na některé univerzitě ve Švýcarsku, Německu, Holandsku nebo Itálii, a tuto
        práci přikládal ke svým žádostem jako doklad o své kvalifikaci.
        
        Einstein vzal zavděk i krátkodobým působením výpomocného učitele na technice ve Winterthuru
        a soukromé škole v Shaffhausenu, ale situace zůstávala frustrující. Nakonec zasáhl jako deus
        ex machina jeho bývalý spolužák \textsc{Grossmann}. Pomocí přímluvy u svého otce vymohl pro
        Einsteina místo experta III. třídy na patentovém úřadě v Bernu. Einstein tam nastoupil r.
        1902, v r. 1906 povýšil na expetra II. třidy a působil na tomto místě do r. 1909. To bylo
        jeho povolání strážce majáku, kde mohl v klidu posuzovat různé, třeba i fantastické
        přihlášky patentů a přitom promýšlet své neméně fantastické ideje a koncepce. V létech 1902
        až 1904 publikoval Einstein čtyři menši práce tykající se termodynamiky zářeni a kinetické
        teorie. Tyto práce nevzbudily větší pozornost a nic nenasvědčovalo tomu, že v Einsteinovi
        vyrůstá myslitel, který brzy převrátí celou fyziku naruby.

        Rok 1905 je nazýván Einsteinův zázračný rok, \emph{\uv{annus mirabilis}}. Mladý, téměř
        neznámý referent patentového úřadu v Bernu, bez spojení s některou ze slavných univerzit
        nebo vědeckých laboratoří, bez provádění dlouhodobých pracných experimentů, vlastně jen
        silou svého ducha a soustředění v úplné izolaci položil základ třem hlavním proudům moderní
        fyziky - atomové fyzice, kvantové fyzice a teorii relativity. 

        Tento zázrak se udál v podstatě v rozmezí pouhých několika měsíců a historicky ho lze
        srovnat snad jen s výkony stejně mladého Newtona. Historikové si lámou hlavu tím, jak je to
        možné. Tento einsteinovský převrat samozřejmě vyplynul z předchozího vývoje vědy, ale cesty
        tohoto vývoje se náhle proťaly a zkoncentrovaly v jednom mozku obdařeném velkou fantazii,
        intuicí, nezávislosti a odvahou myšlení. V dalším Einsteinově vědeckém životě se často
        stávalo, že zůstával sám nepochopen většinou ostatních fyziků, a nakonec se ukázalo, že
        pravdu měl on.

        Obvykle se uvádí, že již v roce 1905 publikoval Einstein své tři slavné práce. Jindy se
        dočteme, že tyto práce byly vlastně čtyři nebo že jich bylo pět. Stejně tak bychom mohli
        říci, že Einstein publikoval toho roku 26 prací, neboť pracoval i jako recenzent pro časopis
        Annalen a otiskl v něm též 21 recenzí prací jiných autorů. Jako charakteristiku
        Einsteinových slavných prací bude nejvhodnější uvést přímo jeho vlastní slova z dopisu
        příteli Habichtovi z roku 1905:

        \emph{\uv{Já vám za to slibuji čtyři práce, z nichž první bych mohl už brzy poslat, protože
        jeden exemplář budu mít k dispozici co nevidět. Pojednává o záření a o energetických
        vlastnostech světla a je velmi převratná… Druhá práce je určení velikosti atomu z difuze a
        vnitřního tření zředěného roztoku neutrální látky. Třetí dokazuje, že za předpokladu
        molekulární teorie tepla musí nepatrné částice o velikosti 1/1000 milimetru, které jsou
        rozpuštěny v roztoku provádět zjistitelný, neuspořádaný pohyb, který je vyvolán tepelným
        pohybem molekul. Jde o pohyby malých částic, které byly pozorovány fyziology a nazvaný
        \uv{Brownův molekulární pohyb}. Čtvrtá práce je zatím v konceptu a týká se elektrodynamiky
        pohybujících se těles za použití modifikované nauky o prostoru a času …}} 
        
        O první z těchto prací jsme se už zmiňovali: podávala vysvětlení fotoelektrického jevu a
        dávala nový, reálný obsah Planckové hypotéze světelných kvant. Einsteinova rovnice
        popisující tento jev, jednoduchá a elegantní, nebyla však ještě přesnými experimenty
        ověřena, a představa o světelných kvantech se zdála mnoha fyzikům příliš fantastickou. O
        ověřeni Einsteinova vztahu se pokusilo později několik anglických a amerických fyziku, mezi
        nimi \textsc{O. W. Richardson}, \textsc{A H. Compton} a \textsc{R. A. Millikan}.

        \begin{tcnote}
          \textsc{Owen Williams Richardson} (1879-1959), který působil v Cambridgi, Princetonu a
          Londýně, je znám svými pracemi o tepelné emisi elektronů a iontů z kovů, kterou popsal
          matematickým zákonem po něm pojmenovaným. Tyto práce mu získaly Nobelovu cenu r. 1928.
          Podotkněme, že tepelnou emisi elektronů objevil známý americký vynálezce a fyzik
          \textsc{Thomas Alva Edison} (1847-1931). \textsc{Richardson} potvrdil Einsteinův vztah r.
          1912. Ještě přesnější měření provedl americky fyzik \textsc{Robert Millikan}. Je zajímavé,
          že Millikan s Einsteinovou teorii nesouhlasil a měření prováděl proto, aby ji vyvrátil.
          Nakonec v r. 1915 musel však čestně uznat, že experiment tuto teorii perfektně potvrzuje,
          a navíc umožňuje dobře určit Planckovu konstantu. Chování kvant rentgenového záření jako
          částic při srážkách s volnými elektrony pak prokázaly experimenty, které prováděl
          \textsc{Arthur Holly Comptnon} (1892-1962) v 1923. Compton byl absolventem Princetonské
          univerzity učil na univerzitách v St. Louis a v Chicagu a za války stál v čele známé
          „Metalurgické laboratoře", kde se rozvíjel americký uranový program. Nobelovu cenu dostal
          Compton v roce 1927.
        \end{tcnote}

        Ke kvantové teorii záření přispěl Einstein zásadním způsobem i v několika pozdějších
        pracích. Tak v roce 1912 objevil základní zákon fotochemie, podle něhož každý pohlcený foton
        vyvolává jednu elementární fotochemickou reakci. V roce 1916 našel souvislost mezi
        vyzařováním a pohlcováním fotonů při přechodech atomů mezi dvěma energetickými stavy
        (hladinami), novým způsobem odvodil Planckův vyzařovací zákon a zároveň předpověděl jev
        \textbf{indukované emise záření}\footnote{ \uv{Vyzařování a pohlcování záření podle kvantové
        teorie}, \foreignlanguage{ngerman}{\uv{Strahlungs-Emission und Absorption nach der
        Quantentheorie}} v Pojednáních Německé fyzikální společnosti 2, 18 (1916) 318 a \uv{Ke
        kvantové teorii záření}, \foreignlanguage{ngerman}{\uv{Zur Quantentheorie der Strahlung},
        Physikalische Zeitschrift 18 (1917) 121}}, který se později stal základem laserové techniky,
        Einstein je tak ne-li duchovním otcem, tedy alespoň dědečkem laserů.

        Druhá Einsteinova práce „zázračného roku" \uv{O novém měření rozměrů molekul} („Eine neue
        Bestimmung der Moleküldimensionen") byla míněna jako doktorská dizertační práce, ale nebyla
        curyšskou Polytechnikou přijata. V časopise Annalen vyšla sice až r. 1906, ale jako
        samostatná práce byla prezentována v červenci 1905.

        Třetí práce \emph{\uv{O pohybu částic suspendovaných v nehybných kapalinách, který vyplývá z
        molekulární kinetické teorie tepla}}\footnote{\foreignlanguage{ngerman}{Über die von der
        molekularkinetischen Theorie der Warme geforderte Bewegung von in ruhenden Flüssigkeiten
        suspendierten Teilchen}} \cite{Einstein1905} má v historii fyziky převratný význam. Podává
        matematickou teorii chaotických trhavých pohybů malých částeček vznášejících se v kapalině
        pod nárazy neuspořádaně se pohybujících molekul. Pohyb molekul atomů se tak vlastně
        projevuje svými účinky navenek a zviditelňuje se.

        Tento pohyb pozoroval poprvé skotský botanik \textsc{Robert Brown} v r. 1827 pylových zrnek
        rozptýlených ve vodě. Brown se domníval, že jde o hemžení živých částeček. Dnes je Brownův
        pohyb vděčným tématem názorných fyzikálních experimentů. Nezávisle na Einsteinovi pracoval
        na tomto problému i polský fyzik \textsc{Marian Smoluchowski} (1872-1917), později profesor
        univerzity v Krakově. Své výsledky uveřejnil sice později než Einstein, ale oba fyzikové si
        pak korespondovali a vzájemně spolupracovali. Teoretická analýza Brownova pohybu provedená
        Einsteinem a Smoluchowskim byla brzy na to (1908) přesvědčivě potvrzena experimenty, které
        provedl francouzský fyzik \textsc{Jean Baptiste Perrin} (1870-1942), profesor pařížské
        univerzity. V roce 1926 dostal Nobelovu cenu za práce „O spojité struktuře látky a zvláště
        za objev sedimentační rovnováhy". Perrin mikroskopicky proměřoval, jaké je výškové rozložení
        částeček rozptýlených v kapalině v tíhovém poli, tedy studoval jakýsi miniaturní model
        atmosféry. Potvrdil tím molekulárně statistickou teorii látky Einsteina a Smoluchowského a
        zároveň našel nový způsob stanovení Avogadrovy konstanty. 
        
        Dá se říci, že po dvou a půl tisíciletích, která uplynula od doby řeckého filozofa
        \textsc{Demokrita}, podala fyzika v podobě Einsteinovy-Smoluchowského teorie a Perrinových
        experimentů konečně nezvratný důkaz o existenci atomů. Poslední pochybovači museli uznat,
        svou porážku. Škoda, že se toho nedožil Boltzmann, který po celý život existenci atomu
        prosazoval a který ukončil svůj život v depresích jen dva roky před konečným vítězstvím
        atomismu. Dnes můžeme fyzikálními metodami zaznamenat přítomnost jednotlivých atomů,
        manipulovat s nimi, sestavovat z nich reklamní loga a pozorovat je řádkovacími mikroskopy.
        
        Kdyby Einstein kromě svého příspěvku ke kvantové teorii a atomové fyzice neudělal nic víc,
        měl by zajištěno své nesmrtelné místo v dějinách fyziky. Rok 1905 je ale také rokem zrodu
        teorie relativity. Čtvrtá Einsteinova práce, snad nejslavnější ze všech, nesla název
        \emph{\uv{K elektrodynamice pohybujících se těles}}\footnote{Zur Elektrodynamik bewegten
        Körper} a obsahovala základní postuláty speciální teorie relativity. Víme už, jak se fyzika
        marně snažila vyrovnat s otázkou éteru, vysvětlit, proč se světlo šiří éterem stejnou
        rychlosti bez ohledu na rychlost zdroje. Rada fyziků, jako \textsc{Lorentz},
        \textsc{Poincaré}, \textsc{Fitzgerald} a další, také dospěla k poznatkům, které se pak staly
        součástí teorie relativity - připomeňme zkracovaní délek pohybujících se těles, Lorentzův
        „místní čas", Lorentzovy transformace a další.
        
        Einstein vyšel ze dvou postulátů - \textbf{principu relativity}, podle něhož všechny
        fyzikální zákony (nejen mechanické) mají stejný tvar ve všech inerciálních vztažných
        soustavách, a \textbf{principu stálosti rychlosti světla}, které se šíří ve všech takových
        soustavách stejnou rychlosti. Z těchto postulátů vyplynuly všechny předchozí dílčí výsledky,
        vysvětlení Dopplerova jevu, Fizeauova měření rychlosti světla v proudící vodě a řada
        dalších, zdánlivě paradoxních předpovědi. Narozdíl od Lorentze měl Einstein odvahu
        připustit, že čas skutečně plyne jinak různých vztažných soustavách, že je vlastně čtvrtou,
        dokonce imaginární souřadnicí. \textsc{Hermann Minkowski} pak vytvořil matematickou teorii
        takového čtyřrozměrného pseudoeukleidovského prostoru. Způsob, jakým se Einstein zbavil
        éteru, je skutečně geniální. Místo aby se snažil vysvětlit, proč pohyb v éteru neovlivňuje
        rychlost světla, stálost rychlosti světla prostě postuloval. Nad éterem mohl jen pokrčit
        rameny - fyzika nepociťuje vůči éteru nějakou nevraživost, ale prostě ho nepotřebuje.

        Občas uvádí, že Einstein chtěl svou teorií vysvětlit negativní výsledek Michelsonova
        experimentu. Není tomu tak. Především to nebyl Einsteinův styl. Einstein nevytvářel teorie k
        vysvětlení nějakého nejasného experimentálního faktu, ale jako výsledek vlastních
        intelektuálních úvah. Svými závěry si byl tak jist, že jejich pozdější experimentální
        potvrzení přijímal jako samozřejmost. Někdy se ovšem i zmýlil, ale génius se pozná i v
        omylech. Jak udává, v době, kdy pracoval na teorii relativity, si otázku Michelsonova
        experimentu neuvědomoval. Když se po mnoha letech při své návštěvě Pasadeny v Kalifornii
        začátkem roku 1931 s Michelsonem sešel, zbývalo Michelsonovi už jen několik měsíců života.
        Oba Albertové, velikán teoretik a velikán experimentátor si vyznali vzájemnou úctu, ale
        Michelson nikdy teorii relativity nepřijal a Einstein se o pokus vlastně příliš nezajímal.

        Z Einsteinovy speciální teorie relativity vyplynul i vztah mezi hmotností a energií, snad
        nejznámější rovnice z celé fyziky. Einstein formuloval tento vztah v páté práci zázračného
        roku 1905, která vyšla v září (\emph{\uv{Závisí setrvačnost tělesa na jeho energetickém
        obsahu?}}\footnote{Ist die Tragheit eines Korpers von seinem Energieinhalt abhängig?}. V
        tomto třístránkovém článku hovoří Einstein ale pouze o změně energie tělesa - vyzáří-li
        těleso energii, jeho klidová hmotnost poklesne. Slavný vzorec \(E = mc^2\), podle něhož
        klidová hmotnost tělesa odpovídá obrovské energii, se objevil až o dva roky později v článku
        „O setrvačnosti energie vyplývající z principu relativity"\footnote{Uber die vom
        Relativitätsprinzip geforderte Tragheit der Energie}, a to v podobě \(\mu = E_0/V^2\)
        (Einstein označoval rychlost světla velkým \(V\)).

        \begin{tcnote}
          Také v tomto případě byli na stopě slavnému vzorci někteří další - \textsc{Poincaré}, málo
          známý ital \textsc{Olinto de Pretto} nebo vídeňský fyzik \textsc{Friedrich Hasenhörl}
          (1874-1915), jemuž ve vzorci vyšel ještě koeficient \(\sfrac{4}{3}\). Einsteinovi bývá
          někdy předhazováno, že ve svých článcích necitoval práce svých předchůdců (v článku
          \emph{\uv{K elektrodynamice …}} děkuje pouze příteli \textsc{Bessovi} za to, že stal věrně
          při něm a povzbuzoval ho). Ve skutečnosti ale Einstein na tyto a další práce nenavazoval a
          nikdo z ostatních fyziků teorii relativity nedomyslel a neformuloval. Možná že byli na
          stopě, ale jiná věc je ulovit kořist.
        \end{tcnote}
        
        Nebylo mnoho těch, kdo teorii relativity hned od počátku přijali a ocenili. Patřil k nim
        např. \textsc{Planck}, \textsc{Nernst}, \textsc{Laue}, \textsc{Minkowski}, v Paříži
        \textsc{Langevin}, v Moskvě \textsc{Umov} a několik dalších. Většina fyziků byla na
        rozpacích. Zejména experimentátoři Einsteinovi oponovali vzhledem k tomu, že závěry teorie
        nebyly dostatečné experimentálně ověřeny. Např. \textsc{Walther Kaufmann} (1871 až 1947) z
        Göttingenu, který už v r. 1902 změřil, jak závisí hmotnost elektronu na jeho rychlosti,
        docházel k jiným závěrům než Einstein. O vyřešení těchto rozporů se zasloužil především
        Planck. Einstein doufal, že by jeho vztah mezi hmotnosti a energii bylo možno ověřit při
        radioaktivních přeměnách atomu, ale tato měření nebyla dostatečně přesná.

        V letech, která následovala po \emph{\uv{zázračnému roku}}, byl Einstein vědecky velmi
        plodný, publikoval řadu prací o kvantové teorii záření, ale především v něm postupně
        uzrávala myšlenka \textbf{obecné teorie relativity}, která vycházela z tzv. \textbf{principu
        ekvivalence}. Ten je vlastně založen už v dávném sporu s aristoteliky, zda všechna tělesa
        padají v gravitačním poli se stejným zrychlením, jinak řečeno, zda platí úměrnost mezi
        setrvačnou a gravitační hmotností těles.

        Tuto úměrnost ověřoval v letech 1889 až 1908 přesnými experimenty pomoci torzních vah
        nejznámější maďarský fyzik \textsc{Loránd Eötvös} (1848-1919). Platí-li tato úměrnost bez
        ohledu na chemické složení tělesa, bylo by možno převést gravitaci na setrvačné síly a
        zrovnoprávnit všechny, nejen inerciální vztažné soustavy. Neinerciální soustavy se pohybují
        vzhledem k inerciálním se zrychlením a po zakřivených trajektoriích, a to vede ke vzniku
        setrvačných sil. Pokud tyto síly ztotožníme, alespoň lokálně, s gravitačními, budeme moci
        gravitaci tak vlastně vůbec zrušit! Můžeme prohlásit za důsledek zakřivení prostoru, resp.
        časoprostoru v blízkosti velkých hmot. Je totiž jedno, prohlásíme-li, že se např. světelný
        paprsek (tvořený fotony) při průchodu v blízkosti Slunce odchyluje jeho gravitaci nebo že
        Slunce ve svém okolí zakřivuje časoprostor a paprsek se pohybuje volně, setrvačností v tomto
        zakřiveném časoprostoru po nejkratší možné trajektorii (geodetické čáře). Takové řešení
        problému gravitace by Newtona určitě překvapilo a Einstein se Newtonovi také výslovně
        omluvil za to, že se opovážil opravovat jeho gravitační zákon. Obecná teorie relativity byla
        výlučně Einsteinovým objevem a patří největším intelektuálním výkonům vědy.

        V roce 1908 se Einstein stal soukromým (neplaceným) docentem na univerzitě v Bernu a v
        následujícím roce mimořádným profesorem univerzity v Curychu. Tim také skončila éra jeho
        poklidného bádáni v patentovém úřadě. I když život Einsteina a jeho rodiny v Curychu byl
        celkem příjemný a pedagogické působení bylo pro něho novou zkušeností, přijal Einstein
        nakonec výhodnější nabídku místa řádného profesora a vedoucího ústavu teoretické fyziky na
        německé univerzitě v Praze.

        Do Prahy přijel Einstein s Milevou a oběma syny na jaře 1911. Ubytoval se na Smíchově v
        Lesnické, tehdy Třebízského ulici. Dům byl nový, měl elektřinu, ale hygienické podmínky byly
        tehdy v Praze nevalné. Vltavská voda se musela převařovat a čistotou nemohla soupeřit s
        vodou švýcarských jezer. Einstein se hned ujal výuky, 5 hodin týdně přednášel mechaniku a
        termodynamiku a vedl dvouhodinové seminární cvičení ve svém ústavu ve Viničné ulici. V
        podstatě se však ani on, a zejména ani jeho žena Mileva, ačkoli byla slovanského původu, s
        pražským prostředím dost dobře nesžili.

        Praha byla tehdy světem tří kultur české, německé a židovské, které spolu příliš
        nekomunikovaly. Einstein navštěvoval společnost židovských intelektuálů s nimiž měl společné
        kulturní a hudební zájmy. Tak poznal vynikajícího matematika \textsc{Georga Picka}, který
        později zahynul v koncentračním táboře, spisovatele \textsc{Maxe Broda} a \textsc{Franze
        Kafku}, význačného indologa \textsc{Moritze Winternitze} a další. Einstein si v Praze
        oblíbil její tichá zákoutí, kochal se pohledem na město z letenských strání a rád chodil po
        keplerovských místech. S Keplerem se cítil spřízněn, byl mu mnohem bližší než Galileo
        Galilei.

        Za svého pobytu v Praze publikoval Einstein 11 prací a dále promýšlel svou teorii gravitace.
        Obvykle se citují jeho slova z předmluvy k českému vydání knihy o teorii relativity z r.
        1923: \emph{\uv{Těší mě, že tato malá knížka… vychází nyní v národní řeči oné země, v níž
        nalezl jsem soustředění nutné k tomu, abych základní myšlenku obecné teorie relativity,
        kterou jsem pojal již r. 1908, ponenáhlu přeodíval určitější formou.}}

        V době Einsteinova pražského pobytu se v Bruselu sešel první Solvayův kongres, na nějž byl
        Einstein rovněž pozván. Chemik \textsc{Ernest Solvay}, bohatý mecenáš a průmyslník, vynalezl
        nový, levnější způsob výroby sody (uhličitanu sodného) a ovládl světový trh s touto
        důležitou surovinou chemického průmyslu. 

        Po dohodě s \textsc{Nernstem} se rozhodl přispět k financování rozvoje moderní fyziky, s niž
        amatérsky koketoval, svoláváním pravidelných kongresů, na které by bylo pozváno vždy 25
        nejpřednějších fyziků své doby, aby spolu diskutovali o nejžhavějších otázkách fyziky.

        První Solvayův kongres se konal od 29. řijna 1911 v Grandhotelu Metropole Bruselu (obr.
        \ref{fyz:fig0949}). \textsc{Lindemann} britský občan, který vyrůstal v Německu, byl
        všestranný fyzik, který se stal blízkým Einsteinovým přítelem a po návratu do Anglie působil
        jako Churchillův poradce. Kongresu předsedal uznávaný nestor teoretických fyziků Lorentz,
        sekretářem konference byl belgický průkopník bezdrátové telegrafie a vzduchoplavby
        \textsc{Goldschmidt}

        Tématem prvního kongresu byla, jak jinak, \emph{\uv{teorie záření a kvanta}}. Solvayův počin
        byl velmi záslužný, umožnil, aby se vždy každé tři roky výkvět teoretických fyziků při
        osobním setkání mohl zabývat nejaktuálnějšími problémy a v diskusích, často ostrých, si
        vyjasňovat své názory. Do roku 2005 se sešlo 23 Solvayových kongresů. Tento kongres byl
        věnován tématu \emph{\uv{kvantová struktura prostoročasu}}\footnote{The Quantum Structure of
        Space and Time}.

        \luagraphicx[1]{fyz_fig0949.jpg}{Výkvět fyziků na \wikiSolvayConference v r. 1911:
        Zúčastnili se ho - sedící z leva do prava  \textsc{W. Nernst}, \textsc{M. Brillouin},
        \textsc{E. Solvay}, \textsc{H. Lorentz}, \textsc{E. Warburg} z Berlína, \textsc{J. Perrin},
        \textsc{W. Wien}, \textsc{M. Curie}, and \textsc{H. Poincaré}. Stojící zleva do prava:
        \textsc{R. Goldschmidt}, \textsc{M. Planck}, \textsc{H. Rubens}, \textsc{A. Sommerfeld} z
        Mnichova, \textsc{F. Lindemann}, \textsc{M. de Broglie}, \textsc{M. Knudsen} z Kodaně,
        \textsc{F. Hasenöhrl} z Vídně, \textsc{G. Hostelet}, \textsc{E. Herzen}, \textsc{J. H.
        Jeans}, \textsc{E. Rutherford}, \textsc{H. Kamerlingh Onnes} z Leidenu, \textsc{A. Einstein}
        a \textsc{P. Langevin}.}{fyz:fig0949}

        Jedním z nejpamátnějších se stal 5. kongres z roku 1927 na téma \emph{\uv{elektrony a
        fotony}}, kde se Einstein střetl s \textsc{Nielsem Bohrem} v diskuzi o interpretaci kvantové
        mechaniky. Einstein, který měl Iví podíl na vzniku kvantové fyziky a byl vlastně objevitelem
        částicově vlnových vlastností fotonu a fotonové statistiky a také zastáncem de Broglieovy
        hypotézy o vlnových vlastnostech částic, se nemohl smířit s tím, jakým směrem se kvantová
        mechanika začala ubírat. Vadilo mu, že by výsledky měření měly záviset na přístupu
        pozorovatele a že jsme schopni postihovat vždy jen jednu stránku chování částic. Zdálo se
        mu, že kvantová mechanika hypertrofuje úlohu náhody a reagoval známým výrokem, že
        \emph{\uv{Bůh přece nehraje v kostky}}. Bohr ho pak v diskusi napomenul, aby nepředepisoval
        Bohu, jak se má chovat. Celá diskuse byla ovšem velmi užitečná a ovlivnila myšlení fyziků na
        desetiletí. Zdá se, že i Einsteina jako fyzika s mimořádnou fantazií fantastický vývoj
        kvantové fyziky nakonec přece jen zaskočil.

        První Solvayův kongres se ovšem konal ještě v době před vznikem kvantové mechaniky a v
        těchto otázkách nebylo vůbec jasno. Rozpaky, ale i optimismus, které panovaly v roce 1911
        nad zářením a kvanty, nejlépe vyjádřil na kongresu předsedající \textsc{Lorentz}:
        \emph{\uv{Za tohoto stavu věcí krásná hypotéza o elementech energie, jak ji poprvé vyložil
        Planck a později uplatnil Einstein, Nernst a další u mnoha jevů, se nám jeví jako zázračný
        paprsek světla. Otevřela nám neočekávané výhledy. Dokonce i ti, kdo na ni pohlížejí s
        určitým podezřením, musí připustit její důležitost a plodnost}}

        Tisk, který nemohl veřejnost informovat o nepochopitelném obsahu jednaní, se zajímal spíše o
        přátelské vztahy paní Curieové a Langevina. \textsc{Paul Langevin} (1872-1946), teoretik
        feromagnetizmu, přední francouzský vědec a učitel, hrdina hnutí odporu 2. světové války,
        patřil stejně jako paní Curieová k nejbližším Einsteinovým přátelům.

        Einsteinův pobyt v Praze netrval dlouho. Už v srpnu 1912 se Einstein vrátil do Curychu,
        možná i na naléhání Milevy. V Curychu ostatně dostal konečné místo řádného profesora a
        vedoucího katedry. Na Prahu však vzpomínal rád a vrátil se do ní krátce ještě jednou v roce
        1921, aby tu přednášel o teorii relativity a také vyjádřil své sympatie českým fyzikům na
        Karlově univerzitě. Ani na curyšské technice však Einstein dlouho nepobyl: v červenci 1913
        ho osobně navštívil Planck s Nerstem a nabídli mu prestižní pozici profesora Berlínské
        univerzity a ředitele Fyzikálního ústavu císaře Viléma. Jeho zástupcem se stal \textsc{Max
        von Laue}.

        V roce 1915, kdy v Evropě už zuřila první světová válka, se Einstein odvážně angažoval v
        protiválečném úsilí. V záři odjel do neutrálního Švýcarska, aby se tu sešel s jiným
        významným pacifistou, francouzským spisovatelem \textsc{Romainem Rollandem}, a v témž roce
        si ještě „odskočil" do Holandska za \textsc{Lorentzem}, který mu svého času nabízel místo na
        univerzitě v Leidenu. Ačkoli byl plně zaměstnán otázkami obecné teorie relativity, stačil
        Einstein ještě předpovědět známy \textbf{Einsteinův-de Haasův jev} týkající se souvislosti
        mezi mechanickým a magnetickým moment. Při přemagnetováni feromagnetické tyče zavěšené podél
        magnetické osy by se tyč měla zároveň pootočit. Einstein předpověděl a experimentálně ověřil
        tento jev společně s významným holandským fyzikem de Haasem. \textsc{Wander Johannes de
        Haas} (1878-1960), odborník v oblasti supravodivosti, byl později dlouholetým ředitelem
        slavné laboratoře nízkých teplot v Leidenu.

      \subsubsection{Einstein a kosmos}\label{fyz:IchapIIsecIVssecIsssecVIII}
        \textsc{Einstein} strávil v Berlíně 19 let za složité a často hrozivé společenské situace, v
        níž se ne vždy dobře orientoval. \textsc{Max Planck} patřil přitom k jeho oporám a
        ochráncům. Jako snad jediný z německých fyziků Einstein hned na počátku 1. světové války
        nepodlehl nacionální euforii a nejen že odmítl podepsat ostudné prohlášení představitelů
        německých univerzit, ale jako jeden z mála podepsal manifest odsuzující válku. Ke konci
        války přepracovaný Einstein dlouhodobě onemocněl žloutenkou a žaludečními vředy, v nemoci ho
        ošetřovala jeho se sestřenice \textsc{Elsa Löwenthalová}. V roce 1919 se pak Einstein
        rozvedl s Milevou a oženil s Elsou.

        Na prahu 1. světové války, po příjezdu do Berlína, Einstein usilovně pracoval na obecné
        teorii relativity, nové teorii gravitace založené na představě o zakřiveném prostoru. S
        matematickou stránkou výpočtů mu pomáhal přítel \textsc{Marcel Grossmann}. Einstein se
        zpočátku domníval, že gravitační zakřivení prostoru se dá vyjádřit pomoci jednoho, v
        podstatě Newtonova gravitačního potenciálu. Jak už vytušil v Praze, uvažoval, že světelný
        paprsek vycházející z hvězdy se při průchodu v blízkosti slunečního okraje v silném
        gravitačním poli zakřiví a poloha hvězdy na obloze se zdánlivě posune o \num{0.87} úhlových
        vteřin. To je ovšem možné pozorovat pouze v okamžiku úplného zatmění Slunce, kdy jsou
        stálice v blízkosti slunečního okraje pozorovatelné.

        Einsteinův přítel z berlínské observatoře astronom \textsc{Erwin Freundlich} dychtil ověřit
        tento výsledek, a vypravil se v létě 1914 na Krym, kde mělo nastat dobře pozorovatelné
        zatmění Slunce. Výpravu částečně financovaly Kruppovy zbrojařské závody. Němečtí astronomové
        stačili vybalit své dalekohledy fotoaparáty ve Feodosii, nedaleko ruské námořní základny v
        Sevastopolu, ale než mohli začít se svou prací, vypukla 1. světová válka. Není divu, že byli
        považováni za německé vyzvědače a jako váleční zajatci internováni. Pro Einsteina to bylo
        svým způsobem štěstí, protože měření by bývala nepotvrdila jeho výsledek. Freundlich později
        uprchl před nacizmem z Německa, v. 1933 do Istanbulu, kde se stal zakladatelem tureckého
        astrofyzikálního výzkumu, v r. 1937 působil na pražské německé univerzitě, pak odjel do
        Holandska a nakonec na univerzitu St. Andrews ve Skotsku.

        Teprve v roce 1915 dospěl Einstein ke správnému tvaru rovnic obecné teorie relativity a
        zjistil, že gravitaci je nutno popisovat soustavou 10 nelineárních parciálních
        diferenciálních rovnic pro 10 potenciálů. Věc je tedy mnohem složitější, než si představoval
        Newton, a zdálo se, že se vůbec nepodaří tak složitou soustavu matematických rovnic vyřešit.
        Einstein přednesl svůj výsledek na zasedání Berlínské akademie věd v prosinci 1915 a
        uveřejnil ho v práci \emph{\uv{Základy obecné teorie relativity}}\footnote{Grundlagen der
        Allgemeinen Relativitätstheorie} v následujícím roce. Ale už pomocí poruchové teorie se dalo
        zjistit, že pro nepříliš silné gravitační pole, jaké vytváří např. naše Slunce, vyplývají
        odchylky od Newtonovy dynamiky. Tak se ukázalo, že perihelium Merkuru se musí stáčet o
        dodatečný úhel \ang{;;43} za století, a tím byl rozehnán další z mráčků nad fyzikou 19.
        století. Pro odchylování světelného paprsku hvězd procházejícího v blízkosti okraje Slunce
        vyšla hodnota dvakrát větší, než Einstein původně předpokládal, totiž \num{1.75} úhlových
        vteřin. Tento zdánlivě podružný, detailní výsledek znamenal úplný převrat v Einsteinové
        životě a založil jeho světovou slávu.

        Brzy po skončení světové války, 29. května 1919, se naskytla možnost pozorovat sluneční
        zatmění v Brazílii a v rovníkové Africe. Astronomického ověření nové teorie, která měla
        opravit, ne-li vyvrátit Newtonovou mechaniku, se ujali angličtí vědci. Královský astronom
        \textsc{Frank W. Dyson} v Greenwichi získal v povalečné úsporné době prostředky k uspořádání
        výprav k pozorování slunečního zatmění v Sobralu v brazilském státě Ceará a na ostrově
        Principe v Guinejském zálivu u afrického pobřeží. Brazilskou část výpravy vedl zkušený
        greenwichský astronom \textsc{Andrew C. Crommelin}, africkou část přesvědčený vyznavač
        teorie relativity \textsc{Arthur Stanley Eddington} (1882-1944), přední anglický astrofyzik
        a od r. 1914 člen londýnské Královské společnosti. Eddington působil na Greenwichské
        hvězdárně, od r. 1913 byl profesorem Cambridžské univerzity a ředitelem univerzitní
        observatoře, kam byly později aktivity historické a již nevyhovující Greenwichské hvězdárny
        přeneseny. Eddington se zabýval hlavně termodynamikou slunečního a hvězdného plazmatu a
        otázkou zdrojů zářivé energie hvězd. Jako kvaker sdílel s Einsteinem i pacifistické názory.
        Říká se, že kdosi prý Eddingtona oslovil: \emph{\uv{Jste prý jeden ze tří lidí na světě,
        který té zatracené Einsteinově teorii rozumí.}} Eddington se dlouze zamyslel a pak řekl:
        \emph{\uv{Nemohu si vzpomenout, kdo je ten třetí}}.

        Přes částečnou nepřízeň počasí obě expedice Einsteinovu předpověď potvrdily. 6. listopadu
        musel \textsc{J. J. Thomson}, předseda londýnské Královské společnosti, na jejím zasedání
        pod Newtonovým portrétem konstatovat pravdu má Einstein, a nikoli Newton. Události se chopil
        americký tisk, nová média, rozhlas a kino, začala šířit Einsteinovu slávu po celém světě.
        Fyzikové byli pohoršeni - za vědce přece mluví jeho dílo, a nikoli reklama. I Einstein byl
        zaskočen, ale nakonec se zapojil do hry a odpovídal na stovky dopisů a dotazů, vyjadřoval se
        ke všem možným i nemožným otázkám. Byl žádán o půjčky, přímluvy anebo jen o podpis, jeho
        osobní život se stal veřejnou věcí. Vyčerpáni válečnými útrapami, strádáním a epidemiemi se
        lidé chtěli radovat, prahli po senzacích. Dospívající dívky omdlévaly při pomyšlení, že
        světlo hvězd se odchyluje o \ang{;;1.75} a že hvězdy nejsou na nebi tam, kde je vidíme.

        Einstein byl pozván na přednáškové turné do Anglie a Francie, v roce 1921 navštívil Spojené
        státy, Indii, Čínu, r. 1922 Japonsko a na zpáteční cestě do Evropy Palestinu, kde kladl
        základní kamen Hebrejské univerzity v Jeruzalémě. V témž roce mu měla být udělena Nobelova
        cena za rok 1921, nikoli však za teorii relativity, nýbrž za vysvětlení fotoelektrického
        jevu a \emph{\uv{za práce pro rozvoj teoretické fyziky}}. Mohlo by se to zdát podivné, ale
        fyzikální obec stále ještě nepovažovala teorii relativity za dostatečně prokázanou. Einstein
        nemohl cenu převzít osobně, a tak nastalo dohadováni, zda má cenu za něho převzít německý
        nebo švýcarský vyslanec ve Stockholmu. Nakonec ji Einsteinovi předal Švédský vyslanec v
        Berlíně. V roce 1925 si Einstein ještě zajel na přednáškové turné do Brazílie.

        Obecná teorie relativity má pro silné gravitační pole \emph{závažné důsledky kosmologické}.
        Ačkoliv se to zdálo neuvěřitelné, podařilo se už roce 1916 německému matematikovi
        \textsc{Karlu Schwarzschildovi} (1873-1916) krátce před smrti najít řešení Einsteinových
        rovnic pro kulově symetrické gravitující těleso, tedy vlastně zobecnění Newtonova
        gravitačního zákona. Z tohoto řešení vyplynula také možnost existence singularit,
        superhustých stavů hmoty, které zakřivují prostoročas natolik, že nevypouštějí vůbec žádné
        světlo, a nesou proto označení \textbf{černé díry}. Einstein na základě svých rovnic
        předpověděl také existenci gravitačních vln, které jsou dnes předmětem usilovného hledání, a
        gravitačních čoček. Velké hmoty ve vesmíru mohou zakřivovat světelné paprsky a soustřeďovat
        je podobně jako světelné čočky a tak nám přibližovat vzdálené objekty. Velmi silná
        gravitační pole způsobují též podle Einsteina posuv spektrálních čar k červené části
        spektra, tzv. \textbf{gravitační červený posuv}. Pozorovacího důkazu těchto svých předpovědí
        se Einstein už ovšem nedožil.

        \begin{tcnote}
          Schwarzschildovo řešení bylo po dlouhá desetiletí jediným známým řešením Einsteinových
          rovnic. Teprve až v r. 1963 novozélandský teoretický fyzik \textsc{Roy Kerr} (nar. 1934)
          našel řešení, které platí pro rotující hmotná tělesa, např. černé díry. Klasická mechanika
          neuvažovala o tom, že by gravitační síla tělesa mohla záviset na tom, zda těleso rotuje
          nebo ne. V obecné teorii relativity však rotující těleso zakřivuje prostoročas kolem sebe
          jiným způsobem než nerotující a působí proto jinou silou. Kerrovo řešení bylo velmi
          překvapující. Kerr pocházel z Christchurche na Novém Zélandu, studoval v Oxfordu, působil
          ve Spojených státech a od r. 1971 byl profesorem matematiky a fyziky na Canterburské
          univerzitě v Christchurchi až do svého odchodu do důchodu.
        \end{tcnote}

        Už v.r. 1917. si Einstein uvědomil, že chce-li aplikovat své rovnice na cely vesmír tak, aby
        tento vesmír zůstával v čase neměnný a celková gravitace vesmíru zůstávala vyrovnána, musí
        své rovnice doplnit o malou konstantu ad hoc zvanou \textbf{kosmologický člen (kosmologická
        konstanta)}. Z matematického hlediska to bylo vlastně takové nehezké \uv{vyspraveni} jinak
        elegantní a krásné rovnice. V r. 1922 však neznámý ruský fyzik, vlastně meteorolog
        \textsc{Alexandr Alexandrovič Fridman} (1888-1925) uveřejnil článek, v němž ukázal, že
        Einsteinovy rovnice připouštějí docela možné řešení, podle nichž je vesmír zakřivený,
        uzavřený do sebe, konečný a bez hranic, a přitom se může rozpínat nebo smršťovat. Einstein s
        tímto názorem zprvu nesouhlasil, ale brzy uznal svou chybu a kosmologický člen ze svých
        rovnic odstranil; označil ho dokonce za svůj největší \emph{omyl}. 
        
        Právě 20. léta 20. století přinesla nové astronomické poznatky týkající se zejména struktury
        a chování galaxii. Jak víme, \textsc{Galileo Galilei} na úsvitu nové fyziky jako první
        pomocí dalekohledu rozpoznal povahu Mléčné dráhy, naší Galaxie, a zjistil, že je složena z
        nesmírného množství hvězd. Kolem roku 1925 americký astronom \textsc{Edwin Powell Hubble}
        (1889 - 1953) pomocí tehdy největších dalekohledů na observatořích Wilson a Mt. Palomar v
        Kalifornii zjistil, že naše Galaxie není ve vesmíru jediná a že to, co bylo dříve považováno
        za drobné chomáčky vzdálených mlhovin, jsou vlastně galaxie podobné naši se stovkami miliard
        hvězd. Galaxie se tak staly novým předmětem zkoumání astronomů. Dnes víme, že se galaxie
        pohybuji obrovskými rychlostmi, rotují, srážejí se, prolínají a vzájemně se \uv{požírají}.
        Vytvářejí dále vyšší seskupení, kupy galaxií a celou hierarchii vesmírných struktur.
        
        \textsc{Edwin Hubble} studoval na univerzitě v Chicagu, pak získal prestižní Rhodesovo
        stipendium do anglického Oxfordu, kde absolvoval studium práv a španělštiny. Byl také
        sportovně nadaný a za obou světových válek si získal i vojenské zásluhy. Po návratu ze
        studií v Oxfordu učil nějaký čas na střední škole, a když v r. 1919 dostal nabídku stát se
        členem astronomického týmu na hvězdárně Mr. Wilson, nic nenasvědčovalo tomu, že bude jedním
        z nejvýznamnějších hvězdářů 20. století.

        Při studiu spekter světla vysílaného vzdálenými galaxiemi v r. 1929 Hubble zjistil, že
        jejich spektrální čáry jsou posunuty k červenému konci spektra tím víc, čím je galaxie od
        nás dál. Vzdálenost galaxií Hubble určoval metodou proměnných hvězd, \textbf{cefeid}, jak
        je v astronomii obvyklé. Měřitelné rozměry vesmíru se tak rázem zvětšily na desítky milionů
        světelných let a stalo se zřejmým, že ani Země, ani Slunce, ani naše Galaxie nejsou nějakým
        jeho středem. Ale nejenom to. Nejvěrohodnější vysvětlení červeného posuvu spektrálních čar
        podle Hubbleova vztahu podává Dopplerův jev, podle něhož se tyto galaxie od nás vzdalují
        velkými rychlostmi tak, že se prostoročas vesmíru rozpíná. Tato možnost plyne i z
        Einsteinových rovnic obecné teorie relativity. Einstein si to ovšem uvědomil, mrzelo ho,
        rozpínání vesmíru nejen nepředpověděl, ale dokonce se ho snažil pomoci své kosmologické
        konstanty vyloučit.

        Hubbleovy objevy patřily k nejvýznamnějším v historii astronomie a můžeme se právem ptát,
        proč za ně Hubble nedostal Nobelovu cenu. Důvodem bylo to, že astronomie nebyla dlouho
        oficiálními hodnotiteli považována za součást fyziky, ale za víceméně popisnou samostatnou
        přírodní vědou. To se ovšem změnilo po druhé světové válce, která na čas odsunula tyto
        otázky do pozadí, a počínaje 70. léty 20. století začala být udělována celá řada Nobelových
        cen právě za objevy z oblasti astrofyziky. Hubble však přesto nepřišel tak úplně zkrátka.
        Byl po něm pojmenován vůbec nejdražší vědecký přístroj, zrcadlový dalekohled o průměru
        \SI{240}{\cm} za 2 miliardy dolarů, který byl v březnu 1990 umístěn na oběžnou dráhu kolem
        Země. I když musel pak být dodatečně na oběžné dráze v r. 1993 poopraven, podařilo se s ním
        získat tisíce krásných a detailních snímků vesmírných objektů, které by se pomocí i větších
        dalekohledů na povrchu Země pro atmosférické a světelné znečištění získat nepodařilo.

        Za svého berlínského pobytu zůstával sice Einstein vědecky aktivní, sledoval vývoj kvantové
        fyziky a přispíval k němu, ale zároveň ho kritizoval a pouštěl se do hlubokých a dnes již
        proslulých vědeckých sporů, zejména s Nielsem Bohrem, v nichž nakonec podlehl. Ve své
        vlastní vědecké práci se stále vracel k problému vytvoření jednotné teorie pole, která by
        spojila gravitaci a elektromagnetismus a byla by vlastně jakýmsi dalším zobecněním obecné
        teorie relativity. Tyto snahy, které sledoval do konce svého života, ho zaváděly stále více
        do slepých uliček a vysilovaly ho.

        S narůstající vlnou antisemitismu se začaly množit i útoky proti Einsteinovi ze strany
        nacisticky smýšlejících fyziků. V r. 1931 se objevil známý ubohý pamflet \emph{\uv{100
        autorů proti Einsteinovi}}, který útočil na teorii relativity jako na „židovskou fyziku“.
        Einstein se nad něj povznesl slovy, že pokud by měl pamflet pravdu, stačil by autor jeden.
        Začátkem 30 let navštívil Einstein s Elsou pětkrát Spojené státy, přednášel v Kalifornském
        technickém institutu v Pasadeně a v Ústavu pokročilých studií v Princetonu. Při první plavbě
        v r. 1930 se loď před vjezdem do Panamského průplavu zastavila v Havaně, kde se Einstein
        sešel s kubánskými vědci, navštívil chudinskou čtvrt a koupil si typický kubánský klobouk

        V Německu se zatím Hitler připravoval k převzetí moci, fyzikové židovského původu byli
        ohroženi a postupně emigrovali. Ze své poslední cesty do Ameriky v r. 1933 se Einstein už do
        Německa vrátit nemohl, na jeho hlavu byla vypsána cena 50 000 marek, byl v nepřítomnosti
        odsouzen k trestu smrti, jeho dům a knihy byly zničeny. Einsteinova nevlastní dcera Margot
        zachránila alespoň jeho vědecký archív tím, že ho převezla na francouzské vyslanectví v
        Berlíně. Einstein přerušil všechny styky s německými vědeckými institucemi k velké lítosti
        Plancka, který se marně snažil za něho orodovat. Nějakou dobu se Einstein zdržel v Belgii
        jako blízký přítel královny, pak v Anglii v Oxfordu a zajel ještě do Paříže, aby se
        rozloučil s fyzikem Langevinem umírající paní Curieovou. V listopadu 1933 spolu s Elsou a
        oddanou sekretářkou Helenou Ducasovou vyplouvá Einstein z Le Havru do Anglie a odtud do
        Spojených států, aby nakonec zakotvil v americkém Princetonu.

        Od r. 1938 žil v Americe v Kalifornii Einsteinův starší syn Hans Albert se svou rodinou;
        stal se uznávaným odborníkem na vodní stavby a hydrotechniku. S otcem se však v Americe
        příliš nestýkal.

        Za válečných a poválečných let žil Einstein v ústraní ve svém princetonském domě jako jakási
        živoucí ikona. Sice přednášel a stýkal se s mladými fyziky, ale ti k němu vzhlíželi sice s
        úctou, ale i s pocitem jakési generační bariery. Byl pro ně jedním ze zakladatelů moderní
        fyziky, který se ale už ocitl mimo hlavní proud fyzikálního výzkumu, a pro jeho stále
        bezvýslednou snahu o jednotnou teorii pole, k níž se Einstein vracel ještě na smrtelném
        lůžku, neměli pochopení. Zpočátku za Einsteinem přijížděli z Evropy jeho vědečtí přátelé,
        pak fyzikové prchající před nacismem a Einstein jim poskytoval všemožnou pomoc. Změnil
        dokonce své pacifistické smýšlení a vyzýval k ozbrojenému odporu proti zlu. Jak se ještě
        zmíníme, propůjčil své jméno i ke známé iniciativě směrem k prezidentu
        \textsc{Rooseveltovi}, která nakonec vedla k zahájení amerického jaderného programu a
        konstrukci prvních jaderných bomb.

        V posledních letech svého života se Einstein cítil osamělým, postupně umírali jeho blízcí a
        přátelé a pro své politické postoje a styky se v očích amerických úřadů stal „podezřelým".
        Angažoval se ovšem v mírových iniciativách, vystupoval proti nespravedlnostem ve světě, ale
        zůstával v něm pocit horkosti. Na Německo zanevřel úplně, už se tam nevrátil. V roce 1952 mu
        byl nabídnut post prezidenta Izraele, ale tuto poctu odmítl (zřejmě k úlevě všech
        zúčastněných). Zemřel náhle na prasklou výduť aorty 18. dubna 1955. Jeho popel byl rozptýlen
        na neznámém místě a jeho mozek podroben lékařskému výzkumu; podstatu jeho geniality však
        lékaři na preparátu nezjistili. Na Einsteinovu počest byl transuranový prvek s atomovým
        číslem 99 nazván einsteinium.

        Zdálo by se, že s Einsteinovou smrti odešel geniální klasik fyziky a dnešní doba ho už dávno
        předběhla. Ve skutečnosti je ale otázkou, kdo koho předběhl. Dnes, po 50 letech od
        Einsteinovy smrti, se vytvoření jednotné teorie částic a polí opět stává cílem nových
        generací fyziků a obrovské částky investované do velkých urychlovačů a detektorů mají tomuto
        cíli napomoci. Nic na tom nemění skutečnost, že cesty k tomuto cíli mohou být jiné, než si
        Einstein představoval. Pokud jde o výzkum vesmíru, nabral nový dech krátce po Einsteinově
        smrti a jeho bouřlivý rozvoj probíhá přesně podle scénáře obecné teorie relativity.
        Einsteinovo jméno je dnes ve fyzikální literatuře citováno mnohem častěji než za jeho
        života.

        Astrofyzice šlo především o to zjistit, odkud se bere obrovská energie hvězd, kterou mohou
        vyzařovat po miliardy let. To ovšem nebylo možné před vznikem jaderné fyziky a teorie
        relativity. Jaderná fyzika se na přelomu 20. a 30. let 20. století teprve rodila, dokonce
        ještě nebyl ani objeven neutron a k možnosti uvolňování jaderné energie ve velkém měřítku
        byli fyzikové většinou skeptičtí. Přesto už v r. 1920 velký astronom Eddington, inspirován
        Einsteinovým vztahem mezi hmotností a energií, přišel s domněnkou, že energie Slunce a hvězd
        se uvolňuje při jaderných reakcích slučováním jader lehčích atomů na těžší. Setkal se s
        nepochopením a výsměchem.

        Myšlenka, že energie hvězd pramení z termojaderných reakci slučování lehkých atomových
        jader, se objevila znovu r. 1929. Přišli s ní velšský astronom \textsc{Robert D'Escourt
        Atkinson} (1898-1982) a \textsc{Friedrich George Houtermans} (1903-1966). Houtermans byl
        synem holandského bankéře, narodil se v Sopotech u Gdaňska, studoval v Göttingenu a jeho
        život by vydal za dobrodružný román. Pracoval v Anglii, v Sovětském svazu i v nacistickém
        Německu, v obou těchto zemích byl na výsluní i v žalářích, od r. 1952 byl profesorem
        univerzity v Bernu. Zkoumal možnost řetězové jaderné reakce, principy laserů, měřil stáři
        meteoritů izotopovými metodami.

        Jaderné reakce a jejich cykly, které jsou podle Einsteinova vztahu mezi hmotností a energií
        zdrojem obrovské vyzařované energie hvězd, prozkoumal \textsc{Hans Albrecht Bethe}
        (1906-2005), Einsteinův přítel a nositel Nobelovy ceny za rok 1967. Bethe se narodil ve
        Štrasburku, přednášel na německých univerzitách, protože měl židovskou matku, musel v r.
        1933 emigrovat do Spojených států. Významně přispěl k teoretické jaderné fyzice a účastnil
        se také prací na přípravě americké vodíkové bomby.

        To už astrofyzikové zkoumali i různá vývojová stadia hvězd a jejich strukturu a zjistili, že
        hvězdy mají svůj život, vznik a zánik a že jejich osud záleží především na jejich hmotnosti.
        \textsc{Subramanian CHandrasekhar} (1910-1995), americký astrofyzik indického původu,
        zjistil, že pokud hmotnost hvězdy překročí tzv. \textbf{Chandrasekharovu mez}, hvězda se po
        vyčerpání jaderné energie zhroutí a stane se z ní bílý trpaslík. Na základě Chandrasekharovy
        předpovědi objevili astronomové postupně několik tisíc bílých trpaslíků, které často tvoří
        velmi hmotnou složku dvojhvězdy. Chandrasekhar dostal svou Nobelovu cenu r. 1983.
        Proměřování spekter bílých trpaslíků vedlo k celkem uspokojivému potvrzení
        \textbf{gravitačního červeného posuvu} spektrálních čar, který vyplývá z Einsteinovy teorie.

        Je-li ovšem hmotnost hvězdy ještě větší, může explodovat jako supernova a zanechat po sobě
        nesmírně hustý zbytek známý jako neutronová hvězda, případně může kolabovat až na černou
        díru. Těmito exotickými stádii vývoje hvězd se zabýval jiný americký jaderný fyzik
        \textsc{John Archibald Wheeler} (1911 až 2008), který také r. 1967 název \textbf{černá díra}
        zavedl. Jak víme, existence černých děr vyplývá z řešení Einsteinových rovnic a spekulovali
        o nich už kdysi \textsc{Michell} a \textsc{Laplace}. Wheeler působil od r. 1938 jako
        profesor Princetonské univerzity a zabýval se problémy jaderných reakcí, teorií jaderného
        reaktoru a projekty získávání termojaderné energie. Při dramatických kolapsech a explozích
        hmotných hvězd dochází k vytváření atomových jader těžkých prvků, tzv.
        \textbf{nukleosyntéze}. Nacházíme-li tedy na Zemi a v našem těle atomy těžkých prvků,
        znamená to, že kdysi musela poblíž Země explodovat supernova a že jsme poeticky řečeno
        \emph{\uv{hvězdami zrození}}. Teorií vzniku prvků v nitru hvězd vypracoval další z
        amerických jaderných fyziků, \textsc{William Alfred Fowler} (1911-1995), který se r. 1983
        podělil s Chandrasekharem o Nobelovu cenu.
        
        Jak se zdokonalovaly pozorovací metody nejen v optické oblasti, ale i v oblasti jiných
        vlnových délek elektromagnetického záření, podařilo se rozeznat strukturu galaxií a
        galaktických kup, pohyb jednotlivých hvězd v galaxiích, chování a rentgenové vyzařování
        mezihvězdného plynu v galaxiích, ale také třeba pohyb galaxii v mezigalaktickém prostoru a
        jejich vzájemné srážky. Vysvětlení pozorovaných jevů pomocí Einsteinovy teorie gravitace
        vyžaduje znát hmotnost těchto galaxii. Tu můžeme určit, provedeme-li jakousi bilanci hvězd a
        jiných svíticích objektů v galaxii; zdrojem elektromagnetického záření jsou přitom atomy a
        atomová jádra tvořená především hmotnými protony a neutrony (tyto částice řadíme mezi tzv.
        \textbf{baryony}, proto někdy mluvíme o svíticí hmotě nebo baryonové hmotě). Při tom musíme
        vzít v úvahu, že galaxie obsahuje i slabé svíticí nebo vůbec nesvítící objekty, jako jsou
        bílí trpaslíci, neutronové hvězdy, popřípadě černé díry. Hmotnost galaxií a galaktických kup
        můžeme určovat také podle jejich gravitačních účinků na světelné paprsky, které vycházejí ze
        vzdálenějších zdrojů a procházejí těmito galaxiemi. V příznivých případech dochází k tomu,
        že se bodový zdroj světla zobrazí jako prstenec a bude se jevit jasnější Je to efekt tzv.
        \textbf{gravitační čočky}, který předpověděl Einstein už v r. 1936

        Uvedená měřeni, která vlastně probíhají dosud, ale ukázala, že gravitační síly v galaxiích
        jsou větší, než by odpovídalo jejich zjištěné hmotnosti, že galaxie jsou kompaktnější jakoby
        svírány mohutnějšími gravitačními silami. Kvantitativně bylo odhadnuto, že pozorovaná
        baryonová hmota galaxie tvoří méně než jednu čtvrtinu hmotnosti potřebné k vysvětlení
        pozorovaných jevů. Větší část tvoři dosud nezjištěná neviditelná, skrytá, \textbf{„temná“
        hmota}. Tato záhadná hmota se projevuje pouze svými gravitačními účinky, nepůsobí na ni ani
        elektromagnetické, ani silné jaderné sily. Uvažovalo se o tom, že by mohla být tvořena
        neutriny: otázka, zda mají neutrina klidovou hmotnost a jak velkou, není však zatím
        definitivně rozřešena. Podstata \uv{temné hmoty} zůstává tedy temnou a patri zřejmě do dějin
        budoucí fyziky. 

        Mnohem fantastičtější situace však nastala v posledním desetiletí v souvislosti s rozpínáním
        vesmíru, které vyplynulo z Hubbleova objevu červeného posuvu spektrálních čar ve spektrech
        vzdálených galaxii. Na základě Hubbleova vztahu mezi rychlostí ubíhajících galaxií a jejich
        vzdálenosti bylo možno usoudit, že vesmír, prostor i čas vznikly z jakéhosi nesmírně
        zhuštěného stavu, matematicky řečeno singularity, o nepředstavitelně velké energii mohutnou
        explozi před \num{13.5} miliardami \num{+-200} miliony let a od té doby se rozpínají. Je
        dobře si uvědomit, že tato exploze nenastala v nějakém určitém bodě, \emph{\uv{středu
        vesmíru}}, od něhož se všechny galaxie vzdalují. Rozpíná se prostor, resp. prostoročas,
        takže se všechny galaxie vzdaluji od sebe navzájem. Při rozpínání ovšem vesmír zároveň
        chladl a z původní horké směsi elementárních částic a antičástic se postupně utvářely
        nukleony, atomy, molekuly, hvězdy a galaxie a také elektromagnetické zářeni. Tento scénář je
        v souladu s tzv. \textbf{standardním modelem} stavby hmoty v částicové fyzice. Hypotéza o
        rozpínání vesmíru nemohla být ovšem založena jen na interpretaci spekter galaxií pomocí
        Dopplerova jevu, bylo třeba najit další experimentální potvrzení teto hypotézy.
        Elektromagnetické záření vzniklé anihilaci částic a antičástic krátce na počátku rozpínaní
        vesmíru nemohlo nikam zmizet a při rozpínání vesmíru mohlo jen chladnout: výpočty ukazovaly,
        že by dnes toto „zbytkové" \textbf{reliktní záření} mělo mít charakter zářeni černého tělesa
        o teplotě několika kelvinů a vlnových délkách s maximem několika milimetrů. 

        Existenci reliktního záření předpověděl americký fyzik ukrajinského původu \textsc{George
        Gamow} (1904-1968) v r. 1948. Gamow se původně zabýval otázkami jaderné fyziky, teorií
        radioaktivního rozpadu alfa, ale také astrofyzikou, kosmologií a biologií. Uvědomil si, že s
        našim vesmírem prožíváme stav obrovského výbuchu. Počátek tohoto výbuchu dostal vlastně
        posměšný komiksový název \textbf{Big Bang!}, \emph{Velký třesk}. Dal mu ho Gamowův odpůrce
        \textsc{Fred Hoyle}, který v hypotézu Velkého třesku nevěřil a snažil se udržet model
        stacionárního vesmíru, v němž stále vzniká nová hmota, \emph{\uv{z ničeho}}.

        Reliktní elektromagnetické záření, které nevychází z žádného zdroje a zaplňuje celý vesmír,
        bylo nakonec objeveno \textsc{A. Penziasem} a \textsc{R. Wilsonem} v r. 1964 a znamenalo
        pádnou podporu hypotéze Velkého třesku. Našly se i další důkazy. Jedním z nich je chemické
        složení vesmíru, především poměr množství vodíku a helia jako nejrozšířenějších prvků. Po
        Velkém tresku musela zůstat i reliktní neutrina, která se také nemohla nikam ztratit a která
        jsou dnes postupně registrována

        Druhá světová válka, která sice zbrzdila rozvoj astronomie, urychlila na druhé straně vývoj
        mohutných raket, které nakonec posloužily ke vzniku kosmonautiky a také ke zdokonaleni
        techniky velmi krátkých elektromagnetických vln pro potřeby radiolokace, jež se nakonec
        uplatnila i v radioastronomii. Vedle optických dalekohledů začaly oblohu prohledávat
        obrovské radioteleskopy a pátrali po zdrojích rádiových a infračervených vln z vesmíru.
        Počátkem 60. let 20 století byly zejména v Cambridge na obloze zmapovány zdroje radiových
        vln, které přicházely z hlubin vesmíru. V té době se podařilo zjistit, které svítící objekty
        tyto radiové vlny vysílají. Přitom se ukázalo, že ve spektru těchto objektu dochází k
        silnému červenému posuvu, takže se od nás vzdalují podle Hubbleova vztahu obrovskými
        rychlostmi. Nejde vlastně o hvězdy, ale nejvzdálenější pozorované zdroje světla ve vesmíru,
        možná jádra vzdálených galaxii. Jsou to jakési \emph{„kvazistelární rádiové zdroje"}, a
        proto také dostaly název \textbf{kvasary}. Vzhledem k intenzitě jejich záření nemůže jejich
        energie pramenit z termojaderných reakcí, ale nejspíš z pohlcování okolní hmoty, včetně
        celých hvězd a jejich vtahování do černé díry.

        Rozpínaní vesmíru nebylo vždy rovnoměrné. V dávné minulosti musela zřejmě nastat tzv.
        \textbf{inflační fáze}, kdy se vesmír náhle velmi prudce rozepnul. Svědčí o tom skutečnost,
        že dnes je pozorovaný vesmír pozoruhodně homogenní, rozloženi hmoty v něm je víceméně
        stejnorodé. Astrofyzikové si samozřejmě kladou otázku, jak bude rozpínání vesmíru
        pokračovat, zda bude trvat věcně (?), nebo zda gravitační přitažlivé sily přinutí vesmír po
        nějaké době se opět smršťovat. To ovšem záleží na tom, kolik je ve vesmíru gravitačně
        působící hmoty
        
        V letech 1987-88 dvě skupiny astronomů, v kalifornském Berkeley a v Dánsku, na základě
        pozorování spekter výbuchů supernov v nesmírně vzdálených galaxiích dospěli k závěru, že se
        rozpínaní vesmíru v současné době urychluje. Vyvodili odtud závěr, že vesmír musí obsahovat
        námi zatím nepozorovanou energii, která působí proti přitažlivým gravitačním silám. Říka se
        jí \emph{\uv{temná energie}} a patří k nejnovějším záhadám astronomie. Odhaduje se dokonce,
        že pozorujeme pouhá \SI{4}{\percent} hmoty ve vesmíru, dalších \SI{22}{\percent} tvoří už
        zmíněná temná gravitující hmota v galaxiích a celých \SI{74}{\percent}  antigravitující
        temná energie. To by pro nás bylo ovšem velmi tristní znamenalo by to, že o vesmíru nevíme
        skoro nic. Uvažuje se o temné energii v souvislosti s vlastnostmi vakua, o existenci nových,
        dosud neznámých částic a polí, o tom, zda gravitační zákon na velkých vzdálenostech není
        třeba nějak zobecnit a vlastně i zda správně interpretujeme červený posuv ve spektrech
        dalekých supernov. A dochází i na starou dobrou kosmologickou konstantu, kterou Einstein
        kdysi vyspravoval gravitační rovnice. Vyprovodil ji sice ze své teorie dveřmi, ale zda se,
        že se zase vrací do fyziky okny.
        
        Už v r. 1913 si Einstein uvědomil, že jeho rovnice obecné teorie relativity připouštějí i
        vlnová řešení, a tedy možnost existence \textbf{gravitačních vln}. Tyto vlny, vlastně vlnky
        křivosti prostoročasu, by se měly šířit rychlostí světla a měly by vznikat při prudkém
        zrychlení nebo nesymetrických kolapsech obrovských vesmírných hmot, jaké představují např.
        výbuchy supernov. V pozemských laboratořích pochopitelně nic podobného vytvořit neumíme, a
        tak nezbývá než sledovat takové jevy ve vesmíru. Gravitační vlny jsou příčné a při příchodu
        k Zemi rozechvívají, deformují hmotná tělesa. Tato relativní deformace je ovšem nesmírně
        malá, menší než jedna triliontina procenta, a právě takovou přesnost se podařilo dosáhnout v
        r. 2015. První \emph{přímé} pozorování gravitačních vln ze splynutí vesmírných objektů
        proběhlo 14. září 2015 v detektoru LIGO označované jako GW150914, v době, kdy byl tento
        detektor ještě v testovací fázi. 

        Přestože detekce gravitačních vln vyžaduje nesmírně přesná měření, už od 60. let 20. století
        se našli nadšenci, \emph{\uv{lovci gravitačních vln}}, kteří se pokoušeli tyto vlny
        zachytit. Průkopníkem byl americký fyzik \textsc{Joseph Weber} (1919 - 2000), známý tím, že
        už v r. 1952 navrhl zesilovač elektromagnetických vln na principu indukované emise, tedy
        maser a laser. Weber používal dva velké pružně upevněné hliníkové válce rozmístěné na velké
        vzdálenosti a sledoval případy, kdy se oba válce dopadající gravitační vlnou z vesmíru
        rozkmitají současně. Tím měl být vyloučen vliv místních otřesů. Přes značnou publicitu
        těchto pokusů se Weberovi gravitační vlny ulovit nepodařilo. V Rusku k nadšeným
        experimentátorům - lovcům gravitačních vln - patřil ruský fyzik \textsc{Vladimír Borisovič
        Braginskij} (1931). Nevýhodou těchto původních detektorů gravitačních, bylo také to, že se
        nevědělo na jakém kmitočtu se má zesilovat.

        První roky nového tisíciletí byly ve znamení nástupu nové generace detektoru gravitačních
        vln, nákladných obřích laserových širokopásmových interferometrů o délce ramen několika
        kilometrů. Patři k nim např. interferometr GEO 600 v Ústavu A. Einsteina u Hannoveru v
        Německu, LIGO v Hanfordu ve státě Washington a v Livingstonu v Luisiane v USA a VIRGO poblíž
        italské Pisy. Do budoucna se uvažuje i o interferometrech umístěných na družicích. Nákladné
        stavby těchto detektorů umožnil astronomický objev \textbf{pulzarů}, rychle rotujících
        nesmírně hmotných neutronových hvězd v r. 1967. Jejich další zkoumání, umožnilo nepřímé
        pozorování gravitačních vln. 

        Při proměřování oblohy pomocí radioteleskopu v Cambridgi pod vedením \textsc{Antonyho
        Hewishe} (1924) byla zpracováním pozorovaných údajů pověřena mladá doktorandka
        \textsc{Jocelyn Bellová} (1943). Při této práci musela prohlédnout pásy registračního papíru
        o délce \SI{5.6}{\km}. Jocelyn si všimla, že z některých zdrojů na obloze přicházejí velmi
        pravidelné krátké radiové signály, jako by je vysílali \emph{„malí zelení mužičkové"}.
        Ukázalo se však, že to bohužel (?) nejsou signály mimozemských civilizací, ale že je
        vysílají rychle rotující neutronové hvězdy, jejichž obrovská gravitace vtlačila elektrony do
        atomových jader a přeměnila hvězdu v jedno obrovské atomové jádro. Tim se jednak potvrdila
        existence do té doby jen tušených \textbf{neutronových hvězd} a zahájila se éra výzkumu
        těchto exotických astronomických objektů, chovajících se přesně podle rovnic obecné teorie
        relativity. Jocelynin školitel Hewish, který existenci pulzaru nepředpokládal, nehledal je,
        a také nenašel, dostal za objev Nobelovu cenu r. 1974, slečna Jocelyn, která je objevila,
        cenu nedostala. V témž roce byl oceněn Nobelovou cenou též sir \textsc{Martin Ryle}
        (1918-1984) za mnohonásobné zpřesnění radioastronomických měření. Paní Bellová Burnellová
        dnes zastává prestižní funkci prezidentky Britského fyzikálního institutu.

        Od té doby bylo objeveno na 1 000 pulzarů rotujících různými rychlostmi, prozkoumány jejich
        vlastnosti a magnetická pole a prokázáno, že jde o zbytky po explozích supernov. Podobně
        jako hvězdy se některé pulzary vyskytují v párech, dvojicích, jejichž složky obíhají
        vzájemně kolem sebe Nejslavnějším z nich se stal binární pulzar PSR B1913 + 16 v souhvězdí
        Orla, který objevili v r. 1974 američtí radioastronomové \textsc{Joseph Hooton Taylor Jr.}
        (1941) a \textsc{Russel Alan Hulse} (1950) pomoci radioteleskopu v Arecibu na ostrově
        Portorico. Tento obrovský radioteleskop a zároveň vysilač rádiových pulzů slouží i pokusům o
        navázáni kontaktů s mimozemskými civilizacemi. Binární pulzar 1913 + 16 se chová přesně
        podle Einsteinovy teorie. Při oběhu obou složek dochází k odchylkám od eliptické dráhy,
        projevuje se stáčeni periastra o nepřehlédnutelných \ang{4,24} za rok, jsou pozorovány
        gravitační červený posuv, dilatace času, gravitační zpoždění signálu, a především zkracování
        oběžné periody o \SI{0.04}{\s} za rok. Toto zkracování je způsobeno tím, že pulzar neustále
        vyzařuje gravitační vlny, ztrácí energii a obě jeho složky na sebe asi za 200 milionů let
        vzájemně spadnou. Astrofyzikové tak mají nepřímý důkaz existenci gravitačních vln a tento
        pulzar považují za jakousi učebnici Einsteinovy obecné teorie relativity. Není proto divu,
        že pánové Taylor a Hulse do stali za svůj objev v roce 1993 Nobelovu cenu.

        I když řada pozorovaných astronomických jevů zůstává stále nevyjasněna, zůstává Einsteinova
        obecná teorie relativity jako teorie gravitace pevným základem moderní astrofyziky. Ani v
        budoucnosti neztratí svou platnost, bude ale jistě dále zobecněna, tak jako Einstein
        zobecnil Newtonovu teorii gravitace. Jestliže Einstein silou svého ducha položil základ
        teorii relativity, kvantové fyzice a atomové fyzice, stal se v nemenší míře i jedním ze
        zakladatelů kosmologie a zákonodárcem vesmíru.


    \subsection{Kvantová Fyzika}\label{fyz:IchapIIsecIVssecII}
      Když jsme načrtli představu elektromagnetického pole, v němž se mohou šířit vlny, brzy
      zjistíme, že tyto vlny se chovají nezvykle, jako kdyby to ani vlny nebyly. Při vyšších
      frekvencích se více podobají \emph{částicím}! Jejich neobvyklé chování vysvětluje
      \emph{kvantová mechanika}, jejíž vznik je spojován s obdobím těsně po roce 1920. Před rokem
      1920 pozměnil Einstein obraz trojrozměrného prostoru a nezávislého času nejdříve na kombinaci,
      kterou nazýváme \emph{prostoročasem} a potom na \emph{zakřivený} prostoročas, aby vystihl
      gravitaci. „Scéna“ se změnila na prostoročas a o gravitaci předpokládáme, že je modifikací
      prostoročasu. Zjistilo se dokonce, že zákony pro pohyb částic jsou nepřesné. Mechanické zákony
      „setrvačnosti“ a „síly“ jsou \emph{nesprávné} - Newtonovy zákony neplatí ve světě atomů.
      Zjistilo se, že věci se v malém měřítku chovají úplně jinak než věci ve velkém měřítku. To
      dělá fyziku obtížnou, ale velmi zajímavou. Obtížnou proto, že chování věcí malých rozměrů je
      pro nás „nepřirozené“, nemáme v tomto směru přímé zkušenosti. Věci se tu chovají úplně jinak,
      než jsme zvyklí, a proto není možné popsat jejich chování jinak, než analyticky. Takový popis
      je těžký a vyžaduje mnoho představivosti.
      
      Kvantová mechanika má mnoho zvláštností. Především vylučuje předpoklad, že částice má určitou
      polohu a určitou rychlost. Abychom ukázali, do jaké míry je klasická fyzika správná, uvedeme
      pravidlo kvantové mechaniky, které říká, že není možné současně vědět, kde se něco nachází a
      jak rychle se to pohybuje. Neurčitost v hybnosti a neurčitost v poloze jsou
      \emph{komplementární} a jejich součin je konstantní. Můžeme to zapsat následujícím způsobem:
      \(\Delta x \Delta p \frac{\si{\planckbar}}{2\pi}\). Podrobněji bude o tomto principu mluveno
      později. Vysvětluje se tím velmi záhadný paradox: jsou-li atomy složeny z kladných a záporných
      nábojů, proč se záporný náboj prostě neusadí na kladném náboji (tyto náboje se přitahují) a to
      tak těsně, že by ho úplně vyrušil? \emph{Proč jsou atomy tak velké}? Proč je jádro uprostřed a
      elektrony okolo něho? Zpočátku se myslelo, že příčinou je velký rozměr jádra; jenže jádro je
      velmi malé. Atom má průměr okolo \SI{10e-10}{\meter}. Jádro má průměr asi \SI{10e-15}{\meter}.
      Kdybychom měli atom a chtěli bychom vidět jeho jádro, museli bychom ho zvětšit tak, aby dosáhl
      velikosti místnosti a i potom by bylo jádro malé jako skvrnka, kterou sotva spatříte okem, ale
      téměř \emph{všechna hmotnost} atomu připadá na toto nepatrné jádro. Co brání elektronu prostě
      spadnout na jádro? Právě uvedený princip. Kdyby elektrony byly v jádru, znali bychom přesně
      jejich polohu a princip neurčitosti by si potom vyžadoval, aby měly velmi velkou (ale
      \emph{neurčitou}) hybnost, tj. velmi velkou \emph{kinetickou energii}. S takovou energií by se
      odtrhly od jádra. Dochází proto ke kompromisu: elektrony si ponechají jakýsi prostor pro tuto
      neurčitost a potom se ve shodě s tímto pravidlem pohybují s jistým minimálním množstvím
      pohybu. (Vzpomeňte si, že atomy krystalu při ochlazení na absolutní nulu neustaly ve svém
      pohybu, ale přece jen kmitaly. Proč? Kdyby se přestaly pohybovat, věděli bychom, kde se
      nacházejí a že mají nulový pohyb a to by bylo v rozporu s principem neurčitosti. Nemůžeme
      vědět, kde jsou a jak rychle se pohybují; proto atomy musí neustále kmitat!)
      
      Jinou, velmi zajímavou změnou v ideách a filozofii vědy, kterou přinesla kvantová mechanika,
      je nemožnost přesně předpovědět, co se za jakýchkoli daných okolností odehraje. Například, je
      možné připravit atom, který bude emitovat světlo, a můžeme zjistit, kdy k této emisi došlo
      tím, že zachytíme foton (o tomto si brzy řekneme více). Nemůžeme však dopředu předpovědět, kdy
      se uskuteční emise světla, nebo v případě více atomů, který z nich bude emitovat světlo. Možná
      se domníváte, že je to proto, že v atomu se nacházejí jakási vnitřní „kolečka“, která jsme
      ještě nerozeznali. Ne, taková vnitřní kolečka neexistují! Příroda, tak jak ji dnes chápeme, se
      chová tak, že je principiálně nemožné přesně předpovědět, co se skutečně stane v daném
      experimentu. 
      
      Opět se vrátíme ke kvantové mechanice a základní fyzice, ale nebudeme zabíhat do podrobností
      kvantově mechanických principů, protože jsou dost těžké k pochopení. Budeme prostě
      předpokládat jejich existenci a ukážeme, k jakým následkům vedou. Jedním z následků je, že
      věci, které jsme považovali za vlny, se chovají jako částice a částice zase jako vlny; ve
      skutečnosti se tedy všechno chová stejně. Není rozdíl mezi vlnou a částicí. \textbf{Kvantová
      mechanika sjednocuje myšlenku pole, jeho vln a částic vjedno.} Při nízkých frekvencích je
      aspekt pole více zřejmý, resp. užitečnější pro přibližný popis vyjádřený řečí naší každodenní
      zkušenosti. Se vzrůstem frekvence však zařízení, které obvykle používáme v experimentu,
      poskytuje spíše důkazy o částicích. I když mluvíme o vysokých frekvencích, musíme přiznat, že
      v oblasti frekvencí nad \SI{10e12}{\Hz} nebyl zatím zjištěn žádný jev přímo související s
      frekvencí. K existenci vyšších frekvencí docházíme pouze úvahou vycházející z energie částic a
      předpokladu správnosti \emph{vlnově-korpuskulární představy kvantové mechaniky}.
      
      Takto docházíme i k novému pohledu na \emph{elektromagnetickou interakci}. Kromě elektronu,
      protonu a neutronu existuje nový druh částice. Tuto částici nazýváme foton. Nový pohled na
      interakci elektronů a protonů, tj. \emph{elektromagnetickou teorii}, která zároveň
      \emph{splňuje} zákonitosti \emph{kvantové mechaniky}, nazýváme \emph{kvantovou
      elektrodynamikou}. Tato základní teorie \emph{interakce světla a hmoty}, nebo
      \emph{elektrického pole a nábojů}, je dosud největším úspěchem fyziky. V této jediné teorii
      máme základní zákony, jimiž se řídí všechny známé jevy s výjimkou gravitace a jaderných
      procesů. Pomocí kvantové elektrodynamiky můžeme vysvětlit všechny známé zákony mechaniky,
      elektřiny a chemie. Plynou, zní zákony srážek kulečníkových koulí, pohyb vodičů v magnetickém
      poli i tepelná kapacita oxidu uhelnatého, barva neonových reklam, hustota soli, reakce vodíku
      a kyslíku při vzniku vody - to vše jsou následky jediného zákona. Všechny tyto detaily je
      možné získat, je-li situace dost jednoduchá na to, abychom ji mohli přibližně popsat. To sice
      není splněno téměř nikdy, často však můžeme pochopit více či méně, co se vlastně děje. Dosud
      se neobjevily žádné výjimky ze zákonů kvantové elektrodynamiky, až na atomová jádra. O jádrech
      však nemůžeme říci, jestli jde v jejich případě o výjimku, protože vlastně nevíme, jaké
      procesy v nich probíhají. Při budování teorie jádra musíme překonat tři hlavní problémy:
      \begin{enumerate}[noitemsep]
      \item Není znám přesný tvar sil působících mezi nukleony v jádře,
      \item rovnice popisující pohyb nukleonů v jádře jsou velmi komplikované - problém  
            matematického popisu,
      \item jádro má zároveň příliš mnoho nukleonů (nedá se popsat pohyb každé jeho částice) i    
            příliš málo (nedá se popsat jako makroskopické spojité prostředí).   
      \end{enumerate}
      Proto se musíme spokojit pouze s modely atomového jádra. 
      
      V podstatě je kvantová elektrodynamika teorií celé chemie a všech životních procesů, je-li
      možné život v konečném důsledku redukovat na chemii, nebo vlastně na fyziku, protože chemie
      vede k fyzice (a ta část fyziky, která se uplatňuje v chemii, je již dobře známá). Navíc,
      kvantová elektrodynamika - ta úžasná vědní disciplína - předpověděla mnoho nových věcí.
      Především mluví o vlastnostech fotonů velmi velkých energií, paprscích gama apod. Předpověděla
      i jinou, velmi pozoruhodnou věc: kromě elektronu musí existovat jiná částice se stejnou
      hmotností, ale s opačným nábojem, tzv. \emph{pozitron} a elektron s pozitronem mohou při
      srážce anihilovat, přičemž se vyzáří světlo nebo paprsky gama (což je vlastně totéž, neboť
      světlo i záření gama se liší polohou ve frekvenční škále elektromagnetických vln). Zobecnění
      poznatku, že ke každé částici existuje antičástice, se ukazuje být pravdivým. V případě
      elektronů má antičástice jiné jméno - nazývá se pozitronem, ale u většiny jiných částic
      mluvíme o anti-tom a tom, např. o antiprotonu nebo antineutronu. Do kvantové elektrodynamiky
      se vkládají \emph{dvě čísla} a o většině ostatních čísel ve světě se předpokládá, že jsou
      následkem těchto dvou. Tato dvě vkládaná čísla nazýváme hmotností a nábojem elektronu. Ve
      skutečnosti to však není úplně tak, neboť máme celý soubor chemických čísel, která hovoří o
      tom, jak těžká jsou jádra. To nás přivádí k další kapitole.
    
    \subsection{Atom}\label{fyz:IchapIIsecIVssecIII}
      \luagraphic[0.8]{fyz_fig0895.pdf}{\wikiAtomJadro: Stylizovaný model atomu helia s atomovým
      poloměrem \protect\SI{30}{\protect\pm}. \uv{Mlha} znároňuje  elektronový obal, sestávající z
      orbitalu 1s, přičemž odstín vyjadřuje hustotu pravděpodobnosti výskytu 2 elektronů
      (integrovanou podél přímky pohledu). Oblast atomového jádra, je vyznačena růžově; jeho
      zvětšenina, na které jsou červeně zobrazeny 2 protony a fialově 2 neutrony, je však jen
      schematická. Ve skutečnosti je i jádro helia (a vlnové funkce jednotlivých nukleonů) kulově
      symetrické. Jádro je tedy kladně nabitou částí atomu, která tvoří jeho hmotnostní i prostorové
      centrum (jádro představuje \protect\SI{99.9}{\protect\percent} hmotnosti atomu). Průměr jádra
      činí přibližně \protect\SI{10e-15}{\protect\m}, což je přibližně \(\num{100 000}\times\) méně
      než průměr celého atomu. Existence atomového jádra byla poprvé pozorována v Rutherfordově
      experimentu, na jehož základě vznikl tzv. planetární model atomu.}{fyz:fig0895} 
    %-----------------------------------------------------------------------------------------------
    \subsection{Jádra a Částice}\label{fyz:IchapIIsecIVssecIV}     
      \emph{Z čeho jsou jádra a jak drží pohromadě}? Zjistilo se, že jádra jsou udržována obrovskými
      silami. Při uvolnění těchto sil se uvolňuje energie, která je obrovská v porovnání s chemickou
      energií, tak jak je obrovský výbuch atomové bomby v porovnání s výbuchem trinitrotoluenu. U
      atomové bomby jde totiž o změny uvnitř jádra, zatímco výbuch trinitrotoluenu souvisí se
      změnami elektronového obalu atomů. Proto si klademe otázku: co jsou to za síly, které udržují
      protony a neutrony v jádře pohromadě? Tak, jako je možné elektrické působení přisoudit částici
      - fotonu, předpokládal Yukawa, že i síly mezi neutrony a protony mají svá pole a kmity tohoto
      pole se chovají jako částice. Kromě neutronů a protonů by proto měly existovat jiné částice a
      Yukawa odvodil vlastnosti těchto částic z již známých charakteristik jaderných sil. Například,
      předpověděl, že by měly mít hmotnost dvěstě až třistakrát větší než elektron; a div se světe -
      v kosmickém záření byly objeveny částice s takovouto hmotností! Později se ukázalo, že to
      nebyla ta správná částice. Tuto částici nazvali \(\mu\text{-mezon}\) neboli \emph{mion}.

      Trochu později, v roce 1947 nebo 1948, byla objevena jiná částice, \(\pi\text{-mezon}\) neboli
      \emph{pion}, která vyhovovala Yukawovu kritériu. Abychom získali jaderné síly, musíme k
      protonu a neutronu přidat pion. A teď si řeknete: „Och, jak velkolepé! - pomocí této teorie
      vybudujeme nukleodynamiku, ve které budou mít piony takovou úlohu, jakou jim přisoudil Yukawa
      a všechno bude vysvětleno“. Ta věc má však háček! Ukázalo se, že výpočty v této teorii jsou
      tak složité, že se dodnes nikomu nepodařilo odvodit všechny důsledky této teorie, nebo ji
      porovnat s experimentem; a to se už táhne spoustu let!
      
      Máme tedy teorii, ale nevíme, jestli je správná nebo nesprávná. Víme však už, že je trochu
      chybná, nebo aspoň neúplná. Zatím co jsme marnili čas teorií a snažili se odvodit její
      důsledky, experimentátoři některé věci objevili. Například, objevili \(\mu\text{-mezon}\)
      neboli mion a my ani nevíme, jaká je jeho úloha. V kosmickém záření se našel velký počet
      dalších „přebytečných“ částic. Dnes máme přibližně třista takových částic a je velmi těžké
      porozumět vztahům mezi těmito částicemi a pochopit, na co je příroda potřebuje, nebo která z
      nich na které závisí. Dnes tyto různé částice nechápeme jako různé aspekty téže věci a
      skutečnost, že máme tak mnoho nesouvisejících částic, je odrazem toho, že máme tak mnoho
      nesouvisejících informací bez dobré teorie. Po ohromném úspěchu kvantové elektrodynamiky máme
      jisté znalosti z jaderné fyziky, ale jen hrubé znalosti, částečně experimentální a částečně
      teoretické. Vycházíme přitom z charakteru sil působících mezi protony a neutrony a sledujeme,
      co z toho vyplyne, ale v podstatě nechápeme, odkud ty síly pocházejí. Kromě toho nebylo
      dosaženo téměř žádného pokroku. Objevili jsme velký počet chemických prvků. Mezi těmito prvky
      se najednou objevila souvislost, neočekávaná souvislost zakotvená v Mendělejevově periodické
      tabulce prvků. Například, sodík a draslík jsou téměř shodné ve svých chemických vlastnostech a
      v Mendělejevově tabulce se nacházejí ve stejném sloupci. Hledala se tabulka Mendělejevova typu
      pro nové částice. Taková tabulka nových částic byla sestavena nezávisle Gell-Mannem v USA a
      Nishijimou v Japonsku. Základem jejich klasifikace je nové číslo, jež je možno, podobně jako
      elektrický náboj, přiřadit každé částici a které se nazývá její „podivností“ S (od anglického
      slova strangeness). Toto číslo se, podobně jako elektrický náboj, zachovává v reakcích
      vyvolávaných jadernými silami.  
    
    \subsection{Kosmonautika}\label{fyz:IchapIIsecIVssecV} 
    
    \luagraphicx[1]{fyz_fig0955.jpg}{Mezinárodní vesmírná stanice (anglicky International Space
      Station, rusky \foreignlanguage{russian}{Междунаро́дная косми́ческая ста́нция}, MKC), známější
      pod zkratkou ISS, je v současné době jediná trvale obydlená vesmírná stanice. První díl
      stanice, modul Zarja, byl vynesen na oběžnou dráhu 20. listopadu 1998. Od 2. listopadu 2000,
      kdy na stanici vstoupila první stálá posádka, je trvale obydlena. V současné době je posádka,
      která se každých 6 měsíců obměňuje, tvořena 7 členy. Stanice je umístěna na nízké oběžné dráze
      Země ve výšce kolem 400 km. Při průměrné rychlosti okolo
      \protect\SI{7700}{\protect\m\protect\per\protect\s}
      (\protect\SI{27720}{\protect\km\protect\per\protect\hour}) pak pravidelně obíhá Zemi s
      periodou cca 92 minut. Na fotografii je zachycena stanice z odlétajícího raketoplánu, dne 17.
      dubna 2010. Kredit: Wikipedia}{fyz:fig0955}
      
      
      \subsubsection{Pohled na Zemi z vesmíru}\label{fyz:IchapIIsecIVssecIVsssecI}  
      \subsubsection{Projekt Apollo}\label{fyz:IchapIIsecIVssecIVsssecII}    
%---------------------------------------------------------------------------------------------------
  \section{Integrační tendence ve fyzice}\label{fyz:IchapIIsecV}
    Není to tak dávno, co se fyzikové dělili na dvě velké skupiny - experimentátory a teoretiky.
    Příslušník každé skupiny věděl, že se bez členů druhé skupiny neobejde. Výsledkem byla plodná
    spolupráce plná zdánlivé řevnivosti a úsměvných historek. S nástupem výpočetní techniky se vše
    změnilo. Postupně vznikala skupina třetí, která se zabývá numerickými simulacemi. Bez nich si
    dnes fyziku nedovedeme představit. Numerické simulace umožňují první ověření výsledků nových
    teorií bez nákladných experimentů. Při zpracování experimentálních dat pomáhají hledat procesy,
    které se za naměřenými údaji skrývají. V současnosti má fyzika tři nedílné celky: teorii,
    experiment a numerické simulace. 
    
    Fyzika zaznamenává v průběhu staletí dvě základní tendence. První z nich je postupné členění na
    další a další podobory. Tento vývoj souvisí s prohlubujícím se poznáním a je přirozenou cestou v
    každé vědní disciplíně. Postupně vznikají specialisté na stále užší a užší obory, vytvářejí si
    svůj vlastní vědecký jazyk a schopnost komunikace odborníků z dříve blízkých oblastí fyziky se
    stále zhoršuje. Na druhé straně dochází k hlubšímu pochopení souvislostí mezi jednotlivými
    částmi fyziky a k jejich postupnému sjednocování do univerzálnějších teorií. Možná se jednou
    podaří sjednotit fyzikální pohled na všechny základní přírodní interakce do jedné jediné teorie,
    kterou dnes nazýváme Teorie všeho (anglicky TOE, Theory Of Everything). Tyto integrační tendence
    ve fyzice jsou znázorněny na obrázku 1. 

    \luagraphic[1]{fyz_fig0924.pdf}{Integrační tendence ve fyzice.
      \cite[s.~12]{Kulhanek2019}}{fyz:fig0924}
    
    Mechanika jakožto vědecká fyzikální disciplína vznikala od 17. století. První známější vědecké
    experimenty prováděl \textsc{Galileo Galilei} (1564-1642). Teoretickou konstrukci klasické
    mechaniky, jakožto nástroje pro předpověď pohybu těles v daném silovém poli, navrhnul
    \textsc{Isaac Newton} (1642-1727) ve svých \emph{Principiích (Philosophiæ Naturalis Principia
    Mathematica)} z roku 1687. V 18. století dovršil konstrukci klasické mechaniky \textsc{Joseph
    Louis Lagrange} (1736-1813), který mechanické úlohy formuloval nezávisle na volbě souřadnicové
    soustavy za pomoci variačního počtu.
    
    V 19. století se úspěšně dařilo poznávat a postupně chápat elektrické a magnetické děje. Na
    experimentech se podílela celá řada významných fyziků, například \textsc{Hans Oersted}
    (1777-1881), \textsc{André Ampère} (1775-1836), \textsc{ichael Faraday} (1791-1867),
    \textsc{Heinrich Hertz} (1857-1894), \textsc{Oliver Heaviside} (1850-1925) a další. Celé toto
    údobí vyvrcholilo poznáním, že jevy elektrické a magnetické mají shodnou povahu a společný
    původ. V roce 1873 publikoval \textsc{James Clerk Maxwell} (1831-1879) pojednání \uv{A Treatise
    on Electricity and Magnetism}, které obsahovalo rovnice, jež završily klasickou elektrodynamiku
    do jednoho jediného celku obsahujícího jak děje elektrické, tak magnetické. 
    
    Na konci 19. století podlehlo mnoho fyziků iluzi, že fyzika jako věda je dokončena. Byly známy
    zákony mechaniky na jedné straně a zákony elektřiny a magnetizmu na straně druhé. Na první
    pohled se zdálo, že veškeré přírodní děje jsou důsledkem těchto dvou vědních disciplin a
    budoucnost fyziky je pouze v aplikaci známých zákonů na neznámé situace. Šlo samozřejmě o krutý
    omyl, který se rychle projevil na počátku dvacátého století, kdy nebylo možné tehdejšími
    znalostmi vysvětlit řadu fyzikálních dějů. 

    Ukázalo se, že jak klasická mechanika, tak klasická elektrodynamika nedokáží uspokojivě popsat
    svět na úrovni atomů. Důsledkem toho byla neschopnost objasnit chování elektronu v atomárním
    obalu, vysvětlit záření absolutně černého tělesa, pochopit fotoelektrický jev a smířit se s
    projevy objektů mikrosvěta, které vykazovaly někdy částicové a jindy vlnové vlastnosti. Zrodila
    se kvantová mechanika, ve které neplatí \(ab = ba\), a nekomutativnost se stala nově objeveným
    rysem přírody na mikroskopické úrovni. Kvantová mechanika s sebou přinesla celou řadu těžko
    představitelných jevů - kvantování energie a momentu hybnosti, dualismus vln a částic, relace
    neurčitosti, nejednoznačnost aktu měření a pravděpodobnostní interpretaci výsledků vedoucí na
    nedeterminizmus kvantové fyziky.

    A to byl teprve začátek. Spin elementárních částic objevený v roce 1925 znamenal další výrazný
    posun lidstva v chápání přírody. Je důsledkem relativistické fyziky, která se od počátku 20.
    století rozvíjela paralelně s kvantovou mechanikou. Spojení kvantové mechaniky se speciální
    relativitou vedlo na Diracovu rovnici, která se stala základem kvantového popisu pohybu
    elektronu. \textsc{Paul Adrien Maurice Dirac} (1902-1984) navrhnul svou rovnici v roce 1928 a
    téhož roku z ní odvodil existenci pozitronu, antičástice k elektronu. Pozitron byl
    experimentálně objeven až o 4 roky později \textsc{Carlem Andersonem} (1905-1991). Za svou práci
    získal Dirac Nobelovu cenu za fyziku pro rok 1933. V letech 1946 až 1949 byla dokončena první
    kvantově polní teorie - \emph{kvantová teorie elektromagnetického pole}, které dnes říkáme
    \textbf{kvantová elektrodynamika} (\emph{QED, Quantum Electro-Dynamics}). Za její formulaci
    získali Nobelovu cenu za fyziku pro rok 1965 \textsc{Richard Feynman} (1918-1988),
    \textsc{Shin-Itiro Tomonaga} (1906-1979) a \textsc{Julian Schwinger} (1918-1994). Kvantová
    elektrodynamika je kvantovou analogií Maxwellových rovnic. Elektromagnetická interakce je
    způsobena polními částicemi, v tomto případě fotony, které si mezi sebou posílají nabité
    částice. Klasický pojem síly ztrácí svůj smysl. Feynmanovi se podařilo složité rovnice
    interpretovat za pomoci názorných grafů, kterým dnes říkáme \emph{Feynmanovy diagramy}. Na
    obdobném základě byla později vytvořena také současná \textbf{kvantová teorie slabé a silné
    interakce}. Základním rysem těchto teorií jsou tzv. \emph{kalibrační symetrie}, které předurčují
    způsob působení dané interakce na elementární částice.

    Od počátku 60. let probíhaly snahy o spojení elektromagnetické a slabé interakce do jednoho
    jediného celku. Podařilo se to \textsc{Stevenu Weinbergovi} (1933), \textsc{Abdusu Salamovi}
    (1926-1996) a \textsc{Sheldonu Glashowovi} (1932). Za svou práci získali Nobelovu cenu za fyziku
    pro rok 1979. Jimi předpovězené polní částice slabé interakce \(W^+\), \(W^-\) a \(Z^0\) byly
    objeveny na přelomu let 1983 a 1984 v evropském středisku jaderného výzkumu CERN. Jejich
    objevitelé, \textsc{Carlo Rubbia} (1934) a \textsc{Simon van der Meer} (1925-2011) získali
    Nobelovu cenu ještě téhož roku (1984).

    K pochopení silné interakce přispěl již ve 30. letech japonský fyzik \textsc{Hideki Yuakawa}
    (1907-1981). Za svou práci získal Nobelovu cenu za fyziku pro rok 1949. Současná kvantově polní
    teorie silné interakce se nazývá \textbf{kvantová chromodynamika} (\emph{QCD, Quantum
    Chromo-Dynamics}) a za její formulaci a zejména za objev asymptotické volnosti silné interakce
    kvarků a gluonů získali Nobelovu cenu za fyziku pro rok 2004 \textsc{Frank Wilczek} (1951),
    \textsc{David Gross} (1941) a \textsc{David Politzer} (1949).
    
    Kvantová mechanika slavila v průběhu 20. století mimořádné úspěchy. Jednoduchá teorie popisující
    mechanické děje postupně přerostla v polní kvantovou teorii schopnou úspěšně popsat hned tři ze
    čtyř základních přírodních interakcí. Tato cesta se samozřejmě neobešla bez potíží a problémů,
    nicméně vyústila v dnešní \textbf{standardní model elementárních částic a interakcí}. Bez
    kvantové teorie a hlubokého pochopení zákonitostí mikrosvěta bychom dnes neměli ani počítače ani
    jinou elektroniku. 

    Na počátku 20. století ale vznikala ještě jedna, neméně úspěšná teorie - obecná relativita. Z
    Maxwellovy elektrodynamiky plynulo, že rychlost světla by ve vakuu měla být univerzální
    konstantou a že by se neměla sčítat s rychlostí zdroje elektromagnetického vlnění. Tento
    výsledek byl na první pohled v rozporu s klasickou mechanikou, ve které se rychlost zdroje s
    rychlostí signálu sčítá. Řada experimentů potvrdila správnost elektrodynamiky. Bylo tedy třeba
    přeformulovat mechaniku tak, aby byla v souladu s Maxwellovou elektrodynamikou. To se v roce
    1905 podařilo Albertu Einsteinovi v rámci tzv. \textbf{speciální teorie relativity}. Daň za
    sjednocení obou teorií byla veliká. Čas spolu s prostorem přestaly být absolutní. Délka letící
    tyče a časový úsek mezi dvěma událostmi ve skutečnosti závisejí na volbě souřadnicové soustavy
    pozorovatele.
    
    Einsteinovy snahy o zobecnění speciální relativity na neinerciální souřadnicové soustavy vedly v
    roce 1915 ke vzniku obecné relativity - zcela nové teorie gravitace, která popisuje tuto
    interakci za pomoci zakřiveného času a prostoru. Za základ nové teorie lze chápat dvě myšlenky:
    \begin{itemize}[noitemsep]
      \item každé těleso svou přítomností zakřivuje časoprostor kolem sebe;
      \item každé těleso se v tomto zakřiveném časoprostoru pohybuje po nejrovnějších možných
            drahách - tzv. \emph{geodetikách}.
    \end{itemize}

    Nové chápání času a prostoru bylo zcela revoluční. Samotná tělesa se podílejí na vytváření času
    a prostoru, bez nich by čas a prostor neexistoval. Otázka, jak by vypadal vesmír bez přítomnosti
    těles, přestává mít smysl.
    
    Fyzika dvacátého století se tak stala v jistém smyslu poněkud schizofrenní. Tři ze čtyř
    interakcí jsou popsány za pomoci výměnných (polních) částic v rámci kvantové teorie pole. A
    jedna interakce, gravitační, je popsána za pomoci pokřiveného světa obecné teorie relativity.
    Vyřešení mnoha fyzikálních hádanek s sebou přineslo ještě větší záhady. Existuje jednotná teorie
    všech čtyř interakcí? Je možné spojit kvantovou teorii a obecnou relativitu do jedné jediné
    teorie? Odpověď na tyto otázky zatím neznáme. Velké úspěchy slaví různé strunové teorie, ve
    kterých jsou částice chápány jako jednorozměrné kmitající útvary ve vícerozměrném světě, ale zda
    jde o krok správným směrem či nikoli, není v tuto chvíli jasné. V roce 2010 se objevila hypotéza
    holandského fyzika Erika Verlindeho, podle které by gravitace nemusela být skutečnou silou, ale
    jen statistickým projevem růstu entropie v mikrosvětě. Těžko odhadnout, zda tato odvážná
    myšlenka najde podporu v dalších experimentech, nebo jde o slepou uličku.
    
    Pokud vás zajímají základní vlastnosti přírody a jejich teoretický popis, je třeba v první řadě
    začít se studiem klasické mechaniky, na kterou úzce navazuje mechanika kvantová. Další studium
    polních problémů zase není možné bez znalosti statistické fyziky \cite[s.~14]{Kulhanek2019}. 