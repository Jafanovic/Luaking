% !TeX spellcheck = cs_CZ
%---------------------------------------------------------------------------------------------------
% file kul1ch03.tex
%---------------------------------------------------------------------------------------------------
% =============================== Kapitola: Nádherná teorie ========================================
\setchaptertoc
\chapter{Nádherná teorie - Sto let obecné teorie relativity}\label{feyIchIII}
\section{Prolog}\label{feyIchIIIsecI}
  \begin{table}[]
    \centering
    \begin{tabular}{ll}
      \multicolumn{2}{l}{\cellcolor[HTML]{C0C0C0}Literatura}          \\ \hline
        & Nádherná teorie - Sto let obecné teorie relativity          \\
        & autor                                                       \\
      \multirow{-3}{*}{Obrázek}           
        & link            
    \end{tabular}
  \end{table}
  Vystoupení \textsc{Arthura Eddingtona} (1882 – 1944) na společném zasedání Královské společnosti a
  Astronomické královské společnosti dne 6. listopadu 1919 tiše skoncovalo s vládnoucím paradigmatem
  fyziky gravitace. V monotónně pronášené slavnostní řeči líčil svou výpravu k malému ostrůvku
  Príncipe u západního pobřeží Afriky, na kterém instaloval teleskop a fotografoval úplné zatmění
  Slunce, přičemž se především snažil zachytit málo zřetelný oblak hvězd rozprostřených kolem
  Slunce. Měřením jejich polohy prokázal, že teorie gravitace objevená patronem britské vědy Isaakem
  Newtonem, která byla pokládaná za pravdivou po dvě stě let, není správná. Místo ní, prohlásil
  Eddington, je třeba přijmout novou správnou teorii navrženou Albertem Einsteinem, známou pod
  jménem „obecná teorie relativity“. 

  V té době už byla tato teorie známá jak pro svou schopnost vyrovnat se s popisem vesmíru jako
  celku, tak pro svou neobyčejnou komplikovanost. Po skončení slavnostní přednášky posluchači vyšli
  ven do londýnského podvečera. K Eddingtonovi se přitočil polský fyzik \textsc{Ludwik Silberstein}
  (1872 – 1948), který už napsal knihu o Einsteinově méně obecné teorii zvané „speciální teorie
  relativity“, a sledoval proto Eddingtonovu přednášku s velikým zájmem. Řekl: \emph{„Profesore
  Eddingtone, vy musíte být jedním z těch tří lidí na světě, kteří rozumějí obecné teorii
  relativity.“ Když Eddington váhal s odpovědí, dodal: „Nebuďte tak skromný, pane profesore!“
  Eddington se na něj zahleděl a řekl: „Právě naopak, nejsem vůbec skromný, snažím se jen přijít na
  to, kdo je ten třetí.“} 

  Tato kapitola je takovou biografií obecné teorie relativity. Einsteinova myšlenka, že prostor a
  čas jsou navzájem provázány, žila ve dvacátém století svým vlastním životem a byla zdrojem
  potěšení i frustrace těch nejskvělejších mozků. Obecná relativita je teorie, která neustále
  přináší údiv nad svými důsledky a nezvyklé nové pohledy na svět přírody, s nimiž měl problémy i
  sám Albert Einstein. A jak teorie putovala od jednoho učence ke druhému, přicházely stále nové a
  nečekané objevy s těmi nejpodivuhodnějšími výsledky. Představa černých děr byla počata na
  bojištích první světové války a své zralosti dosáhla v rukou otců americké a sovětské atomové
  bomby. Rozpínání vesmíru bylo poprvé předpovězeno belgickým knězem a ruským matematikem a
  meteorologem. Nové podivné astronomické objekty, pro které bylo potřeba vzít v úvahu obecnou
  relativitu, byly objeveny náhodou. \textsc{Jocelyn Bellová} v Cambridge objevila\footnote{Viz
  sekce \ref{fyz:IchapIIsecIVssecIsssecVIII} v kapitole \ref{fyz:IchapII}} neutronové hvězdy pomocí
  drátěného pletiva nataženého na vratké konstrukci ze dřeva a hřebíků. 

  \luagraphicx[1]{fyz_fig0956.png}{Princeton University Press představila Digital Einstein Papers,
    web s otevřeným přístupem pro \ColPapEinsteinWeb, pokračující publikaci Einsteinova rozsáhlého
    písemného odkazu obsahujícího více než 30 000 jedinečných dokumentů. Stránka představuje 15
    svazků, které dosud vydali redaktoři projektu Einstein Papers Project a které pokrývají spisy a
    korespondenci Alberta Einsteina (1879-1955) od jeho mládí do roku 1927. Mnoho dobových
    fotografií lze nalézt na \ImageEinsteinWeb spravovaný Americkým fyzikálním
    institutem.}{fyz:fig0956}

  Obecná teorie relativity byla též předmětem největších intelektuálních bitev dvacátého století. V
  hitlerovském Německu byla cílem perzekucí, byla napadána ve stalinském Sovětském svazu a v
  padesátých letech dvacátého století se setkávala s nepřízní i ve Spojených státech. Stavěla proti
  sobě v boji o formu konečné teorie vesmíru ta největší jména. Vystupovala ve vážné rozepři o to,
  zda vesmír vznikl velkým třeskem, nebo zda je věčný, a také v diskusi o tom, co je skutečně
  fundamentální strukturou prostoru a času. Sváděla též dohromady velmi různá společenství.
  Uprostřed studené války spojili své síly britští, sovětští a američtí fyzikové, aby vyřešili
  problém původu černých děr. 

  Příběh obecné relativity se však netýká jen minulosti. Během posledních deseti let se přesvědčivě
  ukázalo, že platí-li obecná relativita, je většina vesmíru temná, že vesmír je naplněn látkou,
  která nejen že nevyzařuje světlo, ale ani ho nepohlcuje a neodráží. Svědčí o tom drtivé empirické
  důkazy. Téměř třetina hmoty ve vesmíru je tvořena takzvanou temnou hmotou, těžkou neviditelnou
  látkou, která obklopuje galaxie jako roj rozzlobených včel. Zbývající dvě třetiny tvoří takzvaná
  temná energie, která se snaží prostor roztáhnout. A pouhá čtyři procenta jsou z hmoty, kterou
  běžně známe - z atomů různých prvků. Z této hmoty jsme i my, což nám dává ve vesmíru nedůležitou
  pozici. 

  To je ovšem pravda v případě, že obecná teorie relativity je skutečně správná za všech podmínek.
  Není vyloučeno, že jsme se ocitli na samé hranici platnosti obecné relativity a ani ona není tou
  úplnou teorií gravitace. 

  Einsteinova teorie je důležitá i pro novou fundamentální teorii přírody, kvůli které planou mezi
  teoretickými fyziky divoké vášně. Teorie strun, která se snaží jít za Newtona i Einsteina a chce
  všechno v přírodě sjednotit, spočívá na složité struktuře prostoročasu ve vyšších dimenzích s
  podivnými geometrickými vlastnostmi. Je mnohem ezoteričtější, než kdy byla Einsteinova teorie, a
  některými vědci je oslavována jako ta pravá konečná teorie všeho, zatímco jiní ji prohlašují za
  romantickou fantastiku, dokonce ani ne vědeckou. Teorie superstrun je odpadlické náboženství,
  které se odštěpilo od pravověrné obecné relativity a nikdy by bez ní nevzniklo. Mnoho
  praktikujících relativistů se však na tuto herezi dívá se skepsí. 

  Temná hmota, temná energie, černé díry, jež dnes ovládají astrofyziku, i teorie strun - to všechno
  jsou potomci obecné teo- rie relativity. Při přednáškách na různých univerzitách, při diskusích na
  různých seminářích i na zasedáních Evropské kosmické agentury, která zodpovídá za činnost řady
  světově důležitých vědeckých satelitů, jsem si uvědomil, že jsme uprostřed důležité přeměny
  moderní fyziky. Máme dnes řadu talentovaných mladých vědců, kteří se dívají na obecnou relativitu
  pohledem založeným na zkušenosti vybudované ve století géniů. Dolují z Einsteinovy teorie poklady
  v ní ukryté pomocí bezprecedentní výpočetní techniky a zkoumají alternativní teorie, jež by ji
  mohly sesadit z trůnu. Ve vesmíru hledají exotické objekty, jejichž pozorování by mohlo potvrdit
  nebo vyvrátit základní principy obecné teorie relativity. Stále větší společenství vědců se
  spojuje při konstrukci kolosálních přístrojů, které nám dovolují pohlédnout do vesmíru dále, než
  jsme kdy byli schopni, nebo družic, jež mohou potvrdit či vyvrátit ty nejpodivnější předpovědi
  obecné teorie relativity. 

  Příběh obecné relativity je úžasný a poučný a stojí za vyprávění. Dnes, kdy už jsme doopravdy
  vstoupili do jednadvacátého století, stojí před námi řada důležitých objevů a trápí nás spousta
  nezodpovězených otázek. V příštích několika letech se jistě objeví něco zásadního a až to přijde,
  musíme být připraveni tomu porozumět. Domnívám se, že zatímco dvacáté století bylo érou kvantové
  fyziky, ve století jednadvacátém dojde k plnému uplatnění obecné teorie relativity.

\section{Jaké to je volně padat}\label{feyIchIIIsecII}
  Během podzimu 1907 pracoval Einstein pod značným tlakem. Byl požádán o přehledový článek o konečné
  formě své speciální teorie relativity pro Jahrbuch der Radioaktivität und Elektronik (Ročenka pro
  radioaktivitu a elektroniku). Shrnout tak fundamentální výsledky v krátkém přehledu byl těžký úkol
  a on se mu mohl věnovat jen ve svém volném čase. Od osmi ráno do šesti večer od pondělka do soboty
  jej totiž bylo možno najít na bernském Federálním patentovém úřadě v nové Poštovní a telegrafní
  budově, kde byl zaměstnán. Zde pečlivě studoval různá novátorská elektrická „udělátka“ a zkoumal,
  zda nejde o nesmysly. Jeho šéf mu dal radu: „Když se chopíte žádosti, přijměte zásadu, že všechno,
  co vynálezce tvrdí, je špatně.“ Einstein si to vzal k srdci. Po většinu dne musel nechávat své
  vlastní teorie a objevy na pokoji. Své vlastní výpočty měl v šuplíku, kterému říkal „mé oddělení
  teoretické fyziky“. 

  Einsteinův přehledový článek stručně shrnoval podstatu triumfálního sňatku staré mechaniky Galilea
  Galileiho a Isaaka Newtona s novým elektromagnetismem Michaela Faradaye a Jamese Clerka Maxwella.
  Vysvětloval některé podivnosti, jež Einstein předtím objevil, například, že hodiny jdou pomaleji,
  když se pohybují a že předměty se zkracují ve směru svého pohybu. V článku byl zmíněn i magický
  vztah mezi hmotností a energií i skutečnost, že nic se nemůže pohybovat rychleji než světlo.
  Rozbor principu relativity měl osvětlit, jak skoro celé fyzice vládne jednotný systém zákonů.  

  V roce 1905 napsal Einstein během pár měsíců několik zásadních článků, které podstatně změnily
  fyziku. V tomto výbuchu inspirace odhalil i to, že světlo se chová jako soubor nedělitelných
  balíčků energie, podobných v určitém ohledu částečkám hmoty. Dále ukázal, že nervózní chaotické
  chování částeček pylu či prachu v kapalině je důsledkem nárazů chaoticky poskakujících molekul
  kapaliny, které vibrují a odrážejí se jedna od druhé. A zabýval se problémem, který trápil fyziky
  téměř půl století: proč se zdá, že fyzikální zákony vypadají různě podle toho, ve kterém vztažném
  systému je sledujete. Spojil je dohromady právě svým principem relativity. 

  Všechny tyto výsledky představovaly ohromující objevy, které Einstein učinil jako nízko postavený
  úředník na patentovém úřadě v Bernu. Byl tam i v roce 1907 a probíral se vynálezy, které se tam
  dostávaly s žádosti o patent. Stále se mu nedařilo posunout se do důstojného akademického světa,
  který mu nebyl, jak se zdálo, příliš nakloněn. Einstein totiž během svého studia vůbec nevypadal
  jako někdo, kdo přepíše základy fyziky. Studoval na curyšské polytechnice (Schule für Fachlehrer
  des Polytechnikums Zürich) a nepůsobil nijak výrazně, navíc vynechával přednášky, které ho
  nebavily, a dělal si nepřátelé z lidí, kteří mohli rozvoji jeho génia pomoci. Jeden z profesorů mu
  řekl: „Vy jste velmi chytrý mladík … Ale máte jednu ohromnou chybu, nikdy si nenecháte nic říct.
  “Když mu jeho školitel nedovolil pracovat na tématu podle vlastního výběru, Einstein odevzdal
  nevýraznou závěrečnou esej a tím snížil své hodnocení natolik, že nemohl získat místo asistenta na
  žádné z univerzit, na které podal přihlášku. 

  Od absolutoria v roce 1900 až do roku 1902, kdy zakotvil na patentovém úřadě, byl jeho život plný
  neúspěchů. Jeho frustraci zvýšilo, že doktorská práce, kterou předložil v roce 1901 na Curyšské
  univerzitě, byla v následujícím roce zamítnuta. Einstein v ní vyvracel některé myšlenky velkého
  teoretického fyzika Ludwiga Boltzmanna z konce devatenáctého století. Einsteinův ikonoklasmus se
  mu nevyplatil. Doktorát získal až v roce 1905, když jako disertaci podal jeden ze svých kouzelných
  článků „Nové určení rozměrů molekul“. Doktorský titul „podstatně ulehčil vztah s lidmi“, jak
  konstatoval Einstein, který se přece jen naučil určité diplomacii. 

  Zatímco Einstein bojoval, jeho přítel Marcel Grossmann byl na cestě stát se důstojným profesorem.
  Metodický, pilný student, oblíbený u svých profesorů, takový byl Grossmann, jehož detailní pečlivě
  vedené poznámky z přednášek zachraňovaly Einsteina se špatnou docházkou na přednášky. Během
  společných studií v Curychu se Grossmann stal blízkým přítelem Einsteina a jeho pozdější ženy
  Milevy Marićové; všichni tři absolvovali ve stejném roce. Na rozdíl od Einsteina se další
  Grossmannova kariéra vyvíjela hladce. Získal místo asistenta v Curychu a v roce 1902 i doktorát.
  Po krátkém období, kdy učil na středních školách, se stal profesorem deskriptivní geometrie na
  ETH, Eidgenössische Technische Hochschule v Curychu, zatímco Einsteinovi se nepodařilo získat ani
  místo středoškolského učitele. Jen díky doporučení Grossmannova otce, jenž využil své známosti s
  šéfem patentového úřadu v Bernu, získal Einstein zaměstnání alespoň jako patentový expert. 

  Pro Einsteina však bylo i toto místo požehnáním. Po letech finanční nestability a závislosti na
  otcových příspěvcích si mohl dovolit oženit se s Milevou a založit v Bernu rodinu. Relativní
  poklid na patentovém úřadu s jasně definovanými povinnostmi a monotónní prací bez vyrušování mu
  dával možnost promýšlet si věci opravdu do hloubky. Práce, za kterou byl placen, mu zabírala každý
  den jen několik hodin a to mu poskytovalo čas, aby se mohl soustředit na své problémy. Seděl za
  malým dřevěným stolem jen s několika knihami a poznámkami ze svého „oddělení teoretické fyziky“ a
  prováděl pokusy v mysli. V těchto myšlenkových experimentech (německý termín Gedankenexperimente
  se dnes užívá i mezinárodně) si představoval různé konstrukce a situace, ve kterých studoval
  důsledky fyzikálních zákonů a přemýšlel o jejich vztahu k reálnému světu. Bez skutečné laboratoře
  si přehrával v hlavě hru s přesně stanovenými pravidly a uvědomoval si ty okamžiky, na které se
  musí soustředit důkladněji. Einstein znal právě tolik matematiky, aby své myšlenky mohl zapsat a
  vytvořit skvělé klenoty, jež posléze zásadně změnily směr fyziky. 

  Jeho nadřízení v patentovém úřadě byli přitom s jeho prací spokojeni a povýšili ho na experta II.
  třídy, nevšímali si však jeho rostoucí reputace. V roce 1907, kdy ho německý fyzik Johannes Stark
  požádal o sepsání přehledného článku s názvem „O principu relativity a jeho důsledcích“, Einstein
  stále zpracovával svou denní dávku patentů. Na jeho napsání dostal dva měsíce a během práce na něm
  si uvědomil, že princip relativity je neúplný. Měl-li být opravdu obecný, potřeboval kompletně
  přepracovat. 

  Článek v Jahrbuch byl shrnutím důsledků původního Einsteinova principu relativity. Tento princip
  říká, že přírodní zákony musí vypadat stejně ve všech inerciálních vztažných soustavách. Základní
  myšlenka tohoto principu nebyla nová a byla vlastně ve fyzice přítomná několik století.

  Fyzikální zákony mechaniky jsou pravidla, podle nichž se objekty pohybují. Určují, jak jsou tělesa
  urychlována či zpomalována pod vlivem sil. V sedmnáctém století předložil anglický fyzik a
  matematik Isaac Newton soubor zákonů, jež určovaly, jak objekty reagují na mechanické vnější síly.
  Tyto pohybové zákony konzistentně vysvětlují, co se stane, když se srazí dvě kulečníkové koule,
  jak se pohybuje náboj vystřelený z děla či míč vržený do vzduchu. 

  Inerciální vztažný systém je takový, vzhledem ke kterému se všechny hmotné body, na něž nepůsobí
  žádné síly, pohybují rovnoměrně přímočaře. Takový systém není jediný - všechny systémy, jež se
  pohybují rovnoměrně přímočaře vzhledem k jednomu inerciálnímu systému, jsou také inerciální.
  Čtete-li tyto řádky v pohodlném křesle ve vašem pokoji nebo v kavárně, jste se značnou přesností v
  inerciálním systému. Jiným klasickým příkladem je rychle se pohybující vlak se zacloněnými okny.
  Jedete-li v něm, tak potom, co se urychlil na určitou rychlost a tou pak jede, nemáte žádnou
  možnost zjistit, že se pohybujete. Principálně není možné zjistit rozdíl mezi systémem, který je
  pevně spojen se Zemí, a systémem, který se vzhledem k ní pohybuje rovnoměrně, byť i velikou
  rychlostí. Experimentujete-li v jednom inerciálním systému a necháváte na objekty působit různé
  síly, dostanete stejné výsledky, jako když tytéž pokusy provádíte v jiném inerciálním systému.
  Zákony přírody jsou totožné ve všech inerciálních systémech. 

  Devatenácté století přineslo zcela novou sadu zákonů, které vzájemně propojily dvě fundamentální
  síly - elektřinu a magnetismus. Na první pohled se elektřina a magnetismus jeví jako úplně
  rozdílné jevy. Elektřina produkuje světlo v našich domovech a je zodpovědná za blesky, magnetická
  síla drží magnetické příchytky na našich ledničkách a stáčí magnetickou střelku kompasu k severu.
  Skotský fyzik James Clerk Maxwell však ukázal, že na obě tyto síly se můžeme dívat jako na různé
  projevy jedné základnější elektromagnetické síly. Jak se tato základní síla projevuje, závisí na
  pohybu pozorovatele. Člověk sedící vedle tyčového magnetu bude pozorovat magnetické působení,
  nikoli elektrické. Ale člověk, který kolem prosviští, zaregistruje kromě magnetismu i působení
  elektrické. Maxwell sloučil tyto dvě síly do jedné, jež je popsána jednotným způsobem bez ohledu
  na polohu či rychlost pozorovatele. 

  Když ale chcete sloučit Newtonovy pohybové zákony s Maxwellovými zákony elektromagnetismu,
  narazíte na problém. Jestliže ve světě platí obě sady těchto zákonů, dá se v principu zkonstruovat
  zařízení z magnetů, drátů a kladek, jež nebude pociťovat žádnou sílu v jednom inerciálním systému,
  ale bude ji cítit v jiných inerciálních systémech a tím narušovat pravidlo, že inerciální systémy
  jsou navzájem nerozlišitelné. Zdá se tedy, že Newtonovy zákony a Maxwellova pravidla pro
  elektromagnetismus jsou navzájem nekonzistentní. Tuto „asymetrii“ v přírodních zákonech chtěl
  Einstein napravit. 

  Do roku 1905 Einstein zformuloval svůj omezený princip relativity, dnes zvaný „speciální“, na
  základě celé řady myšlenkových experimentů, které pro tento účel vymyslel. Jeho duševní kutilství
  kulminovalo ve dvou postulátech. První byl prostě novou formulací principu: Zákony fyziky musí
  vypadat stejně ve všech inerciálních systémech. Ten druhý byl radikálnější: Ve všech inerciálních
  vztažných systémech má rychlost světla ve vakuu vždy tutéž hodnotu 299 792 kilometrů za sekundu.
  Na základě těchto postulátů lze upravit pohybové zákony Newtonovy mechaniky tak, že když se
  skombinují s Maxwellovými zákony elektromagnetismu, inerciální systémy jsou skutečně
  nerozlišitelné. Einsteinův nový princip relativity však vedl k některým ohromujícím důsledkům. 

  Sladění Newtonových a Maxwellových zákonů, aby byl splněn Einsteinův princip relativity,
  vyžadovalo úpravu pohybových zákonů. V klasickém Newtonově vesmíru je rychlost aditivní. Kulka
  vystřelená z lokomotivy ve směru jízdy se pohybuje rychleji, než kulka vystřelená ze stejné
  zbraně, jež je však na kolejích v klidu. Soudili bychom, že pro světlo vrhané reflektorem na
  lokomotivě by tomu mělo být obdobně. V Einsteinově vesmíru to však není pravda - v něm je
  stanovená nejvyšší povolená rychlost na 299 792 kilometrů za sekundu a tuto bariéru nemůže
  překonat ani ta nejvýkonnější raketa. Jenže se děje něco velmi podivného. Například cestující
  sedící ve vlaku, který se pohybuje rychlostí blízkou rychlosti světla, bude stárnout pomaleji, než
  člověk sedící na nádraží a pozorující projíždějící vlak. Vlak sám se bude pozorovatelům stojícím
  kolem trati zdát kratší, než když je v klidu. Čas se zpomaluje a délky se zkracují. Takové podivné
  jevy svědčí o tom, že ve skutečnosti je za tím něco hlubšího - čas a prostor jsou navzájem
  propojené a v určitém smyslu jsou si podobné. 

  Svým principem relativity Einstein fyziku zjednodušil, i když se zdálo, že to má zvláštní
  důsledky. Jenže na podzim roku 1907, když začal psát svůj přehledový článek, Einstein přiznal, že
  i když jeho teorie zdánlivě funguje dobře, není stále úplná. Postulátu relativity nevyhovovala
  Newtonova teorie gravitace.

  Před příchodem Alberta Einsteina měl ve světě fyziky Isaac Newton postavení boha. Newtonovo dílo
  se pokládalo za vrcholný úspěch moderního myšlení. Na konci sedmnáctého století sjednotil
  gravitační sílu působící na těch nejmenších i největších škálách a vystihl ji jednoduchým vztahem,
  který popisoval gravitační účinky jak v kosmu, tak v každodenním životě. 
  
  Newtonův zákon univerzální gravitační přitažlivosti je velice jednoduchý. Říká, že gravitační
  přitažlivost mezi dvěma tělesy je přímo úměrná hmotnostem obou těles a nepřímo úměrná druhé
  mocnině jejich vzdálenosti. Když tedy hmotnost jednoho z těles zdvojnásobíme, přitažlivost mezi
  nimi se také zdvojnásobí, zatímco když zdvojnásobíme jejich vzdálenost, přitažlivost se čtyřikrát
  zmenší. Po dvě století Newtonův zákon skvěle vysvětloval řadu jevů a přesně určoval dráhy známých
  planet. Spektakulárním úspěchem Newtonovy teorie byla předpověď existence planet do té doby
  neznámých. 
  
  Od konce osmnáctého století se hromadily důkazy, že dráha planety Uran vykazuje záhadné
  nepravidelnosti. Pečlivá pozorování astronomům nakonec dovolila určit tuto dráhu s velikou
  přesností. Není vůbec snadná záležitost zjistit, jakou dráhu Uranu Newtonova teorie předpovídá. Je
  totiž třeba brát v potaz nejen přitahování Uranu Sluncem, ale i gravitační účinek ostatních
  planet, sice podstatně menší, ne však bezvýznamný. Působení ostatních planet má za důsledek, že
  přesná dráha Uranu je značně složitá. Astronomové a matematici zveřejňovali tabulky drah planet,
  které určovaly, kde se v určitou dobu ta která planeta nachází. A když pak srovnávali vypočtenou
  dráhu Uranu s údaji získanými pozorováním, byla zde vždy diskrepance, kterou nedovedli vysvětlit.
  
  Francouzský astronom a matematik Urbain Le Verrier byl v počítání planetárních drah mimořádně
  obratný. Když obrátil svou pozornost k Uranu, vycházel z toho, že Newtonova teorie gravitace
  přesně platí - u ostatních planet byla výtečná shoda mezi výpočtem a pozorováním. Řekl si, že
  je-li Newtonova teorie správná, tak poruchy musí působit nějaký neznámý objekt, jehož účinek se do
  výpočtů dosud nezahrnoval. A tak Le Verrier udělal odvážný krok a předpověděl existenci nové,
  fiktivní planety a vypracoval vlastní tabulky, které ji braly v úvahu. K jeho radosti německý
  astronom Gottfried Galle v Berlíně zaměřil teleskop na místo, kde se měla tato planeta podle Le
  Verrierových výpočtů nacházet, a v zorném poli se mu skutečně objevila velká planeta. Jak to Galle
  napsal do dopisu Le Verrierovi: „Monsieur, planeta, jejíž polohu jste naznačil, skutečně existuje.
  “
  
  Le Verrier posunul Newtonovu teorii dále než kdokoli před ním a jeho smělost byla odměněna.
  Neptunu se po několik desetiletí říkalo „Le Verrierova planeta“. Marcel Proust užil ve svém
  \emph{Hledání ztraceného času} Le Verrierův objev jako analogii „vymýcení nepořádků“ a Charles
  Dickens se o něm zmiňuje ve své drsné detektivní povídce Detektivní policie\footnote{Le Verrierův
  objev ovlivnil i zápletku romaneta Jakuba Arbese Etiopská lilie.}. Na výsluní své slávy pak Le
  Verrier obrátil svou pozornost k Merkuru. Zdálo se, že i ten má neočekávaně podivnou dráhu. 
  
  Kdyby se kolem Slunce pohybovala jediná planeta, pak by se podle Newtonovy teorie gravitace
  pohybovala po jednoduché uzavřené dráze ve tvaru elipsy. Planeta by se tak měla pohybovat stále a
  stále, periodicky by putovala od bodu nejbližšího ke Slunci do bodu nejvzdálenějšího a zase zpět.
  Poloha bodu, ve kterém je planeta ke Slunci nejblíže, takzvaného perihelia, by byla v prostoru
  stále táž. Některé planety se pohybují po drahách, jež jsou téměř kruhové - jejich eliptická dráha
  má malou excentricitu. Příkladem takové planety je naše Země. Jiné, jako právě Merkur, mají dráhy
  protáhlejší. 
  
  Skutečné dráhy planet jsou narušeny účinkem planet ostatních, ale od elips se liší jen málo. Le
  Verrier však zjistil, že i když se započítá účinek ostatních planet na Merkur, jeho dráha se stále
  liší od předpovědi na základě Newtonovy teorie. Perihelium Merkurovy dráhy se při každém oběhu
  nepatrně posunuje. Efekt je velice malý, za celé století činí pouhých asi 40 úhlových vteřin.
  (Celý kruh se dělí na 360 stupňů, každý stupeň 25 na 60 úhlových minut a každá minuta na 60
  úhlových vteřin. Protože Merkur oběhne Slunce asi za 88 dní, za století vykoná více než 400
  oběhů.) Tato anomálie, známá jako posun perihelia Merkura, nebyla vysvětlitelná jen účinkem
  pozorovaných planet. Le Verrier byl však přesvědčen, že Newton musí mít pravdu, a tak v roce 1859
  vyslovil hypotézu, že v blízkosti Slunce je ještě jedna planeta, velikostí srovnatelná s Merkurem,
  kterou pokřtil Vulkán. Uvědomoval si ovšem, že je to smělá a podivná hypotéza. Jeho slovy: „Jak
  mohla planeta v těsné blízkosti Slunce, tedy extrémně jasná, uniknout pozorování při úplném
  slunečním zatmění?“
  
  Le Verrierova hypotéza rozpoutala hon na novou planetu Vulkán. Během následujících desetiletí byl
  několikrát objev objektu v blízkosti Slunce ohlášen, ale pečlivé zkoumání pak ukázalo, že jde o
  planý poplach. Pátrání po Vulkánu Le Verrierovou smrtí v roce 1877 neskončilo, v astronomickém
  povědomí se však usazovalo, že čtyřicetivteřinovou anomálii musí vysvětlit něco jiného než
  neviditelná planeta. 
  
  Když se Einstein začal v roce 1907 zamýšlet nad otázkou gravitace, bylo to především proto, že
  musel Newtonovu teorii smířit s principem relativity. V podvědomí si však uvědomoval, že je též
  třeba vysvětlit Merkurovu anomálii. Byl to těžký úkol. 
  
  Newtonovo vysvětlení gravitace narušuje oba postuláty Einsteinovy speciální teorie relativity.
  Podle Newtona jsou gravitační efekty okamžité. Jestliže jsou dvě tělesa náhle umístěna blízko
  sebe, gravitační působení jednoho na druhé se projeví bez zpoždění - gravitace nepotřebuje žádný
  čas k překonání vzdálenosti mezi nimi. Jenže jak to bylo možné, když podle nové Einsteinovy teorie
  se nic, žádný signál, žádný efekt, nemohl pohybovat rychleji než světlo? A stejně významná byla i
  skutečnost, že Einsteinův princip relativity sice nastolil harmonii mezi mechanikou a
  elektromagnetismem, Newtonův gravitační zákon však zůstal stranou. Newtonovská gravitace vypadala
  v různých inerciálních systémech různě. 
  
  První Einsteinův krok na jeho dlouhé cestě k pochopení gravitace a zobecnění teorie relativity se
  uskutečnil jednoho dne, když seděl ve svém křesle v patentovém úřadě v Bernu, zcela ztracený ve
  svém myšlenkovém světě. Po letech Einstein připomínal myšlenku, která ho tehdy napadla a která ho
  vedla dále k jeho teorii gravitace: „Když někdo volně padá, tak nepociťuje svou vlastní váhu.“
  
  Představte si, že jste Alenka v králičí díře a volně padáte, aniž vás něco může zastavit. Protože
  padáte pod vlivem gravitace, rychlost vašeho pádu neustále roste. Zrychlení je přesně rovné
  gravitačnímu tahu, a proto máte pocit, že nemusíte vyvíjet žádné úsilí, abyste se udrželi v
  poloze, kterou zaujímáte. Nepociťujete svou váhu, i když je samozřejmě trochu děsivé se takto
  řítit králičí dírou. Nyní si představte, že spolu s vámi padají i různé předměty: kniha, šálek
  čaje, bílý králík vyděšený stejně jako vy. Protože všechny předměty padají pod vlivem gravitace,
  padají všechny se stejným zrychlením jako vy. Budou tedy stále kolem vás, padají zároveň s vámi.
  Kdybyste chtěli zjistit z jejich pohybu vůči vám, jak je velká gravitační síla, jež vás táhne
  dolů, nemáte šanci. Vy se budete cítit bez tíže, a jakoby bez tíže budou vypadat i okolní objekty.
  To vše naznačuje, že mezi zrychleným pohybem a gravitačním tahem je nějaký důvěrný vztah - v
  popsaném případě jedno kompenzuje druhé. 
  
  Volný pád králičí dírou je ale možná přece jen příliš adrenalinová záležitost. Ve vašem okolí se
  toho děje velmi mnoho. Sviští kolem vás vzduch a představa, že nakonec narazíte na dno, vám
  nedovolí jasně uvažovat. Zkusme něco jednoduššího a trochu poklidnějšího. Představme si, že jste v
  přízemí vysoké budovy nastoupili do výtahu. Výtah se rozjede a během těch prvních pár sekund, kdy
  zrychluje, se cítíte o něco těžší. Vyjeli jste do nejvyššího patra a pak jedete zase dolů. Nyní se
  během doby, kdy výtah nabírá rychlost, cítíte trochu lehčí. Jakmile výtah získá plnou rychlost a
  tou se pak pohybuje, necítíte se ani lehčí, ani těžší, než jste běžně zvyklí. Ale když výtah
  zrychluje nebo zpomaluje, je narušen váš běžný pocit vlastní váhy a máte dojem, že gravitace
  zesílila či zeslábla. Jinými slovy, naše vnímání gravitace je závislé na zrychlení soustavy, ve
  které se nacházíme. 
  
  Toho dne v roce 1907, kdy začal uvažovat o svém volně padajícím člověku, si Einstein také
  uvědomil, že mezi gravitací a zrychlením musí být hluboká podobnost a že tudy vede cesta k
  zabudování gravitace do jeho teorie relativity. Kdyby pozměnil svůj princip relativity tak, že
  podle nové formulace mají přírodní zákony stejný tvar nejen ve všech inerciálních systémech, ale i
  v těch systémech, jež se vůči inerciálním pohybují se zrychlením, snad by se mu podařilo propojit
  gravitaci s mechanikou a elektromagnetismem. Ještě nevěděl jak to udělat, ale tento brilantní
  vhled do fyzikální situace byl počátečním krokem k zobecnění teorie relativity. 
  
  Pod tlakem německých vydavatelů tedy dokončil svůj přehledový článek „O principu relativity a jeho
  důsledcích“. Zahrnul do něj ale i kapitolu, ve které rozebíral, co je třeba udělat, aby princip
  relativity platil i pro gravitaci. Uvedl několik důsledků, mezi nimi to, že přítomnost gravitace
  ovlivní rychlost světla a způsobí i to, že hodiny v gravitačním poli půjdou obecně pomaleji.
  Efekty jeho zobecněného principu relativity by mohly vysvětlit i anomálii v pohybu Merkura. Na
  konci článku poznamenal, že tyto efekty by mohly být užity k testování jeho teorie, mnoho věcí je
  však třeba ještě dopracovat. 
  
  Koncem roku 1907 se již chýlila ke konci Einsteinova „skvělá opuštěnost“. Jeho články z roku 1905
  pomalu ale jistě docházely ohlasu. Začal dostávat řadu dopisů od uznávaných fyziků, kteří ho
  žádali o kopie jeho článků a diskutovali s ním o jeho myšlenkách. Einstein byl tímto vývojem
  nadšen. Svěřil se jednomu příteli: „Moje články se setkávají s velkým uznáním a dávají podněty k
  dalším výzkumům.“ Jeden z jeho obdivovatelů podotkl: „Musím přiznat, že mne ohromilo, když jsem
  četl, že musíte sedět osm hodin denně v kanceláři. Ale historie je plná špatných vtipů!“ Ne, že by
  měl tak špatný život. Jeho zaměstnání v Bernu mu umožnilo založit s Milevou rodinu. V roce 1904 se
  jim narodil syn, kterého pokřtili Hans Albert. Einsteinova pravidelná pracovní doba mu dovolila
  trávit dost času doma a vyrábět podomácku hračky pro malého synka. Ale to už byl na počátku cesty
  do akademického světa. 
  
  V roce 1908 byl jmenován soukromým docentem na bernské univerzitě, což ho opravňovalo přednášet
  pro platící studenty. Učení pro něj bylo velkou zátěží a jako přednášející získal velmi špatnou
  pověst. Přesto se v roce 1909 stal docentem na univerzitě v Curychu. Zůstal tam však pouze něco
  přes rok. V roce 1911 mu byla nabídnuta profesura na německé univerzitě v Praze. Zde ho
  pedagogické povinnosti tolik nezatěžovaly a on se opět cítil tak, jako když pracoval na patentovém
  úřadě a nemusel učit. Opět mohl snít o zobecnění teorie relativity.

\section{Nejcennější objev}\label{feyIchIIIsecIII}
  Albert Einstein se jednou svěřil Ottu Sternovi, svému příteli a kolegovi: „Víte, jakmile začnete
  počítat, dříve než si to uvědomíte, beznadějně se do toho zahrabete.“ Ne, že by neznal matematiku
  poměrně dobře. Ve škole v ní exceloval a ovládal ji jistě natolik, aby uměl své myšlenky
  matematicky vyjádřit. V jeho článcích byla fyzikální argumentace dokonale vyvážená s matematickým
  formalismem, který dával jeho myšlenkám jasnou formu. Ale předpovědi důsledků jeho zobecněné
  teorie byly matematicky nedotažené - jeden z jeho curyšských profesorů popsal jeho argumentaci
  jako „matematicky neobratnou“. Einstein matematikou jakoby trochu pohrdal, říkal, že je to
  „nadbytečné vzdělání“ a vyslovoval poznámky jako: „Od té doby, co se na mou teorii vrhli
  matematici, jí nerozumím ani já sám.“ Ale v roce 1911, když se znovu vrátil k myšlenkám o
  gravitaci, uvědomil si, že potřebuje právě matematiku, aby se pohnul trochu dále. 
  
  Zamyslel se nad svým principem relativity a vrátil se k účinku světla na gravitaci. Představme si,
  že letíme prostorem v kosmické lodi, daleko od všech planet i hvězd. Okénkem na pravoboku vniká do
  lodi světelný paprsek ze vzdálené hvězdy, proletí lodí a opouští ji okénkem na levoboku.
  Představme si dále, že v lodi jsou sedačky jako v letadle. Je-li loď vzhledem ke vzdáleným hvězdám
  v klidu, paprsek ji opouští okénkem ve stejné řadě, jako je to, kterým do kabiny vnikl. Jestliže
  ovšem loď letí velikou rychlostí, paprsek opustí kabinu okénkem o jednu či více řad dále k zádi
  lodi, protože během doby, kterou potřeboval k průletu kabinou, se loď o kus posunula. Paprsek bude
  vnikat do okénka pod určitým úhlem. Když je ale rychlost lodi neměnná, pod stejným úhlem se bude
  šířit kabinou a pod týmž úhlem bude loď opouštět. Pozorovatelům v lodi se paprsek bude stále jevit
  jako přímka. Jiné to je, když se loď během průletu paprsku zrychluje. Paprsek se v tom případě
  dostane až ke vzdálenějšímu okénku a pozorovatelům v kabině se bude jevit zakřivený. 
  
  A zde se projevil Einsteinův úžasný vhled do povahy gravitace. Pozorovatelé v kosmické lodi, která
  se urychluje vzhledem k inerciálnímu systému, budou pociťovat přesně totéž co pozorovatelé v lodi,
  v níž směrem k zádi působí odpovídající gravitační síla. Jak si Einstein uvědomil, na té
  nejjednodušší úrovni se nedá odlišit gravitace od vlivu zrychlení neinerciálního systému.
  Kosmonaut sedící v kosmické lodi, jež přistála na povrchu planety, na kterou díky její gravitaci
  padají předměty s určitým zrychlením, bude pociťovat totéž, co cítí, když loď letí širým prostorem
  daleko od gravitaci budících těles, jestliže loď zapnula tryskový pohon a zrychluje se. A také
  šíření paprsku lodí bude v obojím případě vypadat stejně. Gravitace tedy ohýbá světelný paprsek
  podobně, jako to dělá čočka. 
  
  Aby byl ovšem gravitační ohyb světla pozorovatelný, musí být gravitace opravdu silná - gravitační
  pole na povrchu planety na měřitelný efekt nestačí. Einstein navrhl jednoduchý pozorovací test,
  využívající mnohem masivnějšího objektu, než je planeta. Při něm se měl měřit ohyb světelného
  paprsku, který se šíří od zdroje za Sluncem těsně kolem jeho okraje. Úhlová pozice vzdálených
  hvězd by se měla změnit o nepatrnou hodnotu, o necelou jednu úhlovou vteřinu, jestliže paprsek od
  hvězdy, která ho vyslala, běží těsně kolem slunečního disku. To je hodnota na hranici měřitelnosti
  tehdejšími přístroji. Pozorování by se muselo uskutečnit při úplném zatmění Slunce, protože jinak
  Slunce hvězdy ve svém okolí přezařuje, takže není šance je pozorovat. 
  
  Einstein sice vymyslel způsob, jak testovat platnost svých nových myšlenek, na cestě ke své nové
  teorii však zatím nedosáhl žádného podstatného pokroku. Stále ale promýšlel tutéž představu
  člověka padajícího volným pádem, která ho napadla ještě na patentovém úřadě. I když ho netížily
  pedagogické povinnosti a mohl většinu času věnovat myšlenkovým experimentům a přemítání o nové
  teorii, nebyl ve svém pražském působišti šťastný. Jeho rodina se rozrostla - ještě před příchodem
  do Prahy se narodil syn Eduard. Mileva se však v Praze cítila nepříjemně a opuštěně, daleko od
  světa, kterému přivykla v Bernu a pak Curychu. A tak se Einstein v roce 1912 chopil příležitosti
  vrátit se do Curychu, tentokrát však na místo profesora na ETH. 
  
  Během svého pobytu v Praze si Einstein uvědomil, že k vyjádření svých myšlenek potřebuje jiný
  jazyk, i když byl nedůvěřivý k příliš abstraktní matematice; bál se, že by mohla zatemnit jeho
  krásné fyzikální myšlenky. Pár týdnů po příjezdu do Curychu navštívil jednoho ze svých nejstarších
  přátel, matematika Marcela Grossmanna, a obrátil se na něj s prosbou: „Musíš mi pomoci, nebo se
  zblázním!“Grossmann byl skeptický k lajdáckému způsobu, kterým fyzikové postupují při řešení
  problémů, ale svého přítele se snažil podpořit ze všech sil. 
  
  Einstein chtěl popsat co se děje, když jsou předměty urychlovány gravitací nebo když se pohybují
  vzhledem k urychlovanému systému. Jejich dráhy jsou zakřivené, nejsou to přímky, jako když se v
  inerciálním systému pohybuje bod, na který nepůsobí žádné síly. Ukazovalo se, že k popisu pohybu v
  zrychlených soustavách - který měl též odpovídat pohybu pod vlivem gravitace - bude třeba sáhnout
  k obecnější geometrii, než byla geometrie euklidovská. Grossmann dal Einsteinovi učebnici
  neeuklidovské neboli riemannovské geometrie. 
  
  Téměř sto let před tím, než Einstein začal přemýšlet nad zobecněním svého principu relativity,
  kolem roku 1820, učinil německý matematik Carl Friedrich Gauss podstatný krok k osvobození od
  Eukleidovy geometrie. Eukleidés stanovil pravidla pro čáry a tvary v plochém prostoru. Jeho
  geometrii se stále učíme ve škole a víme tedy, že k dané přímce lze nějakým bodem vést právě jednu
  přímku, která se s danou přímkou neprotne, tedy právě jednu rovnoběžku. Víme také, že přímky se
  obecně protnou právě v jednom bodě. Také si ze školy pamatujeme, že součet úhlů v trojúhelníku je
  180 stupňů a že čtverec má u čtyř vrcholů čtyři pravé úhly. V této geometrii platí řada pravidel,
  která aplikujeme, když například rýsujeme na čtvrtce papíru nebo na tabuli, a zjišťujeme, že
  fungují výborně. 
  
  Co by se ale dělo, kdybychom museli pracovat na pokrouceném papíře? Co když se budeme pokoušet
  rýsovat geometrické obrazce na hladký fotbalový míč? Pak zjistíme, že naše prostá euklidovská
  pravidla neplatí. Vezměme si zeměpisný globus. Když nakreslíme kolmo k rovníku dva poledníky, měly
  by být rovnoběžné, neměly by se nikdy protnout. Ale víme, že se protnou na obou pólech, přestože
  poledník i rovník jsou v dobrém smyslu nejpřímější čáry. Můžeme si vybrat speciální poledníky -
  greenwichský a poledník na devadesátém stupni třeba západní délky. Ty se protnou na severním pólu
  pod úhlem 90 stupňů a spolu s rovníkem vytvářejí trojúhelník, u nějž součet úhlů není 180, nýbrž
  třikrát 90, tedy 270 stupňů. Slavné pravidlo o součtu úhlů v trojúhelníku tedy také neplatí. 
  
  Ve skutečnosti každá uzavřená plocha - povrch koule, pneumatiky či preclíku - má svou vlastní
  geometrii, pro kterou platí vlastní pravidla. Gauss stanovil pravidla pro každý povrch, který si
  můžeme představit. Přistupoval k problému demokraticky - na všechny povrchy se musíme dívat
  stejným způsobem a jejich popis musí být určen nějakým obecným pravidlem. Gaussova geometrie byla
  účinná a obtížná. V padesátých letech devatenáctého století rozvinul Gaussovy myšlenky jiný
  německý matematik Bernhard Riemann a zobecnil je do více dimenzí. Vznikla složitá a náročná oblast
  matematiky, natolik obtížná, že i Grossmann, který ji Einsteinovi doporučil, měl pocit, že Riemann
  ji formuloval příliš abstraktně, než aby mohla být k praktickému užitku pro fyziku. Riemannova
  geometrie byla složitá, objevovala se v ní celá řada funkcí a nelineárních konstrukcí, ale
  skrývala veliké možnosti. Kdyby ji Einstein ovládl, mohla by mu ukázat cestu k vysněné teorii.
  
  Proto se do ní ponořil a snažil se zvládnout tento nadějný nástroj k zobecnění teorie relativity.
  Stála před ním veliká výzva - něco jako naučit se od základů sanskrt a hned v něm napsat román.
  
  Počátkem roku 1913 se Einstein seznámil s novou geometrií a spolu s Grossmannem napsali dva
  články, které představovaly náčrt (německy Entwurf) nové teorie. Jednomu z kolegů tehdy Einstein
  řekl: „Otázka gravitace je vyjasněná k mé plné spokojenosti.“Články byly napsány v jazyce nové
  matematiky, Grossmann v nich vysvětloval užitý matematický aparát pro širší obec fyziků.
  Einsteinovi se podařilo dát fyzikálním zákonům tvar, který zůstával stejný ve všech vztažných
  systémech, nejenom v systémech inerciálních. Uměl tak zapsat elektromagnetismus i Newtonovy
  pohybové zákony, tak jako je uměl dříve zapsat v souladu se speciálním principem relativity v
  inerciálních systémech. Ve skutečnosti měl úspěch se všemi zákony s výjimkou gravitace. Nový
  gravitační zákon, který Einstein s Grossmannem navrhli, stále stál stranou zákonů ostatních a
  obecnému principu relativity nevyhovoval. Ani s podporou nové matematiky tento pozměněný
  gravitační zákon nesplňoval požadavky, které Einsteinovi diktovala jeho intuice. Nicméně Einstein
  byl přesvědčen, že je na správné cestě a že teď už stačí jen malé úpravy, aby jeho teorie byla
  úplná. Byl však příliš optimistický. Konečný úsek cesty k jeho nové teorii prostoročasu
  nepřipomínal hladký běh, bylo na něm ještě mnoho klopýtnutí. 
  
  V roce 1914 se Einstein konečně usadil. Byl povolán do čela nově založeného Fyzikálního ústavu
  císaře Viléma v Berlíně, kde byl velmi dobře placen a neměl žádné pedagogické povinnosti. Stal se
  členem Pruské akademie věd a byl obklopen špičkovými fyziky, jako byli Max Planck nebo Walther
  Nernst. To bylo pro něj skvělé postavení, v rodinném životě však současně přišla roztržka s
  manželkou. Mileva už byla unavená cestováním po Evropě a zůstala s oběma syny v Curychu. Od té
  doby žili rozloučeni až do rozvodu v roce 1919. V témže roce se Einstein oženil se svou mladší
  sestřenicí Elsou Löwenthalovou, se kterou začal nový život a zůstal s ní až do její smrti v roce
  1936. 
  
  Einstein přijel do Berlína na začátku první světové války a měl pocit, že je „chycen v blázinci
  “německého nacionalismu. Extrémní nacionalismus ovlivňoval téměř vše. Kolegové z jeho okolí
  odcházeli na frontu nebo vyvíjeli nové zbraně, například obávaný hořčičný plyn - yperit. V říjnu
  1914 se objevila Výzva kulturnímu světu jednoznačně podporující německou vládu. Podepsalo ji 93
  německých vědců, spisovatelů a dalších představitelů kulturního světa a byla zamýšlená jako obrana
  proti údajným dezinformacím šířeným o Němcích po světě. Výzva, která vešla ve známost pod názvem
  Manifest devadesáti tří, tvrdila, že Němci nejsou zodpovědní za válku, která právě propukla. Text
  Manifestu se opatrně vyhýbal faktu, že Němci napadli Belgii a zničili město Lovaň, a tvrdil, že
  „naši vojáci se nedotkli života ani majetku jediného belgického občana“. Byl to text útočný,
  podvratný a z velké části nepravdivý. 
  
  Einstein byl šokován tím, co se kolem něho dělo. Jako přesvědčený pacifista a internacionalista se
  připojil k podpoře antimanifestu nazvaného Výzva Evropanům. V něm se jeho signatáři, mezi nimi
  Einstein a hrstka jeho kolegů, distancovali od Manifestu devadesáti tří, ostře kritizovali ty,
  kteří ho podepsali, a vyzývali „vzdělané lidi všech států“, aby bojovali proti zuřící ničivé
  válce. Tato iniciativa však zůstala vcelku bez ohlasu. Pro okolní svět byl Einstein jedním z
  dalších z německých vědců, kteří podporují Manifest devadesáti tří, a tedy nepřítel. Takto se na
  něj dívali přinejmenším v Británii. 
  
  Angličan Arthur Eddington proslul svými jízdami na kole na značné vzdálenosti. Svou cyklistickou
  vytrvalost hodnotil pomocí jím definovaného čísla E. Toto číslo reprezentovalo počet dní v jeho
  životě, kdy ujel více než E mil. Pochybuji, že mé číslo E dosahuje větší hodnoty než 5 nebo 6. Asi
  jsem více než šest mil za den neujel na kole víckrát než šestkrát - nic moc, já vím. Když
  Eddington zemřel, jeho číslo E bylo 87, to znamená, že podnikl 87 jízd, při kterých najel více než
  87 mil. Mimořádná životní energie a vytrvalost mu skvěle sloužily a pomohly mu dosáhnout
  vrcholných výsledků během celé jeho životní dráhy. 
  
  V době, kdy se Einstein úporně snažil založit svou vědeckou kariéru, se Eddington rychle
  dopracoval vrcholného postavení v anglické vědě. Při prosazování svých myšlenek byl někdy
  arogantní, pohrdavý a nepříjemně tvrdohlavý, byl ale také vytrvalým vědcem, který málokdy chyběl
  při některém výjimečně obtížném astronomickém pozorování a který se stále učil i té
  nejabstraktnější matematice. Byl vychován v hluboce věřící kvakerské rodině, od dětství mimořádně
  vynikal ve škole. V šestnácti letech začal studovat matematiku a fyziku v Manchesteru. Nakonec
  skončil v Cambridgi, kde byl studentem s nejlepšími výsledky v ročníku, jemuž se v Cambridgi
  říkalo Senior Wrangler. Když získal magisterský titul, stal se okamžitě asistentem královského
  astronoma a členem Trinity College v Cambridge.2 
  
  Univerzita v Cambridgi je špičková vědecká instituce a Eddington byl obklopen skvělými učenci. Byl
  zde například objevitel elektronu Joseph John Thomson nebo Alfred North Whitehead a Bertrand
  Russell, kteří společně napsali dílo Principia Mathematica, jež se stalo biblí logiků. Časem
  přibyli Ernest Rutherford, Ralph Fowler, Paul Dirac - samé velké osobnosti fyziky dvacátého
  století. Eddington do této společnosti skvěle zapadal. Po několika letech, kdy působil na
  greenwichské observatoři v Londýně, se do Cambridge vrátil. V pouhých jednatřiceti letech zde
  získal prestižní plumeovskou profesuru astronomie a experimentální filosofie (tradiční název po
  jejím 2 Titul Senior Wrangler lze přeložit jako „Nejlepší z diskutérů“. Název má historický původ
  z dob, kdy se na univerzitách vysoce cenila schopnost disputace. V Cambridgi získávají titul
  Wrangler studenti matematických disciplín, kteří dosáhli nejlepších výsledků. Senior Wrangler je
  pak nejlepší z nich. Královský astronom, Astronomer Royal je prestižní titul udělovaný vždy
  jednomu vynikajícímu astronomovi. Donedávna býval Astronomer Royal automaticky i ředitelem
  greenwichské observatoře. V současné době je královským astronomem astrofyzik sir Martin Rees.
  Pozn. překl. prvním držiteli v roce 1704 Johnu Plumeovi). Stal se též ředitelem cambridgeské
  observatoře na předměstí Cambridge, kde bydlel se svou sestrou a matkou. V následujících letech
  byl uznávaným vůdcem britské astronomie. V Cambridge už zůstal až do konce svých dnů. Aktivně se
  účastnil života kolejí s formálními večeřemi a neustálými debatami, pravidelně navštěvoval
  zasedání Královské astronomické společnosti, kde prezentoval své výsledky, a občas cestoval do
  různých konců světa za účelem astronomických pozorování. 
  
  S Einsteinovými myšlenkami o gravitaci se Eddington poprvé setkal právě na jedné takové cestě.
  Einstein předpověděl ohyb světla a několik astronomů se této myšlenky chopilo a snažili se tento
  efekt změřit. Vydávali se po celé zeměkouli, do Ameriky, Ruska i Brazílie, aby zachytili úplné
  zatmění Slunce za podmínek umožňujících měřit malý odklon světelných paprsků od vzdálených hvězd.
  Během pozorování zatmění v Brazílii se Eddington s jedním z těchto astronomů setkal. Byl to
  Američan Charles Perrine a Eddingtona upoutalo, čím se zabývá. Po návratu do Cambridge se rozhodl,
  že se s Einsteinovými novými myšlenkami o gravitaci seznámí podrobněji. 
  
  Když vypukla první světová válka, Eddington byl jedním z osamocených hlasů, které odporovaly
  fanatickému nacionalismu, jenž se zmocňoval jeho kolegů i celé jeho vlasti. Uvádělo ho to v
  zoufalství. V základním časopise britských astronomů The Observatory vyšla řada ostrých článků od
  vlivných britských astronomů proti spolupráci s německými vědci. Oxfordský profesor astronomie
  Herbert Turner to vyjádřil velice jasně: „Můžeme Německo znovu přijmout do mezinárodního
  společenství a zároveň snížit naše standardy mezinárodního práva na jeho úroveň, nebo je vyloučit
  a standardy pozdvihnout. Třetí cesty není.“ Animozita proti všemu německému byla tak silná, že
  prezident Královské astronomické společnosti, který měl německé předky, byl vyzván k rezignaci. Po
  dobu války byly vztahy mezi britskými vědci a jejich německými kolegy zcela zmrazeny. 
  
  Eddington uvažoval a jednal jinak. Jako kvaker byl vášnivě proti válce. Během rostoucí nenávisti
  proti Němcům se vyjadřoval nesouhlasně. „Nemyslete na symbolického Němce, ale na svého bývalého
  přítele, například profesora X,“ nabádal své kolegy. „Nazývejte ho Hunem, pirátem, zabíječem dětí
  a snažte se myslet na něj s nenávistí. Uvidíte, jak je to absurdní.“A nejen to. Eddington odmítal
  i vstoupit do armády a bojovat. Když viděl, jak někteří z jeho přátel a kolegů odpluli na pevninu
  na frontu a pak padli v boji, začal vést kampaň proti válce. Protože mu byla udělena výjimka na
  základě „národní důležitosti“ - Eddington byl pro národ důležitější jako astronom než jako
  infanterista - udělal si tak málo přátel. 
  
  Sám v Berlíně, obklopen válečným zmatkem, pracoval Einstein na dokončení své teorie. Vypadala
  dobře, ale potřebovala ještě vylepšit po matematické stránce. Tak se vydal na univerzitu v
  Göttingen, tehdejší mekku moderní matematiky, aby navštívil Davida Hilberta. Hilbert byl vědecký
  kolos a vládl světu matematiky. Pozměnil přístup k celému oboru a snažil se položit neotřesitelné
  základy, ze kterých by bylo možné odvodit veškerou matematiku. V matematice by nezůstala žádná
  volnost, vše by se dalo odvodit ze základních principů pomocí dobře stanovaných formálních
  pravidel. Matematické pravdy měly být skutečnými pravdami jen tehdy, když by byly odvozeny pomocí
  těchto pravidel. Tento scénář byl nazván „Hilbertův program“. 
  
  Hilbert se obklopil řadou kolegů, kteří patřili k těm nejvýznačnějším světovým matematikům. Jedním
  z jeho spolupracovníků byl například Hermann Minkowski, který Einsteinovi ukázal, jak se vztahy
  jeho speciální teorie relativity dají zapsat v mnohem elegantnějším matematickém jazyce, který
  však Einstein tehdy pokládal za zbytečný. (Minkowski ovšem už v roce 1909 zemřel.) Hilbertovi
  studenti a asistenti - jako byl Hermann Weyl, John von Neumann a Ernst Zermelo - se pak sami stali
  vůdčími osobnostmi matematiky dvacátého století. Hilbert se svou skupinou v Göttingen měl
  ambiciózní plán: založit úplnou teorii fyzického světa na několika základních principech, podobně
  jako to chtěl udělat v matematice. Einsteinovo snažení pro něj představovalo integrální součást
  celého projektu. 
  
  Během své krátké návštěvě v Göttingen v červnu 1915 Einstein přednášel a Hilbert si dělal
  poznámky. Podrobně diskutovali o detailech - Einstein byl silný ve fyzice a Hilbert zase v
  matematice. Neudělali však žádný pokrok. Pro Einsteina, který se stále trochu bál matematiky a
  jehož porozumění riemannovské geometrii stále nebylo dokonalé, bylo těžké plně pochopit Hilbertovy
  detailní technické připomínky. 
  
  Krátce poté, co se vrátil ze zdánlivě neplodné návštěvy, začal pochybovat o své nové teorii
  relativity. Už věděl, že není opravdu obecná - když v roce 1913 dokončil spolu s Grossmannem
  články o zobecněné teorii, byl si vědom, že gravitační zákon do ní stále úplně nezapadá. A některé
  její předpovědi nebyly v pořádku. Sice předpovídala posun perihelia Merkura, ale předpověď
  nesouhlasila plně s pozorováním. Einstein se musel na své rovnice podívat znova. 
  
  Během pouhých tří neděl se rozhodl odvrhnout gravitační zákon, který s Grossmannem navrhli a který
  obecnému principu relativity nevyhovoval. Hledal zákon, který by platil ve všech vztažných
  systémech, tak jako tomu bylo s ostatními fyzikálními zákony. Při jeho formulaci chtěl vycházet z
  matematického aparátu riemannovské geometrie, kterou se naučil od Grossmanna. Každých pár dní
  pozměňoval to, co krátce před tím udělal, měnil některé předpoklady, jež měl zákon splňovat, a
  nahrazoval je jinými. A přitom se více a více nořil do komplikované matematiky, kterou se naučil.
  Zjišťoval, že i když ho zatím během jeho zářné kariéry vedla především skvělá fyzikální intuice,
  musí dávat pozor, aby nezastínila velké poselství, jež přinášela matematika. 
  
  Koncem listopadu 1915 najednou zjistil, že je u cíle. Konečně měl obecný zákon pro gravitaci, jenž
  obecnému principu relativity vyhovoval. V měřítkách sluneční soustavy dával s vysokou přesností
  totéž, co Newtonova gravitační teorie, což bylo nutnou podmínkou vzhledem ke skvělé zkušenosti,
  kterou s newtonovskou gravitací astronomové měli. Navíc teorie až neuvěřitelně přesně předpovídala
  Le Verrierův posun perihelia Merkura. Také předpovídala ohyb světelných paprsků v blízkosti
  těžkých objektů. Ohyb však měl být dvakrát větší, než vyplývalo ze vzorce, ke kterému došel ve
  svých pražských úvahách. 
  
  Einsteinova úplná obecná teorie relativity nabízela zcela nový způsob chápání fyziky, který
  nahrazoval newtonovský pohled, jenž vládl po staletí. Podle Einsteinovy teorie byla gravitace
  určena sadou rovnic, jimž se říká Einsteinovy rovnice pro gravitační pole. Idea, která za nimi
  stála - propojit Gaussovu a Riemannovu geometrii s gravitací - byla velice krásná, fyzikové rádi
  užívají slova „elegantní“, výsledné rovnice byly ovšem ve skutečnosti značně komplikované. Bylo to
  deset složitě spolu svázaných nelineárních rovnic pro deset veličin, které určují geometrii
  prostoročasu. Nejde je proto řešit jednu po druhé pro jednotlivé veličiny, musí se řešit všechny
  současně. Po matematické stránce je to obecně krajně obtížná záležitost. Přesto ale tyto rovnice
  mnoho slibovaly - v principu měly popsat řadu procesů přírodního světa kolem nás. Mezi jejich
  řešeními jsou ta, jež popisují pád jablka ze stromu i pohyb planet ve sluneční soustavě. Vyřešením
  Einsteinových rovnic bychom se měli dozvědět mnoho o tajemstvích vesmíru. 
  
  Nové rovnice pro gravitační pole Einstein představil Pruské akademii věd v krátkém třístránkovém
  článku s datem 25. listopadu 1915. Jeho gravitační zákon byl zásadně odlišný od předchozích
  formulací. Došel k závěru, že to, co vnímáme jako pohyb pod vlivem gravitace, není v zásadě nic
  jiného, než pohyb v zakřivené geometrii prostoročasu, přičemž zakřivení této geometrie je vyvoláno
  hmotnými objekty. Einstein tak dospěl k opravdu obecné teorii relativity. 
  
  Einstein však nebyl sám. Hilbert se probíral svými poznámkami z Einsteinových přednášek v
  Göttingen a aniž to Einstein věděl, pokusil se sám najít správné gravitační rovnice. Zcela
  nezávisle došel ke stejnému gravitačnímu zákonu jako Einstein. Dvacátého listopadu, pět dní před
  Einsteinovým vystoupením před Pruskou akademií věd, předložil své vlastní výsledky Královské
  společnosti věd v Göttingen. Vypadalo to, že Hilbert Einsteina předstihl. 
  
  V týdnech po ohlášení těchto výsledků byly vztahy mezi Hilbertem a Einsteinem trochu napjaté.
  Hilbert napsal Einsteinovi, že si nic nepamatuje z přednášky, ve které Einstein hovořil o svých
  pokusech nalézt správné gravitační rovnice, a kolem Vánoc se Einstein upokojil, že nešlo o podraz.
  Jak vyjádřil ve svém dopise, „mezi námi zavládl trochu nedobrý pocit“, ale „naše přátelství
  pokládám za nezkalené“. Zůstali opravdu přáteli a kolegy, protože Hilbert neuplatňoval na
  Einsteinovo magnum opus žádný nárok a o rovnicích, na které přišel on i Einstein, hovořil do konce
  života jako o „Einsteinových rovnicích“. 
  
  Einstein byl tak na konci cesty k nalezení nové teorie gravitace. Postupně při hledání svých
  rovnic podlehl síle matematiky. Od té doby se už nenechával vodit jen myšlenkovými experimenty,
  nýbrž i matematickou elegancí. Matematická krása jeho konečné teorie ho omračovala. Gravitační
  rovnice označil za „nejcennější objev svého života“. 
  
  K Eddingtonovi se dostávaly některé separáty Einsteinových prací z Prahy, Curychu a konečně z
  Berlína přes jeho přítele holandského astronoma Willema de Sittera. Práce ho velmi zaujaly,
  propadl úplně novému pohledu na gravitaci formulovanému v obtížném jazyce. I když byl především
  astronomem, jehož hlavním cílem bylo pozorovat vesmír a interpretovat pozorování, přijal výzvu a
  nastudoval matematický jazyk Riemannovy geometrie, ve kterém Einstein svou novou teorii
  formuloval. A tento zájem se mu vyplatil, jmenovitě když Einstein jasně specifikoval předpověď
  ohybu světelných paprsků, jež se měla stát testem jeho teorie. Úplné zatmění Slunce mělo nastat
  29. května 1919 a bylo přirozené, aby se vedoucím pozorovacího týmu, který měl ověřit Einsteinovu
  předpověď, stal právě Eddington. 
  
  Byl zde jen jeden problém, zato veliký: Evropa byla ve válce. Eddington byl pacifista a Einstein
  byl podle názoru Eddingtonových kolegů ve spolku s nepřítelem. V roce 1918 dosáhla válka svého
  vrcholu a obava, aby německá armáda nezvítězila, vedla k dalším odvodům. Eddington byl povolán do
  armády. 
  
  Když se stal nadšeným advokátem Einsteinových myšlenek, sklízel za to antipatii svých kolegů.
  Jeden z nich prohlásil ve snaze odmítnout vše německé: „Snažili jsme se věřit, že přehnané a
  falešné požadavky dnešních Němců jsou důsledkem nějaké momentální nemoci, která propukla v nedávné
  době. Ale případy jako je tento naznačují, že se možná jedná o něco hlubšího.“ Eddington sice měl
  pro vedení expedice na zatmění podporu královského astronoma Franka Dysona, jenže jen o vlásek
  unikl uvěznění, protože odmítal bojovat. Britská vláda ustavila v Cambridgi tribunál, který měl
  zkoumat Eddingtonův postoj. S pokračujícím jednáním zaujímal tribunál k Eddingtonovi stále
  nepřátelštější stanovisko a vypadalo to, že výjimka pro něj bude zamítnuta. Pak ale vstoupil do
  hry Frank Dyson, který prohlásil, že Eddington je klíčovou osobou pro zdar expedice: „Za daných
  podmínek bude zatmění sledovat jen málo vědců a profesor Eddington je výjimečně kompetentní osoba
  pro tato pozorování, takže doufám, že tribunál mu dovolí je uskutečnit.“Zatmění tribunál zaujalo a
  Eddington dostal znovu výjimku z branné služby z důvodu „národního zájmu“. Tak jej Einstein
  zachránil před frontou. 
  
  Už jsme uvedli, že Einsteinova teorie předpovídala, že světlo ze vzdálených hvězd se bude ohýbat
  při průchodu kolem hodně hmotného tělesa, jakým je například Slunce. Eddington navrhl experiment,
  při kterém se vzdálená hvězdokupa, tzv. Hyady, měla fotografovat ve dvou různých dobách v roce.
  Nejdříve se tak měla určit poloha hvězd v hvězdokupě za jasné noci, když na ně nic nerušilo pohled
  - na dráze mezi nimi a Zemí nebylo žádné těleso, jež by mohlo šíření světla ovlivňovat. Pak se
  mělo měření provést znovu v době úplného zatmění, tedy v době, kdy se paprsky od těchto hvězd
  pohybovaly v blízkosti Slunce a téměř všechno světlo od samotného Slunce bylo blokováno Měsícem.
  Dne 29. května 1919 měly Hyady ležet právě za Sluncem a podmínky tak byly ideální. Srovnání
  obojího měření - jednou se Sluncem u dráhy paprsků, jednou bez něj - mělo ukázat, zda došlo k
  nějakému ohybu. A bude-li ohyb činit 1,7 úhlové vteřiny, bude to v souladu s Einsteinovou
  předpovědí. Úkol byl jasný a zdánlivě jednoduchý. 
  
  Jednoduché to však vůbec nebylo. Těch několik míst na Zemi, kde se dalo úplné zatmění pozorovat,
  bylo od Evropy hodně vzdálených. Astronomové se museli vydat se svými přístroji na dlouhou cestu
  světem, který se vzpamatovával z ničivé války. Eddington s kolegou z cambridgeské observatoře
  Edwardem Cottinghamem se rozhodli pro ostrůvek Príncipe. Astronomové Andrew Crommelin a Charles
  Davidson byli vysláni jako záložní tým do vesnice Sobral na severovýchodě Brazílie v chudé a
  prašné oblasti blízko rovníku. 

  \luagraphic[1]{fyz_fig0917.jpg}{Jedna z Eddingtonových fotografií zatmění Slunce z roku 1919, 
    potvrzujících platnost Einsteinovy obecné teorie relativity. Kredit: Wikipedia}{fyz:fig0917}
  
  Príncipe (Princův ostrov) je malý ostrov v Guinejském zálivu. Tehdy to byla portugalská kolonie,
  známá produkcí kakaa. Svěží zelený ostrov s horkým a vlhkým podnebím, který občas zasahují
  tropické bouře, má několik velkých plantáží, roças, na nichž několik portugalských majitelů
  využívalo k obdělávání půdy původní obyvatele. Po několik desetiletí tyto plantáže zásobovaly
  kakaovými boby společnost Catbury. Počátkem dvacátého století bylo kakaové plantážnictví nařčeno z
  využívání otrocké práce a majitelé přišli o své kontrakty, což ekonomiku na Príncipe zničilo. V
  době Eddingtonova příjezdu ostrov upadal v zapomnění. 
  
  Eddington instaloval svůj dalekohled na vzdáleném konci Roça Sundy. Čekání na zatmění si krátil
  každodenními tenisovými zápasy na jediném kurtu na ostrově a modlil se, aby bouře nebo mraky
  nezhatily jeho poslání. Cottingham se staral o teleskop a doufal, že horko nezdeformuje obrázky.
  
  
  Ráno v den zatmění hustě pršelo a nebe bylo dokonale neprůhledné, hodinu před vrcholem zatmění se
  však začalo vyjasňovat. Eddington a Cottingham zachytili první pohledy na Slunce ještě v době, kdy
  zatmění už nastávalo a byla zakrytá část slunečního disku. Ve čtvrt na tři bylo nebe jasné a
  Eddington s Cottinghamem mohli začít fotografovat. Udělali celkem 16 snímků na fotografické desky,
  jež zachycovaly zatmělé Slunce, za nímž na pozadí bylo vidět hvězdy příslušející k hvězdokupě
  Hyady. Ke konci zatmění už bylo nebe bez mráčku. Eddington telegrafoval Frankovi Dysonovi: „Mraky
  se trhají. Nadějné.“
  
  Přes zamračený začátek bylo pozorování na Príncipe zachráněno. V Sobralu na brazilském
  severovýchodě byl dokonale jasný horký den a zatmění se dalo výborně sledovat od samého počátku.
  Crommelin a Davidson, obklopeni místními obyvateli, kteří nadšeně pozorovali úkaz, nafotografovali
  celkem 19 desek, jež doplnily 16 snímků Eddingtona a Cottinghama. I oni plni nadšení
  telegrafovali: „Zatmění skvělé.“ V tu chvíli ještě nevěděli, že dobré pozorovací podmínky s jasným
  nebem nebyly jejich experimentu příznivé. Panující vedro pokroutilo přístroj natolik, že se snímky
  na fotografických deskách staly bezcennými. Pozorování ze Sobralu nakonec přispěla k celkovému
  výsledku jen daty pořízenými menším záložním teleskopem. 
  
  Astronomové se nemohli vrátit domů narychlo a tak se začalo s analýzou fotografických desek až
  pozdě v červenci. Z šestnácti desek, které Eddington nafotografoval, jen pouhé dvě zachycovaly
  dost hvězd, aby se ohyb dal dostatečně dobře měřit. Získaná hodnota ohybu byla 1,61 úhlových
  vteřin s chybou 0,3 vteřiny. To bylo v souladu s Einsteinovou předpovědí 1,7 vteřiny. Analýza
  desek ze Sobralu byla alarmující. Naměřená hodnota byla 0,93 úhlové vteřiny, což se výrazně lišilo
  od relativistické předpovědi a bylo velmi blízko předpovědi newtonovské, jenže šlo o desky, které
  byly deformované teplem. Když se však prozkoumala sobralská záložní pozorování malým teleskopem,
  vyšla naopak hodnota 1,98 s chybou 0,12 úhlové vteřiny, což bylo zase blízko Einsteinově
  předpovědi. 

  Dne 6. listopadu badatelé předložili své výsledky na společném zasedání Královské společnosti a
  Královské astronomické společnosti. V sérii přednášek, jimž předsedal Frank Dyson, byla měření z
  expedic za zatměním předložena váženým členům společností. Řečníci se shodli na tom, že když se
  vezmou v úvahu problémy, kterým čelila sobralská expedice, měření působivě potvrzují Einsteinovu
  předpověď. 
  
  Prezident Královské společnosti J. J. Thomson tato měření označil za „nejdůležitější výsledek z
  oblasti gravitace od Newtonových dob“. A dodal: „Jestliže uznáme, že Einsteinova argumentace je
  správná - a jeho teorie přežila přísné testy v souvislosti s posunem perihelia Merkura a měření
  při nedávném zatmění - pak tento výsledek patří k největším úspěchům lidského myšlení.“
  
  Den po zasedání v Burlingtonském paláci se Thomsonova slova objevila v londýnských Times. Hned za
  titulky oslavujícími výročí příměří a památku padlých stálo: „Revoluce ve vědě - Nová teorie
  vesmíru - Newtonovy myšlenky překonány“ a články popisovaly výsledky expedice za zatměním. Novinky
  a názory týkající se Einsteinovy nové teorie a Eddingtonovy expedice se šířily anglicky mluvícím
  světem jako požár. Desátého listopadu dostihly Ameriku, kde New York Times přinesly své vlastní
  senzační titulky jako „Všechna světla na nebi šejdrem“, „Einsteinova teorie triumfuje“, „Hvězdy
  nejsou tam, kde se zdály být nebo kde byly vypočteny, nikdo se ale nemusí bát“. 
  
  Eddingtonova hra se vyplatila. Tím, že novou obecnou teorii relativity pochopil a provedl její
  test, se stal prorokem nové fyziky. Od tohoto okamžiku se Eddington stal jedním z mála učenců, na
  které se všichni odvolávali při diskusích o nové teorii gravitace a jeho názor byl vyhledáván jako
  vodítko k interpretaci a vývoji Einsteinovy teorie. 
  
  A z Einsteina udělala Eddingtonova působivá mise superhvězdu. Eddingtonovy výsledky změnily
  Einsteinův život a jeho obecné teorii relativity přinesly - přinejmenším na chvíli - takovou
  popularitu, jakou nezažila žádná jiná vědecká teorie. Einstein po dvou stovkách let nerušené vlády
  sesadil z trůnu Newtona. Jeho teorie byla sice neprůhledná a zakuklená do matematického jazyka,
  kterému rozumělo jen pár lidí, skvěle však prošla Eddingtonovým testem. Navíc Einstein už nebyl
  nepřítel. Válka skončila, a i když vleklé nepřátelství k německým vědcům přetrvávalo, Einsteinovi
  bylo odpuštěno. Teď už se obecně vědělo, že Manifest devadesáti tří nepodepsal, i to, že vlastně
  ani nebyl Němec, nýbrž švýcarský Žid. Jak napsal v článku v Times krátce po Eddingtonově
  historickém oznámení před Královskou astronomickou společností: „V Německu jsem označován za
  německého učence a v Anglii jsem švýcarský Žid. Když bude potřeba ze mne udělat černou ovci, budu
  pro Němce švýcarský Žid a pro Angličany německý vědec.“
  
  Z neznámého patentového úředníka se sklonem k drzosti, jehož obdivovalo jen pár specialistů z jeho
  oboru, se Einstein stal kulturní ikonou, kterou zvali k přednáškám do Ameriky, Japonska a všude v
  Evropě. A jeho obecná teorie relativity, která měla své kořeny v myšlenkových experimentech, jež
  prováděl ve své bernské kanceláři, byla teď formulována jako nový a zcela rozdílný způsob jak
  dělat fyziku. Matematika v teorii relativity zaujala pevnou a důležitou pozici a teorie byla
  formulována sadou složitých, ale krásných rovnic, které byly zralé k vypuštění do světa. Nyní bylo
  i na jiných, aby rozpoznali, co vše je v nich vlastně obsaženo.

\section{Správná matematika, ohavná fyzika}\label{feyIchIIIsecIV} 
  Einsteinovy rovnice pole byly komplikované, obsahovaly spleť neznámých funkcí, v principu je však
  mohl řešit každý, kdo na to měl dostatečné schopnosti a vytrvalost. V desetiletí, jež následovalo
  po Einsteinově objevu, se kromě jiných vědců do tohoto úkolu pustil nadaný sovětský matematik a
  meteorolog Alexander Friedmann a brilantní belgický vědec, kněz, tedy francouzsky abbé, Georges
  Lemaître. Oba na základě rovnic obecné teorie relativity nalezli naprosto nový model vesmíru. Ten
  ovšem představoval pohled na svět, který Einstein už po delší dobu odmítal přijmout. Díky jejich
  práci ovšem teorie začala žít vlastním životem, jenž předčil Einsteinova očekávání. 
  
  Když Einstein v roce 1915 formuloval své rovnice, chtěl je vyřešit sám. Jako dobrý začátek se mu
  zdálo nalezení takového řešení, jež by dobře modelovalo vesmír jako celek. Do tohoto úkolu se
  pustil v roce 1917 a přijal několik jednoduchých předpokladů, jak by hledané řešení mělo vypadat.
  Podle jeho teorie rozložení hmoty a energie diktovalo prostoročasu, jak se má vyvíjet. Aby
  modeloval vesmír jako celek, musel vzít v úvahu všechnu hmotu a energii. Nejjednodušším a
  nejlogičtějším předpokladem bylo, že hmota je ve vesmíru rozložena rovnoměrně, tedy všude v
  prostoru se stejnou hustotou. Z tohoto předpokladu Einstein vyšel. Navázal na uvahy, které změnily
  astronomii v šestnáctém století, kdy Mikuláš Koperník vystoupil s odvážným tvrzením, že Země není
  centrem vesmíru, nýbrž obíhá kolem Slunce. Tato „koperníkovská“ revoluce pokračovala v dalších
  stoletích a naše poloha ve vesmíru se stávala stále méně významnou. V polovině devatenáctého
  století se ukázalo, že ani postavení Slunce není tak důležité; leží na nevýznamném místě v jednom
  ze spirálních ramen naší galaxie zvané Mléčná dráha. Když tedy Einstein hledal řešení svých rovnic
  za uvedeného předpokladu, jenom rozšiřoval do logických důsledků myšlenku, že by vesmír měl
  vypadat všude víceméně stejně: nikde ve vesmíru nemá být preferované místo, tím méně jeho střed.
    
  Za předpokladu, že vesmír je naplněn homogenně rozdělenou hmotou, se rovnice pole podstatně
  zjednodušily a daly se celkem snadno řešit, vedlo to však k velmi podivnému výsledku: vesmír se
  podle Einsteinových rovnic musel vyvíjet. Jednotlivé kousky hmoty a energie se měly vzhledem k
  sobě navzájem pohybovat organizovaným způsobem. Ve velkých měřítkách by nic nezůstávalo v klidu.
  Nakonec by se všechna hmota soustředila do jediného bodu, celý vesmír by zkolaboval a přestal
  existovat. 
  
  Představa astronomů o vesmíru byla v roce 1916 poměrně omezená. Měli vcelku dobře zmapovanou
  Mléčnou dráhu, bylo však zcela nejasné, co leží mimo ni. Nic nenapovídalo, jak vypadá vesmír jako
  celek. Všechna pozorování ukazovala, že hvězdy se trochu pohybují, ale nijak dramaticky a určitě
  ne uspořádaným způsobem ve velkých měřítkách. Einsteinovi, tak jako většině lidí, se nebe zdálo
  statické a nic nenaznačovalo, že by se vesmír jako celek hroutil, nebo rozpínal. Fyzikální intuice
  - i předsudky - jej měly k tomu, že se rozhodl možnost vyvíjejícího se vesmíru ze své teorie
  vymýtit. Do svých rovnic pole přidal novou konstantu, která dostala jméno „kosmologická“. Ta
  dokázala přesně vykompenzovat přitažlivou gravitaci „obyčejné“hmoty a energie. Všechna „obyčejná“
  hmota a energie, kterou si Einstein představoval rovnoměrně rozprostřenou ve vesmíru, se snaží
  podle jeho teorie vesmír smrštit, jak odpovídá naší intuici o přitažlivosti gravitace. Ale
  kosmologická konstanta funguje v rovnicích tak, že se snaží veškerou hmotu naopak rozptýlit,
  působí tedy odpudivě. Toto přitahování a odpuzování mohlo vesmír udržet v delikátní rovnováze,
  takže vesmír jako celek by byl v průměru neměnný, statický, jak podle Einsteina měl vypadat.3
  
  Způsob, jakým se Einstein snažil vyhnout nezbytnosti vývoje vesmíru, jeho teorii velmi
  zkomplikoval. Jak později přiznal, „zavedení kosmologické konstanty znamenalo značné narušení
  logické jednoduchosti teorie“. Jednomu příteli dokonce řekl, že „zavedením konstanty jsem udělal
  teorii gravitace něco, za co bych si zasloužil skončit v blázinci“. Ale zdálo se, že konstanta
  splnila svou úlohu. 
  
  V crescendu, jež vyvrcholilo objevem teorie relativity, Einstein často korespondoval a diskutoval
  s Willemem de Sitterem, holandským astronomem z univerzity v Leidenu. Ten jako občan neutrálního
  státu mohl během první světové války volně korespondovat i s Velkou Británií a tak se jeho
  zásluhou dostávaly zprávy o pokrocích Einsteinovy teorie k Eddingtonovi, který je detailně
  studoval. De Sitter byl tichý člověk, který však sehrál důležitou úlohu při přípravě výpravy za
  zatměním v roce 1919. 
  
  De Sitter byl vzděláním matematik, a tak byl dobře připraven pro práci s Einsteinovými rovnicemi.
  Když od Einsteina obdržel článek popisující statický vesmír, který se zrodil z Einsteinových
  rovnic doplněných kosmologickou konstantou, hned si uvědomil, že to není jediná možnost. Ukázal,
  že se dá dokonce zkonstruovat vesmír, ve kterém není žádná hmota, jen kosmologická konstanta.
  Navrhl realistický model, který mohl obsahovat hvězdy, galaxie a další hmotu, jenže v tak
  nepatrném množství, že tato hmota neovlivňuje prostoročas a není schopná kompenzovat odpudivý
  účinek kosmologické konstanty. Geometrie de Sitterova vesmíru měla tak být plně určená právě
  kosmologickou konstantou. 3 Přitažlivým a na popularizaci vděčným rysem Einsteinova statického
  vesmíru byla skutečnost, že byl prostorově konečný. V jeho trojrozměrném neeuklidovském prostoru
  se přímky, realizovatelné např. světelným paprskem, uzavíraly do sebe, podrobněji viz třeba
  Barrow, J., Kniha vesmírů, Praha: Paseka 2013. Karel Čapek tuto představu s velkým porozuměním
  použil v Krakatitu: „a chodba je na pohled rovná a lesklá jako hamburský tunel a přece se vrací
  kruhem; Prokop vzlyká děsem; to je Einsteinův vesmír…“. Pozn. překl. 
  
  De Sitterův vesmír se jevil statický, právě tak jako vesmír Einsteinův, tedy ve shodě s
  Einsteinovým přesvědčením. Měl ale zvláštní vlastnost, které si všiml sám de Sitter ve svých
  článcích. I de Sitterův vesmír byl statický podobně jako vesmír Einsteinův a jeho geometrie,
  například křivost prostoru v určitém bodě, se s časem neměnila. Jestliže se ale v de Sitterově
  vesmíru rozptýlilo několik galaxií či hvězd - a v našem vesmíru je samozřejmě takových objektů
  řada - začnou se vzájemně pohybovat. I když prostorová geometrie de Sitterova vesmíru je plně
  statická a pro všechny časy zůstává stejná, objekty v tomto vesmíru nezůstanou ve vzájemném klidu.
  
  Své řešení gravitačních rovnic poslal de Sitter Einsteinovi několik týdnů po tom, co od něj
  obdržel článek o statickém vesmíru. Einstein konstatoval, že de Sitterovo řešení je matematicky
  správné, ale vůbec se mu nelíbilo. Výchozí myšlenka, že řešení odpovídá prázdnému prostoru bez
  hmoty a hvězdy jsou jen dynamicky nevýznamné smetí, se mu zdála nepřijatelná. Pro Einsteina bylo
  podstatné brát materiální náplň vesmíru jako referenční soustavu, bez které by nemělo smysl
  hovořit o pohybu, zrychlení nebo otáčení. Einsteinova intuice potřebovala vesmír s materiální
  náplní. Napsal o tom dopis Paulu Ehrenfestovi, ve kterém vyjádřil své podráždění nad představou
  vesmíru bez hmoty. „Přijmout takovou možnost,“psal, „se mi zdá nesmyslné.“ Ale i když Einstein
  reptal, obecná relativita měla pár let po svém vzniku dva statické modely vesmíru, jež se od sebe
  zásadně lišily. 
  
  Zatímco Einstein pracoval na své obecné teorii, Alexander Friedmann bombardoval Rakousko. Byl
  pilotem ruské armády, do které dobrovolně vstoupil už v roce 1914 a sloužil u leteckého průzkumu,
  nejdříve na severní frontě, později ve Lvově. Po krátkou dobu na začátku války Rusko vítězilo. Při
  pravidelných nočních letech přispíval k tomu, aby dohnal ke kapitulaci města obklíčená ruskou
  armádou. Město za městem se tak dostávalo pod ruskou okupaci. 
  
  Friedmann postupoval při bombardování jinak než jeho kolegové. Ti shazovali bomby od oka a
  nepřesně odhadovali, kam asi mohou dopadnout, zatímco on byl podstatně pečlivější. Odvodil vzorec,
  ve kterém vystupovala rychlost letadla, jeho výška a hmotnost bomby a který umožňoval určit, kde
  bombu shodit, má-li zasáhnout cíl. Výsledkem bylo, že Friedmannovy bomby dopadaly přesněji a
  zasahovaly, co zasáhnout měly. Jeho letecké úspěchy byly odměněny vyznamenáním Křížem svatého Jiří
  za hrdinství v boji. 
  
  Před rokem 1914 byla jeho specializací matematika, a tak byl dobře připraven na provádění
  složitých výpočtů. Často se potýkal s problémy, jež byly obtížně řešitelné v době, kdy ještě
  neexistovaly počítače. Friedmann však postupoval odvážně a uměl řešené rovnice zjednodušit tak, že
  odvrhl zbytečný balast a ponechal jen to podstatné. Když nebyly řešitelné ani potom, dařilo se mu
  získávat odpovědi pomoci vtipných grafů a obrázků. Řešil širokou škálu problémů, od předpovědí
  cyklonů až po vliv proudění vzduchu na trajektorie bomb. Obtíže Friedmanna prostě nezdolaly. 
  
  Na začátky dvacátého století se Rusko významně měnilo. Carský režim se potácel od krize ke krizi,
  čelil rostoucí nespokojeností mezi zbídačelým obyvatelstvem a musel hledat své postavení ve víru
  stále nestabilnější Evropy. Friedmann se s nadšením podílel na sociálních změnách kolem sebe. Již
  jako gymnaziální student se účastnil se svými spolužáky protestů v první ruské revoluci v roce
  1905, jež otřásla zemí. Jako univerzitní student v Sankt-Petěrburgu vynikal neobyčejnou bystrostí
  a války se pak účastnil jako pilot, konající též bombardovací mise, učitel aeronautiky i jako
  vedoucí průmyslového závodu, jenž vyráběl navigační přístroje. 
  
  Po válce získal Friedmann postavení profesora na univerzitě v Sankt-Petěrburgu (pozdějším
  Leningradu). Do Ruska už tehdy také dorazil „relativistický cirkus“, jak tomu říkal Einstein.
  Friedmanna lákala krásná složitá matematika obecné teorie relativity, a tak se rozhodl zkusit
  najít řešení gravitačních rovnic pro stejnou situaci, kterou před ním zkoumal Einstein, tedy pro
  vesmír jako celek. Tak jako Einstein zjednodušil rovnice předpokladem, že když se rozložení hmoty
  zprůměruje na velkých měřítkách, je hustota hmoty všude stejná. V takovém případě je vesmír určen
  jedinou veličinou zvanou celková křivost. Podle Einsteina měla být tato veličina rovna jednou
  provždy danému číslu, určenému delikátní rovnováhou mezi kosmologickým členem, resp. kosmologickou
  konstantou, a hustotou hmoty. Tuto hmotu tvoří hvězdy a planety, jejichž hmotnost si představujeme
  rovnoměrně rozprostřenou v prostoru. 
  
  Friedmann Einsteinovy výsledky ignoroval a postupoval samostatně. Když studoval, jak hmota a
  kosmologická konstanta ovlivňují geometrii vesmíru, všiml si pozoruhodné skutečnosti: křivost
  vesmíru se obecně vyvíjí s časem. „Obyčejná“ hmota ve vesmíru, kterou tvoří hvězdy a galaxie a
  kterou Friedmann i Einstein předpokládali rovnoměrně rozloženou, se snaží prostor smršťovat. Je-li
  kosmologická konstanta kladné číslo, nutí prostor naopak roztahovat se, rozpínat. Einstein oba
  efekty, snahu stahovat se a rozpínat se, vybalancoval, takže prostor zůstával v klidu. Friedmann
  si ale všiml, že toto statické řešení byl jen velmi speciální případ mezi možnými řešeními. Obecné
  řešení vypadalo tak, že vesmír se musí s časem měnit. Jestli se rozpíná, nebo se naopak smršťuje,
  závisí na počátečním stavu a na tom, zda převládne vliv obyčejné hmoty či kosmologické konstanty.
  
  V roce 1922 Friedmann zveřejnil vlivný článek „O křivosti prostoru“, ve kterém ukázal, že nejenom
  Einsteinův, ale i de Sitterův vesmír jsou jen speciálními případy modelů vesmíru, jehož chování
  může být mnohem roztodivnější. Ta obecnější řešení odpovídala vesmíru, který se s rostoucím časem
  buď rozpíná, nebo smršťuje. Jedna třída modelů se dokonce mohla rozpínat, dosáhnout jakéhosi
  maxima a pak se opět smršťovat a zdálo se, že se to může opakovat v nekončících cyklech.
  Friedmannovy modely dokonce zbavily kosmologickou konstantu důvodu jejího zavedení. Einstein se
  domníval, že jeho vesmír, kde je hroucení kompenzováno odpudivým účinkem kosmologické konstanty,
  je jediný možný. Jakmile se stal jen jednou z možností mezi třídou modelů, které se obecně
  rozpínaly či smršťovaly, a to i tehdy, když kosmologická konstanta byla nenulová, její existence
  se stávala pochybnou. V závěru článku píše Friedmann trochu pohrdlivě: „Kosmologická konstanta …
  je neurčená … je to prostě libovolná konstanta.“Tím, že opustil Einsteinův požadavek, že vesmír má
  být statický, vlastně ukázal, že konstanta je v rovnicích zbytečná. Jestliže se vesmír vyvíjí,
  není nutné teorii komplikovat další neurčenou konstantou, jak to udělal Einstein. 
  
  Byl to článek, který přišel jako blesk z čistého nebe. Friedmann se nikdy neúčastnil diskusí s
  Einsteinem, neseděl na přednáškách, které Einstein konal pro Pruskou akademii věd. Byl to
  outsider, který podlehl vlně nadšení, která se vzedmula po Eddingtonově expedici za zatměním.
  Friedmann byl především matematický fyzik a vše, co na novém poli kosmologie udělal, bylo to, že
  zde využil zručnosti a matematických dovedností, kterých užíval při studiu bomb a počasí.
  Výsledky, kterých dosáhl, nebudily v Einsteinovi dobré pocity. 
  
  Myšlenka vyvíjejícího se vesmíru se Einsteinovi zdála absurdní. Když četl Friedmannův článek
  poprvé, zcela odmítal. že by jeho teorie mohla vést k něčemu takovému. Friedmann nemůže mít pravdu
  a Einstein se to snažil prokázat. Friedmannův článek pečlivě četl a podařilo se mu najít něco, co
  pokládal za základní chybu. Jakmile se tato chyba odstranila, i z Friedmannových výpočtů plynul
  statický vesmír, který dříve objevil Einstein. Einstein rychle publikoval poznámku, ve které
  tvrdil, že „význam Friedmannova výpočtu spočívá v tom, že znovu potvrdil nutnost statičnosti a
  neměnnosti vesmíru“. 
  
  Friedmann se cítil Einsteinovou poznámkou ponížený. Byl přesvědčen, že žádnou chybu neudělal a že
  početní chybu udělal naopak Einstein. Napsal mu dopis, ve kterém ukazoval, kde se omylu dopustil,
  a zakončil jej přáním: „Zjistíte-li, že mé výpočty jsou správné, buďte tak laskav a uvědomte o tom
  redaktory Zeitschrift für Physik.“Odeslal dopis do Berlína a doufal, že Einstein zapracuje rychle.
  
  Einstein však tento dopis nikdy neobdržel. Díky své slávě byl neustále nucen účastnit se různých
  seminářů a konferencí, cestovat po světě od Holandska a Švýcarska až po Palestinu a Japonsko,
  takže dlouho nebyl v Berlíně, kde na něj Friedmannův dopis čekal a pokrýval se prachem. Jen
  náhodou došlo k tomu, že na leidenské observatoři narazil na jednoho z Friedmannových kolegů a o
  Friedmannově odpovědi se dozvěděl. A tak se stalo, že Einstein po téměř šesti měsících uveřejnil
  „opravu ke své opravě“Friedmannova hlavního výsledku a připustil, že „existuje časově proměnné
  řešení“ pro vesmír. Ale pořád byla situace taková, že Friedmann jen dokázal existenci časově
  proměnného řešení Einsteinových rovnic. Podle Einsteina to byla pouhá matematika, ne realita. On
  sám byl stále ovládán předsudkem, že vesmír musí být statický. 
  
  Friedmann získal proslulost tím, že opravil velkého guru. Ale i když obecnou teorii relativity
  stále popularizoval ve své zemi, která se teď nazývala Sovětský svaz, a školil v této oblasti i
  nějaké doktorandy, vrátil se k meteorologii. Zemřel v pouhých 37 letech na tyfovou horečku, kterou
  se nakazil na výletě na Krym. Jeho matematické modely vesmíru, který se vyvíjí, se na několik let
  uložily ke spánku. 
  
  Georges Lemaître se dostal k matematice i k náboženství v mladém věku. Už ve škole byl zručný v
  řešení rovnic a velice bystrý při luštění matematických hádanek. V Bruselu absolvoval jezuitské
  gymnázium a odešel studovat důlní inženýrství. Stále ještě studoval, když byl v roce 1914 povolán
  do armády. V době, kdy Einstein a Eddington bojovali proti válce, Lemaître bojoval v zákopech,
  když Němci napadli Belgii. Němci zničili město Lovaň a rozlítili mezinárodní společenství.
  Následovalo sepsání neblaze proslulého Manifestu devadesáti tří německých vědců, který nadlouho
  otrávil vztahy mezi britskou a německou vědou. Lemaître byl příkladný voják, stal se dělostřelcem
  a povýšil do hodnosti dělostřeleckého důstojníka. Podobně jako Alexander Friedmann i on dal své
  matematické schopnosti do služeb balistiky. Na konci války mu byl udělen belgický válečný kříž za
  udatnost. 
  
  Válečné krveprolití, strašné účinky chloru v zákopech a brutalita na frontě Lemaîtra hluboce
  ovlivnily. Po skončení vojenské služby pokračoval nejen ve studiu matematiky a fyziky, ale
  vstoupil i do kněžského semináře Maison Saint Rombaut a v roce 1923 byl vysvěcen na kněze. Zbytek
  svého života se nadšeně věnoval matematice a zároveň duchovnímu životu. Po čase získal titul
  čestného kanovníka a v roce 1960 jej papež Jan XXIII. jmenoval osobním prelátem a prezidentem
  Pontifikální akademie věd ve Vatikánu. A tento kněz-vědec obrátil ve dvacátých letech svou
  pozornost k řešení Einsteinových rovnic pro vesmír. 
  
  Einsteinova obecná teorie relativity přitahoval Lemaîtra už na univerzitě v Lovani. Přednášel zde
  o ní na seminářích a psal na toto téma krátké přehledy. V roce 1923 pobýval nějakou dobu v
  anglické Cambridgi, bydlel zde v domě katolických kněží. Spolupracoval s Eddingtonem, který jej
  hlouběji seznámil se základy teorie relativity a vedl jej k práci na hledání správné teorie
  vesmíru. Lemaître na Eddingtona velmi zapůsobil, shledával ho „skvělým studentem, rychle
  uvažujícím, jasnozřivým a s velkou matematickou zručností“. Když se v roce 1924 Lemaître přesunul
  do Cambridge v Massachusetts, jeho hlavním tématem se stal nevyřešený problém jak nejpřesněji
  modelovat vesmír, kterému se věnoval během práce na doktorské disertaci na MIT (Massachusetts
  Institute of Technology, proslulá univerzita v americkém Bostonu). 
  
  Když se v roce 1923 Lemaître obrátil ke kosmologii, byl stále ve hře jak Einsteinův, tak de
  Sitterův model. Byly to jediné dva matematické modely, které řešily Einsteinovy rovnice, ale nic
  víc, pozorování neupřednostňovala ani jeden z nich. Vyvíjející se vesmír Friedmannův neměl
  vážnější odezvu a Einsteinova preference statického vesmíru zabránila tomu, aby se někdo další
  vydal Friedmannovou cestou. Převažující názor stále byl, že vesmír je statický. Ale Eddingtona
  zaujal de Sitterův model, ve kterém byly hvězdy geometrií vesmíru rozptylovány. De Sitter
  argumentoval, že toto by mohlo jeho vesmír observačně odlišit od Einsteinova. V takovém vesmíru by
  totiž vzdálené objekty vypadaly zvláštně - světlo z nich by mělo větší vlnovou délku než světlo z
  objektů blízkých. 
  
  Světlo si můžeme představovat jako soubor vln o různých vlnových délkách, jež odpovídají různým
  energetickým stavům. Červené světlo má větší vlnovou délku, což odpovídá nižšímu energetickému
  stavu, než světlo modré. Když se díváme na světlo hvězdy, galaxie nebo jiného jasného objektu,
  světlo jimi vyzařované je směsí různých barev. Základní „kvanta energie“světla - fotony - mají
  energii úměrnou své frekvenci, to znamená nepřímo úměrnou vlnové délce. De Sitter si uvědomil, že
  jednotlivé vlny vysílané vzdalujícím se zdrojem budou mít o něco větší vlnové délky, viditelné
  světelné spektrum se posune k větším vlnovým délkám. A v de Sitterově vesmíru by platilo, že čím
  vzdálenější objekt, tím větší by byl jeho červený posun. 
  
  Jev červeného posunu, tedy to, že světlo ze vzdálených galaxií se zdálo červenější než světlo z
  galaxií blízkých, ukazoval na to, že de Sitterovu vesmíru plně nerozumíme. Eddington jej studoval
  podrobněji spolu Hermannem Weylem, jedním ze žáků Davida Hilberta z Göttingen. Nalezli přesnou
  relaci mezi vzdáleností zářícího objektu a jeho červeným posunem: byl-li objekt od Země dvakrát
  vzdálenější, byl jeho červený posun dvojnásobný. Tento efekt byl nazván de Sitterovým jevem. 
  
  Když v roce 1924 Lemaître pečlivěji prostudoval de Sitterův vesmír i Eddingtonovy a Weylovy
  výpočty, všiml si, že gravitační rovnice v de Sitterově článku byly zapsány zvláštním způsobem. De
  Sitter zkoumal statický vesmír, ten měl ale tu podivnou vlastnost, že měl střed a pro pozorovatele
  ve středu existoval horizont, za kterým nebylo nic vidět. To bylo v rozporu s Einsteinovým
  požadavkem, že vesmír musí být ve všech místech stejný. Když Lemaître zapsal de Sitterův vesmír
  takovým způsobem, že ve všech bodech opravdu vypadal stejně, horizont zmizel a vesmír se celkově
  choval naprosto jinak. V Lemaîtrově jednodušším zápisu se křivost prostoru vyvíjela v čase a
  geometrie se vyvíjela tak, že jednotlivé pevné body v prostoru se navzájem vzdalovaly. Právě tento
  vývoj vysvětloval de Sitterův jev - Lemaître narazil na vyvíjející se vesmír. Lemaîtrův objev, že
  červený posun je spojen s rozpínáním vesmíru, však přinesl něco, co ve Friedmannově objevu, který
  vznikl o trochu dříve, chybělo: jev mohl být zkoumán v reálném světě astronomickým pozorováním.
    
  Lemaître dotáhl svou analýzu dále a hledal další řešení. Ke svému údivu zjistil, že statické
  modely Einsteina i de Sittera jsou v rámci Einsteinovy teorie prostoru a času případy zcela
  výjimečnými. De Sitterův vesmír může být přepsán do tvaru, v němž vypadá ne jako statický, nýbrž
  jako vyvíjející se vesmír, Einsteinův model byl nestabilní, nepatrná porucha by jej vyvedla z
  klidu. Jestliže se v Einsteinově vesmíru objevila jen nepatrná nerovnováha mezi hmotou a
  kosmologickou konstantou, vesmír se rychle začal buď rozpínat, nebo smršťovat, přestal být
  statický, tedy takový, jakým si ho Einstein tak moc přál mít. Le maître ve skutečnosti zjistil, že
  Einsteinův i de Sitterův model jsou jen dvěma výjimečnými členy obrovské rodiny modelů, které se
  obecně rozpínají či smršťují. 
  
  De Sitterově modelu se dostalo i určité astronomické podpory. Už v roce 1915, tedy ještě dříve,
  než de Sitter předložil svůj model a upozornil na jeho charakteristický znak, naměřil americký
  astronom Vesto Slipher červený posun na světlých obláčcích rozesetých po obloze, jimž se říkalo
  mlhoviny. Jeho zjištění spočívalo na studiu spektra těchto mlhovin. Jednotlivé chemické prvky, ze
  kterých jsou složeny zářící objekty, ať je to žárovka, žhavé uhlí, hvězda nebo mlhovina, vysílají
  charakteristický vzorek vlnových délek světla. Ve spektrometru se tyto vlnové délky objeví jako
  série čar připomínající čárový kód. A tomuto čárovému kódu se říká spektrum objektu. 
  
  Slipher pomocí svých přístrojů na Lowellově observatoři ve Flagstaffu v Arizoně proměřoval spektra
  mlhovin v různých místech oblohy. Pak porovnával naměřená spektra s čárami stejných prvků,
  získanými v pozemské laboratoři. (Tato spektra byla dobře známá, takže ve skutečnosti nemusel tyto
  pokusy sám provádět.) Konstatoval, že naměřená spektra byla proti očekávání lehce posunuta. Čárové
  kódy byly posunuty buď k červenému, nebo naopak k modrému konci spektra. 
  
  Tento posun naznačoval, že pozorované objekty se vzhledem k nám pohybují. Jestliže se objekt
  vzdaluje, vysílané vlnové délky se prodlužují a světlo se zdá červenější. Naopak, jestliže se
  objekt přibližuje, vlnové délky se zkracují a objekt se zdá být modřejší. Tento efekt je znám jako
  Dopplerův jev a asi jste si ho povšimli v souvislosti se zvukem. Jede-li směrem k vám ambulance,
  zvuk její sirény se nám zdá vyšší, než když se od vás vzdaluje. Efekt se týká všech vln, tedy i
  světelných, a proto na jeho základě byl Slipher schopen určit pohyb vzdálených objektů vesmírem.
  
  Slipherovy výsledky nebyly zcela udivující. Předpokládal, že nebeské objekty se pohybují, otáčejí
  se kolem sebe díky vzájemnému gravitačnímu působení. Ve skutečnosti jeho výsledky zprvu
  naznačovaly, že jedna z nejjasnějších mlhovin, mlhovina v souhvězdí Andromeda, se k nám
  přibližuje, protože její světlo bylo posunuto do modra. Ale Slipher byl systematický a zaznamenal
  spektra několika dalších mlhovin. A co zjistil, to už udivující bylo: skoro všechny mlhoviny se
  podle jeho výsledků od nás vzdalují. V pohybu mlhovin byl určitý trend. 
  
  V roce 1924 mladý švédský astronom Knut Lundmark vzal Slipherova data a udělal hrubý odhad, jak
  asi daleko od nás mohou různé mlhoviny být. Neuměl to určit přesně a vůbec si nebyl jistý svými
  výsledky. Ale přece jen mu vycházel určitý trend - čím byla mlhovina vzdálenější, tím větší byl
  červený posun jejího spektra, tedy tím rychleji se vzdalovala. 
  
  V roce 1927 abbé Lemaître znovu odvodil trend, jenž vyplýval z de Sitterova modelu a který
  korespondoval se Slipherovým měřením. Jeho výpočty předpovídaly červený posun a mezi ním a
  vzdáleností vzdálených galaxií měl být lineární vztah. Když se nakreslil graf, kde se na
  vodorovnou osu nanášela vzdálenost a na svislou červený posun příslušné galaxie, všechny galaxie
  by měly ležet na přímce. Aniž znal Friedmannovy práce, Lemaître své výsledky sepsal, zveřejnil je
  však v poněkud obskurním, málo čteném belgickém časopise. Shrnul tam své výpočty a v krátkém
  odstavci diskutoval observační důkazy, kde uvedl lineární vztah, který objevil Eddington, Weyl a
  on. Observační důkazy nebyly moc přesvědčivé a byly zatíženy velikými chybami, ale přesto bylo
  podivuhodné, jak do sebe vše zapadalo. 
  
  K velkému Lemaîtrovu zklamání vůdčí teoretici v relativitě včetně jeho někdejšího školitele
  Eddingtona jeho článek zcela ignorovali. V následujícím roce se na konferenci setkal s Einsteinem,
  toho však jeho práce vůbec neoslovila. Upozornil Lemaîtra, že jeho výpočty jsou jen replikou
  výpočtů Alexandera Friedmanna. Einstein připustil, že Friedmannovy výpočty jsou správné, nalezená
  řešení však pokládal za pouhou matematickou kuriozitu, jež nemá co dělat se skutečným vesmírem,
  který podle Einsteina musel být statický. Svoje ohodnocení Lemaîtrových výsledků shrnul odmítavým
  bonmotem: „I když jsou vaše výpočty správné, vaše fyzika je ohavná.“ A tímto rozsudkem zmizely
  Lemaîtrovy vesmíry někde v pustině, alespoň na chvíli. 
  
  Edwin Hubble byl mnohem více respektován pro svou pozorovatelskou zručnost než pro kouzlo své
  osobnosti. Studoval na Chicagské univerzitě, kde se stal boxerským šampionem, nebo to alespoň o
  sobě tvrdil. Potom strávil několik let na univerzitě v Oxfordu, kde získal nesnesitelný anglický
  akcent, jenž mu pak zůstal po zbytek života. Své pompézní chování doplňoval tvídovým oblekem a
  dýmkou, nezbytnými rekvizitami anglického venkovského šlechtice. Po Oxfordu Hubble bojoval v první
  světové válce, stejně jako Lemaître a Friedmann, do bojových akcí se však dostal až na samém jejím
  konci. 
  
  Koncem dvacátých let dvacátého století budila Hubbleova práce velikou pozornost, protože pár let
  před tím narazil na zlatou žílu. Od počátku století bylo jasné, že žijeme ve velkém víru hvězd,
  které tvoří naši galaxii Mléčnou dráhu. Nad astronomií však visela nezodpovězená otázka: byla
  Mléčná dráha jedinou galaxií, osamělým ostrovem v prázdnotě prostoru, nebo bylo ve vesmíru galaxií
  mnoho? Když se podíváme na noční oblohu, vidíme kromě hvězd a planet slabě svítící záhadné
  obláčky, tytéž mlhoviny, jejichž spektra Slipher měřil. Byly tyto mlhoviny jen vyvíjející se
  hvězdy v galaxii Mléčné dráhy, nebo jiné vzdálené galaxie a Mléčná dráha je jen jednou galaxií z
  mnoha? 
  
  Hubble tuto otázku rozhodl tím, že změřil vzdálenost jedné určité mlhoviny v souhvězdí
  Andromeda. Uvědomil si, že k tomu může použít proměnných hvězd zvaných cefeidy. Když porovnal, o
  kolik jsou temnější cefeidy v Andromedě než cefeidy o stejné periodě v Mléčné dráze, mohl
  odhadnout jejich vzdálenost.4 Vzdálenost mlhoviny v Andromedě mu vyšla obrovská, kolem milionu
  světelných let, což byl pětiaž desetinásobek odhadovaného průměru Mléčné dráhy. Mlhovina v
  Andromedě tedy nemohla patřit k Mléčné dráze - byla příliš daleko. Přirozené vysvětlení bylo, že
  je to jiná galaxie podobná té naší. A když to platilo o mlhovině v Andromedě, proč by to nemělo
  platit i o ostatních mlhovinách? Tímto jediným měřením v roce 1925 Hubble vesmír podstatně
  zvětšil. 
  
  V roce 1927 se Hubble zúčastnil zasedání Mezinárodní astronomické unie v Holandsku. Slyšel zde o
  senzaci, kterou vyvolala de Sitterova, Eddingtonova a Weylova předpověď kosmologického červeného
  posunu, a dozvěděl se též, že Slipherova měření mohou být prvním krokem k údajům, jež předpověď
  potvrdí. Lundmarkův pokus sestavit graf, který by dával červený posun dohromady se vzdáleností
  mlhovin, byl publikován v roce 1924, rok před Hubbleovým měřením vzdálenosti mlhoviny v Andromedě
  a byl přijímán se značným skepticismem. Abbé Lemaître užil Hubbleovo měření vzdáleností ve svém
  článku z roku 1927, jenže jej publikoval francouzsky, takže ho nikdo nečetl. Hubble 4 Vztah mezi
  periodou proměnných hvězd cefeid a jejich jasem objevila v roce 1912 výpočtářka harvardské
  observatoře Henrietta Swan Leavittová (1868-1921). Hubble její zásluhy uznával a vyslovoval se, že
  za svůj objev měla získat Nobelovu cenu. Pozn. překl. vycítil svou příležitost vstoupit do hry a
  změřit de Sitterův efekt, nevšímat si předchozích pokusů a postavit se do pozice objevitele.
  
  \luagraphic[1]{fyz_fig0918.jpg}{Dr. Albert Einstein s ostatními vědci po návštěvě největšího
  dalekohledu na světě na observatoři Mount Wilson v pohoří San Gabriel poblíž Pasadeny.
  }{fyz:fig0918}

  Získal ke spolupráci pracovníka z řad techniků observatoře na Mount Wilson Miltona Humasona. Noc
  co noc seřizoval Humason hranoly v dalekohledu na Mount Wilson vysoko v horách nad Pasadenou v
  Kalifornii a proměřoval spektra. Byla to nevděčná práce. Kupole observatoře byla chladná a temná a
  Humason měl rozbolavěné nohy od železné podlahy. Bolela ho záda z nekonečného sezení u přístroje a
  koukání do okuláru, když se snažil najít spektrální čáry vybraných mlhovin. Věděl, že musí dojít
  dále než Slipher a zkoumal i velmi slabé mlhoviny. Čím byly nezřetelnější, tím by měly být
  vzdálenější. Jenže pracoval s přístrojem, který nebyl určen přesně pro tento účel. Určit spektrum
  mu trvalo dva až tři dny, zatímco jiné teleskopy to zvládly za několik hodin. 
  
  Zatímco Humason měřil červené posuny mlhovin, Hubble se věnoval určení jejich vzdáleností.
  Sledoval jas jednotlivých mlhovin a porovnával výsledky. Když to srovnal s mlhovinou v Andromedě,
  jejíž vzdálenost znal, dostal hrubý odhad jejich vzdáleností. Potom vzdálenosti zkombinoval se
  Slipherovými a Humasonovými měřeními červeného posunu a zkoumal, jestli platí lineární závislost,
  jak předpovídal de Sitterův jev. 
  
  V lednu 1929 Humason a Hubble měli určený červené posuny pro 46 mlhovin. Červený posun 21 z nich
  změřil již dříve Slipher. Hubble výsledky zanesl do grafu, v němž na vodorovné ose byly
  vzdálenosti a na svislé rychlosti jednotlivých mlhovin určené z červeného posunu. Odchylky od
  přímky byly stále velké, ale s lineární závislostí souhlasily mnohem lépe než průkopnické pokusy
  Lundmarka a Lemaîtra. A hlavně byl jasně vidět celkový trend, že čím je galaxie vzdálenější, tím
  je červený posun větší. 
  
  Hubble publikoval graf v krátkém článku „Vztah mezi vzdáleností a radiální rychlostí
  mimogalaktických mlhovin“; Humason zde nebyl uveden jako spoluautor. O Lundmarkovi, který měl
  vlastně prvenství, se sice zmiňuje, ale silně nadceňuje důležitost svých měření. V závěrečném
  odstavci napsal: „Vynikajícím znakem těchto výsledků je skutečnost, že vztah červený posun -
  vzdálenost může být důsledkem de Sitterova jevu a tak získaná numerická data mohou být použita v
  diskusi o celkovém zakřivení vesmíru.“ Humason publikoval svá měření červeného posunu ve skromném
  krátkém článku, odeslaném týž den jako článek Hubbleův, a jeho práce obsahovala data o mlhovinách
  dvakrát vzdálenějších, než uvažoval ve svém článku Hubble. I tyto vzdálenější mlhoviny se zdály
  vyhovovat vztahu, který nalezl Hubble. De Sitterův jev byl na světě. 
  
  I když práce Lundmarka a Lemaîtra vznikly dříve, až Hubbleův objev lineárního vztahu mezi
  vzdáleností a červeným posunem byl katalyzátorem, který odstartoval moderní kosmologii. V letech,
  která následovala po Hubbleově vlivném článku z roku 1929, myšlenky Einsteina, de Sittera,
  Friedmanna a Lemaîtra, jež v předchozích deseti letech dozrávaly, se konečně spojily do
  jednoduchého obrazu. I když zárodky důkazu vzdalování galaxií byly již v předběžných analýzách
  Sliphera, Lundmarka a Lemaîtra, byly to až články Hubblea a Humasona, které astronomy přesvědčily
  o existenci de Sitterova efektu. 
  
  Rok po předložení Hubbleova článku zveřejnil Eddington rozbor de Sitterova jevu a Hubbleových
  pozorování v astronomickém časopise The Observatory, v témže časopise, kde během první světové
  války uveřejňoval své pacifistické výzvy. Abbé Lemaître, který měl nyní trvalé místo na univerzitě
  v Lovani, článek četl a byl ohromen. Nebyla zde jediná zmínka o jeho pracích - jeho mnohem
  jednodušší model rozpínajícího se vesmíru byl zapomenut. Lemaître okamžitě poslal Eddingtonovi
  dopis, kde připomínal svou práci z roku 1927, ve které ukázal, že existují řešení Einsteinových
  rovnic popisující rozpínající se vesmír. Na konci dopisu připojil: „Posílám vám několik kopií
  tohoto článku. Možná najdete příležitost, jak je poslat de Sitterovi. Já jsem mu článek svého času
  poslal, ale asi ho nečetl.“ Eddington se cítil zahanbený. Jeho „brilantní“ a „jasnozřivý“student
  ho průběžně informoval o svých pracích v relativitě, ale on si jeho článků nevšímal a na jeho
  práci zapomněl. Hned začal propagovat Lemaîtrův pohled na vesmír a přesvědčil de Sittera, aby
  zavrhl svůj vlastní model a přijal model Lemaîtrův. Nyní bylo na Einsteinovi, aby zaujal nové
  stanovisko k rozpínajícímu se vesmíru. 
  
  Einstein žil během posledních let v záři reflektorů a to ho rozptylovalo od sledování divokého
  vývoje v kosmologii, k němuž docházelo díky pracím Friedmannovým, Lemaîtrovým a pozorováním
  vzdalujících se galaxií. Ale v roce 1930 si uvědomil i on, že ve vzduchu je něco velkého. Během
  návštěvy v Cambridge, kde pobýval u Eddingtona a jeho sestry, se nakazil Eddingtonovým nadšením
  pro Hubbleovy výsledky a Lemaîtrův vesmír. Na jedné ze svých mnoha cest se zastavil v Kalifornii a
  na Mount Wilson se setkal s Hubblem. Vedli spolu velmi obtížnou diskusi o nové vizi vesmíru.
  Problém byl v tom, že Einstein stále ještě nemluvil anglicky plynně a Hubble neuměl německy vůbec,
  ale oba se shodli, že představa rozpínajícího se vesmíru je postupně přijímána většinou astronomů
  i fyziků. A tak na další cestě, tentokráte do Leidenu, se Einstein setkal s de Sitterem a oba
  přijali novou kosmologii, která se vynořila z Einsteinovy teorie, a dokonce navrhli svou verzi
  rozpínajícího se vesmíru. Tím zavrhli kosmologickou konstantu, kterou v roce 1917 Einstein do
  svých rovnic dodatečně přidal jen proto, aby dovolovaly statický vesmír. 
  
  Potom, co objevil na základě Einsteinových rovnic rozpínající se vesmír, chtěl Lemaître postoupit
  ještě dále. Uvědomil si, že obecná teorie relativity něco říká o počátku času. Jakmile totiž
  přijmete, že se vesmír rozpíná, napadne vás hned samozřejmá otázka, jak a kdy toto rozpínání
  začalo. Když sledujete vesmír zpět v čase, dostanete se k okamžiku, kdy byl celý vesmír stažen do
  jediného bodu. To je bizarní situace, zcela nepodobná čemukoli, co spatřujeme v přirozeném světě
  kolem sebe. Nicméně je to něco, co má ve Friedmannových a Lemaîtrových modelech své místo:
  počáteční okamžik, kdy vesmír začal existovat. 
  
  A tak Lemaître přišel s velmi radikální myšlenkou o tom, jak vesmír započal. Měl to být skutečný
  počátek všeho. Podle jeho představy se vesmír vynořil z jediného jsoucna - primordiálního atomu,
  či „primordiálního vejce“, jak to rád nazýval. Tento atom zplodil vše, co náš vesmír dnes
  naplňuje. Atom se měl rozpadat podle zákonů kvantové mechaniky, které vědci v té době jen tak tak
  začínali rozumět, analogicky radioaktivnímu rozpadu částic, který byl pozorován v laboratořích.
  Potomstvo prvotního atomu se opět rozpadalo na další částice a tak dále a dále. 
  
  Byl to jednoduchý, spekulativní, téměř biblický model, ale Lemaître se důsledně snažil od svého
  návrhu oddělit náboženství. Jako kněz byl více než ostatní vědci v nebezpečí, že zanáší víru tam,
  kde by měla mít místo jen vědecká hypotéza. V časopise Nature zveřejnil krátký článek s názvem
  „Počátek světa z hlediska kvantové teorie“. Název článku hovořil za vše. Nebylo to o božím zásahu
  či teologické konstrukci. Byl to praktický důsledek chladných, nestranných přírodních zákonů. Svůj
  názor sumarizoval takto: „Jestliže svět povstal z jediného kvanta, na samém počátku neměly smysl
  ani pojmy prostoru a času. Ty získaly dobrý smysl teprve tehdy, když se původní kvantum rozmnožilo
  v dostatečný počet nových kvant. Je-li moje představa správná, pak počátek světa se odehrál
  nepatrně před počátkem prostoru a času.“
  
  V lednu 1931 sdělil Eddington ve své předsednické řeči pro Britskou matematickou společnost svůj
  názor na Lemaîtrovu nejnovější myšlenku: „Představa počátku nynějšího řádu přírody mě odpuzuje.“
  
  Eddington propagoval Lemaîtrovy práce o rozpínání vesmíru a přesvědčil Einsteina, aby představu
  statického vesmíru opustil. Právě Eddingtonovi vděčil Lemaître za svůj mezinárodní věhlas. Ale
  nejnovější Lemaîtrova myšlenka byla pro Eddingtonův žaludek přece jen příliš. Posouvala
  Einsteinovu teorii prostoru a času za její platné hranice, alespoň podle Eddingtonova názoru, a
  tak se proti ní veřejně postavil. 
  
  Podobně jako Einstein dlouho odmítal rozpínání prostoru ve Friedmannových a Lemaîtrových modelech,
  i Eddington odmítal přijmout, co mu říkala matematika. Místo toho navrhl jiné řešení. Hubbleovy a
  Humasonovy výsledky naznačovaly, že Einsteinův vesmír je třeba zavrhnout, jenže možná ne úplně.
  Lemaître ve snaze najít všechna možná řešení gravitačních rovnic odhalil katastrofickou vlastnost
  Einsteinova statického vesmíru - tento vesmír byl nestabilní. Jestliže se do něj vložilo jen
  nepatrně více hmoty, jediná galaxie, hvězda, či dokonce jen jediný atom, začal se hroutit do bodu.
  Podobně, když se něco hmoty odebralo, začal se nezadržitelně rozpínat a na konec se choval jako
  jeden z rozpínajících se vesmírů Friedmanna či Lemaîtra. Ale tato katastrofická vlastnost se
  Eddingtonovi hodila, právě touto nestabilitou chtěl vysvětlit expanzi vesmíru. 
  
  Eddingtonův návrh nebyl zcela dokončený a vypadal záplatovaně, byl ale uvěřitelný a jednoduchý.
  Vesmír začal tak, jak to navrhl Einstein, to znamená jako statický a neměnný. Slovo „začal“ se sem
  však vlastně nehodí - vesmír mohl v tomto stavu být nekonečně dlouho, až do té doby, kdy hmota v
  něm začala tvořit hrudky - proč, to zbývalo vysvětlit. Tyto hrudky byly zárodky galaxií a hvězd a
  volný prostor mezi nimi vyvolal v Einsteinově vesmíru nestabilitu, takže se začal rozpínat. Vesmír
  bez časových změn elegantně přešel ve vesmír expandující. 
  
  Zatímco Eddingtona Lemaîtrův radikální návrh o původu vesmíru nepřesvědčil, s Einsteinem to bylo
  jiné. V zimě roku 1933 Einstein i Lemaître cestovali po Spojených státech a setkali se v příjemném
  kampusu Kalifornského technologického ústavu v Pasadeně, kde měl abbé přednést dvě přednášky.
  Jejich setkání na Solvayské konferenci v roce 1927 neproběhlo příliš dobře - Einstein tam odmítl
  Lemaîtrovy modely. Tentokrát však to bylo jiné - Lemaître zde vystupoval jako vůdčí osobnost nové
  kosmologie. Během svého pobytu se Einstein s Lemaîtrem často procházeli po zahradách Athenaea,
  ponořeni do konverzace. Podle Los Angeles Times měli oba muži „vážný výraz, který svědčil o tom,
  že debatují o současném stavu kosmických záležitostí“. Bylo stylové, že Einstein naslouchal
  Lemaîtrově přednášce na tom místě, kde bylo vzdalování galaxií objeveno. Na konci jednoho z
  Lemaîtrových seminářů povstal a prohlásil: „To je to nejkrásnější a nejuspokojivější vysvětlení
  stvoření, které jsem kdy slyšel.“
  
  Po více jak deseti letech, kdy ho jeho intuice sváděla na scestí, Einstein konečně uviděl světlo.
  Byl to zajímavý obrat událostí. Tvůrce obecné teorie relativity nedokázal unést to, co jeho teorie
  říkala o vesmíru, a proto ji upravil. Jen díky tomu, že Friedmann a Lemaître dokázali uchopit
  teorii relativity v celé její slávě, se jim povedlo předpovědět vyvíjející se, expandující vesmír,
  který byl potvrzen pozorováním. Einsteinovo ocenění povzneslo Lemaîtra v očích populárního tisku.
  Tak jako se již dříve stal populární hvězdou Einstein, Lemaîtra teď tisk nazýval „vůdčím světovým
  kosmologem“ a za jednoho ze zakladatelů moderní kosmologie jej pokládali i v odborných kruzích.
  Jeho myšlenky i myšlenky Alexandera Friedmanna byly podstatné pro revoluci v kosmologii, ke které
  došlo o třicet let později.

\section{Kolabující hvězdy}\label{feyIchIIIsecV} 
  Obecná teorie relativity nebyla hlavním předmětem zájmu Roberta Oppenheimera. Věřil jí tak jako
  skoro každý rozumný fyzik ve třicátých letech dvacátého století, nedomníval se však, že je nějak
  obzvláště důležitá pro nejpalčivější problémy fyziky té doby. O to zvláštnější je, že právě on
  objevil jednu z nejexotičtějších předpovědí obecné teorie relativity, totiž jak se v přírodě
  vytvoří černá díra. 
  
  Oppenheimerovým hlavním zájmem byla druhá velká teorie první poloviny dvacátého století, kvantová
  mechanika. V té si vysloužil první úspěch a získal věhlas. Studoval ji v Evropě a nakonec založil
  jednu z vůdčích skupin kvantové fyziky ve Spojených státech, která pracovala na Kalifornské
  univerzitě v Berkeley. V jistém smyslu to byl právě rozvoj kvantové teorie, který odsoudil
  Einsteinovu teorii obecné relativity k určité stagnaci a izolaci, a způsobili to právě lidé jako
  Oppenheimer. Ale v roce 1939 se snažil pochopit se svým studentem Hartlandem Snyderem, co se stane
  na konci životního cyklu těžkých hvězd. Oppenheimer nalezl těžko pochopitelné podivné řešení
  rovnic obecné teorie relativity, i když to bylo řešení, které se dalo posledních dvacet pět let
  tušit. Ukázal totiž, že je-li hvězda velká a dostatečně hustá, pak zkolabuje a zmizí z našeho
  dohledu. Jak to popsal on sám: „Hvězda se snaží odříznout od veškeré komunikace se vzdáleným
  pozorovatelem; pro něho zůstane pozorovatelné jen její gravitační pole.“Jako by se kolem
  kolabujícího balíku hmoty a energie vytvořila tajuplná opona, která jej skrývá před vnějším
  světem, a prostoročas jako by se zabalil do nepředstavitelně utaženého uzlu. Zpoza opony nemůže
  uniknout vůbec nic, dokonce ani světlo. Oppenheimerovo řešení byla nová podivnost, která se
  vyklubala z Einsteinových rovnic a kterou mnozí nedokázali spolknout. 
  
  Skoro čtvrt století před tím, než Oppenheimer a Snyder našli své řešení, poslal německý astronom
  Karl Schwarzschild Einsteinovi dopis, v jehož závěru psal: „Jak vidíte, válka se ke mně chová
  laskavě, dovoluje mi, abych se vydal do vaší země idejí, i když dunění děl je rozhodně v pozemské
  vzdálenosti.“ Bylo to v prosinci 1915 a Schwarzschild psal ze zákopů na východní frontě. Do armády
  se přihlásil dobrovolně hned po propuknutí války v roce 1914, i když jako ředitel postupimské
  observatoře narukovat nemusel. Jak ale o něm později řekl Eddington, „Schwarzschildovo zaměření
  bylo praktičtější“. Podobně jako Friedmann, i on ve vojenské službě využíval své fyzikální
  znalosti, dokonce poslal berlínské akademii článek „Účinek větru a hustoty vzduchu na dráhu
  projektilu“. 
  
  Když byl v Rusku, dostal poslední číslo Sitzungsberichte der Preussischen Akademie der
  Wissenschaften (Zprávy ze zasedání Pruské akademie věd). Zde nalezl Einsteinův krátký, ale
  fascinující výklad jeho nové obecné teorie relativity. Dal se do práce na Einsteinových rovnicích,
  snažil se najít jejich řešení v té nejjednodušší a přitom astronomicky nejzajímavější situaci,
  která ho napadla. Na rozdíl od Friedmanna a Lemaîtra, kteří o něco později hledali řešení pro
  vesmír jako celek, studoval něco skromnějšího - jak vypadá prostoročas kolem sféricky symetrického
  objektu, jakým je hvězda nebo planeta. 
  
  Při práci s tak složitými rovnicemi, jako jsou Einsteinovy, je třeba hledat nějaká zjednodušení.
  Schwarzschild se zaměřil na statické, v čase se nevyvíjející řešení, jaké se dalo očekávat kolem
  hvězdy v rovnováze. Navíc chtěl takové řešení, které by mělo sférickou symetrii, to znamená, že
  všechny veličiny by závisely pouze na vzdálenosti od středu hvězdy. 
  
  Schwarzschildovo řešení bylo neobyčejně jednoduché, dalo se zapsat během okamžiku. Do určité míry
  bylo dokonce snadno srozumitelné. Když se totiž pozorovatel postavil hodně daleko od středu
  hvězdy, prostoročas popsaný tímto řešením působil na částice stejně, jak předpovídal zákon, jenž
  před staletími stanovil Isaac Newton - gravitační síla byla úměrná hmotnosti centrálního tělesa a
  ubývala s druhou mocninou vzdálenosti. V menší vzdálenosti se Schwarzschildovo řešení od Newtona
  sice lišilo, ale jen velmi málo - právě jen tak, aby umělo přesně vysvětlit posun perihelia
  Merkura, tedy jev, jehož vysvětlení bylo velikým úspěchem Einsteinova snažení. 
  
  Když jste se však dostali k hvězdě opravdu velmi blízko, začaly se dít podivné věci. Kdyby hvězda
  byla velmi malá, ale přitom velmi hmotná, to znamená velmi hutná, byla by obklopena sférickou
  plochou clonící všechno pod ní - toutéž plochou, kterou o mnoho let později nalezli Oppenheimer se
  Snyderem. Tato sféra by měla devastující účinek na vše, co by se dostalo pod ni - už by se to
  nikdy nemohlo dostat ven, cesta pod tuto sféru by byla cestou bez návratu. Abyste se totiž mohli
  dostat zpátky, museli byste se pohybovat rychlostí větší než rychlost světla, což bylo podle
  Einsteinovy teorie nemožné. Schwarzschild objevil to, co bylo o více než půl století později
  nazváno černá díra. 
  
  Schwarzschild rychle sepsal své výsledky a poslal je dopisem Einsteinovi s prosbou, aby je
  přednesl na zasedání Pruské akademie věd. Einstein souhlasil a ve své odpovědi uvedl: „Nečekal
  jsem, že přesné řešení půjde formulovat tak jednoduchým způsobem.“ Koncem ledna 1916 představil
  Einstein Schwarzschildovo řešení světu. 
  
  Schwarzschild sám se už k podrobnějšímu zkoumání svého řešení nikdy nedostal, tím méně k výpočtům
  analogickým práci Oppenheimera a Snydera. Po několik dalších měsíců trpěl pemphigem, puchýřnatým
  autoimunitním onemocněním kůže. Tělo ho zradilo a on zemřel v květnu 1916. 
  
  Einstein a jeho následovníci Schwarzschildovo řešení rychle přijali. Bylo jednoduché, dobře se s
  ním pracovalo a bylo perfektní pro vytváření fyzikálních předpovědí. Snadno se z něho dal odvodit
  například pohyb planety kolem Slunce a posun perihelia Merkura. Přesně předpovědělo i hodnotu
  ohybu světelného paprsku v poli Slunce, kterou pak ověřil Eddington na Príncipe. Schwarzschildovo
  řešení sloužilo velmi dobře a jeho důsledky podporovaly obecnou teorii relativity, byla tu však
  záhada spojená s podivnou sférou kolem velmi hustých a hmotných hvězd, jež měla polykat vše, co
  skrze ni prošlo. 
  
  Nedalo se popřít, že řešení záhadnou sféru obsahovalo a že se jednalo o korektní řešení
  Einsteinových rovnic. Existovalo ale i v přírodě? 
  
  Ve dvacátých letech dvacátého století Arthur Eddington obrátil svou pozornost ke tvoření a vývoji
  hvězd. Na základě fyzikálních zákonů popsaných matematicky korektními rovnicemi chtěl plně
  charakterizovat jejich strukturu. Napsal: „Když na základě matematické analýzy porozumíme řešení …
  získáme znalost přizpůsobenou fyzikálnímu problému.“ Se správnou matematikou v ruce je pak řešení
  problému jednoduše řešením rovnic, tak jak tomu je v obecné teorii relativity. V roce 1926 vydal
  Eddington knihu Vnitřní stavba hvězd, která se okamžitě stala biblí stelární astrofyziky.
  Eddington nebyl jen světovou autoritou v oblasti obecné relativity - byl i vůdčí osobností ve
  fyzice hvězd. 
  
  Až do té doby hvězdy představovaly záhadu. Především nikdo nevěděl, odkud se bere obrovská
  energie, kterou hvězdy vyzařují. A právě Eddington přišel s věrohodným vysvětlením, co může být
  hvězdným „palivem“, zdrojem energie. Abychom jeho myšlence porozuměli, musíme se podrobněji
  podívat na ten nejjednodušší atom - atom vodíku. Ten je tvořen dvěma částicemi, kladně nabitým
  protonem a záporně nabitým elektronem. Elektron je vázán k protonu elektromagnetickou silou;
  kladný a záporný náboj se navzájem přitahují. Proton je téměř 2000krát těžší než elektron, takže
  jeho hmotnost představuje prakticky celou hmotnost atomu vodíku. 
  
  Atom helia sestává ze dvou protonů a dvou elektronů, jeho jádro však obsahuje ještě dvě elektricky
  neutrální částice zvané neutrony, prakticky stejně hmotné jako protony. Tedy jednoduchý model
  atomu helia vypadá tak, že kolem jádra tvořeného dvěma protony a dvěma neutrony krouží dva
  elektrony. Protože skoro veškerá hmotnost atomu helia je hmotností jádra a proton má skoro stejnou
  hmotnost jako neutron, očekávali bychom, že hmotnost helia bude čtyřnásobkem hmotnosti vodíku. Ale
  není tomu tak, helium je o něco lehčí, asi o 0,7 \% předpokládané hmotnosti. Nějaká hmotnost heliu
  chybí. A kde chybí hmotnost, chybí podle Einsteinovy speciální teorie relativity určitá energie.
  To byl pro Eddingtona pokyn. 
  
  Vyslovil domněnku, že zdrojem energie vyzařované hvězdami je právě přeměna vodíku na helium. V
  jádru hvězdy, kde zuří žhavé peklo, jsou vodíkové atomy stlačovány k sobě. Některé protony se
  radioaktivním rozpadem změní v neutrony a protony s neutrony pak vytvářejí heliová jádra. Při
  tomto procesu se při vzniku každého atomu helia vyzáří určité malé množství energie, ale při
  obrovském počtu atomů tvořících hvězdu to dává energii odpovídající energii vyzařované hvězdou.
  Jestliže většina Slunce byla na počátku tvořena vodíkem, uplyne asi 9 miliard let, než se přeměna
  vodíku v helium dokončí. Víme, že Země je stará asi 4,5 miliardy let, takže tato čísla nejsou v
  rozporu. 
  
  Eddington ve své knize položil pevné základy stelární astrofyziky. Nejen že vysvětlil, odkud se
  bere energie, kterou hvězdy vyzařují, ale vyložil také, proč se nezhroutí pod vlivem vlastní
  gravitační přitažlivosti: rovnováhu proti tahu gravitace udržuje záření, které se dere na povrch.
  Hvězdy se staly perfektním systémem, jenž se dal popsat pomocí matematických rovnic. Ale Vnitřní
  stavba hvězd přece jen nevysvětlovala všechno. Eddington uměl dobře popsat život hvězd pomocí své
  matematické pyrotechniky, neřešil však jejich smrt. Logická úvaha mu samozřejmě říkala, že palivo
  hvězdy se jednou vyčerpá a tím zmizí záření, které bránilo jejímu zhroucení pod vlivem vlastní
  gravitace. Ve své knize psal: „Hvězda se zřejmě ocitne v nepříjemných nesnázích poté, co jí dojde
  zásoba subatomární energie. … Je to zvláštní problém a dá se vyslovit řada hypotéz o tom, co se
  skutečně stane.“ Samozřejmě jednou z podivuhodných možností bylo to, co nabízelo Schwarzschildovo
  řešení. Jak Eddington psal, „Gravitační síla může být tak veliká, že zabrání i světlu vzdálit se,
  takže bude padat zpátky k hvězdě jako kámen vržený ze Země“. To však bylo pro Eddingtona příliš
  radikální, pokládal to jen za matematickou kuriozitu. Říkal, že „pokud nalezneme výsledek, aniž
  bychom mu porozuměli, tvrzení, které nepředvídatelně vyskočí z bludiště matematických rovnic,
  nemáme žádný základ pro to, že se to bude v přírodě skutečně realizovat“.
  
  Když ale chceme zůstat realisty, co se může stát po vyhoření nukleárního paliva? V roce 1914 byl
  pozorován náznak, jak by to mohlo na hřbitově hvězd vypadat. Při pozorování Síria, nejjasnější
  hvězdy na obloze, téměř třicetkrát jasnější než Slunce, si astronomové povšimli jeho málo jasného
  podivného průvodce, který kolem Siria obíhal. Dostal jméno Sirius B a měl velice zvláštní
  vlastnosti. Jeho hmotnost byla zhruba stejná jako hmotnost Slunce, zatímco jeho poloměr byl mnohem
  menší, než je poloměr Země. To tedy znamenalo, že je neobyčejně hutný. Na počátku dvacátých let
  minulého století dostal jméno bílý trpaslík a mezi osazenstvem hvězdné zoo představoval záhadu.
  Zdálo se ale, že by to mohl být konečný produkt životního cyklu hvězdy. Klíč k vysvětlení
  struktury a osudu bílých trpaslíků se nacházel v nově objevené kvantové fyzice. 
  
  Kvantová fyzika rozdělila přírodu na ty nejmenší stavební kameny a znovu je složila podivným
  způsobem. Vznikla v souvislosti s bizarním jevem objeveným v devatenáctém století, kdy fyzikové
  zjistili, že chemické prvky a sloučeniny absorbují a emitují světlo podle zvláštního pravidla -
  pohlcují a vyzařují jen některé vlnové délky. Výsledkem je „čárový kód“, o kterém jsme mluvili v
  souvislosti s odhalením červeného posunu Vesto Slipherem a Miltonem Humasonem. Klasická
  newtonovská fyzika spojená s Maxwellovou elektrodynamikou, jež v té době vládla, neuměla tento
  podivný jev vysvětlit. Během svého „zázračného roku“ 1905 se Einstein zabýval jiným experimentálně
  zjištěným fenoménem - fotoelektrickým jevem. Jestliže světlo dopadá na kov, jeho atomy je pohlcují
  a občas z kovu vyskočí elektron. Objevitel tohoto jevu Philipp Lenard jej popsal takto:
  „Vystavíme-li kovovou desku ultrafialovému světlu, okolní vzduch získává záporný elektrický náboj.
  “Mohlo by se zdát, že k vyvolání tohoto jevu stačí kov vystavit dostatečnému množství světla, ale
  tak tomu není. Elektron se emituje jen tehdy, má-li světlo dostatečnou frekvenci. Einstein
  vyslovil hypotézu, že světlo se skládá z balíčků energie, kvant, podobně jako se hmota skládá z
  elementárních částic. A k fotoelektrickému jevu dojde jen tehdy, když světelné energetické kvantum
  má dostatečnou velikost, přičemž energie kvanta je určena frekvencí světla. Einstein tyto částice
  světla nazval „světelná kvanta“, později získaly název fotony. 
  
  Jak se na přelomu devatenáctého a dvacátého století zlepšovala experimentální technika, příroda se
  přestala jevit spojitá a hladká, najednou byla zrnitá a diskrétní. Jinými slovy, vypadala
  kvantovaná. Na začátku dvacátého století se počal vynořovat provizorní model přírody v těch
  nejmenších měřítkách, pro nějž platila zatím trochu nesourodá pravidla, která říkala, jak atomy
  interagují se světlem. Einstein sice k vývoji nové vědy výrazně přispěl, celkově ji však sledoval
  s určitou nedůvěrou. Nová pravidla působila neohrabaně a nezapadala do elegantního matematického
  obrazu, který přinesly jeho principy relativity. 
  
  Zákony kvantové fyziky získaly konečnou přijatelnou podobu v roce 1927. Dva fyzici, Werner
  Heisenberg a Erwin Schrödinger, přišli nezávisle na sobě s novou teorií, která uměla konzistentně
  vyložit kvantové vlastnosti atomů. A podobně jako Einstein, když pracoval na obecné teorii
  relativity, museli i tito fyzici odít své verze teorie do nové matematiky. Heisenberg svou teorii
  zapisoval pomocí tabulek čísel zvaných matice, s nimiž se muselo nakládat podle určitých pravidel.
  Na rozdíl od násobení dvou čísel, když vynásobíte matici B maticí A, dostanete obecně něco jiného,
  než když vynásobíte matici A maticí B. To někdy vede ke zvláštním výsledkům. Schrödinger naopak
  popisoval realitu, tedy atomy, jádra a elektrony, ze kterých se skládá všechno kolem nás, pomocí
  materiálních vln. To jsou velmi exotické objekty, řídící se rovnicí po něm nazvanou. Jak se
  později ukázalo, Heisenbergův i Schrödingerův přístup jsou navzájem ekvivalentní. 
  
  Nejznámějším výsledkem nové kvantové teorie byl princip neurčitosti. V klasické newtonovské fyzice
  se objekt pod vlivem vnějších sil pohyboval předpověditelným způsobem. Jestliže jste znali přesné
  polohy a zároveň rychlosti jednotlivých členů nějakého systému, mohli jste přesně předpovědět jeho
  vývoj, který byl jednoznačně určen. Ale podle nové kvantové teorie bylo nemožné znát zároveň
  přesně polohy a rychlosti nějakého objektu. Jestliže si opravdu pečlivý a tvrdohlavý
  experimentátor umíní, že zjistí naprosto přesně polohu nějaké částice, a dokáže-li to, pak nebude
  mít sebemenší představu, jakou bude mít částice rychlost. Můžete si představit, že je to podobné
  jako s rozvztekleným zvířetem uvězněným v kleci - čím více se snažíte omezit jeho pohyby, tím více
  se rozzlobí a bude útočit na stěny svého vězení. Když je zavřete do opravdu malé klece, pak jeho
  nárazy na stěny budou velmi silné. Kvantová fyzika zanesla neurčitost a nahodilost do samého jádra
  fyzikálního popisu. A právě nahodilost mohla vysvětlit problém bílých trpaslíků. 
  
  Subrahmanyan Chandrasekhar bezmezně toužil dělat velké věci. Chandra, jak se mu záhy začalo říkat,
  se narodil v bohaté bráhmanské rodině a vždy byl zaujatým a pilným studentem. Vynikal v matematice
  a byl velmi zručný při provádění složitých výpočtů. Během svých studií na univerzitě v Madrasu se
  seznámil s novými myšlenkami, které přicházely z Evropy, a byl ohromen prací velikých vědců, kteří
  budovali novou fyziku dvacátého století. Od mladého věku se s horečným nadšením snažil proniknout
  na bojiště moderní fyziky. Později vzpomínal: „Motivací mé práce bylo v začátcích i to, že jsem
  chtěl světu ukázat, co všechno dokáže Ind.“
  
  Chandra byl uchvácen novou kvantovou fyzikou. Přečetl všechny nové učebnice, ke kterým se dostal,
  a mezi nimi byla i Eddingtonova právě vyšlá kniha Vnitřní stavba hvězd. Opravdu mimořádně ho
  zaujalo dílo německého fyzika Arnolda Sommerfelda Stavba atomů a spektrální čáry. Jím inspirován,
  začal studovat statistické vlastnosti kvantových systémů a jejich vzájemnou interakci. Výsledky
  shrnul do článků, které ho proslavily. Jeden z jeho prvních článku vyšel v prestižním časopise
  Proceedings of the Royal Society v době, kdy Chandrasekharovi ještě nebylo osmnáct let. Bylo
  jasné, že má schopnosti na to, aby se podílel na budování nové kvantové fyziky, a rozhodl se proto
  vydat se do Evropy. Za cíl si zvolil Anglii a jako první působiště Cambridge, kde chtěl získat
  titul PhD. 
  
  Během dlouhé plavby z Indie učinil svůj ohromující objev, který ovlivnil celý jeho další život.
  Byl posedlý prací, a proto chtěl cestu využít ke studiu. Pustil se do článku Ralpha Fowlera,
  jednoho z Eddingtonových kolegů z Cambridge, ve kterém se jeho autor snažil vyřešit problém bílých
  trpaslíků a zdálo se, že se mu to povedlo. Fowler aplikoval na astrofyziku dva principy kvantové
  teorie. Prvním byl Heisenbergův princip neurčitosti, podle kterého nešlo přišpendlit částici na
  určité místo a zároveň přesně stanovit její pohybový stav, to znamená její rychlost. Tím druhým
  byl vylučovací princip, podle kterého dva elektrony či protony nemohou být v atomu ve stejném
  kvantovém stavu - to nedovolovaly vlastnosti exotické vlnové funkce, kterou postuloval Erwin
  Schrödinger. Jako by mezi elektrony či protony působila jakási síla, která jim zabrání zaujmout
  stejný stav. 
  
  Tyto dva principy se Fowler snažil užít na Sirius B. Argumentoval tak, že materiál tvořící bílého
  trpaslíka je tak hustý, že se na něj můžeme dívat jako na plyn elektronů a protonů silně
  stlačených dohromady. Elektrony jsou mnohem lehčí než protony, proto se mohou volněji pohybovat, a
  létají tedy v plynu mnohem divočeji. Vylučovací princip pak elektronům zakazuje, aby vstupovaly do
  teritoria toho druhého, a jak hustota roste, každý elektron má méně a méně prostoru, ve kterém se
  může pohybovat. A jakmile je omezen prostor, ve kterém se elektrony mohou nacházet, princip
  neurčitosti je nutí do vyšších rychlostí, a to je provázeno růstem tlaku v elektronovém plynu.
  Uvnitř bílého trpaslíka roste kvantový tlak, jenž soupeří s gravitační silou, která se snaží
  hvězdu smrštit. V určitém stavu se kvantový tlak a gravitační přitažlivost mohou přesně vyrovnat a
  hvězda se ocitne v rovnováze. To je případ bílého trpaslíka. Ten už září jen málo, protože v jeho
  centru se vyčerpalo nukleární palivo, ale protože jej drží kvantový tlak, může zůstávat v
  rovnovážném stavu neomezeně dlouho. Fowlerův výklad vyjasnil Eddingtonův problém - zdálo se, že
  hvězdy mohou skončit jako bílí trpaslíci. Tím se měl uzavřít příběh hvězdné evoluce a vyřešit
  mezera, která byla v Eddingtonově Vnitřní stavbě hvězd.
  
  Chandra prostudoval Fowlerovy výsledky a udělal něco velmi jednoduchého - určil hustotu elektronů,
  která se dala v bílém trpaslíkovi předpokládat. Číslo, které mu vyšlo, bylo sice obrovské, ale
  přijatelné, to konečně tvrdil i Fowler ve svém článku. Pak ale udělal to, co Fowler opomněl -
  určil, jak velké skutečně budou elektronové rychlosti. Chandra udělal jednoduchý výpočet a byl
  šokován - elektrony tam musely poletovat rychlostí blízkou rychlosti světla. A zde se Fowlerův
  argument dostal do potíží, úplně totiž ignoroval pravidla speciální teorie relativity, jež jsou
  důležitá, když se objekty pohybují rychlostí blízkou rychlosti světla. Fowler se chybně domníval,
  že elektrony v bílém trpaslíkovi se mohou pohybovat libovolně rychle, dokonce i nadsvětelnou
  rychlostí. 
  
  Chandra se tedy rozhodl Fowlerovu chybu napravit. Neodchyloval se od jeho postupu až do okamžiku,
  kdy by se elektrony ve zkolabované hvězdě začaly pohybovat rychleji než světlo. Když ale byl bílý
  trpaslík tak hutný, že by k tomu docházelo, použil vzorců Einsteinovy speciální teorie relativity,
  takže k překročení rychlosti světla nemohlo dojít. Vyšla mu podivuhodná věc. Jestliže byl bílý
  trpaslík příliš těžký, to znamená i příliš hutný, tlak elektronového plynu už nedokázal zabránit
  hroucení hvězdy, gravitační síla se s ním nemohla dostat do rovnováhy. Jinými slovy, existovala
  určitá maximální hmotnost, kterou mohl bílý trpaslík mít. V prvních výpočtech Chandrovi vycházelo,
  že nemůže být větší než asi 90 \% hmotnosti Slunce. (Později se ukázalo, že správná hodnota je
  vyšší, asi 140 \% hmotnosti Slunce). Jako bílý trpaslík mohla skončit jen hvězda o hmotnosti
  menší, než byla určitá maximální hmotnost. Hmotnější hvězdu by tlak kvantového elektronového plynu
  před nevyhnutelným kolapsem nezadržel, nedokázala by podpírat sebe samu - gravitace by zvítězila.
  
  Po svém příjezdu do Cambridge ukázal Chandra své výsledky Eddingtonovi a Fowlerovi, ti je ale
  ignorovali. Na nalezené nestabilitě bylo něco zneklidňujícího, co ohrožovalo celou stavbu, kterou
  Eddington tak nadějně vybudoval a Fowler doplnil, takže lidé z Cambridge zachovávali opatrnost.
  Během čtyř let Chandra dopracoval svůj argument a jeho důvěra ve správnost výsledků, které
  odvodil, vzrostla. V roce 1933 získal doktorát a ve věku pouhých dvaceti dvou let se stal
  asistentem na Trinity College. V roce 1935 uvedl své výsledky do elegantní formy a měl je přednést
  na jednom z pravidelných měsíčních zasedání Královské astronomické společnosti. 
  
  11. ledna 1935 stál Chandra v Burlingtonském paláci v Londýně před výkvětem astronomů Královské
  astronomické společnosti. Pečlivě a systematicky přednesl své výsledky a rozebral detaily svého
  devatenáctistránkového článku, který se měl objevit prestižním časopise Monthly Notices of the
  Royal Astronomical Society, který vydávala Královská společnost. Svou přednášku zakončil slovy:
  „Hvězda s velikou hmotností nemůže skončit jako bílý trpaslík a nezbývá než přemýšlet o dalších
  možnostech.“Tento podivný výsledek byl založený na matematice a fyzice, kterou všichni uznávali, a
  bylo tedy třeba brát jej vážně. Když Chandra skončil, následoval zdvořilý potlesk a pár otázek.
  Tím to skončilo. 
  
  Prezident společnosti se pak obrátil na Eddingtona a pozval ho na podium, aby hovořil o svém
  vlastním článku s názvem „Relativistická degenerace“. Eddington přednesl svůj krátký,
  patnáctiminutový příspěvek, ve kterém pečlivě probíral Chandrovo tvrzení, že výpočty vyvracejí
  Fowlerovu představu o bílých trpaslících. A nakonec Chandrův neprůstřelný argument sumárně zavrhl.
  Pro Eddingtona byly Chandrovy výsledky „redukcí ad absurdum formule pro relativistickou degeneraci
  “. Byl pevně přesvědčen, že „hvězdu mohou zachránit různé události“ a navíc tvrdil, že „musí
  existovat přírodní zákony, které hvězdě zabrání chovat se absurdním způsobem“. Eddingtonova
  autorita byla taková, že většina posluchačstva Chandrův proslov okamžitě odmítla. Jestliže se
  domnívá Eddington, že je to špatně, tak to musí být špatně. 
  
  Chandra se postavil proti mocnému Eddingtonovi a prohrál. Chtěl sabotovat Eddingtonův krásný
  příběh o tom, jak hvězdy žijí a umírají, a to se Eddingtonovi nelíbilo. Kdyby totiž gravitační
  kolaps nic nezadrželo, muselo by se brát vážně Schwarzschildovo řešení se všemi bizarními
  důsledky. Po letech Chandra řekl: „Eddington si jasně uvědomil, že existence maximální hmoty má za
  následek, že v přírodě musí existovat černé díry. Tento závěr však odmítal přijmout. … Kdyby se s
  tím smířil, mohl předstihnout všechny ostatní o 40 let. Svým způsobem je to škoda“. 
  
  Chandra se do Cambridge vrátil zničený. Jeho střetnutí s Eddingtonem ho poznamenalo na celý život.
  O několik let později získal místo na Yerkesově observatoři v Chicagu. Přestal přemýšlet o bílých
  trpaslících i o tom, co se stane, je-li hmotnost hvězdy opravdu veliká. Povede to k vytvoření
  Schwarzschildova řešení v plném rozsahu, anebo tomu něco zabrání? Tuto otázku zodpověděl právě
  Robert Oppenheimer, o kterém jsme mluvili na začátku této kapitoly. 
  
  J. Robert Oppenheimer byl pevně spjat s kvantovou fyzikou. Byl vychován v bohaté newyorské rodině,
  v bytě, kde se na něj ze stěn dívaly van Goghovy obrazy, a měl to nejlepší vzdělání. Nejdříve
  studoval na Harvardu, v roce 1925 však přešel do Cambridge. Jeho harvardský školitel mu napsal
  doporučení, ve kterém sice uváděl, že „Oppenheimer je určitě handicapován tím, že mu chybí praxe v
  základních fyzikálních laboratorních technikách“, ale k tomu dodával, že „stěží najdete
  slibnějšího studenta“. Oppenheimerův pobyt v Cambridgi však netrval dlouho  a byla to katastrofa.
  Po nervovém zhroucení napadl jednoho z kolegů a navíc přiznal, že jiného se chystal otrávit.
  Rozhodl se z Cambridge odejít a zkusit své štěstí v Göttingen. 
  
  Ve zdejším království Davida Hilberta s velkým úsilím rozvíjeli kvantovou teorii a pro
  Oppenheimera to byla skvělá příležitost, jak se dostat do frontové linie revoluce moderní fyziky.
  Jeho školitelem byl Max Born, jenž se nesmazatelně zapsal do dějin kvantové mechaniky, a
  Oppenheimer s ním během dvou let napsal několik důležitých článků. Například
  Bornova-Oppenheimerova aproximace se stále na univerzitách učí jako metoda výpočtů při zkoumání
  kvantového chování molekul. Oppenheimer získal doktorský titul v roce 1927 a o několik let později
  se vrátil do Spojených států, kde získal akademickou pozici na Kalifornské univerzitě v Berkeley.
  
  Zde rozsvítil jeden z majáků teoretické fyziky ve Spojených státech. Oppie, jak byl přátelsky
  přezdíván, se vyznal ve všem možném, od umění a poezie až po fyziku a plachtění. Bystře a
  neobyčejně rychle se zmocňoval nových představ a přesunoval se od projektu k projektu. Přispíval k
  vývoji v mnoha oblastech fyziky, a i když jeho příspěvky někdy nebyly mimořádně hluboké, vždy byly
  cenné a aktuální. Býval netrpělivý a někdy i nepříjemný, když s někým nesouhlasil nebo nemohl něco
  rychle pochopit, jeho osobní magnetismus a energie z něj však dělaly přirozeného vůdce, který svou
  skupinu skvěle podporoval a inspiroval. Pomalu a pečlivě si vybíral skupinu brilantních a
  nadšených studentů a badatelů, se kterými pak řešil mnoho nových problémů, o kterých se
  diskutovalo v Evropě. Wolfgang Pauli, který si všiml, že Oppenheimer má ve zvyku si při přemýšlení
  mumlat „hm, hm“, přezdíval jeho skupinu „chlapci hm, hm“. Berkeley bylo pro Oppenheimera Kodaní i
  Göttingen. 
  
  A najednou, poté, co se téměř deset let soustřeďoval skoro výhradně na kvantové problémy, ho v
  roce 1938 zaujala Einsteinova obecná teorie relativity. Podobně jako Chandra k ní přistoupil z
  kvantového úhlu pohledu, zajímalo ho, jak by kvantové problémy mohly působit proti gravitačnímu
  smršťování prostoru. 
  
  Každé léto jezdil se svými studenty a spolupracovníky do jižní Kalifornie, kde přijali
  pohostinství v Caltechu ve slunné Pasadeně. Zde mohli hovořit nejen s dalšími fyziky, nýbrž i s
  astronomy, kteří sledovali Hubbleův úspěch a byli svědky Lemaîtrových přednášek o primordiálním
  atomu. Obecná teorie relativity zde byla stále v plné slávě a právě tady Oppenheimer četl článek
  ruského fyzika Lva Davidoviče Landaua o tom, jak by to vypadalo, kdyby jádro hvězdy bylo tvořené
  samými neutrony. 
  
  Landau byl jednou z vůdčích osobností sovětské fyziky. Dospíval v období ruské revoluce a byl to
  skutečně brilantní fyzik, který těžil z vlny modernizace, která proběhla novým Ruskem. Tak jako
  Oppenheimer, i on prožil nějaký čas v zahraničí, studoval ve velkých laboratořích v Evropě a stál
  u zrodu kvantové fyziky. Již v devatenácti letech napsal článek, ve kterém novou fyziku aplikoval
  na chování atomů a molekul. Když se vrátil do Leningradu, bylo mu dvacet tři let. Dostávalo se mu
  obdivu od starších kolegů a byl rychle pohlcen sovětským systémem. 
  
  Landauův zájem o zkoumání složitých systémů z hlediska kvantové fyziky ho dovedl k novému možnému
  zdroji energie ve hvězdách - k neutronům. Neutrony jsou neutrální částice, které se nacházejí v
  jádrech atomů. Tehdy už bylo známo, že spojování protonů a neutronů, nebo naopak jejich odebírání
  z některých atomových jader může vést k uvolňování obrovského množství nukleární energie. Landau
  proto vyslovil domněnku, že kdyby jádra hvězd byla bohatá na neutrony, bylo by možné získat z nich
  dostatek nukleární energie, jež by se přeměňovala ve světlo. Kdyby byly neutrony v jádrech hvězd
  stlačeny na sebe tak, jako jsou v jádrech atomů, mohly by být nadějným palivem. Takhle hutný
  nukleární materiál by byl nepředstavitelně těžký - čajová lžička této látky by měla hmotnost
  několika tun. Kdyby se do takového jádra hvězdy dostal atom z jejího obalu, rozbil by se na
  střepy, jež by byly částečně absorbovány a částečně by se vyzářily jako světlo. Podle Landaua to
  bylo právě neutronové jádro, jež bylo odpovědné za záření hvězd, a díky jemu Slunce svítilo. Na
  problému pracoval dále a snažil se spočítat, jak velké toto jádro musí být, aby bylo stabilní.
  Vyšlo mu, že jeho hmotnost musí být větší než tisícina hmotnosti Slunce. Taková jádra měla být
  skrytá v nitru hvězd, postupně vyhořívat a tím produkovat jejich světlo. 
  
  Když ale Landau sepisoval své myšlenky, zachvátila ho vlna policejních represí, jež se tehdy
  šířila zemí. Dva měsíce po tom, co v Nature zveřejnil svůj článek „Původ hvězdné energie“, byl
  zatčen NKVD. Byl přistižen při vydávání antistalinského pamfletu, ve kterém byl Stalin označen za
  fašistu „s divokou nenávistí ke skutečnému socialismu“, který se „podobá Hitlerovi a Mussolinimu“.
  Pamflet měl být rozšiřován během májového průvodu roku 1938. Landau byl uvězněn ve vězení
  Lubljanka vzápětí poté, co jeho článek v Nature byl v předních sovětských novinách Izvestija
  označen jako důvod k hrdosti na sovětskou fyziku. 
  
  Oppenheimer byl překvapen stručností Landauova článku a jednoduchostí předložené myšlenky. Rozhodl
  se proto Landauovy výpočty zopakovat. Podařilo se mu to za pomoci tří nadaných studentů. Prvním z
  jeho spolupracovníků byl Robert Serber. Společně rozcupovali Landauovu myšlenku, že ve Slunci se
  může skrývat neutronové jádro zahalené horkými plyny, jež podpírají hvězdu. Došli k závěru, že je
  chybná. Oppenheimer a Serber uveřejnili svůj dopis redakci, podobně stručný jako ten Landauův, v
  říjnu 1938 v časopise Physical Review. V té době už Landau strádal v Lubljance. Další krok udělal
  Oppenheimer s jiným studentem, Georgem Volkoffem, se kterým studoval stabilitu neutronového jádra.
  Jejich výpočty, zveřejněné v lednu 1939, jsou krásnou směsí obratného matematického zjednodušení
  Einsteinovy teorie, hlubokého vhledu na základě fyzikální intuice a obtížných výpočtů. Ukázali, že
  neutronová jádra jsou neuvěřitelně nestabilní uspořádání, a proto je rozhodně nelze pokládat za
  zdroj energie hodně velikých hvězd. Prokázali tedy, že Landauova myšlenka nefunguje. 
  
  Na konci svého článku Oppenheimer a Volkoff zdůrazňují, že k porozumění dlouhodobému osudu
  neutronových jader jsou nezbytné „úvahy o nestatických řešeních“. A to souviselo s posledním
  kouskem skládačky, který Oppenheimer prostudoval s dalším studentem, Hartlandem Snyderem.
  Tentokrát se jednalo o problém plně zapadající do obecné teorie relativity, o úvahy, jež
  zasahovaly dále, než se kdo před tím pokusil dostat. Oppenheimer a Snyder spočítali, jak se
  prostor a čas (a také neutronové jádro) budou vyvíjet potom, co se neutronová hvězda stane
  nestabilní. Uplatnili při tom chytrý nápad. Jednoho fiktivního pozorovatele postavili hodně daleko
  od imploze a druhého naopak přímo na povrch hroutícího se neutronového jádra a pak porovnali
  jejich pozorování. Zjistili, že oběma pozorovatelům se situace bude jevit velmi rozdílně. 
  
  Vzdálený pozorovatel uvidí implozi neutronového jádra, které se podle něho bude smršťovat. Čím
  více se však povrch neutronového jádra bude přibližovat k záhadné oponě, kterou objevil
  Schwarzschild, tím se mu kolaps bude zdát pomalejší. Od určitého okamžiku pak bude hroucení z jeho
  hlediska tak pomalé, že pohyb povrchu se prakticky zastaví. Vlnové délky světla unikajícího z
  neutronové hvězdy se budou roztahovat červeným posunem víc a víc, jak se bude povrch blížit
  kritické ploše. Pro vnějšího pozorovatele tedy jako by se hvězda přestala vyvíjet a jako by
  přestávala komunikovat s vnějším světem. Něco velmi podobného popsal o více než deset let dříve i
  Eddington ve své knize Vnitřní stavba hvězd. Napsal: „Hmota vyprodukuje tolik křivosti … že
  prostor kolem hvězdy se uzavře a nás nechá venku (tj. nikde).“
  
  Ale pozorovatel sedící na povrchu kolabující hvězdy zažívá něco zcela jiného. Sleduje
  nezadržitelné hroucení neutronového jádra, vidí, jak povrch tohoto jádra projde kritickým povrchem
  a spadne do vnitřku oblasti ohraničené magickou Schwarzschildovou sférou. A nejen to, ke zkáze
  odsouzená osoba uvidí, jak se vytváří obávaná Schwarzschildova sféra, tato hranice nenávratna,
  zpoza které se nic nemůže dostat ven. Jinými slovy, sedíte-li na správném (či vlastně nesprávném)
  místě, můžete sledovat postupný vývoj celého Schwarzschildova řešení. 
  
  Oppenheimer a Snyder tak dokončili Eddingtonův scénář života hvězd až k jejich smrti, když
  ukázali, že opravdu masivní hvězdy se zhroutí tak, že se kolem nich postupně vytváří úplné
  Schwarzschildovo řešení. To znamenalo, že Schwarzschildovo řešení nemusí být jen matematickou
  kuriozitou, řešením Einsteinových rovnic obecné teorie relativity, které se v přírodě nikde
  nerealizuje. Že tyto podivné útvary mohou skutečně existovat a musí se pojmout do astrofyziky tak
  jako hvězdy, planety či komety. Obecná teorie relativity tak odhalila další zvláštní a
  neočekávanou vlastnost vesmíru. 
  
  Oppenheimerův a Snyderův článek se objevil ve Physical Review 1. září 1939, právě v den, kdy
  nacistická armáda vtrhla do Polska. Ve stejném čísle tohoto časopisu vyšel i článek od dánského
  fyzika Nielse Bohra a amerického fyzika Johna Archibalda Wheelera. I oni se zajímali o neutrony a
  jejich interakci v extrémních podmínkách, záměr jejich článku „Mechanismus jaderného štěpení“ však
  byl naprosto jiný. Bohr a Wheeler se snažili modelovat strukturu velmi těžkých jader, například
  jádra uranu a jeho izotopů. Kdyby se jim to podařilo, mohli zjistit, kolik energie se dá z těchto
  jader uvolnit. 
  
  Během třicátých let minulého století začínali vědci stále lépe rozumět zvěřinci atomových jader.
  Eddington se domníval, že produkce světla ve hvězdách může být důsledkem slučování vodíku na
  helium v nitru hvězd, při kterém se uvolňuje určitá energie. Tomuto procesu se říká jaderná fúze,
  ke které může docházet u lehkých prvků. Na druhé straně se zjistilo, že velmi těžká jádra se mohou
  naopak rozpadat na jádra lehčí, a i tento proces je provázen uvolňováním energie - tomu se říká
  jaderné štěpení. Obecně se přemýšlelo o tom, jak udělat jaderné štěpení efektivním. Bylo by možné
  spustit jaderné štěpení obláčku těžkých atomů dodáním malého množství energie tak, aby rozpad
  každého individuálního atomu způsobil štěpení atomů dalších? Jinými slovy, dá se spustit řetězová
  reakce? 
  
  Článek Bohra a Wheelera ukazoval cestu k jadernému štěpení a pomáhal dalším fyzikům porozumět,
  proč uran 235 a plutonium 239 jsou prvky vhodné pro řetězovou reakci. V následujících letech
  jaderné štěpení dominovalo fyzice a zastiňovalo ostatní oblasti. Armáda skvělých fyziků věnovala
  svůj intelekt snaze zapřáhnout štěpení do práce a Robert Oppenheimer byl jejím důležitým členem.
  
  Během svého pobytu v Berkeley Oppenheimer vytvořil skvělou skupinu mladých badatelů a studentů,
  která byla schopna poprat se s každým problémem. Vybudoval si pověst výborného organizátora a
  vedoucího pracovního týmu a dokázal vyřešit se svou skupinou každou úlohu, která ho opravdu
  zajímala. Jeho kolegové v Berkeley začali v cyklotronu na Berkeley Hills syntetizovat těžké
  nestabilní prvky. V roce 1941 jeden z nich, Glenn Seaborg, objevil plutonium a tím otevřel jednu z
  cest k řetězové reakci. Oppenheimer byl zachvácen vírem událostí a objevů, jež charakterizovaly
  rozvoj nukleární fyziky během druhé světové války. 
  
  Byl také značně rozhořčený. Zprávy o zacházení se Židy v Německu a osudy velké řady skvělých
  fyziků, kteří hledali bezpečí ve Spojených státech, ho šokovaly. Při svém budování skupiny v
  Berkeley se také díval kolem sebe a zapojil se do organizování intelektuální aktivity přílivu
  evropských uprchlíků. I když se nechtěl politicky příliš angažovat, začal politice věnovat
  pozornost. A po začátku války se nukleární štěpení stalo hlavním předmětem jeho zájmu. 
  
  V roce 1942 byl pověřen velením operační skupiny fyziků se základnou v Los Alamos v Novém Mexiku,
  jejímž jediným úkolem bylo uskutečnit a řídit řetězovou reakci nukleárního štěpení. Skupina byla
  vyzbrojena řadou mladých - i ne tak mladých - skvělých mozků, jež patřily například Johnu von
  Neumannovi, Hansi Bethemu, Edwardu Tellerovi či mladému Richardu Feynmanovi. Projekt Manhattan
  namířil své úsilí k výrobě první atomové bomby a během tří let se to zde pracujícím vědcům
  podařilo. Když byly v srpnu 1945 na Hirošimu a Nagasaki shozeny dvě atomové bomby, „Little Boy“ a
  „Fat Man“, bylo zabito na dvě stě tisíc lidí. To byly smutné důsledky Oppenheimerovy schopnosti
  ovládnout nukleární síly v tak krátké době. Úspěch atomové bomby vydobyl kvantům centrální
  postavení na jevišti světové fyziky.
  
  Protože se veškerá pozornost soustředila na válku a nukleární projekt, důležitý Oppenheimerův a
  Snyderův článek o černých dírách nebudil pozornost a byl po řadu let téměř zapomenut. To, co mohlo
  být nadějným zrodem jednoho z největších konceptů obecné teorie relativity, bylo odloženo na
  neurčito. Ani dva staří harcovníci obecné teorie relativity Albert Einstein a Arthur Eddington se
  nepokoušeli zachránit před nezájmem Oppenheimerův a Snyderův objev. 
  
  Eddington stále trval na tom, že Chandrasekharovy výpočty jsou chybné a zavádějící a že bílí
  trpaslíci jsou konečným stadiem evoluce všech hvězd. Nespoutaný kolaps hvězd až do stadia, kdy
  „gravitace se stane tak silnou, že udrží i veškeré záření“, se mu stále zdál absurdní. Chandra
  vzpomínal téměř o půl století později: „Já mohu jen říci, že stále nemohu pochopit, proč právě
  Eddington, který byl jedním z prvních a nejnadšenějších příznivců obecné teorie relativity,
  shledával tak nepřijatelným závěr, že v průběhu přirozeného vývoje hvězd se může vytvořit černá
  díra.“
  
  I Einstein se bránil myšlence, že extrémní rozšíření Schwarzschildova řešení - černá díra - má
  nějaké místo v přírodním světě. Reagoval na tuto možnost stejně, jako na Friedmannovu a Lemaîtrovu
  představu rozpínajícího se vesmíru - správná matematika, ale ohavná fyzika. Více než dvacet let,
  jež uplynulo od objevu Schwarzschildova řešení, odmítal jeho bizarnější důsledky. Pak v roce 1939,
  kdy Oppenheimer a Snyder publikovali svou práci, vydal článek, kde zkoumal, co se stane s oblakem
  částic, jež padají pod účinkem své vlastní gravitace. Tvrdil v něm, že částice se nikdy nezhroutí
  pod sféru o kritickém poloměru. Byl tvrdohlavý a problém formuloval tak, že dostal odpověď, kterou
  si přál: neexistují žádné černé díry. Neměl pravdu a podobně jako Eddington propásl příležitost
  nechat svou teorii obecné relativity zazářit v plné slávě. 
  
  Zájem skoro všech fyziků byl teď někde jinde - přitahovaly je úspěchy kvantové teorie. Většina
  talentovaných mladých fyziků publikovala v této oblasti a snažila se ji dále rozvíjet, hledala
  ještě velkolepější objevy a aplikace. Einsteinova obecná teorie relativity se všemi svými
  podivuhodnými předpověďmi a exotickými výsledky byla odsunuta do pozadí a odsouzena na osamocenou
  cestu pustinou.

\section{Zcela bláznivý Einstein?}\label{feyIchIIIsecVI}
  Poslední léta svého života žil Einstein velice prostě. Budil se pozdě v bílém šindelovém domě na
  Mercer Street v srdci Princetonu v New Jersey, kde žil se svou sestrou Majou. (Jeho žena Elsa
  zemřela v roce 1936, tedy krátce po jejich příjezdu.) V týdnu pak odcházel do Fuld Hall v
  Institutu pro pokročilá studia, kde působil od roku 1933. Během let se stal známou postavou
  princetonského kampusu. I když byl neobyčejně slavný, byl osamělý. 
  
  Einstein byl získán jako jeden z prvních stálých členů Institutu, který byl založen rodinou
  Bambergerů jako útočiště pro brilantní mozky. Byl zde obklopen proslulými kolegy. Pracoval zde
  John von Neumann, matematik, který pracoval na atomové bombě a byl jedním vynálezců moderních
  počítačů. Po nějakou dobu zde byl i matematik Hermann Weyl, jeden z oblíbenců Davida Hilberta,
  který jako jeden z prvních začal rozvíjet obecnou teorii relativity. Byl zde i Kurt Gödel, filosof
  a logik, který způsobil poprask ve filosofii dvacátého století svými větami o neúplnosti. A byl
  zde samozřejmě také Robert Oppenheimer, jenž se v roce 1947 stal ředitelem ústavu. Na chodbách
  Institutu se Einstein mohl setkávat s význačnými návštěvníky, architekty kvantové teorie či
  moderní matematiky. Většinou však vyhledával soukromí své pracovny. 
  
  Po několika hodinách odešel domů na oběd, po kterém si zdříml. Pak se zase vrátil do své pracovny,
  usedl do své oblíbené židle, nohy si zakryl dekou a počítal, psal a zabýval se spoustou dopisů,
  které se na něho z celého světa hrnuly. Dopisy od hlav států a vysokých činitelů prokládaly dopisy
  mladých začínajících vědců a fanoušků. Na konci dne ho čekala večeře, po ní před spaním poslouchal
  rádio a trochu četl. 
  
  Pro člověka, který se těšil takové slávě, to byl nezvykle tichý život. Nebyl zapomenut. Veřejnost
  ho znala stejně dobře jako Charlie Chaplina nebo Marilyn Monroe. Byl členem nesčetných akademií a
  řada měst mu nabídla klíče od bran. Obálka časopisu Time, na níž byl vyobrazen, se stala ikonou
  nové technické éry. Tu a tam si světové celebrity našly cestu k jeho domu. Navštívil jej
  Džaváharlál Néhrú i jeho dcera Indíra Gándhíová, tak jako i premiér Izraele David Ben-Gurion.
  Proslulé Juilliardovo smyčcové kvarteto mu jednou uspořádalo v jeho domě improvizovaný koncert.
  
  Přes všechnu tuto slávu se Einstein choval spíše samotářsky. I když měl několik mladých asistentů,
  kteří s ním pracovali, vždy si nechával nějaký čas na práci individuální. Obecná teorie relativity
  byla pořád jeho pýchou i zdrojem radosti a čas od času se k ní vracel. Studoval Friedmannova,
  Lemaîtrova a Schwarzschildova řešení a snažil se najít řešení složitější, ale snad realističtější.
  Obecná teorie relativity toho stále hodně skrývala, ale čas s ní netrávilo mnoho lidí; většina
  fyziků své úsilí investovala raději do kvantové teorie. I sám Einstein trávil většinu času prací
  na nové ambiciózní teorii, která ho stravovala téměř po tři desetiletí. A byl kvůli ní
  ostrakizován. 
  
  Einstein z padesátých let minulého století nemohl být více nepodobný Einsteinovi z let dvacátých.
  Tehdy po svém obrovském vědeckém úspěchu cestoval, přičemž s ním všude jednali jako s královskou
  osobou, měl populární přednášky, debatoval s jinými fyziky a v diskusích nejdříve odmítal a pak
  zase s nadšením přijímal myšlenku rozpínajícího se vesmíru. Dostalo se mu pocty konstrukcí
  Einsteinovy věže v Postupimi na předměstí Berlína, kde se měla provádět pozorování zkoumající
  důsledky jeho teorie. Byl tou nejváženější osobou i na mezinárodních konferencích, kam byl zván,
  aby vyslovil své mínění o nejnovějším vývoji fyziky. 
  
  Viděl ovšem také crescendo antisemitských nálad ve své německé vlasti a ve třicátých letech poznal
  drsnou realitu provázející vzestup nacistické strany a jejích následovníků. Jeho cestování bylo
  omezenější, množily se výhružky smrtí a přesto, že jeho sláva stále ještě rostla, začal cestovat
  po Evropě za různými závazky se stále většími obavami. 
  
  I když byl částečně chráněn před vírem kolem sebe a pořád byl ceněným národním pokladem, i jeho se
  dotýkaly antisemitské rány pod pás. Už krátce po objevu obecné teorie relativity zahájila kampaň
  proti jeho nové teorii skupina fyziků sdružená v takzvané „Pracovní skupině německých vědců pro
  zachování čistoty vědy“. Tato skupina hanobila teorii relativity jako příklad „masového bludu“ a
  snažila se prokázat, že Einstein je plagiátor. Hnutí nalezlo jako svého mluvčího i fyzika světové
  proslulosti: Philippa Lenarda. 
  
  V Maďarsku narozený Philipp Lenard získal v roce 1905 Nobelovu cenu za výzkum katodových paprsků a
  jeho experimentální práce byly podkladem pro Einsteinovu teorii světelných kvant z roku 1905. Až
  do objevu obecné teorie relativity byly jeho vztahy s Einsteinem zdvořilé. Relativitu ale divoce
  napadal - podle něho byla příliš obskurní a příliš překračovala hranice toho, co pokládal za
  „fyzikální zdravý rozum“. Lenard uveřejnil řadu článků proti Einsteinově teorii v Jahrbuch,
  stejném časopise, kde Einstein v roce 1907 poprvé zveřejnil své myšlenky, jež nakonec vedly k
  obecnému principu relativity. Slovní válka mezi nimi pokračovala a Einstein v ní Lenarda pohrdlivě
  označil za experimentátora, který není schopen jeho myšlenkám porozumět. Lenard se cítil uražen a
  požadoval veřejnou omluvu. Veřejná hádka měla nedobrý dopad jak na Einsteina, tak na Lenarda a
  „antirelativisty“. 
  
  V roce 1933 měl už Einstein Německa dost. Když se dostala k moci nacistická strana, rozhodl se
  přetrhat své svazky s Berlínem. Opustil Německo, které vstupovalo do svých nejtemnějších dnů, a
  jeho teorie se stala cílem útoků hnutí Deutsche Physik (Německé fyziky). Se vzestupem nacistické
  strany se Philippovi Lenardovi daleko lépe hájila jeho kauza, teď dokonce hlasitě podporovaná
  jiným známým fyzikem a laureátem Nobelovy ceny Johannesem Starkem. Podle Lenarda a Starka byla
  Einsteinova teorie, židovská fyzika, zákeřným jedem, který otravoval německou kulturu. Musela být
  vymazána ze systému, měly-li se uskutečnit velké plány nacistické ideologie. 
  
  Léta, jež následovala po Einsteinově odchodu z Německa, byla svědky systematického ničení fyziky v
  německé vědecké komunitě. Přitom německá fyzika byla u většiny velikých změn, které přinesl
  začátek dvacátého století. Po vypuknutí války byli všichni židovští profesoři zbaveni svých
  akademických pozic. Mnozí z těch největších myslitelů moderní fyziky, kteří stáli u vybudování
  kvantové teorie, jako byl například Erwin Schrödinger a Max Born, opustili Německo. Někteří
  dokonce během druhé světové války pracovali na projektu konstrukce spojenecké atomové bomby. 
  
  Když bylo německé fyzikální společenství vážně oslabeno, Johannes Stark se rozhodl stát se vůdcem
  nové árijské fyziky. V cestě mu stál jeden z otců moderní kvantové teorie Werner Heisenberg.
  Heisenberg sice nebyl Žid, to ale Starkovi nevadilo. Do oficiálního časopisu SS napsal článek
  označující Heisenberga za „bílého Žida“, který patří k úpadkové německé fyzice právě tak jako ti,
  co byli vyhnáni. Jenže Stark zde kupodivu narazil. Heisenberg chodil do školy s Heinrichem
  Himmlerem, velitelem SS, a ten ho ochránil před dalším hanobením. Nakonec se Heisenberg stal
  vedoucím německého projektu, jehož cílem byla konstrukce atomové bomby, což konsternovalo řadu
  jeho kolegů, kteří z hitlerovského Německa uprchli. 
  
  Einsteinův odchod odsoudil německý výzkum v oblasti jeho teorie ke stagnaci. V časech Výmarské
  republiky byl sice ctěn jako národní hrdina, během nacistické éry však z německé kultury zmizel. V
  učebnicích zůstaly některé myšlenky, které ho vedly ke speciální teorii relativity, jeho jméno
  však nebylo uváděno. Obecná teorie relativity došla v Německu zase uznání až po druhé světové
  válce. 
  
  Nebylo to však jen Německo, kde se Einsteinovy myšlenky ocitly pod palbou. Teorie relativity a
  kvantová mechanika se dostávaly do sporu i s dialektickým materialismem, integrální součástí
  marxismu, přijatým za oficiální ideologii v Sovětském svazu, tedy zemi stojící na opačné straně
  politického spektra. Dialektický materialismus byl v druhé polovině devatenáctého století vytvořen
  Karlem Marxem na základě myšlenek německých filosofů Georga W. F. Hegela a Ludwiga Feuerbacha. Na
  jeho dalším vývoji se podílel Friedrich Engels a řada dalších následovníků, jmenovitě Vladimír
  Iljič Lenin. Josef Vissarionovič Stalin v článku z roku 1938 „Dialektický a historický
  materialismus“podal svůj výklad dialektického materialismu a posvětil jej jako oficiální sovětskou
  ideologii. V této filosofii byla základem všeho hmota a vše ostatní se z ní vynořovalo. Realita
  byla definována jako způsob chování materiálního světa a jeho vzájemných propojení a hmota
  předcházela jakékoli formě myšlení a idejí. Jak konstatoval Marx ve svém stěžejním díle Kapitál:
  „Ideální svět není nic jiného než materiální svět odrážený lidskou myslí a přeložený do formy
  myšlení.“
  
  Stoupenci marxismu se snažili vše vysvětlovat pomocí různých konstituentů materiálního světa a
  jejich interakcí. Vše v materiálním světě přispívá k vývoji vesmíru a velmi dramatické změny mohou
  být způsobeny nahromaděním změn nepatrných. Existence a vývoj hmoty jsou podle marxistů objektivně
  reálné, nezávislé na pozorovatelích a interpretaci. Lidské poznání je schopné se k objektivní
  realitě postupně přibližovat v řadě aproximativních kroků. Tento proces se však nikdy nevyčerpá a
  naše poznání nikdy nebude úplné. 
  
  Většina fyziků, ne-li všichni, by neměla problém s materialistickým přístupem jako takovým, ke své
  práci přistupují jako praktičtí materialisté, aniž cítí potřebu to výrazně deklarovat. Rozhodně
  jim ale vadí, jestliže se jim filosofové snaží předpisovat, jak při výzkumu postupovat, a učí je
  „správné metodologii“, propagované určitou filosofickou školou. Marxismus-leninismus nebyl jen
  určitou filosofickou školou - byla to mocná doktrína podporovaná sovětským státem, činící si nárok
  na první slovo ve všech oblastech. V napjaté atmosféře třicátých až padesátých let dvacátého
  století filosofické diskuse o interpretaci kvantové mechaniky či relativity mohly vyústit v
  obvinění fyziků z neloajality a takové označení mívalo nebezpečné následky. 
  
  Einsteinova relativistická fyzika právě tak jako nové radikální myšlenky v kvantové fyzice se díky
  své obtížné pochopitelnosti a často nejasné filosofické interpretaci staly oblíbeným terčem
  sovětských filosofů vědy. Einsteinova teorie prostoročasu byla napadána z řady stran. Předně byla
  prohlašována za exemplární příklad idealismu. Vynořila se na základě Einsteinových myšlenkových
  experimentů, jež měly často malý či žádný vztah k přirozenému světu každodenní zkušenosti. Za
  druhé byla formulována ve velmi obtížném matematickém jazyce a měla složitou interpretaci, takže
  pro mnohé filosofy, jejichž matematické vzdělání nebylo moc hluboké, byla nesrozumitelná. A
  konečně, což bylo nejhorší, plynul z ní absurdní obraz vesmíru s počátkem, který se zdál blízký
  náboženskému obrazu stvoření, a proti náboženství marxismus tvrdě bojoval. Relativistické
  kosmologii nepřidalo ani to, že jejím čelným představitelem byl kněz Lemaître, představitel
  zkažené dekadentní buržoazní společnosti. Při tomto tvrdém odsouzení se nějak zapomínalo na to, že
  vyvíjející se vesmír navrhl jako první skvělý ruský a později sovětský matematik Alexander
  Friedmann. Debata občas vzplála a pak zase na nějakou dobu utichla. Bylo by ale nepatřičným
  zjednodušením dívat se na ni jako na spor mezi brilantními fyziky a nevzdělanými ortodoxními
  filosofy. Řada matematiků a fyziků - a byli mezi nimi i velmi známí vědci - se k šikům filosofů
  připojila a diskuse byla často zatížena skupinovými zájmy a dalšími faktory, jež měly málo
  společného s vlastním předmětem diskuse. 
  
  V roce 1952 vlivný sovětský filosof a historik vědy Alexander Maximov uveřejnil článek nazvaný
  „Proti reakčnímu einsteinismu ve fyzice“. Článek vyšel v trochu obskurním časopise sovětského
  námořnictva Krasnyj flot, reakce fyziků však byla velmi ostrá. Čelný relativista té doby Vladimir
  Fock, někdejší Friedmannův student, odpověděl článkem „Proti ignorantské kritice moderní fyziky“.
  Před zveřejněním článku se Fock, Lev Davidovič Landau a další fyzici obrátili na sovětské
  politické vedení se žádostí o pomoc. V soukromém dopisu adresovaném Lavrentiji Berijovi, blízkému
  spolupracovníku Stalina a vedoucímu sovětského nukleárního a termonukleárního projektu, si
  stěžovali na „abnormální situaci v sovětské fyzice“ a Maximovův článek uváděli jako příklad
  agresivní ignorance, jež brzdí pokrok sovětské fyziky. Fockův článek vyšel a Fock oznámil, že má v
  této věci podporu strany a vlády. Rozzuřený Maximov trval na svém názoru a stěžoval si u Beriji,
  ale v roce 1954 Fockova a Landauova skupina zvítězila. V té době mělo ovšem nejvyšší politické
  vedení Sovětského svazu na starosti naléhavější věci než rozebírat jemnosti Einsteinových teorií.
  Navíc měl ovšem Landau a jeho kolegové na své straně velmi silný argument - zkonstruovali
  sovětskou atomovou bombu a tak teorie, na jejichž základu byla postavena, byly zřejmě v pořádku,
  bez ohledu na jejich filosofickou interpretaci. V polovině padesátých let minulého století
  ideologické boje mezi filosofy a fyziky skončily a relativisté měli nadále vcelku pokoj. Jednou z
  posledních zaregistrovaných bitev této války bylo upozornění na „ideologicky závadnou“ veřejnou
  přednášku o rozpínajícím se vesmíru Jevgenije Lifšice, který spolu s Landauem napsal proslulý Kurz
  teoretické fyziky. Toto upozornění dostal Ústřední výbor komunistické strany, který se jím sice
  zabýval, ale nemělo to žádné následky. 
  
  Války s marxistickými filosofy neměly nic společného s politickými represemi v letech 1937-1938 i
  jindy, které stály výjimečně talentované sovětské fyziky, jako byli Matvej Bronštejn, Lev
  Šubnikov, Semen Šubin a Alexander Witt, život a další skončili v žaláři nebo byli donuceni k
  exilu. Přesto ale rozvoj Einsteinovy teorie relativity v SSSR byl pomalý z podobných důvodů jako
  na Západě - převažoval zájem o kvantovou teorii. K tomu přistupovala snaha o přežití země při
  prudké industrializaci, grandiózní vítězná bitva s evropským fašismem a následný nukleární závod
  během studené války. 
  
  Odmítali-li sovětští filosofové obecnou teorii relativity pro její údajný matematický idealismus,
  tak ještě více proti mysli jim byla pozdější Einsteinova práce. Po příjezdu do Princetonu ho
  posedla myšlenka vytvořit teorii velkého sjednocení. Obecná teorie relativity byla jeho srdci
  stále ještě blízká, ale chtěl uskutečnit něco ještě většího a lepšího. Obecnou teorii relativity
  chtěl začlenit do teorie, jež by uvedla celou základní fyziku do jednotného rámce. Einstein
  doufal, že se mu povede ukázat, že nejenom gravitace, ale též elektřina a magnetismus a možná i
  některé podivné efekty přisuzované kvantům mohou být popsány jako geometrie prostoročasu. Ale
  zatímco jeho cesta k obecné teorii relativity byla výsledkem jeho neobyčejné fyzikální intuice
  podpořené silou riemannovské geometrie, k nové výzvě přistupoval zcela rozdílně. Vzdal se intuice
  a nechal se vést jen složitou matematikou. 
  
  Einstein nepřišel jen s jedinou velkou sjednocenou teorií. Po třicet let klopýtal od teorie k
  teorii, někdy nějakou možnost zavrhl, aby se k ní o deset let později vrátil. Jedním z jeho pokusů
  bylo zavést pětirozměrný prostor místo čtyřrozměrného. Dodatečná prostorová dimenze byla sbalená
  do sebe a prakticky nepozorovatelná. Její geometrie, respektive s ní spojená křivost, měla
  reprezentovat elektromagnetické pole, buzené náboji a proudy přesně tak, jak to navrhl v druhé
  polovině devatenáctého století James Clerk Maxwell. 
  
  Myšlenka pětirozměrného prostoru nebyla původně Einsteinova. Pocházela od dvou mladých vědců,
  Theodora Kaluzy, skromného soukromého docenta univerzity v Královci, a Oskara Kleina, mladého
  švédského fyzika, který pracoval pod Nielsem Bohrem. Spolu detailně rozpracovali, jak by
  pětirozměrný prostoročas mohl téměř dokonale vystihovat elektromagnetismus. Kaluzův a Kleinův
  vesmír, nad kterým Einstein strávil téměř dvacet let života, je znečištěn jakousi zvláštní formou
  hmoty, nekonečně mnoha různými částicemi s nejrůznějšími hmotnostmi, jež by měly být všude kolem
  nás a zakřivovat zbývající geometrii prostoročasu. Einstein doufal - nikdy se mu to však
  nepodařilo dokázat -, že tato dodatečná pole by mohla být nějak svázána s vlnovou funkcí, kterou
  do fyziky zavedl Schrödinger. Teorie tohoto typu Einstein opustil koncem třicátých let; je velmi
  zajímavé, že Kaluzova-Kleinova teorie se do fyzikálního uvažování vrátila v sedmdesátých letech
  minulého století, kdy se myšlenka sjednocení všech interakcí stala hitem teoretické fyziky. 
  
  Mnohem více času Einstein věnoval jiné teorii, jež měla navzájem těsněji svázat elektromagnetismus
  a gravitaci. Vzal matematický rámec obecné relativity, jazyk, který zavedl Riemann několik desítek
  let předtím, a rozvolnil jej. V obecné relativitě určuje geometrii a dynamiku prostoročasu celkem
  deset neznámých funkcí, které je třeba určit z rovnic pole. Skutečnost, že je zde tolik neznámých
  funkcí, které jsou navzájem složitě propojeny, činí řešení rovnic pole obecné teorie relativity
  neobyčejně složitým problémem. Do své nové teorie chtěl Einstein přidat dalších šest funkcí, z
  nichž tři by popisovaly elektrické a tři magnetické pole. Problémem ovšem bylo, jak těchto
  šestnáct funkcí svázat dohromady tak, aby teorie byla stále dobře definovaná a daly se z ní vyčíst
  jednoznačné fyzikální předpovědi. Kdyby se to bylo povedlo, výsledkem by byla teorie, ve které by
  s geometrií prostoročasu byly spojeny jak elektromagnetismus, tak gravitace. Připomeňme, že v
  obecné teorii relativity geometrie určuje gravitaci. Einstein se snažil zkonstruovat matematicky
  krásnou teorii, přes mnoholetou práci se mu to však uspokojivě nepovedlo. 
  
  Einstein byl něčemu na stopě - hledání velkého sjednocení všech interakcí dominovalo fyzice koncem
  dvacátého století. Během svého života však byl opuštěným hledačem svatého grálu fyziky, velká
  většina fyzikálního společenství pokládala jeho cíl za neuskutečnitelný. I když mezi vědci byl
  osamělou postavou, laická veřejnost byla jeho úsilím fascinována. Čas od času se Einstein ocitl na
  titulní straně hlavních novin. V listopadu 1928 oznamovaly New York Times: „Einstein na pokraji
  velkého objevu“ a neméně nadšený byl i článek, který zde vyšel o rok později na základě krátkého
  rozhovoru s Einsteinem. Zájem a nadšení sdělovacích prostředků vydrželo i v další čtvrtině
  dvacátého století. V roce 1949 prohlašovaly New York Times: „Nová Einsteinova teorie dává
  univerzální klíč k vesmíru“ a o pár let později, v roce 1953, vytrubovaly: „Einstein nabízí novou
  teorii sjednocující zákony vesmíru“. Ale přes všechen zájem médií kolegové Einsteina moc
  neposlouchali a jeho pokusy o jednotnou teorii pole z velké většiny odmítali. 
  
  Odchodem do Spojených států unikl Einstein proudu pomluv, kterému bylo jeho dílo vystaveno v
  Německu. Jenže i v jeho nové vlasti zájem o relativitu klesal. Mladé bystré studenty kolem něho,
  kteří měli potenciál posunout obecnou relativitu kupředu, přitahovala kvantová teorie, především
  její aplikace na elementární částice a základní interakce. 
  
  V jistém smyslu to bylo pochopitelné. Obecná teorie relativity přinesla několik velkých úspěchů,
  jako byl výklad posunu perihelia Merkura a ohyb světla v gravitačním poli Slunce. Vedla také k
  odhalení rozpínání vesmíru, což od základu změnilo náš světový názor. Jenže se zdálo, že další
  řešení rovnic pole vedou k matematickým důsledkům, jež se zdály fyzikálně nevěrohodné. Příkladem
  bylo Schwarzschildovo řešení v plném rozsahu a s ním související řešení Oppenheimera a Snydera pro
  kolabující hvězdu. Tato řešení snad mohla popisovat situace, které se ve vesmíru skutečně
  vyskytují, nikdo je však neviděl, takže nebylo jasné, zda se nejedná jen o matematické exoty. Na
  druhé straně kvantová fyzika se dala ověřovat v laboratořích a na jejím základě se dala stavět
  různá zařízení. Bylo ale jasné, že v obecné teorii relativity se dají objevit ještě další
  podivuhodnosti. Na jednu takovou ukázal logik Kurt Gödel. 
  
  Během chůze do Institutu Einstein nebyl vždy osamělý. Tohoto excentrického profesora v pomačkaných
  šatech, s rozcuchanými vlasy a laskavým pohledem často doprovázel malý človíček, vždy zahalený do
  těžkého převlečníku a s očima za silnými brýlemi. Jak se Einstein roztěkaně sunul směrem k Fuld
  Hall, jeho společník ho sledoval a tiše naslouchal Einsteinovým monologům, které jen zřídka
  přerušoval poznámkou pronesenou vysokým hlasem. Byl to Einsteinův blízký přítel Kurt Gödel,
  člověk, který zcela zaměnil moderní matematiku. K velkému Einsteinovu překvapení zasáhl tento muž
  významně i do obecné teorie relativity. 
  
  Gödel intelektuálně dozrával ve Vídni, která v té době byla významným kulturním centrem. Vídeňské
  kavárny byly místem hlubokých debat a druhým domovem osobností jako Ernst Mach, Ludwig Boltzman,
  Rudolf Carnap, Gustav Klimt a mnoha dalších. Nejprestižnějšími z těchto neformálních setkání byly
  debaty světoznámého Vídeňského kroužku. Aby se někdo mohl stát členem Vídeňského kroužku, musel
  tam být přizván. Gödel byl jedním z mála vyvolených. 
  
  Na rozdíl od Einsteina byl Gödel od dětství výborným žákem, který základním vzděláním procházel se
  skvělými známkami ze všech předmětů, a byl vynikajícím studentem i na univerzitě. I on koketoval s
  fyzikou, na rozdíl od Einsteina však zvítězila matematika. Zaujal ho problém jak převést celou
  matematiku do jednotného logického rámce. Rychle zvládl rozvoj v této oblasti, za nějž byli
  zodpovědní jak matematici, tak filosofové a jehož cílem bylo vytvořit neprůstřelnou matematickou
  teorii, odolnou proti vší iracionalitě a nepodloženým úvahám. Takový byl plán, který nastínil
  David Hilbert, matematický kníže z Göttingen. 
  
  David Hilbert pevně věřil, že veškerá matematika se dá založit na hrstce předpokladů, axiomů.
  Pečlivou systematickou aplikací logických pravidel by se pak dal každý jednotlivý matematický
  výrok ve vesmíru odvodit z malého souboru axiomů. Nic by z tohoto systému nebylo vynecháno.
  Ověření kteréhokoli matematického faktu, od tvrzení že 2 + 2 = 4 až po velkou Fermatovu větu, by
  se získalo logickým důkazem. Hilbertův program byl motorem matematiky té doby a na něj upřel
  pozornost i Kurt Gödel. 
  
  Kurt Gödel se ponořil do vídeňského života, tiše se účastnil schůzek Vídeňského kroužku a sledoval
  nekonečné debaty mezi logiky a matematiky o tom, jak rozšířit Hilbertův program na celou přírodu.
  Při tom ale soustavně nahlodával základní principy tohoto programu. A pak najednou celý Hilbertův
  plán zdemoloval důkazem jím předložené věty o neúplnosti. 
  
  Věta o neúplnosti říká něco neuvěřitelně jednoduchého. Kdykoliv matematicky popisujete nějaký
  systém, začínáte se souborem axiomů a pravidel. Gödel ukázal, že ať je tento systém axiomů
  jakýkoli, vždy existují věci, jež se z něho nedají odvodit. Když zakopnete o nějakou skutečnost,
  kterou z vašeho souboru axiomů nemůžete dokázat, tak ji můžete k vašemu souboru axiomů prostě
  přidat jako další axiom. Jenže Gödelova věta říká, že těch nedokazatelných pravdivých tvrzení je
  nekonečně mnoho. Když tedy postupně sbíráte nedokazatelné pravdy a přidáváte je do vašeho systému,
  váš elegantní deduktivní systém se bude postupně nadouvat. Po čase bude gigantický, ale stále
  neúplný. 
  
  Gödelova věta torpedovala Hilbertův program a řadu matematiků vyvedla z míry. Hilbert sám zpočátku
  nechtěl Gödelův výsledek přijmout. Nakonec jej uznal a neúspěšně se jej snažil včlenit do svého
  programu. Jiní filosofové napsali zavádějící kritiky, které zase nechtěl přijmout Gödel. Anglický
  filosof Bertrand Russell se s Gödelovou větou nikdy necítil příjemně. Ludwig Wittgenstein, který
  určoval filosofické myšlení první poloviny dvacátého století, větu o neúplnosti prostě zavrhl a
  prohlásil ji za irelevantní. Nebyla a Gödel to dobře věděl. 
  
  Gödel Vídeň miloval, ale okolnosti ho nakonec donutily hledat útočiště tam, kde nalezl poslední
  přístav i Einstein. Během několika návštěv ve třicátých letech si postupně zvykal na prostředí
  Institutu pro pokročilá studia, kde se spřátelil s Einsteinem a diskutoval s von Neumannem.
  Uvědomil si, jaký je intelektuální kalibr emigrantů, kterým ústav poskytl útulek. Po velmi
  ošklivém incidentu, při kterém byl zbit proto, že vypadal jako Žid, se rozhodl k odchodu z Vídně.
  
  Mezi Einsteinem a Gödelem přeskočila jiskra přátelství okamžitě. Einstein říkal, že se mu vyplatí
  jít do ústavu jen proto, „aby měl privilegium jít domů s Kurtem Gödelem“. Když Gödel onemocněl,
  Einstein ho navštěvoval a pečoval o něj. Když Gödel zažádal o americké občanství a schylovalo se k
  přísaze, kterou musí žadatel složit, nalezl v americké ústavě něco, co vnímal jako logickou
  nekonzistenci, jež by mohla dopustit, aby v zemi zavládla tyranie. Einstein se tehdy do
  záležitosti vložil a Gödela k přísaze doprovodil, aby mu zabránil sabotovat slavnostní udělení
  jeho vlastního občanství. 
  
  I když hlavním Gödelovým zájmem byla matematika, měl rád i fyziku a často trávil s Einsteinem
  hodiny diskusemi o relativitě a kvantové mechanice. Oba nechtěli přijmout náhodnost v kvantové
  mechanice, ale ani zde se Gödel nezastavil. Domníval se, že katastrofální trhlina je i v obecné
  teorii relativity. 
  
  Gödel studoval Einsteinovy rovnice pole a podobně jako Friedmann, Lemaître a mnoho dalších se
  snažil najít takové jejich řešení, které by bylo matematicky schůdné, a přesto by mohlo
  reprezentovat reálný vesmír. Víme, že Einstein uvažoval vesmír vyplněný hmotou - atomy, hvězdami,
  galaxiemi -, jež v jeho modelu měla být rovnoměrně rozložená. Když jste se tímto vesmírem
  pohybovali, vypadal všude stejně, neměl žádná význačná místa nebo směry, žádný střed. Friedmann i
  Lemaître přejali - každý po svém - tento Einsteinův předpoklad a nalezli jednoduchá řešení, ve
  kterých se geometrie vesmíru vyvíjela s časem. Gödel si dovolil model malinko zkomplikovat. Jím
  předpokládaná změna byla dostatečně malá, aby se rovnice ještě daly přesně řešit, ale přece jen
  tak významná, že způsobila něco zajímavého. Předpokládal, že vesmír rotuje, otáčí se kolem
  centrální osy stále dokola, jako kolotoč. V tomto novém vesmíru, který nalezl, mohl být
  prostoročas popsán pomocí času, tří prostorových souřadnic a geometrií v každém bodě prostoru - to
  bylo stejné jako ve vesmírech Friedmannových a Lemaîtrových. Byly zde však určité rozdíly. Předně
  ve Friedmannových a Lemaîtrových vesmírech se měl pozorovat červený posun, o kterém Slipher a
  Hubble ukázali, že se v reálném vesmíru skutečně vyskytuje. Tento jev v Gödelově vesmíru chyběl.
  Tento vesmír tedy zřejmě nemohl vysvětlit rozpínání vesmíru objevené Slipherem, Hubblem a
  Humasonem. Na tom ale tolik nezáleželo. Důležité bylo, že je to řešení Einsteinových rovnic, tedy
  matematicky možný vesmír v Einsteinově obecné teorii relativity. 
  
  Toto řešení se však od až dosud nalezených vesmírů lišilo dramatickým způsobem. Pozorovatelka
  (samozřejmě i pozorovatel) ve Friedmannově či Lemaîtrově vesmíru se mohla vydat na jeho průzkum,
  vyšetřovat různá místa v něm a jak se takto pohybovala, stárla a svůj minulý život nechávala za
  sebou. Minulost, přítomnost a budoucnost měly jasný smysl. Ne tak ve vesmíru Gödelově. Zde mohlo
  dojít k tomu, že když se pozorovatelka pohybovala vesmírem dostatečně rychle, mohla najednou
  natrefit na sebe samu v minulosti. S dobrou přesností mohla vidět sebe samu v mladších letech,
  před tím okamžikem, než se vydala na výlet vesmírem. Jinak řečeno, Gödelův vesmír dovoloval cesty
  zpět v čase. 
  
  V Gödelově fantastickém vesmíru se dalo přemisťovat tam a zpět v čase, znovu navštívit svou
  minulost, napravit chyby mládí, omluvit se příbuzným, kteří již dávno zemřeli, a vyhnout se
  špatným rozhodnutím s ohledem na to, co jste viděli v budoucnosti. Také to ovšem znamenalo, že se
  zde mohly odehrát věci, které nedávaly dobrý smysl, a vznikat nebezpečné paradoxy. Předpokládejme,
  že jsme na výletě do minulosti narazili na svou babičku, když byla mladým děvčetem, a nějakým
  způsobem došlo k hrozné události - zabili jste ji. Když jste ji vymazali z existence, nemohla dát
  život vašemu otci či vaší matce. Tím jste ale negovali i možnost vlastní existence, takže jste se
  nemohli vrátit v čase a spáchat ten hrozný čin. Kdybyste ale žili v Gödelově vesmíru, nic by vám v
  takovýchto věcech nebránilo, odhlédneme-li od technických a morálních hledisek. Gödelův výsledek
  ukazoval, že Einsteinova obecná teorie relativity připouští řešení, jež dovolují cesty zpět v čase
  a tím i paradoxy, v rozporu s tím, jaká je naše zkušenost se světem kolem nás. Jestliže
  Einsteinova teorie skutečně odráží přírodu, Gödelův absurdní vesmír je reálnou fyzikální možností.
  
  Gödel předložil své výsledky na slavnostním semináři konaném v roce 1949 u příležitosti
  sedmdesátých narozenin Alberta Einsteina. Jeho řešení bylo prezentováno jasně a stručně. Bylo ale
  tak podivné, že nikdo nevěděl, co k němu říci. Chandra, který se posledních dvacet let bránil
  Eddingtonově kritice, napsal krátké sdělení, ve kterém uváděl, kde je podle něho chyba v Gödelově
  odvození. Jenže tentokrát to byl systematický a pečlivý Chandra, jenž se dopustil matematické
  chyby. Astronom z Caltechu H. P. Robertson, který spolu s Friedmannem a Lemaîtrem prosazoval
  myšlenku rozpínání vesmíru, napsal o rok později přehledný článek o stavu kosmologie a v něm
  Gödelův vesmír pohrdavě zavrhl. 
  
  A Einstein? Ten zapnul svou příslovečnou intuici, která hrála klíčovou roli při všech jeho velkých
  objevech, ať už to byla speciální či obecná teorie relativity nebo jiné práce. Byla to ovšem
  pohříchu tatáž intuice, jež ho vedla k zamítnutí Friedmannova a Lemaîtrova řešení a k ignorování
  plného významu řešení Schwarzschildova. Gödelovo řešení komentoval tak, že „Gödelův vesmír je
  důležitým příspěvkem k obecné teorii relativity“, ale je třeba rozmyslet, zda jej nelze „vyloučit
  na fyzikálním základě“. 
  
  Gödelovo řešení Einsteinových rovnic se zdálo být příliš bizarní, než aby mohlo mít nějaký reálný
  dopad na skutečný přírodní svět. Gödel sám až do své smrti v roce 1978 stále pátral po nějakých
  astronomických pozorováních, jež by podporovala jeho řešení a ukazovala, že má reálnou fyzikální
  důležitost. V nějakém smyslu však bylo Gödelovo řešení exemplárním příkladem toho, co na obecné
  relativitě mnohé fyziky odpuzovalo: byla to matematická teorie, jež měla i bizarní řešení. Taková
  teorie by neměla žádnou relevanci pro reálný svět. 
  
  Když se v roce 1935 Institut pro pokročilá studia poprvé snažil získat Roberta Oppenheimera, bylo
  to v době, kdy se slibně rozjížděla jeho škola v Berkeley a začínala získávat jméno, a tak nabídku
  tehdy odmítl. Po krátké návštěvě v Institutu napsal svému bratrovi, že „Princeton je ústav duševně
  chorých: jeho solipsistické kapacity osamoceně září v naprosté opuštěnosti. Einstein je zcela
  bláznivý.“ Nikdy se mu nepodařilo zcela zbavit nepříznivých dojmů z Einsteinových pozdějších
  prací. 
  
  V roce 1947 však Oppenheimer přijal místo ředitele Institutu. Jeho jmenování se setkalo i s
  odporem. Einstein a Hermann Weyl prosazovali na toto místo rakouského fyzika Wolfganga Pauliho,
  objevitele vylučovacího principu, jednoho ze základních kamenů kvantové fyziky. Ve své kampani se
  opírali o to, že „Oppenheimer neučinil žádný tak fundamentální objev, jakým je Pauliho vylučovací
  princip“. Oppenheimerova pověst skvělého organizátora však byla taková, že vedoucí místo bylo
  svěřeno jemu. Hned se pustil do oživování ducha ústavu, jeho způsob vedení byl neformální a měl
  švih. Hlavní článek v Time z roku 1948 konstatoval: „Na seznamu hostů Oppieho hotelu najdeme v
  tomto roce historika Arnolda Toynbeeho, básníka T. S. Eliota, filosofa práva Maxe Radina - a
  literárního kritika, úředníka i leteckého manažera. Není jasné, kdo budou další - možná psycholog,
  předseda vlády, malíř nebo hudební skladatel.“ Institut rozhodně netrpěl výlučností. 
  
  O obecnou relativitu ztratil Oppenheimer zájem nedlouho poté, co se jí krátce věnoval se studenty
  v Berkeley. Tehdy spolu se svým studentem Hartlanem Snyderem napsal jeden z nejdůležitějších
  článků pro rozvoj obecné teorie relativity. V Princetonu však mladým pracovníkům rozmlouval, aby v
  této oblasti pracovali. Během jeho kralování jeden z mladých členů institutu Freeman Dyson napsal
  domů: „Obecná teorie relativity je v současné době jedním z nejméně nadějných polí výzkumu.“
  Einsteinova teorie se nezdála moc užitečnou alespoň do doby, kdy nové experimenty odhalí více o
  podivných vlastnostech prostoročasu, nebo někdo nalezne způsob, jak navzájem propojit obecnou
  relativitu s kvantovou teorií. 
  
  Oppenheimer nebyl jediným z vůdčích fyziků, kteří obecnou teorii relativity odmítali jako nadějnou
  oblast bádání. Vzestup kvantové teorie zastínil Einsteinovu teorii natolik, že článek s touto
  problematikou nebylo snadné publikovat. Redaktorem časopisu Physical Review byl Samuel Goudsmit,
  holandský fyzik žijící v Americe, jenž byl jednou z důležitých postav raného vývoje kvantové
  teorie. Stal se jím po své imigraci do Ameriky Samuel Goudsmit (1902-1978) proslul především
  zavedením představy spinu elektronu, kterou v roce 1925 předložil spolu Georgem E. Uhlenbeckem.
  Pozn. překl. a umínil si, že z Physical Review udělá nejdůležitější světový fyzikální časopis,
  který v konkurenčním boji porazí časopisy evropské. Na obecnou teorii relativity se díval se
  skepsí. Tak jako Oppenheimer nevěřil, že se toho dá mnoho pořídit s tak ezoterickou teorií s
  omezenými možnostmi aplikace a testovatelnosti. Hrozil redakčním pokynem, který by prakticky
  zastavil publikování článků týkajících se „obecné teorie relativity a fundamentální teorie“ ve
  Physical Review. Zabránil tomu apel princetonského profesora Johna Archibalda Wheelera, který si v
  té době začínal uvědomovat krásu Einsteinovy teorie, a proto se Goudsmitovu záměru vzepřel. 
  
  Mezi Oppenheimerem a Einsteinem se však nakonec vyvinul přátelský vztah, srdečný, i když ne zcela
  blízký. Ukazují to různé jejich činy, prokazující vzájemnou loajalitu. Jednou například
  Oppenheimer připravil starému pánovi překvapení k narozeninám. Nechal mu instalovat stožár s
  rádiovou anténou na jeho domě v Mercer Street, aby mohl večer poslouchat svou oblíbenou hudbu. V
  Einsteinovi pak našel věrného spojence během svých nejtemnějších dnů. 
  
  Oppenheimer zaznamenal v Berkeley raketový vzestup a jako vedoucí projektu Manhattan prokázal
  vlasti neobyčejné služby. Stal se členem sedmičlenného představenstva Komise pro atomovou energii,
  která dohlížela na poválečné atomové projekty a praktické využívání atomové energie. V této úloze
  vzbudil nelibost, když odmítal některé fantastičtější projekty, například letadlo s jaderným
  pohonem, které by mělo neomezený dolet, hlavně však nesouhlasem s konstrukcí „superbomby“, tedy
  vodíkové bomby, proti níž by bomby svržené na Hirošimu a Nagasaki byly svým účinkem trpaslíky.
  Nadělal si tím mocné nepřátelé. A tito nepřátelé na něj zaútočili v období antikomunistické
  hysterie v éře McCarthyho v padesátých letech. 
  
  V roce 1953 byl v článku v časopise Fortune Oppenheimer silně kritizován „za trvalou kampaň, která
  má zvrátit vojenskou politiku Spojených států“, a obviněn z toho, že je vedoucím komplotu, jenž má
  sabotovat konstrukci vodíkové bomby. V tomto roce byla Oppenheimerovi odebrána bezpečnostní
  prověrka a byl prohlášen za hrozbu bezpečnosti Spojených států. V roce 1954 si vyžádal slyšení
  před příslušným výborem a byl částečně očištěn, bezpečnostní prověrku však zpět nezískal. Výnos ze
  slyšení říkal: „Zjistili jsme, že chování a styky dr. Oppenheimera stále ještě odrážejí vážná
  opomenutí z hlediska požadavků bezpečnostního systému.“ Oppenheimer ztratil svou pozici člena
  washingtonské elity. 
  
  Einstein nikdy nechápal Oppenheimerovo opojení mocí. Proč Oppenheimer tak stál o postavení
  odpovídající vysokému státnímu úředníkovi? Einstein jako jeden z představitelů světového pacifismu
  nikdy nepochopil, proč Oppenheimer, jenž byl myšlenkám světového míru nakloněný, se nikdy nesnažil
  hlasitěji veřejně vystoupit proti závodům ve zbrojení. On sám nemlčel, objevoval se v televizi se
  svými poselstvími národu, varoval před zlem „superbomby“. V novinách se objevovaly titulky jako
  „Einstein varuje svět: Postavte vodíkovou bombu mimo zákon, nebo zhyňte“. 
  
  Ve svých posledních nejopuštěnějších dnech se Einstein zase stal známým. Z vnějšího pohledu to
  byla ironická situace. V jednom poschodí Institutu Einstein pomáhal koncipovat pacifistické výzvy
  proti šíření jaderných zbraní, zatímco v jiném poschodí se Oppenheimer skláněl nad plány vodíkové
  bomby. Jenže Einstein si mohl svůj postoj dovolit. Byl příliš známý, než aby ho mohla ohrozit
  protikomunistická hysterie. Klíčová postava americké nukleární hegemonie Oppenheimer byl sesazen z
  trůnu a byl ponižován bezpečnostními výslechy. Zůstával opatrný, aby nemohl být nějak spojován s
  komunistickou hrozbou, zatímco Einstein veškerou opatrnost odložil. Veřejně odsuzoval výslechy
  před bezpečnostním výborem. Do New York Times například napsal: „Co může menšina intelektuálů
  dělat proti zlu? Upřímně řečeno, jako východisko vidím jedině revoluční cestu nespolupráce ve
  stylu Gándhího.“Veřejně vyzýval ty, kteří byli k výslechu pozváni, aby odmítli vypovídat s odkazem
  na pátý dodatek k ústavě. 
  
  Na poslední Einsteinovy roky vrhala stín nemoc. V roce 1948 mu byla diagnostikována výduť břišní
  aorty, jež mohla mít fatální následky. V průběhu let se pomalu zvětšovala a Einstein se
  připravoval na nevyhnutelné. Když v roce 1955 dosáhl sedmdesátých šestých narozenin, byl už příliš
  nemocný, než aby mohl plánovat cestu do Bernu na konferenci pořádanou na oslavu padesátého výročí
  speciální teorie relativity. V polovině dubna výduť nakonec praskla a Einstein po několika dnech v
  nemocnici zemřel. 
  
  Pohřeb byl rychlý a neokázalý. Kremaci navštívila hrstka lidí a jeho popel byl rozptýlen v
  soukromí. Z jeho pohřbu zbylo jen pár fotografií, které ukazují, že to byla tichá a skromná
  událost. Jeho mozek byl zachován potomstvu v naději, že by mohl posloužit jako klíč k jeho
  genialitě. Bernská konference se uskutečnila, měla teď charakter chvály a oslavy Einsteinova díla.
  
  Oppenheimer byl jako vedoucí Institutu opakovaně žádán o hodnocení Einsteinova života a díla. Při
  takových příležitostech sice vysoce oceňoval Einsteinovy výsledky, když ovšem byl k tomu dotlačen,
  nedokázal skrýt lehký nesouhlas s Einsteinovou prací posledních let. Neměl problém prohlásit, že
  „Einstein byl skutečně největším fyzikem a přírodním filosofem naší doby“, v článku o Institutu
  pro časopis Time z roku 1948 se však vyjádřil pro senzacechtivé novináře méně nadšeně: „V těsně
  provázaném bratrstvu fyziků se bohužel dochází k poznání, že Einstein je mezník, nikoli vůdčí
  osobnost; překotný vývoj fyziky ho zanechal daleko za čelem závodu.“ V rozhovoru pro L’Express asi
  deset let po Einsteinově smrti šel Oppenheimer ještě dále, když řekl, že „na konci svého života už
  Einstein nevytvořil nic cenného“. 
  
  K Einsteinovu odchodu ze světa došlo v období stagnace obecné teorie relativity. Byla zastíněná
  kvantovou fyzikou a řada vůdčích fyziků ji odmítala. Aby získala novou životní energii, k tomu
  bylo potřeba nové krve a nových objevů.

\section{Dny rádia}\label{feyIchIIIsecVII}
  Dny rádia V roce 1949 hluboce zapůsobily na posluchače BBC přednášky Freda Hoylea, vysílané v
  sérii nazvané Povaha vesmíru. Výmluvný mladý cambridgeský profesor promlouval k milionům lidí a
  poučoval je o historii a vývoji vesmíru. Podobně jako před ním Einstein, Lemaître a řada dalších
  vědců zpřístupňoval teorii relativity širokému okruhu lidí a oni to přijímali s potěšením. Protože
  mu ještě nebylo ani čtyřicet let, mohl se pro veřejnost stát novým mluvčím obecné teorie
  relativity a nahradit v této úloze Einsteina, Eddingtona a Lemaîtra. 

  Jenže Hoyle prohlašoval, že Lemaître nemá pravdu. Podle Hoylea rozpínání vesmíru z ničeho byl
  nesmysl a otcové obecné relativity měli teorii opravit tak, aby dávala rozumnější výsledky.
  Tvrdil, že je hloupost předpokládat, že vesmír náhle vznikl z ničeho. Stav kosmologických teorií v
  té době popisoval slovy: „Tyto teorie by založeny na hypotéze, že všechna hmota byla stvořena v
  jediném,velkém bum‘, které se odehrálo v určitém čase ve vzdálené minulosti.“ Anglický termín,
  který použil a který se dnes do češtiny překládá jako „velký třesk“, byl „Big Bang“ a mínil ho
  posměšně. Myslel si, že existuje mnohem lepší řešení - nekonečný vesmír, který se stále regeneruje
  stacionárním tvořením hmoty. 

  Hoyle se pustil do války s relativisty a velký počet posluchačů rádia mu dodával sílu. Jeho
  přednášky vyznívaly tak, jako by jeho teorie stacionárního vesmíru byla standardní kosmologií a
  rozpínající se vesmír z dvacátých let byl překonanou představou. To však nebyla pravda. Hoyle a
  jeho dva spolupracovníci Hermann Bondi a Thomas Gold ve skutečnosti deformovali obecné povědomí o
  tom, co se v teoretické fyzice děje, a to velmi zlobilo jejich kolegy. Jeden astronom v reakci na
  jeho rozhlasové přednášky prohlásil, že Hoyle zašel až na samý okraj decentní popularizace
  astronomie, a on se obává, že Hoyleova jednostrannost a neskromnost poškozuje obraz profese.

  Přes Hoyleovu mediální popularitu by jeho teorie stacionárního vesmíru zůstala asi jen místní
  záležitostí a diskuse o ní by se omezily jen na cambridgeský okruh. Jenže otázky, které teorie
  stacionárního vesmíru vyvolala, inspirovaly mladé vědce při rozvíjení možností nového okna do
  vesmíru. A právě nové možnosti pozorování způsobily renesanci ve výzkumu obecné relativity v
  následujících desetiletích. 

  Není udivující, že se takový samorost jako Hoyle vynořil právě v Cambridgi, říši Arthura
  Eddingtona. Ten totiž později v životě vypadl z hlavního proudu vědy a byl zcela posedlý svou
  velmi ezoterickou teorií vesmíru - trochu to připomínalo Einsteinovu snahu nalézt jednotnou teorii
  pole. I on v posledních desetiletích svého života pracoval na fundamentální teorii, která by
  zahrnula všechno - gravitaci, relativitu, elektřinu, magnetismus i kvantový popis. Ale ostatním
  vědcům se jeho svět čísel, symbolů, a magických souvislostí jevil spíše jako numerologie,
  nepřipomínal elegantní matematiku, která stála za obecnou teorií relativity. A tak byl Eddington
  ve svém úsilí ještě osamocenější než Einstein a do své smrti v roce 1944 žil v relativní izolaci.
  Zanechal po sobě neúplný rukopis, který posmrtně vyšel v roce 1947 pod hrdým titulem Fundamentální
  teorie. Je to obskurní kniha, nečitelná a prakticky zapomenutá, smutný odkaz velikého vědce, který
  pomohl obecné teorii relativity k uznání. Jeden astronom o ní prohlásil: „Bez ohledu na to, zda
  přežije jako velké dílo vědy, je to určitě pozoruhodné dílo umění.“ Wolfgang Pauli, objevitel
  vylučovacího principu, tak důležitého pro porozumění bílým trpaslíkům, se o Eddingtonově teorii
  vyjádřil s pohrdáním. Prohlásil ji za „úplný nesmysl, za romantickou poezii, ne fyziku“. 

  Fred Hoyle, obrýlený Angličan s kulatým obličejem, který přečetl Eddingtonovu populární knihu
  Hvězdy a atomy už ve svých dvanácti letech, přišel do Cambridge v roce 1933, v době, kdy Eddington
  rozvíjel svou teorii hvězd a přel se s mladým Chandrou o konečný osud bílých trpaslíků. Byl to pro
  něho protipól ostatního, zcela mu nevyhovujícího vzdělání, během kterého mu „bylo dovoleno se
  nechat unášet proudem“. Ale v Cambridge zazářil, jako student získal několik cen a doktorát pak
  získal v oblasti kvantové fyziky. V roce 1939 se stal asistentem na St. John’s College a získal
  prestižní výzkumné místo. Rozhodl se také změnit své zaměření a místo kvantové teorii se věnovat
  astrofyzice. Inspirován Eddingtonovou Vnitřní stavbou hvězd začal přemýšlet o problému, jak hvězdy
  hoří a jak získávají své palivo. Jeho pozdější práce na toto téma se stala klíčem k porozumění
  tomu, jak jaderné procesy ve hvězdách vedou k vytváření těžších prvků. 

  Hoyle přešel k astrofyzice v roce 1939, tedy v roce, kdy vypukla druhá světová válka. V důsledku
  toho se v následujících šesti letech jeho úsilí soustředilo především na práci pro vojenské účely,
  a to speciálně na vývoj radaru. Podobně jako ve Spojených státech ty nejbystřejší mozky přitáhl
  projekt vývoje atomové bomby, ve Velké Británii se během války řada talentů soustředila právě na
  technologie spojené s rádiovými vlnami. Spousta nových překvapivých myšlenek v této oblasti vedla
  k praktickému využití při sledování letadel, lodí a ponorek. Dědictví válečného vývoje radaru je
  zde stále s námi, moderní společnost se přímo koupe v rádiových vlnách. Z technologií založených
  na elektromagnetických vlnách jmenujme například rozhlas a televizi, bezdrátový přenos signálu pro
  mobilní telefony, řízení letadel a raketových střel. 

  Při své práci na radaru se Hoyle setkal se dvěma mladými fyziky, Hermannem Bondim a Thomasem
  Goldem. Bondi byl židovský emigrant z Rakouska. Když mu bylo šestnáct let, navštívil jednu z
  populárních přednášek, kterou během svého pobytu ve Vídni přednesl Arthur Eddington. Vzbudila v
  něm touhu jít studovat matematiku do Cambridge a ta se nakonec uskutečnila. Bondi si intelektuální
  ovzduší Cambridge zamiloval. Později napsal: „Zde bych chtěl žít až do konce svých dnů.“ Jako
  státní příslušník nepřátelského státu byl Bondi v první fázi druhé světové války internován v
  Kanadě. Zde se seznámil s jiným židovským emigrantem z Vídně Thomasem Goldem, který studoval v
  Cambridgi techniku, jehož ale také vzrušovaly Eddingtonovy populární knihy. Když byli nakonec
  uvolněni z internace, pracovali s Hoylem na válečných úkolech. Ve volném čase však diskutovali o
  novém vývoji v kosmologii a astrofyzice, přičemž každý z nich uplatňoval trochu jiný přístup k
  problémům. Hoyle byl plný odvážných intuitivných myšlenek, Bondi preferoval matematický přístup a
  Gold byl pragmatik. 

  Když válka skončila, vrátili se všichni do Cambridge, každý na jinou kolej. Poválečná Cambridge
  byla drsnější a pracovníci univerzity se obměnili. Řada učenců ji opustila, válečné zkušenosti
  způsobily, že mnoho vědců budovalo svou kariéru mimo akademickou oblast. Ceny realit však během
  války vzrostly v důsledku přílivu pracovníků zvenčí. Bondi a Gold skončili ve společném domě
  nedaleko za městem. Hoyle často trávil všední dny u nich a do svého venkovského domu se vracel jen
  na víkendy. 

  Večery trávil Hoyle většinou s Bondim a Goldem, které nutil do diskusí o myšlenkách, jež ho právě
  napadly. Gold líčí, že „Hoyle … se někdy opakoval, někdy byl až nepříjemný, stále se k nějakému
  bodu vracel bez jasného důvodu“. Jednou z jeho posedlostí byla Hubbleova pozorování rychlosti
  expanze vesmíru. 

  Během let, jež uplynula od Hubbleových a Humasonových pozorování de Sitterova efektu, Friedmannův
  a Lemaîtrův expandující vesmír v astrofyzice pevně zakotvil. Lemaîtrův primordiální atom byl
  příliš ezoterický a jeho existence nebyla podpořena žádným pozorováním, ale co se týče jeho modelu
  vesmíru, o tom se soudilo, že je v jádru správný. Astrofyzici přijímali obraz, podle kterého se
  vesmír od určitého času rozpíná, a to, jak vstoupil do existence, se nechávalo jako problém, jenž
  se nějak vyřeší v budoucnu. A tento model byl bezpochyby velkým úspěchem astrofyziky a obecné
  teorie relativity. 

  S Friedmannovým a Lemaîtrovým vesmírem byl však jeden veliký problém, se kterým si v té době
  nevěděli astrofyzici rady a který se objevil hned, když Hubble provedl svá převratná měření.
  Hubble totiž ze svých měření vyvodil, že rychlost expanze je přibližně 500 kilometrů za sekundu na
  megaparsek. To znamená, že galaxie, jež je od nás vzdálená jeden megaparsek (tedy asi 3 miliony
  světelných let), by se od nás měla vzdalovat rychlostí 500 kilometrů za sekundu, dvakrát
  vzdálenější galaxie pak rychlostí 1000 kilometrů za sekundu atd. Z této veličiny, které se dnes
  říká Hubbleova konstanta, se dalo podle Friedmannových a Lemaîtrových vývojových modelů určit, jak
  dlouho expanze již trvá, jinak řečeno, jak je vesmír starý - vycházelo, že je mu přibližně
  miliarda let. 

  To se může zdát jako pořádně dlouhá doba, jenže takový věk vesmíru byl nedostatečný. Už v roce
  1920 se podařilo z radioaktivního datování určit, že naše Země je starší než dvě miliardy let. A
  práce astronoma Jamese Jeanse dokonce naznačovaly, že stáří hvězdokup je mezi stovkou a tisícovkou
  miliard let. Hodnoty pro stáří hvězdokup byly sice později podstatně sníženy, ale přesto nebylo
  pochyb, že to vypadá, jako by byl vesmír mladší, než je materiál, který obsahuje. To prostě
  nemohlo být pravda, ale zdálo se, že z tohoto paradoxu není úniku. V roce 1932 to shrnul Willem de
  Sitter: „Obávám se, že nám nezbývá, než tento paradox přijmout a snažit se si na něj zvyknout“. A
  do doby, kdy se o expandující vesmír začali zajímat Hoyle, Bondi a Gold, se situace nezlepšila. 

  Když se cambridgeské trio začalo zabývat kosmologií, paradox stáří se zdál být nápadným selháním
  Friedmannových a Lemaîtrových modelů. Hoyleovi, Bondimu a Goldovi ale vadilo mnohem více něco
  mnohem hlubšího a týkalo se to základních představ. Když se ve Friedmannových a Lemaîtrových
  modelech postupovalo zpátky v čase, tak počátek vesmíru odpovídal tomu, že jeho veškerý obsah byl
  soustředěn v jediném bodě. Jinými slovy, prostor, čas a hmota začaly existovat v jednom jediném
  okamžiku. To bylo pro Hoylea a jeho přátelé nepřijatelné. Jak to vyjádřil Hoyle, „jednalo se o
  iracionální proces, který se nedá popsat vědeckým jazykem“. Mohou zákony fyziky popsat stvoření
  něčeho z ničeho? To se zdálo nemyslitelné. Představa, že základní předpoklady nemohly být nijak
  konfrontovány s pozorováním, byla pro Hoylea „zřetelně nepřijatelná koncepce“. Jeho rozhodné
  námitky připomínaly původní Eddingtonovy argumenty proti Lemaîtrově primordiálnímu vejci. 

  Nakonec to byl film Smrt noci z roku 1945, který Hoylea a jeho kolegy dovedl k novému pohledu na
  vesmír. Film byl horor s kruhovou strukturou, konec se elegantně blížil počátku. Bez skutečného
  počátku a konce to byla jako by klaustrofobní vize nekončícího vesmíru. A to se líbilo Hoyleovi,
  Bondimu a Goldovi. Co když byl vesmír právě takový? Pak by neexistoval žádný počáteční čas ani
  primordiální vejce. 

  Bondi a Gold na problém počátečního času - nebo „velkého třesku“, o kterém později hovořil Hoyle -
  pohlíželi z téměř abstraktního estetického hlediska. Po staletí se popis vesmíru vzdaloval od
  představy, že v něm zaujímáme nějaké význačné postavení. Vesmír v pojetí Friedmanna a Lemaîtra,
  právě tak jako podle starší představy Einsteina, neměl nikde žádný význačný bod, byl prostorově
  homogenní. Mezi všemi prostorovými body panovala plná demokracie. Nebylo by ale možné tento
  předpoklad, nazývaný kosmologický princip, ještě zobecnit? Proč nevyslovit předpoklad, že
  ekvivalentní jsou nejen všechny prostorové body v určitém čase, nýbrž že jsou ekvivalentní všechny
  body v prostoročase, že vesmír se jeví stejný v každém bodě i v každém okamžiku? Nebyl by žádný
  počátek, nýbrž jen věčný vesmír, který by zůstával po všechny časy ve stacionárním stavu. 

  Hoyle se ujal úkolu vyjasnit všechny detaily takového přístupu. Ve Friedmannově a Lemaîtrově
  vesmíru se s expanzí rozřeďovala energie, její hustota s časem pomalu klesala. Měl-li být vesmír
  ve stacionárním stavu, hustota energie musela zůstávat stále stejná, a jestliže se takovýto vesmír
  rozpínal, energie se v něm musela nějak doplňovat. A tak se Hoyle rozhodl Einsteinovy rovnice
  upravit, podobně jako to udělal Einstein, když vytvářel model statického vesmíru. Hoyle postuloval
  existenci něčeho, co nazval kreační pole, neboli C-pole. Hoyleův vesmír byl tak udržován ve
  stacionárním stavu tajemným zdrojem energie, který ovšem nikdo nikdy nepozoroval. V Hoyleově
  vesmíru neplatil jeden z těch nejsvětějších fyzikálních zákonů - zákon zachování energie. Hoyle
  argumentoval, že to není tak velká oběť, protože stačí, aby vznikl „jeden atom za sto let v objemu
  velkém jako Empire State Building“. To je skoro nic. 

  V roce 1948 vyšly v časopise Monthly Notices of RAS dva články na toto téma - jeden od Hoyla,
  autory druhého byli Bondi a Gold. Jejich přijetí se různilo. Jeden z otců kvantové mechaniky
  Werner Heisenberg, který byl právě v Cambridgi, když tam Hoyle představoval své C-pole, prohlásil,
  že to byla ta nejpodnětnější myšlenka, kterou v Cambridgi získal. Zato E. A. Milne, oxfordský
  profesor matematiky, ji okamžitě zavrhl. Prohlásil: „Nevěřím, že hypotéza spojitého tvoření hmoty
  je nezbytná, stejně jako si nemyslím, že je dostatečný důvod k předpokladu, že vesmír byl stvořen
  v nějakém určitém okamžiku.“ Max Born, který byl v Göttingen školitelem Roberta Oppenheimera,
  nemohl prostě Hoyleův návrh přijmout, protože „přežil-li nějaký zákon všechny revoluční změny ve
  fyzice, tak je to právě zákon zachování energie“. A největší guru obecné relativity Albert
  Einstein věnoval Hoyleovu modelu jen malou pozornost, nazval jej „romantickou spekulací“. To, co
  se cambridgeskému triu zdálo být jednoduchým a přirozeným řešením fundamentálního kosmologického
  problému, bylo odmítnuto jako příliš exotické a zbytečné. Hoyle byl frustrován reakcí kolegů,
  kterou vnímal jako pošetilou. Jeho slovy, byl „unaven vysvětlováním fyziky, matematiky a logiky
  tupým myslím“. 

  A pak se mu naskytla skvělá příležitost, jak propagovat model mnohem účinněji než článkem nebo
  seminářem. BBC plánovala sérii rozhlasových přednášek cambridgeského historika Herberta
  Butterfielda. Ten ale na poslední chvíli odřekl a tak byl pozván mladý Hoyle, který měl určité
  zkušenosti s rozhlasem, aby ho nahradil. Tématem měl být vesmír a kosmologie, celkem pět
  přednášek. V nich mohl Hoyle rozvinout otázky kosmologie, problém mladého vesmíru se starými
  galaxiemi a ukázat, že Friedmannův a Lemaîtrův vesmír přinesl více problémů, než kolik jich
  vyřešil. A samozřejmě mohl propagovat ctnosti svého stacionárního vesmíru. Mohl pominout všechny
  konvenční přístupy a celé zemi předložit své myšlenky jako hotovou věc. Jeho teorii bude teď znát
  kdekdo. 

  Jeho přednášky byly neuvěřitelně úspěšné a Hoyle se stal známou osobností a mediální hvězdou.
  Veřejnost vřele přijala jeho popis vesmíru a ten zaujal své místo v obecné představě. Ale tím, že
  využil této příležitosti k propagaci svého vlastního modelu na úkor mezi vědci uznávanějšího
  modelu Friedmanna a Lemaîtra, popudil Hoyle své kolegy, což jejich nepříznivou reakci na
  stacionární vesmír ještě zesílilo. Hoyleovi se sice podařilo rozšířit svou myšlenku ve veřejnosti,
  vědečtí kolegové ji však tím silněji odmítali. Hoyle později uváděl, že na počátku padesátých let
  měl problémy s publikací svých článků. 

  Nicméně stacionární vesmír nakonec zaujal své místo po boku Friedmannova a Lemaîtrova vesmíru jako
  možná alternativa. Byl v souladu s velkými astronomickými objevy dvacátých let minulého století,
  jež sprovodily ze světa Einsteinův statický vesmír. Za pár let se ovšem otevřelo další okno do
  vesmíru a vrhlo na všechny tyto modely nové světlo. 

  „Myslím, že není neodůvodněné říct, že motivací [Martina] Rylea pro rozvinutí programu mapování
  rádiových zdrojů … byla především odplata,“ napsal Hoyle na adresu svého někdejšího
  spolupracovníka. Říci něco takového nebylo moc hezké, ale bylo v tom možná zrnko pravdy. Martin
  Ryle byl totiž nevy- rovnaný, popudlivý člověk, soutěživý a podezřívavý. I v Cambridgi se izoloval
  od svých kolegů, pracoval blízko radioteleskopů v jejich servisní budově, kterou bylo bývalé
  nádraží Lord’s Bridge. Byl to „přístřešek v polích“, jak to nazval jeden z jeho kolegů. Ryleova
  kariéra byla plná úspěchů, v roce 1972 získal prestižní postavení královského astronoma a v roce
  1974 Nobelovu cenu, přesto se ale stále choval, jako by ho ovládal stihomam, a podobnou náladu
  šířil i ve svém týmu. 

  Martin Ryle též pocházel z radarové generace. Byl synem cambridgeského profesora, studoval v
  Oxfordu, kde v roce 1939 absolvoval s výborným výsledkem. Podobně jako Bondi, Gold a Hoyle, i
  Martin Ryle pracoval během války na radaru a vyvinul způsob rušení německého radaru i naváděcího
  systému německých raket. Po válce odešel do Cambridge a zde uplatnil své zkušenosti s radarem při
  budování nového odvětví - radioastronomie. V této oblasti získal v lecčems dominantní postavení.
  Nebyl ale jediný, kdo se radioastronomií zabýval. Bernard Lovell, který též strávil válku prací na
  radaru a po válce působil v Manchesteru, začal budovat jeden z největších řiditelných
  radioteleskopů ve světě. A v Austrálii radioastronomickou skupinu založil Joseph Pawsey, který
  během války pracoval na radaru pro australské námořnictvo. 

  První kroky v radioastronomii udělal počátkem třicátých let minulého století Karl Jansky, inženýr
  Bellových telefonických laboratoří v New Jersey. Měl najít příčinu, proč konverzace pomocí
  vysílaček, ba i rozhlasové vysílání, jsou někdy tak špatně srozumitelné v důsledku šumu neznámého
  původu. Jansky se vůbec nezajímal o tajemství vesmírného prostoru, chtěl prostě vylepšit poslech
  rádia. 

  Rádiové vlny se chovají velmi podobně jako vlny světelné, jen jejich vlnové délky jsou miliardkrát
  větší než u viditelného světla. Světlo, které vidíme a které představuje podstatnou část záření
  Slunce, má vlnové délky více než milionkrát kratší než jeden metr. Rádiové vlny mají proti tomu
  vlnové délky obrovské, od milimetrů až po stovky metrů. Jansky zjistil, že zdroj neznámého šumu
  leží ve vesmíru. Mléčná dráha vyzařuje ve dne v noci značné množství rádiových vln. Slunce je sice
  mnohem jasnější než celá Mléčná dráha, rádiových vln však zase tolik nevysílá. V roce 1933 Jansky
  publikoval článek „Elektrické poruchy zjevně mimozemského původu“, kde systematicky studoval
  všechny možné zdroje poruch a vytvořil mapu zobrazující, odkud mimozemské rádiové vlny přicházejí.
  Jeho metoda zároveň odhalila nový způsob, jak pozorovat vesmír. Ne pomocí obřích optických
  teleskopů na vrcholech hor, nýbrž pomocí drátěného pletiva, oceli a kovových talířů rozmístěných
  na otevřených pláních. Astronom nemusí jen opticky sledovat slabounké světelné zdroje daleko od
  nás, informaci mu přinášejí i rádiové vlny přicházející ze vzdáleného prostoru. 

  Janskyho objev však vědci většinou ignorovali. Když vedení Bellových laboratoří navrhl stavbu nové
  vylepšené antény, byl jeho projekt zamítnut, protože astronomie nebyla v oblasti zájmu laboratoří.
  A tak se Jansky začal věnovat jiné problematice. Osobitý radioinženýr a amatérský astronom z
  Wheatonu v Illinois jménem Grote Reber se dočetl o Janskyho výzkumech v časopise Popular Astronomy
  a začal stavět na dvoře svého domu ve Wheatonu větší a lepší anténu, než měl k dispozici Jansky.
  Reberova anténa měla devítimetrový talíř s ocelovou konstrukcí před ním, kde bylo zařízení
  zachycující odražené rádiové vlny. Byl to první radioteleskop podobný těm, které vídáme dnes. S
  ním Reber proměřil podrobněji rádiové záření Mléčné dráhy a vypracoval detailní mapu rádiového
  nebe. Své výsledky zaslal do časopisu Astronomical Journal. Chandru, který byl v té době jeho
  šéfredaktorem, výsledky zaujaly a udivila ho Reberova vytrvalost, takže článek nechal otisknout. V
  roce 1940 se tedy objevil Reberův článek „Kosmický šum“, ve kterém byly jeho mapy rádiového nebe. 

  Reberovy mapy Mléčné dráhy byly zajímavé a pomohly přesněji určit, odkud jsou tajemné vlny vlastně
  vyzařovány. Ale Reberova měření odhalila ještě něco velmi pozoruhodného. Na mapě bylo několik
  izolovaných bodů, odkud vycházelo obrovské množství rádiových vln. Reberovi se podařilo určit
  jejich polohu v blízkosti různých souhvězdí - Labutě, Kassiopeie a Býka - a v těch místech se
  nevyskytovaly objekty vyzařující viditelné světlo. Reber ve skutečnosti objevil nový typ
  astronomických objektů, kterým se začalo říkat rádiové zdroje či rádiové hvězdy. 

  „Kosmický šum“otevřel nové okno do vesmíru. Před novou generací astronomů leželo teritorium, které
  si žádalo průzkum, a Martin Ryle byl k tomu připraven. Ryle se svými cambridgeskými
  spolupracovníky ve čtyřicátých letech začal společně s Lovellovou a Pawseyovou skupinou mapovat
  vesmír pomocí rádiových vln. Ryle využil svých zkušeností s technikou, se kterou se seznámil při
  práci s radarem, a navrhl novou generaci radioteleskopů, díky které se Cambridge stala jedním z
  nejvýznačnějších světových center radioastronomie. Ale ve svém důsledku to vedlo ke střetu s
  Hoylem a jeho spolupracovníky. 

  Martin Ryle byl více radioamatér a elektroinženýr než kosmolog, takže trochu udivovalo, že se
  pustil do křížku s „teoretiky“, jak posměšně nazýval Hoylea a jeho kolegy. Jenže k tomu došlo.
  Jeho první snahou bylo najít nejjasnější rádiové zdroje, jako byly ty, co pozoroval Reber, a určit
  přesně jejich polohu. Jenže se dopustil chyby. Zdálo se mu jasné, že tyto objekty musí být pevně
  zakotveny v Mléčné dráze. V roce 1950 v jasně sepsaném článku argumentoval, že většina rádiových
  zdrojů leží v naší galaxii. Možná existuje pár výjimek, ale ve své většině musí být blízko. To, co
  tvrdil, vypadalo zcela rozumně a dávalo to smysl. Své výsledky Ryle přednesl na setkání Královské
  astronomické společnosti v roce 1951. V auditoriu byli jeho kolegové z Cambridge Gold a Hoyle,
  kteří v diskusi ležérně prohlásili, že by se mohlo jednat o zdroje mimogalaktické. Ryle, který
  svou argumentaci pečlivě promýšlel, se cítil dotčen a Golda a Hoyla odbyl slovy: „Domnívám se, že
  teoretici špatně pochopili experimentální data.

  “Byla to srážka dvou kultur. Intelektuální teoretičtí astronomové, zběhlí v matematice a fyzice a
  nabízející elegantní - i když podivné - teorie vysvětlující vesmír jako celek zde stáli proti
  „kutilům“, rádiovým operátorům, kteří stavěli složité konstrukce a hráli si s elektronikou. Rylea
  zlobil postoj jeho kolegů, který vnímal jako povýšenou blahosklonnost. Domníval se, že rozumí
  empirickým datům lépe než ti, kteří jsou zvyklí pracovat jen s papírem a tužkou. K Ryleově
  neštěstí ale Hoyle a Gold měli pravdu, postupně se ukazovalo, že stále více objektů může být
  ztotožněno s objekty mimo Mléčnou dráhu. Skutečně byly extragalaktické a Ryle musel uznat, že
  teoretici datům rozumějí. 

  Jenže Ryle nepřijal svou porážku s pokorou. Jestliže tyto rádiové zdroje leží mimo galaxii, pak by
  mohly něco říct o vesmíru jako celku. Takže Ryle začal shromažďovat více a více dat a používat je
  proti Hoyleovu a Goldovu dítěti, stacionárnímu vesmíru. Postupoval tak, že začal uspořádávat
  rádiové zdroje podle jejich jasu a tuto funkci se snažil spojit s vlastnostmi vesmíru jako celku.
  Čím je zdroj vzdálenější, tím je pochopitelně méně jasný, takže jas zdroje se dá pokládat za
  indikátor jeho vzdálenosti. Vesmír je veliký a je v něm spousta místa, a tak se dá očekávat, že
  budeme pozorovat více málo jasných, tedy vzdálenějších zdrojů než těch blízkých, jasných. Ukázalo
  se, že poměr jasných zdrojů ke slabým je dobrým indikátorem toho, v jakém vesmíru žijeme. Když se
  díváme na vzdálené zdroje, tak světlu trvalo déle, než nás dostihlo, a proto vidíme vesmír v době,
  kdy byl mladší. Jestliže bychom žili v Hoyleově, Goldově a Bondiho stacionárním vesmíru, hustota
  zdrojů by zůstávala konstantní a tak počet zdrojů v určité oblasti by byl přímo úměrný jejímu
  objemu. Ve vyvíjejícím se vesmíru, jako je vesmír Friedmanna a Lemaîtra, byla v minulosti hustota
  větší, takže slabších, vzdálených zdrojů by mělo být mnohem více, než těch blízkých, jasných. Z
  poměru počtu slabých zdrojů k těm jasným by mělo být možné určit, zda žijeme ve vesmíru
  stacionárním, nebo ve vesmíru s velkým třeskem. 

  Ryle publikoval seznam téměř dvou tisíc rádiových zdrojů v takzvaném katalogu 2C (C znamená
  Cambridge), který byl nadstavbou mnohem menšího seznamu padesáti zdrojů, známého jako katalog 1C.
  K jeho uspokojení zde bylo mnohem více slabých zdrojů ve srovnání s jasnými a to nebylo
  konzistentní s teorií stacionárního vesmíru. Ryle v této statistice viděl nemilosrdného zabijáka
  Hoyleovy teorie a okamžitě tomu udělal patřičnou reklamu. V prestižní přednášce, o kterou byl
  požádán v Oxfordu a kterou proslovil v květnu 1955, hrdě prohlásil na adresu svých rivalů:
  „Přijmeme-li závěr, že většina rádiových zdrojů leží mimo naši galaxii, a tomuto závěru je těžké
  se vyhnout, pak se zdá, že pozorování nemohou být v žádném případě vysvětlena teorií stacionárního
  vesmíru“. Zdálo se, že Ryle model Hoylea a Golda zcela rozbořil. 

  Po Ryleově oxfordské přednášce byl Hoyle se svými spolupracovníky v defenzivě. Hoyle vzal data
  vážně, ale Gold bral výsledky s určitým podezřením. Říkal Hoyleovi: „Nevěř jim, může v tom být
  spousta chyb, nelze je brát vážně.“ A Gold měl tehdy pravdu. Tentokrát Rylea zaskočila jeho
  vlastní kohorta, ti samí kutilové, kteří přetvářeli radioastronomii v důvěryhodnou vědu. Dva mladí
  australští astronomové, Bernard Mills a Bruce Slee ze Sydney, analyzovali znovu data katalogu 2C a
  došli ke zcela jiným závěrům než Ryle. Nesnažili se přijít s katalogem tisíců zdrojů, jenž by mohl
  konkurovat tomu Ryleovu, nýbrž se soustředili na malou podmnožinu z celkového přehledu, asi na tři
  sta zdrojů, a ty proměřili se všemi detaily. Tento malý katalog byl zvolen tak, aby se s Ryleovým
  překrýval a mohl se s ním porovnat. 

  Když Mills a Slee publikovali své výsledky, zcela tím zpochybnili důvěryhodnost Ryleových dat. Ve
  svém článku uvedli, že „katalog detailně porovnali se současným cambridgeským katalogem… a
  shledali, že oba katalogy jsou nekonzistentní“. Mills a Slee to vysvětlovali „nízkou rozlišovací
  schopností cambridgeského radiointerferometru“. Ryleovy výsledky prostě nebyly dostatečně přesné -
  Mills a Slee pracovali s přesnějším radioteleskopem a jejich výsledky stacionární vesmír
  nevylučovaly. A do debaty vstoupili ještě vědci z Jodrell Bank, z konkurenční skupiny ve Spojeném
  království: „Radioastronomie musí ještě udělat podstatný pokrok, než bude moci říci kosmologům
  něco hodně užitečného.“ Zdálo se, že radioastronomové se na výsledcích svých měření nemohou
  shodnout, takže jich lze těžko použít k testování kosmologických modelů. Hoyle a jeho
  spolupracovníci mohli zase jásat. 

  Ryle počal v Cambridgi pracovat na další generaci katalogu zdrojů. Jeho i celý tým mrzel debakl,
  který utrpěli se svými výsledky, a tak příští tři roky strávili nad novým katalogem, nazvaným jak
  jinak než katalog 3C. Ryle pevně věřil, že nové výsledky bez milosti zahubí nesmysl, který
  propagoval Hoyle a jeho spolupracovníci. Když byl v roce 1958 katalog 3C konečně na světě, Martin
  Ryle byl přesvědčen, že teď se má o co opřít. Bitva však stále nebyla vyhraná. Bondi zůstával
  skeptický a poukazoval na to, že Ryle má tendenci tvrdit, že jeho výsledky jsou lepší, než ve
  skutečnosti jsou. Bylo skutečností, že Ryle už několikrát tvrdil, že vesmír Hoylea a jeho
  spolupracovníků vyvrátil, a když pak někdo data analyzoval podrobněji, ukázalo se, že chyby měření
  jsou ve skutečnosti větší, než Ryle uváděl, takže stacionární vesmír se ocitl znovu ve hře. Jak
  Bondi veřejně prohlásil, „za posledních deset let se to stalo více než jednou“. 

  V únoru 1961 předložil Ryle svou analýzu výsledků shromážděných teď už v katalogu 4C na zasedání
  Královské astronomické společnosti. Argumentoval, že tyto výsledky jsou skutečně neslučitelné s
  modelem stacionárního vesmíru, že jasných rádiových zdrojů je příliš málo v poměru k těm slabým.
  Jeho slovy, „pozorování poskytují nezvratné svědectví proti teorii stacionárního vesmíru“. Noviny
  se chopily Ryleova prohlášení a přinesly články s titulkem „Bible měla pravdu“ či titulky
  podobnými, ve kterých se hovořilo o existenci počátečního okamžiku stvoření. Týmy v Austrálii a
  Spojených státech tyto výsledky podpořily a tak se zdálo. že Ryle skutečně zvítězil. 

  Hoyle a jeho spolupracovníci byli znepokojeni, avšak nepřestali být přesvědčeni o své pravdě.
  Jak Bondi řekl pro New York Times krátce po tom, co Ryle zveřejnil svůj rozbor, „určitě si
  nemyslím, že spojité tvoření hmoty je mrtvé“, k čemuž dodal: „S podobným tvrzením přišel profesor
  Ryle už v roce 1955 a pozdější pozorování ukázala, že bylo nesprávné.“ V Ryleově úsilí vyvrátit
  stacionární model bylo něco iracionálního, i když se pozorovací údaje rok co rok zlepšovaly. Pro
  Hoylea, Bondiho a Golda však rádiová data model nezabila, alespoň v té době ne.

  Boj mezi Hoylem a Rylem, jenž se odehrával v Cambridgi, se může zpětně jevit jako zbytečné
  rozptylování v nezadržitelném vývoji obecné teorie relativity a kosmologie. Navíc mimo Spojené
  království jevila ve skutečnosti zájem o Hoyleův model jen hrstka lidí. Mnoha vědcům se celá
  debata zdála bláznivá, téměř nevědecká, protože v ní velkou roli hrálo osobní napětí a vendeta
  mezi jejími aktéry. Návštěvníci Cambridge si všímali jedovaté atmosféry mezi Ryleovým týmem a
  Hoyleovou skupinou. 

  Ale toto soupeření vedlo k významnému vědeckému pokroku. Fred Hoyle si získal pověst jednoho z
  největších astrofyziků druhé poloviny dvacátého století. Spolu s Williamem Fowlerem a Geoffreyem a
  Margaretou Burbidgeovými ze Spojených států nakonec vyvinuli skvělou teorii tvoření prvků ve
  hvězdách. Někteří lidé se domnívají, že to byla jeho osobitá povaha, která způsobila, že se
  nepodílel na Nobelově ceně za rok 1983 (získali ji Subramanyan Chandrasekhar a William Fowler). V
  roce 1973 opustil Cambridge, přesídlil do Lake District a věnoval se psaní románů. 

  Hermann Bondi vytvořil skvělou relativistickou skupinu na londýnské King’s College a Thomas Gold
  se stal vedoucí osobností při stavbě největšího radioteleskopu v Arecibo na Portoriku. Skupina
  Martin Rylea získala pověst, že má sklon k tajnůstkářství a paranoidnímu chování, v následujících
  dvou desetiletích však stála u několika významných objevů v radioastronomii. Martin Ryle získal
  v roce 1974 Nobelovu cenu. Rozvoj radioastronomie a tajuplná podstata rádiových zdrojů sehrály
  důležitou roli v rozvoji obecné relativity, která vstoupila do nové fáze.

\section{Wheelerovštiny}\label{feyIchIIIsecVIII} 
  John Archibald Wheeler se dostal k obecné relativitě prostřednictvím nukleární fyziky a kvantové
  teorie. Na jaře roku 1952 se Wheeler začal zamýšlet nad osudem hvězd tvořených neutrony. Neutrony
  jsou stavebními kameny jádra a Wheeler se jimi ve svém předchozím bádání hodně zabýval. Upoutala
  ho Oppenheimerova předpověď, že výsledkem gravitačního kolapsu hvězdy může být singularita, tedy
  bod, ve kterém hustota a křivost vzrostou do nekonečna a který se během kolapsu vytvoří ve středu
  hvězdy. Wheelerovi se zdálo, že se singularitami něco není v pořádku, že nemohou být reálnými
  fyzikálními objekty a že musí existovat nějaký způsob, jak se jich zbavit. Aby porozuměl této
  bizarní předpovědi, Wheeler se rozhodl, že se obecnou relativitu dobře naučí a jako nejlepší cesta
  k tomu mu připadalo vyučovat ji studenty v Princetonu. A tak v roce 1952 Wheeler přednášel první
  kurz obecné relativity v princetonském fyzikálním ústavu, domově Einsteina, Gödela a Oppenheimera.
  Do té doby se obecná relativita pokládala za příliš abstraktní disciplínu, vhodnou jen pro
  matematické oddělení. Byl to důležitý start, Wheeler na tuto přednášku později vzpomínal jako na
  své „první krůčky v teritoriu, které uchvátilo mou imaginace a poutalo můj badatelský zájem po
  zbytek mého života“. 

  Wheeler byl „radikální konzervativec“, jak ho označil jeden z jeho studentů. Rozhodně
  konzervativně vypadal, vždy bezvadně oblečený, v tmavém obleku a s kravatou, pečlivě učesaný, s
  vyleštěnými botami - dokonalý obrázek tradičního, trochu konvenčního džentlmena. Plně se věnoval
  svým studentům, byl vždy loajální ke svým spolupracovníkům a byl staromódně zdvořilý a pozorný.
  Někdy ovšem pronášel ty nejpodivnější výroky, často i tajemné věty o záhadách vesmíru, které z něj
  dělaly spíše gurua New Age či osvíceného hippie. 

  Jako vědec byl Wheeler směsicí snílka a praktika. Na jedné straně ho zajímaly výbušniny a různá
  mechanická zařízení, na druhé straně ho okouzlovala nová magická pravidla teorie atomů. Na
  univerzitě studoval technický obor a přitom objevil plnou krásu matematiky. Jeden z jeho učitelů
  matematiky mu dal radu jak si počínat, narazí-li na problém. Wheeler vzpomínal, že když se při
  výkladu objevil nějaký matematický problém, tento přednášející říkal, že on jako Ir si s
  překážkami poradí tak, že je obejde. A tato rada ovlivnila styl Wheelerovy práce na celý život.
  Odvážně se pustil do jakéhokoli problému a naučil se, co k jeho řešení potřeboval, až když to
  skutečně bylo potřeba. V roce 1932 mu bylo pouhých jednadvacet let a už měl doktorát z kvantové
  fyziky. 

  John Wheeler se stal kvantovým fyzikem v době, kdy velké objevy Schrödingera a Heisenberga
  začínaly přinášet ovoce. Jako mladý asistent v Princetonu zkoumal s proslulým dánským fyzikem
  Nielsem Bohrem kvantové vlastnosti jader a jejich interakci. Wheelerova a Bohrova práce o štěpení
  jader byla zveřejněna ve stejný den, kdy vyšla práce Oppenheimera a Snydera o gravitačním kolapsu,
  a stala se jedním ze základních kamenů celého projektu Manhattan. 

  Wheelerův konzervativismus pramenil z jeho vášnivé víry v americký způsob života, jeho instituce a
  jeho hodnoty. Na projektu atomové bomby začal pracovat hned po útoku na Pearl Harbour, podílel
  se na stavbě obřích reaktorů, potřebných k výrobě plutonia pro bomby. Jeho bratr padl v boji v
  roce 1944 a Wheeler měl do konce života pocit viny, že neučinil dost, aby se na výrobě atomové
  bomby začalo pracovat dříve. Jak později říkal svým kolegům, kdyby byla bomba dokončena dříve,
  mohla se použít už během války s Německem. Ztráty na životech by byly obrovské, ale přesto ne tak
  hrozné, jaké byly v posledním roce války. 

  Jeho vlastenectví někdy vedlo ke sporům s kolegy. V padesátých letech minulého století byl přizván
  Edward Teller k práci na projektu Matterhorn, jenž měl zajistit Spojeným státům získání vodíkové
  bomby, termonukleární zbraně fungující na principu jaderné fúze. Nabídku přijal, i když řada jeho
  kolegů, včetně Roberta Oppenheimera, byla silně proti. Wheeler byl jedním z mála fyziků, kteří se
  zdrželi podpory Oppenheimera při jeho obvinění z narušení národní bezpečnosti.

  V politice byl sice konzervativní, ale ve vědě byl velmi osobitou figurou, která přicházela s
  těmi nejpodivnějšími nápady, jež bořily zavedené vědecké koncepce. Jedním z Wheelerových studentů
  v Princetonu byl Richard Feynman, brilantní mládenec z New Yorku, který se stal jednou z hlavních
  postav poválečného vývoje kvantové fyziky. Pod Wheelerovým vedením došel ke zcela revolučnímu
  způsobu popisu vzájemného působení částic, který dával jasný návod k výpočtu jejich chování na
  pozadí prostoročasu. Právě Wheeler naučil Feynmana nezvyklému uvažování a odvážnému prosazování
  myšlenek. 

  Wheeler byl tím pravým, kdo mohl poskládat různé dílčí výsledky obecné teorie relativity do
  jednotného obrazu. Byl praktický, ale také vizionář. Byl konzervativní v tom smyslu, že byl pln
  respektu k fyzice a astrofyzice, která k teorii vedla, na druhé straně však byl nakloněn neotřelým
  pohledům na problémy a rozdílným přístupům k nim. A především to byl inspirující učitel, který
  vzdělával a podporoval novou generaci fyziků, jež vnesla do obecné teorie relativity nového ducha.

  Když se Wheeler obecnou relativitu naučil, zcela ho pohltila. Byla tak elegantní a její
  experimentální ověření bylo natolik přesvědčivé, že nemohla nebýt pravdivá, i když pozorovatelných
  důsledků nebylo na počet mnoho. To ale neznamenalo, že by se nesnažil zkoumat její meze. Věřil, že
  „dotáhneme-li teorii do extrémů, zjistíme, kde by se mohly skrývat trhliny v její struktuře“.
  Rozhodl se tedy důkladně prostudovat ty největší podivnosti obecné teorie relativity. V tomto
  procesu dával svým originálním nápadům vtipné nálepky - tomuto stylu se začalo říkat
  wheelerovština.

  Jedním z nápadů bylo zavést do obecné relativity elektrický náboj tak, aby tam skutečný náboj
  vlastně nebyl. „Náboj bez náboje“, to bylo wheelerovské heslo, které tuto myšlenku provázelo, a on
  ji rozvinul se svým talentovaným studentem Charlesem Misnerem. Představa byla taková, že na dvou
  vzdálených místech byly v prostoru díry, propojené navíc prostoročasovou trubicí, zvanou „červí
  díra“. Do tohoto tunelu se na jedné straně nořily elektrické siločáry, které vycházely otvorem na
  straně druhé. Vstupy do tunelu se pak chovaly jako kladný a záporný elektrický náboj, navzájem se
  přitahovaly. Ústí červích děr se pak chovala jako elektrické náboje velmi daleko od sebe, jedno
  jako kladný, druhé jako záporný, nebyla však spojena s žádnými skutečnými nabitými částicemi. Byla
  to důmyslná myšlenka, která se dala pěkně znázornit, skutečné matematické zpracování však bylo
  neobyčejně obtížné. 

  „Hmota bez hmoty“ - to byla další wheelerovština. Einsteinova teorie vysvětluje, jak spolu hmotné
  objekty interagují. Wheeler se však snažil dosáhnout Einsteinových výsledků, aniž by hmotné
  objekty zaváděl jako samostatné pojmy. Podle obecné teorie relativity světlo zakřivuje prostoročas
  stejně, jako to dělá hmota. Wheeler vyšel z myšlenky, že kdybychom stlačili svazek světelných
  paprsků tak, že by prostoročas dostatečně zakřivily, vzniklý útvar by se navenek choval jako
  hmotný objekt. Takový chumáč světla - Wheeler ho nazval geon - by měl váhu a přitahoval by jiné
  geony. Světelné paprsky by musely být svinuty do tvaru pneumatiky, chovaly by se ale tak, jako
  kdyby měly hmotnost, aniž by tam byly nějaké skutečné materiální částice. S jiným studentem Kipem
  Thornem se Wheeler snažil zjistit, zda takové objekty mohou existovat, aniž by se okamžitě
  rozpadly díky své nestabilitě. 

  A pak byl ovšem problém sňatku obecné teorie relativity s kvantovým popisem přírody. To byl
  problém velmi vážný a opravdu extrémní, takže Wheeler nemohl odolat pokušení ho vyřešit. Vyslovil
  několik svých představ o prostoročasu v malých škálách. Jednou z nich bylo, že kdybyste jej
  detailně sledovali na těch nejmenších měřítkách, objevovaly by se zvláštní efekty. Ve velkých
  rozměrech se prostoročas může zdát hladký, spojitě zakřivený přítomností masivních objektů (včetně
  Wheelerových geonů a červích děr). V opravdu malých rozměrech bychom však nalezli makroskopicky
  nepozorovatelnou hrubozrnnost. Skutečně neobyčejně výkonným mikroskopem, jaký dodnes není k
  dispozici, bychom však viděli turbulentní zmatek, vše smíchané dohromady. To si vynucují kvantové
  relace neurčitosti, v malých měřítkách by měl prostoročas vypadat jako vířící pěna. Fundamentální
  hrubou strukturu prostoročasu nepozorujeme jen proto, že naše vidění je rozmazané. 

  Wheeler však nepřicházel jen s podivnými nápady a odvážnými scénáři. Znepokojovaly ho také
  zvláštní singularity, které vykukovaly z prací Schwarzschilda, Oppenheimera a Snydera o kolapsu
  masivních hvězd - právě tahle problematika jej původně přitáhla k obecné teorii relativity.
  Wheeler pevně věřil, že singularity musí být podivné matematické artefakty, které v přírodě
  neexistují. Jak později vzpomínal: „Po mnoho let byla pro mne představa kolapsu do toho, co teď
  nazýváme černá díra, nepřijatelná. Prostě se mi nelíbila.“

  Snažil se to tedy napravit tím, že hledal nějaký nový fyzikální proces, který by přišel ke slovu v
  okamžiku, kdy kolaps začne hmotu v jádru hvězdy stahovat do nestydatě vysokých hustot. Bylo to pro
  něj úplně nové pole. I když byl Wheeler jedním z největších expertů na jadernou fyziku, fyzika
  neutronů v centru gravitačního kolapsu byla jiná věc. Potřeboval zjistit, co se stane, jsou-li
  neutrony směstnány mnohem hustěji než v Landauově či Oppenheimerově neutronové hvězdě nebo v
  kterékoli bombě, s níž se setkal během své práce pro americkou armádu. To vyžadovalo imaginaci a
  odhad a právě v takovém způsobu práce Wheeler vynikal. Ale přes veškerou kreativitu nedošli
  Wheeler a členové jeho týmu k jinému výsledku než Oppenheimer a Landau - i oni zjistili, že
  existuje maximální hmotnost, při jejímž překročení neutronová hvězda už nedokáže odolat vlastní
  gravitaci a skončí kolapsem. Ale Wheeler se nedokázal vyrovnat s představou singulárního konce a
  tak svůj boj nevzdával. 

  Čím více byl Wheeler fascinován obecnou teorií relativity a posedlý svým zápasem o vyloučení
  singularit, tím více studentů získával, aby ho na této cestě následovali. I oni, právě tak jako
  jejich učitel, byli ohromeni silou teorie a s nadšením zkoumali, co se dá v jejím rámci udělat.
  Rok co rok Wheelerova skupina vyprodukovala spoustu myšlenek. Některé z nich byly výstřední,
  některé rozumné, ale všechny bez výjimky podmanivé. Wheelerův vliv na vývoj obecné relativity
  ovšem překračoval hranice Princetonu. Jednou z jeho důležitých zásluh byla tichá podpora Bryce
  DeWitta z Univerzity Severní Karoliny v Chapel Hill. 

  Na Bryce DeWittovi bylo cosi obdivuhodného. Měl v sobě neobyčejnou sílu, jakousi přísnost
  připomínající starozákonního proroka, a povzbuzoval k velkým výkonům i své okolí. Nedovoloval sobě
  ani jiným nedbalost, vše se muselo dělat pečlivě a pořádně. Když se konečně nějaká myšlenka
  dopracovala k článku a publikaci, byla pevně podložená. 

  DeWitt byl také cestovatel, „vesmírný cestovatel“, jak si rád říkal. Jako mladík byl pilotem ve
  druhé světové válce. Graduoval na Harvardu a pak se pohyboval po celém světě - přes Princeton a
  Curych si odskočil i do Tata Institutu v Bombaji. Pobyt v Indii „nebyl moc užitečný v
  profesionálním smyslu, ale potěšil jeho toulavého ducha“, jak později vzpomínal jeden z jeho
  kolegů. 

  DeWitt se s manželkou usadil v Kalifornii. Jeho ženou se stala francouzská matematička Cécile
  DeWitt-Morette, se kterou se poznal v Princetonu. Nalezl práci v Lawrence Livermore Laboratory,
  kde pracoval na modelování nukleárních dělostřeleckých nábojů. Rodina ale potřebovala peníze na
  zakoupení domu, tak se DeWitt odhodlal napsat soutěžní esej - vítězná práce měla být odměněna 1000
  dolary. Tento esej změnil nejen osud DeWitta, ale i obecné teorie relativity.

  „Nadace pro výzkum gravitace“ byla duchovním dítětem Rogera Babsona, obchodníka, jehož vášní byla
  gravitace. Své bohatství získal na trhu s akciemi, kde aplikoval svou verzi Newtonových
  fyzikálních zákonů: „Co naroste, to zase poklesne. … Trh s akciemi spadne svou vlastní vahou.
  “Nebyla v tom žádná hluboká věda, ale Babson byl člověk s obsesí. Jeho starší sestra se utopila
  jako malé dítě a on z toho vinil gravitaci. Podle jeho verze tragédie „se nedokázala ubránit
  gravitaci, která ji uchvátila a stáhla pod hladinu jako drak“. Po celý život Babson do gravitace
  investoval. Sbíral newtonovské památeční předměty, podporoval různé podivné myšlenky, především
  ale založil Nadaci pro výzkum gravitace. 

  Babson si nadaci původně představoval tak, že bude podporovat každoroční soutěž o nejlepší esej na
  téma gravitace. Kandidáti předloží esej o rozsahu nejvýše 2000 slov, ve kterém budou navrhovat jak
  spoutat gravitaci a tak se propracovávat ke konečnému Babsonovu cíli - k objevu antigravitace.
  Nadace bude podporovat vývoj antigravitačních zařízení, jež budou izolovat od gravitace,
  absorbovat ji nebo odrážet. Atom byl nedávno spoután a Babsonovi se zdálo, že je ten pravý čas
  ovládnout i gravitaci. Svou soutěží o nejlepší esej chtěl povzbudit to nejlepší v poválečné
  fyzice. 

  Původní ohlas Babsonovy výzvy byl nevýrazný. Mezi rokem 1949 a 1953 se v esejích objevilo několik
  vágních návrhů. Jejich náměty byly roztodivné. Eseje pocházely dílem od akademických pracovníků,
  dílem od starších studentů a dílem od amatérů, kteří trápili své mozky, aby přišli s něčím, co by
  splňovalo Babsonovy požadavky. Jenže požadovaná tematika byla příliš exotická, a tak soutěž
  přinášela jen podivné výtvory nekompetentních osob a k vědeckému pokroku rozhodně nesloužila. 

  Babsonova výzva nebyla rozhodně nic budícího respekt - žádný rozumný fyzik si nemyslel, že se dá
  zkonstruovat nějaké antigravitační zařízení. Soutěž ale byla ozvěnou rostoucího zájmu o gravitaci.
  Po druhé světové válce se americká ekonomika úspěšně rozvíjela a optimismus se stával součástí
  každodenního života. Byl to začátek atomové éry, zrození věku nových technologií. Lidé z
  byznysu, kteří měli peníze na investování, hodně sázeli na gravitaci, která slibovala velké věci
  hned za nukleární energií. Na představě antigravitace bylo něco velmi přitažlivého, pocházela v
  podstatě z vědeckofantastické literatury. Už v roce 1901 H. G. Wells ve svém románu První lidé na
  Měsíci hovořil o magické substanci „cavoritu“, která obrací směr gravitace a vynese na Měsíc první
  cestovatele. 

  V tisku se v padesátých letech minulého století poměrně často objevovaly články o nových způsobech
  vesmírného cestování, jež by umožnila antigravitace, a titulky jako „Zázračné kosmické lodi
  umožněné přechytračením gravitace“ nebo „Budoucí letadla mohou vzdorovat gravitaci a přitom se
  obejít se bez vztlaku vzduchu“. Snilo se o „systémech gravitačního pohonu“. Populární tisk si
  představoval letadla či kosmické lodi, které užívají gravitace místo tryskového pohonu. I v New
  York Herald Tribune se objevil článek s názvem „Pokoření gravitace je cílem špičkových vědců v USA
  “, který popisoval, jak se letecké společnosti jako Convair, Bell Aircraft a Lear, Inc. věnují
  výzkumu gravitace, jež by „jednou mohla být ovládnuta tak jako světlo nebo rádiové vlny“.

  Společnost Glenn L. Martin Company (později známá jako Lockheed Martin) založila výzkumný ústav
  pro pokročilá studia. Jeho cílem bylo sledovat nové myšlenky v teoretické fyzice se speciálním
  důrazem na problémy gravitace a zkoumání možností gravitačního pohonu. Ústav najímal výborné
  fyziky a speciálně relativisty, jejichž výzkum měl směřovat k tomuto futuristickému cíli. Letectvo
  Spojených států přišlo se skromnějším a podstatně lacinějším projektem na půdě Aeronautické
  výzkumné laboratoře, sídlící na Wrightově-Pattersonově letecké základně v Daytonu v Ohiu. Tato
  laboratoř zaměstnávala také skupinu opravdových relativistů, kteří se však věnovali základnímu
  výzkumu v gravitaci a jednotných teoriích pole. V jejich výsledcích nebylo nic, co by se týkalo
  antigravitace, a po nějakou dobu byla tato skupina kvalitním výzkumným centrem v oblasti obecné
  teorie relativity, jež úspěšně konkurovalo jiný podobným uskupením rozesetým po světě. Letectvo
  pumpovalo peníze i do dalších skupin, které se zabývaly obecnou relativitou. Jen málo vědců bralo
  úsilí o objev antigravitace vážně a badatelé se zdržovali optimistických předpovědí v tomto směru,
  rádi však přijímali peníze, jejichž udělování bylo motivováno ezoterickými představami o podstatě
  reality. 

  Uprostřed této euforie byl DeWittův pokus o získání vítězství v Babsonově soutěži více než
  podivný: napadl totiž utkvělou představu sponzora. V soutěžním eseji, který v roce 1953 poslal
  Nadaci pro výzkum gravitace, DeWitt rázně zavrhl Babsonův ambiciózní cíl vyvinout „praktická
  zařízení, jako jsou reflektory gravitace, izolátory od gravitace nebo magické slitiny, které mění
  gravitaci v teplo“. Pomocí Einsteinovy teorie prostoročasu zdůvodnil, proč „jakýkoli frontální
  útok na problém spoutání gravitační síly po naznačené linii je ztrátou času“. Konstatoval, že „je
  možno bezpečně prohlásit, že všechna schémata gravitačního pohonu jsou nemožná“. DeWitt napadl
  bláznivé nápady a zvítězil. DeWittův esej byl rozhodně zcela jiný než práce předchozích
  soutěžících. Byla to skutečná věda rozhodně se zdržující spekulací a mluvící o skutečných
  vědeckých cílech, jimiž se musí příští výzkum v teorii gravitace zabývat. Byl to nesnadný úkol, a
  jak to DeWitt formuloval, „gravitaci se za posledních třicet let dostalo jen malé pozornosti“.
  Obecná teorie relativity byla „výjimečně obtížná“, vyžadovala „ezoterickou matematiku“ a „její
  základní rovnice je téměř beznadějné obecně řešit“. Ve skutečnosti „fenomén gravitace nedostatečně
  chápou i ty nejlepší mozky“. 

  Roger Babson se vůbec necítil uražen, první skutečně závažný soutěžící ho zaujal. Objevil se někdo
  seriózní, skutečný vědec, který mohl jeho soutěži dodat vážnosti. A DeWittův esej prestiži soutěže
  skutečně pomohl, v dalších letech počet soutěžících dramaticky vzrostl. V následujících
  desetiletích se výherci soutěže Nadace pro výzkum gravitace stala řada fyziků, kteří významně
  ovlivnili další vývoj obecné teorie relativity. Navíc se vítězné eseje zabývaly výhradně gravitací
  - antigravitace byla zapomenuta. DeWitt se později o výhře v soutěži vyjádřil tak, že to bylo
  „nejrychleji vydělaných 1000 dolarů, jaké kdy získal“, ale ve skutečnosti mu účast v soutěži
  přinesla mnohem a mnohem větší prospěch, než si uvědomoval. 

  Roger Babson měl přítele Agnewa Bahnsona, kterému gravitace též učarovala a který získal značné
  jmění prodejem průmyslových klimatizací. I on chtěl podpořit výzkum gravitace podobně jako Babson,
  nevěděl však, jak to nejlépe udělat. Když mu Babson ukázal DeWittův vítězný esej, pochopil, že
  DeWitt by mu mohl pomoci založit seriózní uznávanou instituci, ve které by skuteční myslitelé
  mohli sledovat své zájmy. V jedné inaugurační brožuře nově založeného Institutu pro fyziku pole,
  známého pod zkratkou IOFP anglického názvu Institute of Field Physics, Bahnson napsal: „V obecném
  povědomí je představa gravitace často spojována s různými fantastickými možnostmi. Institut však v
  současné době neočekává nějaké praktické dopady svého výzkumu.“Neplánovala se žádná antigravitační
  zařízení ani žádné letouny pohybující se díky gravitačnímu tahu. Své osobní fantazie si Bahnson
  uskutečňoval psaním sci-fi románů a skutečnou gravitaci ponechal vědcům. 

  Bahnson se obrátil na Johna Archibalda Wheelera o radu, jak při založení ústavu postupovat.
  Wheeler si dobyl ve Washingtonu velikou reputaci svými pracemi o jaderných zbraních a obecně byl
  brán jako zkušený fyzik, vždy ochotný podporovat vládu při projektech směřujících k národní
  obraně. Zpovzdálí sledoval DeWittovu kariéru a tiše podpořil myšlenku, aby do nového institutu
  umístěného na Chapel Hill v Severní Karolině byli jako první vědci pozváni Bryce a Cécile
  DeWittovi. 

  Mohlo by se zdát, že ústav vznikl jen pro uspokojení marnivosti sponzora, ale s Wheelerovou
  autoritou v pozadí a s DeWittovými jako jeho prvními badateli byl brán s respektem po celé zemi a
  písemně ho podpořila řada „šedých eminencí“, které tleskaly zařízení, kde se mohl provádět čistý
  výzkum, nenarušovaný požadavky průmyslu, armády či nového atomového věku. A jádrem výzkumného
  programu byla gravitace. 

  Inaugurační akcí Institutu pro fyziku pole byla konference organizovaná DeWittem v lednu 1957 pod
  názvem „Role gravitace ve fyzice.“ Konference byla zároveň i inaugurací nové éry. Její účastníci
  byli povětšině mladí a dosud méně známí, ale byla mezi nimi i řada nových vůdčích osobností v
  obecné teorii relativity. Sjeli se na Chapel Hill, aby během několika dnů pořádně rozebrali
  Einsteinovu teorii. Konferenci podpořil Agnew Bahnson a letectvo Spojených států. Letectvo dokonce
  některé účastníky na místo konání konference bezplatně dopravilo. 

  Nepřijeli sem jen relativisté. Konference se rozhodl zúčastnit i bývalý student J. A. Wheelera
  Richard Feynman, který kompletně přebudoval kvantovou teorii a navrhl nový způsob, jak kvantový
  popis zavést. Ač patřil do kvantového světa, zajímalo ho, co je v obecné teorii relativity nového.
  Později vyprávěl, jak přijel do Chapel Hill. Nevěděl, kde přesně se konference koná, vzal si
  taxík, ale řidič to také nevěděl - proč by také měl? Feynman mu řekl: „Podívejte, hlavní část
  konference začala včera, takže včera se zde musela motat spousta lidí, co tam jeli. Já vám je
  popíši. Vypadali trochu nepřítomně, při chůzi hovořili mezi sebou a nedávali pozor, kam jdou, a
  pořád si žbrblali něco jako gé - mí - ný.“ Gé - mí - ný (psáno gµν) je matematický symbol pro
  metriku, ve které je zakódována geometrie prostoročasu. A řidič věděl, kam má jet.

  Všem účastníkům konference bylo od počátku jasné, že má-li se obecná teorie relativity vytáhnout
  ze stínu, kde strádala poslední tři desetiletí, musí se něco udělat. Richardu Feynmanovi bylo
  zřejmé, proč byla obecná teorie relativity zanedbávána: „Je tu vážná potíž, a tou je nedostatek
  experimentů. Navíc nemůžeme doufat, že nějaké experimenty přibydou, takže musíme hledat způsob,
  jak si poradit se situací, kdy žádné experimenty nejsou k dispozici.“ Bez experimentů se obor
  nemůže rozvíjet, ale Feynman doporučoval, jak pokračovat. Říkal, že obecná teorie relativity je
  obtížná, ale ne zase tak hrozně obtížná, a tak „je nejlepší předstírat, že nějaké experimenty
  máme, a počítat. V této oblasti nás netáhnou kupředu experimenty, nýbrž imaginace.“ 

  Feynmanův názor souzněl s obecným pocitem na chapelhillské konferenci, kde byla řada mladých lidí
  z nové generace relativistů nebo čerstvě graduovaných studentů. Byli plni nových myšlenek a
  připraveni k boji. Jak se konference rozvíjela, výstřední nápady si konkurovaly se střízlivými
  myšlenkami zkušených harcovníků. Přednášky střídaly debaty plné bystrých argumentů. Když Thomas
  Gold předložil novou verzi teorie stacionárního vesmíru, DeWitt napadl její základní předpoklad -
  Hoyleovo kreační pole - tím, že zpochybnil mechanismus, kterým mělo být narušeno zachování
  energie. Když někdo propagoval potřebu vytvořit sjednocenou teorii gravitace a elektromagnetismu v
  duchu teorií, o jaké se po léta snažil Albert Einstein, Feynman byl nemilosrdný. Proč by
  elektromagnetismus měl být jedinou silou sjednocenou s gravitací? Co ostatní síly a vše ostatní,
  co v přírodě nacházíme? Mnoho se diskutovalo o myšlence, kterou byli posedlí zejména Wheeler a
  DeWitt, totiž o propojení gravitace a kvantové teorie - tento problém se objevoval v mnoha formách
  a přestrojeních. A může být prostoročas zbrázděn gravitačními vlnami, tak jako hladina jezera
  vlnami od parníku? Vlnami podobnými elektromagnetickým vlnám, které předpovídá Maxwellova teorie?
  O těchto i dalších tématech se vedly mezi účastníky živé debaty v sekcích věnovaných diskusím. 

  John Wheeler přišel s velikým plánem revoluce ve fyzice na základě obecné teorie relativity a s
  kohortou svých studentů a postdoktorandů (zkráceně postdoků) propagoval své myšlenky. S úvahami o
  relativistické teorii se posunuli ještě dále, tak daleko, že jejich koncepce zněly jako žert. Na
  jejich menu byl nejen „elektromagnetismus bez elektromagnetismu“, ale i „náboj bez náboje“, „spin
  bez spinu“ a „elementární částice bez elementárních částic“. Po celou konferenci hrál Wheelerův
  klan hlavní roli, když házel mezi účastníky nové a nové myšlenky, aby je promyšleně zvážili,
  případně odvrhli. John Wheeler byl ve svém živlu. 

  Na elementárnější úrovni diskutovali relativisté shromáždění na Chapel Hill o tom, zda lze na
  základě obecné teorie relativity dělat realistické předpovědi. Má-li mít teorie prestiž, musí být
  prediktivní. Například maxwellovská elektrodynamika byla kromobyčejně úspěšná proto, že
  předpověděla skoro vše, co se týká světla, elektřiny a magnetismu. Avšak i když Schwarzschild,
  Friedmann, Lemaître a Oppenheimer na základě obecné teorie relativity leccos předpověděli, omezili
  se na velice zjednodušené idealizované systémy a vůbec nebylo jasné, jak se dostat za tato
  zjednodušení. A tak si účastníci konference na Chapel Hill kladli otázku: je možné obecně řešit
  Einsteinovy rovnice pole a dělat reálné předpovědi, jak se bude prostoročas vyvíjet? Zdálo se, že
  příšerně zamotaná struktura obecné teorie relativity znemožňuje i správnou volbu počátečních
  podmínek, natož výpočet vývoje z nich. Neobyčejně složitým úkolem se ukázalo i řešení
  Einsteinových rovnic na počítačích. 

  Setkání bylo vzrušujícím fórem pro mladé příznivce obecné teorie relativity překypující
  kreativitou, které povzbuzovala vynalézavost Johna Wheelera a Feynmanova imaginace. Ale teorie
  prostoročasu byla stále na mrtvém bodě. Sebedůvtipnější matematika, všechny návrhy na sjednocení,
  debaty o gravitačních vlnách a Wheelerových červích dírách, geonech a prostoročasové pěně by byly
  k ničemu, kdyby se je nepodařilo propojit s reálným světem. 

  Od prvního velkého testu obecné relativity, Eddingtonova měření při zatmění, uplynulo tehdy už
  skoro čtyřicet let a od Hubbleova objevu rozpínání vesmíru tři desítky let. V době konference na
  Chapel Hill nebyla ani žádná čerstvá potvrzení Einsteinovy teorie, ani nic, co by jí mohlo
  protiřečit. Jeden z Wheelerových princetonských kolegů Robert Dicke shrnul situaci ve svém
  příspěvku „Experimentální základ Einsteinovy teorie“ takto: „Relativita se zdá být čistě
  matematickým formalismem s malým dopadem na to, co měříme v laboratoři.“ Odpověď na otázku
  fyzikální relevance obecné teorie relativity však nečekala v laboratoři, nýbrž ve hvězdách.

  V roce 1963 pracoval holandský astronom Maarten Schmidt s dalekohledem pojmenovaném po Georgi
  Ellerym Haleovi, patronu observatoře na Mount Palomar. Zajímal ho jeden zdroj v katalogu 3C od
  radioastronomů Martina Rylea a Bernarda Lovella. Zatímco Wheeler dodával obecné teorii relativity
  novou energii, radioastronomové podrobněji studovali rádiové zdroje ve svých přehledech. Tak jako
  všichni pozorovatelé hvězd, chtěli se i oni dozvědět, co jejich rádiové zdroje skutečně jsou. K
  tomu jich potřebovali najít co nejvíce a potřebovali je prohlížet skutečně podrobně, aby zjistili,
  co rádiové vlny skutečně emituje. 

  Během jednoho desetiletí Ryle a Lovell zvýšili přesnost svých měření více než o jeden řád -
  pomáhala jim jejich nápaditost, kterou uplatňovali už při vývoji radaru. Tato přesnost měření už
  jim dovolovala ukázat na obloze skutečné místo, kde se zdroj nacházel. Ryleův katalog 3C obsahoval
  už několik set zdrojů s přesně určenou polohou. 

  Lovellova skupina se zaměřila na Cygnus A, jeden z rádiových zdrojů, které identifikoval Grote
  Reber v souhvězdí Labutě na pozadí šumu rádiových vln z naší galaxie - v Ryleově katalogu nesl
  označení 3C405. Byl to velmi podivný objekt, sestávající ze dvou laloků rádiových vln, z nichž
  každý měl téměř pravoúhlý tvar. Byly to gigantické struktury o průměru asi sto světelných let a
  zdálo se, že je vyživuje něco mezi nimi. Když pak astronomové zaměřili teleskopy na jiný zdroj
  označený jako 3C48, nenašli složitou strukturu, jaká se rozprostírala kolem zdroje Cygnus A, nýbrž
  jasný bodový útvar, v jehož světle dominovalo záření z modrého konce spektra. Vypadal jako hvězda,
  jednoduchá a bez význačných rysů. Když ale proměřili jeho spektrum, aby zjistili, z čeho je 3C48
  složen, nebyli schopni přiřadit les spektrálních čar, který nalezli, ke spektru z žádné známé
  hvězdy a nemohli identifikovat žádné prvky, jež by tomuto spektru odpovídaly. A takto
  neidentifikovatelných zdrojů se nalezlo více. Kosmických rádiových zdrojů byla spousta a u žádného
  se neumělo určit ani jeho složení, ani jeho vzdálenost. 

  Maarten Schmidt se soustředil na zdroj, který nesl označení 3C273. Vypadal jako hvězda, jenže
  žádná hvězda podobné spektrum neměla. Když ale svá měření prostudoval podrobněji, všiml si něčeho
  pozoruhodného: spektrální čáry přesně odpovídaly spektru vodíku, jenomže dramaticky posunutému
  červeným posunem o téměř 16 \%. Pak se všechny čáry daly přesně identifikovat. Jenže tak obrovský
  červený posun dovoloval jen dvojí závěr. Buď mohl být 3C273 celkem blízko a z nějakých důvodů se
  od nás vzdaloval rychlostí blízkou rychlosti světla, nebo tato rychlost vzdalování byla způsobena
  rozpínáním vesmíru, pak ale zdroj musel ležet v obrovské vzdálenosti. Schmidt byl ohromen. Ten
  večer, kdy si toho všiml, řekl své ženě: „Dnes se mi v práci stalo něco příšerného.“

  To byl opravdu důležitý objev. Schmidt zjistil, že tyto po vesmíru roztroušené objekty jsou
  vzdálené miliardy světelných let. To, že jsou pozorovatelné ať už optickými nebo rádiovými
  teleskopy znamenalo, že musí vyzařovat nepředstavitelné množství energie. Objekty 3C273 a 3C48
  vyzařovaly tolik světla jako stovka galaxií dohromady. Byly jako supergalaxie, s mnohem větším
  výkonem, než byl kdy ve vesmíru pozorován. 

  Tyto zdroje byly zároveň relativně velmi malé, měly jen zlomek rozměrů běžné galaxie. A totéž se
  zjistilo i o dalších zdrojích z katalogu 3C - některé byly desetkrát, některé i stokrát menší než
  běžná galaxie. Při bližším ohledání neměřily napříč více než několik bilionů kilometrů, „podle
  kosmologických standardů byly jako burské oříšky“, jak napsal v té době časopis Time. V
  kolosálních vzdálenostech v malé oblasti prostoru bylo produkováno obrovské množství energie.

  Něco tak těžko vysvětlitelného představovalo pokušení pro Freda Hoylea. Ten stále pokračoval v
  boji za stacionární vesmír, získal však zatím velikou reputaci jako expert na strukturu hvězd.
  Spolu s Williamem (Willym) Fowlerem a Geofreyem a Margaretou Burbidgeovými poskytli detailní
  vysvětlení, jak se mohou ve hvězdách syntetizovat jednotlivé chemické prvky. Ukázali též, že
  všechny těžší prvky, jež v přírodě nacházíme, mohly vzniknout termonukleárními reakcemi v nitrech
  hvězd. 

  Fowler a Hoyle tedy přišli s myšlenkou, že rádiové hvězdy jsou skutečně hvězdy, avšak odlišné od
  běžných hvězd. Dostaly jméno superhvězdy. Měly mít hmotnost rovnou milionům až stovkám milionů
  hmotnosti našeho Slunce, takže během svého života mohly produkovat obrovské množství energie. A
  jejich život byl krátký, své jaderné palivo spotřebovaly rychle. Potom následoval kolaps do rychle
  probíhající divoké smrti. Svými superhvězdami však Hoyle a Fowler posunuli naše porozumění vývoji
  hvězd, jehož základní pravidla nalezl Eddington, do království, kde se měla výrazně projevovat
  obecně relativistická teorie gravitace. Einsteinova teorie lákala. 

  V úmorném vedru léta 1963 se sešla v texaském Dallasu malá skupina relativistů. Seděli kolem
  bazénu, popíjeli Martini a diskutovali o těžkých objektech, které odhalil Maarten Schmidt. Byla to
  internacionální skupina. Jak to jeden z nich zhodnotil: „Američtí vědci kromě geologů a geofyziků
  by se zde těžko usadili. Krajina kolem je asi tak přitažlivá jako Paraguay.“ Ale Texas se měl stát
  významným centrem relativistické fyziky, což byla především zásluha úsilí sympatického Vídeňáka
  Alfreda Schilda. 

  Schild prožil toulavé dětství a mládí, což byl důsledek zmatků v třicátých a čtyřicátých letech
  dvacátého století. Narodil se v Turecku a jako dítě žil v Anglii. Podobně jako Bondi a Gold byl za
  války internován v Kanadě. Tam studoval pod vedením jednoho z Einsteinových žáků Leopolda Infelda
  a napsal disertaci o kosmologii. Byl na setkání na Chapel Hill v roce 1957, prožil tam nadšení pro
  novou fázi obecné teorie relativity a v témž roce také získal profesorské místo na Texaské
  univerzitě v Austinu. 

  Když Alfred Schild přišel do Austinu, Texas představoval tak trochu stojaté vody, ale byl velice
  bohatý díky příjmům z ropného průmyslu, které tekly do místního hospodářství. Schild byl schopen
  přesvědčit univerzitu, aby naftové peníze vhodně užila a dovolila mu založit jeho vlastní centrum
  pro relativitu. Za zády s americkým letectvem, které stále věřilo v možnou magickou sílu
  gravitace, nebyla o peníze nouze. A zatímco matematici v Austinu se na Schildovu práci dívali
  spatra, fyzici byli spokojeni. 

  Schild začal hledat do svého centra talenty a měl na ně evidentně štěstí. Skupina mladých
  relativistů pocházejících z Německa, Anglie a Nového Zélandu přeměnila texaský Austin v povinnou
  zastávku každého relativisty hodného toho jména. Schildovo působení se však netýkalo jen Austinu.
  V nově vzniklém Jihozápadním centru pro pokročilá studia hledali vhodný program výzkumu a právě
  zde vstoupil do hry Schild. Navrhl, aby se zaměřili na relativitu a vedení centra souhlasilo. V
  Dallasu vzniklo další relativistické centrum, ve kterém působila řada zahraničních vědců. 

  V tom červencovém odpoledni texaští relativisté obléhající bazén sestavili schéma, které přivedlo
  do Texasu vědce z celého světa k diskusi o relativitě. Neměl to být nový Chapel Hill, malý a
  neformální. Tentokrát měl dorazit i zástup astronomů, které chtěli relativisté přitáhnout k
  přemýšlení o Einsteinově teorii s důrazem na povahu rádiových hvězd, „kvazistelárních rádiových
  zdrojů“. Na základě Schmidtových měření z minulého března bylo zřejmé, že tyto podivné objekty
  měly příliš velkou hmotnost a byly příliš daleko, než aby je stačilo zkoumat na základě starého
  Newtonova gravitačního zákona. Byly zřejmě obrovské a Chandra s Oppenheimerem ukázali, že pro
  takové útvary, jež nemusí být schopné vzdorovat své vlastní gravitaci, může hrát zásadní roli
  obecná teorie relativity. Organizátoři ve zvacím dopise uváděli, že „energie, jež vedla k
  vytvoření rádiových zdrojů, může být dodávána gravitačním kolapsem superhvězd“. Konferenci nazvali
  „Texaské symposium o relativistické astrofyzice“ a měla se konat v prosinci 1963 v Dallasu. 

  Málem ovšem došlo k jejímu zrušení. V Dallasu byl zavražděn prezident John Fitzgerald Kennedy a
  mnozí z předpokládaných účastníků měli obavy o svou bezpečnost. Dallaští relativisté tedy požádali
  starostu, aby je ujistil, že žádné nebezpečí nehrozí. To zabralo a do Dallasu se sjelo více než
  tři sta vědců, aby vyslechli nejčerstvější novinky o rádiových zdrojích a diskutovali o tom, co z
  těchto nových poznatků plyne. Byl mezi nimi i Robert Oppenheimer, který předtím v Princetonu
  odrazoval studenty od práce v relativitě. Nové rádiové hvězdy ho však zaujaly. Jak se vyjádřil,
  „byly to neuvěřitelně krásné objekty … spektakulární, neobyčejně grandiózní objekty bez precedentu
  “. Později vzpomínal, jak mu to připomínalo konference o kvantové fyzice o dvě desítky let dříve,
  „kdy panoval velký zmatek, ale bylo k dispozici velké množství dat“. Byla to pro něj vzrušující
  doba. 

  Konference probíhala tři dny, kdy jak astronomové, tak relativisté diskutovali o podivných
  „kvazistelárních rádiových zdrojích“ v Ryleově katalogu 3C. Jeden z účastníků pro ně doporučil
  název „kvasary“, což bylo kratší a snadněji se vyslovovalo, a tento název se ujal. Relativistům
  bylo jasné, že kvasary jsou natolik hmotné a koncentrované, že se Schwarzschildovo podivné řešení
  musí brát v úvahu v celém svém rozsahu, mají-li pozorovací data dávat nějaký smysl. To zvyšovalo
  význam Oppenheimerových a Snyderových výpočtů. A astrofyzikům a astronomům připadala pozorovací
  data tak tajemná, že začali sledovat, co k tomu říkají relativisté. Možná - ale bylo to pořád jen
  veliké možná - je opravdu nutné vtáhnout do hry obecnou teorii relativity, mají-li se rozumně
  vysvětlit výsledky pozorování. 

  V Dallasu byl přítomen i John Wheeler a byl připraven vyjádřit svůj názor. O obecnou relativitu se
  začal zajímat zhruba před deseti lety. Pro něj zůstávalo velkou nezodpovězenou otázkou to, co
  nazýval „problém finálního stavu“. Chtěl zjistit, k čemu dochází na konci gravitačního kolapsu.
  Pořád nemohl uvěřit Oppenheimerově a Snyderově předpovědi, že se vytvoří singularity, a byl
  přesvědčen, že obecná relativita bude mít klíčovou úlohu při vysvětlení, proč k tomu nedojde. Přes
  svůj předsudek se cítil povinen probrat všechny možnosti a získat účastníky, aby se připojili k
  jeho hledání. Před svým vystoupením vzal Wheeler křídu a pečlivě tabuli vyplnil propracovanými
  obrázky a rovnicemi, jež demonstrovaly, o čem přemýšlel téměř celé desetiletí. Nakreslil grafy,
  které ukazovaly, jak hvězda kolabuje vlivem své vlastní gravitace a jak obecná relativita
  předpovídá nezadržitelnou cestu hvězdy k jejímu finálnímu stavu. Kolem byly rozházeny rovnice,
  kousky Einsteinových rovnic pole, shrnutí výsledků kvantové fyziky, směs vztahů, na kterých
  zakládal své brilantní úvahy posledních let. Wheelerova přednáška byla apologií obecné teorie
  relativity a zdůvodněním, proč její důsledky musí brát vážně každý poctivý astrofyzik.

  Pro řadu astronomů byly důsledky příliš fantastické. Jeden z účastníků si vzpomínal na výraz
  „naprosté nevíry“ na tváři jednoho z „význačných účastníků“. Jiní ale byli potěšeni, že se Whee
  ler konečně zapletl s reálným vesmírem. Zdálo se, že obecná teorie relativity, o které tak dlouho
  přemýšlel, konečně získává důležitost pro vysvětlení údajů získaných novými pozorováními
  rádiových zdrojů. 

  Časopis Life o konferenci napsal: „Vědci, kteří popustili otěže své představivosti natolik, že
  dříve by to bylo příliš i pro autory sci-fi, odcházeli z konference obtíženi záhadami stejně, jako
  byli na začátku jednání… Povaha rádiových zdrojů se zdá být tak podivuhodná, že nelze vyloučit ani
  ta nejfantastičtější vysvětlení.“ Na závěrečné slavnostní večeři Thomas Gold shrnul ve své řeči
  podivuhodný vývoj událostí, ke kterému během konference došlo: „Ukázalo se, že relativisté se svou
  sofistikovanou prací nejsou jen velkolepým kulturním ornamentem, nýbrž že mohou být užiteční i
  ostatní vědě! Všichni můžeme být spokojeni. Relativisté proto, že jsou najednou experty v oblasti,
  o které donedávna nevěděli, že existuje. Astrofyzikové proto, že najednou zvětšili svou říši o
  území obecné teorie relativity.“ Skončil opatrnou poznámkou: „Doufejme, že je to tak v pořádku.
  Jaká by byla hanba, kdybychom relativisty museli zase vyloučit.“ 

  John Wheeler s neuvěřitelnou jasnozřivostí dozíral na vzkříšení skomírající Einsteinovy teorie.
  Tím, že věnoval svůj obrovský intelekt a tvořivost výchově nové generace brilantních mladých
  relativistů a podporoval nová centra rozesetá po celé zemi, živil novou komunitu plnou života,
  připravenou hluboce přemýšlet o problémech gravitace. A existovala empirická data, jež čekala na
  vysvětlení. Díky astronomům, fyzikům a matematikům připraveným zabývat se velkými otázkami stálo
  Texaské sympozium na prahu nové éry. Obecná teorie relativity byla opět ve hře. 

\section{Singularity}\label{feyIchIIIsecIX}
  Zatímco většina posluchačstva sledovala vystoupení Johna Wheelera na Texaském symposiu v roce 1963
  bez hlubšího porozumění, jeden mladý matematik poslouchal jeho přednášku před pečlivě připravenými
  rovnicemi a grafy s velikým napětím. „Wheelerovo vystoupení na mě udělalo skutečně neobyčejný
  dojem,“ vzpomíná Roger Penrose. A přestože Wheeler tvrdošíjně odmítal existenci singularit, kladl
  podle Penrose správnou otázku: Mohou být singularity podstatnou složkou obecné teorie relativity?
  Wheelerova řeč na Texaském symposiu byla počátkem dekády, které se pak přezdívalo „zlatý věk
  obecné relativity“ (jak to zavedl jeden z Wheelerových studentů Kip Thorne), a Roger Penrose se
  stal jednou z největších postav tohoto období. 

  Penrose tráví svůj život hrou s prostoročasem: tu jej kus odřízne a pak zase nějak slepí
  dohromady, tu zobrazuje jeho nejzazší okraje. On totiž vidí věci jinak, má matematický vhled
  kombinovaný s vnitřním porozuměním prostoru a času. Jeho obrázky známé jako Penroseovy diagramy
  určitým způsobem rozbalují prostoročas a odhalují jeho nejzvláštnější vlastnosti. Ukazují, co se
  děje se světlem, když projde Schwarzschildovým povrchem, jak se světlo chová, když je sledujeme
  zpět k velkému třesku, a dokonce i to, jak se může prostor a čas zdeformovat, takže vypadá jako
  zpěněná mořská hladina. 

  Penrose začala obecná teorie relativity přitahovat již jako studenta matematiky v Londýně. Její
  základy se naučil z knihy Erwina Schrödingera příznačně nazvané Struktura prostoročasu. O
  detailech ale začal uvažovat až na základě přednášek Freda Hoylea, jež propagovaly jeho teorii
  stacionárního vesmíru. Na vesmíru, který Hoyle popisoval, bylo něco fascinujícího a zároveň
  podivného - neshodoval se s Penroseovým chápáním relativity. Rozhodl se navštívit svého bratra
  Olivera, také matematika, který studoval na PhD v Cambridgi. Doufal, že Oliver mu pomůže pochopit
  podivnou teorii, která ho tak lákala. 

  Cambridge padesátých let minulého století byla přes usedlou atmosféru staletí starých budov a
  tradiční rituály kolejí i univerzity vzrušujícím místem. Paul Dirac, anglický fyzik, který mimo
  jiné ukázal, že kvantové teorie Heisenberga a Schrödingera jsou ekvivalentní, zde fascinujícím
  způsobem přednášel kvantovou mechaniku. Hermann Bondi zde měl přednášky z obecné teorie relativity
  a kosmologie a spolu s Fredem Hoylem aktivně propagovali svou teorii stacionárního vesmíru. A pak
  tu byl Dennis Sciama. 

  Penrose a jeho bratr Oliver se jednoho večera sešli v restauraci Kingswood v Cambridgi a
  diskutovali o rozhlasových přednáškách Freda Hoylea. Roger prostě nemohl pochopit Hoyleovo
  tvrzení, že ve stacionárním vesmíru se budou galaxie od sebe vzdalovat tak rychle, že časem zmizí
  za kosmickým horizontem. Podle něho to mělo probíhat jinak a byl přesvědčen, že to jasně vyplývá z
  jeho diagramů. Oliver ukázal k vedlejšímu stolu a řekl: „Zeptej se Dennise, ten ví o stacionárním
  vesmíru všechno.“Odvedl Rogera k Sciamovu stolu a navzájem je představil. Mezi oběma okamžitě
  přeskočila jiskra. 

  Sciama byl jen o čtyři roky starší než Roger, ale v Einsteinově teorii se už výborně vyznal a byl
  pro ni plný nadšení, jež přenášel dalších skoro padesát let na nespočet studentů a
  spolupracovníků. Rok před Einsteinovou smrtí byl na stáži v princetonském Institutu pro pokročilá
  studia a v jednom rozhovoru s Einsteinem prohlásil hrdě, ale trochu unáhleně, že „je tam proto,
  aby obhájil,starého Einsteina‘ před tím novým“. Einstein se jeho drzosti velmi smál. Sciama
  pracoval nějakou dobu pod vedením Paula Diraca, tedy pokud to vzhledem k Diracovým zvláštnostem
  šlo. Upoutal ho model vesmíru Hoylea, Bondiho a Golda, ale přestože ve stacionární vesmír věřil,
  pečlivě sledoval výsledky Rylea a jeho skupiny, které získávala takřka „za humny“Cambridge.
  Uvědomil si, že by pro Hoyleův model mohly mít smrtící důsledky. 

  Tehdy v restauraci Kingswood Penrose Sciamovi vysvětlil, proč galaxie z dohledu nezmizí. Budou se
  jen jevit stále méně zřetelné a z dálky se bude zdát, že zamrzají v čase. Obdobně se podle
  Oppenheimerových a Snyderových výpočtů jeví vnějšímu pozorovateli i povrch hvězdy, který prochází
  Schwarzschildovým horizontem. Sciama si všímal jisker v Rogerových očích a moc se mu líbil nový
  pohled na prostoročas, který Penrose přinášel. Přátelili se pak až do Sciamovy smrti (Dennis
  Sciama zemřel v roce 1999). 

  Roger Penrose přešel na doktorské studium matematiky do Cambridge, dále se však věnoval
  matematickým podivnostem, které nacházel v geometrii prostoročasu. Ze všech sil se jim snažil
  porozumět. Když dokončil doktorské studium, podlehl svodům obecné relativity a rozhodl se v ní
  pracovat. Několik příštích let strávil cestováním po světě, pracoval s Wheelerem v Princetonu, s
  Hermannem Bondim v Londýně a Peterem Bergmannem v Syracuse. Nakonec se na podzim 1963 připojil k
  Schildově texaské skupině v Austinu. 

  Texas byl pro relativitu skvělé místo a tamější badatelé si nemuseli dělat starosti s penězi,
  protože výzkumu se dostávalo skvělé podpory. Penrose to komentoval: „Nikdy jsme se neptali, odkud
  všechny ty peníze pocházejí nebo proč někdo pokládá za užitečné utratit tolik peněz na relativitu.
  Myslel jsem si, že to musí být nějaký omyl.“

  Jedním z mladých Penroseových kolegů byl Roy Kerr z Nového Zélandu. Kerr trávil dlouhou dobu v
  texaském vlhku a vedru tím, že se snažil najít složitější, ale realističtější řešení Einsteinových
  rovnic. Nakonec předložil elegantní soustavu rovnic popisující prostoročas s poměrně jednoduchou
  geometrií, která byla zobecněním geometrie Schwarzschildovy. Zatímco Schwarzschildovo řešení
  popisovalo prostoročas symetrický kolem bodového centra, ve kterém sídlila neblahá singularita,
  Kerrovo řešení bylo symetrické vzhledem k přímce procházející celým prostoročasem. Dalo se
  interpretovat tak, že odpovídalo prostoročasu rotujícímu kolem zmíněné přímky. Toto řešení
  záviselo na parametru, který charakterizoval míru rotace. Byl-li tento parametr rovný nule, řešení
  přešlo v řešení Schwarzschildovo. 

  Penrose začal okamžitě Kerrovy výsledky studovat. S novými kolegy v Austinu trávil hodiny
  diskusemi o tomto řešení, které přeformuloval do svého matematického jazyka. Penroseův přístup
  zaujal Schilda stejně tak jako Sciamu. Jeho matematický vhled a jeho diagramy vrhly na Kerrovo
  řešení zcela nové světlo. Kerr zaslal své pozoruhodné řešení do časopisu Physical Review Letters.
  Redakce tohoto časopisu přijala o několik let dříve rozhodnutí nepublikovat nic, co se týká obecné
  teorie relativity, teď však byl článek přijat a vyšel v září 1963. To bylo jen pár měsíců před
  Texaským symposiem v Dallasu, kde mohl Kerr předložit své výsledky astrofyzikům.

  Schild měl obavu, že by Kerrovo vystoupení mohlo být příliš suché a matematické, a proto požádal
  Penrose, aby o novém řešení promluvil on. To Penrose odmítl - řešení bylo přece Kerrovo dítě.
  Ukázalo se však, že Schildovy obavy nebyly zcela neopodstatněné. Když Kerr přišel na podium,
  polovina účastníků opustila sál. Kerr byl mladý a neznámý a navíc relativista ve smečce
  astrofyziků, kteří měli v tu chvíli dle svého názoru důležitější věci na práci. Kerr tedy mluvil k
  ne moc zainteresovanému zbytku, který, jak Penrose vzpomíná, mu nevěnoval moc pozornosti. Jen
  velmi málo posluchačů si uvědomilo, že Kerrovo řešení je velikým krokem k takovému zobecnění
  Schwarzschildova prostoročasu, jež bude obecnější, realističtější a pro astrofyziku užitečnější.
  Kerr napsal krátké shrnutí do sborníku z konference, ale ten, kdo shrnoval hlavní výsledky, se o
  jeho řešení vůbec nezmínil. Pro astrofyziky to pořád byla příliš abstraktní obecná relativita, než
  aby byli ochotni se jí zabývat. 

  Na prvním Texaském symposiu nebyl ani jeden sovětský fyzik. Velká část skvělé sovětské
  intelektuální síly byla pohlcena sovětskými nukleárními projekty a nemohla se obecnou teorií
  relativity zabývat. Ale podobně jako se nová generace relativistů ve Spojených státech zrodila z
  projektu Manhattan a ve Spojeném království z radaru, i v Sovětském svazu se na oživení obecné
  relativity v šedesátých letech minulého století podílela řada nukleárních fyziků. 

  Sovětský nukleární výzkum začal pozdě, země se během druhé světové války musela věnovat
  naléhavějším úkolům. Od roku 1939 po článku Johna Wheelera a Nielse Bohra o mohutné produkci
  energie při nukleárním štěpení těžkých prvků najednou v západních časopisech publikace o jaderném
  štěpení zmizely. Pro Sověty to znamenalo, že toto přerušení má nějaký důvod. V roce 1942 sovětský
  fyzik Georgij Flerov napsal o této podivné skutečnosti Stalinovi a ten začal být podezřívavý.
  Pochopil, že Američané zřejmě pracují na bombě a že Sověti nesmějí zůstat stranou. Jakmile válka
  skončila, Stalin shromáždil svou vědeckou elitu, aby začala pracovat na vývoji atomové bomby. Mezi
  těmito vědci byli i Lev Landau a Jakov Zeldovič. 

  Lev Landau byl jednou z obětí perzekuce během velkého teroru na konci třicátých let. Jeho pobyt ve
  vězení z něj udělal hluboce zatrpklého muže zklamaného režimem, jemuž byl ovšem vydán na milost a
  nemilost. Stal se legendou, s jeho jménem byla spojena řada objevů od kvantové teorie po
  astrofyziku. Vytvořil školu teoretické fyziky a jeho žáci napínali své intelektuální schopnosti na
  samý kraj, jen aby s ním mohli pracovat. Aby se někdo mohl stát Landauovým aspirantem, musel
  složit jedenáct obtížných zkoušek - říkalo se tomu „Landauovo teoretické minimum“. Zkoušejícím byl
  sám Landau a projít touto procedurou trvalo dva roky. Touto „Landauovou bariérou“ prošlo málo
  uchazečů, jen jim pak bylo dovoleno pracovat se samotným velikánem. 

  Běloruský fyzik židovského původu Jakov Zeldovič, jen o pár let mladší než Landau, byl předčasně
  vyzrálý student. V sedmnácti letech se stal laboratorním asistentem, ve čtyřiadvaceti získal
  doktorát a rychle se stal jednou z největších sovětských autorit na problémy spalování a zážehu.
  Bylo samozřejmé, že byl přizván k účasti na vývoji bomby a této práci se věnoval s nadšením. Mezi
  léty 1945 a 1963 se podílel na vývoji první sovětské atomové bomby, kterou Američané nazývali
  Joe-1, když zaregistrovali její zkušební výbuch v srpnu 1949. Pracoval i na jejím potomkovi
  „superbombě“. Sovětský svaz dostihl Ameriku a stal se další jadernou velmocí. 

  Zatímco Zeldovič se věnoval vývoji bomby s nadšením, Landau, stále roztrpčený pobytem v Lubljance
  a chovající ke Stalinovi hlubokou nenávist, byl ke spolupráci donucen. Zeldovič Landaua obdivoval,
  Landaův postoj k němu byl však méně vřelý, právě tak jako k celému projektu bomby. Když se
  Zeldovič snažil sovětský projekt jaderné bomby rozšířit, Landau ho nazval „čubkou“. Když Stalin
  zemřel, Landau prohlásil: „Tak je to za námi. Je pryč. Už se ho nemusím bát a už nikdy nebudu
  pracovat na nukleárních zbraních.“ Nicméně oba muži byli za své zásluhy o projekt sovětské atomové
  bomby několikrát vyznamenáni Stalinovou cenou a titulem „Hrdina socialistické práce“. Landau
  obdržel v roce 1962 i Nobelovu cenu. 

  V šedesátých letech minulého století Zeldovičova hvězda stále stoupala. Landau se však stal obětí
  automobilové nehody, kvůli které ztratil své dušení schopnosti a pro fyziku byl ztracený. Jeho
  místo zaujali nejbližší spolupracovníci. Dva mladí lidé, Isaak Chalatnikov a Jevgenij Lifšic,
  kteří oba překonali Landauovu bariéru, byli skvěle připraveni na to, aby si poradili s obtížemi
  Einsteinovy teorie a zakousli se do problému, co se stane, když se hmota hroutí pod účinkem
  vlastní gravitace.

  Oppenheimerovo a Snyderovo řešení bylo jednoduchou aproximací, představovalo dokonale symetricky
  rozloženou hroutící se látku. Dokonalá symetrie se zpočátku nelíbila lidem jako byl Wheeler, který
  v ní viděl přílišnou idealizaci. Povrch Země je pokryt nepravidelnostmi - vysokými horami a
  hlubokými oceány. Co když je povrch kolabující hvězdy podobně nepravidelný? Mohly by takové
  nepravidelnosti způsobit, že některé části povrchu se hroutí rychleji než jiné, takže by ke středu
  nedorazily současně, mohly by se odrazit a znovu rozepnout? Pak by se singularity nemusely nikdy
  vytvořit. 

  Ruští vědci se postavili k otázce tak, že Oppenheimerovu a Snyderovu symetrii uvolnili. Ve
  výpočtech Lifšice a Chalatnikova se může prostoročas kroutit a deformovat v každém směru jinak.
  Představme si, že sledujeme povrch zárodečného uskupení hmoty velmi hmotné hvězdy, která se začala
  hroutit směrem ke svému centru. Předpokládali bychom, že její povrch bude nepravidelný. Vrcholy a
  dolíky vrásek na povrchu hvězdy budou obecně přitahovány silou jiné velikosti. Místo aby veškerá
  hmota padala do středu, bude část hmoty urychlována jiným směrem. To bude obecně bránit vzniku
  singularity, která se vytvoří jen tehdy, je-li hmota na počátku rozmístěna přesně symetricky.
  Chalatnikov a Lifšic došli k závěru, že singularita se ve skutečnosti nikdy nevytvoří - tento
  udivující závěr zveřejnili v časopise Žurnal experimentalnoj i teoretičeskoj fiziky. Podle nich
  bylo Schwarzschildovo právě tak jako Kerrovo řešení abstrakcí, která se nikdy v přírodě
  nerealizuje. Zdálo se tedy, že Einstein a Eddington měli přece jenom pravdu. 

  Sovětští fyzici občas, i když vzácně, získali povolení zúčastnit se některé konference, která se
  konala na Západě. Třetí konference o obecné relativitě a kosmologii konaná v roce 1965 v Londýně,
  která navazovala na setkání na Chapel Hill, měla více než dvě stě účastníků. Když zde předložil
  své výsledky Chalatnikov, dostalo se jim široké pozornosti. Bylo patrno, že obecná teorie
  relativity prožívá renesanci i v Sovětském svazu, na druhé straně železné opony se však o tom
  mnoho nevědělo. Hlavní sovětské fyzikální časopisy se sice překládaly do angličtiny, ale překlady
  vycházely vždy opožděně. 

  Penrose tiše seděl a naslouchal Chalatnikovově přednášce. Věděl, že prezentované výsledky jsou
  nesprávné, zdálo se mu však „nediplomatické“ to vyslovit. Jeho názor ale byl takový, že způsobem,
  jakým Chalatnikov a Lifšic postupovali, se nic podstatného dokázat nedá: „Je tam prostě příliš
  mnoho předpokladů. Takhle se singularity odstranit nedají.“ Penrose naopak uměl dokázat, že
  singularity se při kolapsu hvězdy vytvoří vždycky, což byl pravý opak Chalatnikovova tvrzení.
  Přitom Penroseovy výsledky byly zcela obecné díky tomu, že užíval své vlastní metody pro zkoumání
  prostoročasu. 

  Od svého prvního setkání se Sciamou v cambridgeské restauraci Kingswood před deseti lety
  rozpracoval Penrose své diagramy do obecných pravidel, jak sledovat šíření světla a pohyb částic
  prostoročasem. Byl teď schopen vzít jakýkoli prostoročas a rozhodnout podle jeho základních
  vlastností a charakteru jeho materiální náplně o jeho osudu, například zda zkolabuje do bodu, nebo
  se naopak rozepne do nekonečna. Když svá pravidla aplikoval na problém gravitačního kolapsu, tedy
  na „problém finálního stavu“, jak jej nazýval Wheeler, zjistil nevyhnutelnost vzniku singularit.
  Své výsledky shrnul do článku „Gravitační kolaps a prostoročasové singularity“, který zaslal do
  časopisu Physical Review Letters. V něm shrnul, že „odchylky od sférické symetrie nezabrání vzniku
  prostoročasových singularit“. I po padesáti letech je tento článek mistrovským dílem úspornosti,
  jasnosti a matematické přesnosti: má jen tři stránky, na nichž je stručně uveden problém,
  matematické prostředky i důkaz tvrzení, ilustrovaný Penroseovým diagramem. 

  V době Chalatnikovovy londýnské přednášky byl už Penroseův článek odeslaný. Bylo pravděpodobné, že
  bude přijat k publikaci, a předpokládalo se, že vyjde v prosinci téhož roku. Penroseova technika
  však byla pro většinu posluchačů nezvyklá, především pro relativisty ze Sovětského svazu. Když se
  jeden z Wheelerových studentů Charles Misner přihlásil do diskuse a oponoval Chalatnikovovi na
  základě Penroseových výsledků, byla to předem prohraná bitva. Tyto výsledky byly Rusům podezřelé a
  striktně odmítli připustit, že by v jejich přístupu mohla být chyba. „Krčil jsem se v koutě,
  “vzpomíná Penrose, „bylo to hrozně nepříjemné.“

  Ale Penrose měl pravdu. Tvrzení, které získalo jméno „věta o singularitách“, má dalekosáhlé
  důsledky. Znamená, že pokud je obecná relativita správná, tak by Schwarzschildovo i Kerrovo
  řešení, tyto podivné prostoročasy se singularitami, měly skutečně v přírodě existovat. Nejsou to
  jen matematické konstrukce a Eddington s Einsteinem pravdu neměli. Čtyři roky po publikaci
  Penroseova článku Chalatnikov a Lifšic uznali svou porážku. Společně se studentem Vladimirem
  Bělinským se podívali na své výpočty znovu a ke svému zděšení v nich nalezli chybu. Zatímco v roce
  1961 se domnívali, že kolaps, který by vytvářel singularitu, byla velmi výjimečná událost, jejíž
  výskyt v přírodě by byl velmi nepřirozený, teď spolu s Bělinským zjistili pravý opak. Jejich
  výsledek jiným způsobem potvrzoval Penroseovy závěry: singularity v realistických situacích
  opravdu vznikají. Pokorně publikovali své nové výsledky a veřejně tak přiznali svůj omyl. 

  Penrose tedy matematicky prokázal nevyhnutelnost vzniku singularit a tím vyřešil Wheelerův problém
  finálního stavu. Brzy přišlo i hlubší potvrzení. 

  Martin Ryle byl při svém prvním pokusu vyvrátit teorii stacionárního vesmíru na základě pozorování
  rádiových zdrojů sice neúspěšný, jeho pozorovací data se však stále zlepšovala. Když v roce 1961
  uveřejnil svůj katalog rádiových zdrojů 4C, většina astronomů souhlasila, že řada chyb ve starších
  katalozích už byla odstraněna. Ale konec stacionárního vesmíru způsobili především jeho vlastní
  vyznavači. 

  Velkým propagátorem Hoyleovy teorie stacionárního vesmíru byl Dennis Sciama. Byl ale také
  fascinován kvasary a jednomu svému studentovi, Martinu Reesovi, zadal úkol podívat se na Ryleova
  nová měření z jiného úhlu. Rees přijal jednodušší a mnohem čistší způsob rozboru dat, než byl
  Ryleův. Vzal třicet pět kvasarů, u kterých byl dobře změřen červený posun, a rozdělil je do tří
  skupin. Jednu tvořily kvasary s malým červeným posunem, to znamená ty, jež byly od Země nepříliš
  vzdálené a signály z nich k nám neputovaly příliš dlouho. Ve druhé byly kvasary se středně velkým
  červeným posunem a konečně ve třetí kvasary s posunem hodně velkým, tedy kvasary, jejichž signály
  k nám byly vyslány v dávné minulosti. 

  Reesova myšlenka byla jednoduchá, ale mimořádně chytrá. Ve stacionárním vesmíru, který se nevyvíjí
  v čase, by všechny tři skupiny měly být přibližně stejně početné. Ve skutečnosti však Rees
  nenalezl prakticky žádné kvasary ve skupině nejbližší a středně vzdálené, téměř všechny byly ve
  skupině těch nejvzdálenějších. Jinými slovy, počet kvasarů se měnil s časem - nejvíce jich
  existovalo v minulosti - a tak vesmír nemohl být ve stacionárním stavu. Rees to jasně ukázal -
  teorie stacionárního vesmíru nefungovala. „Právě tento rozbor způsobil, že Dennis konvertoval,
  “vzpomíná Rees. Od toho okamžiku Sciama věřil v Lemaîtreovu teorii, neboli v teorii velkého
  třesku, jak ji nazval ve své přednášce Hoyle. Tento název byl pak všeobecně přijat.

  Poslední hřebík do rakve stacionárního vesmíru přišel z New Jersey na druhé straně Atlantiku. Arno
  Penzias a Robert Wilson prováděli měření v Holmdelu, v jednom z míst patřících k Bellovým
  laboratořím, kde se měřily rádiové vlny. Velikou anténu, která původně sloužila pro
  telekomunikační účely, chtěli upravit pro sledování rádiového záření z galaxie. K zmapování Mléčné
  dráhy potřebovali nejprve určit přesnost svého zařízení. Tak zamířili anténu nejdříve do prázdnoty
  a zkoumali, co vidí. 

  Co ale viděli, to nebylo očekávané „nic“. Penzias a Wilson zcela určitě něco viděli, nebo přesněji
  slyšeli: nízký jemný hvízdot přicházející z prázdného prostoru. Bez ohledu na to, jak upravovali
  svůj přístroj, tajemného signálu se nemohli zbavit. Tihle dva muži totiž nechtěně zakopli o relikt
  raného vesmíru, fosilii velkého třesku. 

  Koncem čtyřicátých let minulého století ruský fyzik žijící ve Spojených státech George Gamow
  předpověděl existenci velmi chladné lázně záření, prostupující celý prostor. Vyšel z myšlenky
  abbého Lemaîtra, že vesmír započal jako horká hustá polévka, ze které se nakonec vynořily všechny
  prvky. Jeho argumentace byla zhruba následující. 

  Představme si, že na počátku je vesmír plný vodíkových atomů. Atom vodíku je základním stavebním
  kamenem chemie, jsou to proton a elektron, které drží pohromadě elektromagnetická síla. Jestliže
  vodíkový atom bombardujeme dostatečně silným proudem energie, elektrony se mohou od protonů
  odtrhnout a zbydou protony vznášející se ve vesmírném prostoru. Nyní si představme vodíkový plyn v
  horké lázni. Atomy do sebe narážejí, pohybují se sem a tam a bombardují je energetické fotony
  záření. Čím teplejší je lázeň, to znamená čím větší je energie fotonů, tím pravděpodobnější je, že
  se elektrony od protonů odtrhnou. Je-li prostředí opravdu hodně horké, zůstane intaktní jen velmi
  málo vodíkových atomů. Místo atomů vodíku bude vesmír vyplněn volnými protony a elektrony.

  Ve skutečném raném vesmíru byl ovšem nejen vodík, byl zde i určitý počet neutronů a ty spolu s
  protony daly vzniknout jádrům helia, za teplot vyšších než několik tisíc stupňů však v něm bylo
  jen málo úplných atomů. Většinou v něm byly samostatné elektrony a jádra těch nejlehčích prvků,
  především vodíku a helia. Jak běžel čas, vesmír se rozpínal a tím chladnul. Téměř všechny
  elektrony se spojily s jádry. Vesmír byl naplněn vodíkovými a heliovými atomy s nepatrnou příměsí
  těžších prvků a malým množstvím záření. 

  Právě toto záření viděli Penzias s Wilsonem - byl to jasný důkaz horkého stavu vesmíru v jeho
  dětských letech. To odpovídalo vesmíru, jenž započal velkým třeskem. Následující krok v cestě za
  velkým třeskem udělal Stephen Hawking, další student Dennise Sciamy. 

  Mladý Hawking lecčíms připomínal Einsteina a když byl malý, spolužáci mu tak říkali. Ve škole
  nezářil, byl zlobivý a nepořádný žák, který rád bavil své spolužáky. Pak ho ale začala bavit věda
  a když se přihlásil na studia do Oxfordu, u přijímacích zkoušek skvěle obstál. Studium zde
  shledával velmi lehkým a na své tutory a přednášející dělal veliký dojem. Doktorské studium dělal
  v Cambridgi pod vedením Sciamy. Zde Hawking obrátil svůj zájem k vesmíru a ukázal na jeden
  důležitý důsledek objevu Penziase a Wilsona. 

  Stephena Hawkinga, o rok staršího než Martin Rees, fascinovala matematika obecné teorie
  relativity. Během jeho doktorského studia mu však byla diagnostikována Lou Gehringova nemoc a
  lékaři mu dávali jen několik let života. Zpráva ho zdrtila, ale ještě dva zbývající roky
  doktorského studia byl stále celkem v pořádku. To ho pobídlo, aby se soustředil na práci a snažil
  se porozumět, k čemu došlo na začátku universální expanze při samotném velkém třesku. Je možné, že
  počáteční singularita je v expandujícím vesmíru podobně nevyhnutelná, jako ve Wheelerově finálním
  stavu? 

  V dalších letech Hawking bojoval s pokračující nemocí a snažil se ve vědě vykonat co nejvíce.
  Podařilo se mu dokázat, že v expandujícím vesmíru skutečně je za normálních podmínek počáteční
  singularita. V dalších letech pak spolu s Georgem Ellisem, nadaným Sciamovým studentem z Jižní
  Afriky, ukázal, že každý vesmír s reliktním zářením, jaké objevili Penzias a Wilson, musí mít
  singulární počátek. Nakonec společně s Rogerem Penrosem stanovili kompletní sadu vět, jež
  pokrývala prakticky všechny v té době představitelné modely rozpínajícího se vesmíru. Penroseovy a
  Hawkingovy věty ukázaly, že singularity jsou nevyhnutelné, a to jak v minulosti, tak v
  budoucnosti. 

  Na prvním Texaském symposiu zazněla celá řada spekulací o vzdálených mohutných zdrojích rádiových
  vln v Ryleově katalogu. Soudilo se, že by to mohly být superhmotné hvězdy, hroutící se podle
  předpovědi obecné relativity. Chandra svého času ukázal, že superhmotní bílí trpaslíci by byli
  nestabilní a měli by se zhroutit. Výsledkem kolapsu by mohla být neutronová hvězda. Bílí trpaslíci
  byli dobře známé objekty, o existenci neutronových hvězd však chyběly jakékoli důkazy. To se však
  změnilo, když v roce 1965 přišla do Cambridge Jocelyn Bellová a začala pracovat na své doktorské
  práci v Ryleově skupině. 

  Bellová nepracovala přímo s Rylem, nýbrž s jeho o něco mladším kolegou Anthonym Hewishem. Ten jí
  postavil radioteleskop z dřevěných tyčí a drátěného pletiva, s jehož pomocí mohla určovat polohu
  kvasarů vysílajících signál o frekvenci 81,5 megahertzů. Jak na to Jocelyn vzpomíná, „první léta
  znamenala spoustu tvrdé práce pod širým nebem nebo ve velmi chladném přístřešku.“ Ale tato práce
  byla pro ni i přínosem: „Když jsem s tím skončila, mohla jsem mávat perlíkem.“ V roce 1967 Bellová
  zaznamenávala data pomocí grafického zapisovače a denně analyzovala asi třicet metrů záznamu a
  hledala charakteristické znaky kvasarů; celé nebe se vešlo na 120 metrů papíru. 

  Na záznamech však bylo něco podivného. Po každých 120 metrech narážela v záznamu na asi
  půlcentimetrový výběžek, kterému nerozuměla. Neuměla si představit, odkud se signál bere a co
  znamená. Signál tam nepochybně byl, jakési cvrlikání z určitého místa na nebi. „Začali jsme mu
  přezdívat malí zelení mužíčci,“ vzpomíná Jocelyn Bellová. „Domů jsem chodila značně otrávená.“Tým
  se nakonec rozhodl svůj tajemný výsledek zveřejnit. 

  V únorovém čísle Nature v roce 1968 se objevil článek nazvaný „Pozorování rychle pulzujícího
  rádiového zdroje“. V tomto článku Bellová, Hewish a další spoluautoři oznamovali, že „na
  Mullardově radioastronomické observatoři byly zaznamenány podivné signály“. Dále vyslovili
  odvážnou hypotézu: „Zdá se, že záření přichází od objektu uvnitř naší galaxie a mohlo by být
  spojeno s oscilacemi bílého trpaslíka nebo neutronové hvězdy.“Spekulovali o tom, že maxima v
  grafickém záznamu mohou představovat oscilace nebo pulzace těchto hutných kompaktních rádiových
  zdrojů. 

  Tisk po tomto objevu skočil a dotazoval se Hewishe, v čem spočívá jeho důležitost. Bellová si ale
  vzpomíná, že jí kladli i „veledůležité“otázky, jako například jestli je vyšší nebo stejně vysoká
  jako princezna Margareta, jaké jsou její míry, s kolika chlapci chodila, zkrátka na věci, kvůli
  kterým jsou podle nich ženy na světě. V deníku Sun měla zpráva o novém objektu titulek „Dívka,
  která odhalila malé zelené mužíčky“. Byl to ale Daily Telegraph, který přišel se jménem pro
  podivné objekty - jakýsi žurnalista navrhl název „pulzary“ jako zkratku termínu „pulzující rádiové
  hvězdy“. 

  Radioastronomie opět přinesla něco velikého a zase to bylo náhodou. Jednalo se o opravdu významný
  objev, v roce 1974 byla za něj udělena Nobelova cena školiteli Bellové Tonymu Hewishovi a Martinu
  Ryleovi. Jocelyn Bellová vyšla zcela naprázdno a mnozí to pokládají za jednu z největších
  nespravedlností v dějinách udílení Nobelovy ceny. O téměř dvacet let později se Bellová zúčastnila
  ceremoniálu udílení Nobelovy ceny jako host jiného astronoma Josepha Taylora jr., který cenu
  získal v roce 1993. „Konečně jsem se sem dostala“, komentovala to bez jakékoli hořkosti. 

  Pulzary byly prvním hmotným důkazem existence neutronových hvězd. Tyto hvězdy ve skutečnosti
  nepulzují, nýbrž rotují, a díky tomu vysílají periodické signály. Byl to ten pověstný chybějící
  článek v teorii gravitačního kolapsu, který studovali Landau a Oppenheimer a systematicky jej
  prozkoumali Wheeler a jeho žáci. Neutronová hvězda byla posledním stupněm před nezadržitelným
  kolapsem končícím Penroseovou singularitou. 

  Když Jakov Zeldovič změnil oblast svého výzkumu, udělal to beze strachu. Jeden z jeho studentů
  vzpomíná na jeho radu: „Osvojit si prvních deset procent znalostí v určité oblasti je sice těžké,
  ale zajímavé. … Cesta od deseti k devadesáti procentům, to už je čirá radost z tvoření. … Projít
  však úsek dalších devíti procent, to je nekonečně obtížné a za hranicemi možností koho- koli.
  Poslední procento je beznadějné. Je proto rozumné se přeorientovat na jiný problém dříve, než je
  příliš pozdě.“

  Tak jako Wheeler, i Zeldovič přešel od jaderného výzkumu k relativitě po své čtyřicítce a vytvořil
  jeden z nejlepších výzkumných týmů na světě. Články, které Zeldovič se svými žáky psal, často
  užívaly ve fyzice nezvyklého stylu, například: „Kmotr psychoanalýzy Sigmund Freud nás učí, že
  chování v dospělosti závisí na zkušenostech z raného dětství…Současná struktura vesmíru má být
  důsledkem jeho raného chování.“ Jejich články se četly jako zhuštěné eseje, byl v nich jen malý
  počet rovnic nezbytných k osvětlení základního výsledku. Jejich překlady do angličtiny bylo často
  těžké dešifrovat. Časem byly ale oceněny tak, jak si to zasloužily - byly to pravé klenoty
  relativistické astrofyziky. 

  Prvním Zeldovičovým cílem na novém poli bylo hledání zamrzlých hvězd, jak se na Východě říkalo
  Schwarzschildovým či Kerrovým kolabujícím hvězdám. Zamrzlé hvězdy jsou neviditelné, nevyzařují
  žádné světlo a nemají žádný povrch, který by mohl zářit či odrážet světlo. Zeldovič ale nemohl
  přijmout, že by neexistoval způsob, jak tyto podivné objekty pozorovat; křiví a deformují totiž
  prostoročas kolem sebe. Vyvíjejí neobyčejně silný tah na vše, co se k nim přiblíží. A tak Zeldovič
  a jeho studenti začali diskutovat o tom, jaký vliv by přítomnost zamrzlé hvězdy měla na objekty v
  jejím okolí a co by ji umožnilo pozorovat aspoň nepřímo. Kdyby se například v blízkosti zamrzlé
  hvězdy nacházelo Slunce, muselo by kolem ní obíhat, podobně jako Měsíc obíhá Zemi. Protože samotná
  zamrzlá hvězda by byla neviditelná, vypadalo by to, že Slunce poskakuje na obloze po podivné
  dráze, v jejímž centru se nic nenachází. Dívejte se proto po poskakujících hvězdách, vyzýval
  Zeldovič, po takových, které sice vidíme jako osamocené, ale chovají se jako jedna složka
  dvojhvězdy. 

  Zeldovič ale předložil další hypotézu. Viditelní partneři nebudou zamrzlé hvězdy jen poklidně
  obíhat, silné gravitační účinky zamrzlých hvězd je budou trhat na kusy. Předpokládal, že látka z
  viditelné komponenty bude padat k zamrzlé hvězdě a přitom se bude urychlovat až na rychlost
  blízkou rychlosti světla. Přitom se bude mísit a srážet, což povede k jejímu zahřívání - tomuto
  procesu se říká akrece. A horký materiál bude vyzařovat energii. Akrece v blízkosti
  Schwarzschildova horizontu je tak efektivní, že látka může vyzářit energii ekvivalentní až deseti
  procentům své klidové hmotnosti - takový energii generující proces nemá ve vesmíru obdoby. V
  krátkém článku v časopise Doklady Akademii nauk z roku 1964 vyslovil Zeldovič předpoklad, že
  produkce energie v blízkosti zamrzlé hvězdy bude tak obrovská, že to postačí k vysvětlení produkce
  energie v intenzivně zářících kvasarech, které objevili radioastronomové. Ve stejnou dobu došel k
  obdobnému závěru i americký astronom z Cornellovy univerzity Edwin Salpeter, podle něhož mohutné
  rádiové záření z kvasarů může vycházet z objektů, jejichž hmotnost může překračovat milion
  slunečních hmotností, neboli, jak to formuloval, „z extrémně hmotných objektů relativně malých
  rozměrů“. 

  Tady se však Zeldovič nezastavil. Společně se svým mladým kolegou Igorem Novikovem aplikoval svůj
  argument na binární systém tvořený normální hvězdou obíhanou hvězdou zamrzlou. Došli k závěru, že
  obrovská gravitační přitažlivost zamrzlé hvězdy připraví normální hvězdu o všechny svrchní vrstvy
  plynu. Je to podobné, jako kdybychom „měli vypustit vanu velikosti jezera Loch Lomond odtokem,
  jaký mají běžné vany“, řekl jednou Roger Penrose. Síly, kterým bude plyn vystaven, jsou tak
  obrovské, že se vyzáří veliké množství světla o velmi krátkých vlnových délkách, takzvaného
  rentgenového záření. „Pátrejte po zdrojích rentgenového záření“, tak zněla výzva Zeldoviče a jeho
  žáků astronomickému světu. 

  Jak bylo stále přesvědčivější, že existuje vztah mezi zamrzlými či zkolabovanými hvězdami a
  kvasary, tak se ve vědeckých článcích astronomů a astrofyziků stále častěji objevovalo jméno
  Schwarzschild. Jak ale po několika letech vzpomínal Wheeler, název „plně zkolabovaný gravitační
  objekt“, jenž se užíval ve Spojených státech, byl dlouhý a nehezký. „Když jste to řekli desetkrát
  za sebou, začali jste zoufale pátrat po lepším termínu.“Na konferenci v Baltimoru v roce 1967
  navrhl někdo z posluchačstva termín černá díra. Wheeler jej přijal a název se ujal. 

  V roce 1969 jeden ze Sciamových kolegů v Cambridgi Donald Lynden-Bell napsal v úvodu k jednomu ze
  svých článků: „Byl by však chybný závěr, že takové velmi hmotné objekty v prostoročase by byly
  nepozorovatelné. Jsem toho názoru, že nepřímo je pozorujeme už po řadu let.“ Argumentoval, že
  velmi hmotné černé díry v centrech galaxií by vysávaly okolní hmotu takovým způsobem, jak to
  popsal Penrose - jako prudce vířící vodu vytékající z umyvadla. Rotující plyn kolem černé díry by
  vytvořil plochý disk podobný Saturnovým prstencům a celý systém by se otáčel kolem osy. Jádra
  galaxií, krmená těmito akrečními disky, by byla opravdovým majákem světla a Lyndon-Bell uměl
  ukázat, jak je energie vytvářena a emitována. Detailní modely kvasarů budovali i Martin Rees s
  Dennisem Sciamou. Tyto modely měly vyložit všechny podivné vlastnosti kvasarů - jejich rozměry,
  vzdálenost, rychlost, s jakou se mihotají a pulzují, a kolik energie vysílají. Během několika
  následujících let byli Rees, Lynden-Bell a jejich studenti a postdokové v Cambridgi schopni
  navrhnout krásný, pečlivě vypracovaný model ohňostroje obklopujícího kvasary a rádiové zdroje.
  Všechny kusy skládačky začaly do sebe zapadat. 

  A pak se konečně začaly dostávat ke slovu i zdroje rentgenového záření Zeldoviče a Novikova. V
  šedesátých letech minulého století začal tým vedený italským fyzikem Riccardem Giacconem vysílat
  rakety nad zemskou atmosféru, která rentgenové záření pohlcuje, a tam několik minut pátraly po
  rentgenových paprscích. A nalezly je, jejich zdroji byly jasné skvrnky přezařující planety
  slunečního systému. V roce 1970 byl ze základny blízko Mombasy v Keni vypuštěn satelit Uhuru,
  jehož jediným cílem bylo mapovat rentgenovou oblohu. Byl to velmi úspěšný experiment, nalezlo se
  asi tři sta rentgenových zdrojů. 

  Mezi řadou objektů, které Uhuru zaměřil, byl výjimečně jasný zdroj v souhvězdí Labutě (Cygnus)
  označený jako Cygnus X-1. Ten byl poprvé zpozorován už v roce 1964 při raketových letech, ale
  měření z Uhuru ukázala, že bliká neobyčejně rychle, několikrát za sekundu. To bylo neklamným
  znamením, že se jedná o velmi kompaktní objekt. Na měření z Uhuru rychle navázala pozorování na
  rádiových a optických frekvencích, jež identifikovala zdroj s vlastnostmi, které předpověděli
  Zeldovič s Novikovem - hvězdu, která je pomalu zbavována své obálky a vypadá, Singularity že
  drobně poskakuje, jak obíhá kolem neviditelného hustého objektu, jehož hmotnost je okolo
  osminásobku hmotnosti Slunce. Konečně zde byl první důkaz existence černé díry - ne nezvratný, ale
  rozhodně silný. Zdroj Cygnus X-1 byl s největší pravděpodobností černou dírou - malý, neviditelný,
  který však byl mohutným zdrojem rentgenového záření. 

  V létě 1972 uspořádali Bryce a Cécile DeWittovi letní školu v Les Houches ve francouzských Alpách.
  Mezi přednášejícími byli mladí relativisté, žáci Sciamy, Wheelera a Zeldoviče, kteří se zatím
  stali světově uznávanými autoritami: Brandon Carter a Stephen Hawking z Cambridge, Kip Thorne,
  jeho žák James Bardeen, Remo Ruffini z Caltechu a Princetonu, i Igor Novikov, který reprezentoval
  Moskvu. To byli noví proroci černých děr. „Dnes je už dobře známý příběh, jak se obecná teorie
  relativity během pouhého desetiletí dostala z klidného přístavu na okraji zájmu do přední linie
  vědeckého bádání, kde přitahuje víc a více talentovaných mladých lidí,“ napsali DeWittovi v úvodu
  ke sborníku z Les Houches. „Žádný objekt či koncept necharakterizuje současný stav vývoje tak
  výstižně jako černá díra.“ Setkání v Les Houches bylo vyvrcholením dekády fenomenálních objevů. 

  Einstein a Eddington se hluboce mýlili. I Wheeler se vzdal a kolem roku 1967 uznal, že příroda se
  neděsí singularit v prostoročase obecné relativity. Schwarzschildovo řešení, objevené před dávným
  časem na bitevním poli východní fronty, i Kerrovo řešení, zkonstruované v texaském vedru, byly
  reálné a musí v přírodě existovat - jsou to výsledné prostoročasy gravitačního kolapsu.
  Předpověděla je obecná teorie relativity, jsou ve své podstatě jednoduché a gravitační kolaps k
  nim nevyhnutelně vede. Dovedou přitom úžasné věci: zbavují hvězdy obalu a jsou zdrojem energie
  kvasarů. Rádiové nebe neustále vysílá podivné záblesky a objevené rentgenové signály ukazují, že
  vycházejí od velice kompaktních hustých objektů. Měření ještě neřekla své poslední slovo, ale
  reálná existence černých děr se zdá být nevyhnutelná. Na to, které z podivných objektů na nebi
  jsou černé díry, se uzavírají sázky, ale jejich existence se už pokládá za téměř jistou. 

  Vědci shromáždění v Les Houches v předchozích letech také ukázali, že najdou-li se v přírodě černé
  díry, musí být tak jednoduché jako Schwarzschildovo či Kerrovo řešení. 

  Ezra („Ted“) Newman ze Syracuské univerzity lehce rozšířil Kerrovo řešení, které pak představovalo
  černou díru, jež byla navíc elektricky nabitá. Plné řešení Einsteinových rovnic představující
  černou díru pak záviselo na pouhých třech parametrech: její hmotnosti, momentu hybnosti,
  charakterizujícím její rotaci, a jejím náboji. To byl ohromující výsledek. 

  Proč nemůže mít černá díra na jedné straně trochu více hmoty, něco jako horu na zemském povrchu, a
  naopak na druhé straně trochu méně hmoty, jakési údolí? Snadno si představíte černé díry, které
  mají tři stejné základní charakteristiky, a přitom vypadají velmi rozdílně. Jenže matematika
  obecné relativity dokazuje, že rozdíly by velmi rychle vymizely. Hory by se zploštily, údolí by se
  vyplnila, deformované oblasti by se vyhladily. Černé díry se stejnou hmotností, rotací a nábojem
  by se rychle usadily do stavu, že by byly zcela nerozeznatelné. Wheeler popsal tento monotónní
  vzhled tak, že „černá díra nemá vlasy“. 

  Setkání v Les Houches ukázalo, čeho dosáhnou skvělé mozky, když společně řeší velký problém.
  Martin Rees vzpomíná na toto období: „Černým dírám se pokoušely porozumět tři skupiny - moskevská,
  cambridgeská a princetonská. A já jsem měl vždy pocit, že jsou mezi nimi kolegiální vztahy.“ A
  skutečně v této době skoro naprosté izolace Západu a Východu jejich pracovní setkání posunovala
  vědu kupředu. Kip Thorne a Stephen Hawking navštívili Zeldoviče v Moskvě a diskutovali s ním o
  akrečních discích, gravitačním kolapsu a singularitách. Stejně důležité byly i krátké a obtížné
  cesty sovětských fyziků na Západ. Jak vzpomínal Novikov na svou návštěvu Texaského symposia v roce
  1967, tentokrát konaného v New Yorku: „Přes naši zoufalou snahu nasbírat co nejvíce informací a
  mluvit s co nejvíce kolegy bylo fyzicky nemožné pokrýt vše zajímavé.“ O několik let později na
  setkání v Les Houches v roce 1972 napsali Thorne a Novikov společný článek o akrečních discích. 

  Během deseti let se obecná teorie relativity změnila. Texaská symposia se konala pravidelně a
  setkávali se zde stovky astrofyziků, z nichž někteří se pokládali za relativisty. Jak řekl Roger
  Penrose: „Viděl jsem proměnu černých děr z matematických úloh v něco, v co lidé opravdu věřili.“

  Generace, která vzešla ze zlatého věku obecné relativity, byla odměněna vedoucími místy na
  některých špičkových univerzitách. Ve Velké Británii Martin Rees a Stephen Hawking získali
  prestižní postavení v Cambridgi a Roger Penrose v Oxfordu. Wheelerovi studenti se stali profesory
  v Caltechu, Marylandu a na dalších význačných univerzitách a akademická kariéra čekala i členy
  Zeldovičova týmu v Sovětském svazu. Einsteinova teorie se stala částí hlavního proudu ve fyzice, a
  to triumfálním způsobem. 

\section{Útrapy se sjednocením}\label{feyIchIIIsecX}
  V roce 1947 se Bryce DeWitt, čerstvý absolvent univerzity, setkal s Wolfgangem Paulim a řekl mu,
  že pracuje na kvantování gravitačního pole. Nemohl pochopit, proč dvě velké teorie dvacátého
  století - kvantová teorie a obecná teorie relativity - nejsou navzájem propojeny. „Co je tak
  zvláštního na gravitačním poli, že si udržuje svou izolovanost? Co kdyby je někdo násilím vtáhl do
  hlavního proudu fyziky a kvantoval je?“ To byl DeWittův argument. Pauli jeho nápad plně nepodpořil
  a řekl: „To je velice důležitý problém, ale jeho vyřešení bude chtít někoho opravdu chytrého.
  “Nikdo nezpochybňuje DeWittovu neobyčejnou inteligenci, ale během uplynulých více než padesáti let
  příroda jeho snaze úspěšně odolávala. 
  
  Gravitační pole, jak je popisovala obecná teorie relativity, bylo přitom jedinou výjimkou, která
  pokusům o kvantování a sjednocení popisu odolávala. Rozvoj kvantové teorie po druhé světové válce
  vedl k nové mocné teorii, která spojila dohromady všechny síly - tedy všechny síly kromě právě
  gravitace. O jednotnou teorii gravitace a elektromagnetismu se po léta snažili Einstein i
  Eddington, oba bez úspěchu. Kvantová teorie byla jiná. Její správnost ověřovaly gigantické
  experimenty na urychlovačích částic v Evropě i Spojených státech. Byl to příběh šťastného
  manželství mezi nádhernou matematikou a praktickými experimenty. 
  
  Přes tyto úspěchy však byli i takoví přední vědci, kteří poválečný rozvoj kvantové teorie nevítali
  s nadšením. Například Paul Dirac tvrdil, že kvantová teorie částic a sil je nehezká a že je
  příkladem zmateného uvažování. Nelíbilo se mu, že používá nepěkného triku - utíká od základních
  problémů tím, že nechává záhadně zmizet nekonečné hodnoty některých veličin. Byl přesvědčen, že je
  to právě tato magie, jež brání gravitaci, aby se připojila k slavnému sjednocení všech sil. 
  
  Na Paulu Diracovi bylo cosi neproniknutelného. Byl to štíhlý vysoký muž, který ve společnosti
  skoro nepromluvil. A když už mluvil, jeho slova byla vždy až příliš přesná a úzce zaměřená. Byl
  samotářský a stydlivý, dával přednost samostatné práci a byl posedlý matematickou krásou, o níž
  byl přesvědčen, že prostupuje celou realitu. Jeho články byly matematické drahokamy, jež však měly
  dalekosáhlé důsledky pro reálný svět. Studoval původně v Bristolu technické disciplíny, pak však
  krátce po svém dvacátém roce přišel do Cambridge a záhy se z něj stal jeden z proroků nové
  kvantové teorie. Stal se asistentem na St. John’s College v Cambridgi a zanedlouho byl jmenován
  lucasovským profesorem. Toto prestižní postavení zastával v sedmnáctém století Isaac Newton.
  Cambridge poskytla Diracovi útočiště, kde se mohl skrýt před světem a zároveň ovlivňovat generace
  fyziků, mezi nimi i některé astrofyziky a relativisty, kteří v šedesátých letech minulého století
  dodali obecné teorii relativity novou energii. Jeho doktorskými studenty byl jak Fred Hoyle, tak
  Dennis Sciama a jeho přednášky navštěvoval i Roger Penrose, který obdivoval jejich jasnost a
  přesnost. 
  
  Byl to právě Paul Dirac, kdo objevil základní rovnici pro elektron. Rovnice nese Diracovo jméno a
  byla prvním krokem k sjednocení Einsteinova principu relativity a základů kvantové teorie. Rovnice
  kvantové fyziky nám říkají, jak se kvantový stav nějakého systému, například elektron vázaný k
  protonu v atomu vodíku, vyvíjí v čase. To velice jasně rozlišuje mezi úlohou prostoru a času.
  Einsteinova speciální teorie relativity spojila prostor a čas do jednoho nedělitelného jsoucna -
  prostoročasu. Kombinuje také koherentním způsobem zákony mechaniky a zákony světla. Paulu Diracovi
  se podařilo formulovat zákony kvantové teorie ve stejném pojmovém rámci. Jsou-li mikročástice
  popsány Diracovou rovnicí, pak celá fyzika včetně kvantové mechaniky vyhovuje speciálnímu principu
  relativity. 
  
  Částice ve vesmíru se dají rozdělit do dvou velkých skupin - na fermiony a bosony. Fermiony jsou
  základními stavebními kameny atomárních látek, zatímco bosony jsou nositeli síly mezi nimi. K
  fermionům tedy kromě dalších částic patří elektrony, protony a neutrony. Když jsme hovořili o
  bílých trpaslících a neutronových hvězdách, zmiňovali jsme se o podivné kvantově mechanické
  vlastnosti fermionů - podléhají Pauliho vylučovacímu principu. Ten říká, že dva fermiony se
  nemohou nacházet ve stejném kvantovém stavu. Proto jsou-li stlačovány do menšího a menšího objemu,
  navzájem se odpuzují a budí kvantový tlak. Víme už, že Fowler, Chandra a Landau vysvětlili pomocí
  tohoto tlaku rovnováhu bílých trpaslíků a neutronových hvězd, pokud jsou jejich hmotnosti menší
  než hmotnosti kritické. Na druhé straně bosony Pauliho vylučovacímu principu nepodléhají, a proto
  mohou být stlačeny do objemu libovolně malého. Příkladem bosonu je foton, který přenáší
  elektromagnetickou sílu. 
  
  Diracova rovnice popisuje kvantově mechanické chování elektronu, tedy fermionu, a zároveň vyhovuje
  i Einsteinově speciální teorii relativity. Můžeme z ní vyčíst pravděpodobnost, se kterou se určitá
  částice nachází v dané poloze v prostoru, nebo jakou má v daném okamžiku rychlost. Nevybírá
  speciálním způsobem prostor, nýbrž určuje kvantový stav v prostoročase, tak jak to vyžaduje
  speciální teorie relativity. Diracova rovnice dává hluboký vhled do přirozeného světa
  elementárních částic. I jejího autora překvapilo, když zjistil, že předpovídá existenci
  antičástic. Antičástice mají stejnou hmotnost jako odpovídající částice, ale opačný náboj.
  Antičástice k elektronu dostala jméno pozitron. Pozitron má všechny vlastnosti stejné jako
  elektron, jenom jeho náboj je kladný. Podle Diracovy rovnice musí v přírodě existovat jak
  elektrony, tak pozitrony. Navíc se ukázalo, že pár elektron-pozitron může samovolně vyskočit z
  vakua, jako by vznikl z ničeho. To se jeví hodně bizarní a je těžké tomu porozumět. Ještě těžší to
  bylo v době, kdy Dirac svou rovnici formuloval, protože pozitron nikdy nikdo do té doby neviděl.
  Dirac byl s tvrzením, že pozitrony opravdu existují, opatrný. V roce 1932 však byly nalezeny v
  kosmickém záření. Rok nato Dirac získal Nobelovu cenu. 
  
  Diracova rovnice odstartovala revoluci v chápání částic a sil v přírodě. Jestliže se kvantová
  fyzika elektronu dala popsat ve stejném rámci jako elektromagnetické pole - tedy v souladu s
  Einsteinovým speciálním principem relativity -, proč by se elektromagnetické pole nemohlo
  kvantovat podobným způsobem jako elektron? Elektromagnetické pole popisuje světelné vlny, ale jeho
  kvantováním dostaneme fotony, hypotetická kvanta energie, jejichž existenci Einstein předpokládal
  při svém vysvětlení fotoefektu v roce 1905. Kvantová teorie jak elektronů, tak světla, nazvaná
  kvantová elektrodynamika, často označovaná stručně jako QED podle anglického Quantum
  Electrodynamics, byla dalším krokem ve sjednocení popisu částic a sil. Krátce po druhé světové
  válce ji vybudovali Richard Feynman, Julian Schwinger a Sin-Itiro Tomonaga a signalizovala nový
  přístup ke studiu kvantové fyziky - kvantový popis částic (elektronů) a sil (elektromagnetického
  pole) koherentním způsobem. Kvantová elektrodynamika byla fenomenálním úspěchem. Dovolovala
  předpovídat vlastnosti elektronů a elektromagnetického pole s bezprecedentní přesností - více než
  právem byli její tvůrci odměněni Nobelovou cenou. 
  
  I když však kvantová elektrodynamika skvěle fungovala, Dirac se na ni díval s velkou nelibostí.
  Klíčem k jejímu úspěchu byla totiž metoda výpočtu, která odporovala jeho víře v matematickou
  jednoduchost a eleganci. Říká se jí renormalizace. Abychom pochopili, o co se jedná, všimněme si,
  jak se v kvantové elektrodynamice počítá hmotnost elektronu. Hmotnost elektronu byla laboratorně
  změřena s velkou přesností a činí 9,1 desetiny miliardtiny miliardtiny miliardtiny gramu - je to
  opravdu nepatrná hodnota. Jenže první výpočet na základě kvantové elektrodynamiky dá pro hmotnost
  elektronu hodnotu nekonečnou. Je to proto, že kvantová elektrodynamika dovoluje vznik a opětný
  zánik fotonů a elektronových-pozitronových párů, tedy částic a antičástic, jejichž existence
  vyplynula z Diracovy rovnice. Tyto páry vznikají jakoby z ničeho, i když „žijí“ jen velmi krátkou
  dobu. A všechny tyto virtuální částice vyskakující z vakua přispívají k měřitelné hmotnosti
  elektronu tak, že součet všech příspěvků dává nakonec nekonečnou hodnotu. A tak kvantová
  elektrodynamika aplikovaná necitlivě dává všude kolem nás spoustu nekonečen. Jenže Feynman,
  Schwinger a Tomonaga argumentovali tak, že protože víme, že hmotnost elektronu je konečná, tak
  tuto nekonečnou hodnotu prostě odstraníme a nahradíme měřenou hmotností - tento proces, jenž však
  podléhá určitým pravidlům, se právě nazývá renormalizace. 
  
  Může se ovšem zdát, že celé kouzlo renormalizace spočívá v tom, že nekonečna se zahodí a nahradí
  se hodnotou konečnou, přičemž celý postup není jednoznačný, je v něm určitá libovůle. Dirac sám
  prohlásil, že je „se situací velice nespokojen“. Argumentoval: „To prostě není seriózní
  matematika. V rozumné matematice se zanedbává něco, co je velmi malé - ne co je nekonečně velké, a
  to jen proto, že se nám to nelíbí!“ Vypadalo to jako velice zmatené uvažování, i když nikdo nemohl
  popřít, že jinak kvantová elektrodynamika skvěle fungovala. 
  
  Kvantová elektrodynamika byla prvním krokem na dlouhé pouti ke sjednocení. Mezi třicátými a
  šedesátými lety minulého století se však vyjasnilo, že kromě elektromagnetické a gravitační síly
  existují ještě dvě další síly neboli interakce, jež by měly být popsány v jednotném rámci. Jednou
  z nich byla takzvaná slabá síla, jejíž existenci navrhl italský fyzik Enrico Fermi, aby vysvětlil
  jeden typ radioaktivity známý jako rozpad beta. Při rozpadu beta se neutron mění v proton a při
  tom uvolní elektron. Tento proces se však nedal vyložit jen působením elektromagnetické síly a tak
  Fermi vyslovil předpoklad, že existuje ještě jiná dosud neznámá síla. Tato síla však měla působit
  jen na velmi krátké vzdálenosti v řádu jaderných rozměrů a měla být mnohem slabší než
  elektromagnetismus - odtud její název slabá interakce. 
  
  Tou druhou silou, zvanou naopak silná interakce, je síla, jež spolu váže protony a neutrony.
  Působí i mezi fundamentálnějšími částicemi, takzvanými kvarky, ze kterých jsou složeny protony,
  neutrony a řada dalších elementárních částic, zvaných hadrony. I tato síla působí jen na krátkých
  vzdálenostech, je však mnohem silnější, jak napovídá i její jméno. Před fyziky stála obdobná
  výzva, jako před Jamesem Clerkem Maxwellem v devatenáctém století. Jemu se tehdy podařilo
  sjednotit do jediné síly elektrickou a magnetickou sílu. O necelých sto let později stál před
  fyziky úkol sjednotit popis čtyř fundamentálních interakcí: elektromagnetické, silné, slabé a
  gravitační. 
  
  Silná a slabá interakce se systematicky a detailně zkoumaly v padesátých a šedesátých letech. Jak
  se jim začínalo lépe rozumět, vyvstávala jejich matematická podobnost s interakcí
  elektromagnetickou, a to napovídalo, že všechny tři interakce jsou projevem interakce jediné, jež
  se jen projevuje různým způsobem za různých okolností. Na konci šedesátých let pak Steven Weinberg
  z Massachusetts Institute of Technology (MIT), Sheldon Glashow z Harvardu a Abdus Salam z Imperial
  College v Londýně přišli s myšlenkou, jak spojit dohromady alespoň dvě z těchto interakcí,
  elektromagnetickou a slabou - začalo se hovořit o elektroslabé interakci či síle. Spojit je i se
  silnou interakcí se v té době sice ještě nepodařilo, ale jejich matematická podobnost byla tak
  veliká, že se pevně věřilo v sjednocení všech těchto tří interakcí do „teorie velkého sjednocení
  “elektromagnetismu se slabou a silnou interakcí. Důležité bylo, že v sedmdesátých letech se
  podařilo dokázat, že jak elektroslabá, tak silná interakce jsou renormalizovatelné, podobně jako
  kvantová elektrodynamika. Všechna nepříjemná nekonečna v těchto teoriích se dala nahradit
  konečnými hodnotami, takže teorie dávaly ověřitelné předpovědi. Kombinace elektroslabé a silné
  interakce dostala název „standardní model“ a některé její přesné předpovědi byly ověřeny na
  urychlovačích částic v CERN, evropském centru nukleárního výzkumu ve švýcarské Ženevě. Tato téměř
  úplně sjednocená kvantová teorie tří interakcí - elektromagnetické, silné a slabé - byla přijata
  všemi, až na drobné výjimky, k nimž patřil především Paul Dirac. I on sice obdivoval mladou
  generaci fyziků, která vybudovala standardní model pomocí nádherné matematiky, opakovaně však
  brojil proti nekonečnům a jejich odstranění trikem s renormalizací, který pokládal za podvodný. V
  několika veřejných přednáškách, kde se vůbec zmínil o standardním modelu, peskoval své kolegy za
  to, že se důrazněji nepokusili o vytvoření lepší teorie bez nekonečen. Ke konci své cambridgeské
  kariéry byl však Dirac stále izolovanější, protože tvrdohlavě odmítal pokrok, který v kvantové
  teorii nastal. Miloval sice soukromí, ale přesto ho mrzelo, že zbytek fyzikálního světa ignoruje
  jeho námitky proti kvantové elektrodynamice a na něj se dívá jako na přežitek starých časů.
  Ponechal si svou studovnu v St. John’s College, vyhýbal se však ústavu, kde měl profesuru.
  Nevěnoval pozornost velkým objevům v obecné teorii relativity, které přicházely od lidí jako
  Dennis Sciama, Stephen Hawking, Martin Rees a dalších. Jak vzpomíná jeden z jeho cambridgeských
  současníků, „Dirac byl jako duch, kterého jsme málokdy potkali a nikdy s ním nemluvili.“ V roce
  1969 se vzdal své lucasovské profesury a přijal profesorské místo na Floridě. V jeho posledních
  letech ho však jistě nepřekvapilo, že se obecná teorie relativity odmítá podřídit renormalizační
  technice. 
  
  Bryce DeWitt si vůbec neuměl představit, jaký zápas ho čeká při snaze kvantovat gravitaci. Na
  Harvardu pracoval s Julianem Schwingerem a viděl tak zrod kvantové elektrodynamiky od samého
  počátku. Když se rozhodl věnovat se kvantové gravitaci, chtěl postupovat obdobně jako u
  elektromagnetismu a zreprodukovat úspěch kvantové elektrodynamiky. Mezi oběma teoriemi byla určitá
  podobnost, zejména v tom, že obě popisovaly síly dlouhého dosahu. V kvantové elektrodynamice se
  přenos interakce mezi nabitými částicemi popisoval jako výměna nehmotné částice neboli fotonu.6 Na
  elektromagnetismus se 6 Nehmotnou částicí se rozumí částice, jejíž klidová hmotnost by byla rovna
  nule. Takové přiřazení klidové hmotnosti je ovšem jen formální, tato částice v klidu být nemůže,
  vždy se pohybuje rychlostí světla. Pozn. překl. můžeme dívat jako na moře fotonů, které přeskakují
  z jednoho náboje na druhý a přitom je buď odpuzují, nebo naopak táhnou k sobě, podle toho, zda jde
  o souhlasné, nebo opačné náboje. DeWitt přistoupil ke kvantové teorii gravitace podobným způsobem,
  jen nahradil elektron jinou nehmotnou částicí, gravitonem. Gravitony mají přeskakovat mezi
  hmotnými částicemi a tím je táhnout k sobě navzájem. Výsledkem takového přitahování je pak to, co
  nazýváme gravitační silou. Je vidět, že tento přístup opouští všechny krásné myšlenky o geometrii
  prostoročasu, které k obecné teorii relativity vedly. DeWitt o gravitaci uvažoval jen jako o další
  síle, na kterou se dají uplatnit všechny technické postupy užívané v kvantové elektrodynamice.
  
  Následujících dvacet let se DeWitt snažil budovat kvantovou teorii gravitonů, ale zjišťoval, že je
  to gargantuovský problém. Opět se ukázalo, že Einsteinovy rovnice jsou příliš komplikované a
  navzájem propletené, než aby se na ně daly přenést postupy z elektromagnetismu. DeWitt pozorně
  procházel postup kvantování ostatních polí a zaměřoval se na podobnosti obtíží. Jenže léčba
  problémů kvantových teorií ostatních tří polí u gravitace nezabírala a sjednocení s ostatními
  silami se Einsteinovo gravitační pole urputně bránilo. Ve svém boji o kvantovou gravitaci nebyl
  DeWitt osamocený - Matvej Bronštejn,7 Paul Dirac, Richard Feynman, Wolfgang Pauli i Werner
  Heisenberg dospěli k představě gravitonu tak jako on. I architekti úspěšného modelu elektroslabé
  interakce Steven Weinberg a Abdus Salam se pokoušeli aplikovat osvědčené techniky standardního
  modelu na gravitaci, ale shledali ji příliš složitou. 
  
  DeWittova práce budila velký zájem různých skupin. Povzbuzoval ho John Wheeler a obdobným
  problémům se věnovali někteří jeho studenti, nad kvantovou gravitací uvažoval i pákistánský fyzik
  Abdus Salam, Dennis Sciama v Oxfordu a Stanley Deser v Bostonu. Obecně však vztah fyziků ke
  kvantové 7 Matvej Petrovič Bronštejn (1906-1938), skvělý ruský fyzik popravený za stalinských
  čistek. Pozn. překl. gravitaci byl smíšený a často chladný. Michael Duff, bývalý student Abduse
  Salama, vzpomíná, jak na konferenci v Cargèse na Korsice byl jeho příspěvek o kvantové gravitaci
  přijat s pobavením. Sciamův student Philip Candelas, který zkoumal vlastností různých polí
  žijících na prostoročasech s různými geometriemi, zaslechl, jak si členové fyzikální fakulty v
  Oxfordu o jeho práci říkají, že „to není fyzika“. Kvantová gravitace byla stále něco příliš
  abstraktního ve srovnání s kvantovou teorií jiných polí a zabývat se jí představovalo pro mnoho
  fyziků ztrátu času. 
  
  V únoru 1974 hospodářství Spojeného království stagnovalo. Cena nafty vystřelila vzhůru,
  neefektivní vlády se snažily brzdit rostoucí inflaci, země byla ochromena spory v průmyslu.
  Pracovní týden byl občas zkracován kvůli úspoře energie a kvůli vypínání proudu se večerní jídlo
  často jedlo při svíčkách. Za těchto temných dnů se uskutečnilo setkání, jež mělo informovat o
  pokrocích v kvantování gravitace. Skoro pětadvacet let poté, co DeWitt začal na této problematice
  pracovat, se konalo oxfordské Symposium o kvantové gravitaci a přes ponurou ekonomickou situaci na
  něm vládla euforie. Předpovědi standardního modelu byly právě spektakulárně potvrzeny na velkém
  urychlovači v CERN a čekalo se, že teď přichází na řadu kvantová gravitace.
  
  Jak ale vystupovali jednotliví řečníci a přednášeli své myšlenky, ukazovalo se, že tu
  nejpřímočařejší a zdánlivě nejnadějnější cestu brzdí stále stejný problém, na který narážel i
  DeWitt. Nebylo možné zapomenout na geometrii a gravitaci pojímat pouze jako další sílu. Zazněla
  zde i parafráze Bible, kterou vyslovil Wolfgang Pauli: „A protož, co Bůh rozdělil, člověk
  nespojuj!“Obecná teorie relativity se totiž od kvantové elektrodynamiky i standardního modelu
  podstatně lišila. Kvantová elektrodynamika i standardní model částic dovolovaly renormalizovat
  všechny hmotnosti i náboje jednotlivých částic a změnit je tak, aby vedly k rozumným výsledkům.
  Ale stejná technika se při aplikaci na obecnou teorii relativity hroutila. Nekonečna se hromadila
  a odmítala se dát renormalizovat. Když jste se jich zbavili v jedné části teorie, nutně se
  objevila v jiné a odstranit je zároveň všude se ukázalo jako nemožné. Gravitace, tak jak ji
  popisuje obecná teorie relativity, je příliš složitá, než aby se s ní dalo zacházet jako s
  ostatními interakcemi. Na oxfordském symposiu pronesl Mike Duff v závěru svého příspěvku
  zlověstnou předpověď: „Vypadá to, že vše se proti nám spiklo a že jen zázrak by mohl způsobit, aby
  se gravitace dala renormalizovat.“Kvantová gravitace se dostala do slepé uličky a obecná
  relativita se vzpouzela tomu, aby se spojila s ostatními silami v jednotném obraze. Jak smutně
  konstatoval článek o symposiu v časopise Nature: „Prezentace M. Duffa jen potvrdila výjimečnou
  vzdálenost, kterou je třeba urazit k dosažení minimálního pokroku.“Neúspěch byl o to víc
  zkrušující, když se uvážil obrovský pokrok, kterého se v předchozích letech dosáhlo v
  relativistické astrofyzice, fyzice černých děr a kosmologii, už vůbec nemluvě o úspěchu
  standardního modelu elementárních částic. 
  
  Oxfordské symposium se mohlo zdát přiznáním porážky, nebýt překvapivé přednášky cambridgeského
  fyzika Stephena Hawkinga o černých dírách a kvantové fyzice. V ní Hawking ukázal, že přece jen
  existuje nadějné místo, kde se kvantová teorie a obecná relativita zajímavým způsobem spojují.
  Černé díry totiž podle Hawkinga nebyly naprosto černé, nýbrž zářily jemným světlem a Hawking
  tvrdil, že to umí dokázat. Tento zvláštní výsledek značně změnil přístup ke kvantové gravitaci v
  následujících čtyřech desetiletích. 
  
  Od raných sedmdesátých let minulého století byl Stephen Hawking význačnou osobností na
  cambridgeské scéně, kde pracoval v oddělení pro aplikovanou matematiku a teoretickou fyziku. Ještě
  mu nebylo třicet, ale v obecné teorii relativity byl už velmi známým jménem. Vyšel ze stáje
  Dennise Sciamy a spolu s Rogerem Penrosem ukázal, že singularity existovaly ve vesmíru od počátku
  času. V sedmdesátých letech posunul svou pozornost od kosmologie k černým dírám a spolu s
  Brandonem Carterem a Wernerem Israelem definitivně ukázali, že černé díry nemají vlasy: během
  svého vývoje ztratí veškeré vzpomínky na to, jak se tvořily, a všechny černé díry se stejnou
  hmotností, rotací a nábojem vypadají naprosto stejně. Dokázal také zajímavé tvrzení o velikosti
  černých děr: splynou-li dvě černé díry dohromady, je plocha Schwarzschildova horizontu, neboli
  horizontu událostí, výsledné černé díry větší, než je součet ploch horizontů původních černých
  děr. Prakticky to znamená, že sečte-li se plocha horizontů černých děr před a po nějaké fyzikální
  události, výsledná plocha je vždy větší. 
  
  Těchto významných výsledků dosáhl Hawking přesto, že jeho tělo sužovala Lou Gehrigova nemoc. Během
  šedesátých let ještě procházel po ústavu s hůlkou a občas se opíral o zeď, jeho stav se ale pomalu
  zhoršoval, až už se nemohl pohybovat bez pomoci. Jak pomalu ztrácel schopnost používat základní
  nástroje teoretického fyzika, psaní a kreslení, naučil se promýšlet velmi složité postupy jen v
  hlavě. Jen díky tomu byl stále schopen obdivuhodně řešit komplikované úlohy z obecné relativity i
  kvantové teorie. 
  
  Dá se říci, že jeho velký objev záření černých děr vznikl z nespokojenosti s článkem mladého
  Wheelerova doktoranda Jacoba Bekensteina. Bekenstein chtěl smířit černé díry s druhou větou
  termodynamickou. Použil jeden z Hawkingových výsledků a přišel s tvrzením, které na první pohled
  vypadalo naprosto absurdně. Hawkingovi přišlo jako zcela spekulativní a naprosto chybné. 
  
  Abychom pochopili, o co se Bekenstein snažil, musíme udělat krátkou odbočku do klasické
  termodynamiky, tedy odvětví fyziky, které se zabývá teplem, prací a energií. Druhá věta, či lépe
  druhý princip termodynamiky říká, že entropie, což je míra neuspořádanosti systému, vždy roste.
  Všimněme si klasického příkladu jednoduchého termodynamického systému - nádoby s molekulami plynu.
  Jsou-li všechny molekuly stlačeny v klidu v jednom rohu nádoby, je entropie systému nízká - je tam
  totiž jen velmi malá neuspořádanost. Stacionární částice se také nesrážejí se stěnami nádoby, to
  znamená, že i teplota systému je nízká. A teď si představme, že se molekuly začnou pohybovat.
  Volně poletují nádobou, náhodně se rozdělují po celé nádobě a tím se systém přesouvá do stavu s
  vyšší entropií, protože molekuly jsou rozloženy neuspořádaněji. Při svém pohybu molekuly narážejí
  na stěny nádoby, předávají jim určitou energii a zvyšují tak jejich teplotu. Čím rychleji se
  molekuly pohybují, tím rychleji roste entropie, dokud nedosáhne maxima. Čím rychleji se pohybují,
  tím méně je pravděpodobné, že by se usadily v klidném, uspořádaném stavu nízké entropie. Ale nejen
  to, rychlejší molekuly také předávají více energie stěnám a tak dále zvyšují jejich teplotu. To
  ukazuje na dvě věci. Jednak na to, že entropie nádoby s plynem roste, jak předpovídá druhý
  termodynamický princip, a také na souvislost mezi entropií a teplotou. 
  
  Bekenstein chtěl vyřešit paradox co se stane, když do černé díry hodíte krabici s nějakým
  haraburdím. V krabici může být leccos - encyklopedie, vodíkový plyn, kus železa - ale abychom si
  to nekomplikovali, zůstaňme u naší nádoby s plynem. Nádoba zmizí v černé díře a přijde ke slovu
  věta, že černá díra nemá vlasy. Pak už není možné zjistit, co do černé díry spadlo - všechny
  informace o tom se ztratily. To ale znamená, že zmizela i informace o neuspořádanosti v nádobě,
  která přestavovala entropii tohoto systému a tím se zmenšila i entropie vesmíru. Zdá se tedy, že
  existence černých děr narušuje druhý termodynamický princip. 
  
  Bekenstein chtěl jeho platnost zachránit pomocí jednoho Hawkingova výsledku. Jestliže něco hodíte
  do černé díry, velikost plochy horizontu událostí se buď nezmění, nebo vzroste, ale nikdy se
  nezmenší. Bekenstein tedy dospěl k závěru, že má-li ve vesmíru platit druhý termodynamický
  princip, černé díry musí nést určitou entropii a její velikost je spojená s obsahem plochy
  představující horizont událostí. Vzrůst obsahu této plochy bude kompenzovat ztrátu míry
  neuspořádanosti, jež byla pohlcena černou dírou, takže entropie vesmíru nikdy nepoklesne. Jenže
  když Bekenstein dovedl své řešení paradoxu s entropií do konečného důsledku, dospěl k něčemu
  bizarnímu. Má-li totiž černá díra určitou entropii, musí mít i určitou teplotu, jak jsme to viděli
  na našem příkladu s nádobou s plynem. I Bekensteinovi se zdálo, že zachází příliš daleko, a proto
  ve svém článku napsal: „Zdůrazňujeme, že veličinu označenou jako T nemůžeme chápat jako teplotu
  černé díry. To by vedlo k celé řadě paradoxů a taková identifikace by nebyla užitečná.“
  
  I přes tuto Bekensteinovu opatrnost Hawkinga jeho předpoklad pobouřil. Podle zákonů termodynamiky
  se nedá entropie černé díry zvýšit jinak, než že díra nějaké teplo vyzáří. A to už se Hawkingovi
  zdálo příliš. Černé díry byly podle něho opravdu černé - mohlo do nich něco padat, ale zcela
  určitě z nich nic nemohlo vycházet. Vlastnost horizontu, že jeho plocha se nemůže zmenšovat, sice
  chování entropie připomíná, ale není to pravá entropie. Plocha horizontu je jen analogií entropie.
  Leccos ale naznačovalo, že Bekenstein má pravdu a Hawking se mýlí. Byl to především objev Rogera
  Penrose z roku 1969, podle kterého rotující černá díra, popsaná Kerrovým řešením, může vyzařovat
  energii. Představme si částici, která se pohybuje rychlostí blízkou rychlosti světla a obíhá
  Kerrovu černou díru. Dále si představme, že se částice rozpadne na částice dvě. Jedna z nich je
  „pozřena“horizontem, zatímco druhá se může urychlit a odletět pryč s větší energií, než se kterou
  se k černé díře přiblížila. Celková energie systému i celého vesmíru zůstane při tom zachována.
  Tímto bizarním procesem, kterému se říká Penroseova superradiace, černé díry efektivně vyzařují
  energii a dá se říci, že zvláštním způsobem září. 
  
  Ale objevovaly se i další podobné myšlenky. V roce 1973 navštívil Stephen Hawking Jakova Zeldoviče
  a jeho mladého kolegu Alexeje Starobinského a ti mu sdělili, že podle jejich výpočtů bude Kerrova
  černá díra vyzařovat energii pocházející z kvantového vakua v jejím okolí. Černá díra tak bude
  fakticky zářit. 
  
  Hawking pak začal uvažovat na základě kvantové fyziky o částicích v okolí horizontu událostí
  představujícího povrch černé díry, kde se mohou odehrávat divné věci, a zjistil něco opravdu
  podivného. Podle kvantové teorie se z vakua neustále rodí páry částice-antičástice. Za normálních
  okolností páry vzápětí zase mizí, částice s antičásticemi anihilují. Jenže Hawking poukázal na to,
  že situace v blízkosti horizontu událostí je značně jiná. Tam mohou být některé antičástice
  vtaženy pod horizont černé díry, zatímco jejich partnerské částice zůstanou nad ním. Tento proces
  probíhá neustále a výsledkem je, že černá díra chrlí do svého okolí proud částic nesoucích určitou
  energii. Hawking detailně rozebral, co se děje v případě, že částice mají nulovou klidovou
  hmotnost jako fotony. Došel k tomu, že černá díra bude z hlediska vzdáleného pozorovatele zářit -
  bude-li to černá díra, která vznikla kolapsem hvězdy, její svit bude neobyčejně slabý, bude
  vypadat jako velice matná hvězda. A podobně jako jiným hvězdám, například Slunci, jí půjde připsat
  teplotu, která určuje spektrum vysílaného záření. Světlu vyzařovanému Sluncem odpovídá povrchová
  teplota asi 6000 kelvinů. Podle obecné teorie relativity jsou černé díry objekty, které pohlcují
  vše, co na ně dopadne, a nic nevyzařují. Hawking však ukázal, že vezmou-li se v úvahu kvantové
  efekty, černá díra přece jen není úplně černá, vyzařuje určité světlo a lze jí připsat i teplotu.
  
  Byl to přesný matematický výsledek s dalekosáhlými důsledky. Hawking dokázal, že tato teplota je
  úměrná převrácené hodnotě hmotnosti černé díry. Například černá díra o hmotnosti Slunce by měla
  teplotu pouhou jednu miliardtinu kelvinu, zatímco černá díra o hmotnosti Měsíce by měla teplotu 6
  kelvinů. Když černá díra září, ztrácí určitou hmotnost, je to však neobyčejně pomalý proces. Černé
  díře o hmotnosti našeho Slunce by trvalo nesmírně dlouho, než by se zbavila veškeré své hmoty,
  čili „vypařila se“, jak to nazval Hawking. Avšak černá díra se vypařuje tím rychleji, čím je její
  hmotnost menší, a celý proces se tedy zrychluje. Například černá díra o hmotnosti bilionu
  kilogramů, což je z astrofyzikálního hlediska malá černá díra, by se vypařila během celého věku
  života vesmíru, přičemž v poslední desetině sekundy jejího života by se vypařování jevilo jako
  mocný výtrysk energie. Jak napsal Hawking v článku „Exploze černých děr?“v časopise Nature, „podle
  astronomických standardů by to byla vcelku malá exploze, přesto ale větší, než exploze milionu
  megatunových vodíkových bomb“. 
  
  Výsledky o záření černých děr přednesl Stephen Hawking na oxfordském symposiu už na kolečkovém
  křesle. Argumentoval jasně a když rekapituloval své výsledky, bylo vidět, že mluví o něčem
  průlomovém. Po jeho přednášce následovalo téměř úplné ticho. Jak vzpomíná student Dennise Sciamy
  Philip Candelas, „všichni Hawkinga sledovali s velikým respektem, nikdo však pořádně nechápal, o
  čem mluví“. A sám Hawking později uváděl, že „se setkal s velikou nedůvěrou. … předsedající sekce…
  tvrdil, že je to nesmysl“. V Nature vyšlo shrnutí symposia, v němž autor uváděl, že „největším
  tahákem konference byl nezdolný S. Hawking“, nicméně bylo patrno, že k predikci explodujících
  černých děr je skeptický. Psal totiž, že „jakkoli vzrušující tato možnost je, nelze si představit
  pravděpodobný fyzikální mechanismus, jenž by k takovému dramatickému výsledku vedl“. 
  
  Na Hawkingův objev si vědecká obec musela zvyknout a to chtělo určitý čas. Někteří fyzikové si
  však uvědomili důležitost jeho objevu okamžitě. Dennis Sciama označil Hawkingův článek za „jeden z
  nejkrásnějších článků v dějinách fyziky“ a několik svých studentů vybídl k tomu, aby dosažené
  výsledky rozvíjeli dále. John Wheeler mluvil o Hawkingově výsledku jako o „bonbonu, který se
  rozplývá na jazyku“. A Bryce DeWitt Hawkingovy výsledky znovu odvodil jiným způsobem a napsal
  přehled výsledků o záření černých děr, který přesvědčil další skupinu vědců o reálnosti jevu. 
  
  Je však třeba zdůraznit, že Hawkingovy práce o záření černých děr se netýkaly kvantové gravitace.
  V nich se sice uvažovalo o kvantových efektech chování částic v gravitačním poli, ale gravitační
  pole samo se nekvantovalo. Neuvažovalo se o žádných pravidlech chování gravitonů ani o procesech,
  kterých by se gravitony účastnily, tedy o problematice, ve které neuspěli DeWitt a další. Úspěšně
  se v nich však propojovala gravitace s kvantovou teorií ostatních polí a získávaly se matematicky
  přesně podložené zajímavé výsledky, které by budoucí kvantová teorie gravitace, podaří-li se ji
  vybudovat, mohla vysvětlit z hlubšího hlediska. A tak záření černých děr bylo na několik příštích
  let zdrojem naděje pro úplnou kvantovou gravitaci. Sám Hawking pilně pracoval nejen na kvantové
  teorii objektů v prostoročase, nýbrž i na kvantování prostoročasu samotného. Se svými studenty
  intenzivně studoval kvantovou gravitaci dalších čtyřicet let. Jeho dílo v té době už docházelo
  všeobecného hlubokého uznání, deset let po odchodu Paula Diraca ze stolce lucasovského profesora
  byl na toto místo ustanoven právě on. 
  
  Když se jeden mladý student zeptal Johna Wheelera, jak se nejlépe připravit na práci v kvantové
  gravitaci, má-li se stát expertem spíše na obecnou relativitu, nebo na kvantovou fyziku, ten mu
  odpověděl, že by se měl začít zabývat něčím úplně jiným. Byla to moudrá rada. Tvrdohlavá nekonečna
  ze všech sil bránila kvantování obecné teorie relativity a zdálo se, že všechny pokusy o hledání
  kvantové gravitace jsou odsouzeny k nezdaru. 
  
  Bylo ale také jisté, že tam, kde se obecná relativita a kvantová teorie setkávají, dějí se
  neočekávané věci. Černé díry mají entropii a emitují teplo - to odporovalo původním představám
  relativistů, že černé díry jsou dokonale černé a nic z nich nemůže vycházet. Ale Bekensteinovy a
  Hawkingovy výpočty vrhaly nové světlo i na teorii kvant, obecná teorie relativity tam přispívala
  zvláštními efekty. Pro běžné fyzikální systémy, jako je nádoba naplněná plynem, je jejich entropie
  nějak spojena s jejich objemem. Čím větší objem je k dispozici, tím více způsoby lze částice plynu
  rozmístit a dát vzniknout neuspořádanosti, jejíž mírou je entropie. Přímý vztah mezi objemem a
  entropií se rozebírá v každé základní učebnici termodynamiky. Bekenstein a Hawking však zjistili,
  že entropie černé díry je vztažena k obsahu jejího povrchu, ne k objemu černé díry. Je to, jako by
  naše nádoba s plynem nějak ukládala svou entropii ve stěnách nádoby, ne v náhodném pohybu molekul
  plynu. Jak se entropie ukládá na povrchu černé díry, o které víme, že je jednoduchá a bez vlasů a
  jen stejnoměrně vyzařuje světlo prostřednictvím Hawkingova záření? 
  
  Vzpurná a tajemná kvantová gravitace, která by se měla vyrovnat i se všemi ohromujícími novými
  výsledky týkajícími se černých děr, zůstává výzvou pro mladé bystré fyziky. Kvantová gravitace
  bude intelektuálním bitevním polem možná na několik příštích desetiletí, stejně náročná válečná
  operace však probíhala i na jiném poli obecné relativity. Zde se nepracovalo jen s matematikou a
  myšlenkovými experimenty, ale i s velmi složitými přístroji. Cílem bylo zachytit slabounké
  gravitační vlny, které se šíří tkaninou prostoročasu od navzájem se srážejících černých děr. 

\section{Vidět gravitaci}\label{feyIchIIIsecXI} 
  Josephu Weberovi se před lety dostalo pocty, že byl prohlášen za prvního, kdo pozoroval gravitační
  vlny. Byl to on, kdo téměř osamocený začal s experimenty, jejichž cílem bylo pozorovat gravitační
  vlny, a na konci šedesátých a počátkem sedmdesátých let minulého století byly jeho výsledky
  oslavovány jako jeden z největších úspěchů v obecné relativitě. Jeho hvězda však časem pomalu, ale
  jistě zapadla. V roce 1991 si jednomu novináři postěžoval: „Jsme číslo jedna v oboru, ale od roku
  1987 jsme nedostali žádnou subvenci.“

  Weberův osud se může zdát křiklavě nespravedlivý. Když byl na vrcholu své kariéry, o jeho
  výsledcích se diskutovalo na všech velkých konferencích o obecné relativitě spolu s neutronovými
  hvězdami, kvasary, horkým velkým třeskem a zářením černých děr. Byly předmětem mnoha článků, které
  se snažily o jejich vysvětlení. Weber měl nakročeno k Nobelově ceně. A potom, stejně rychle, jako
  vystoupal ke slávě, se ocitl v pozadí akademické vědy. Opuštěn kolegy a odmítán vědeckými nadacemi
  byl odsouzen k pomalé vědecké smrti - stal se nešťastnou a nepříjemnou poznámkou pod čarou v
  dějinách obecné teorie relativity. Někteří fyzici by dokonce možná řekli, že teprve po jeho pádu
  nastal skutečný útok na problém gravitačních vln. 

  Gravitační vlny jsou pro gravitaci tím, čím jsou elektromagnetické vlny pro elektřinu a
  magnetismus. Když James Clerk Maxwell sjednotil elektřinu a magnetismus, ukázal také, že podle
  jeho teorie existují elektromagnetické vlny, které pak experimentálně předvedl Heinrich Hertz.
  Tyto vlny mohou kmitat s nejrůznějšími frekvencemi a podle této frekvence mají různé vlastnosti.
  Při určitých frekvencích jsou to vlny viditelného světla, které mohou vnímat naše oči. Při nižších
  frekvencích jsou to vlny rádiové, jejichž zásluhou hrají naše rádiové přijímače a které bezdrátově
  přenášejí informace do našich laptopů i signály od kvasarů v těch nejvzdálenějších oblastech
  vesmíru. 

  Jen několik měsíců po dokončení obecné teorie relativity Einstein ukázal, že jeho teorie
  předpovídá existenci vln v řadě ohledů podobných vlnám elektromagnetickým. Tyto vlny byly vráskami
  na samotném prostoročasu. Prostoročas můžeme přirovnat k rybníku - vhodíte-li do něho oblázek,
  šíří se od místa dopadu vlny ve všech směrech. Vrásky na prostoročase, kterým říkáme vlny
  gravitační, přenášejí z místa na místo určitou energii, v tom jsou podobné jak vlnám na vodě, tak
  vlnám elektromagnetickým. 

  Jenže na rozdíl od elektromagnetických vln je gravitační vlny neobyčejně obtížné najít - účinnost,
  se kterou odnášejí energii z nějakého gravitací vázaného systému, je velice malá. Když Země obíhá
  kolem Slunce ve vzdálenosti 150 milionů kilometrů, budí gravitační vlny, které odnášejí ze
  sluneční soustavy určitou energii, a v důsledku toho se poloměr dráhy Země zmenšuje. Jenže změna
  tohoto poloměru je nepatrná, za den činí asi průměr protonu. To znamená, že za celou dobu své
  existence se Země přiblížila ke Slunci jen o milimetr. A i když je ve vesmíru nějaký silný zdroj,
  který produkuje velké množství gravitačních vln, tyto vlny se projevují po dlouhé cestě prostorem
  jen jako velmi jemný šepot. Prostoročas je lepší si představovat ne jako rybník, nýbrž jako blok z
  oceli, který se jen nepatrně zachvěje, když se do něj kopne. 

  Mnoho fyziků odmítalo existenci gravitačních vln přijmout. Einstein argumentoval skoro půl
  století, že tyto vlny existují, mnoho fyziků však odmítalo uznat jejich realitu a dívalo se na ně
  jako na další z matematických podivností, ke kterým obecná teorie relativity vede a jež vyžadují
  ještě hlubší pochopení této Jedním z těch, kdo odmítli skutečnou existenci gravitačních vln, byl
  Arthur Eddington. Když Einsteinovy výpočty zopakoval, nabyl přesvědčení, že takzvané gravitační
  vlny jsou jen matematickým artefaktem vzniklým zvoleným popisem prostoročasu, a že je proto možné
  se jich zbavit lepší volbou vztažného systému. Na rozdíl od elektromagnetických vln, jež se
  pohybují rychlostí světla, se vlny, jejichž reálnost Eddington zavrhl, mohly pohybovat „rychlostí
  myšlenky“, tedy rychlostí libovolně větší než rychlost světla. Einstein uznal svou chybu v
  původním výpočtu a zamítnutí reálnosti gravitačních vln dokonce podpořil článkem v časopise
  Physical Review, který napsal v roce 1936 spolu se svým mladým asistentem Nathanem Rosenem. Tento
  článek pracoval s přesným řešením Einsteinových rovnic a autoři v něm reálnost gravitačních vln
  odmítali. 

  S nejpřesvědčivějšími argumenty ve prospěch gravitačních vln přišel na konferenci na Chapel Hill v
  roce 1957 Hermann Bondi, v té době vedoucí relativistické skupiny na King’s College v Londýně. Ten
  popsal velmi jednoduchý myšlenkový pokus. Vezměme tyč a navlékněme na ni dva kroužky tak, aby byly
  od sebe v malé vzdálenosti. Poloměr kroužků nechť je takový, že sice mohou stále po tyči klouzat,
  ale mezi nimi a povrchem tyče působí určité tření. Když přes takové zařízení přejde gravitační
  vlna, dostatečně tuhou tyč to téměř neovlivní, ale kroužky vlna posune od sebe a pak zase zpátky -
  připomíná to pohyb bóje na mořské hladině pod účinkem vln. V důsledku tření mezi kroužky a tyčí se
  uvolní určitá energie a zdrojem této energie nemůže být nic jiného než gravitační vlna. Z toho
  učinil závěr, že vlna musí nést určitou energii. Richard Feynman, který byl mezi účastníky
  konference, přišel s podobným argumentem a většinu posluchačů to o reálnosti gravitačních vln
  přesvědčilo. Obecná teorie relativity tedy předvídala existenci fyzikálně reálných gravitačních
  vln; zbývalo je experimentálně objevit. Diskuse na Chapel Hill okouzlila Joe Webera. Ať si Bondi,
  Feynman a ostatní sedí kolem stolu a diskutují o gravitačních vlnách - on je opravdu půjde hledat.

  Weber skutečně patřil k těm, kdo se pokoušejí o nemožné. Od mládí byl nadšený kutil, jako
  středoškolák si vydělával opravováním rádií. Byl vizionářem, který se neustále snažil posunout
  techniku za hranice toho, co se pokládalo za možné, a jednoduchými prostředky se snažil stavět
  aparatury, jež měly provádět měření na samé hranici fyzikálních možností. Jeho elán se promítal do
  všech oblastí jeho života - každé ráno uběhl tři míle a pak pracoval celý den, což praktikoval i
  dlouho po své sedmdesátce. Weber vystudoval Námořní akademii Spojených států, ze které vyšel jako
  elektroinženýr. Během druhé světové války působil jako lodní kapitán, pak byl ale vzhledem ke svým
  znalostem v oblasti elektroniky a rádiového vysílání postaven do čela námořního programu
  elektronických obranných zařízení. Po skončení války se stal profesorem elektroinženýrství na
  Marylandské univerzitě, zde však změnil oblast svého zájmu a získal PhD ve fyzice. 

  V polovině padesátých let se začal zajímat o gravitaci. V tomto směru ho povzbuzoval John Wheeler
  a vyslal ho na rok do Evropy, aby se seznámil s nejnovějšími pokroky v obecné teorii relativity.
  Po svém návratu byl Weber připraven na stavbu přístroje na registraci gravitačních vln. Zvažoval
  různé možnosti, jak tuto aparaturu navrhnout, a pořídil řadu nákresů. Z různých nápadů ho nejvíce
  zaujal jeden, jehož myšlenka byla jednoduchá - vyrobit velké těžké válce z hliníku a zavěsit je na
  pevných vláknech na strop. Na plášti každého z válců byly dokola upevněny neobyčejně citlivé
  detektory, které vysílaly elektrické signály, jestliže válec vibroval. Vibrace mohlo vyvolat
  leccos - zazvonění telefonu, kolemjedoucí auto, bouchnutí dveřmi. Proto Weber válce pečlivě
  odizoloval od všech možných podnětů, jež by mohly jejich vibrace vyvolat. Když je pak konečně
  uvedl do provozu - ustálilo se pro ně jméno Weberovy válce - začal sbírat veškeré signály z
  detektorů. Jakmile se podařilo vyloučit všechny ostatní vnější podněty, válce registrovaly jen
  nepatrné záchvěvy, jejichž zdrojem by mohly být právě gravitační vlny. Na registrovaných
  záchvěvech však bylo něco podezřelého. Kdyby totiž šlo opravdu o účinek gravitačních vln, musely
  by být zdrojem tak mohutné exploze v blízkém vesmíru, že by měly být pozorovatelné teleskopy.
  Signály byly příliš silné, než aby mohly být přičítány gravitačním vlnám, a tak Weber musel své
  zařízení upravit. 

  Aby si byl jist, že určitý záchvěv byl vyvolán opravdu gravitační vlnou, umístil jeden ze svých
  čtyř válců do Národní argonnské laboratoře u Chicaga, téměř tisíc kilometrů od své laboratoře na
  Marylandské univerzitě. Jestliže by se válce na obou místech zachvěly současně, silně by to
  nasvědčovalo tomu, že vibrace byly vyvolány gravitační vlnou přicházející z veliké dálky, a ne
  například přirozenými tepelnými kmity válců - ty by nebyly korelované. Weber tedy porovnával
  záznamy na obou od sebe vzdálených zařízeních. Jestliže se nějaká nadprůměrná výchylka objevila na
  obou detektorech zároveň, bylo pravděpodobné, že byla vyvolána vnějším podnětem, který zasáhl obě
  zařízení současně, a že nejde o shodu náhodných kmitů. Tak by tomu bylo v případě gravitační vlny
  přicházející odněkud z vesmíru. Weber tedy hledal právě tyto „koincidence“, jak je nazýval. Zapnul
  tedy svá zařízení a čekal. 

  V roce 1969, deset let poté, co s experimenty začal, měl konečně výsledek, který mohl ukázat
  světu. Nalezl řadu koincidencí mezi výchylkami nejen na válci u Chicaga a v Marylandu, ale na
  všech čtyřech válcích. To bylo příliš mnoho shod, než aby byly náhodné. Kmity všech válců v
  unisonu musel způsobit vnější podnět. V době experimentu neproběhlo žádné zemětřesení ani
  významnější elektromagnetická bouře, nic, čím by se měřený efekt dal vysvětlit. Zdálo se, že Weber
  gravitační vlny opravdu objevil. 

  V následujících letech Weber své zařízení zdokonaloval, aby předešel pochybám, že na záznamu vidí
  to, co vidět chce. Záchvěvů válců bylo málo a byly ukryty v přirozeném šumu zařízení. Válce totiž
  musely vykazovat i náhodné tepelné kmity, při kterých molekuly a atomy materiálu, z něhož byly
  vyrobeny, náhodně poskakují sem a tam. Tyto kmity nejsou stejně velké a ty, které převyšují
  průměr, by mohly být zaměněny za hledaný signál, zvláště při sledování prostým okem. Nechal tedy
  signály registrovat automaticky pomocí speciálního počítačového programu. Použil i vtipný trik:
  porovnal mezi sebou záznamy, které byly proti sobě uměle časově posunuté. Tepelné kmity jsou totiž
  opravdu náhodné, takže tento posun by neměl ovlivnit počet koincidencí. Byly-li ale koincidence
  reakcí na vnější signál od gravitační vlny, posunutí by jejich počet zmenšilo. A podle Webera k
  tomu opravdu docházelo - to se zdálo být dobrým důkazem, že koincidence jsou vyvolány gravitační
  vlnou. 

  V roce 1970 už experiment běžel dostatečně dlouho, aby si Weber mohl troufnout dokonce na určení
  směru, ze kterého signál přicházel. Vycházelo mu, že zdroj údajných gravitačních vln leží v centru
  naší galaxie, a to vypadalo přijatelně. V jednom článku napsal: „Dobrým rysem je, že zde leží 10
  miliard slunečních hmot a je rozumné, že zdroj se nachází tam, kde je soustředěna největší část
  hmoty galaxie.“

  Jak rostlo Weberovo přesvědčení, že opravdu registruje gravitační vlny, rostl i zájem zbytku
  světa. Všichni fyzici byli jeho objevem udiveni, protože takové přímočaré odhalení gravitačních
  vln se nečekalo, nebyl ale důvod o výsledcích a priori pochybovat. Relativisté se jen snažili
  pochopit, co přesně znamenají. Například Roger Penrose spočítal, co by bylo výsledkem kolize dvou
  gravitačních vln - mohl-li by být výsledek tak bouřlivý, aby vysvětlil Weberův experiment. Stephen
  Hawking vypracoval myšlenkový experiment, při kterém se srážely dvě černé díry. Doufal, že kolizi
  bude doprovázet výtrysk gravitačního záření, jež by mohlo být při Weberových experimentech
  pozorováno. A v těchto prvních letech po oznámení objevu Weberova sláva rostla. Byl požádán o
  interview pro časopis Time a jeho práce byla popularizována v New York Times a v mnoha dalších
  tiskovinách ve Spojených státech i Evropě. A nové výsledky stále přicházely.

  Weberův objev byl ohromující a měření byla příliš skvělá, než aby nebudila pochybnosti. Zdálo se,
  že Weber objevil skutečně mohutný zdroj gravitačních vln, mnohem silnější, než se předpokládalo,
  že může existovat. Weberovy válce i detektory k nim připojené byly sice velmi sofistikované a
  citlivé, ne ale tak citlivé, aby mohly registrovat vlny od teoreticky předpokládatelných zdrojů. K
  rozkmitání Weberových válců by bylo zapotřebí neuvěřitelně silných gravitačních vln. A to byl
  problém. I když vlny měly podle Webera přicházet z centra naší galaxie, kde bylo dost materiálu,
  jenž se mohl hroutit do černých děr či kolidovat a rozkmitávat tak okolní prostoročas, toto
  centrum bylo vzdálené od Země 20 000 světelných let. Vlny odtud vyslané by během své dlouhé cesty
  velice zeslábly a jejich intenzita by po dosažení Země byla nepatrná. Weber uváděl, že energie
  pozorovaných vln je ekvivalentní hmotnosti tisíce hvězd velikosti Slunce, zničených v centru
  galaxie během jednoho roku - to byl opravdu kolosální proces. 

  Martin Rees z Cambridge byl k Weberovým výsledkům nedůvěřivý od samého počátku. Spolu se svým
  bývalým školitelem Dennisem Sciamou a Georgem Fieldem z Harvardu odhadli, kolik energie se ve
  formě gravitačních vln z centra galaxie může skutečně uvolňovat. Rees se svými spolupracovníky
  zjistil, že to nemůže být více než ekvivalent asi dvou set zaniklých hvězd ročně, jinak by se
  totiž galaxie nadouvala a to by se projevilo na pohybu blízkých hvězd, jež ze Země pozorujeme.
  Výpočet byl přibližný, proto byli se svými závěry opatrní. Ve svém článku říkali: „Obrovské ztráty
  hmotnosti, jež naznačují Weberovy experimenty, nejsou těmito astronomickými úvahami vyloučeny,
  bylo by však velmi žádoucí, aby tato měření byla zopakována dalšími pracovníky.“Webera to
  neznepokojilo, protože argument předkládaný Reesem, Fieldem a Sciamou byl teoretický. Možná, že
  byla chybná teorie, ale nepochyboval, že jeho experimentální výsledky jsou správné.

  Vynořili se Weberovi následovníci. V Moskvě, Glasgow, Mnichově, v Bellových laboratořích, ve
  Stanfordu i Tokiu se budovaly nové detektory. Některé byly přesnou kopií Weberovy aparatury a
  všechny vycházely z jeho původních návrhů. Postupně byly uváděny do chodu a začaly se objevovat
  první výsledky. Jenže až na pár koincidencí nalezených detektorem v Mnichově žádný z přístrojů
  nevykazoval tolik koincidencí, kolik údajně nacházel svou aparaturou Joe Weber. Prostě tam nebyly.
  Ani to Webera neznejistilo. On na svých přístrojích pracoval deset let a byl přesvědčen, že
  ostatní experimenty jsou méně citlivé než ty jeho, a proto není divu, že nenacházejí žádný signál.
  Chtějí-li fyzici kritizovat jeho výsledky, musí postavit přesně stejný detektor jako on, jeho
  přesnou kopii. Některé skupiny, mezi nimi ta v Glasgow a v Bellových laboratořích v Holmdelu, sice
  tvrdily, že jejich přístroje jsou přesnými kopiemi, ale Weber odpovídal, že nejsou dokonalými
  kopiemi. 

  Jenže na Weberových experimentech bylo leccos podezřelého. Určit citlivost přístrojů byla ošidná
  věc. Ale více znepokojivé bylo, že Weber dělal určité chyby, a přesto koincidence nacházel.
  Původně tvrdil, že signály přicházejí z centra galaxie, protože koincidence dosahovaly maxima
  jednou za čtyřiadvacet hodin, kdy byly válce ve správné poloze, tj. kolmo na předpokládaný směr
  vlny. Jenže Weber přehlédl jednu skutečnost - gravitační vlně by v její pouti vůbec nevadila
  přítomnost Země a tak by perioda koincidencí měla být jen dvanáct hodin - docházelo by k nim
  jednou, když by byly válce přivrácené k centru galaxie a podruhé, když je rotace Země odnese na
  stranu opačnou. Když si Weber uvědomil tuto chybu v úvaze, analyzoval svoje měření znovu a
  zjistil, že mezi koincidencemi byla skutečně jen dvanáctihodinová perioda. Našel totiž
  koincidence, kterých si původně nevšiml. Zdálo se, že ve svých naměřených údajích nalezne skutečně
  cokoli, co potřebuje. Relativista Bernard Schutz vzpomíná, že „kolegové byli tehdy velmi
  podezřívaví. Weber svá měření detailně nezveřejnil, abychom se na ně mohli všichni podívat, a
  zdálo se, že on v nich najde, cokoli chce.“

  Ještě závažnější problém se objevil, když Weber spojil své síly s experimentální skupinou na
  univerzitě v Rochesteru. Když porovnával záchvěvy marylandských válců s těmi z Rochesteru, našel
  řadu koincidencí, tedy záchvěvů, které proběhly současně na obou místech. To měl být jasný důkaz,
  že je vyvolaly gravitační vlny. Jenže pak se ukázalo, že Weber neporozuměl, jak v Rochesteru
  označovali čas jednotlivých událostí. Události, které zdánlivě koincidovaly, se ve skutečnosti
  odehrály se čtyřhodinovým odstupem. Jakmile byl tento omyl napraven, Weber data probral znovu a
  opět našel významné koincidence. 

  Weberův objev se najednou zdál být důsledkem chyb a nesprávných závěrů. On byl schopen najít
  koincidence kdekoli a koincidence podle něj znamenaly gravitační vlny. Weberova nesporná schopnost
  obcházet chyby měla zničující účinek na jeho reputaci. Už vůbec mu nepomohlo, že nikdo nebyl
  schopen jeho výsledky reprodukovat. Respektovaný experimentální fyzik Richard Garwin napsal do
  Physics Today článek „Detekce gravitačních vln zpochybněna“, ve kterém systematicky kritizoval
  Weberovu analýzu měření i celý experiment a konstatoval, že Weberovy koincidence „nejen nejsou,
  ale ani nemohou být důsledkem gravitačních vln“. Většina relativistů se k Weberovi otočila zády.
  Jeho články, svého času velmi ceněné, přestaly být sledovány. Stále více kolegů odmítalo jeho
  experimenty podporovat a dotace pro jeho práci vyschly. Koncem sedmdesátých let minulého století
  Weber vypadl z vedoucího postavení ve vědecké obci. 

  Weberovy experimenty sice byly diskreditovány, jenže uvedly do pohybu něco mnohem významnějšího.
  Ze zmatků kolem nich vzniklo nové pole bádání. Astronomové si uvědomili, že gravitační vlny by
  otevřely nové významné okno do vesmíru. Pozorování gravitačních vln by doplnila pozorování pomocí
  světelných a rádiových vln i rentgenového záření. Dovolovala by vidět pomocí gravitačních vln do
  těch nejvzdálenějších oblastí vesmíru, jejichž zkoumání neumožňují ani ty největší teleskopy. K
  optické, rádiové a rentgenové astronomii by se připojila astronomie gravitačně vlnová. 

  V roce 1974 dva američtí astrofyzici Joe Taylor a Russell Hulse objevili dvojici neutronových
  hvězd obíhajících kolem sebe. Jedna z nich byla pulzar, který vysílal k Zemi záblesky světla s
  periodou několika tisícin sekundy, a proto se jeho pohyb kolem druhé hvězdy dal pohodlně sledovat.
  Taylor a Hulse dokázali měřit polohy obou hvězd s neuvěřitelnou přesností, a tak objevili novou
  perfektní laboratoř pro ověřování obecné teorie relativity. Podle Einsteina měly dva objekty na
  oběžné dráze kolem sebe vyzařovat energii do okolního prostoročasu a v důsledku toho se měly
  zmenšovat poloměry jejich oběžných drah. Obě tělesa se nakonec na sebe zhroutí. Einstein sice
  později tento názor opustil, ale výpočty zde už byly a čekaly na ověření. A Hulseův a Taylorův
  pulzar se pro tento účel skvěle hodil. 

  V roce 1978 na devátém Texaském symposiu, které se ten rok konalo v Mnichově, Joe Taylor oznámil
  výsledek. Sledoval milisekundový pulzar po čtyři roky a mohl bezpečně prohlásit, že jeho dráha se
  smrštila přesně podle Einsteinovy předpovědi. Navzájem se obíhající neutronové hvězdy ztrácejí
  energii gravitačním zářením. Tento důkaz existence gravitačního záření byl sice nepřímý, ale jiné
  vysvětlení neexistovalo. Množství vyzářené energie přesně souhlasilo s předpovědí a měření byla
  čistá a jednoznačná. Gravitační vlny skutečně existují. 

  Z popela Weberových měření povstalo nové odvětví experimentální vědy. Různé skupiny po celém světě
  začaly stavět vlastní detektory. Někteří experimentátoři imitovali Weberovy detektory, ale
  dramaticky je chladili, aby co nejvíce potlačili tepelné kmity. Jiní měnili tvar přijímačů,
  například jim dávali kulový tvar, aby reagovaly na vlny přicházející ze všech směrů. Ale signál,
  který hledali, byl tak slabý, že k jeho zachycení byl třeba receptor s naprosto fantastickou
  citlivostí. Zachytit vrásky prostoročasu je opravdu nesmírně obtížný úkol. Objevil se jeden
  přístup, který byl mnohem účinnější, ale také mnohem nákladnější - použití laserových
  interferometrů. 

  Laserový interferometr využívá těch nejlepších nástrojů moderní fyziky. Předně užívá laserového
  paprsku, neuvěřitelně tenkého kužele světla - lze jím osvětlit cíl o velikosti hrotu tužky na míle
  daleko. Byl to právě Joe Weber, kdo jako jeden z prvních přišel s konceptem laseru ještě dříve,
  než se začal zabývat gravitací. Udělal to ve stejné době jako Charles Townes z Columbijské
  univerzity, ale nikdy se mu nedostalo plného uznání, a tak nebyl jedním z těch, kteří byli za
  objev laseru v roce 1964 vyznamenáni Nobelovou cenou. 

  Laserová interferometrie však využívá i té vlastnosti světla, že světelné vlny se chovají podobně
  jako vlny na oceánu - mohou spolu interferovat. Znamená to, že setkají-li se dva vrcholy vln,
  dojde k zesílení výsledné vlny, která bude mít vrchol daleko vyšší - říkáme, že dochází ke
  konstruktivní interferenci. Setká-li se naopak vrchol a údolí dvou vln, obě vlny se navzájem
  vyruší - jedná se o destruktivní interferenci. A samozřejmě existuje mnoho možností mezi oběma
  extrémy. 

  Těchto dvou vlastností světla z laseru se dá užít ke sledování nepatrného pohybu objektů
  způsobeného gravitační vlnou. Návod je zhruba následující: Zavěste dvě drobné hmotné kuličky v
  určité vzdálenosti od sebe a namiřte na ně laserové paprsky. Ty se od nich odrazí a budou spolu
  interferovat. Výsledný interferenční obraz bude záviset na vlnové délce užitého světla a na přesné
  vzdálenosti, kterou paprsky urazí. Když se jedna z kuliček pohne, interferenční obrázek se změní.
  Sledováním změn interferenčního obrazu je možné určit i velice nepatrný pohyb vyvolaný dopadající
  gravitační vlnou. Toto měření je mnohem citlivější, než bylo měření pomocí Weberových válců. 

  Laserová interferometrie představovala zcela nový vědecký postup, přinejmenším pro relativisty.
  Teorie relativity bývala především záležitostí tužky a papíru, experimentů bylo jen málo. Některé
  se uskutečňovaly na základě spolupráce mezi univerzitami a různými institucemi, ale metody práce
  byly neporovnatelné s částicovou a jadernou fyzikou s jejich obrovskými urychlovači a reaktory.
  Teď ale nastupovala nová kultura. Experimenty s gravitačními vlnami vyžadovaly desítky či dokonce
  stovky milionů dolarů. Skupiny pracující na těchto projektech už nečítaly několik pracovníků,
  musely je tvořit stovky vědců a techniků. 

  Teď se muselo experimentovat opravdu promyšleně, muselo být jasné, co se vlastně hledá. Bylo
  zřejmé, že gravitační vlny, jež by se měly zaznamenávat, musí být ze zdrojů, pro jejichž popis je
  obecná relativita nezbytná. Milisekundový pulzar Hulseho a Taylora vypadal ještě vcelku nenápadně
  - prostě jen velmi těsná dvojhvězda tvořená neutronovými hvězdami. Ale gravitační vlny z ní
  dokázaly odsávat energii natolik efektivně, že se to projevovalo na změně drah obou komponent.
  Neutronové hvězdy jsou objekty na samé hranici katastrofálního hroucení prostoročasu - právě
  studium podobných objektů pozvedlo Einsteinovu teorii relativity k její plné slávě.

  Možným zdrojem bohatého výtrysku gravitačních vln by mohl být výbuch supernovy. Supernovy jsou
  hroutící se hvězdy, které po několik sekund září jasněji než miliardy hvězd v naší galaxii
  dohromady. Než se z nich stane neutronová hvězda nebo černá díra, jsou tím nejjasnějším objektem
  na obloze. Jsou neobyčejně mocnými zdroji elektromagnetických vln a astrofyzici spekulovali o tom,
  zda uvolněná energie nezatřese okolním prostoročasem natolik, aby nedošlo i k mocnějšímu výtrysku
  gravitačních vln. V roce 1987 došlo k výbuchu supernovy v blízkém sousedství naší galaxie, ve
  Velkém Magellanově mračnu. To je vzdáleno asi 160 000 světelných let, takže supernova byla výborně
  pozorovatelná normálními teleskopy. Bohužel ve chvíli výbuchu nebyl v provozu žádný z detektorů
  gravitačních vln s výjimkou detektoru Joe Webera. Nikoho neudivilo, když tvrdil, že něco
  zaznamenal, a jak bylo už v té době zvykem, všichni to ignorovali. 

  Potíž se supernovami je v tom, že jsou nevypočitatelné, neumíme předpovědět, kdy a kde k výbuchu
  dojde. Mohutná exploze supernovy může sice vyslat výtrysk energie, jenže než gravitační vlna
  doputuje k pozemskému detektoru, velice se oslabí, a může být zaměněna za událost, jež byla jen
  výsledkem šumu. Potřebovali bychom čistý signál, který i kdyby byl slabý, měl by nezaměnitelný
  tvar a průběh a my bychom ho poznali podobně bezpečně, jako když hledáme známou tvář v zástupu
  lidí. 

  Některé takové procesy astrofyzici znali. Tvar signálu od neutronové dvojhvězdy Hulseho a Taylora
  se dal vypočítat s dostatečnou přesností, abychom věděli, co očekávat. Na rozdíl od zmatených
  signálů pocházejících z kosmických výbuchů by tato gravitační vlna byla pravidelná a periodická,
  jako zvuková vlna od sirény, a její perioda by se jen pomalu měnila s časem v korespondenci s tím,
  jak by se měnily dráhy a tím i perioda oběhů dvojhvězdy. Signál by byl zřetelný, jednoduše
  popsatelný a možná by šel i snadno zaznamenat. 

  Proč se ale zastavit zde? Proč nezaútočit na větší cíl? Neutronová hvězda obíhající kolem černé
  díry, která do ní nakonec spadne, by vysílala mnohem silnější signál, a samozřejmě dvojhvězda
  tvořená dvěma černými dírami by produkovala v prostoročase opravdu mohutné vlnky. Takové kolem
  sebe obíhající černé díry by vysílaly bručení gravitačních vln, jehož frekvence by rostla s tím,
  jak by se černé díry k sobě přibližovaly, a vysílání by zakončil mohutný výtrysk gravitačního
  záření v okamžiku, kdy by se obě černé díry spojily a vytvořily jednu. Po takovém signálu by měly
  detektory pátrat: nejdříve signál ze spirálovitého sestupu černých děr, potom charakteristické
  „cvrlikání“, než se díry spojí, a pak závěrečné zazvonění. Relativistické dvojhvězdy jsou klenoty
  skryté na nebeské klenbě a gravitační detektory by je měly objevit.

  Zdálo se to jednoduché - prostě hledat hroutící se dvojhvězdy tvořené neutronovými hvězdami nebo
  černými dírami - ale zásadní informace chyběla. Co detektor skutečně uvidí? Jak budou vypadat
  charakteristické části signálu v okamžiku, kdy dospějí k detektoru? Pozorovatelé, nové pokolení
  astronomů gravitačních vln, budou muset vědět přesně, nejen přibližně, jak signál vypadá. A aby
  tuto informaci měli, je třeba se vrátit k problému, který obecnou relativitu provází od jejího
  vzniku. Musí se řešit Einsteinovy rovnice, které určí, jak budou vysílané gravitační vlny vypadat.
  Desetiletí zkušeností však ukázala, jak se Einsteinovy rovnice umí bránit pokusům o zkrocení.
  Jedinou cestou je v tomto případě spolehnout se na velmi výkonný počítač a na něm závěrečné
  stadium spojení dvou černých děr simulovat. 

  Charles Misner, jeden ze studentů a spolupracovníků Johna Wheelera, varoval před záludností
  Einsteinových rovnic už na chapelhillské konferenci v roce 1957. „Snažíte-li se zkrotit pomocí
  počítače nelineární šelmy, které nám odkázal Albert Einstein, může to mít jen dvojí konec: buď se
  zastřelí programátor, nebo počítač vybuchne.“ To druhé se přesně stalo. Když se jeden z bývalých
  Wheelerových studentů Robert Lindquist pokoušel v roce 1964 řešit svůj model, program „explodoval
  “. Jak se černé díry dostávaly blíž a blíž k sobě, narůstaly chyby řešení a velice brzy začal
  počítač chrlit bezcenná data - dostal numerický „průjem“. Chyby se nedaly odhalit a Lindquist to
  vzdal. 

  V roce 1970 to byl Bryce DeWitt, kdo se pokoušel počítačově sledovat kolizi dvou černých děr. I
  když jeho vášní byla kvantová gravitace, počítačovým simulacím složitých rovnic se naučil v době,
  kdy v livermorské Lawrencově národní laboratoři v Kalifornii pracoval s Edwardem Tellerem na
  projektu vodíkové bomby. V Texasu zadal svému studentovi Larrymu Smarrovi úkol zjistit, kolik
  energie se uvolní při srážce dvou černých děr. Úlohu řešili na velikém počítači Texaské univerzity
  a byli schopni stanovit hrubý odhad, jak budou vysílané gravitační vlny vypadat. Pak ale začaly
  narůstat chyby a z počítače vycházel samý brak. Byl to záblesk pohledu na tvar výsledné vlny, ale
  byl příliš hrubý, než aby byl užitečný. Na konci výpočtu totiž zvedly své ošklivé hlavy
  prostoročasové singularity a výsledek zardousily. 

  Následující tři desetiletí se kolizi dvojhvězd snažily bez velkého úspěchu simulovat týmy
  programátorů. Dosáhly určitého pokroku, ale jak vzpomíná princetonský relativista Frans Pretorius,
  „primitivní věci nefungovaly, nikdo nevěděl proč a vědci jako by poletovali v temnotě. Co činilo
  problém tak náročným, byla počítačová nákladnost.“ V devadesátých letech se problém kolize černých
  děr pokládal ve Spojených státech za jednu z největších výzev pro počítačovou fyziku. Řada skupin
  badatelů po celé zemi dostávala miliony dolarů na nákup superpočítačů a jejich programování. Tu a
  tam se objevilo určité vylepšení a výsledky se o něco posunuly, než narůstající chyby zabránily
  pokračování výpočtu. Problematika počítačového řešení gravitačních rovnic se stala samostatným
  odvětvím výzkumu - hovoří se o numerické relativitě.

  Řešení rovnic pro kolizi černých děr se ukázalo jako neobyčejně obtížný problém, stejně obtížný
  jako experimentální nalezení gravitačních vln, a soustředila se na ně veliká pozornost. Věnovalo
  se mu mnoho mladých studentů, kteří obětovali značnou část své profesionální kariéry na to, aby
  dosáhli jen malého zlepšení výpočtu. Bylo to jako hrát složitou počítačovou hru, často osamoceně,
  bez nějaké okamžité odměny, bez rozlišení úrovní hry a bez slavných vítězství. 

  Pro řadu fyziků představa relativity splynula právě s numerickou relativitou. Skupina zabývající
  se obecnou teorií relativity by nebyla úplná bez alespoň jednoho pracovníka snažícího se numericky
  vyřešit problém kolize černých děr s důrazem na produkci gravitačních vln. Na řadě konferencí si
  numeričtí relativisté vzájemně sdělovali své triky a ukazovali grafy, jež zobrazovaly jejich
  výsledky. Ale rovnice se nevzdávaly. A tvary vln, které byly výsledkem jejich simulací, nedávaly
  velkou naději, že se je podaří najít pomocí detektorů. 

  Pretorius vzpomíná na názory, jež v těchto temných časech vládly: „Vážně se uvažovalo o tom, že
  problém nebude vyřešen do té doby, než detektory gravitačních vln přinesou nějaké výsledky.“Měření
  z detektorů by mohla poskytnout důležité pokyny o tom, jak by měl vypadat výsledek simulace. 

  Úsilí věnované numerické relativitě však mělo i širší důsledky. Na konci sedmdesátých a počátkem
  osmdesátých let minulého století vyvinul Larry Smarr komplikované numerické programy, s nimiž
  prováděl výpočty na těch největších počítačích, ke kterým získal přístup. Řadu svých výpočtů
  prováděl v Německu a zlobilo ho, že své programy nemohl jednoduše používat doma ve Spojených
  státech. V polovině osmdesátých let se mu podařilo přesvědčit vládu, aby založila síť
  superpočítačových center, jež by sloužila všem odvětvím vědy. 

  Sám Smarr se stal ředitelem jednoho z těchto center - National Center for Supercomputing
  Applications v Illinois. Právě jeho tým přišel v devadesátých letech s prvním webovým prohlížečem
  Mosaic, s jehož pomocí dokázal vizualizovat na dálku přes internet výsledky svých výpočtů. Černé
  díry a numerická relativita se tak účastnily zrodu Sítě, jež je dnes tak samozřejmou součástí
  našeho života. 

  Zatímco numeričtí relativisté zkoušeli vše možné, rozbíhaly se plány na stavbu opravdu efektivního
  zařízení na měření gravitačních vln. Teď už nemělo dojít k žádným falešným objevům, Weberova éra
  byla minulostí. Nový detektor měl pracovat na principu interferometrie, ovšem požadavky kladené na
  jeho konstrukci byly enormní. Dráha laserového paprsku měla být natolik dlouhá, aby se na
  interferenčním obrazci projevily i malé změny její délky, jaké mohly způsobit gravitační vlny.
  Délka interferometru byla i několik kilometrů. Laserové paprsky se nechávaly více než stokrát
  odrazit zrcátky na sledovaných tělesech, než jim bylo dovoleno spolu interferovat. Zrcátka musela
  být dokonale hladká a přesně rovnoběžná. Posun způsobený realistickými gravitačními vlnami by byl
  totiž neobyčejně malý. Výtrysk gravitačních vln způsobený zánikem dvojhvězdy by vedl k posunu
  rovnému asi tak průměru protonu. Postavit interferometr, který by mohl registrovat gravitační vlny
  přicházející ze vzdáleného vesmíru, je téměř nemožné. Laserový paprsek by musel uběhnout řadu
  kilometrů, aniž by se odchýlil z přímé cesty o více než průměr atomu. Zařízení se musí postavit
  tak, jako by se sledovaná tělesa vznášela volně ve vzduchu, musí být dokonale izolované od všeho
  šumu působeného pozemským okolím, s přesně vyrobenými zrcátky. Je třeba dokonale vyloučit účinek
  přílivu a odlivu, otřesy působené kamiony na dálnicích či vibrace elektrické sítě. 

  Detektory musí být ve všech směrech perfektní a přitom velké. Když se začalo uvažovat o detekci
  gravitačních vln pomocí interferometrů, bylo záhy jasné, že jejich rozměry a náklady na jejich
  stavbu drasticky omezí jejich počet. V Evropě spojili své úsilí Britové a Němci a postavili
  interferometr o délce asi 600 metrů. Je umístěn blízko města Sarstedt v Německu a nese jméno
  GEO600. Mnohem větší interferometr s rameny 3 kilometry postavili Francouzi a Italové u italského
  města Cascina. Jmenuje se VIRGO podle hvězdokupy v souhvězdí Panny (latinsky Virgo). V Japonsku
  byl vybudován malý interferometr TAMA o délce asi 300 metrů. 

  Tou největší chloubou laserové interferometrie je projekt LIGO, Laser Interferometry Gravitational
  (Wave) Observatory, ve Spojených státech. V čele tohoto projektu stáli původně dva experimentální
  fyzici Rainer Weiss z MIT Ronald Drever z Caltechu a teoretik Kip Thorne. Interferometr LIGO byl
  navržen počátkem sedmdesátých let minulého století, jeho zrod byl však obtížný a udál se po
  etapách. Mělo to být vůbec největší ze všech interferometrických zařízení. Projekt vlastně
  zahrnuje interferometry dva, jeden v Hanfordu ve státě Washington a druhý v Livingstonu ve státě
  Louisiana. Se dvěma detektory položenými daleko od sebe se dají vyloučit lokální šumy způsobené
  zemětřeseními, dopravou a podobně. A pokud se měření z LIGO spojí s údaji z jiného zařízení,
  například z GEO600, dá se přesně určit směr, odkud vlny přicházejí. Při přípravě projektu si ovšem
  nikdo nebyl jistý, co přesně by detektor měl měřit a jestli bude dostatečně citlivý. Nejdříve se
  měl postavit gigantický prototyp, jenž měl vyzkoušet, zda může skutečně měřit to, co relativisté a
  experimentátoři chtěli - to mělo zabrat více než desetiletí. Teprve potom měl být interferometr
  LIGO „upgradován“ a mohl začít pátrat po zajímavých údajích. Jedná se o dlouhodobý projekt, ale
  odměna za to, kdyby se opravdu podařilo gravitační vlny pozorovat, je více než lákavá. Pomocí
  gravitačních vln bychom mohli zkoumat vesmír zcela novým způsobem doplňujícím pozorování pomocí
  optických či rádiových vln. Byl by to také důležitý důkaz ve prospěch obecné teorie relativity. I
  když nyní mezi relativisty panuje přesvědčení, že gravitační vlny skutečně existují, nikdo je
  dosud přímo neviděl. Odhalí-li LIGO gravitační vlny, bude to svou důležitostí odpovídat objevu
  elektronu, protonu a neutronu v první polovině dvacátého století. Takový objev bude určitě odměněn
  Nobelovou cenou. 

  Nadšení pro LIGO nebylo ovšem všeobecně sdílené. Předpokládalo se, že projekt bude stát stovky
  milionů dolarů, a ty budou samozřejmě chybět na jiné výzkumné projekty, především na jiné
  experimenty s gravitačními vlnami, ale i na experimenty v jiných oblastech vědy. Výraz „observatoř
  “v jeho názvu trochu dráždil i astronomy, kteří se obávali, že LIGO odsaje peníze z jiných oborů
  astronomie. Tony Tyson z Bellových laboratoří, který se podílel na raných experimentech s
  gravitačními vlnami, napsal v roce 1991 v článku pro New York Times: „Většina astrofyziků se
  domnívá, že ze změřených signálů od gravitačních vln bude velice obtížné získat nějaké užitečné
  informace i v případě, že se je vůbec podaří změřit.“ A vůdčí princetonský astrofyzik Jeremiah
  Ostriker řekl pro New York Times, že „svět si bude muset počkat, až někdo přijde s levnější a
  spolehlivější metodou, jak gravitační vlny změřit“. Opozice mnoha astrofyziků byla hlasitá, někdy
  až divoká. Když v roce 1990 byla skupina astronomů vedená Johnem Bahcallem z princetonského
  Institutu pro pokročilá studia požádána o vypracování pořadníku astronomických projektů, jež by
  měly mít prioritu ve státní podpoře, projekt LIGO mezi ně vůbec nezařadila. 

  Americká Národní vědecká nadace, která poskytuje vládní podporu vědě, projekt LIGO nejdříve
  dvakrát zamítla. Trvalo pět let, než byl třetí návrh konečně schválen a dostal podporu 250 milionů
  dolarů - to se zdálo být přemrštěnou částkou na projekt, u kterého bylo pravděpodobné, že jeho
  výsledek bude nulový a že jej nepůjde technicky realizovat. Ale v roce 1992, po téměř dvaceti
  letech snění a příprav, se úžasný projekt konečně mohl rozběhnout. 

  Frans Pretorius se narodil v Jižní Africe v době, kdy Kip Thorne a jeho spolupracovníci už
  diskutovali o plánech na LIGO. Pretorius vyrostl ve Spojených státech a Kanadě a dokončil PhD na
  Univerzitě Britské Kolumbie ve Vancouveru, jednom z nervových center numerické relativity. Získal
  místo asistenta na Caltechu, kde působil i Kip Thorne, a mohl si vybrat, čím se bude zabývat.
  Rozhodl se pro studium splývajících černých děr, ale zvolil zcela originální postup. Nezapojil se
  do velkého týmu programátorů, kteří pracovali na stále nepřekonaném problému, jak simulovat
  vzájemný oběh černých děr po zmenšujících se spirálách, přechodné „cvrlikání“ a finální splynutí
  děr. Pracoval sám a začal od nuly. Prostudoval nezdařené pokusy z posledních desetiletí a nasbíral
  přitom některé myšlenky, které vypadaly slibně, a potom začal psát numerický program od samého
  počátku, do kterého získané ideje zabudoval. Měl neuvěřitelný instinkt pro to, co by mohlo
  fungovat a co ne. Einsteinovy rovnice se v jeho pojetí značně zjednodušily, takže připomínaly
  rovnice pro elektromagnetismus. A vývoj elektromagnetických vln je vcelku snadné vyřešit. 

  Hotový program spustil. Výpočet běžel několik měsíců - na tuto dobu Pretorius vzpomíná jako na
  „čistou agonii“. Ale s rostoucím údivem zjišťoval, že program stále běží, žádná numerická
  katastrofa ho nezarazí od chvíle, kdy obě černé díry začnou kolem sebe kroužit, až do okamžiku,
  kdy splynou. V průběhu celého procesu systém vyšle výtrysk gravitačních vln a skončí jako jedna
  rychle rotující černá díra. Na světě byl popis zdroje gravitačních vln, který všichni zoufale
  hledali, Pretorius konečně vyřešil Einsteinovy rovnice na počítači. Výpočet založil na myšlenkách,
  které se objevily už před ním, ale jemu se podařilo je skloubit tím správným způsobem. 

  O svých výsledcích referoval na konferenci o obecné teorii relativity, která se konala v lednu
  2005 v Albertě. Konečně se podařilo rozlousknout Einsteinovy rovnice a simulovat vzájemný oběh
  dvou černých děr, jež se přitahují tak silně, že nakonec splynou, když předtím vyslaly balík
  gravitačních vln. „Vzbudilo to značné vzrušení“, vzpomíná Pretorius, „kolegy to natolik zaujalo,
  že zorganizovali samostatné zasedání, kde mohli klást detailní otázky. O půl roku později oznámily
  další dvě skupiny, že vyřešily vývoj dvou splývajících černých děr zcela jinými metodami. I ony
  byly schopny sledovat celý proces hroucení a vysílání gravitačních vln. Jako by Pretoriův objev
  odstranil mentální blok, který se do té doby nad ostatními týmy vznášel. Výsledky, které se začaly
  na různých stranách objevovat, potvrzovaly Pretoriův výpočet. Byla cítit euforie a uvolnění,
  konečně se dala popsat forma gravitačních vln, jež produkují dvojhvězdy tvořené černými dírami.
  Pozorovatelé budou moci identifikovat příslušné tajemné signály, jež se skrývají v šumu
  interferometrů. 

  Konec svého života prožil Joseph Weber jako zatrpklý člověk. Při každé diskusi o gravitačních
  vlnách sršel hněvem, jehož obětí se stávalo posluchačstvo těch několika konferencí, kterých se
  zúčastnil. Rozzlobil se při každém pokusu klást mu nějaké otázky. On viděl gravitační vlny dříve
  než kdokoli jiný a to mu nikdo nevezme. Freeman Dyson, který byl jedním z jeho bývalých
  podporovatelů, jej prosil, aby ustoupil. Napsal mu: „Velký člověk se nemusí obávat přiznat veřejně
  svůj omyl a prohlásit, že změnil názor. Vím, že jste čestný muž. Jste dost silný na to, abyste
  přiznal, že jste se mýlil. Když to uděláte, sice se zaradují vaši nepřátelé, ale vaši přátelé se
  zaradují mnohem více. Zachráníte svou pověst vědce.“ 
  
  Nic takového Weber neudělal. Stal se naopak brzdou výzkumu gravitačních vln a aktivně bojoval
  proti projektu LIGO. Weber se dříve objevoval v tisku natolik často, že si v obecném povědomí
  vybudoval postavení předního experta na gravitační vlny a někteří důležití činitelé byli ochotni
  mu naslouchat. Když na začátku devadesátých let vědci udělali třetí zoufalý pokus o získání dotace
  pro LIGO, Weber napsal dopis Kongresu, kde prohlašoval, že podporovat toto velmi nákladné zařízení
  by bylo vyhazováním peněz. Tvrdil, že jeho válce viděly gravitační vlny za zlomek peněz
  požadovaných na LIGO, není tedy žádný důvod utrácet sta miliony dolarů. Jeho ostouzení projektu
  však mělo malý dopad. Během svého života udělal Weber řadu podivných prohlášení. Proto, jak
  vzpomíná Bernard Schutz, „v době, kdy napadal LIGO, už nikdo nebyl na jeho straně“. Weber jenom
  zhoršoval své postavení - nyní byl nepřítelem oboru, který sám stvořil. 
  
  Zemřel v roce 2000, dříve než bylo LIGO uvedeno do provozu. Zajistit, aby vše správně fungovalo,
  zabralo neobyčejně dlouhou dobu. Objevovaly se stále nové problémy, jež způsobovaly odklady. V
  osmdesátých a devadesátých letech uzavřel Kip Thorne se svými kolegy řadu sázek, že gravitační
  vlny budou objeveny do konce tisíciletí - a všechny prohrál. I na začátku nového tisíciletí byl
  režim LIGO několikrát narušen. Jednou to způsobily pily dřevorubců v Louisianském lese, jež
  rozkmitaly detektor v Livingstonu, jindy záhadné záchvěvy nukleárních reaktorů v okolí Hanfordu.
  Když ale bylo LIGO v roce 2002 konečně spuštěno a pak bylo několik let v provozu, dosahovalo
  žádané citlivosti. Byl to první stupeň experimentu, jehož základy byly položeny v devadesátých
  letech minulého století. Detektory byly schopny zachytit vibrace o rozkmitu menším než průměr
  protonu, jak se v projektu předpokládalo. Tým obsluhující LIGO dokonce oznámil, že zařízení je
  citlivější, než se předpokládalo. Byl to zkrátka úspěšný projekt, přestože nic nezpozoroval.
  Interferometr LIGO sice nebyl stále dost citlivý na to, aby zachytil gravitační vlny, ale ukazoval
  cestu kupředu. Teď se pracuje na dalších vylepšeních detektorů, které snad umožní pozorovat
  záhadné vlnky na prostoročase, jež Einstein kdysi předpověděl. 
  
  Na rozdíl od Weberových měření je pozorování pomocí LIGO během na dlouhou vzdálenost. Než budou
  vlny skutečně zaznamenány, budou muset pracovat tisíce techniků mnoho desítek let. Zřejmě už u
  toho nebude trio otců LIGO, Ron Dever, Kip Thorne a Rainer Weiss, kterým je dnes kolem osmdesáti
  let - možná své životy zasvětili něčemu, co nikdy neuvidí. Ale důvěra v existenci gravitačních vln
  je dnes obecně pevná - nejen proto, že je Einsteinova teorie předpovídá, ale i proto, že je
  pozorujeme nepřímo pomocí poruch drah milisekundových pulzarů. Je to jen otázka času, kdy místo
  Weberových krátkých rázů zaregistrujeme dlouhodobý jemný šepot prostoročasových vlnek
  prostupujících i Zemí.

\section{Temný vesmír}\label{feyIchIIIsecXII}
  V roce 1996 se v Princetonu konalo setkání nazvané Kritické dialogy o kosmologii, což byla
  vzájemná střetnutí dvojic vynikajících vědců, kteří diskutovali o různých problémech týkajících se
  vesmíru. Organizátoři vybrali některá palčivá témata a doufali v ostrý střet oponentů. Pozvané
  dvojice vědců - vynikajících astronomů, fyziků a matematiků - se nechovaly podle obvyklých
  pravidel konferencí, diskutující se snažili roztrhat na kusy tezi, kterou zastával jejich oponent.
  Byl to zvláštní, ale oživující způsob, jak diskutovat o vědě. 
  
  Sérii soubojů zahájil Martin Rees, který se zasloužil o porozumění černým dírám a teorii velkého
  třesku a stal se jedním z vrcholných představitelů relativistické astrofyziky. Argumentoval, že
  kosmologie je „tou nejzákladnější vědou a největší z věd o životním prostředí“. Kosmologie totiž
  nabízí tu nejkrásnější aplikaci nádherné matematiky a fyziky, kterou během dvacátého století
  rozvinuli Einstein, Dirac a řada dalších. Navíc se potýká s mnoha pozorováními galaxií, kvasarů a
  hvězd a snaží se vysvětlit zdánlivě zmatený mechanismus, který dohromady dává velký obraz vesmíru.
  Rees řekl, že cíl kosmologie je velmi nesnadný a kontroverzní, zatím jsme od něho daleko, ale je
  nanejvýš důležitý. 
  
  V době princetonského setkání byl obraz vesmíru opravdu bizarní. Zdálo se, že mu rozumíme méně,
  než jsme si až dosud mysleli. Velkou část náplně vesmíru měly představovat exo- tické substance,
  které v pozemských laboratořích nikdo nespatřil. Dostaly název „temná hmota“a „temná energie“.
  Měli jsme důkazy, že skutečně existují, protože podstatně ovlivňují prostoročas, neuměli jsme je
  však přímo pozorovat. Téma temného vesmíru se dostalo na pořad jednání jednoho odpoledne, když se
  debatovalo o struktuře vesmíru ve velkých měřítkách. Byla to právě tato tematika, která přilákala
  ke kosmologii i mě. 
  
  Zahledíme-li se na nebe, vidíme složitou strukturu světlých bodů. Galaxie jsou vázány v kupy,
  vlákna a stěny a mezi nimi jsou velké oblasti prázdna. Jejich uspořádání je roztodivné a složité.
  Jak tyto velkorozměrné struktury vznikly? To byla jedna z nejtíživějších otázek pro účastníky
  setkání, a protože odpovědi byly značně nejisté, organizátoři věnovali této problematice celé
  odpoledne. J. Richard Gott, vytáhlý princetonský astronom, mluvící pomalým jižním nářečím, se
  snažil ve svém vystoupení obhajovat hledisko selského rozumu. Na první pohled se vesmír zdá
  prázdný, tak Gott propagoval model vesmíru téměř zbavený veškeré hmoty, ve kterém se pomalu
  vyvíjela tapiserie galaxií a kup galaxií, kterou na obloze vidíme. Jiný mladý astronom z
  Princetonu David Spergel naopak tvrdil, že vesmír není vůbec prázdný, nýbrž je naplněn
  neviditelnou formou hmoty, takzvanou temnou hmotou. Spergelova temná hmota měla sestávat z
  jakýchsi fundamentálních částic, které nevystupují ve standardním modelu a nebyly zatím
  experimentálně pozorovány. Nejpodivnější předpoklad celého odpoledne však vyslovil skvělý
  teoretický kosmolog z Chicaga Michael Turner: „Proč nepředpokládat, že vesmír je naplněn energií
  kosmologické konstanty?“V Turnerově vesmíru by asi dvě třetiny celkové energie připadaly právě na
  konstantu, kterou Einstein zavedl a pak zase tak rázně odmítl před téměř sedmdesáti roky.
  Posluchačstvo tím nebylo nadšeno. Všechno, jen ne kosmologickou konstantu - vždyť to byl
  Einsteinův největší omyl. 
  
  Gladiátorskému souboji vesmírů předsedal Phillip James (Jim) Peebles, který zastával v Princetonu
  profesuru pojmenovanou po Albertu Einsteinovi. Peebles byl vysoký, štíhlý chlapík s protáhlými
  rysy, který jako by vypadl z Modiglianiho obrazu, džentlmen, který diskusi moderoval se
  starosvětskou zdvořilostí.Snažil se sice udržet diskusi v patřičných mezích, občas se mu ale
  obličej rozzářil téměř dětským úsměvem, když ze sálu vyletovaly různé posměšné poznámky. „Kritické
  dialogy“byly zorganizovány částečně jako konference k oslavě jeho šedesátých narozenin, což byla
  velmi vhodná pocta. V posledních třiceti letech byl totiž Peebles hlavním architektem teorie
  stavby vesmíru ve velkých měřítkách z hlediska moderní kosmologie. V raných sedmdesátých letech
  vydal útlou knihu s názvem Fyzikální kosmologie, což bylo shrnutí jeho přednášek pro postgraduální
  studenty, které konal v Princetonu v roce 1969. Přednášek se účastnil i John Wheeler, který si
  dělal poznámky, a podle Peeblese to byl on, kdo ho přesvědčil, že má přednášky vydat knižně. V
  úvodu k Fyzikální kosmologii se Peebles o kosmologické konstantě krátce zmiňuje, říká však, že
  „kosmologická konstanta Λ (velké řecké „lambda“ je tradiční označení této konstanty) je v těchto
  přednáškách zmiňována jen zřídkakdy“. Pokládal ji za zbytečnou komplikaci, „malé nepatřičné
  tajemství kosmologie“. Každý z relativistických kosmologů věděl, že matematika obecné relativity
  ji dovoluje, protože by ale fyziku činila bizarní a nepohodlnou, každý se tvářil, jako by
  neexistovala. Nyní, o čtvrt století později, se kosmologická konstanta znova dostávala na scénu,
  přestože ji většina Peeblesových kolegů zavrhovala. A drala se zpět plna pomstychtivosti. 
  
  Když Jim Peebles přišel v roce 1958 do Princetonu jako čerstvý absolvent Manitobské univerzity,
  byl tam John Wheeler a jeho spolupracovníci, kteří zkoumali černé díry a finální stav hvězd. Ale
  Wheeler nebyl jediným veleknězem relativity v Princetonu; byl tam i Robert Dicke. Dicke si v
  polovině padesátých let uvědomil, jak malý či spíše žádný pokrok nastává v experimentálním ověření
  Einsteinovy obecné teorie relativity. Vytvořil v Princetonu svou vlastní skupinu, jež se věnovala
  gravitaci a ve které se debatovalo o obecné relativitě, a co bylo ještě důležitější, kde se i
  experimentálně testovala. „Ve své kariéře jsem se brzy dostal na oběžnou dráhu kolem Boba a tím k
  práci na vzrušujících věcech,“ vzpomíná Peebles. K Dickeho skupině se připojil jako doktorand a po
  získání PhD zaměřil svůj zájem na testování teorie gravitace. V Princetonu zůstal následujících
  padesát let. 
  
  Peebles říká, že kosmologie byla v té době „velmi omezeným předmětem - vystupovala v ní dvě nebo
  tři čísla a věda se dvěma nebo třemi čísly na mne působila depresivně“. Fyzikálními aspekty
  kosmologie se zabývalo jen pár vědců a celkově byl tento směr výzkumu málo sledovaný. To
  Peeblesovi vyhovovalo, mohl se svým vlastním tempem věnovat problémům, které ho přitahovaly.
  Doktorát získal na poli kvantové fyziky a pak se už věnoval vyjasňování některých otázek
  kosmologie. Začal zkoumáním, co se skutečně dělo s atomy a jádry v době, kdy byl vesmír velice
  hustý a horký - tedy v „primordiální ohnivé kouli“, jak to nazvali jeho kolegové v Princetonu.
  Pracoval jako řemeslník. Zavřen ve své pracovně popisoval rovnicemi stránku za stránkou, pomalu
  procházel své výpočty a vybrušoval svůj přístup. 
  
  Metoda Peeblesova učitele Roberta Dickeho byla rozdílná. Podle Peeblese „uznával důležitost
  fyzikálních teorii, jenže ho zajímaly jen ty, jež vedly k experimentům uskutečnitelným v blízké
  budoucnosti“. Proto Dickeho tým začal pátrat po reliktním záření zanechaném primordiální ohnivou
  koulí. Vyvinuli novou formu detektoru, který mohl skenovat oblohu ze střechy budovy fyzikálního
  ústavu, ale hledané záření nenašli dostatečně brzy. Jednoho úterý koncem roku 1964 seděl tým v
  Dickeho pracovně na pravidelném týdenním setkání, když zazvonil telefon. Dicke ho zvedl a několik
  minut s někým mluvil. Když zavěsil, prohlásil: „Předběhli nás!“Volal mu totiž Arno Penzias, aby mu
  řekl, že spolu Robertem Wilsonem právě v Bellových laboratořích našli důkaz existence reliktního
  záření. Během několika měsíců Dicke a jeho tým výsledky z Bellových laboratoří potvrdili, to ale
  už bylo pozdě - Nobelovu cenu dostali jen Penzias a Wilson. 
  
  Podle Peeblese bylo něco v nepořádku s obrazem vesmíru, který v šedesátých letech předkládaly
  fyzikální učebnice. Vystupovala v nich dvě rozdílná témata. Prvním byla historie a vývoj vesmíru
  jako celku, tedy příběh odvíjející se od Friedmanna a Lemaîtra. Vysvětloval, jak se prostor, čas a
  hmota vyvíjely v těch největších možných měřítkách. Druhým tématem byly objekty, které pozorují
  astronomové - galaxie a kupy galaxií. Tyto galaxie jsou sice součástí vesmíru, jejich přítomnost
  však jako by moc nesouvisela s jeho základním vývojem, jako by to byly jen mohutné světelné víry
  namalované na prostoročasu. Je pravda, že galaxie nám toho o vesmíru říkaly mnoho - jak rychle se
  rozpíná a jaká je celková hmotnost v něm obsažená. Peebles však cítil, že vývoj vesmíru musí být s
  vývojem galaxií nějak úže propojen, že hrají klíčovou roli při vývoji velkých struktur ve vesmíru
  a že s vývojem vesmíru je těsně propojen i jejich původ. Nemohly se objevit z ničeho, tyto velké
  chuchvalce světla, plynu a hvězd přece nespadly do prostoročasu až dodatečně. To znamená, že i ony
  mají závažné postavení v Einsteinově obecné teorii relativity, a otázka zněla jaké. To byla pro
  Peeblese ta správná výzva: obtížný obecný problém, kterým se chtěl jen málokdo zabývat. 
  
  Role gravitace při formování individuální galaxie je zřejmá. Je-li k dispozici dostatek hmoty a ta
  má dostatečnou kinetickou energii, aby zabránila kolapsu pod určitou mez, výsledný chuchvalec se
  stane galaxií, jejíž jednotlivé části jsou k sobě poutány vzájemným gravitačním přitahováním. Méně
  zřejmé bylo, jak se na gravitačních efektech uvnitř určité galaxie podílí celkové rozpínání
  vesmíru. Už abbé Lemaître poukazoval na to, že mezi nimi musí být určité spojení, a ruský teoretik
  George Gamow rozvažoval, jak se galaxie budou vytvářet v rozpínajícím se vesmíru. Nikdo však
  nepředložil jasný výpočet podkládající jejich spekulace. V roce 1946 jeden z Landauových žáků
  Jevgenij Lifšic vzal Einsteinovy rovnice a na jejich základě zkoumal, co se ve vesmírných
  měřítkách stane s mnohem menšími strukturami, jakými jsou jednotlivé galaxie. Jeho výpočty
  naznačovaly, jakým způsobem se vynoří struktury ve velkých měřítkách - v souladu s Einsteinovými
  rovnicemi budou malé poruchy hladké struktury prostoročasu postupně narůstat a galaxie se usadí v
  oblastech s větším zakřivením. Tam se nahromadí materiál v galaxiích sdružených do kup galaxií a
  tyto struktury dnes pozorujeme. 
  
  Když Peebles rozebíral, jak se chovaly atomy a jádra v horkém raném vesmíru, uvědomil si, že zde
  může ležet vysvětlení, jak vznikly zárodky galaxií již krátce po velkém třesku. Udělal určitý
  hrubý odhad věku vesmíru, hustoty atomů a teploty reliktního záření a zjistil, že se kolapsem
  mohly vytvářet struktury s celkovou hmotností až tisíců miliard hmotností Slunce, srovnatelné s
  naší galaxií Mléčné dráhy. Jak už předpokládal Gamow, raný vesmír byl tím ideálním pozemkem pro
  pěstování galaxií. Peebles nebyl jediný, kdo se zabýval detaily tvoření galaxií. Mladý doktorand
  na Harvardu Joseph Silk došel k závěru, že kolabující chuchvalce, ze kterých se časem vytvoří
  galaxie, musely zanechat otisk na struktuře primordiální ohnivé koule - jemný vzorek horkých a
  chladnějších oblastí reliktního záření, jež krátce předtím objevili Penzias s Wilsonem. Silkovy
  výsledky měly ohlas u Rainera Sachse a jeho studenta Arthura Wolfa v Austinu. Ti zjistili, že
  reliktní záření by bylo ovlivněno kolapsem hmoty ve vesmíru i ve větších rozměrech. A ke stejnému
  závěru došla i Zeldovičova skupina v Sovětském svazu. Tyto výsledky ukazovaly na to, že pohled na
  nepravidelnosti v obrazu reliktního záření, pocházejícího z doby, kdy bylo vesmíru jen několik set
  let, je pohledem i na první okamžiky tvorby galaxií. I když zde bylo mnoho nejasností, Gamowova a
  Peeblesova fyzikální kosmologie začínala přinášet ovoce. 
  
  Peebles chtěl popsat expanzi vesmíru - horký začátek, vznik atomů a gravitační kolaps vedoucí k
  tvorbě galaxií - pomocí učebnicové fyziky, kde se kombinovala obecná teorie relativity,
  termodynamika a zákony ovládající světlo. Se svým doktorandem Jerem Yu z Hong Kongu sepsali úplný
  soubor rovnic, jež dovolovaly popsat vývoj vesmíru od nejranějších momentů po velkém třesku až po
  dnešek. Peeblesův vesmír započal v homogenním horkém stavu směsi plynu a záření, narušeném jen
  jemnými poruchami. Když se tyto poruchy začaly vyvíjet, působil na ně tlak neuspořádaného
  lepkavého plazmatu tvořeného volnými protony, neutrony a elektrony. Postupně se vytvořilo určité
  množství jader helia s malou příměsí jader lehkých prvků a o něco později se vytvořily neutrální
  atomy. Pak následovala další fáze: atomy a molekuly se začaly v důsledku vzájemného gravitačního
  působení shlukovat, až vytvořily svítící ostrovy hmoty rozeseté v prostoru. Tak z horkého velkého
  třesku vznikly galaxie a kupy galaxií. 
  
  V Peeblesově a Yuově vesmíru si struktura rozptýlení galaxií v prostoru zachovává vzpomínku na
  horký počátek vesmíru. Reliktní záření, jehož dnešní teplotu Penzias s Wilsonem odhadli na 3
  kelviny, nese obraz původních nehomogenit, jež byly zárodky pozdějších galaxií. Jednotným a
  konzistentním řešením rovnic pro vývoj vesmíru nalezli Peebles a Yu nový účinný způsob, jak
  studovat Einsteinovu obecnou teorii relativity: pozorovat rozložení galaxií v prostoru, tedy
  strukturu vesmíru ve velkých měřítkách, a pomocí těchto pozorování odhalit, jak se prostor
  vyvíjel.
  
  Peebles a Yu napsali silný a strhující příběh, reakcí na něj však bylo spíše mlčení. „Našemu
  článku nikdo nevěnoval zvláštní pozornost,“ vzpomíná Peebles. Tím, že autoři sváděli dohromady
  různé oblasti fyziky, vstupovali na intelektuální území nikoho. Nebyla to přesně vzato ani
  astronomie, ani obecná teorie relativity, ani základní fyzika. Peeblese se však nedostatečný ohlas
  nedotkl. Dále pracoval na svém obrazu vesmíru, občas podporován studenty nebo mladými
  spolupracovníky, většina výpočtů však byla jeho. 
  
  Nyní když Peebles měl svůj model vesmíru, musel se podívat na některé výsledky pozorování, aby
  zjistil, je-li na správné cestě. V raných padesátých letech prostudoval francouzský astronom
  Gérard de Vaucouleurs z Texaské univerzity takzvaný Shapleyův-Amesův katalog, evidující více než
  tisíc galaxií, a vypozoroval v něm „proud galaxií“, probíhající přes celou oblohu, jenž byl větší
  než každá kupa galaxií. Byla to jakási „superkupa“ či „supergalaxie“. Jeho práce nebyla příliš
  příznivě přijata. Walter Baade, astronom z Caltechu, výsledek odmítl: „Pro existenci supergalaxie
  nejsou žádné důkazy.“Podobně reagoval Fritz Zwicky, který prohlásil naplno: „Žádná superseskupení
  neexistují.“ I Peebles byl k de Vaucouleursovým závěrům skeptický. Jeden z jeho studentů však
  vzpomíná, jak citoval slova svého učitele Boba Dickeho, že „dobré pozorování je mnohem cennější
  než nová průměrná teorie“. Rozhodl se tedy pořídit mapu velkých struktur ve vesmíru sám. Spolu se
  svými spolupracovníky dospěl k některým překvapujícím výsledkům. Když dva mladí badatelé z
  Harvardu Marc Davis a John Huchra zjistili, že i podle jejich mnohem podrobnějšího obrazu
  rozložení galaxií na obloze obrovské struktury existují, Peebles byl ohromen. Prohlásil: „Napsal
  jsem několik pěkně kousavých článků o tom, jak astronomové byli často svedeni na nesprávnou cestu
  snahou nalézt v šumu nějakou strukturu, a doprovodil jsem to příklady. Je zřejmé, že je třeba mít
  nějaký jasný mechanismus, který k vytváření takových struktur vede.“ Jenže časem zjistil, že
  galaxie skutečně tvoří složitou tapiserii zdí, vláken a kup, které se začalo říkat kosmická síť.
  Strukturu ve velkých měřítkách, kterou předpovídaly Peeblesovy počítačové modely, začali
  astronomové objevovat v reálném světě. 
  
  V roce 1979 Stephen Hawking spolu s jihoafrickým relativistou Wernerem Israelem vydali sborník
  článků o obecné relativitě, který měl uctít sté výročí narození Alberta Einsteina. Přispěli do
  něho vedoucí badatelé v kosmologii, fyzice černých děr i kvantové gravitaci. Bob Dicke a Jim
  Peebles dodali krátký esej „Kosmologie s velkým třeskem - záhady a všeléky“, kde na několika
  stránkách vysvětlili, co pokládají za fundamentální problémy neuvěřitelně úspěšné teorie. 
  
  Jaké problémy zde byly? Především se zdálo, že vesmír je příliš hladký. Dělaly se sice pokusy, jak
  homogenitu vesmíru vysvětlit, ale Dickemu a Peeblesovi se nezdály věrohodné. A byly i další
  problémy. Proč se prostorová geometrie, na rozdíl od geometrie prostoročasu, zdá tak jednoduchá?
  Vypadalo to, že geometrie prostoru není celkově zakřivená a v průměru ve vesmíru platí pravidla
  euklidovské geometrie, která známe ze školy. Jsou to poučky jako „Dvě rovnoběžky se nikdy
  neprotnou“ a „Součet úhlů v trojúhelníku je 180 stupňů“. Vesmír s nulovou prostorovou křivostí
  sice obecná teorie relativity dovoluje, je to ale velmi speciální případ. Podle Einsteinových
  rovnic se prostorová křivost, která se původně jen málo lišila od nulové, s časem rychle zvětšuje.
  To znamená, že vesmír, který dnes nemá téměř žádnou křivost, musel mít v minulosti křivost ještě
  mnohem menší. Žijeme v neobyčejně nepravděpodobném vesmíru. A konečně, galaxie a struktury
  vytvořené z galaxií, které ve vesmíru pozorujeme, se musely odněkud vzít. Aby vesmír vypadal tak,
  jak vypadá, musely být v minulosti podmínky jemně vyladěny. Při velkém třesku získal vesmír právě
  tak velikou tendenci se rozpínat, aby byla překonána snaha gravitace jeho expanzi brzdit. Ne ale
  zase tak extrémně velkou, aby se rychle rozepjal natolik, že by zbyla v podstatě prázdnota. A
  článek směřoval k té zásadní otázce: co se stalo na samém počátku? 
  
  Kromě článku Dickeho a Peeblese obsahoval sborník i krátký esej Jakova Zeldoviče, ve kterém autor
  rozebíral ten nejranější vesmír, a jeho úvahy trochu připomínaly přístup abbého Lemaîtra k
  primordiálnímu atomu. V horkém raném vesmíru mohla hrát roli spousta zajímavých jevů, které mohly
  ovlivnit jeho budoucí vývoj. Zeldovič vyzýval částicové fyziky i relativisty, aby tyto jevy
  zkoumali. 
  
  Článek Dickeho a Peeblese, právě tak jako esej Zeldoviče, byl prorocký. Jen o rok později se
  kosmologie úplně otočila naruby v důsledku jednoduché modifikace předpokladů o vývoji raného
  vesmíru. V nevybroušené formě visela už dlouho ve vzduchu myšlenka, které se řádně chopil až Alan
  Guth, postdok ve Stanfordském centru lineárního urychlovače, když přišel s představou kosmické
  inflace. Guth si uvědomil. že podle některých teorií velkého sjednocení, které se snaží sjednotit
  popis elektromagnetické, slabé a silné interakce, se v určitém okamžiku může svět nacházet ve
  stavu, kdy energie jednoho z polí je neuvěřitelně vysoká a převyšuje všechno ostatní. V tomto
  stavu by se vesmír rozpínal a nadouval neobyčejně prudce. Guth o tomto procesu hovořil jako o
  „inflaci“. Původní Guthova myšlenka neobstála, ukázalo se totiž, že jednou započatou inflaci nelze
  zastavit, objevilo se však několik dalších scénářů inflace, jež byly životaschopné. 
  
  Představa inflace otevřela v kosmologii zcela novou cestu. Podle ní měla být v minulosti vesmíru
  perioda, která si zasluhovala podrobnějšího zkoumání, protože právě v této periodě prudkého
  nadouvání vznikly zárodky galaxií. Hypotéza se dotýkala i problémů, o kterých psali Dicke a
  Peebles - inflace totiž měla vesmír roztáhnout téměř okamžitě do stavu, kdy neměl žádnou křivost.
  Představte si, že máte kulatý míč tak velký, že ho pohodlně udržíte v rukou, a obří pumpu, jež vám
  dovolí míč téměř okamžitě nafouknout do rozměrů zeměkoule. Podíváte-li se zblízka na jeho povrch
  po nafouknutí, pak se vám kus povrchu, který máte před očima, bude jevit dokonale plochý. Podobně
  by inflace uvedla vesmír do stavu, ve kterém by se všechny nepravidelnosti vyhladily. Jakákoli
  větší porucha na tváři vesmíru by byla odklizena do nedohledna. Z inflačního obrazu také
  vyplývalo, že zárodky pozdějších vesmírných struktur vznikly právě v tomto období. Kvantové
  fluktuace na tváři vesmíru, které se objevily na počátku inflační periody, by se během intenzivní
  inflace roztáhly a později by hrály roli kondenzačních jader, ze kterých se dále vyvíjely galaxie.
  Inflace zprostředkovala propojení „vnitřního prostoru s vnějším“, jak to výstižně popisovali
  astronomové z Chicaga. „Vnitřním prostorem“rozuměli svět kvantový a svět základních sil a „vnější
  prostor“byl vesmír, kde se uplatňovala obecná teorie relativity. A tak výzkum rozvíjený v
  předchozím desetiletí Peeblesem, Zeldovičem, Silkem a dalšími získal nový cíl: nebeské struktury
  ve velkých měřítkách, rozložení galaxií a reliktní záření by v sobě měly nést informace o spojení
  mezi vnitřním a vnějším prostorem. Vědce to začalo zajímat. 
  
  V roce 1982 se Peebles pokusil zkonstruovat nový vesmír. Starý model, který vybudoval s Jerem Yu z
  atomů a záření, s pozorováním nesouhlasil, předpověděné rozložení galaxií neodpovídalo. Realita
  prostě odmítala podrobit se elegantním výpočtům. A to nebylo všechno. Během posledních deseti let
  se ukazovalo, že galaxie jsou složitější útvary, než jsme si předtím mysleli. Dělo se v nich něco
  podivného. 
  
  Americká astronomka Vera Rubinová zjistila, že galaxie musí držet pohromadě něco neznámého,
  protože se otáčejí příliš rychle. Rubinová zaměřila svůj teleskop na galaxii v Andromedě, která se
  jevila jako vír hvězd a plynu pohybující se dokola rychlostmi dosahujícími stovek kilometrů za
  sekundu. Alespoň tak se to jevilo v teleskopu. Nejvíce světla vycházelo ze středu galaxie, a tak
  se Rubinová domnívala, že právě tam bude soustředěno nejvíce hmoty zodpovědné za gravitační
  přitažlivost. Když se ale dívala na hvězdy od středu hodně vzdálené, zjišťovala, že právě ty se
  pohybují velmi rychle, tak rychle, že se zdálo nepochopitelné, jak může gravitační přitažlivost
  galaktické hmoty zabránit tomu, aby neodlétly z galaxie pryč. Bylo to, jako by se oběžná rychlost
  Země, která je v rovnováze s gravitační přitažlivostí Slunce, náhle zdvojnásobila či
  ztrojnásobila. Kdyby Slunce nějak nezvětšilo svoji přitažlivost, Země by opustila dráhu kolem
  Slunce a odlétla někam do širého prostoru. Pozorování Rubinové ukazovala, že v galaxii něco drží
  vnější hvězdy na jejich drahách, něco neviditelného, ale s velkou gravitační přitažlivostí.
  
  Podobnou skutečnost zjistil už ve třicátých letech minulého století Fritz Zwicky, ale jeho měření
  zůstala téměř čtyřicet let nepovšimnutá. Zwicky sledoval kupu galaxií v souhvězdí Coma (Vlasy
  Bereniky) a sečetl veškerou hmotnost, kterou tam mohl pozorovat. Potom měřil rychlost galaxií,
  které kupu tvoří, a také zjistil, že se pohybují rychleji, než odpovídá přitažlivosti pozorované
  hmoty. V článku, který publikoval v roce 1937 ve Švýcarsku, uváděl: „Hustota svítící hmoty kupy v
  souhvězdí Coma je malá oproti hustotě jakési temné hmoty.“
  
  Jim Peebles měl s galaxiemi podobný problém. Spolu s mladým spolupracovníkem z Princetonu Jerrym
  Ostrikerem sestavili jednoduchý počítačový model tvoření galaxií. Předpokládali, že částice jejich
  modelu se gravitačně přitahují a po spirálách se k sobě přibližují. Jakmile ale nechali model
  rotovat, galaxie se od sebe rozletěly. V centru se utvořil chuchvalec, ze kterého se vyvinula
  ramena a galaxii roztrhala. Ostriker a Peebles se snažili svůj model stabilizovat tím, že své
  rotující částice vnořili do koule z neviditelné hmoty. Tato koule materiálu - nazvali ji halo -
  dokázala svou gravitací udržet galaxii pohromadě. Halo muselo být temné, tedy neviditelné
  teleskopy. A paradoxně se ukazovalo, že jeho hmotnost musí být podstatně větší, než hmotnost
  atomární látky ve hvězdách. Koncem sedmdesátých let Sandra Faberová, která pracovala v Santa Cruz
  v Kalifornii, a Jay Gallagher z Illinois napsali článek shrnující starší výsledky získané
  pozorováním galaxií a konfrontovali je s výsledky Peeblese a jeho kolegů při simulacích tvoření
  galaxií. Článek uzavřeli tvrzením: „Pokládáme za pravděpodobné, že objev neviditelné hmoty přetrvá
  a bude to jeden z velkých výsledků moderní astronomie.“
  
  V roce 1982, když Peebles začal tvořit nový model vesmíru, rozhodl se zahrnout do něj jak atomy,
  tak temnou hmotu. Ve skutečnosti předpokládal, že z tajemné temné hmoty je vytvořen téměř celý
  vesmír. Tato neznámá hmota musela být tvořena těžkými částicemi a byla neviditelná proto, že
  neinteragovala se světlem. Peeblesův model s chladnou temnou hmotou byl jednoduchý a dovolil
  předpovědět, jak by mělo vypadat rozmístění galaxií a jak by vypadaly nepravidelnosti v reliktním
  záření. Časem se ukázalo, že tento přístup bude mít veliký dopad na kosmologii. Peebles však
  vzpomíná: „Náš model jsem nebral příliš vážně … pracoval jsem s ním, protože byl jednoduchý a
  zdálo se, že souhlasí s pozorováním.“
  
  I když Peeblesův model nezahrnoval inflační éru, o které se v té době hodně uvažovalo, přesto
  dobře zapadal do ducha doby. Předpokládal existenci hmotných částic, které se měly nějak vylíhnout
  z fyziky částic, a spojoval tak vnitřní a vnější vesmír. Model s chladnou temnou hmotou,
  standardně označovaný jako CDM (podle anglického Cold Dark Matter), přijala řada astronomů a
  fyziků, kteří dopracovávali detaily, jak se galaxie skutečně tvořily. Marc Davis v Berkeley a
  mexický astronom Carlos Frenk se spojili se dvěma britskými astronomy Georgem Efstathiou a Simonem
  Whitem. Společně pracovali na počítačovém programu, který dovoloval sledovat tvoření galaxií a kup
  galaxií ve virtuálních vesmírech. Počítačové simulace tohoto „gangu čtyř“, jak se jim přezdívalo,
  dovolovaly sledovat vývoj mnoha set tisíc navzájem gravitačně interagujících částic vytvářejících
  velké vesmírné struktury. 
  
  I když byl model CDM populární a obecně přijímaný, řada věcí nefungovala tak, jak by měla. V
  Peeblesově modelu CDM mohl být vesmír starý nejvýše 7 miliard let, což bylo příliš málo.
  Astronomové objevili v galaxiích husté kapsy hvězd, zvané kulové hvězdokupy. V těchto jasných
  objektech byly staré hvězdy, jež se musely vytvořit v rané historii vesmíru, kdy byl vesmír
  naplněn téměř výhradně vodíkem a heliem. To znamenalo, že tyto staré hvězdy existují nejméně 10
  miliard let. A model narážel i na další problémy. Je-li vesmír tvořen především chladnou temnou
  hmotou, poměr temné hmoty k hmotě atomární by byl asi 25 : 1. 
  
  Astronomové však přes veškeré úsilí nemohli najít, kde se tolik temné hmoty nachází. Z rotačních
  rychlostí galaxií a z teploty galaktických kup (čím vyšší teplota, tím větší gravitační
  přitažlivost) mohli usoudit, kolik temné hmoty by vytvářelo odpovídající gravitační přitažlivost,
  která by udržela dané seskupení v rovnováze. Takto ale vycházel poměr neviditelné hmoty k hmotě
  atomární asi 6 : 1. Je pravda, že metody vážení temné hmoty byly hrubé a nepřesné, deficit však
  byl příliš velký, než aby ho vysvětlila chyba měření. Proto také téměř okamžitě po vzniku modelu
  CDM se Peebles snažil vytvořit modely alternativní. Jak vzpomíná, „koncem osmdesátých a počátkem
  devadesátých let se objevila řada kandidátů“. 
  
  Gang čtyř na tom nebyl lépe. Vyráběl spoustu virtuálních modelů vesmíru a porovnával je s tím
  skutečným. Žádný se mu nepodobal. Zdálo se, že skutečný vesmír je ve velkých měřítkách mnohem
  strukturovanější než kterýkoli z virtuálních „falešných“vesmírů. Některé nesouhlasy mezi modelem a
  pozorováním se daly trochu zmenšit určitou manipulací s výsledky, pravdou však bylo, že Peeblesův
  model plně nefungoval. 
  
  Většina astronomů a fyziků však model vesmíru s chladnou temnou hmotou v zásadě přijímala,
  přestože přesně nesouhlasil s pozorováním. Byl koncepčně jednoduchý, dal se snadno sladit s
  inflací a vysvětloval pozorování dokazující přítomnost temné hmoty v galaxiích. Jeho příznivci se
  jej snažili dále rozvinout a nějak upravit. Jednou z cest k tomu bylo vzkříšení Einsteinovy
  kosmologické konstanty. Pro mnohé to však bylo něco naprosto vyloučeného. 
  
  Od roku 1917, kdy ji Einstein poprvé zavedl, byl osud kosmologické konstanty složitý. Einstein sám
  ji po objevu rozpínání vesmíru rychle zavrhl, několik jeho kolegů s ní však počítalo i nadále. Do
  svých modelů vesmíru ji zahrnul jak Eddington, tak abbé Lemaître, který šel dokonce tak daleko, že
  ji prohlašoval za hustotu energie vakua. V roce 1967 Zeldovič poukázal na vážný problém s
  kosmologickou konstantou. Sečetl energii všech virtuálních částic, které se ve vesmíru neustále
  vynořují a zase zanikají, a zjistil, že výsledné rozložení energie a tlaku bude sice vypadat tak,
  jako kdyby v Einsteinových rovnicích vystupovala kosmologická konstanta, ale obrovská. A to
  skutečně obrovská. Úplně přesně vzato vycházela její hodnota nekonečná, ale použitím určitého
  matematického triku se dala snížit na konečnou, byť nesmírně velkou. Její velikost byla o mnoho
  řádů větší, než bylo slučitelné s pozorováním - nepředstavitelně větší, než jakou ji kdysi
  předpokládal Einstein. 
  
  Zeldovičovy výpočty jasně ukázaly, že existuje-li ve vesmíru energie vakua - a tím i kosmologická
  konstanta -, měla by být mnohem větší, než dovolovala pozorování. Zdálo se proto, že musíme
  připustit existenci nějakého dosud neznámého fyzikálního mechanismu, který tuto hodnotu
  vykompenzuje, takže konstanta nakonec vyjde rovná nule. A tak se kosmologové rozhodli ji ignorovat
  a předpokládat, že neexistuje. 
  
  Jenže znovu a znovu se ukazovalo, že kosmologická konstanta - známá jako „lambda“ - se nabízí jako
  záchrana při řešení různých problémů s temnou hmotou. Sám Peebles nalezl v roce 1984 použitelný
  vesmír s temnou hmotou, ve kterém kosmologická konstanta dávala 80 \% celkové energie. A když do
  svých výpočtů zahrnul „lambda“ gang čtyř, tedy Davis, Frenk, Efstathiou a White, mnoho problémů
  spojených s jednoduchým scénářem CDM najednou zmizelo. 
  
  V roce 1990 George Efstathiou, tehdy na Oxfordské univerzitě, uveřejnil se spolupracovníky v
  časopise Nature článek „Kosmologická konstanta a chladná temná hmota“. V něm srovnával velké
  struktury ve vesmíru skutečném se strukturami, které se svými spolupracovníky získal počítačovými
  simulacemi, v nichž vystupovala i kosmologická konstanta. Užili k tomu katalogu s miliony galaxií,
  které nashromáždili za mnoho let. V úvodu uváděli: „Tvrdíme, že teorie CDM může dát prostorově
  plochý model, jenž je v souladu se současným pozorováním, jestliže 80 \% veškeré energie vesmíru
  zajišťuje kladná kosmologická konstanta.“ V článku pak detailně srovnávají předpovědi modelu s
  tehdy dostupnými výsledky pozorování. 
  
  V roce 1995 Jerry Ostriker a Paul Steinhardt, jeden z otců inflační teorie, uveřejnili v Nature
  článek, kde tvrdili, že „favorizovaným vesmírem se zdá být vesmír s kritickou hustotou a velkou
  kladnou kosmologickou konstantou“. Všechno zřejmě ukazovalo na existenci „lambda“, vědci ale byli
  stále zdrženliví. Jak napsal v roce 1984 John Peebles: „Problém je … že výběr není plausibilní.“
  Jak uváděli Efstathiou s kolegy v závěru svého článku: „Nenulová kosmologická konstanta by
  znamenala vážné problémy pro fundamentální fyziku.“ George Blumenthal, Avishai Dekel a Joel
  Primack ze Santa Cruz v Kalifornii konstatovali, že kosmologická konstanta vyžaduje
  nepravděpodobně jemné vyladění všech parametrů teorie. Podle Jerryho Ostrikera a Paul Steinhardta
  experimentální důkazy vedly k vážnému problému: „Jak můžeme vysvětlit nenulovou hodnotu
  kosmologické konstanty z teoretického hlediska?“ Záhada kosmologické konstanty se už nedala držet
  pod pokličkou. 
  
  Když v roce 1996 Michael Turner z Chicagské univerzity na princetonském setkání hájil
  kosmologickou konstantu proti Richardu Gottovi a Davidu Spergelovi a neměl snadnou práci.
  Pozorování svědčila v jeho prospěch, jenže kosmologická konstanta zůstávala pro jeho kolegy stále
  nestravitelnou. Byly s ní koncepční i estetické problémy. Kdyby měl Turner hájit Boží zásah, měl
  by to pravděpodobně snazší. Když debata skončila, vítězem byl vyhlášen standardní model s chladnou
  temnou hmotou a nulovou kosmologickou konstantou. John Peebles byl představením fascinován. V roce
  1996 se kosmologie změnila způsobem, který si nepředstavoval ve svých nejdivočejších snech. Byl to
  on, Jakov Zeldovič a Joe Silk, kdo byli osamělými pionýry teoretického zkoumání velkých struktur
  ve vesmíru. Peebles vypracoval techniky, které byly používány nejen v teorii, ale i při analýze
  pozorování. A nyní mladá generace teoretiků posouvala s velkou energií jeho teoretickou analýzu
  dopředu a dále ji rozvíjela. Zároveň astronomové mapovali nebe se stále větší přesností. 
  
  V této nové éře se však Peebles ocitl v nevděčné roli opozičníka na poli, které sám vytvořil.
  Nelíbil se mu přílišný zápal, se kterým jeho kolegové přijímali CDM, a stále se snažil hledat
  alternativy. Ale jak říkal jeho učitel Bob Dicke, dobrá pozorování přetrumfnou všecko. A
  přetrumfnutí se dočkali i Peebles a podpůrci CDM. 
  
  Jedním z hlavních vedoucích projektu družice COBE (Cosmic Background Explorer) byl George Smoot.
  Ten v roce 1992 prohlásil: „Jste-li věřící, tak toto je jako pohled do tváře Boží.“COBE byl
  družicový experiment, jehož cílem bylo podrobně mapovat reliktní záření, jež je pozůstatkem
  velkého třesku. Jedním z úkolů bylo zjišťovat změny jeho jasnosti při malých změnách směru
  pozorování a citovaný výrok se týkal vlnek na pozadí reliktního záření, malých nepravidelností,
  jejichž existenci předpověděli Peebles, Silk, Novikov a Sunjajev už o pětadvacet let dříve.
  Pátralo se po nich dlouho a obtížně. Čas ubíhal a zčeření hladiny reliktního záření se stále
  nedařilo pozorovat. Teoretici naštěstí zatím snížili odhad, jak musí být tyto nepravidelnosti
  veliké. Úlevu přinesla v roce 1992 až měření z družice COBE, která nesla radiometr postavený podle
  návrhu Roberta Dickeho, jenž dokázal měřit drobné rozdíly v intenzitě záření. Smoot za tato
  pozorování získal později Nobelovu cenu. 
  
  Vrásky, které viděla COBE, nebyly ještě pozorovány s dostatečným rozlišením. Přitom šlo o
  neobyčejně důležité údaje. Jak vyplývalo z prací Zeldoviče, Novikova a Peeblese, bohatá tapiserie
  teplejších a chladnějších skvrn v reliktním světle mapovala geometrii prostoru. Byla-li geometrie
  prostoru skutečně euklidovská, rozměry skvrn by měly představovat asi jeden úhlový stupeň. A
  měření geometrie prostoru bylo podle obecné teorie relativity cestou k určení energie celého
  vesmíru. Byly třeba další jemnější experimenty. Řada skupin po celém světě vyvinula přístroje,
  které dokázaly měřit reliktní záření přesněji a s větším rozlišením. Bylo to, jako kdyby se parta
  neohrožených výzkumníků vydala mapovat právě objevený kontinent. Když se na zlomu tisíciletí
  porovnaly všechny výsledky, ukázalo se, že horké a chladné skvrnky na obloze mají skutečně úhlový
  průměr kolem jednoho stupně, a že tedy prostorová geometrie vesmíru je plochá. Výsledek souhlasil
  s předpovědí inflačního modelu a výpočty velkých vesmírných struktur, jež plynuly z CDM a
  existence kosmologické konstanty. 
  
  Konečný triumf kosmologické konstanty však nepřišel z Peeblesem milovaných velkých struktur, nýbrž
  z pozorování supernov explodujících ve vzdáleném vesmíru. První náznak se objevil v roce 1998 na
  výročním zasedání Americké astronomické unie. Tým astronomů ze západního pobřeží, který pracoval
  na takzvaném „Supernovovém kosmologickém projektu“, oznámil, že gravitační přitažlivost atomární a
  temné hmoty nedokáže brzdit vesmírnou expanzi. Vědci pracující na tomto projektu totiž zjistili,
  že expanze vesmíru se nejen nezpomaluje, nýbrž dokonce zrychluje. To znamenalo, že vesmír je buď
  mnohem prázdnější, než se soudilo, nebo že kosmologická konstanta má kladnou hodnotu a působí
  odpudivou silou, která vesmírnou expanzi zrychluje. Projekt se vlastně snažil o totéž, co dělali
  Hubble a Humason koncem dvacátých let minulého století - snažil se měřit červený posun vzdálených
  objektů. Jenže místo na galaxie se pozorovatelé nyní dívali na individuální supernovy, na hvězdy
  vybuchující s takovou silou, že jejich záření vycházející z malé oblasti převyšuje záření celé
  galaxie. Tyto supernovy jsou viditelné i z velikých vzdáleností, mnohem větších, než mohli
  pozorovat Hubble s Humasonem. Bylo to sice v duchu těchto dvou průkopníků, teď už to však nebyla
  práce dvou lidí, nýbrž veliká operace několika týmů obsluhujících pozemské dalekohledy a Hubbleův
  kosmický teleskop. Měření byla obtížná a dovést je k dokonalosti trvalo více než deset let. 
  
  Kromě tohoto projektu běžel i nezávislý projekt nazvaný „Hledání supernov s velkým Z“, to znamená
  s velikým červeným posunem. Jeho výsledky byly obdobné: expanze vesmíru se zrychluje, což znamená
  kladnou kosmologickou konstantu. 
  
  Oba týmy projevovaly váhavost, když měly zveřejnit, co jejich měření ukazují. Jejich sdělení na
  konferenci Americké astronomické unie byla velmi opatrná, možná až příliš. Skutečný důsledek
  jejich výsledků vedl k bohatým kuloárovým diskusím a vzbudil i pozornost tisku. Den po oznámení
  výsledků se ve „Washington Post“ objevil článek, kde se uvádělo: „Tato měření též vdechla nový
  život teoriím, podle kterých existuje takzvaná kosmologická konstanta.“ O pár týdnů později šel
  časopis Science ještě dále, když přinesl článek s názvem „Vybuchující hvězdy ukazují na existenci
  odpudivé síly“. Vedoucí „Supernovovém kosmologickém projektu“ Saul Perlmutter však ve svém článku
  brzdil přílišné nadšení a tvrdil, že je třeba vykonat ještě mnoho práce. 
  
  Ale jen o měsíc později tým projektu „Hledání supernov s velkým Z“vystoupil odvážněji. Uvedl, že
  nestačí, aby ve vesmíru bylo méně atomů a temné hmoty, že musí být naplněn něčím, co způsobuje
  jeho urychlování. Televize na celém světě žádaly členy týmu, aby své ohromující výsledky
  vysvětlili široké veřejnosti. Stanice CNN oznámila, že „vědci byli šokováni možností, že expanze
  vesmíru se zrychluje“ a v New York Times citovali vedoucího týmu Briana Schmidta: „Moje vlastní
  reakce je někde mezi okouzlením a hrůzou. Okouzlen jsem proto, že takový výsledek jsem nečekal, a
  hrůza pochází z toho, že většina astronomů mu nebude věřit - tak jako já budou k neočekávanému
  závěru extrémně skeptičtí.“ Druhý tým rychle vydal podobné prohlášení. Nyní to už bylo oficiální -
  „lambda“existuje. Hlavní osobnosti obou týmů, Saul Perlmutter, Brian Schmidt a Adam Riess, byly za
  tento objev poctěny Nobelovou cenou za rok 2011. Po léta, či přesněji po desetiletí panovala velká
  nejistota o tom, jak vesmír vypadá, jak je starý, jaké jsou jeho hlavní stavební kameny. Různé
  návrhy měly svá pro i proti, a kosmologie tak byla spíše záležitostí estetiky než vědou, záleželo
  na osobním vkusu, jaký obraz vesmíru přijímáme. Nyní však zvítězil její nejproblematičtější aspekt
  - kosmologická konstanta. Během několika měsíců byl na světě „model shody“, nebo prostě „Lambda
  CDM“. Tento nový model obsahoval koktejl namíchaný z atomů, chladné temné hmoty a kosmologické
  konstanty. Byl to model, na který po léta ukazovaly simulace velkých vesmírných struktur, jen ho
  skoro nikdo nechtěl přijmout. I Peebles, přes svou obecnou nechuť sledovat hlavní proud, byl
  okouzlen tím, jak vše teď zapadalo dohromady. Ale jak říkal jeho učitel Dicke, byly to výsledky
  pozorování, které tuto změnu způsobily. Musel tak prohlásit: „Nejlepším vysvětlením výsledků
  měření je kosmologická konstanta. Anebo něco, co jako kosmologická konstanta vypadá.“
  
  Když Jim Peebles skončil s výukou v Princetonu a šel na odpočinek, trávil většinu času vycházkami
  a fotografováním přírody. Obdivoval krásu a někdy i podivnost ptáků, se kterými se na svých túrách
  setkával, a teď na to měl více času. Místo aby se zaměřoval na vzory, které galaxie vytvářejí na
  obloze, či jak se chovají jednotlivé galaxie, kochal se krásou lesů a hájů kolem sebe. Jeho
  pečlivý celoživotní pohled a důraz na detail mu umožnil přijmout transformaci kosmologie v tvrdou,
  přesnou vědu. Kosmologie - další aspekt obecné teorie relativity - uzrála a začala žít vlastním
  životem. Peeblesovo klidné a stálé úsilí postavilo studium struktury vesmíru ve velkých měřítkách
  pevně do centra fyziky a astrofyziky. Jeho nezávislost ho dovedla k bizarnímu modelu vesmíru, ve
  kterém 96 procent veškeré energie bylo ve formě neviditelných částic a neviditelné substance,
  kombinaci temné hmoty a kosmologické konstanty. Ve srovnání s východisky, se kterými téměř před
  padesáti lety začínal, to byl surrealistický zvrat událostí. 
  
  Kosmologická konstanta teď už byla všeobecně uznávána. Základní problém s ní však přetrvá:
  obrovský rozpor mezi hodnotou, kterou vypočetl na základě kvantově mechanických úvah Zeldovič, a
  hodnotou získanou pozorováním. Obě veličiny se liší o vice než sto řádů. Dříve tento rozpor vedl
  kosmology k tomu, aby kosmologickou konstantu vůbec vypustili ze svých úvah. Teď ji však museli
  přijmout - prostě zde byla a nešlo se jí zbavit. Ve své učebnici relativistické astrofyziky vydané
  v roce 1967 Zeldovič s Novikovem napsali: „Džina vypuštěného z láhve … do ní lze vrátit jen velmi
  obtížně.“V této analogii je hluboká pravda. Když nyní obecně nastal posun k modelu „Lambda CDM“,
  musíme se s problémem kosmologické konstanty nějak vyrovnat. 
  
  Nebo možná nemusíme. Vznikl nový pokus, jak se kosmologické konstanty zbavit - naplnit prostor
  jakousi substancí, jež rozpíná prostor. Tato nová substance se chová velmi podobně jako
  kosmologická konstanta, začalo se jí však říkat „temná energie“. Do této hypotetické substance,
  jež měla propojit úspěšnou observační kosmologii s fyzikou částic a kvantovou teorií, se vkládaly
  velké naděje. Na problému začali hromadně pracovat staří i mladí kosmologové. Na jedné konferenci
  přednášející předložil v jedné přednášce více než sto modelů temné energie; to je důkaz kreativity
  nové generace kosmologů. Jenže zavedení temné energie neřeší problém, na který upozornil Zeldovič,
  že podle současné kvantové teorie by energie vakua měla být mnohem větší, než je z empirického
  hlediska přijatelné. Diskrepanci se prostě nevěnovala pozornost. Vyřešení tohoto problému by
  znamenalo revoluci v kvantové teorii gravitace, 
  
  Vzestup fyzikální kosmologie v posledních čtyřiceti letech podstatně změnil náš pohled na
  prostoročas a vesmír. Tím, že využívali obecné teorie relativity v těch největších vesmírných
  měřítkách a teoretické důsledky pečlivě propojovali s pozorováním, Jim Peebles a jeho současníci
  otevřeli zcela nové okno pro pohled na realitu. Jejich práce, podpořená obrovskými úspěchy v
  mapování rozložení galaxií a reliktního záření, odhalila bizarní vesmír, plný exotických
  substancí, kterým dodnes moc nerozumíme. Současná věda o vesmíru je velmi vzdálená od té
  disciplíny s dvěma či třemi čísly, se kterou svého času vyslovoval Peebles svou nespokojenost.
  Moderní kosmologie je jedním z největších úspěchů Einsteinovy obecné teorie relativity a celé
  moderní vědy. Zodpověděla mnoho otázek, položila však také řadu nových.

\section{Konec prostoročasu}\label{feyIchIIIsecXIII}
  Relativistovi Stephenu Hawkingovi bylo v roce 1979 nabídnuto jedno z nejprestižnějších postavení v
  Cambridgi, lucasovská profesura matematické fyziky. Kdysi ji zastával i Isaac Newton a v minulém
  století Paul Dirac. Hawking byl jmenován lucasovským profesorem, když mu ještě nebylo ani čtyřicet
  let, tuto poctu si však plně zasloužil. V předchozích dvaceti letech rozhodujícím způsobem přispěl
  k otázce zrodu vesmíru a k fyzice černých děr. Jeho vrcholným úspěchem byl důkaz, že černé díry
  mají teplotu, entropii a září, takže se nakonec zcela vypaří. Hawkingovo záření bylo pro fyzikální
  svět velkým překvapením. O černých dírách se do té doby předpokládalo, že jsou jednoduché a
  opravdu černé. Hawking vycházel z Bekensteinovy hypotézy a ukázal, že černé díry musí v sobě nést
  velké množství neuspořádanosti a že tato neuspořádanost je spojena s velikostí jejich povrchu,
  nikoli s jejich objemem, jak je tomu u jiných fyzikálních systémů. Všechny fyziky však tížila
  otázka: jak je entropie v černé díře uložená? A všichni se domnívali, že tato otázka má hlubokou
  odpověď související s kvantovou gravitací. 
  
  V té době se zdálo, že hledání kvantové gravitace stagnuje. V době Oxfordského symposia v roce
  1975, na kterém Hawking oznámil svůj objev záření černých děr, bylo zřejmé, že obecná teorie
  relativity není renormalizovatelná, že je zanesená neodstranitelnými nekonečny. Obecná teorie
  relativity se lišila od ostatních teorií základních interakcí natolik radikálně, že konvenční
  metody, kterými byl vybudován standardní model částic, pro ni selhávaly. Při kvantování gravitace
  bude třeba použít něco dramaticky odlišného. Ke konci sedmdesátých let se na poli kvantové
  gravitace objevila řada nových myšlenek a mezi různými tábory badatelů v tomto oboru se vytvořily
  hluboké příkopy. Vzájemně si konkurující tábory umíněně lpěly na svých pravidlech, jak se má
  obecná relativita kvantovat, a dogmaticky odmítaly přijmout jiný přístup. Společenství fyziků
  pracujících v kvantové gravitaci se rozpadlo na jednotlivé klany, mezi kterými vzplanula opravdová
  válka. Z tohoto turbulentního prostředí plného popudlivosti se však přece jen zrodilo obecné
  přesvědčení, že je třeba opustit starou představu prostoročasového kontinua a najít nějaký nový
  pohled na realitu. 
  
  Stephen Hawking vždy pronášel odvážná radikální tvrzení, často vizionářská, někdy ale zlomyslná.
  Když přijímal místo lucasovského profesora, pronesl inaugurační řeč s názvem „Je v dohledu konec
  teoretické fyziky?“, ve které předložil svůj názor na budoucí rozvoj tohoto oboru. Vyslovil se, že
  „cíle teoretické fyziky se dosáhne v nedaleké budoucnosti, řekněme do konce století“. Hawking se
  domníval, že sjednocení fyzikálních zákonů a kvantová gravitace jsou na dosah. Pro své odvážné
  prohlášení měl však dobrý důvod díky úspěchům, kterých dosáhla nová myšlenka zvaná supersymetrie.
  Supersymetrií se rozumí hlubší symetrie v přírodě, která propojuje všechny částice i interakce ve
  vesmíru. O každé elementární částici se předpokládá, že má svou supersymetrickou partnerku.
  Částice se dělí na fermiony a bosony. Supersymetrie přiřazuje fermionu bratrský boson a naopak.
  Teorie, která v roce 1976 posunula myšlenku supersymetrie o krok dále, zrcadlila sám prostoročas
  na sebe sama a zaváděla supergravitaci. V době Hawkingovy přednášky se zdálo, že supergravitace je
  to, v co všichni doufají: životaschopný kandidát na kvantovou teorii gravitace. Jenže
  supergravitace nebyla v praxi dobře použitelná. Rozšiřovala prostoročas do dalších dimenzí a
  vyžadovala řešení podstatně komplikovanějších rovnic, než byly ty původní Einsteinovy. Vypočítat
  něco v jejím rámci vyžadovalo měsíce práce a výsledky byly zamořeny nekonečny a novými částicemi,
  jež se v přírodě nedaly nalézt. Malá skupina nadšenců se s ní sice dále zabývala, ale jako
  kvantová teorie gravitace byla obecně záhy opuštěna. Po konci teoretické fyziky se Hawking musel
  poohlédnout jinde. 
  
  I když byl Hawking ve své inaugurační řeči tak optimistický, trápil ho podivný problém, o který
  klopýtl během své práce o záření černých děr. Tento problém se hrozivě vznášel nad všemi pokusy o
  kvantování gravitace a bořil řadu základních fyzikálních předpokladů. Hawking se chtěl o něj
  podělit s vybranými kolegy. Užil k tomu setkání v sídle bohatého podnikatele Wernera Erharda.
  
  Erhard získal peníze i slávu kurzy sebezdokonalování, které pořádal po celých Spojených státech.
  On sám byl ovlivněn mnoha učeními i různými vírami od zen-buddhismu po scientologii, měl však
  velký zájem o fyziku. Každý rok organizoval sérii přednášek o fyzice a zval na ně známé vědce jako
  Hawkinga nebo Feynmana. Když byl v roce 1981 Hawking pozván k přednesení přednášky, rozhodl se
  mluvit o bizarním výsledku, který poprvé publikoval v roce 1976 a který ho od té doby ho
  nepřestával znepokojovat. Přednášku nazvanou „Informační paradox černých děr“vlastně proslovil
  jeden z jeho doktorandů - sám Hawking už nemohl hovořit. 
  
  Jejím tématem byl fyzikální předpoklad, že máme-li o systému plnou informaci, jsme schopni
  rekonstruovat celou jeho minulost. Vezměme si třeba míč, který nám proletí kolem hlavy. Víme-li,
  jak rychle letí a jaký je směr jeho pohybu, umíme vypočítat, odkud přiletěl a kudy přesně se
  pohyboval. Nebo si představme nádobu naplněnou plynem. Kdybychom mohli změřit rychlosti a polohy
  všech molekul, mohli bychom zjistit, kde každá z nich byla v minulosti. Realističtější situace
  jsou často mnohem složitější. Vezměme jako příklad laptop, na kterém píši tuto kapitolu. Abych byl
  schopen přesně rekonstruovat, jak tento laptop vznikl, potřeboval bych k tomu spoustu informací
  nejen o něm samotném, ale i o okolním světě. Fyzikální zákony mi však zaručují, že je to alespoň v
  principu možné. Vyšší stupeň komplikací přináší kvantová mechanika. Abychom mohli určit minulost
  nějakého kvantového stavu, potřebujeme k tomu ještě více informací. Ale v zákonech zachování
  kvantové fyziky je zakotveno zachování informace. Informace je jádro předpověditelnosti a fyzici
  vždy vycházeli ze základního předpokladu, že informace je nezničitelná. 
  
  Tedy - informace se nezničí do té doby, dokud se nesetká s černou dírou. Když například do černé
  díry vhodíte výtisk této knihy, úplně vám zmizí z dohledu. Hmotnost a povrch černé díry o něco
  vzrostou a černá díra bude vyzařovat nějaké světlo. Nakonec se úplně vypaří a zmizí, přičemž po ní
  zbude jen lázeň záření, zbaveného všech charakteristických rysů. Jestliže do černé díry vhodíte
  místo knihy balon naplněný vzduchem, jenž má stejně velkou hmotnost jako kniha, konečným výsledkem
  bude zase lázeň záření, stejná, jako když jste tam hodili knihu. Konečný produkt bude v obou
  případech přesně stejný, i když oba procesy začaly značně rozdílně. Ve skutečnosti nemusíme čekat
  až do chvíle, kdy se černá díra zcela vypaří. Černé díry se stejnou hmotností vypadají stejně po
  celou dobu, co září, a je zcela nemožné zjistit, jestli do jedné spadla kniha a do druhé balon
  naplněný vzduchem. Informace zmizí. Hawking v této skutečnosti viděl paradox. Jestliže existují
  černé díry, budou zářit a vypařovat se, to ale znamená, že vesmír je nepředpověditelný. Představa,
  že mezi příčinou a následkem je přímé spojení, která panovala nejen v Newtonově a Einsteinově
  fyzice, ale i v kvantové mechanice, se musí zahodit. 
  
  Hawkingův argument jeho kolegy šokoval. Řada jich prostě odmítala jeho závěr přijmout. Jestliže se
  totiž informace může ztrácet, pak je to konec fyziky jako prediktivní vědy. Jediný způsob, jak by
  se ztrátě informace dalo předejít, bylo předpokládat, že černé díry jsou bohatší, než jsme si
  mysleli. Snad v nich vládne nějaký nový typ mikrofyziky, který dovoluje informace nějak uložit a
  zároveň zaručuje, že se nakonec uvolní do vnějšího světa. Tuto možnost mohla realizovat jen
  kvantová teorie gravitace. 
  
  V roce 1967 vyhlásil Bryce DeWitt dva navzájem si odporující manifesty o kvantování obecné
  relativity. Už vstoupil do svých čtyřicátých let a skoro dvacet let života strávil bezvýsledným
  řešením problému kvantové gravitace. Nyní měl v ruce tři ruko- pisy, které sumarizovaly jeho
  úsilí. Začalo se jim říkat „trilogie“a pro mnoho fyziků v oboru se staly posvěcenými články víry o
  kvantové gravitaci. DeWitt pečlivě shrnul veškeré výsledky svých předchůdců, jeho rukopisy však
  jasně vymezily možnosti zasnoubení obecné teorie relativity s kvantovou teorií. V jádru
  konzistentním způsobem sumarizoval výsledky své i svých předchůdců. 
  
  První článek trilogie popisoval takzvaný kanonický přístup. Ten navrhli již dříve jiní fyzikové,
  například Peter Bergmann, Paul Dirac, Charles Misner a John Wheeler. Právě tak jako v klasické
  obecné relativitě, i zde hraje ústřední roli geometrie. Kanonický formalismus štěpí prostoročas na
  dvě části: prostor a čas. Obecná teorie relativity přestává být teorií prostoročasu jako nedílného
  celku a stává se teorií vývoje prostorové geometrie v čase. DeWitt ukázal, že kvantovou fyziku sem
  lze zavést pomocí rovnice, která dovoluje počítat pravděpodobnost určité geometrie prostoru.
  Podobně jako Schrödinger objevil rovnici pro vlnovou funkci obvyklého kvantového systému, DeWitt
  napsal rovnici pro vlnovou funkci prostorové geometrie. 
  
  DeWitt kanonický přístup záhy zavrhl, nadšeně se ho však chopil John Wheeler. Oba se setkali na
  letišti Raleigh-Durham v Severní Karolině a DeWitt mu ukázal svou rovnici. DeWitt na setkání
  vzpomíná: „Wheeler byl velice vzrušený a o rovnici začal při každé příležitosti přednášet.“Mnoho
  let ji pak DeWitt nazýval Wheelerova rovnice, zatímco Wheeler jí říkal DeWittova rovnice a všichni
  ostatní pak pro ni používali název Wheelerova-DeWittova. 
  
  DeWittovo srdce však patřilo druhé a třetí části trilogie, které se týkaly kovariantního přístupu
  ke kvantování gravitace. Zde se zcela zapomnělo na geometrii a gravitace vystupovala jako další
  síla, přenášená svou „částicí-poslíčkem“, gravitonem. Tento přístup se snažil kopírovat úspěchy
  kvantové elektrodynamiky a standardního modelu, vedl však k neodstranitelným nekonečnům, jež tak
  dramaticky zabrzdila snahy o kvantování gravitace po Oxfordském symposiu v roce 1974. 
  
  \textbf{Kanonický a kovariantní přístup ke kvantování} ztělesňovaly dvě rozdílné filosofie a o
  kvantování gravitace se pokoušely v různém duchu. V srdci kanonického kvantování ležela geometrie,
  zatímco v kovariantním přístupu vystupovaly virtuální částice, pole a snaha o sjednocení. Tyto dva
  přístupy postavily jejich příznivce proti sobě. 
  
  Prapor kovariantního přístupu nakonec převzal radikálně nový přístup k sjednocení, nazvaný teorie
  strun. Teorie strun se vynořila koncem šedesátých let minulého století jako pokus vysvětlit cho-
  vání celé zoo nových částic, jež se objevovaly na velkých urychlovačích. Její základní myšlenkou
  bylo popisovat tyto částice ne jako bodové objekty, nýbrž pomocí mikroskopických jednorozměrných
  objektů, kousků strun. Částice s různými hmotnostmi nejsou nic jiného, než různé vibrace
  nepatrných strunek, jež se pohybují v prostoru. Trik spočívá v tom, že jeden jediný objekt, jediná
  struna, může popsat všechny částice. Čím vice se struna chvěje, tím větší energii nese a tím těžší
  jsou částice, které popisuje. Teorie strun byla určitým způsobem unifikací, ale zvládnutou zcela
  jiným způsobem, než jaké se do té doby používaly. 
  
  Myšlenka strun jako fundamentálních objektů byla fascinující, ale hned na samém počátku narazila
  na velké obtíže. Jakmile jste z této teorie chtěli vypočítat nějaké fyzikální předpovědi, objevila
  se nekonečná čísla, která nešlo odstranit podobnou renormalizací, jaká se užívá v kvantové
  elektrodynamice a standardním modelu. Navíc se v této teorii objevovaly částice, jež svými
  vlastnostmi připomínaly gravitony, tedy částice, jež měly způsobovat gravitační sílu. Ty by
  samozřejmě měly své místo v kvantové teorii gravitace, neměly však co dělat v teorii, jejímž cílem
  bylo popsat nové experimentálně nalezené exotické částice. 
  
  A tak po počátečním zájmu přišlo rozčarování a uprostřed šedesátých let teorii strun opustili
  téměř všichni fyzikové hlavního proudu. Jeden z mála jejích stoupenců, nositel Nobelovy ceny
  Murray Gell-Mann, o sobě říkal, že je něco jako „patron teorie strun“ a „konzervativec“. Vzpomíná,
  že podpořil ohrožené superstrunové teoretiky na Caltechu, a tak se tam mezi rokem 1972 a 1984
  vykonalo na této teorii hodně práce. 
  
  V roce 1984 jeden z ohrožených strunařů z Caltechu John Schwartz spojil své síly s mladým britským
  teoretikem z Londýna Michaelem Greenem. Společně přišli s návrhem, že teorie strun by mohla být
  užitečnější jako teorie kvantové gravitace. Ukázali, že teorie strun v deseti rozměrech by mohla
  obsáhnout kvantovou gravitaci, jestliže splní některá omezení a bude vyhovovat určitým symetriím.
  V příštím roce skupina částicových fyziků a relativistů, kterou tvořili Edward Witten z
  Princetonu, Philip Candelas z texaského Austinu a Andrew Strominger s Garym Horowitzem, oba ze
  Santa Barbary, pokročili ještě dále. Ukázali, že když šest dodatečných vesmírných dimenzí má
  speciální geometrii zvanou Calabiho-Yauova, pak rovnice teorie strun mají řešení, jež vypadá
  přesně jako supersymetrická verze standardního modelu. Skutečný standardní model měl být jenom o
  krok dále. 
  
  Koncem osmdesátých let minulého století se teorie strun stala idolem. Zdálo se, že má pro každého
  něco. Její matematika byla nová a vzrušující, podobně jako Einsteinovi musela připadat
  neeuklidovská geometrie, když s její pomocí formuloval obecnou teorii relativity. Matematici
  používali těch nejmodernějších nástrojů, nejen neeuklidovské geometrie, ale i teorie čísel a
  topologie, aby zjistili, co všechno může teorie strun nabídnout. Ke konci dvacátého století byla
  teorie strun na vrcholu slávy. V roce 1995 na výroční konferenci o teorii strun, která se konala v
  Kalifornii, přišel Ed Witten s převratným výsledkem, že všechny strunové modely, jež se v
  předchozích letech objevily, jsou navzájem propojeny a že jsou ve skutečnosti jen různými aspekty
  hlubší a bohatší teorie, kterou nazval M-teorie. Jeho slovy: „M znamená magická, mysterium nebo
  membrána - jak kdo chce.“ Wittenova teorie totiž neobsahuje jen jednorozměrné struny, nýbrž i
  vícerozměrné objekty podobné membránám, jež se vznášejí ve vícerozměrném vesmíru. Ujal se pro ně
  kratší termín „brány“. 
  
  Přes euforii a sebevědomí strunařů se však teorie strun nedo- kázala zbavit jednoho základního
  problému - nebyla jen jedna. Bylo možno vytvořit mnoho verzí a když jste vybrali jednu z nich,
  pořád existovalo mnoho řešení, jež mohla být modelem reálného světa. Hrubý odhad vede k 10500
  řešení pro každou verzi teorie strun. Tomuto vpravdě obscénnímu panoramatu možných vesmírů se říká
  „krajina“. Teorie strun není schopná udělat jednoznačnou předpověď, která část „krajiny“ odpovídá
  našemu světu. Řada prominentních skeptiků tvrdila, že teorie strun slibovala příliš mnoho a
  přinesla příliš málo. Richard Feynman krátce před smrtí v roce 1987 poskytl interview, ve kterém
  prohlásil: „Domnívám se, že všechen ten humbuk kolem superstrun je bláznivý a že je to chybná
  cesta. Nelíbí se mi, že se tam nic konkrétního nepočítá. Nelíbí se mi, že myšlenky se
  experimentálně neověřují. Nelíbí se mi, že když se přijde do sporu s experimentem, ukmochtí se
  nějaké řešení … to není v pořádku.“
  
  Podobný ohlas měl Feynmanův pohled u Sheldona Glashowa, který spolu se Stevenem Weinbergem a
  Abdusem Salamem vytvořil mimořádně úspěšný standardní model. Napsal, že „superstrunoví teoretici
  stále ještě neukázali, jak jejich teorie skutečně funguje. Nejsou schopni dokázat, že standardní
  teorie je logickým důsledkem teorie strun. Nejsou si ani jisti, zda jejich teorie popisuje takové
  objekty, jako jsou protony a elektrony.“
  
  Daniel Friedan, prominentní teoretik strun během „první strunové revoluce“ v osmdesátých létech
  minulého století, nedostatky teorie superstrun přiznává: „Dlouhodobým problémem teorie strun je,
  že zcela selhává jako vysvětlení fyziky ve větších rozměrech. … Teorie strun nedává žádné konečné
  vysvětlení našich současných znalostí reálného světa a neumí dělat žádné určité předpovědi.
  Spolehlivost teorie strun nemůže být odhadnuta, tím méně vypočtena. Nemá žádnou důvěryhodnost jako
  kandidát na finální teorii.“Skeptici zůstávali ale v menšině a byli snadno přehlušeni. Kdybyste se
  v osmdesátých a devadesátých letech chtěli začít věnovat kvantové teorii gravitace, nikdo by vám
  nezazlíval názor, že kovariantní přístup jednoznačně zvítězil a jedinou cestu skýtá teorie strun.
  
  Jedna věc na teorii strun obzvláště dráždila fyziky pracující v obecné teorii relativity: v teorii
  strun, tak jako vůbec v kovariantním přístupu, se zcela vytratil sám základ obecné relativity -
  geometrie prostoročasu. Gravitace v ní vystupuje na stejné úrovni jako ostatní tři síly sjednocené
  ve standardním modelu a kvantuje se obdobným způsobem. Malá skupina relativistů proto dávala
  přednost kanonickému přístupu, který propagoval Wheeler a který odmítl DeWitt. V jeho rámci by
  mělo být možné kvantovat samotnou geometrii. V polovině osmdesátých let nalezl v tomto duchu cestu
  vpřed indický fyzik Abhay Ashtekar. Tento nadšený relativista pracující na Syracuské univerzitě
  nalezl vtipný způsob, jak Einsteinovy rovnice přepsat, aby zmizela řada nepříjemných nelinearit a
  aby obecná relativita vypadala mnohem jednodušší. Ashtekarův trik dal Einsteinovým rovnicím
  neočekávanou formu a otevřel alternativní cestu ke zkoumání jejich kvantové povahy. 
  
  Podobně jako Bryce DeWitt, Lee Smolin propadl lásce ke kvantové gravitaci v okamžiku, kdy v
  sedmdesátých letech přišel na Harvard, aby zde získal titul PhD. Jeho školitel Sidney Coleman ho
  nechal ohmatat kvantovou gravitaci tím, že mu dovolil spolupracovat se Stanleyem Deserem z
  nedaleké Brandeisovy univerzity. Jako doktorand velkého úspěchu v této oblasti nedosáhl, nadšení
  pro problematiku mu však zůstalo. Později, když byl na učitelském místě na Yaleově univerzitě v
  New Havenu, si uvědomil, jak Ashtekarova formulace celý problém zjednodušuje. Začal na něm
  pracovat s Theodorem Jacobsonem, bývalým studentem Cécile DeWitt-Moretteové z texaské
  relativistické skupiny. Smolin a Jacobson předně zjistili, že místo aby sledovali geometrii v
  izolovaných bodech, je snazší sledovat vývoj geometrie souboru bodů a zaměřit se na geometrii
  kousků prostoru v daném okamžiku. Stavebními kameny pro kvantovou gravitaci se staly smyčky, něco
  jako stužky v prostoru, jež se daly užít ke konstrukci řešení Wheelerovy-DeWittovy rovnice. Zdálo
  se, že se otvírá schůdná cesta ke zcela novému způsobu uvažování o gravitaci. Smyčky se mohou
  různě řadit a proplétat jako řetízky a mohou vytvářet složité tkaniny, připomínající drátěnou
  košili. Díváme-li se na nějakou tkaninu z dálky, její struktura není patrná. Podobně se
  prostoročasové kontinuum Einsteinovy teorie objeví až při makroskopickém pohledu. Přístup, který
  formulovali Smolin a Jacobson, se nazývá smyčková gravitace. 
  
  Smolina podpořil v jeho hledání mladý italský ikonoklasta Carlo Rovelli, který si málem vylámal
  zuby na složité algebře kvantové gravitace. Rovelli byl duchem rebel. Během studií provozoval v
  Římě alternativní rozhlasovou stanici, italské orgány ho sledovaly pro jeho politické názory a
  riskoval uvěznění pro odmítání branné povinnosti. Alternativní pohledy byla jeho doména. Spolu se
  Smolinem dovedli smyčkový obraz ještě dále, zkoumali, jak se smyčky dají uspořádávat, splétat a
  spojovat uzly. V polovině devadesátých let narazili na starou myšlenku Rogera Penrose jak
  popisovat kvantový systém pomocí jednoduchého matematického lešení, které Penrose nazýval spinová
  síť. Tato síť, podobná veliké konstrukci na šplhání na dětském hřišti, je soustavou spojů a
  vrcholů a každá její součást nese nějakou kvantovou vlastnost. Rovelli a Smolin ukázali, že tyto
  sítě dávají ještě lepší řešení Wheelerovy-DeWittovy rovnice. Přesto ale tyto podivné sítě neměly v
  sobě nic, co by připomínalo intuitivní obraz prostoru a času, který by přijal relativista, jenž má
  nějakou sebeúctu. 
  
  Rovelliho a Smolinovy spinové sítě byly zcela novým pohledem na kvantovou gravitaci. V jejich
  pojetí na kvantové úrovni prostor neexistoval - byl atomizován či rozbit na molekuly, jako voda
  při mikroskopickém pohledu. Voda vypadá při pohledu v lidských měřítkách spojitě, ve skutečnosti
  je však složená z molekul, tedy obláčků protonů, neutronů a elektronů, které se vznášejí v
  prostoru, volně k sobě vázány elektromagnetickou silou. Stejně i prostor se zdá podle Rovelliho a
  Smolina hladký, ale tak by nevypadal pod extrémně silným mikroskopem. Kdybychom mohli rozlišit
  vzdálenosti řádu biliontiny biliontiny centimetru, žádný spojitý prostor bychom neviděli, jen síť.
  
  Smyčková gravitace tak byla odvážným soupeřem teorie strun v kvantování gravitace. Ona i její
  následníci nabízeli kanonickou alternativu kovariantního přístupu. Její autoři se nesnažili o
  sjednocení všech sil; tím, že brali geometrii jako výchozí bod, zachovávali něco z krásy původní
  Einsteinovy myšlenky o obecné relativitě. Určitou ironií je, že v konečné teorii opustili
  prostoročas jako fundamentální koncept. 
  
  V roce 2004, krátce před svou smrtí, proslovil DeWitt přednášku, ve které vyzdvihoval, jak daleko
  se kvantová gravitace dostala. „Podíváme-li se na teorii strun, vidíme, jak se za padesát let naše
  představy dokonale změnily. Na gravitaci jsme se kdysi dívali jako na neškodné pozadí, jež je pro
  kvantovou teorii pole zcela irelevantní. Dnes hraje gravitace hlavní úlohu: její existence
  ospravedlňuje teorii strun! Anglické přísloví říká, že z prasečího ucha neuděláš hedvábnou
  taštičku. Nikdo ji nebral jako fundamentální teorii … až počátkem šedesátých let se vše
  převrátilo. Teorie strun najednou nutně potřebovala gravitaci stejně jako řadu dalších objektů,
  které v ní jsou, ale i nemusí být, obsaženy. Z tohoto hlediska je teorie strun tou hedvábnou
  taštičkou.“
  
  DeWitt v teorii strun nikdy nepracoval, ale bylo jasné, kam se kloní jeho sympatie. Ke kanonickému
  přístupu byl mnohem skeptičtější. Wheelerovu-DeWittovu rovnici nenáviděl, přestože byl jejím
  autorem. Domníval se, že „patří do odpadkového koše historie“, protože „narušuje ducha relativity
  “. Jasně se vyslovil, že „Wheelerova-DeWittova rovnice je chybná … je chybné ji užívat jako
  definici kvantové gravitace nebo jako základ pro rafinovanou detailní analýzu“. Práci Abhaye
  Ashtekara označil za „elegantní“, ale říkal, že „ač dosáhla několika zřejmě důležitých výsledků o
  takzvané ,kvantové pěně‘, zdá se mi celkově nepodstatná“. DeWittova antipatie odrážela ducha
  panujícího v teoretické fyzice, který pasoval teorii strun na vítěze. 
  
  Strunoví teoretici se hrdě hlásí k tomu, co pokládají za svůj úspěch. Mike Duff prohlašuje: „V
  teorii strun a M-teorii jsme dosáhli obrovského pokroku. … A je to jediný pokus o sjednocení.“Řada
  strunových teoretiků se domnívá, že supersymetrie a dodatečné dimenze budou záhy experimentálně
  objeveny a že struny představují jediný možný přístup ke sjednocení. I Stephen Hawking prohlásil,
  že „teorie strun je jediný kandidát na úplnou teorii vesmíru.“ Když se někdo zeptá na konkurenční
  kanonický přístup, který mnozí vidí jako právoplatného dědice Wheelerovy filosofie kvantování
  geometrie, Duff oponenty obviňuje, že pokládají „kvantovou gravitaci“ za synonymum „smyčkové
  kvantové gravitace“. A Duff není zdaleka sám. Oddaný člen strunařského tábora Philip Candelas
  prohlašuje na adresu odpůrců: „Neumějí spočítat ani účinek gravitonu. Jak tedy mohou tvrdit, že
  mají pravdu?“
  
  Na počátku nového tisíciletí se antagonismus mezi oběma tábory hledačů kvantové gravitace ještě
  vyhrotil a dostal se na veřejnost. Po léta se v populárních fyzikálních časopisech i na blogu
  objevovaly články odmítající hegemonii teorie strun. Kolem roku 2006 vyšly dvě knihy, které
  dokonce tvrdily, že teorie strun ohrožuje budoucnost fyziky. Jejich autory byli Lee Smolin, jeden
  z proroků smyčkové teorie gravitace, a Peter Woit, matematický fyzik z Kolumbie. Oba autoři
  tvrdili, že ovlivnitelní mladí fyzici byli přemlouváni pracovat právě na teorii strun a že tato
  práce nepřinesla za třicet let žádný hmatatelný výsledek. Podle nich na akademických pracovištích
  dominovali strunaři a ti zase přijímali na svá pracoviště další strunaře, čímž vylučovali z
  výzkumu lidi, kteří se nedrželi strunařské doktríny. Jak psal Smolin v roce 2005: „Řada fyziků je
  frustrována tím, že tato komunita, jež se stylizuje do dominantního postavení - a na mnoha místech
  v USA skutečně dominantní je - se vůbec nezajímá o dobrou práci v jiných oblastech. Když
  organizujeme setkání o kvantové gravitaci, zveme představitele hlavních konkurenčních teorií
  včetně zastánců teorie strun. Ne, že bychom byli tak morální - prostě se to sluší. Tak tomu ale
  není na výročních setkáních o teorii strun.“Internet planul debatami a tábor strunařů, rozzlobený
  útoky, začal odpovídat. Vyjádření na fyzikálních webových stránkách byla doprovázená stovkami
  komentářů, zmatenou směsicí technických detailů, hlubších analýz i nepochopení. Každý měl svůj
  názor na věc. 
  
  Nepřátelství vůči teorii strun bylo cítit v roce 2011, když Michael Green, který nahradil Stephena
  Hawkinga na místě lucasovského profesora v Cambridgi, proslovil v Oxfordu veřejnou přednášku o
  teorii strun. Green stál spolu s Johnem Schwartzem u startu teorie strun v roce 1984 a já si
  vzpomínám na obrovský úspěch jeho přednášky počátkem devadesátých let v Londýně. Tehdy byla teorie
  strun na vzestupu. Na oxfordské přednášce byla však atmosféra mnohem chladnější. Většina dotazů se
  týkala technických detailů jeho prezentace, několik jich však bylo i trochu kousavých. Žádná
  veřejná přednáška o teorii strun se dnes neobejde bez dotazu: „Je tato teorie testovatelná?
  “Takovou otázku vždy položí někdo ze sympatizantů protistrunařského tábora. 
  
  Je brzy na to předvídat, jak se vyvine antagonismus mezi různými klany hledačů kvantové teorie
  gravitace. V tuto chvíli se zdá, že ti, kdo propagují nestrunový přístup ke kvantové gravitaci,
  mají potíže, ale bez potíží není ani tábor strunařů. 
  
  Pozoruhodným výsledkem celé debaty je, že o problémech kvantové gravitace je dnes informována řada
  lidí. Válka mezi kanonickým a kovariantním přístupem si našla cestu i na televizní obrazovku. V
  populárním seriálu Teorie velkého třesku se dva z protagonistů rozkmotří, protože se nemohou
  shodnout, v jakém přístupu ke kvantové gravitaci by měli vzdělávat své děti. Než se Leslie
  Winkleová rozzlobeně vyřítí z místnosti, křičí na Leonarda Hofstadtera: „To ruší naši dohodu!“
  
  Třicet let potom, co Stephen Hawking předpověděl konec teoretické fyziky a pak zveřejnil svůj
  informační paradox spojený s černými dírami, se nedospělo k souhlasu v otázce kvantové gravitace,
  tím méně k jednotné teorii všech interakcí. Nicméně přes rozladění nad hledáním kvantové gravitace
  v něčem přece jen existuje shoda. Vynořuje se radikálně nový a téměř všeobecně sdílený pohled na
  povahu prostoročasu. Jak teorie strun, tak smyčková gravitace a podobně i další pokusy o
  vybudování kvantové teorie gravitace se vzdávají pohledu na prostoročas jako na základní pojem.
  Tento pohled lze bezprostředně spojit s Hawkingovým objevem záření černých děr a může možná
  vyřešit paradox ztráty informace v černých dírách a otázku konce předpověditelnosti ve fyzice.
  Jedním z klíčových kroků pro řešení Hawkingova paradoxu bude asi porozumět tomu, jakým způsobem
  černé díry ukládají informace, které spolkly, a jak by je mohly uvolnit do vnějšího světa. To
  vyžaduje mít mnohem složitější obraz černé díry, než naivní obecně relativistický koncept
  horizontu a ničeho více. Trochu překvapivě se zdá, že jak smyčková gravitace, tak teorie strun, a
  dokonce i některé ezoteričtější přístupy vrhají na tento problém určité světlo. 
  
  Ve smyčkové kvantové gravitaci je prostoročas atomizován a existuje určitý minimální rozměr, pod
  kterým už nemá dobrý smysl hovořit o obsahu nějakého obrazce či objemu tělesa. Lee Smolin, Carlo
  Rovelli a Kirill Krasnov z Nottinghamské univerzity ukázali, jak tato teorie dovoluje rozdělit
  povrch černé díry na mikroskopické části, z nichž každá obsahuje jeden bit informace. Podle
  příznivců smyčkové gravitace součet těchto jednobitových informací dává dohromady správnou
  entropii černé díry. 
  
  Vyznavači strunové teorie vidí problém trochu jinak. Andrew Strominger a Cumrun Vafa z Harvardu
  ukázali, že v rámci M-teorie, pokládané za inkarnaci teorie strun, se dají odvodit přesné vztahy
  mezi entropií, informací a povrchem černé díry. Pro jisté typy černých děr se jim dokonce podařilo
  ukázat, jak soubor určitých „bran“ dovolí, aby černá díra shromáždila právě to správné množství
  informací. Brány dávají černým dírám tu správnou mikrostrukturu, která řeší Hawkingův paradox.
  Obecněji, zmínění autoři věří, že na černé díry lze pohlížet jako na klokotající směsici strun a
  bran, jejichž konce jsou na jejich horizontu. Tyto kousky bran a strun, jež se odrážejí od
  horizontu, v principu dovolují rekonstruovat veškerou informaci skrytou v černé díře a z ní i
  správnou entropii. 
  
  I když jsou přístupy smyčkové kvantové gravitace a teorie strun radikálně rozdílné, zdá se, že obě
  ukazují správnou cestu k řešení informačního paradoxu. Jestliže totiž informace skutečně sídlí na
  horizontu, mohou krmit Hawkingovo záření, které černá díra vysílá, a tak je postupně předávat do
  vnějšího světa. Toto záření způsobí postupný zánik, „vypaření“ černé díry. Během tohoto procesu se
  ale uvolní všechny informace, jež předtím černá díra nasála, takže se ve skutečnosti žádné
  neztratí. 
  
  Strunoví teoretici jsou ještě smělejší - tvrdí totiž, že to, co zjistili o Hawkingově záření
  černých děr, je dokonce daleko hlubší vlastností všech fyzikálních teorií. Podivnou vlastností
  černých děr se zdá být skutečnost, že množství informací, jež může černá díra uchovávat, je dáno
  velikostí jejího povrchu, nikoli objemu, jak bychom naivně soudili. Toho si všimli už v
  sedmdesátých letech i Bekenstein a Hawking. Je-li ale entropie dána povrchem, znamená to, že
  množství informací, jež se skrývá v nějaké oblasti prostoru, je vždy omezené. Když chceme popsat,
  co se fyzikálně děje v nějakém kousku prostoru, stačí udat, co se děje na povrchu, který danou
  oblast obklopuje. Je to podobné tomu, jak dvourozměrný hologram dokáže zakódovat informace
  třírozměrné scény. Je-li to ale pravda pro každou jednotlivou oblast prostoru, bude to platit i
  pro celý vesmír. A v takovém holografickém vesmíru nebudou důležité detaily toho, co se odehrává v
  jednotlivých prostoročasových bodech. Tato vlastnost je ohromující a Eda Wittena spolu s některými
  jeho strunařskými kolegy vedla k závěru, že prostoročas je „přibližný koncept, jenž se vynořuje na
  klasické úrovni“ a na kvantové úrovni nemá žádný smysl. A obecně se zdá, že u všech přístupů ke
  kvantové gravitaci prostoročas na té nejzákladnější úrovni prostě neexistuje. 
  
  Když John Wheeler se svými studenty začal v padesátých letech uvažovat o prostoročase a kvantové
  teorii, spekuloval o tom, že kdybychom se podívali na prostor skutečně velmi podrobně pomocí
  nepředstavitelně silného mikroskopu, uviděli bychom, že „geometrie v malém má charakter pěny“. Byl
  sice podivuhodně jasnozřivý, ale z hlediska toho, čemu dnes snad začínáme rozumět, se i on a jeho
  spolupracovníci jeví jako konzervativní. Ani „pěna“nevystihuje složitost základu, na kterém pojem
  prostoročasu stojí. 
  
  Zdá se, že jeden ze základních pojmů Einsteinovy veliké teorie, představa geometrie samotného
  prostoročasu, která je vlastně gravitací, si vyžádá určitou revizi. Kvanta nás posunují za hranice
  toho, co je obecná relativita schopná popsat, a pro tyto situace si budeme muset osvojit zcela
  nový způsob uvažování. 
  
  Jsou však i jiné náznaky, že se možná blížíme k hranicím toho, co nám Einsteinova teorie může
  povědět o prostoru, čase, a dokonce i o vesmíru jako celku. Jak zdůrazňoval John Wheeler, když je
  teorie dovedena do extrému, dozvíme se něco nového a udivujícího. V hraničních oblastech můžeme
  zahlédnout záblesk něčeho ještě většího a lepšího, co nakonec překoná veliký Einsteinův objev.

\section{Působivá extrapolace}\label{feyIchIIIsecXIV}
  Skončil jsem přednášku a teď stojím spolu s posluchači v atriu Astronomického ústavu univerzity v
  Cambridgi. Popíjíme levné víno z plastových pohárků, shromáždili jsme se v malém hloučku,
  přešlapujeme z nohy na nohu a snažíme se rozproudit konverzaci. Pozvali mě sem přednášet o
  modifikované gravitaci, o třídě teorií, které by mohly sesadit s trůnu obecnou teorii relativity a
  jež byly motivovány snahou vysvětlit některé kosmologické hádanky. Samotná přednáška proběhla
  hladce. Trochu jsem klopýtl, když jsem odmítl určitý komentář o temné hmotě, ale vše dopadlo
  dobře. Nikdo mi neřekl, že jsem v něčem neměl pravdu, nebyla ani nepříjemná diskuse a já se už
  chystal na zpáteční cestu do Oxfordu. Přispěchal ředitel ústavu George Efstathiou mávaje plastovým
  pohárkem jako zbraní a se zářícíma očima mi řekl: „Děkujeme, že jste přijel. Byla to zajímavá
  přednáška. Vlastně bych řekl, že to byla dobrá přednáška o nedobrém tématu.“Na jeho přátelské
  poplácání po rameni jsem odpověděl zdvořilým úsměvem. Nebylo to poprvé, kdy jsem se setkal s
  podobnou reakcí, a neudivilo mne to. Efstathiou hrál význačnou úlohu při detailní analýze, jak se
  temná hmota mohla podílet na formování velkých vesmírných struktur, a byl první, kdo tvrdil, že
  uspořádání galaxií v sobě skrývá důkaz existence kosmologické konstanty. On sám měl za sebou
  rychlý kariérní vzestup, byl úspěšný a důvěryhodný. 
  
  Pak pokračoval: „Když jsem převzal vedení ústavu, snažil jsem se z něj udělat zónu se zákazem
  modifikací gravitace a domnívám se, že jsem byl vcelku úspěšný.“ Rozzářil se, když skupinka kolem
  nás hleděla na podlahu. „Proč vy se tím zabýváte?“otázal se, aniž ve skutečnosti očekával odpověď.
  
  O několik měsíců dříve jsem se zúčastnil malého workshopu na Královské observatoři v Edinburghu,
  který byl celý věnován diskusím o alternativních teoriích gravitace. Jeho účastníky byla podivná
  směs astronomů, matematiků a fyziků a vládl zde zcela jiný duch. Každé vystoupení bylo odměněno
  bouřlivým potleskem, typickým pro skupiny, jež si vystačí samy. Také se tam nějak vznášel pocit,
  že každé vystoupení snad představuje zjevení nového božského fyzikálního zákona. Každý zde byl
  prorokem. Každý byl Einsteinem. Zdejší soudružská atmosféra mi připomněla můj krátký flirt s
  trockistickou organizací z mých mladých let, kdy jsem cítil blízkou pospolitost se svými kolegy
  agitátory a se všemi jsem se tiše shodoval na vnitřní zkaženosti světa. 
  
  V bojovném nadšení workshopu jsem se cítil velmi nepříjemně. Po mém vlastním příspěvku se mi z
  nadšeného potlesku dělalo téměř špatně a musel jsem opustit místnost. Bylo to ode mne
  nespravedlivé. Ti lidé v posluchárně po léta pracovali na alternativních teoriích gravitace a
  museli zápasit s hlavním proudem, jenž zbožně věřil v Einsteina. Byli mezi nimi vědci, jejichž
  články byly v časopisech opakovaně odmítnuty jen proto, že pracovali na tematice, jež nebyla v
  módě. Byli zvyklí mluvit před nepřátelsky naladěným posluchačstvem. A na tomto workshopu se
  setkali se sympatizujícíma ušima a mohli bez zábran diskutovat o svém cíli: svrhnout Einsteinovu
  obecnou teorii relativity. 
  
  Většina kolegů není příznivě nakloněna myšlence nějak měnit Einsteinovo největší dílo - říkají:
  „Když něco funguje, nech to na pokoji.“ Především ti, kteří se podíleli na slavné renesanci v
  šedesátých letech, kdy se obecná relativita vyhoupla ze svého skromného ustrnulého postavení a
  zazářila jako prostředek k vysvětlení bizarních jevů spojených se smrtí hvězd i osudem vesmíru,
  nevidí důvod, proč měnit tuto krásnou teorii. Tato generace astrofyziků pořád pociťuje magickou
  sílu Einsteinovy teorie. Hloubku jejich loajality jsem si uvědomil na jiném setkání, tentokráte na
  zasedání Královské astronomické společnosti v roce 2010. Ve stejných prostorách, kde Eddington
  oznámil výsledky své expedice za zatměním a kde se obul do Chandrasekhara proto, že vyvolává
  strašidlo gravitačního kolapsu, padl dotaz, kdo ze shromážděných astronomů a astrofyziků je
  přesvědčen o naprosté správnosti Einsteinovy teorie. Zvedlo se několik rukou a při bližším pohledu
  bylo vidět, že patří těm průkopníkům, kteří v šedesátých letech vtáhli obecnou teorii relativity
  do hlavního proudu fyzikálního bádání. Tato skupina je přesvědčená, že obecná teorie relativity je
  příliš zvláštní a příliš krásná, než aby se měla nějak měnit. 
  
  Nikdo nemůže popřít kolosální úspěšnost obecné teorie relativity během dvacátého století, přece
  jen si však zaslouží určitý čerstvý pohled. Věda má prospěch ze skutečnosti, že obecná teorie
  relativity je rozšířením Newtonovy teorie gravitace. I dnes je totiž newtonovská teorie zdravá a
  užitečná; úplně postačuje pro popis účinků gravitace v pozemských podmínkách, například v
  balistice, a s nepatrnými opravami i pro popis planet, a dokonce i evoluce galaxií. Newtonova
  teorie se hroutí až v extrémnějších situacích, kdy je gravitace dostatečně silná. Pak je nezbytné
  přijmout se všemi důsledky obecně relativistický popis. Možná ale čas uzrál k tomu, abychom
  učinili další krok a hledali teorii, která rozšiřuje obecnou teorii relativity na jiné extrémní
  situace. 
  
  Existují určité náznaky, že obecná teorie relativity má jisté problémy, je-li aplikována v
  situacích s velmi silným gravitačním působením, či naopak s působením velmi slabým. Problém
  manželského spojení obecné teorie relativity s kvantovou teorií může být v tom, že obě teorie se
  chovají v opravdu malých měřítkách trochu odlišně, ačkoli by tam měly spolu souhlasit. Obecně
  relativistická předpověď, že temná exotická hmota tvoří 96 procent veškeré hmoty ve vesmíru, může
  prostě znamenat, že tato teorie gravitace není v pořádku. Dnes, téměř sto let od chvíle, kdy
  Einstein předložil svou teorii, je možná dobrá doba pro revizi oblasti její použitelnosti.
  Historie je plná pokusů, jak obecnou teorii relativity modifikovat. Sám Einstein se téměř okamžitě
  po jejím objevu domníval, že obecná relativita je nedokončené dílo, že je součástí něčeho většího.
  Znova a znova se pokoušel vytvořit teorii sjednocující elektromagnetismus a gravitaci tak, aby
  obecnou relativitu rozšiřovala. I Arthur Eddington strávil poslední léta svého života prací na
  vlastní fundamentální teorii, jakémsi magickém spojení matematiky, numerologie a koincidencí, jež
  měla vysvětlit všechno od elektromagnetismu až po prostoročas. Eddingtonovo hledání fundamentální
  teorie bylo ovšem nešťastným podnikem, který pomalu ale jistě nahlodával jeho prestiž.
  
  Cambridgeský fyzik Paul Dirac se domníval, že Einsteinova teorie je dokonalým příkladem toho, jak
  má teorie vypadat. Ke konci života prohlásil: „Krása rovnic poskytovaných přírodou … vyvolává
  silnou emocionální reakci.“ A Einsteinovy rovnice tuto krásu v plné míře mají. Diraca však přece
  jen něco trápilo - byly to různé numerické koincidence mezi přírodními konstantami. Domníval se,
  že pokud jsou základní rovnice skutečně krásné, nemohou to být náhodné koincidence. V přírodě
  existují určitá velká čísla a o těch se Dirac domníval, že vztahy mezi nimi nemohou být náhodné.
  Srovnejme například elektromagnetickou a gravitační sílu mezi elektronem a protonem. Abychom
  dostali velikost vzájemné elektrické síly, musíme vzájemnou gravitační sílu vynásobit číslem, kde
  za jedničkou stojí 39 nul. Je to obrovské číslo, jež je charakteristické i pro větší veličiny,
  například věk vesmíru. I Hermann Weyl a Arthur Eddington argumentovali, že pro koincidence různých
  velikých čísel ve vesmíru musí být nějaký hlubší důvod. Dirac šel o krok dále a tvrdil, že
  gravitační konstanta, která je mírou síly gravitace, se vyvíjí v čase, není tedy konstantou, jak
  předpokládá obecná teorie relativity. 
  
  Dirac tuto myšlenku vyslovil v roce 1930, nikdy ji však nerozpracoval do konce. Nový život vdechli
  Diracově teorii až v letech padesátých a šedesátých minulého století Robert Dicke se svým
  studentem Carlem Bransem v Princetonu a nezávisle i Pascual Jordan v Hamburku. Vytvořili teorii,
  kde se gravitační konstanta skutečně měnila. Byla alternativou k obecné teorii relativity a
  vystupovalo v nich jedno dodatečné pole. Carl Brans na tato léta vzpomíná: „Experimentátoři,
  hlavně ti v NASA, byli velmi rádi, že mají záminku testovat obecnou relativitu, o níž se dlouho
  myslelo, že další experimentování nepotřebuje.“ Ne všichni to ale takto viděli. Jak Brans dále
  vzpomíná, „postupně se řada dalších teoretiků cítila uražena tím, že jsme Einsteinovu teorii
  zašpinili dodatečným polem“. 
  
  Když Paul Dirac odešel v Cambridgi na odpočinek, přešel na Floridskou státní univerzitu, kde
  rozvíjel některé své starší nápady. Občas se svěřil kolegům se svým přesvědčením, že musí
  existovat nějaký lepší a přírodu přesněji vystihující popis gravitace. Ale většinou se varoval,
  aby hovořil o své práci týkající se gravitace, protože cítil, že by to bylo vnímáno jako něco
  spekulativního, ne-li bláznivého. V té době ale vznikla řada jiných pokusů modifikovat obecnou
  teorii relativity motivovaných hlavně snahou nalézt kvantovou teorii gravitace, která by dávala
  konečné výsledky. Vstoupí-li do hry kvantová fyzika, s gravitací se mohou dít podivné věci - to
  proklamoval koncem šedesátých let sovětský fyzik Andrej Sacharov. 
  
  Sacharov byl členem týmu, který vytvořili Igor Kurčatov a Lavrentij Berija, aby soutěžil s
  Američany v nukleárním zbrojení. Patřil do něho i Jakov Zeldovič, Lev Landau a řada dalších. Syn
  učitele fyziky Andrej Sacharov začal v roce 1938 studovat na Moskevské státní univerzitě ve věku
  sedmnácti let. Po dobu války pracoval jako technický asistent a titul kandidáta věd v oboru
  teoretická fyzika získal v roce 1947. Tak jako Zeldovič, i on se vynořil jako zázračný chlapec
  sovětského režimu. Zatím co Landau se z nukleárního programu odpoutal hned po Stalinově smrti,
  Sacharov pracoval na vývoji jaderných a termonukleárních zbraní téměř dvacet let, déle než
  Zeldovič. 
  
  Zeldovič byl kreativní, expanzivní a intuitivní, Sacharov jevil větší zálibu v abstraktních
  problémech. Oba o sobě navzájem mluvili s obdivem. Sacharov mluvil o Zeldovičovi jako o „člověku s
  těmi nejširšími zájmy“, Zeldovič skládal komplimenty osobitému a jedinečnému způsobu, jak Sacharov
  řešil problémy: „Nerozumím tomu, jak Sacharov myslí.“
  
  Od roku 1965 se Andrej Sacharov zaměřil na kosmologii, ale šel svou vlastní originální cestou.
  Zatímco Zeldovič psal spoustu článků, Sacharovova produkce byla spartánštější. Jeho sebrané
  publikace tvoří jen tenký svazek. Ale v tom poměrně malém počtu publikací je několik skutečných
  pokladů, jako jsou práce o tvoření struktur, původu hmoty a povaze prostoročasu. V jednom krátkém
  hutném článku Sacharov argumentuje, že zákony prostoročasu jsou pouhá iluze a vynořují se z
  komplikované struktury kvantové reality. Tvrdí, že hledět na prostoročas je velmi podobné tomu
  dívat se na vodu, krystaly nebo jiný komplexní systém. To, co pokládáme za realitu, je jen hrubý a
  neúplný obraz nějaké základnější reality. Co způsobuje, že voda vypadá jako voda, tedy průzračné
  fluidum s vlastnostmi kapaliny, jsou kvantové vlastnosti molekul a to, jak jsou tyto molekuly
  navzájem volně vázány. Detaily jsou sice jiné, ale Sacharovův široký pohled prorocky naznačil to,
  jak se na prostoročas díváme o čtyřicet let později na základě pokroku v kvantové gravitaci.
  
  Sacharov studoval Einsteinovu teorii a vyslovil hypotézu, že geometrie prostoročasu není ve
  skutečnosti to základní v obdobném smyslu, jako není základním pojmem viskozita vody nebo
  elasticita krystalu. To jsou vlastnosti, které se vynoří z hlubšího popisu reality. Podobně se
  vynořuje z kvantové povahy hmoty gravitace. Udivujícím výsledkem třístránkového jednoduchého
  článku bylo tvrzení, že takto vzniklá gravitace má vyhovovat Einsteinovým rovnicím. Jinými slovy,
  Sacharov tvrdil, že kvantový svět přirozeným způsobem indukuje geometrii prostoročasu. Sacharovova
  indukovaná gravitace obecnou relativitu připomínala, ale její rovnice byly ve skutečnosti ještě
  komplikovanější než rovnice obecné relativity. Jde-li z Einsteinových rovnic pole hrůza,
  Sacharovova indukovaná gravitace je ještě mnohem horší. Rozdíl od Einsteinovy teorie se projeví
  teprve tehdy, když prostoročas je opravdu extrémně zakřivený, jak tomu je v blízkosti černých děr,
  nebo ve velmi raném vesmíru, kde vše bylo horké a stlačené na obrovskou hustotu. A je tomu tak i v
  mikroskopických měřítkách, kdy se do hry dostává Wheelerova kvantová pěna. Když se fyzikální
  zákony aplikují v extrémních situacích, mohou se zhroutit a místo nich nastoupí jiné, jež v
  nějakém smyslu zahrnují ty staré. 
  
  Sacharov publikoval tento článek v roce 1967, kdy měl ovšem v hlavě na prvním místě jiné věci.
  Jeho práce na vodíkové bombě mu vynesly vysoké pocty od sovětského režimu. Tak jako Zeldovič byl
  třikrát odměněn řádem „Hrdina socialistické práce“. Jenže dlouholeté spojení s bombou způsobilo,
  že si obzvláště silně uvědomoval hrozné důsledky, jaké mají závody v nukleárním zbrojení mezi
  Sovětským svazem a Spojenými státy. Silně se vyslovoval proti jadernému zbrojení, což ho stálo
  ztrátu prominentního postavení a režim ho začal ignorovat. V roce 1968 překročil pro režim
  přípustné meze, když publikoval esej „Zamyšlení nad pokrokem, mírovou koexistencí a intelektuální
  svobodou“, kde otevřeně vyslovil své námitky proti jednomu z hlavních sovětských zbrojních
  projektů, vývoji antibalistické raketové obrany. A to byl konec Sacharovova postavení jako
  modelového sovětského občana. Význačný disident byl zbaven svých privilegií, oprávnění pracovat na
  tajných projektech a byl poslán do vyhnanství ve městě Gorkij. Zeldovič se nad tím, co Sacharov
  nazýval „prací pro společnost“, nesouhlasně mračil. Svým nejbližším kolegům říkal: „Lidi jako
  Hawking jsou zcela oddáni vědě. Nic je nemůže rozptýlit.“ Sacharov to však viděl jinak. Ve svých
  pamětech píše: „Cítil jsem, že musím mluvit, jednat, vše ostatní dát stranou, do určité míry i
  vědu.“
  
  Sacharov sice utrpěl určitou ztrátu ve své osobní vědecké kariéře, avšak jeho myšlenka, jak
  kvantová fyzika může pozměnit obecnou relativitu, se v následujících desetiletích stále znovu
  objevovala. Jeho článek předvídal ostřelování obecné relativity novými kvantovými myšlenkami,
  které jí otřásalo v sedmdesátých letech minulého století. Někteří relativisté se domnívali, že
  změna obecné relativity způsobem, který navrhoval Sacharov, přiblíží tuto teorii standardním
  kvantovým teoriím ostatních polí a snad odstraní nekonečna, jež ji zamořovala. Jenže na konci
  sedmdesátých let dokázali Steven Weinberg a Edward Witten, že tato nekonečna se vyrušit nemohou. K
  opravě teorie proto nestačí jen malá úprava - je potřeba udělat něco podstatnějšího. 
  
  Něčím zcela jiným byly „superteorie“ - supergravitace a superstruny - ty zasahovaly do Einsteinovy
  teorie radikálněji. Fundamentální myšlenka, na které spočívala obecná teorie relativity, zůstala
  beze změny - geometrie prostoročasu byla stále ústředním jevištěm pro porozumění gravitaci. Jen to
  nebyl prostoročas o pouhých čtyřech dimenzích, s jakým pracoval Einstein. Rovnice popisující
  geometrii desetinebo jedenáctirozměrného prostoročasu vypadaly obdobně jako ve čtyřech dimenzích.
  Dodatečné dimenze však způsobovaly, že vznikla nová říše dalších fundamentálních částic a silových
  polí, jež ovlivňovaly děje ve čtyřrozměrném prostoročase, který našimi smysly vnímáme. 
  
  Několik osamělých hlasů odolávalo tomuto útoku na obecnou relativitu, ale převažoval názor, že
  obecná relativita si žádá určitého vylepšení, má-li být aplikovatelná v oblastech s vysokou
  hustotou hmoty, jaká je v blízkosti singularit nebo v nejranějším vesmíru, kde se musí projevovat
  kvantové aspekty. 
  
  Einsteinova teorie byla stále skvěle úspěšná, pokud jste kormidlovali mimo minová pole kvantové
  gravitace a nezabývali jste se samotným horkým a chaotickým počátkem vesmíru. Ve velkých měřítkách
  v astrofyzice a kosmologii stále přinášela cenné dary. 
  
  Kdyby astronomie byla průmyslovým odvětvím, výroční zasedání Mezinárodní astronomické unie by byla
  něco jako veletrh, kde se každý snaží něco prodat. Při takovém zasedaní v roce 2000, jež se konalo
  v britském Manchesteru, se více než tisíc lidí chtělo pochlubit svými výsledky a představit nové
  projekty, jež měly v nejbližší době odstartovat. Kosmologové byli na tomto setkání v triumfální
  náladě a tuto euforii jsem sdílel i já. Pár let předtím byla zveřejněna pozorování supernov, jež
  prokazovala urychlování expanze vesmíru, a právě v roce konání konference byly oznámeny výsledky
  měření geometrie vesmíru. Všechny měřené údaje ukazovaly, že vesmír je jednoduchý, i když značně
  exotický vzhledem ke své náplni temnou hmotou a kosmologické konstantě pohánějící zrychlenou
  expanzi. Nebyl důvod ke vzrušeným debatám a nesouhlasným stanoviskům a nebylo moc prostoru na
  osobní preference jednoho či druhého modelu vesmíru. Kosmologie byla konečně dobrou solidní vědou,
  výsledky pozorování byly jasné, konzistentní a nedaly se nijak obejít. 
  
  Na konferencích bývá kromě referátů v jednotlivých sekcích několik přednášek pro všechny
  účastníky. Jednu z těchto plenárních přednášek měl v Manchesteru Jim Peebles. Konference byla
  vlastně oslavou jeho prací o fyzikální kosmologii i toho, jak daleko nás dovedly - vždyť všechny
  důležité objevy posledních let navazovaly tím či oním způsobem na metodiku, na jejímž založení měl
  lví podíl. Jenže Peebles nebyl přítelem ukvapeného jásotu nad úspěchy, byť to i byly úspěchy jeho.
  Ve svém příspěvku tlumil všeobecné nadšení a kladl otázku, proč vlastně chceme mít co nejpřesnější
  údaje o vesmíru. Jeho odpověď zněla: abychom testovali naše hypotézy. Postupně procházel všechny
  aspekty modelu s horkým velkým třeskem. Jak víme, že vesmír měl horký počátek? Odkud se v něm
  vzaly velkorozměrové struktury? Jak vznikly galaxie? A poukázal také na něco vlastně zřejmého.
  Později v konferenčních materiálech to formuloval takto: „Elegantní logika obecné teorie
  relativity a přesnost jejích testů ji doporučily jako prvního kandidáta na budování kosmologických
  modelů.“Varoval ale, že kosmologové by neměli dělat předčasné závěry o její jedinečnosti. Obecná
  teorie relativity výborně funguje v měřítkách sluneční soustavy - posun perihelia Merkura je toho
  skvělým dokladem. Nemáme ale žádný důkaz, že ji lze se stejnou přesností užívat i v rozměrech
  vesmírných. „Je to působivá extrapolace,“ řekl Peebles a měl jistě pravdu, i když většina
  účastníků konference odmítala přijmout význam jeho tvrzení. 
  
  Francouzský astronom Le Verrier kdysi vášnivě argumentoval, že k výkladu dráhy Merkura je nutné
  předpokládat existenci dosud neznámé planety Vulkán, která se nachází v blízkosti Slunce. Víra v
  Newtonovu teorii gravitace vedla Le Verriera k předpovědi něčeho neznámého a exotického, co nikdo
  nepozoroval - bez Vulkánu by newtonovský model nefungoval. Ukázalo se ovšem, že neměl pravdu. K
  vysvětlení detailů dráhy Merkura nebylo třeba hledat novou planetu, nýbrž novou teorii gravitace.
  
  Dnes na počátku 21. století jsme v podobné situaci. Máme skvělou teorii gravitace, která však k
  vysvětlení kosmologických pozorování potřebuje, aby 96 procent náplně vesmíru tvořilo něco přímo
  nepozorovaného, co nelze vidět, ani jinak registrovat. Mohlo by to signalizovat další trhlinu v
  budově, kterou Einstein postavil před téměř sto lety? To, že obecná relativita zřejmě potřebuje
  úpravu, má-li být slučitelná s kvantovou fyzikou, většina vědců přijímá bez potíží. Jinak je tomu
  s připuštěním oprav obecné relativity při její aplikaci v opravdu velkých měřítkách, které by byly
  nutné, měla-li by se eliminovat temná hmota či temná energie. Pro většinu astrofyziků byly nějaké
  úpravy Einsteinovy krásné teorie nepřijatelné tak jako použití kladiva na krásné autoveterána jen
  proto, že se nemůže vejít do garáže.
  
  Izraelský relativista Jacob Bekenstein začal přemýšlet o možné úpravě Einsteinovy obecné teorie
  relativity již začátkem sedmdesátých let dvacátého století, když ještě byl doktorandem Johna
  Wheelera v Princetonu. V té době studoval entropii černých děr, zaujala ho ale také alternativa k
  obecné relativitě, kterou navrhl Dirac. „V určitém okamžiku jsem nechápal, proč se některé věci
  dělají v obecné relativitě jistým způsobem a které předpoklady jsou vlastně důležité, zkrátka proč
  se musí nutně sledovat klasická cesta k relativistické teorii gravitace. Cítil jsem potřebu
  porovnat standardní přístup s jinými teoriemi gravitace, než je obecná teorie relativity“ říkal
  Bekenstein. 
  
  Tím „odlišným pokusem“, na kterém začal pracovat, byla teorie jeho krajana, izraelského
  astrofyzika Mordehaie Milgroma, z osmdesátých let. Milgrom dostal nápad podívat se na působení
  gravitace v galaxiích zcela novým pohledem. Poukázal na to, že argumenty pro existenci temné hmoty
  v rotujících galaxiích jsou založeny na pozorování okrajových oblastí galaxií, kde je gravitační
  síla už velmi slabá. Když se zde aplikuje newtonovská gravitace, pak jasně vychází, že musí být
  přítomna nějaká dodatečná přímo nepozorovaná hmota, která přispívá ke gravitační přitažlivosti
  hmoty viditelné. Není ale chyba uplatňovat i v těchto oblastech Newtonovu teorii? Milgrom vyslovil
  odvážný předpoklad, že hvězdy na okraji galaxie se chovají, jako by byly těžší, takže gravitační
  působení centrálních hvězd na hvězdy na okraji je podstatně efektivnější než podle Newtonovy
  teorie. A protože gravitační přitažlivost je efektivnější, vnější hvězdy se pohybují rychleji. To
  by mohlo vysvětlit výsledky Very Rubinové a dalších astronomů, kteří pozorovali, že vnější hvězdy
  obíhají kolem centra galaxie podstatně rychleji, než se očekávalo. Milgrom nazval svůj přístup
  Modified Newtonian Dynamics, zkráceně MOND. 
  
  Řada astrofyziků se domnívala, že Milgrom zašel se svou úpravou zákonů gravitace příliš daleko.
  Chyběl nějaký vůdčí princip, byla to spekulativní úprava snažící se vystihnout pozorování.
  Bekenstein myšlenku modifikované gravitace, jež odstraňuje potřebu temné hmoty, předložil na
  zasedání Mezinárodní astronomické unie v roce 1982 a ohlas na své vystoupení popisoval takto:
  „Někteří posluchači na mne hleděli, jako kdybych oznámil, že jsem viděl UFO … Skoro všichni se
  domnívali, že představa existence temné hmoty je důležitá a brali ji za svou.“Po dvě další
  desetiletí veliká většina astrofyziků a relativistů Milgromovu myšlenku buď ignorovala, nebo se ji
  snažila vyvrátit. Čas od času se objevil článek, který aplikoval Milgromův zákon na určitou
  astrofyzikální situaci a ukazoval, že nefunguje. Argumentace v takových článcích však byla často
  nepřesná a neúplná, jestliže ale vyvracely MOND, byly pokládány za solidní vědu, a proto byly
  snadno publikovatelné. Naopak obhajoba MOND se brala jako exemplárně špatná věda a zveřejnit
  článek na toto téma vyžadovalo nesmírné úsilí. Jak řekl jeden astronom, MOND bylo neslušné slovo.
  
  Peebles se těchto šarvátek neúčastnil, v roce 2002 však uvedl: „MOND jsme ještě zdaleka
  nevyvrátili a lidem, kteří se touto teorií zabývají, by se mělo dostávat trochu větší podpory.
  “Jacob Bekenstein kritizoval přístup k MOND daleko silněji: „Při hodnocení války mezi MOND a
  temnou hmotou je třeba si uvědomit, že nejde jen o akademický spor. Do hledání temné hmoty se
  vložila spousta peněz. … A nedá se tomu zabránit - do temné hmoty investovala řada lidí i celou
  svou kariéru. Je zřejmé, že kdyby něco podobného teorii MOND došlo uznání, investice do hledání
  temné hmoty by se snížily a bylo by méně pracovních míst.“
  
  Od vzniku teorie MOND se Bekenstein ze všech sil snažil ji vylepšit. Vždy měl sklon k tomu dívat
  se na samé kořeny fyzikálních teorií, a tak mu nedělalo radost ponechat MOND ve stavu, v jakém se
  nacházela. Chtěl najít teorii, jež by byla srovnatelná s obecnou teorií relativity a byla by
  aplikovatelná ve všech měřítkách, počínaje pozemskou fyzikou a konče největšími vesmírnými
  strukturami. Prohlásil: „Rozhodl jsem se postavit se ke kritice čelem a vytvořit příkladnou
  relativistickou teorii.“ V roce 2004 publikoval článek, ve kterém zaváděl novou teorii, která měla
  být rivalem teorie Einsteinovy. Nazval ji zkratkou TeVeS, což znamenalo, že jeho nová teorie je
  tenzorově-vektorově-skalární teorií gravitace. Nevypadala moc pěkně. Už podle názvu v ní
  existovala změť polí, jejichž kombinace vedla k podstatně složitějším rovnicím, než byly rovnice
  Einsteinovy. K tenzorovému poli, určujícímu v obecné relativitě geometrii, přibyla ještě dvě
  dodatečná pole, jedno vektorové a jedno skalární. Působilo to zmateně, ale Bekensteinova teorie
  fungovala. Nejenže vypadala podobně jako MOND, když se aplikovala na galaxie, ale poradila si i s
  vývojem vesmíru a tvorbou těch největších vesmírných struktur. 
  
  Většina kosmologů a relativistů se na TeVeS dívala pohrdlivě. Vadil jim především nedostatek
  elegance, působila zmateně a podle nich se nedostávala k jádru problému. Byla to ale účinná
  konstrukce vytvořená vědcem, který měl mezi relativisty uznávané postavení. Bekensteinův objev
  entropie černých děr se pokládal za jeden z nejhlubších pohledů do moderní obecné teorie
  relativity a kvantové fyziky. Je pravdou, že řada starších čelných fyziků se nechala unést svými
  úspěchy a začali pracovat na podivných myšlenkách, které nikdo nesdílel, to ale nebyl Bekensteinův
  případ. 
  
  On také nebyl ve svém útoku osamělý. Zatímco jeho návrh se snažil vyrovnat s problémem temné
  hmoty, jiní se soustředili na kosmologickou konstantu a temnou energii. Panorama konkurenčních
  teorií obecné relativity se stalo neuspořádanějším, ale také bohatším a boj o správnou teorii
  gravitace nabyl na intenzitě. Ohromující pozorování vykonaná novými teleskopy a přístroji
  postavenými během exploze fyzikální kosmologie dodala další munici. Když se našla nějaká nová
  kosmologická data, která byla vykládána jako podpora obecné relativity, pravidelně to bylo spojeno
  s prohlášeními do tisku a nové objevy se často ocitly na titulních stránkách časopisů. Zároveň
  však se objevila záplava článků, ve kterých autoři dokazovali, že to, co se prezentuje jako jasná
  podpora obecné teorie relativity, není tak zcela solidně podloženo. 
  
  V lednu 2008 signalizoval článek v Nature, že dochází k určitému nenápadnému posunu v názorech.
  Italský tým v něm analyzoval data z pozorování galaxií. Jejich postup se moc nelišil od metodiky
  nastartované Jimem Peeblesem a používané jím a jeho následovníky čtyřicet let. Studiem seskupování
  galaxií mohla italská skupina určit, jak se jednotlivé galaxie pohybují v gravitačním poli, do
  něhož jsou vnořeny. Na tom nebylo nic nového a podobné analýzy se dělaly na základě různých
  přehlídek galaxií. Zajímavé však bylo, jakým způsobem Italové své výsledky interpretovali.
  Graficky vyznačili nejen to, co předpovídá obecná teorie relativity, ale i předpovědi některých
  alternativních teorií gravitace. Některé teoretické předpovědi se přesně shodovaly s pozorováním,
  jiné se však podstatně lišily. Bylo jasné, že teorii je třeba pečlivě porovnat s výsledky
  pozorovatelů. Článek v Nature ohlašoval změnu ducha v přístupu observačních kosmologů. Od
  devadesátých let minulého století se pokládalo za cíl změřit, charakterizovat a identifikovat
  temnou energii, ale tento článek si kladl za cíl testovat obecnou teorii relativity. Byl to návrat
  k ověřování základních předpokladů fyzikální kosmologie. 
  
  V následujících letech se testování obecné teorie relativity stalo jádrem pozorovací kosmologie.
  Pořád chceme vědět, existuje-li temná energie, co je její podstatou a jak se tvoří galaxie, jež
  jsou základními stavebními kameny vesmíru. Ale zjišťujeme, že v současné době v žádostech o
  vědecké granty, v programech seminářů i mezi hlavními přednáškami na konferencích se otázka
  ověřování obecné teorie relativity dostává víc a více do popředí. 
  
  Nad modifikacemi gravitační teorie se dosud ošklíbá mnoho relativistů, ne-li všichni. Případné
  úpravy obecné teorie relativity za účelem jejího sladění s kvantovou mechanikou se tiše přijímají,
  avšak opravovat teorii prostoročasu proto, aby se dosáhlo shody s pozorováním, to je něco jiného.
  V rámci Einsteinovy teorie ještě zbývá vyjasnit mnoho věcí a nějak ji pozměňovat, to je podle
  relativistů zbytečná a neelegantní komplikace. Ale příroda s tím nemusí souhlasit, a když se u
  astronomů projevil obnovený zájem o Einsteinovu teorii relativity, máme možnost zkoumat základní
  zákony prostoročasu na základě pohledu do hlubších a vzdálenějších oblastí vesmíru. 
  
  Myšlenky Diraca, Sacharova a Bekensteina podložené novými pracemi v pozorovací kosmologii
  poskytují nový pohled, který je příliš vzrušující, než abychom ho ignorovali, a kolos kosmologie
  dostává díky nim novou tvář. V nedávné době jsem se s kolegy z Oxfordu a Nottinghamu rozhodl
  sepsat přehled alternativních teorií gravitace. Cítili jsme se jako badatelé v džungli, kteří
  objevují stále nové exotické živočišné druhy. Narazili jsme na tucty teorií, jednu zvláštnější než
  druhou, které navrhovaly různé změny obecné teorie relativity vedoucí k udivujícím a často
  realistickým důsledkům. Náš přehled nabízel ten nejobsáhlejší zvěřinec gravitačních teorií, z
  nichž některé mohly být vážnými konkurenty obecné teorie relativity. Dnes o alternativách k obecné
  relativitě přemýšlí tolik vědců, že na velkých shromážděních relativistů - potomcích DeWittovy
  konference na Chapel Hill a Texaského symposia Alfreda Schilda - probíhají paralelní sekce, na
  kterých se vědci z různých zemí a z různých generací snaží obecnou relativitu rozcupovat. Je to
  sice stále okrajová aktivita, má však mnoho aktivistů. 
  
  To odpoledne v Cambridgi, kdy jsem měl svou přednášku, byl Efstathiou odmítavý. Ale i on, astronom
  s brilantní hlavou a jeden z pionýrů běžného standardního kosmologického modelu s temnou energií a
  temnou hmotou, jenž je založen na obecné teorii relativity, by byl jistě vzrušen, kdyby nová
  astronomická pozorování ukazovala na nějakou novou fyziku. A nová teorie gravitace se všemi
  dalekosáhlými důsledky by si bezpochyby označení „nová fyzika“ zasluhovala. A tak vše závisí na
  tom, zda nová astronomická data přinesou něco opravdu nového.

\section{K něčemu dojde}\label{feyIchIIIsecXV}
  Nedávno jsem působil jako poradce v Evropské kosmické agentuře (European Space Agency, ESA). Tato
  instituce je zodpovědná za vysílání satelitů s vědeckým programem do vesmíru a často spolupracuje
  s NASA. Jedním z nejznámějších plodů této spolupráce je Hubbleův teleskop, který pořídil
  nejjasnější a nejpřesnější snímky těch nejvzdálenějších vesmírných objektů. 
  
  Umělé družice patří k nejmodernějším vědeckým nástrojům. Jsou to nepředstavitelně sofistikované
  laboratoře obíhající na hranicích našeho dosahu a uskutečňují se na nich neobyčejně složité
  experimenty. Jsou ovšem velice nákladné, jejich cena se pohybuje mezi půlmiliardou a mnoha
  miliardami dolarů. Než se dostanou do vesmíru, předcházejí tomu roky nebo i desítiletí pečlivého
  plánování a zpracovávání návrhu, protože musí být zaručeno, že vyslání satelitu se opravdu
  vyplatí. 
  
  Má činnost v ESA spočívala v účasti na diskusích o příštích vesmírných projektech a posuzování
  různých návrhů podaných velkými mezinárodními vědeckými týmy. Během dlouhých a náročných zasedání,
  kdy jsme byli zahrnováni spoustou powerpointových prezentací, časových plánů a finančních
  požadavků, jež mi často vháněly slzy do očí, jsem mnohdy ztrácel chuť do života. Věda se mi
  najednou zdála hodně odlišná od volného bádání, nespoutané kreativity a krásné matematiky, které
  mě k ní přitáhly jako mladého doktoranda. Šokovalo mě také, že diskutujeme o vědeckých projektech
  dalekosáhlého významu tak, jako kdyby šlo o průmyslové podnikání či zakládání nových továren v
  jakési vzdálené zemi. 
  
  Opravdu mě ale překvapilo, jak silnou pozici zastávaly mezi mnoha technickými projekty návrhy
  družicových misí, jež se týkaly obecné teorie relativity. Právě obecná relativita stála v pozadí
  řady technických návrhů a specifických cílů, o kterých jsme diskutovali. Každou chvíli jsme
  dostali žádost o financování satelitového experimentu za zhruba miliardu dolarů, jenž měl buď
  ověřovat některou předpověď Einsteinovy teorie, nebo zkoumat vzdálené oblasti vesmíru či
  vlastnosti hutných hmotných objektů. Jednalo se zde o budoucnost vesmírných projektů v
  jednadvacátém století. Samozřejmě ne všechny projekty bylo možné doporučit k financování, ne
  všechny navrhované družice skutečně vyletí a volba mezi projekty byl těžký oříšek.
  
  Jeden projekt navrhoval, jak zachytit gravitační vlny vysílané při srážce černých děr. Měl být
  postaven obrovský interferometr, v principu fungující obdobně jako LIGO nebo GEO600, ale mnohem
  větší. Nepředstavitelně přesné laserové paprsky se měly šířit mezi třemi satelity obíhajícími
  Slunce, na nichž se měly odrážet od zrcátek vzdálených od sebe miliony kilometrů. Celé zařízení
  dostalo název Laser Interferometer Space Antenna, zkráceně LISA. Princip měření navazoval na
  pozemské projekty, ale zařízení mělo být schopné registrovat tak slabé signály, jaké LIGO nebo
  GEO600 nikdy nezachytí. 
  
  To není všechno. Byla navržena mise, jejímž cílem bylo měřit expanzi vesmíru v době, kdy jeho věk
  byl zlomkem jeho dnešního stáří. Využívalo by se metod fyzikální kosmologie a sestavil by se
  obrovský katalog obsahující stovky milionů galaxií. Rozbor uskupení galaxií v této vesmírné síti
  by pak mohl zjistit buď vliv temné hmoty a temné energie, nebo naopak to, že Einsteinova teorie v
  těch největších škálách neplatí, a potvrdit tak domněnku některých vědců.
  
  Vznikl také návrh kosmické sondy zkoumající vnitřek černé díry. Na cestě do jejího nitra by
  pátrala po silném rentgenovém záření, jež v šedesátých a sedmdesátých letech minulého století
  pomohlo černé díry objevit. Tentokrát by se však získaly informace z hlubších vrstev hroutící se
  hvězdy, kde podle Zeldoviče, Novikova, Reese a Lyndena-Bella mělo docházet k divokým dějům. Poprvé
  by se tak mohlo podařit sledovat fyzikální procesy v blízkosti záhadného horizontu událostí, který
  před dlouhou dobou objevil Schwarzschild. 
  
  Během těchto jednání v ESA jsem si ujasnil, že obecná relativita bude jádrem fyziky a astronomie
  jednadvacátého století. 
  
  Nebude to ovšem snadné. V reálném světě omezených finančních prostředků, chudoby a hospodářské
  recese se musí dobře rozmýšlet, než se dospěje k rozhodnutí vynaložit miliardy eur nebo dolarů na
  vědeckou družici. Není tedy divu, že vláda Spojených států se rozhodla odstoupit od
  spolufinancování projektu LISA, je to ovšem veliká škoda. Projekt LISA měl být konečným krokem v
  pátrání po gravitačních vlnách. Neměl jen prokázat jejich existenci; byl by dokonalou observatoří
  umožňující pomocí gravitačních vln zachycovaných obřím interferometrem sledovat srážky černých děr
  či vzájemně se obíhající neutronové hvězdy. A nejen to, pomocí LISA bychom se dozvěděli mnoho
  nového o dalších exotických předpovědích Einsteinovy obecné teorie relativity. 
  
  První fáze provozu systému LIGO byla velikým úspěchem, a to přesto, že se žádné gravitační vlny
  nepozorovaly. Ukázala totiž, že fantastická technologie zařízení, kombinace laserové techniky,
  kvantových procesů a přesné mechaniky, skutečně funguje a lze ji ještě vylepšit. Další fáze
  projektu, pokročilé LIGO, už by mohla nějaké gravitační vlny skutečně vidět a upravit cestu pro
  LISA. Když ale Američané z projektu vycouvali, osud LISA je více než nejistý. Kdo by chtěl v době
  ekonomické nejistoty krmit obrovské stvoření s tak ezoterickým životním cílem? 
  
  Hledání gravitačních vln je však příliš důležitý úkol, než abychom mohli celý záměr vzdát, a tak
  alespoň Evropa, reprezentovaná ESA, bude pokračovat. Nový interferometr bude menší, ale stále
  ještě s pozoruhodným výkonem. Bude sice stát miliardy, ale přece jen bude méně nákladný než LISA.
  
  Zklamaní relativisté v Americe se semkli a odmítají se úplně vzdát. Řada skupin roztroušených po
  celých Spojených státech pracuje na návrzích zařízení, jež by byla levnější, kompaktnější a méně
  ambiciózní, přesto by ale přinášela cenné informace o těch nejvzdálenějších zákoutích
  prostoročasu. Kdyby tedy Evropané nakonec couvli, je zde ještě plán B. 
  
  Nemusíme čekat na to, až vzlétnou satelity zkoumající gravitační teorii. Už teď se dějí
  fantastické věci. Sledovali jsme historii singularit a viděli jsme, jak byla jejich přítomnost v
  teorii odpudivá pro některé vědce od Alberta Einsteina přes Eddingtona až k Johnu Wheelerovi, i
  když ten nakonec uviděl světlo. Po objevu kvasarů, neutronových hvězd a vesmírného rentgenového
  záření však výbuchy fenomenální kreativity lidí jako Wheeler, Kip Thorne, Jakov Zeldovič, Igor
  Novikov, Martin Rees, Donald Lynden-Bell a Roger Penrose způsobily, že černé díry našly v našem
  povědomí pevné místo. Na konci šedesátých a sedmdesátých let, tedy období, které Kip Thorne nazval
  „zlatým věkem obecné relativity“, získaly černé díry v astrofyzice a fyzice tak pevný statut, jako
  mají hvězdy a planety. 
  
  Mám v knihovně dvě učebnice obecné relativity, které se objevily na konci zlatého věku. Navzájem
  se značně liší. Jednu z nich, nazvanou prostě Gravitace, napsal John Wheeler se dvěma svými
  bývalými skvělými studenty, Charlesem Misnerem a Kipem Thornem. Má více než tisíc stránek velkého
  formátu a černé desky - vypadá jako temný telefonní seznam. Fyzikové ji označují iniciálami autorů
  - MTW. Obsahuje spoustu výstižných ilustrací a je v ní skoro všechno, co byste mohli chtít vědět o
  prostoročase. Dozvíte se tam i o různých nápadech, s nimiž John Wheeler udivoval na konferencích a
  které jsme označili jako „wheelerovštiny“. Ta druhá učebnice je od Stevena Weinberga, jednoho z
  otců standardního modelu. Weinberg získal pověst jednoho z největších intelektů v oblasti kvantové
  fyziky, zabýval se však i obecnou relativitou a jeho kniha Gravitace a kosmologie obsahuje i
  pečlivý a promyšleně podaný úvod do Einsteinovy teorie. Materiál v ní se v mnohém kryje s MTW, i
  když se Weinberg vyhýbá některým bizarnostem. A uvážíme-li, co přinesla léta těsně předcházející
  jejímu vydání, je s podivem, jak málo se věnuje černým dírám. Ty jsou jen opatrně zmíněny, jako by
  se autor domníval, že zde obecná relativita zachází až příliš daleko. 
  
  Snadno pochopíme, proč tehdy panovala určitá opatrnost. Ano, všechny důkazy se zdály ukazovat na
  to, že v blízkém i vzdáleném vesmíru existují hutné těžké objekty, jejichž vlastnosti by bylo
  těžké vysvětlit jinak, než že se jedná o černé díry. Ale nikdo skutečně neviděl černou díru. Vidět
  černou díru je tak trochu paradox. Není co vidět - černá díra je dokonale skryta Schwarzschildovou
  oponou. Jenže to, že černou díru nemůžeme vidět, neznamená, že by nestálo za to se na ni podívat.
  Obrovská černá díra ve skutečnosti sedí přímo ve středu naší galaxie Mléčné dráhy. Její hmotnost
  je asi stomilionkrát větší než hmotnost Slunce a její poloměr je asi 10 milionů kilometrů. Je
  obrovská. Jenže je vzdálená více než deset tisíc světelných let, a proto na obloze zaujímá jen asi
  stomiliontinu stupně, jeví se tedy jako nepatrná tečka, příliš malá na to, aby ji rozlišily
  současné teleskopy. Že tam skutečně černá díra je, víme jen díky chytrosti a vytrvalosti
  astronomů. 
  
  Dvě výzkumné skupiny, jedna v Mnichově a druhá v Kalifornii, trpělivě sledovaly několik hvězd v
  blízkosti centra Mléčné dráhy. Studovaly jejich pohyb po více než deset let a zjistily, že jejich
  dráhy jsou silně zakřivené. Bylo zřejmé, že na ně z centra působí velká gravitační síla. Díky
  pečlivému pozorování byli astronomové schopni určit, kde se silové centrum nachází, i určit
  hmotnost centrálního objektu. Prokázali tak existenci černé díry, a tedy i singularity
  prostoročasu v centru galaxie. 
  
  Ale ani to není všechno. Astronomové a relativisté plánují stavbu teleskopu, který tuto černou
  díru skutečně uvidí. Bude se jmenovat Event Horizon Telescope (Teleskop horizontu událostí) a měl
  by mít rozlišení jedné miliardtiny úhlového stupně, což by byl zlomek toho, co centrální
  galaktická černá díra zaujímá na obloze. Mohl by tedy skutečně vidět Schwarzschildovu oponu, tedy
  povrch černé díry, o kterém Oppenheimer a Snyder říkali, že je to momentka zamrzlá v prostoročase.
  Jevila by se jako temný stín uvnitř víru, který podle hypotézy Zeldoviče a Novikova černou díru
  obklopuje. Pozorovali bychom tedy akreční disk složený z hvězd, plynu a prachu, který singularita
  vtahuje do černé díry. 
  
  Hromadící se důkazy jsou velmi přesvědčivé. Weinbergova zdrženlivost byla ještě pochopitelná, dnes
  je však těžké najít astrofyzika, který by o přítomnosti černé díry v centru naší galaxie
  pochyboval. A netýká se to jen Mléčné dráhy, černou díru v centru by měly mít všechny galaxie -
  superhmotnou pohonnou jednotku, obklopenou gargantuovskými spirálami hvězd. 
  
  Pro média je cokoliv, co má co do činění s obecnou teorií relativity a Einsteinovými velkými
  myšlenkami, přitažlivé a vděčné téma. Obrázky centra naší galaxie vedly například k pořadu „Černá
  díra ve středu naší galaxie“na BBC a k velkému článku s názvem „Pozorování svědčí o existenci
  černé díry v Mléčné dráze“ v New York Times. Právě v den, kdy toto píši, vychází na webové stránce
  BBC komentář mého oxfordského kolegy k současnému pozorování kvasaru, který je prokazatelně
  superhmotnou černou dírou s hmotností miliardy Sluncí. Ohromuje mě, že téměř padesát let po
  měřeních Maartena Schmidta a prvním Texaském symposiu vyvolávají černé díry stále takový zájem.
  
  Neuběhne měsíc, aby se v novinách neobjevilo něco o kosmologii či fyzice černých děr, o počátcích
  vesmíru či o ozvěnách jiných vesmírů z tajemného multiverza. Slova jako černé díry, velký třesk,
  temná hmota, temná energie, multiverzum, singularita či červí díra se staly pojmy popkultury a
  vyskytují se ve hrách na Broadway, v muzikálech či hollywoodských filmech. Obecná relativita
  vstoupila mnoha způsoby do vědecko-fantastické literatury i sci-fi filmů. Tam její úloha překonává
  svou fantastičností i nejdivočejší wheelerovské sny. Dá se říci, že dnes se za experta na obecnou
  relativitu pokládá kdekdo. 
  
  Toto okouzlení je zábavné, někdy ale až komické. Když mne můj syn obvinil z nezodpovědnosti,
  protože jsem nepřímo ovlivnil to, že Velký hadronový urychlovač (Large Hadron Collider, LHC) v
  CERN byl uveden v život, nebyl jediný, kdo zaujímal tak negativní stanovisko. Média opakovaně
  šířila myšlenku založenou na určitém výkladu teorie strun, že po spuštění LHC se vytvoří černá
  díra. Při srážce svazků protonů by se mělo vytvořit mnoho různých částic a také mikroskopické
  černé díry, miniportály do jiných dimenzí. Můj syn také věděl, že černé díry nasají vše ve svém
  okolí. To dnes ví každý. Tak proč proboha by měl někdo postavit tak neuvěřitelně nebezpečnou věc?
  To by přece byl vyložený nesmysl. Jeden takovýto „fyzik“ se dokonce pokoušel prosadit zákaz LHC
  soudní cestou. Když byl dotázán v populárním televizním pořadu, jaká je pravděpodobnost, že ke
  katastrofě skutečně dojde, odpověděl, že padesát procent. Jeho pochmurná předpověď se nevyplnila,
  ženevský urychlovač běží, my jsme zde a žádné miniaturní černé díry se bohužel nenalezly. 
  
  Při populárních přednáškách vždy dostávám otázku: „Co bylo před velkým třeskem?“ Utíkám se k
  různým odpovědím, jednou z nich je, že „před velkým třeskem žádné,před‘ nebylo, protože nebyl čas
  “. Kolegyně Jocelyn Bell Burnellová odpovídá více v duchu zen-buddhismu: „To je jako ptát se, co
  je na sever od severního pólu.“ Kdybych se mohl utéci k matematice, odpověď by byla snazší, jenže
  většina posluchačů by nebyla spokojená. Po desetiletí jsme díky větám Stephena Hawkinga a Rogera
  Penrose o singularitách věřili, že před velkým třeskem skutečně nic nebylo. Byla to jedna z pravd,
  matematických pravd, které pocházejí ze zlatého věku obecné relativity a nedají se obejít. 
  
  V poslední době jsou však mé odpovědi ohledně velkého třesku méně určité a méně definitivní.
  Odpovědi na otázku počátku času se zkomplikovaly vzhledem k vývoji v kvantové gravitaci a
  kosmologii. Když necháte hodiny běžet pozpátku, takže vesmír houstne a jeho teplota neustále
  roste, nastane situace, kdy ke slovu přicházejí pojmy jako kvantová pěna, struny, brány či
  kvantové smyčky. Mnozí vědci tvrdí, že se zde koncept prostoročasu hroutí a mluvit o počáteční
  singularitě ztrácí smysl. 
  
  Co tedy bylo před velkým třeskem? Jednou možností je, že se náš vesmír zrodil z vakua jako bublina
  prostoročasu a rostl a rostl, až získal dnešní podobu. A podobně se z vakua zrodila i řada dalších
  vesmírů. Jiné scénáře se vymýšlejí na základě teorie strun a M-teorie, podle kterých má náš vesmír
  více než čtyři dimenze, tři prostorové a jednu časovou. My pak žijeme na „bráně“ o třech
  prostorových a jedné časové dimenzi, jež představuje náš prostoročas, a účastníme se jejího
  vývoje. Naše domovská brána se chová jako třírozměrný vesmír, který se však občas srazí s jinou
  branou podobnou té naší. Když dojde ke srážce, brány se zahřejí a náš vesmír vypadá jako by prošel
  horkým velkým třeskem. Není zde tedy žádná počáteční singularita, jen nekonečná posloupnost
  horkých velkých třesků, cyklický vesmír, kterým by se pyšnili ortodoxní sovětští filosofové a
  pravděpodobně i Fred Hoyle a jeho souvěrci. Tvůrci tohoto modelu nazvali každý nový velký třesk
  ekpyrósis, což v antickém Řecku značilo periodickou destrukci vesmíru nevyhnutelně následovanou
  jeho znovuzrozením.
  
  Ovšem kvantová gravitace ukazuje na fragmentaci prostoru, kterou by uměl pozorovat „vševidoucí
  “mikroskop. Když bychom přetáčeli vesmírné hodiny zpátky, tak dříve, než bychom dospěli do
  okamžiku, kdy vesmír byl soustředěn v jediném bodu, procházeli bychom obdobím fragmentace
  prostoročasové struktury - před dosažením počáteční singularity by se známá fyzika zhroutila. Ti,
  kteří věří ve smyčkovou kvantovou gravitaci, mají odpověď na to, co bylo před počátkem vesmíru,
  jehož scénář dnes popisujeme - byl zde kolabující vesmír, který se smršťoval tak dlouho, až
  narazil na kvantovou stěnu, od ní se nějak magicky odrazil a začal se rozpínat. Takový vesmír
  prošel fází, které se zjednodušeně říká „odraz“. 
  
  Nemusíme se ale utíkat až k záhadné temné éře, kdy vstupuje do hry kvantová gravitace, na kterou
  panuje tolik rozdílných názorů. S velikou pravděpodobností je prostoročas mnohem rozsáhlejší, než
  jsme si dříve mysleli, a náš vesmír je jedním z nespočetného množství vesmírů, jejichž soubor
  označujeme jako multiverzum. V multiverzu se stále objevují nové vesmíry, rostou do kosmických
  proporcí, každý svým vlastním tempem a každý z nich je udělán jiným způsobem. Kdybychom sledovali
  minulost vesmíru, ve kterém žijeme, zjistili bychom, že je vnořen jako puchýřek do mnohem širšího
  prostoročasu, který existuje věčně. Multiverzum je obrovská divoká říše, která má jednu neměnnou
  vlastnost - stacionární stav tvoření a destrukce. 
  
  Představa multiverza se zrodila spolu s tak zvaným antropickým principem při snaze nalézt řešení
  problému kosmologické konstanty. Observační kosmologie nás přesvědčila, že kosmologická konstanta
  v přírodě skutečně existuje. Jenže její velikost je mnohem menší než obscénně velká hodnota,
  kterou předpovídá standardní kvantová teorie. Zastánci strunové teorie interpretují nedostatek
  jednoznačnosti této teorie tak, že existuje „krajina“ různých možných vesmírů, z nichž každý má
  svou vlastní škálu energie, vlastní symetrie, vlastní typy částic a polí, a co je zásadní, svou
  vlastní kosmologickou konstantu. Každý z těchto vesmírů může existovat, i ty, ve kterých je
  kosmologická konstanta velice malá. 
  
  Antropický princip navrhl jako první Robert Dicke a dále ho rozpracoval Brandon Carter. Princip
  říká, že vesmír je takový, jaký je, proto, že kdyby byl jiný, nebyli bychom tu my. Můžeme totiž
  existovat jako bytosti s vědomím jen díky tomu, že různé veličiny - fyzikální konstanty,
  charakteristiky částic, energetické škály - mají právě ty správné hodnoty. Mezi tyto veličiny
  patří i kosmologická konstanta. Možných vesmírů je nespočet, jenže my můžeme existovat jen v
  málokterém z nich, jen v takovém, kde mají základní konstanty včetně kosmologické tu správnou
  hodnotu. Je tedy přirozené, že žijeme právě v takovém vesmíru, byť byl v multiverzu sebevzácnější.
  
  Někteří badatelé soudí, že když se kosmologie stala tak bohatou a komplexní záležitostí, ocitli
  jsme se na samé hranici toho, co si zaslouží název „věda“. Jedním ze skeptiků, kteří se domnívají,
  že současný přístup už jde příliš daleko, je relativista George Ellis, který spolu s Hawkingem a
  Penrosem prokázal koncem šedesátých let nezbytnost singularit. Stál v přední linii snahy používat
  vesmír jako obrovskou laboratoř pro zkoumání základů Einsteinovy teorie, ale o multiverzu říká:
  „Nevěřím, že existence těch ostatních vesmírů byla prokázána, ani že kdy prokázána bude.“
  „Argument s multiverzem je dobře založená filosofická hypotéza, ale protože ji nelze
  experimentálně testovat, nepatří do vědeckého uvažování.“ V „krajině možností“ lze předpovědět, že
  někde existuje cokoli. Pocit, že věci zašly příliš daleko, zavládl i mezi strunaři. Tento nový
  přístup opouští konečný cíl moderní fyziky: najít jednoduché a jednoznačné vysvětlení všech
  základních sil včetně gravitace. Přijmout myšlenku multiverza znamená se tohoto cíle vzdát.
  Dokonce i Ed Witten, papež moderní teorie strun, není s tímto vývojem spokojen a říká: „Doufám, že
  současná diskuse teorie strun nevede nesprávným směrem.“
  
  Přesto popularita multiverza roste. Tato hypotéza řeší některé velké problémy, například proč
  existuje kosmologická konstanta a proč mají přírodní konstanty právě ty hodnoty, které měříme.
  Běžně se dnes se zprávami o paralelních vesmírech a důkazech o bohatosti a pluralitě světů
  setkáváme v tisku i v jiných médiích. Je to samozřejmě skvělý prostor pro nejrůznější spekulace i
  ještě neopotřebovaná tematika pro romanopisce. Podle Ellise to ale prostě není věda. 
  
  V roce 2009 jsem navštívil Príncipe, malý svěží flíček zeleně u západní Afriky. Právě odtamtud o
  devadesát let dříve poslal Arthur Eddington tehdejšímu prezidentu Královské astronomické
  společnosti Franku Dysonovi stručnou telegrafickou zprávu: „Mraky se trhají. Nadějné.“Následující
  Eddingtonova měření ohybu světelných paprsků při zatmění Slunce udělaly z Einsteinovy obecné
  teorie relativity symbol moderní vědecké teorie a z Einsteina a Eddingtona se staly mezinárodní
  superhvězdy. 
  
  Do ostrovního státečku São Tomé a Príncipe jsem cestoval s pestrou směsicí Britů, Portugalců,
  Brazilců a Němců za účelem odhalení pamětní desky, věnované Královskou astronomickou společností a
  Mezinárodní astronomickou unií, na místě, kde Eddington a Cottingham prováděli svá měření. São
  Tomé a Príncipe se po letech koloniální nadvlády osvobodily a vytvořily na nějakou dobu další
  africký socialistický stát, který se později připojil k světu volného trhu. Světlé nové rekreační
  objekty bohatých Angolanů podivně kontrastují s velkými sešlými statkářskými domy koloniální éry.
  
  Doufal jsem, že hlavní dům v Roça Sundy, kde Eddington konal svá měření, bude v lepším stavu než
  většina opuštěných koloniálních stavení roztroušených po zeleném venkově. Regionální prezident
  ostrůvku Príncipe, na kterém žije sotva pět tisíc lidí, si jej totiž vybral za své prázdninové
  sídlo. Ale i to se ukázalo být jen zbožným přáním - ve skutečnosti i tento dům byl zchátralý,
  neudržovaný a neobyvatelný. 
  
  São Tomé a Príncipe, krásný zapadlý koutek světa, na mne velice zapůsobil a to z několika důvodů.
  Předně se tam na začátku dvacátého století narodila moje babička a od ní jsem o tomto místě mnoho
  slyšel. Ale důležitější bylo, že jsem byl na místě, kde se měnila historie. Zde byla prokázána
  správnost obecné teorie relativity s takovou jistotou, jak jen vědecká teorie může být ověřena.
  Právě zde se obecná relativita stala faktem. 
  
  Byly zde zachovány některé památky na dobu, kdy tu pobýval Eddington, například tenisový kurt s
  rozpraskaným betonem, jenž bojoval ztracenou bitvu s neúprosnou vegetací deroucí se ze země. Vše
  kolem tonulo v záplavě zeleně. Velice se to lišilo od šlechtěné krajiny s vysušenými močály, kde
  Eddington prožil většinu života. Díky naší návštěvě tam přibyla lesklá deska připomínající
  Eddingtonův úspěch. Doufali jsme, že všem návštěvníkům tohoto vzdáleného místa připomene, jak byl
  tento objev důležitý. 
  
  Ohlédneme-li se zpět do roku 1919, žasneme, jak se Einsteinovy a Eddingtonovy myšlenky od té doby
  rozvinuly. Jednoduchý nápad, že světelný paprsek se v zakřiveném prostoročase bude ohýbat, vedl o
  devadesát let později k jednomu z nejvýkonnějších nástrojů v astronomii, který posledních dvacet
  let umožňuje získat studiem ohybu světla zásadní poznatky o stavbě vesmíru. Pozorujeme-li hvězdy v
  blízkých galaxiích, ohyb jimi vyzařovaného světla odhalí rozložení hmoty podél paprsků, a to i
  hmoty, která sama nesvítí. Temná hmota tak hraje roli Slunce v Eddingtonově experimentu, ohýbá
  světlo z hvězd podobně, jako kdyby světlo procházelo čočkou. Hovoří se proto o gravitačních
  čočkách. Ve větší škále se užívají jako čočky kupy galaxií sestávající z tisíců galaxií. Tato
  gigantická uskupení vytvářejí mohutná zakřivení struktury prostoročasu, která ohýbají světlo z
  galaxií ještě vzdálenějších, a tento ohyb dovoluje astronomům „zvážit“ čočky, které jej způsobily.
  
  Proč se ale zastavit na této škále? S typickou vědeckou arogancí začali astronomové a kosmologové
  proměřovat zakřivení světelných paprsků od všech objektů, které jen lze pozorovat. Na tomto
  základě se snaží získat podrobný popis struktury prostoročasu všude kolem nás. Einsteinova a
  Eddingtonova myšlenka povznesená na vyšší úroveň dnes dovoluje podrobně zkoumat geometrii
  prostoročasu kolem nás i testovat, zda k jejímu popisu používáme správných zákonů. 
  
  V průběhu oslav na Príncipe měl kdekdo na rtech Einsteinovo a Eddingtonovo jméno. Samozřejmě, že
  jen málokdo z obyvatel malinkého ostrůvku věděl, o čem se tam mluví. Neobratné proslovy místních a
  cizích hodnostářů toho mnoho neobjasnily. Na oslavách byla hejna dětí a mládeže. Tito mladí lidé
  sice moc nevěděli, co se vlastně oslavuje, ale všichni slyšeli jméno Einstein, a někteří dokonce
  slyšeli o proslulém Angličanovi Eddingtonovi, který před lety ostrov navštívil. Všichni ovšem
  souhlasili, že se jedná o dobrou věc, která ostrůvku dodá na vážnosti. 
  
  Když jsem pozoroval dav účastníků těchto podivných ezoterických oslav, uvědomil jsem si
  univerzalitu i určitou demokratičnost Einsteinovy teorie. Je sice obtížná a nesnadno sdělitelná,
  dá se však shrnout v malém počtu rovnic, jež lze včetně definic jednotlivých veličin zapsat na pár
  stránkách. Historie vývoje obecné teorie relativity se odehrávala na několika kontinentech a
  vytvářel ji vskutku internacionální soubor velmi odlišných postav. Vystupují zde britští
  astronomové, ruský meteorolog, belgický kněz, novozélandský matematik, německý voják, indické
  zázračné dítě, americký expert na atomovou bombu, jihoafrický kvaker a řada dalších osobností,
  které svedla dohromady elegance a síla Einsteinovy teorie. Přemýšlel jsem o tom, jak i dnes nás
  Einsteinova teorie nutí dívat se na vesmír v těch největších měřítkách. Úlohu Príncipe dnes hrají
  třeba observatoře na jihu Afriky nebo v australské poušti a současné teleskopy využívají těch
  nejnovějších výdobytků techniky jednadvacátého století. Zatímco Eddington používal optický
  teleskop, tedy soustavu čoček s okulárem a záznamem na fotografickou desku, nová fáze zkoumání
  vesmíru se hodně spoléhá na rádiové antény v různém uspořádání. Radioteleskopy už daly obecné
  relativitě hodně, teď však chtějí jít radioastronomové dále, než se kdy pokládalo za možné.
  Uvažuje se o vybudování desítek tisíc rádiových antén, rozsetých ve vzdálenostech stovek až tisíců
  kilometrů od sebe. Vytvoří tak síť zvanou SKA podle anglického Square Kilometer Array. Sběrná
  plocha všech antén dohromady má totiž být čtvereční kilometr veliká. Toto zařízení bude rozloženo
  na dvou kontinentech. Některé radioteleskopy budou ležet na rozsáhlých pláních západní Austrálie,
  další budou pokrývat jižní Afriku. Srdce zařízení má být v polopoušti Karoo v Jihoafrické
  republice, ale řada talířových antén bude rozptýlena po celém kontinentě - v Namibii, Mosambiku,
  Ghaně, Keni a na Madagaskaru. Bude to skutečně kontinentální africký projekt. A podobně jako
  Eddingtonovi posloužil k prokázání správnosti obecné teorie relativity ostrůvek Príncipe, projekt
  SKA by měl být schopen testovat Einsteinovu teorii v kosmických škálách s dosud nedosažitelnou
  přesností. Uměl by prokázat, zda v Einsteinově velké myšlence lze skutečně najít nějaké trhliny,
  jež je třeba zacelit. Odhalil by i nepolapitelné gravitační vlny, které stále čekají na své
  objevení. Snad by mohl i odkrýt povahu záhadné temné energie, jež nalezla, jak se zdá, pevné místo
  v stávajících kosmologických modelech. 
  
  Tu noc, kdy jsme slavili Eddingtonův a Einsteinův kolosální úspěch, jsem přemýšlel o tom, že jsme
  teprve na počátku toho, co nám teorie prostoročasu řekne o vesmíru. Jednadvacáté století bude
  zcela určitě stoletím Einsteinovy obecné teorie relativity a já jsem šťasten, že žiji v době, kdy
  tolik nových věcí čeká na objevení. Sto let od chvíle, kdy Einstein přišel s konečnou formou své
  teorie, žijeme v oprávněném očekávání, že dojde k něčemu fantastickému.