% !TeX program = lualatex
% !TeX root = luaking.tex
% !TeX encoding = UTF-8
% !TeX spellcheck = cs_CZ
%---------------------------------------------------------------------------------------------------
% file kul1ch02.tex
\graphicspath{{../src/FYZ/img/}}
%---------------------------------------------------------------------------------------------------
%=============================== Kapitola: Astrofyzika =============================================
\setchaptertoc
\chapter{Hvězdy v Galaxii}\label{kulIchII}
  Pohled na noční oblohu posetou hvězdami zná snad každý. Mnohem méně lidí však už ví, že není
  hvězda vše, co se za ni vydává a poznat, že některé světlé body jsou světlem ne jedné, ale světlem
  miliard hvězd nám ztěžuje jejich obrovská vzdálenost.

  Slunce, okem viditelné hvězdy i naprostá většina hvězd viditelných dalekohledem, spolu s
  mezihvězdnou látkou a skrytou hmotou, která se projevuje jen gravitačně, vytváří komplexní, bohatě
  strukturovaný a vysoce organizovaný gravitačně vázaný systém nazývaný
  \textbf{Galaxie}\footnote{Slovo galaxie bylo odvozeno z řeckého názvu naší vlastní galaxie Mléčné
  dráhy Κύκλος γαλακτικός (Κyklos galaktikos). Galaxii se
  někdy nesprávně říká Mléčná dráha nebo soustava Mléčné dráhy. Mléčná dráha je ovšem jenom součástí
  Galaxie. Abychom zabránili nedorozuměním, je třeba rozlišovat mezi Galaxií a galaxiemi použitým
  vhodných přívlastků, např. \uv{naše Galaxie}, \uv{cizí galaxie} případně \uv{extragalaktické
  soustavy}}. 

  Výzkum struktury Galaxie (na obloze viditelná jako pruh podél ekliptiky - nazývané jako Mléčná
  dráha), našeho hvězdného domova, je paradoxně ztížen skutečností, že samo Slunce leží uvnitř
  tohoto systému; soustavu pak vidíme (nebo v důsledku mezihvězdné extinkce také nevidíme) ve všech
  směrech \cite[s.~268]{Mikulasek2000}. Ostatní galaxie se nám jeví na obloze jako mlhavé obláčky.
  Proto byly dlouho považovány za mlhoviny (ještě Hubble svou monografii o galaxiích nazval
  \emph{Říše mlhovin - The Realm of Nebulae}). Hrubý odhad ukazuje, že se v pozorovaném vesmíru (do
  vzdálenosti 10 miliard let) vyskytuje až 120 miliard galaxií.

  \section{Naše Galaxie}\label{kulIchIIsecI}
    Pokud se při jasné, bezoblačné noci, daleko od světelného znečištění podíváme na oblohu, můžeme
    spatřit slabý pás světla táhnoucí se přes oblohu. Jedná se o Mléčnou dráhu, stříbřitý pás, který
    obepíná celou oblohu. Klene se od souhvězdí Štíra nahoru přes Orla, Labuť, Kassiopeji až do
    zimního souhvězdí Jednorožce. Potom klesá pod obzor, prochází Plachtami, Jižním křížem,
    Kentaurem, Střelcem až do Štíra. V rovníkových oblastech, odkud je vidět téměř celá obloha, ji
    lidé nazývají Nebeský pás, Zářící pás či Mléčný kruh, jinde dostala jméno Nebeská řeka či
    Stříbrná řeka. Někde ji zase považují za nebeskou cestu, pěšinu či dráhu, jako je tomu v
    indoevropských zemích (tedy i ve slovanských jazycích). Nikdo ale neznal její podstatu. Teprve
    výzkumy v 20. století ukázaly, že Mléčná dráha je naše galaxie (Galaxie), viděná od jedné z
    jejích 200 miliard hvězd - ze sluneční soustavy. Jsme blízko roviny souměrnosti disku Galaxie,
    takže Mléčná dráha rozděluje oblohu na dvě stejné poloviny. Galaxie je domovem naší sluneční
    soustavy, stejně jako více než 200 miliard dalších hvězd a jejich planet, tisíců hvězdokup a
    mlhovin. Jedná se o spirální galaxii s centrální příčkou a radiálními rameny, které začínají ve
    středu a vytváří spirálovitý tvar. Slunce (a naše sluneční soustava) se otáčí kolem středu
    Galaxie - galaktického středu, přičemž pro dokončení jednoho oběhu potřebuje přibližně 220
    milionů let. Domníváme se, že Slunce a sluneční soustava obíhá kolem středu Galaxie konstantní
    rychlostí. Za svoji existenci tak vykonalo méně než 25 oběhů kolem středu Galaxie.
    \begin{figure*}
      \centering
      \luafigure[1]{fyz_fig916.jpg}
      \caption{Mléčná dráha}
      \label{fyz:fig916}
    \end{figure*}

    \subsection{Klasifikace}
      Naše Galaxie je obrovská galaxie, její hmotnost včetně korony je (3-6).1012 (3-6 biliony)
      hmotností Sluncí (hmotnost Slunce - 1,985.1030 kg) a její průměr je přibližně 90 000 světelných
      let. Radioastronomové zkoumali rozložení vodíkových mračen a na základě tohoto měření usuzují,
      že se jedná o spirální galaxii s  příčkou Hubbleova typu SBc. Stoupající počet výzkumů dává
      některé důkazy, které svědčí o tom, že Galaxie vypadá jako M61 nebo M83.

  \begin{figure*}
    \centering
    \luafigure[1]{fyz_fig896.jpg}
    \caption{Vesmír}
    \label{fyz:fig896}
  \end{figure*}