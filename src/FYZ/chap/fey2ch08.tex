% !TeX spellcheck = cs_CZ
%{\tikzset{external/prefix={tikz/FYZII/}}
% \tikzset{external/figure name/.add={ch06_}{}}
%---------------------------------------------------------------------------------------------------
% file fey1ch08.tex
%---------------------------------------------------------------------------------------------------
%====================Kapitola: Elektrostatická energie =============================================
\setchaptertoc
\chapter{Elektrostatická energie}\label{fyz:IIchapVI}
  \section{Elektrostatická energie nábojů. Homogenní koule}\label{fyz:IIchapVIsecI}
    Jedním z nejzajímavějších a nejužitečnějších objevů v mechanice byl zákon zachování energie.
    Vzorce pro kinetickou a potenciální energii mechanické soustavy nám pomáhají objevit souvislosti
    mezi stavy soustavy ve dvou různých časech, aniž bychom museli vnikat do podrobností toho, co se
    mezi tím děje. Nyní se chceme zabývat energií elektrostatických soustav. I v nauce o elektřině
    bude princip zachování energie užitečný při objevování mnoha zajímavých věcí.

    Zákon o energii vzájemného působení je v elektrostatice velmi jednoduchý; vlastně jsme ho již
    probírali. Představte si, že máme dva náboje \(q_1\) a \(q_2\) vzdálené od sebe o \(r_{12}\).
    Tato soustava se vyznačuje nějakou energií, neboť na to, aby se oba náboje přivedly do jejich
    současné vzájemné polohy, bylo třeba vynaložit určité množství práce. Práci, která je vykonána
    při přiblížení dvou nábojů z velké vzdálenosti, jsme už počítali. Je rovna
    \begin{equation}\label{fyz:eq868}
      \dfrac{q_1q_2}{4πϵ_0r_{12}}.      
    \end{equation}
    Kromě toho z principu superpozice víme, že v případě mnoha nábojů je celková síla působící na
    každý z nich rovna součtu sil, kterými na něj působí všechny ostatní náboje. Z toho vyplývá, že
    celková energie soustavy více nábojů je rovna součtu členů pocházejících ze vzájemné interakce
    každého páru nábojů. Jsou-li \(q_i\) a \(q_j\) některé dva z těchto nábojů a \(r_{12}\) je
    vzdálenost mezi nimi (\ref{fyz:fig178}), energie tohoto páruje
    \begin{equation}\label{fyz:eq869}
      \dfrac{q_iq_j}{4πϵ_0r_{ij}}.      
    \end{equation}

    \begin{figure}[ht!]  %\ref{fyz:fig178}
      \centering
      \includegraphics[width=0.6\linewidth]{fyz_fig178.pdf}
      \caption{Elektrostatická energie soustavy částic je rovna součtu elektrostatických energií
              všech párů částic v soustavě (\cite[s.~140]{Feynman02}).}
      \label{fyz:fig178}
    \end{figure}

    Výsledná elektrostatická energie \(W\) je součtem energií všech možných párů nábojů v soustavě:
    \begin{equation}\label{fyz:eq870}
      W=\sum_{\mathclap{\substack{\text{všechny}\\\text{páry}}}}\dfrac{q_iq_j4πϵ_0}{r_{ij}}.
    \end{equation}
    Jde-li o rozdělení nábojů specifikované hustotou náboje \(ρ\), je samozřejmě nutné nahradit sumu
    ve vzorci (\ref{fyz:eq870}) integrálem.

    My se budeme touto energií zabývat ze dvou hledisek. Jedním je \emph{využití} pojmu energie v
    elektrostatických úlohách a druhým jsou různé způsoby výpočtu energie. Někdy je snazší vypočítat
    práci vykonanou v nějakém speciálním případě, než vyčíslit sumu nebo příslušný integrál ve
    vzorci (\ref{fyz:eq870}). Jako příklad vypočtěme energii potřebnou na shromáždění náboje do
    koule s homogenní hustotou náboje. Je rovna práci vykonané při přibližování nábojů z nekonečna.

    Představte si, že kouli vytváříme postupným přikládáním tenkých kulových slupek s
    infinitezimální tloušťkou na sebe. V každém stádiu tohoto procesu bereme malé množství náboje a
    přikládáme jej ve tvaru tenké kulové slupky sahající od \(r\) do \(r + \dd{r}\).

    \begin{figure}[ht!]  %\ref{fyz:fig179}
      \centering
      \includegraphics[width=0.6\linewidth]{fyz_fig179.pdf}
      \caption{Energii homogenně nabité koule můžeme počítat tak, že si představíme kouli, jako by
              byla složena ze vzájemně na sebe přiléhajících kulových slupek.
              (\cite[s.~141]{Feynman02}).}
      \label{fyz:fig179}
    \end{figure}

    Pokračujeme tak dlouho, dokud nedosáhneme konečného poloměru a (obr. \ref{fyz:fig179}). Je-li
    náboj koule ve stádiu, kdy byla vytvořena do poloměru \(r\), je při přinášení dalšího náboje
    \(\dd{Q}\) vykonávána práce.
    \begin{equation}\label{fyz:eq871}
      \dd{W}=\dfrac{Q}{r^4πϵ_0r}\dd{Q}.
    \end{equation}
    Je-li hustota náboje v kouli \(ρ\), náboj \(Q_r\) je
    \begin{equation*}
      Q_r=ρ\cdot\dfrac{4}{3}πr^3,
    \end{equation*}
    a náboj \(\dd{Q}\) je vyjádřen takto:
    \begin{equation*}
      \dd{Q}=ρ⋅4πr^2\dd{r}.
    \end{equation*}
    Rovnost (\ref{fyz:eq871}) pak získá tvar
    \begin{equation}\label{fyz:eq872}
      \dd{W}=\dfrac{4πρ^2r^4\dd{r}}{3ϵ_0}.
    \end{equation}
    Celková energie potřebná na vytvoření koule je rovna integrálu \(\dd{W}\) od \(r = 0\) do \(r =
    a\), tj.
    \begin{equation}\label{fyz:eq873}
      W=\dfrac{4πρ^2a^5}{15ϵ_0}.
    \end{equation}
    Chceme-li výsledek vyjádřit pomocí celkového náboje \(Q\) koule, dostaneme
    \begin{equation}\label{fyz:eq874}
      W=\dfrac{3}{5}\dfrac{Q^2}{4πϵ_0a}.
    \end{equation}
    Energie je tedy přímo úměrná druhé mocnině celkového náboje a nepřímo úměrná poloměru. Vztah
    (\ref{fyz:eq874}) můžeme interpretovat také tak, že podle něj je střední hodnota veličiny
    (\(1/r_{ij}\)) pro všechny páry bodů v kouli \(3/5 a\).

  \twocolumn[\section{Energie kondenzátoru. Síly působící na nabité vodiče}\label{fyz:IIchapVIsecII}]
    Nyní uvažujme o energii potřebné k nabití kondenzátoru. Vezmeme-li od jednoho z vodičů tvořících
    kondenzátor náboj \(Q\) a přeneseme-li ho na druhý vodič, vznikne mezi nimi rozdíl potenciálů
    \begin{equation}\label{fyz:eq875}
      U = \dfrac{Q}{C},
    \end{equation}
    kde \(C\) je kapacita kondenzátoru. Kolik práce je vykonáno při nabíjení kondenzátoru? Budeme
    postupovat stejně jako v případě koule a představíme si, že kondenzátor se nabíjel přenášením
    náboje po malých částech \(\dd{Q}\), z jedné jeho desky na druhou. Na přenesení náboje
    \(\dd{Q}\) je spotřebována práce
    \begin{equation*}
      \dd{W} = U\dd{Q}.
    \end{equation*}
    Dosazením \(U\) z (\ref{fyz:eq875}) tento vztah získá tvar
    \begin{equation*}
      \dd{U} = \dfrac{Q\dd{Q}}{C}.
    \end{equation*}
    Když potom integrujeme od nulového do konečného náboje \(Q\) dostaneme
    \begin{align}
      W &= \dfrac{1}{2}\dfrac{Q^2}{C}.  \label{fyz:eq876}     \\
      \shortintertext{Tuto energii je možné napsat i ve tvaru}
      W &= \dfrac{1}{2}CU^2.            \label{fyz:eq877}
    \end{align} 

    Vzpomeneme-li si, že kapacita vodivé koule (vzhledem k nekonečnu) je
    \begin{equation*}
      C_{\text{koule}} = 4πϵ0a,
    \end{equation*}
    můžeme ze vzorce (\ref{fyz:eq876}) ihned dostat vztah pro energii nabité koule:
    \begin{equation}\label{fyz:eq878}
      W=\dfrac{1}{2}\dfrac{Q^2}{4πϵ_0a}.
    \end{equation}
    Ovšem toto je i vztah pro energii \emph{tenké kulové slupky} s celkovým nábojem \(Q\) tato
    energie představuje \(5/6\) energie homogenně nabité koule, vyjádřené vztahem (\ref{fyz:eq874}).

    Nyní se zabývejme aplikacemi pojmu elektrostatické energie. Zkoumejme následující otázky: Jaká
    síla působí mezi elektrodami kondenzátoru? Nebo: Jaký moment síly vzhledem k nějaké ose působí
    na nabitý vodič v přítomnosti jiného vodiče s opačným nábojem? Takové otázky lze snadno
    zodpovědět použitím našeho výsledku (\ref{fyz:eq877}) pro elektrostatickou energii kondenzátoru
    spolu s principem virtuální práce (viz kapitoly \ref{fyz:IchapII}, a \ref{fyz:chap_fey_work},
    díl \ref{part:FYZI}).

    Použijeme tuto metodu na určení síly působící mezi deskami rovinného kondenzátoru.
    Představíme-li si, že mezera mezi deskami se zvětšila o malou hodnotu \(\Delta z\), mechanická
    práce vynaložená vnější silou na posunutí desek byla
    \begin{equation}\label{fyz:eq879}
      ΔA=FΔz,
    \end{equation}
    kde \(F\) je síla působící mezi deskami. Tato práce musí být rovna změně elektrostatické energie
    kondenzátoru.

    Podle vztahu (\ref{fyz:eq876}) měl kondenzátor původně energii
    \begin{equation*}
      W = \dfrac{1}{2}\dfrac{Q^2}{C}.
    \end{equation*}
    Změna energie (nedopustíme-li, aby se náboj změnil) je
    \begin{equation}\label{fyz:eq880}
      ΔW=\dfrac{1}{2}Q^2Δ\left(\dfrac{1}{C}\right).
    \end{equation}
    Porovnáním (\ref{fyz:eq879}) a ({fyz:eq880}) dostaneme
    \begin{align}
      FΔz &=\dfrac{Q^2}{2}Δ\left(\dfrac{1}{C}\right). \label{fyz:eq881} \\
      \shortintertext{což je možné napsat ve tvaru}
      FΔz &=−\dfrac{Q^2}{2C^2}ΔC.                     \label{fyz:eq882}
    \end{align}
    Působící síla, samozřejmě, vyplývá z rozdělení nábojů na deskách, ale jak je vidět, nemusíme se
    starat o jejich detailní rozdělení; všechno, co potřebujeme, je obsaženo v kapacitě \(C\)

    \begin{figure}[ht!]  %\ref{fyz:fig180}
      \centering
      \includegraphics[width=0.6\linewidth]{fyz_fig180.pdf}
      \caption{Jaký moment síly působí na otočný kondenzátor? (\cite[s.~144]{Feynman02}).}
      \label{fyz:fig180}
    \end{figure}

    Je zřejmé, že tuto myšlenku lze rozšířit na vodiče jakéhokoliv tvaru a na ostatní složky síly.
    Ve vztahu (\ref{fyz:eq881}) zaměníme \(F\) složkou, kterou hledáme, a \(\Delta z\) nahradíme
    malým posunutím v odpovídajícím směru. Nebo máme-li elektrodu vyznačující se nějakou osou a je
    třeba najít moment síly \(τ\), napíšeme virtuální práci ve tvaru
    \begin{equation*}
      ΔA=τΔθ,
    \end{equation*}
    kde \(Δθ\) je malé \emph{úhlové posunutí}. V tomto případě musí \(Δ(1/C)\) přirozeně
    představovat změnu veličiny \(1/C\) příslušnou pootočení \(Δθ\). Tak bychom mohli najít moment
    působící na pohyblivé desky v otočném kondenzátoru takového typu, jako je na obr.
    \ref{fyz:fig181}.
    
    Vrátíme se ke speciálnímu případu kondenzátoru s rovnoběžnými rovinnými elektrodami. Pro jeho
    kapacitu můžeme použít vzorec odvozený v kapitole \ref{fyz:IIchapV}:
    \begin{equation}\label{fyz:eq883}
      \dfrac{1}{C}=\dfrac{d}{ϵ_0S},
    \end{equation}
    kde \(S\) je plošný obsah každé desky. Zvětšíme-li mezeru mezi deskami o \(\Delta z\), bude
    platit
    \begin{equation*}
      Δ\left(1C\right)=\dfrac{Δz}{ϵ_0S}.
    \end{equation*}
    Z rovnice (\ref{fyz:eq881}) dostaneme, že síla mezi deskami je
    \begin{equation}\label{fyz:eq884}
      F=\dfrac{Q^2}{2ϵ_0S}.
    \end{equation}
    Všimněme si výrazu (\ref{fyz:eq881}) trochu blíže a podívejme se, zda lze říci, jak tato síla
    vzniká. Vyjádříme-li náboj na jedná desce ve tvaru
    \begin{equation*}
      Q=σS,
    \end{equation*}
    
    \begin{figure}[ht!]  %\ref{fyz:fig181}
      \centering
      \includegraphics[width=0.6\linewidth]{fyz_fig181.pdf}
      \caption{Intenzita elektrického pole se změní při průchodu vrstvou plošného náboje
              existujícího na povrchu vodiče z nuly na hodnotu \(E_0 =
              \frac{\sigma}{\varepsilon_0}\) (\cite[s.~145]{Feynman02}).}
      \label{fyz:fig181}
    \end{figure}
    
  \section{Elektrostatická energie iontového krystalu}\label{fyz:IIchapVIsecIII}

    \begin{figure}[ht!]  %\ref{fyz:fig182}
      \centering
      \includegraphics[width=0.6\linewidth]{fyz_fig182.pdf}
      \caption{Řez krystalem kuchyňské soli v atomovém měřítku. Šachovnicové uspořádání iontů \ce{Na} 
              a \ce{Cl} je stené v obou na sebe kolmých řezech krystalem (obr. 1.7 díl I)
              (\cite[s.~146]{Feynman02}).}
      \label{fyz:fig182}
    \end{figure}

  \section{Elektrostatická energie v atomových jádrech}\label{fyz:IIchapVIsecIV}

    \begin{figure}[ht!]  %\ref{fyz:fig183}
      \centering
      \includegraphics[width=0.6\linewidth]{fyz_fig183.pdf}
      \caption{Síla mezi dvěma protony závisí na všech možných parametrech.
              (\cite[s.~148]{Feynman02}).}
      \label{fyz:fig183}
    \end{figure}

    \begin{figure}[ht!]  %\ref{fyz:fig184}
      \centering
      \includegraphics[width=0.6\linewidth]{fyz_fig184.pdf}
      \caption{Energetické hladiny jader \ce{^{11}B} a \ce{^{11}C} (hodnty udané v
              \si{\mega\electronvolt}). Základní stav \ce{^{11}C} leží o
              \SI{1.982}{\mega\electronvolt} výše než základní stav \ce{^{11}B}
              (\cite[s.~149]{Feynman02}).}
      \label{fyz:fig184}
    \end{figure}
    
  \section{Energie v elektrostatickém poli}\label{fyz:IIchapVIsecV}

    \begin{figure}[ht!]  %\ref{fyz:fig185}
      \centering
      \includegraphics[width=0.6\linewidth]{fyz_fig185.pdf}
      \caption{Každý element objemu \(\dd{V} = \dd{x}\dd{y}\dd{z}\) v elektrickém  poli obsahuje
      energii \(\varepsilon_0/2E^2\dd{V}\) (\cite[s.~154]{Feynman02}).}
      \label{fyz:fig185}
    \end{figure}
    
  \section{Energie bodového náboje}\label{fyz:IIchapVIsecVI}

\todo[inline]{Kapitola fey2ch08 je nedodělaná, obsahuje pouze obrázky}
%} %tikzset
%~~~~~~~~~~~~~~~~~~~~~~~~~~~~~~~~~~~~~~~~~~~~~~~~~~~~~~~~~~~~~~~~~~~~~~~~~~~~~~~~~~~~~~~~~~~~~~~~~~