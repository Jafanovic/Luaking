% !TeX spellcheck = cs_CZ
\begin{mdframed}[style=mdexam]
  \begin{example}\label{FYZ:exam031}
    \emph{Automobil, jehož motor má výkon \qty{40}{\kW}  a účinnost \qty{28}{\percent}, se pohybuje
    rychlostí \qty{120}{\km\per\hour}. Jakou dráhu může urazit, jestliže zásoba paliva v nádrži je
    \qty{30}{\litre}? Výhřevnost paliva je \qty{46}{\mega\joule\per\kg}, jeho hustota
    \qty{750}{\kg\per\cubic\m}. (\cite[s.~33]{Bartuska1997})}
    
    \textbf{Řešení:}\newline 
    Účinnost automobilového motoru je určena vztahem:
    \begin{equation*}
      \eta = \dfrac{W}{Q_1},
    \end{equation*}
    kde \(W\) je práce, kterou vykoná motor automobilu při jízdě na dráze \(s\) a \(Q_1 = m_1H\)
    teplo, které se při jízdě po této dráze uvolní spálením paliva o hmotnosti \(m_1\) a výhřevnost
    \(H\). Pro práci \(W\) proto platí: 
    \begin{equation*}
     W = \eta Q_1 = \eta m_1H = \eta\varrho VH.
    \end{equation*}
    Výkon automobilového motoru při jízdě na dráze \(s = vt\) je \(P=\dfrac{W}{t}\). Odtud pro práci
    \(W\) dostáváme
    \begin{equation*}
      W = Pt = P\dfrac{s}{t}. 
    \end{equation*}
    Z předchozích rovnic pak vyplývá
    \begin{equation*}
      P\dfrac{s}{t} = \eta\varrho VH \;\rightarrow\; s = \dfrac{\eta\varrho VHv}{P}
    \end{equation*}
    číselně (nezapomeňme převést \qty{120}{\km\per\hour} na \qty{33.3}{\m\per\s})
    \begin{align*}
       s  &=\dfrac{\num{0.33}\cdot750\cdot\num{30e-3}\cdot\num{46e6}\cdot\num{33.3}}{\num{40e3}}  \\
       s  &\approx \qty{284}{\km}    
    \end{align*}
    Zkontrolujeme jednotky
    \begin{equation*}
      \dfrac{\unit{\kg\per\cubic\m}\cdot\unit{\cubic\m}\cdot\unit{\joule\per\kg}\cdot\unit{\m\per\s}}
            {\unit{\joule\per\s}}  
    \end{equation*}  
    Automobil se zásobou paliva o objemu \qty{30}{\litre} urzí dráhu \qty{284}{\km}.
  \end{example} 
\end{mdframed}