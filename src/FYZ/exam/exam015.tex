\begin{fyzexam}{Představme si, že pozorujeme dva čluny na řece. Obrázek 1.1 znázorňuje řeku šířky
  \(L\), která teče rychlostí \(v\). Dva čluny startují od téhož břehu řeky stejně velkou rychlostí
  \(V\). Člun \(A\) přepluje řeku do místa na druhém břehu přímo naproti výchozímu bodu, a poté se
  vrátí, zatímco člun \(B\) pluje po proudu na vzdálenost \(L\) a rovněž se vrátí. Vypočítejme dobu,
  potřebnou na obě cesty tam a zpět. \hfill\cite[s.~18]{Beiser1975}}{exam015} 

  {\centering\captionsetup{type=figure}\luafigure[1]{fyz_fig0897.png}\par}

  \vspace{1em}
  Začneme člunem \(A\). Směřuje-li člun \(A\) kolmo napříč řeky, unáší ho proud od cíle na protějším
  břehu (obr. \ref{fyz:fig0898}). Musí proto mířit poněkud proti proudu, aby se vliv proudu
  kompenzoval. K vyrovnání odchylky musí být tato složka rychlosti proti proudu rovna přesně \(-v\),
  aby kompenzovala říční proud o rychlosti \(v\); zbývající složka \(V'\) je pak čistou rychlostí
  člunu napříč řeky. Podle obr. \ref{fyz:fig0898} tyto rychlosti spolu souvisejí vzorcem
  \begin{equation*}
    V^2 = V'^2 + v^2,
  \end{equation*}
  takže skutečná rychlost plavby přes řeku je zde
  \begin{equation*}
    V'^2 = \sqrt{V^2 - v^2} = V\sqrt{1 - \frac{v^2}{V^2}}.
  \end{equation*}

  {\centering
  \captionsetup{type=figure}
  \luafigure[1]{fyz_fig0898.png}
  \captionof{figure}{Má-li člun \(A\) přeplout kolmo řeku, musí mířit šikmo proti jejímu proudu, aby
            vyrovnal jeho vliv.  
  \cite[s.~18]{Beiser1975}
  \label{fyz:fig0898}} \par}
  \vspace{1em}
  Čas potřebný k prvnímu přeplutí řeky, je tedy roven vzdálenosti \(L\) dělené rychlostí \(V'\).
  Jelikož cesta zpátky trvá přesně stejně dlouho, je celková doba \(t_A\) pro cestu tam a zpět
  dvojnásobkem \(L/V'\), neboli 
  \begin{equation*}
    t_A = \frac{2L/V}{\sqrt{1 - \frac{v^2}{V^2}}}.
  \end{equation*}

  Situace člunu \(B\) je poněkud jiná. Pluje-li po proudu, rovná se jeho rychlost vůči břehu součtu
  vlastní rychlosti člunu \(V\) a rychlosti \(v\) říčního proudu (obr. \ref{fyz:fig0899}), takže
  urazí vzdálenost \(L\) po proudu během doby
  \begin{equation*}
    \frac{L}{V+v}.
  \end{equation*}
  Na zpáteční cestě se však člun \(B\) pohybuje vzhledem ke břehu rychlostí, která se rovná rozdílu
  jeho vlastní rychlosti \(V\) a rychlosti \(v\) říčního proudu. Potřebuje tudíž dobu
  \begin{equation*}
    \frac{L}{V-v},
  \end{equation*}
  aby urazil proti proudu vzdálenost \(L\) a vrátil se do výchozího bodu. Celková doba to cesty je
  součtem obou těchto časů
  \begin{equation*}
    t_B = \frac{L}{V+v} + \frac{L}{V-v},
  \end{equation*}
  Převedení na společného jmenovatele (V + v) (V - v) dává 
  \begin{equation*}
    t_B = \frac{L(V-v)+L(V+v)}{(V+v)(V-v)} = \frac{2LV}{V^2 -v^2} 
        = \frac{2L/V}{1 - \frac{v^2}{V^2}},
  \end{equation*}
  tj. více, než činí odpovídající doba ta prvního člunu.

  {\centering
  \captionsetup{type=figure}
  \luafigure[1]{fyz_fig0899.png}
  \captionof{figure}{Rychlost člunu \(B\) vůči břehu je při plavbě po proudu větší o rychlost
            říčního proudu, kdežto při plavbě proti proudu je o stejnou hodnotu menší.
  \cite[s.~19]{Beiser1975}
  \label{fyz:fig0899}} \par}

  Poměr obou časů \(t_A\) a \(t_B\) je
  \begin{equation*}
    \frac{t_A}{t_B} = \frac{2L/V}{\sqrt{1 - \frac{v^2}{V^2}}}.
  \end{equation*}

  Známe-li rychlost \(V\), společnou pro oba čluny, a měříme-li poměr \(t_A/t_B\), můžeme určit
  rychlost \(v\) říčního proudu.
\end{fyzexam}