% !TeX spellcheck = cs_CZ
\begin{mdframed}[style=mdexam]
  \begin{example}\label{fyz:fey_exam022}
    Elementární částice nazvaná kladný kaon (\(K^+\)) má v klidu průměrnou dobu života
    \SI{0.1237}{\micro\s}, tj. jedná se o dobu života měřenou v klidové soustavě kaonu. Vytvářejí-li
    se v atmosféře kladné kaony s rychlostí \num{0.990}c vzhledem k zemské vztažné soustavě, jak
    daleko se v této soustavě během své doby života dostanou?

    \vspace{1em}
    V soustavě pozorovatele na Zemi je vzdálenost \(d\) uražená kaonem spojena s jeho rychlostí \(v
    = \num{0.990}c\) a dobou jeho letu \(\Delta t_k\) vztahem \(d = v\Delta t_k\). (Toto tvrzení
    není ovlivněno relativitou, protože všechny veličiny se měří v téže vztažné soustavě.) Kdybychom
    neužívali speciální relativity, doba letu by činila \SI{0.1237}{\micro\s}, což je doba života
    částice, a tudíž uražená vzdálenost by byla
    \begin{align*}
      d &= v\Delta t_k                                                        \\
        &= \num{0.990}\cdot\SI{3.00e8}{\m\per\s}\cdot\SI{1.237e-7}{\s}        \\
        &= \SI{36.7}{\m}  \qquad\text{(chybná odpověď)} 
    \end{align*}  
    Je však třeba užít speciální relativity a doba letu kaonu v laboratorní soustavě je jeho
    dilatovaná doba života \(\Delta t\). Podle rov. (\ref{fyz:eq582}) můžeme najít \(\Delta t\) ze
    znalosti vlastní doby života kaonu \(t_0 = \SI{0.1237}{\s}\) měřené v jeho klidové soustavě
    \begin{align*}
      \Delta t &= \dfrac{\Delta t_0}{\sqrt{1  - \left(\dfrac{v}{c}\right)^2}}   \\
               &= \dfrac{\SI{0.123 7e-6}{\s}}{\sqrt{1  - (\num{0.990}c/c)^2}}  \\
               &= \SI{8.769e-7}{\s}
    \end{align*}
    To je sedmkrát větší než vlastní doba života kaonu. (Tento výpočet přihlížel k relativitě,
    protože jsme museli přepočítat data z klidové soustavy částice do soustavy laboratorní.) Nyní
    najdeme délku cesty kaonu v laboratorní soustavě jako
    \begin{align*}
      d &= v\Delta t_k                                                      
        = \num{0.990}\cdot\SI{3.00e8}{\m\per\s}\cdot\SI{8.769e-7}{\s}        \\
        &= \SI{260}{\m}  
    \end{align*} 

    {\centering
    \captionsetup{type=figure}
    \luafigure[1]{fyz_fig0915.png}
    \captionof{figure}{Kaon v atmosféře Země žije déle, jak je měřeno pozorovatelem vázaným na Zemi,
               než měřeno vnitřními hodinami kaonu. (\cite[s.~1013]{Halliday2001})
    \label{fyz:fig0915}}
    \par}  

    Tato vzdálenost je sedmkrát větší než podle naší první (nesprávné) odpovědi. Experimenty tohoto
    druhu, které ověřují speciální relativitu, jsou už po desetiletí pro fyzikální laboratoře
    rutinní záležitostí.    
  \end{example}
\end{mdframed}