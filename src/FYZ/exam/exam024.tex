\begin{fyzexam}{Odhadněte na základě rozměrové analýzy tvar vztahu pro úhlovou frekvenci kmitů
  matematického kyvadla \cite[s.~1]{Kulhanek2020}.}{exam024}  

  Předpokládejme, že frekvence kmitů bude záviset na délce závěsu \(l\), na hmotnosti zavěšené
  kuličky \(m\) a na tíhovém zrychlení \(g\), tj.
  \begin{equation*}
    ω= ω(l, m, g).
  \end{equation*}

  {\centering
  \luafigure[0.6]{fyz_fig0925.pdf}
  \captionsetup{type=figure}  
  \captionof{figure}{Matematické kyvadlo}           
  \label{fyz:fig0925}
  \par}

  Dále předpokládejme, že vztah pro úhlovou frekvenci je jednoduchý a lze ho zapsat jako většinu
  fyzikálních vztahů za pomoci mocninných závislostí:
  \begin{equation*}
    ω= l^\alpha m^\beta g^\gamma.     
  \end{equation*}
  Na první pohled se zdá úloha neřešitelná. Máme totiž jedinou rovnici pro tři neznámé \(α\), \(β\),
  \(γ\). Ve fyzice je každá rovnice nejen rovností číselných hodnot, ale i rovností rozměrů. Pokud
  zapíšeme rozměry všech veličin na levé a pravé straně rovnosti, dostaneme
  \begin{equation}\label{fyz:eq748}
    \dfrac{1}{\unit{\s}} = \unit{\m}^α\unit{\kg}^β\left(\dfrac{\unit{\m}}{\unit{\square\s}}\right)^γ.
  \end{equation}
  V posledních dvou vztazích si opět povšimněte, že proměnné jsou sázeny šikmým a jednotky svislým
  řezem písma. Nyní porovnejme mocninné koeficienty u sekundy, kilogramu a metru na obou stranách
  rovnosti:
  \begin{align*}
    \unit{m}:  \qquad 0 &= α + γ, \\
    \unit{kg}: \qquad 0 &= β,     \\
    \unit{s}:  \qquad-1 &= -2γ.
  \end{align*}
  Rovnice (\ref{fyz:eq748}) je skutečně řešitelná. Snadno zjistíme, že \(β = 0\), \(γ = ½\), \(α =
  −½\). Úhlová frekvence kyvadla tedy je: 
  \begin{equation}\label{fyz:eq749}
    ω=\sqrt{\dfrac{g}{l}}.
  \end{equation}
\end{fyzexam}