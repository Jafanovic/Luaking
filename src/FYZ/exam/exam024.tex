% !TeX spellcheck = cs_CZ
\begin{mdframed}[style=mdexam]
  \begin{example}\label{fyz:exam024}
    Odhadněte na základě rozměrové analýzy tvar vztahu pro úhlovou frekvenci kmitů matematického 
    kyvadla \cite[s.~1]{Kulhanek2020}. \newline
    \textbf{Řešení}: Předpokládejme, že frekvence kmitů bude záviset na délce závěsu \(l\), na
    hmotnosti zavěšené kuličky \(m\) a na tíhovém zrychlení \(g\), tj.
    \begin{equation*}
      ω= ω(l, m, g).
    \end{equation*}

    {\centering
    \luafigure[0.7]{fyz_fig925.pdf}
    \captionsetup{type=figure}  
    \captionof{figure}{K příkladu \ref{fyz:exam024}}           
    \label{fyz:fig925}
    \par}

    Dále předpokládejme, že vztah pro úhlovou frekvenci je jednoduchý a lze ho zapsat jako většinu
    fyzikálních vztahů za pomoci mocninných závislostí:
    \begin{equation*}
      ω= l^\alpha m^\beta g^\gamma.     
    \end{equation*}
    Na první pohled se zdá úloha neřešitelná. Máme totiž jedinou rovnici pro tři neznámé \(α\),
    \(β\), \(γ\). Ve fyzice je každá rovnice nejen rovností číselných hodnot, ale i rovností
    rozměrů. Pokud zapíšeme rozměry všech veličin na levé a pravé straně rovnosti, dostaneme
    \begin{equation}\label{fyz:eq748}
      \dfrac{1}{\si{\s}} = \si{\m}^α\si{\kg}^β\left(\dfrac{\si{\m}}{\si{\square\s}}\right)^γ.
    \end{equation}
    V posledních dvou vztazích si opět povšimněte, že proměnné jsou sázeny šikmým a jednotky
    svislým řezem písma. Nyní porovnejme mocninné koeficienty u sekundy, kilogramu a metru
    na obou stranách rovnosti:
    \begin{align*}
      \si{m}:  \qquad 0 &= α + γ, \\
      \si{kg}: \qquad 0 &= β,     \\
      \si{s}:  \qquad-1 &= -2γ.
    \end{align*}
    Rovnice (\ref{fyz:eq748}) je skutečně řešitelná. Snadno zjistíme, že \(β = 0\), \(γ = ½\), \(α =
    −½\). Úhlová frekvence kyvadla tedy je: 
    \begin{equation}\label{fyz:eq749}
      ω=\sqrt{\dfrac{g}{l}}.
    \end{equation}
    Poznámky:
    \begin{itemize}[noitemsep]
      \item K odvození vztahu (\ref{fyz:eq749}) jsme nepotřebovali znát žádné fyzikální mechanizmy.
            Pouhá rozměrová analýza určila jediný možný tvar fyzikálního zákona.
      \item Odvodili jsme pouze tvar zákona, nikoli číselný koeficient před ním. Před odmocninou by
            mohla být jakákoli bezrozměrná konstanta, například \num{2}, \num{3}, \(π\). V našem
            případě je koeficient před odmocninou skutečně roven jedné. K určení koeficientu by
            postačil jeden jediný experiment. Kdybychom ale chtěli experimentálně odvodit celý
            vztah, museli bychom provádět sady měření s různými délkami závěsů, různými hmotnostmi
            těles a v různých tíhových zrychleních.
      \item Výsledný vztah nezávisí na hmotnosti tělesa. Tělesa všech hmotností kývají na konkrétním
            závěsu se stejnou frekvencí. To není náhoda. Jde o velmi důležitou vlastnost gravitace.
            Všechna tělesa se v gravitaci pohybují stejným způsobem. Například malá kulička a cihla
            dopadnou na zem při volném pádu za stejný čas. K této vlastnosti gravitačního pole se
            ještě vrátíme. 
    \end{itemize}
  \end{example}
\end{mdframed}