% !TeX spellcheck = cs_CZ
\begin{mdframed}[style=mdexam]
  \begin{example}\label{fyz:fey_exam014}
    Představme si, že pozorujeme západ Slunce vleže na břehu klidného moře. Spustíme stopky právě v
    okamžiku, kdy Slunce zcela zmizí. Poté vstaneme a zvýšíme tak polohu svých očí o \SI{1.70}{\m}.
    Stopky zastavíme v okamžiku, kdy nám Slunce zmizí podruhé. Jaký je poloměr Země, ukazují-li
    stopky \SI{11.1}{\s}?\newline

    {\centering\captionsetup{type=figure}\luafigure[1]{fyz_fig885.jpg}\par}

    \vspace{1em}
    \textbf{Řešení:} 
    Z obr. \ref{fyz:fig884} vidíme, že při pozorování vleže se zorný paprsek směřující k hornímu
    okraji slunečního kotouče dotýká povrchu Země v místě, ve kterém se právě nacházíme, tj. v bodě
    \(A\). Při druhém pozorování západu Slunce je zorný paprsek tečnou v bodě \(B\). Označme
    symbolem \(d\) vzdálenost mezi bodem \(B\) a polohou očí stojícího pozorovatele. Vzdálenosti
    bodů \(A\) a \(B\) od středu Země jsou rovny poloměru Země \(r\). Z Pythagorovy věty dostaneme
    \begin{equation*}
      d^2 + r^2 = (r + h)^2 = r^2 + 2rh + h^2\;\Rightarrow\; d^2 = 2rh + h^2.
    \end{equation*}

    Výška \(h\) je ovšem zanedbatelná vzhledem k poloměru Země \(r\). Proto je člen \(h^2\) mnohem
    menší než člen \(2rh\) a rovnici lze přepsat ve tvaru
    \begin{equation}\label{fyz:eq573}
      d^2 = 2rh
    \end{equation}

    {\centering
    \captionsetup{type=figure}
    \luafigure[1]{fyz_fig884.pdf}
    \captionof{figure}{Zvedne-li se pozorovatel z polohy vleže (bod \(A\)) a zvýší tak polohu svých
      očí do výšky \(h\), otočí se zorný paprsek vycházející z horního okraje slunečního kotouče o
      úhel \(\varTheta\). (Velikosti výšky \(h\) i úhlu \(\varTheta\) jsou v obrázku mnohem větší,
      než odpovídá skutečnosti.)
    \cite[s.~229]{Halliday2001}
    \label{fyz:fig884}}
    \par}
    \vspace{1em}
    Symbolem \(\varTheta\) jsme označili úhel mezi tečnami v bodech \(A\) a \(B\) (obr.
    \ref{fyz:fig884}). O stejný úhel se za změřenou dobu \SI{11.1}{\s} otočí Slunce na své zdánlivé
    dráze kolem Země. Za celý den, tj. přibližně za \num{24} hodin, se Slunce kolem Země otočí o
    \ang{360}. Pak můžeme psát
    \begin{equation*}
      \dfrac{\varTheta}{\ang{360}} = \dfrac{t}{\SI{24}{\hour}}.
    \end{equation*}
    Dosadíme-li \(t = \SI{11.1}{\s}\), dostaneme
    \begin{equation*}
      \varTheta = \dfrac{\ang{360}\cdot\SI{11.1}{\s}}
                        {\SI{24}{\hour}\cdot\SI{60}{\minute\per\hour}\cdot\SI{60}{\s\per\minute}}
                =\ang{0.04625}.
    \end{equation*}
    Z obr. \ref{fyz:fig884} vidíme, že \(d = r\cdot\tan{\varTheta}\). Dosazením do rov.
    (\ref{fyz:eq573}) dostaneme
    \begin{equation}\label{fyz:eq574}
      r\cdot\tan{\varTheta} = 2rh\quad\Rightarrow\quad 
      r = \dfrac{2h}{\tan^2{\varTheta}}.
    \end{equation}
    Pro číselné hodnoty \(\varTheta = \ang{0.04625}\) a \(h = \SI{1.70}{\m}\) máme konečně
    \begin{align*}
      r\cdot\tan{\varTheta} &= 2rh\quad\Rightarrow   \\
                          r &= \dfrac{2\cdot\SI{1.70}{\m}}{\tan^2{\ang{0.04625}}}=\SI{5.22e6}{\m}.
    \end{align*}
    Tento výsledek se liší od známé hodnoty poloměru Země (\SI{6.378e6}{\m}) o \SI{20}{\percent}.
  \end{example}
\end{mdframed}