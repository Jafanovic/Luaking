% !TeX spellcheck = cs_CZ
\begin{mdframed}[style=mdexam]
  \begin{example}\label{FYZ:exam006}
    Najděte paralaxu Proximy Centauri, která je od nás vzdálená asi \num{4.2} světelného roku 
    \cite[s.~4]{Kulhanek2009}.
    
    {\centering
      \captionsetup{type=figure}
      \luafigure[0.9]{fyz_fig225.pdf}
      \captionof{figure}{K příkladu \ref{FYZ:exam006}: Paralaxa naší nejbližší hvězdy}
      \label{fyz:fig225}
      \par}
      
    \textbf{Řešení}: Díky pohybu Země kolem Slunce se zdá, že blízké hvězdy opisují oproti 
    vzdáleným elipsu. Úhlový poloměr této elipsy se nazývá paralaxa hvězdy. Lze ji změřit jen pro 
    nejbližší hvězdy. Z definice úhlu (jako v předchozím příkladě) tedy vyplývá, že
    \begin{align*}
      \pi &= \frac{R_{ZS}}{l} = \frac{\SI{1.5e11}{\meter}}{\SI{4.2}{\lightyear}} 
           = \frac{\SI{1.5e11}{\meter}}{\num{4,2}\cdot\SI{9.5e15}{\meter}}   \\
          &\cong \SI{3,7e-6}{\radian},
    \end{align*}
    
    což je přibližně \(\ang{;;0,76}\). Vidíme, že i u druhé nejbližší hvězdy po Slunci není 
    paralaxa ani celá \(\ang{;;1}\).
  \end{example}
\end{mdframed}