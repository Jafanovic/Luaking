% !TeX spellcheck = cs_CZ
  \begin{example} Dělová koule o hmotnosti $m = 24\,kg$ opustila hlaveň rychlostí $v = 
    500\,ms^{-1}$ v čase $\tau=0.008\,s$ po zapálení roznětky. Jak velká síla na kouli působila, 
    jestliže předpokládáme rovnoměrně zrychlený pohyb koule v hlavni? Jak velká práce byla vykonána 
    na urychlení koule a jak dlouhá je hlaveň?\newline
    \textbf{Řešení:}
    \begin{itemize}
      \item Délka hlavně: $l=\frac{1}{2}at^2=\frac{1}{2}v\tau=\frac{1}{2}\cdot500\cdot0.008=2\,m$
      \item Síla působící na kouli: $F=m\frac{v}{\tau}=24\cdot\frac{500}{0.008}=1.5\times10^6\,N$
      \item Vykonaná práce při urychlování koule: $A=\frac{1}{2}mv^2=3\times10^6\,J$
    \end{itemize}
    \todo[inline]{exam004 dopočítat - pahýl}
 \end{example} 