% !TeX spellcheck = cs_CZ
\begin{mdframed}[style=mdexam]
\begin{example}
  Jestliže platí, že \(\rot{A} = \rot{B}\), plyne z toho že \(\vec{A}=\vec{B}\)?\newline  
  \textbf{Řešení:} 
  Nikoliv. Mějme následující funkci \(\vec{A} = \vec{B} + \grad{\varPsi}\), kde 
  \(\varPsi(\vec{r})\) je libovolná skalární funkce (tedy \(\vec{A}\neq\vec{B}\)), pak platí
  \[\rot{A} = \rot{B} +\mathrm{rot}\;\grad{\varPsi} = \rot{B}\] (rotace gradientu je pro 
  spojité funkce vždy nulová, viz \ref{fyz:eq239} kapitola \ref{fyz:IIchapIIsecVII})
\end{example}
\end{mdframed}