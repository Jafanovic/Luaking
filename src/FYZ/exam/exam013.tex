% !TeX spellcheck = cs_CZ
\begin{mdframed}[style=mdexam]
\begin{example}\label{fyz:fey_exam013}
    Vypočtěte gradient skalárního pole \(\varphi(r) = \abs{\vec{r}} = r\)\newline  
  \textbf{Řešení:} Gradient vypočteme po složkách.
  \begin{equation*}
    \pder{\varphi}{x} = \pder{\sqrt{x^2+y^2+z^2}}{x} = \dfrac{x}{\sqrt{x^2+y^2+z^2}} = \dfrac{x}{r}.
  \end{equation*}
  Podobné vztahy dostaneme i pro ostatní složky
  \begin{equation*}
    \pder{\varphi}{y} = \dfrac{y}{r} \qquad\qquad \pder{\varphi}{z} = \dfrac{z}{r}
  \end{equation*}
  Tedy \(\grad{\varphi} = \left(\frac{x}{r}, \frac{y}{r}, \frac{z}{r}\right) =
  \dfrac{\vec{r}}{r} = \vec{r}_0\). Výsledkem je jednotkový vektor ve směru \(\vec{r}\).

\end{example}
\end{mdframed}