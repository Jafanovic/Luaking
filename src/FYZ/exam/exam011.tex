% !TeX spellcheck = cs_CZ
% Marsak, Havrankova - Sbirka resenych prikladu z fyziky - Termika a molekulova
% fyzika.pdf
\begin{mdframed}[style=mdexam]
  \begin{example}\label{fyz:fey_exam011}
    Lampa o hmotnosti \qty{5}{\kg} visí uprostřed dlouhého původně vodorovného drátu a způsobí jeho
    prohnutí o \ang{1} od vodorovného směru (viz obr. \ref{fyz:fig0379}). Určete velikost síly
    \(F_n\) napínající drát. 
        
    {\centering
    \captionsetup{type=figure}
    \luafigure[1]{fyz_fig0379.png}
    \captionof{figure}{K příkladu \ref{fyz:fey_exam011} \cite[s.~6]{Havrankova1995}
    \label{fyz:fig0379}}
    \par}
    
    Tíhová síla, kterou lampa prohýbá drát musí být vyrovnána vertikální silou vzniklou jako
    výslednice napěťových sil působících v drátu. Tedy
    \begin{align*}
               G &= 2F_n\sin\alpha = mg, \quad m = \qty{5}{\kg},   \\
      \sin\alpha &= \sin\ang{1} \approx \alpha = \num{0.017}
    \end{align*}
    \begin{equation*}
      F_n = \frac{mg}{2\sin\alpha} = \frac{\num{5}\cdot\num{9.81}}{2\cdot\num{0.017}}
          = \qty{1443}{\newton}
    \end{equation*}
    % \begingroup\makeatletter\def\f@size{7}\check@mathfonts
    % \def\maketag@@@#1{\hbox{\m@th\large\normalfont#1}}%
    \begin{align*}
      \shortintertext{Použitý vzorec můžeme snadno odvodit z kosinové věty:}
      G &= \sqrt{F_1^2 + F_2^2 - 2F_1F_2\cdot\cos2\alpha}   \\
      \shortintertext{\footnotesize s přihlédnutím, že obě síly jsou si rovny a s využitím  
        goniometrického vzorce pro dvojnásobný úhel \(\cos2\alpha = \cos^2\alpha - \sin^2\alpha\) 
        můžeme provést následující úpravy} 
      G &= \sqrt{2F_n^2\cdot(1-\cos2\alpha)}                 \\
        &=\sqrt{2}F_n\sqrt{1-\cos^2\alpha+\sin^2\alpha}      \\
        &=\sqrt{2}F_n\sqrt{2\sin^2\alpha} = 2F_n\sin\alpha
    \end{align*}
    % \endgroup
    Drát bude napínán stejnou silou, jako kdyby na něj působilo tahem těleso tíhy
    \qty{1443}{\newton}, tedy o hmotnosti \qty{147}{\kg}
  \end{example}
\end{mdframed}  