% !TeX spellcheck = cs_CZ
\begin{mdframed}[style=mdexam]
  \begin{example}\label{FYZ:exam030}
    \emph{Určete hmotnost střelného prachu v nábojnici, jestliže střela má hmotnost \SI{20}{\g} a
    při výstřelu získá rychlost \SI{700}{\m\per\s}. Výhřevnost střelného prachu je
    \SI{3.79}{\mega\joule\per\kg} a účinnost zbraně je \SI{28}{\percent}.
    (\cite[s.~31]{Bartuska1997})}
    
    {\centering
    \captionsetup{type=figure}
    \luafigure[1]{fyz_fig0935.png}
    \captionof{figure}{Vysokorychlostní balistická fotografie zachycuje výstřel z pistole Glock
      model 30 S ve výkonné ráži .45 ACP. Poté, co střela opustí hlaveň, pokračuje expanze plynů
      vzniklých hořením výmetné náplně, Rychlost výtoku těchto plynů je po určitou dobu vyšší než je
      rychlost střely a proto tyto plyny stále ovlivňují střelu a to mimo jiné i tak, že zvyšují
      její rychlost. Experimentálně bylo zjištěno, že k maximální rychlosti dosáhne střela ve
      vzdálenosti asi 20 ráží za koncem hlavně.
    \label{fyz:fig0935}}
    \par} 

    \textbf{Řešení:}\newline 
    Při výstřelu se část energie, která se uvolní při spálení střelného prachu přemění na kineticku
    energii střely. Účinnost zbraně je určena vztahem:
    \begin{equation*}
      \eta = \dfrac{E_k}{Q_1},
    \end{equation*}
    kde \(E_k = \frac{1}{2}mv^2\) je kinetická energie střeli a \(Q_1 = m_1H\) je celkové teplo,
    které se při výstřelu uvolní spálením střelného prachu o hmotnosti \(m_1\) a výhřevnosti \(H\).
    Užitím těchto vztahů dostáváme dostáváme
    \begin{equation*}
     E_k = \eta Q_1 = \eta m_1H = \frac{1}{2}mv^2 \;\rightarrow\; m_1 = \dfrac{mv^2}{2\eta H}.
    \end{equation*}
    Čísleně 
    \begin{align*}
       m_1 &= \dfrac{\SI{20}{\g}\cdot(\SI{700}{\m\per\s})^2}
                    {2\cdot\num{0.28}\cdot\SI{3.79}{\mega\joule\per\kg}}                          \\
       m_1 &= \dfrac{\num{0.02}\cdot700^2}{\num{0.56}\cdot\num{3.79e6}}\cdot
              \dfrac{\si{\kg}\cdot\si{\square\m\per\square\s}}
                    {\si{\kg\square\m\per\square\s\per\kg}}                                       \\
       m_1 &\approx \SI{4.6}{\g}    
     \end{align*}
     V nábojnici je střelný prach o hmotnosti \SI{4.6}{\g}.
  \end{example} 
\end{mdframed}