\begin{fyzexam}{Výpočet gradientu ze zadané funkce}{exam008}
  Nadmořská výška libovolného bodu na povrchu kopce je dána formulí
  \begin{equation*}
    h(x, y) = A\cdot\exp{\left[
                           −\left(\frac{x}{l_0}\right)^2
                          −9\left(\frac{y}{l_0}\right)^2
                         \right]},
  \end{equation*}
  kde \(A = \qty{500}{\m}\), \(l_0 = \qty{100}{\m}\). Nalezněte směr největšího stoupání do kopce 
  (malé posunutí po povrchu kopce v tomto směru vyvolá největší přírůstek nadmořské výšky) v bodě 
  \(B = [-30, 10]\,\unit{m}\), (zdroj: \librariaALDBR).
  
\tcbsubtitle[before skip=\baselineskip]{Řešení I:}  
  Směr největšího stoupání vyjadřuje gradient skalární funkce, kterou je popsána povrchová plocha
  kopce. Platí
  \begin{align*}
    \vec{n} &= \grad h = \left(\pder{h}{x}, \pder{h}{y}\right)                       \\
            &= \frac{2A}{l_0}\exp{\left[−\left(\frac{x}{l_0}\right)^2
                −9\left(\frac{y}{l_0}\right)^2\right]}(x,9y)                         \\
            &\approx (-x,-9y)
  \end{align*}
  Nepodstatné konstanty mění jen délku vektoru, nikoliv jeho směr, proto jsou v konečném  výsledku 
  vynechány. V bodě \(B\) je tedy směr největšího stoupání určen vektorem
  \begin{equation*}
    \vec{n}_B \approx (+30, -90) \approx (+10, -30) \approx (+1, -3).
   \end{equation*}

  \vspace{0.5cm}
  {\centering
    \captionsetup{type=figure}
    \luafigure[1]{fyz_fig0201.png}
    \captionof{figure}{Graf funkce}
    \label{fyz:fig0017}
  \par}
  
\tcbsubtitle[before skip=\baselineskip]{Výpis programu:}  
  %---------------------------------------------------------------
  \lstinputlisting[%
    style=luaMatlabStyle, firstline=7,
    caption={\texttt{FYZ000.m}: Výpis programu pro určení směru největšího stoupání do kopce.}
    ]{../src/FYZ/matlab/FYZ000.m}
  %---------------------------------------------------------------  
\end{fyzexam}