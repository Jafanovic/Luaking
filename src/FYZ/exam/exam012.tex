% !TeX spellcheck = cs_CZ
\begin{mdframed}[style=mdexam]
  \begin{example}\label{fyz:fey_exam012}
    Dvě síly stejné velikosti \SI{250}{\newton} svírají úhel \ang{60}. Určete velikost výslednice.
        
    {\centering
    \captionsetup{type=figure}
    \luafigure[0.8]{fyz_fig0380.pdf}
    \captionof{figure}{Grafické určení výslednice dvou sil se společným působištěm, svírajících
      \(\alpha\). Také platí \(\beta = \ang{180}-\alpha\) a pak \(\cos\beta = - \cos\alpha_1\) 
    \label{fyz:fig0380}}
    \par}
    
    Velikost síly \(F_v\) můžeme určit graficky pomocí vektorového rovnoběžníku. Velikosti sil
    \(F_1\) a \(F_2\) vyneseme ve zvoleném měřítku (např. \SI{50}{\newton} = \SI{1}{\cm}),
    vykreslíme rovnoběžník dle obrázku a délku uhlopříčku převedeme zpět ve stejném měřítku na
    velikost síly \(F_v\). Početně určíme velikost výslednice pomocí kosinové věty: 
    \begin{align*}
      F_v^2 &= F_1^2 + F_2^2 - 2F_1F_2\cdot\cos\beta               \\
      F_v^2 &= F_1^2 + F_2^2 - 2F_1F_2\cdot\cos(\ang{180}- \alpha) \\
      F_v   &= \sqrt{2F_1^2 - 2F_1^2\cdot\cos(\ang{180}- \alpha)}  \\
            &= \sqrt{2}F_1\sqrt{1-\cos(\ang{180}- \alpha)}         \\
            &= \sqrt{2}\cdot250\sqrt{1-\cos(\ang{180} - \ang{60})} 
               \cong \SI{433}{\newton}
    \end{align*}
  \end{example}
\end{mdframed}