% !TeX spellcheck = cs_CZ
\begin{mdframed}[style=mdexam]
  \begin{example}\label{FYZ:exam032}
    \emph{Určete hmotnost vařící vody, kterou je třeba přilít do vody o hmotnosti \SI{5}{\kg} a
    teplotě \SI{9}{\degreeCelsius}, aby výsledná teplota vody byla \SI{30}{\degreeCelsius}.
    Předpokládejme, že tepelná výměna nastala jen mezi teplejší a studenější vodou. 
     (\cite[s.~33]{Bartuska1997})}
    
    \textbf{Řešení:}\newline 
    Teplo, které odevzdá vařicí voda, se rovná teplu, které přijme studenejší voda. Platí proto
    \begin{equation*}
      m_Hc(t_2 - t) = m_Sc(t-t_1), \;\rightarrow\; m_H = \dfrac{(t_2 - t)}{(t-t_1)}\cdot m_S.
    \end{equation*}
    číselně
    \begin{align*}
      m_H &= \dfrac{(\SI{30}{\degreeCelsius}  -  \SI{9}{\degreeCelsius})}
                  {(\SI{100}{\degreeCelsius} - \SI{30}{\degreeCelsius})}\cdot\SI{5}{\kg}.         \\
      m_H &= \SI{1.5}{\kg}
    \end{align*}
    Do studenejší vody je třeba přilít vařicí vodu o hmotnosti \SI{1.5}{\kg}.  
  \end{example} 
\end{mdframed}