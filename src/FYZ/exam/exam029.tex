% !TeX spellcheck = cs_CZ
\begin{mdframed}[style=mdexam]
  \begin{example}\label{FYZ:exam029}
    \emph{Určete hmotnost petroleje, který spotřebuje petrolejový vařič k zahřátí vody o hmotnosti
    \SI{3}{\kg} z \SI{10}{\degreeCelsius} na \SI{100}{\degreeCelsius}. Účinnost vařiče
    \SI{34}{\percent}, výhřevnost petroleje je \SI{44}{\mega\joule\per\kg} a měrná kapacita vody
    \SI{4180}{\joule\per\kg\per\kelvin}. (\cite[s.~31]{Bartuska1997})}
    
    {\centering
    \captionsetup{type=figure}
    \luafigure[1]{fyz_fig934.png}
    \captionof{figure}{Abychom si v terénu dokázali teplé jídlo připravit, neobejdeme se bez vody a
    bez spolehlivého vařiče a paliva.
    \label{fyz:fig934}}
    \par} 
    
    \textbf{Řešení:}\newline 
    Při ohřívání vody petrolejovým vařičem se jen část tepla, které se uvolní při spalování
    petroleje, spotřebuje na ohřátí vody. Účinnost vařiče je určena vztahem:
    \begin{equation*}
      \eta = \dfrac{Q}{Q_1},
    \end{equation*}
    kde \(Q = mc(t_2 - t_1)\) je teplo, které přijme voda  hmotnosti \(m\) při ohřátí z teploty
    \(t_1\) na teplotu \(t_2\), a \(Q=m_1H\) teplo uvolněné spálením petroleje o hmotnosti \(m_1\)
    a výhřevnosti \(H\). Užitím těchto vztahů dostáváme
    dostáváme
    \begin{align*}
     Q = \eta Q_1 = \eta m_1H &= mc(t_2-t_1) \;\rightarrow  \\
                               m_1 &= \dfrac{mc(t_2-t_1)}{\eta H}.
    \end{align*}
    Čísleně 
    \begin{align*}
       m_1 &= \dfrac{\SI{3}{\kg}\cdot\SI{4180}{\joule\per\kg\per\kelvin}\cdot\SI{90}{\kelvin}}
                    {\num{0.34}\cdot\SI{44}{\mega\joule\per\kg}}                                  \\
       m_1 &= \dfrac{3\cdot4180\cdot90}{\num{0.34}\cdot\num{44e6}}\cdot
              \dfrac{\si{\kg}\cdot\si{\joule\per\kg\per\kelvin}\cdot\si{\kelvin}}
                    {\si{\J\per\kg}}                                                              \\
       m_1 &\approx \SI{75}{\g}    
     \end{align*}
     Na zahřátí vody o hmotnosti \SI{3}{\kg} z \SI{10}{\degreeCelsius} na \SI{100}{\degreeCelsius}
     na petrolejovém vařiči je zapotřebí \SI{75}{\g} petroleje.
  \end{example} 
\end{mdframed}