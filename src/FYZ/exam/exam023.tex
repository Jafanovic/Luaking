% !TeX spellcheck = cs_CZ
\begin{mdframed}[style=mdexam]
  \begin{example}\label{fyz:exam023}
    Pokud v zápise rozměr zapomeneme, může dojít k zajímavým absurditám:
    \begin{equation*}
      v = \dfrac{s}{t} = \textcolor{red}{\dfrac{6}{3}} = \SI{2}{\m\per\s}
    \end{equation*}
    Zdánlivě nevinný výpočet rychlosti jako podílu dráhy a času je nesmyslný. U červeně označených
    hodnot chybí rozměry. Pokud bychom vzali v úvahu poslední rovnost, máme
    \begin{align*}
      \dfrac{6}{3} &= \SI{2}{\m\per\s}  \rightarrow \\
                 2 &= \SI{2}{\m\per\s}  \rightarrow \\
                 1 &= \si{\m\per\s}     \rightarrow \\
         \si{\s} &= \si{\m} 
    \end{align*}
    Docházíme tak ke zcela jistě nepravdivému tvrzení, že sekunda je totéž co metr. Dejme si proto
    pozor a nezapomeňme psát všude jednotky \cite[s.~1]{Kulhanek2020}. 
  \end{example}
\end{mdframed}