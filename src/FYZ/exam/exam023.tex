\begin{fyzexam}{Jak zdánlivě nevinný výpočet rychlosti jako podílu dráhy a času může být nesmyslný.  
  \[v = \dfrac{s}{t} = \textcolor{red}{\dfrac{6}{3}} = \SI{2}{\m\per\s}\]}{exam023}    

  U červeně označených hodnot chybí rozměry. Pokud bychom vzali v úvahu poslední rovnost, máme
  \begin{align*}
    \dfrac{6}{3} &= \SI{2}{\m\per\s}  \rightarrow 2 = \SI{2}{\m\per\s}  \rightarrow \\
               1 &= \si{\m\per\s}     \rightarrow     \si{\s} = \si{\m} 
  \end{align*}
  Docházíme tak ke zcela jistě nepravdivému tvrzení, že sekunda je totéž co metr. Dejme si proto
  pozor a nezapomeňme psát všude jednotky. \cite[s.~1]{Kulhanek2020}. 
\end{fyzexam}