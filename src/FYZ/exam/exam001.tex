% !TeX spellcheck = cs_CZ
\begin{mdframed}[style=mdexam]
  \begin{example}\label{fyz:fey_exam001}
    \href{http://librarian/stable.php?id=380}{Soustředné válce}: Velmi tenký nevodivý plášť válce o
    poloměru \(b\) a délce \(L\) obklopuje dlouhý, plný nevodivý válec o poloměru \(a\) a délce
    \(L\), kde \(b < a\). V celém objemu vnitřního válce je spojitě vyplněn nábojem o celkové
    velikosti \(+Q\). Na povrchu vnějším plášti válce je spojitě rozprostřen náboj stejné velikosti,
    opačného znaménka \(−Q\). Oblast \(a < r < b\) je prázdná. S využitím Gaussova zákona určete
    intenzitu elektrického pole v celém prostoru.
      
    {\centering
    \captionsetup{type=figure}
    \luafigure[0.5]{fyz_fig0212.pdf}
    \captionof{figure}{K příkladu \ref{fyz:fey_exam001}
    \label{fyz:fig0212}}
    \par}
    
    \begin{enumerate}[noitemsep]
      \item \emph{Jaká je symetrie úlohy?} Cylindrická
      \item \emph{Jaký je směr intenzity elektrického pole?} Vektor intenzity elektrické pole míří v
            radiálním směru cylindrické vztažné soustavy, jeho velikost je konstantní na
            cylindrických plochách s konstantním poloměrem. 
      \item \emph{Kolik různých oblastí v prostoru budeme vyšetřovat?} Budeme vyšetřovat tři
             oblasti, uvnitř \(a\), mezi \(a\) a \(b\), vně \(b\).
      \item \emph{Pro každou oblast v prostoru zvolte Gaussovu plochu, jakou proměnnou zvolíte pro
            parametrizaci těchto ploch? Jaké jsou obory hodnot této proměnné?} V každé oblasti
            budeme používat válcové plochy o poloměru \(r\) a výšce \(h\), které jsou souosé se
            skutečnými válci. Plochu budeme popisovat poloměrem \(r\), jehož obor hodnot je interval
            \(\left\langle 0,\infty\right)\).
      \item \emph{Pro oblast \(r < a\) spočítejte tok Gaussovou plochou, kterou jste si vybrali. Ve
            vyjádření by jste měli mít i neznámou intenzitu elektrického pole.} Obě podstavy
            Gaussova válce nebudou přispívat k celkovému toku plochou (ze symetrie je pole
            rovnoběžné s touto plochou). Tok pláštěm válce tak je
            \begin{equation*}
              \Phi_E = \oiint\vec{E}\dd{S} = 2\pi r h E.
            \end{equation*}
            \item \emph{Pro oblast r < a spočítejte náboj uzavřený ve zvolené Gaussově ploše.}
            Ve vnitřním válci je náboj homogenně rozložen. Náboj uzavřený v Gaussově ploše můžeme
            vyjádřit dvěma způsoby: (1) jako objemový podíl uzavřený v ploše nebo (2) využitím
            objemové nábojové hustoty \(\varrho\). Výpočet je proveden oběma způsoby:
            \begin{enumerate}[noitemsep]
              \item \(Q_{in} = \dfrac{V_{in}}{V_{total}}Q = \dfrac{r^2h}{a^2L}Q\)
              \item \(\varrho = \dfrac{Q}{L\pi a^2} 
                    \Rightarrow Q_{in} = \varrho\pi r^2h = \dfrac{r^2h}{a^2L}Q\)
            \end{enumerate}
      \item \emph{Pro oblast \(r < a\) dejte podle Gaussova zákona do rovnosti vztahy z 5. a 6.
            bodu, vyjádřete velikost intenzity elektrického pole.}
            \begin{equation*}
              2\pi r h E = \frac{1}{\varepsilon_0}\dfrac{r^2h}{a^2L}Q 
              \Rightarrow E = \frac{Qr}{2\pi\varepsilon_0a^2L}
            \end{equation*}
            \emph{Zopakujte stejnou proceduru pro oblast \(a < r < b\), vyjádřete intenzitu
            elektrického pole jako funkci \(r\).} Náboj, který je uzavřen v ploše je konstantní, se
            vzrůstajícím poloměrem r se však mění celková plocha válce.
            \begin{equation*}
              2\pi\,r\,h\,E = \frac{1}{\varepsilon_0}\dfrac{h}{L}Q 
              \Rightarrow E = \frac{Q}{2\pi\varepsilon_0rL}
            \end{equation*}
      \item Zakreslete \emph{intenzitu elektrického pole} jako graf funkce v závislosti na
            parametru, který popisoval Gaussovou plochu. Graf nakreslete pro celý prostor. 
            
            {\centering
              \captionsetup{type=figure}
              \luafigure[0.9]{fyz_fig0213.pdf}
              \captionof{figure}{K příkladu \ref{fyz:fey_exam001}
              \label{fyz:fig0213}}
              \par}
      \item Jaký je \emph{rozdíl potenciálů} mezi body \(r = a\) a \(r = 0\)? Tedy, kolik je
            \(\Delta V = V(a) − V(0)\)? 
            \begin{align*}
              \Delta V  &= V(a) − V(0) = -\int_0^a\vec{E}(r)\dd\vec{r}   \\
                        &= -\int_{0}^{a}E(r)\vec{r}\cdot\vec{r}\dd{r}    \\
                        &= -\int_{0}^{a}\frac{Qr}{2\pi\varepsilon_0a^2L} 
                         = -\frac{Q}{2\pi\varepsilon_0L}
            \end{align*}
      \item Jaký je rozdíl potenciálů mezi body \(r = b\) a \(r = a\)?
            \begin{align*}
              \Delta V  &= V(b) − V(a) = -\int_a^b\vec{E}(r)\dd\vec{r} \\   
                        &= -\int_a^b E(r)\vec{r}\cdot\vec{r}\dd{r}     \\
                        &= -\int_a^b \frac{Q}{2\pi\varepsilon_0rL} 
                         = -\frac{Q}{2\pi\varepsilon_0L}\ln\frac{b}{a}
            \end{align*}
            Opět je rozdíl potenciálů záporný, neboť potenciál v místě \(r = a\) je vyšší než v
            místě \(r = b\).
    \end{enumerate}
  \end{example}
\end{mdframed}