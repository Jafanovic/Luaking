% !TeX spellcheck = cs_CZ
% \href{http://librarian/stable.php?id=380}{Soustředné válce}:
\begin{fyzexam}{Sousé nevodivé válce}{exam001}
    Velmi tenký nevodivý plášť válce o poloměru \(a\) a délce \(L\) obklopuje dlouhý, plný nevodivý
    válec o poloměru \(b\) a délce \(L\), kde \(a < b\). Celý objem vnitřního válce je spojitě
    vyplněn nábojem o celkové velikosti \(+Q\). Na povrchu pláště vnějšího válce je spojitě
    rozprostřen náboj stejné velikosti, opačného znaménka \(−Q\). Oblast \(a < r < b\) je bez
    náboje. S využitím Gaussova zákona určete intenzitu elektrického pole v celém prostoru.
  \tcbsubtitle[before skip=\baselineskip]{Řešení:}  
  \begin{description}[leftmargin=0em,labelindent=0em, style=nextline]
    \item[\emph{Jaká je symetrie úlohy?}] Cylindrická (\ref{fyz:fig0212})
    \item[\emph{Jaký je směr intenzity elektrického pole?}] Vektor intenzity elektrické pole míří
          v radiálním směru cylindrické vztažné soustavy, jeho velikost je konstantní na
          cylindrických plochách s konstantním poloměrem. 
    \item [\emph{Kolik různých oblastí v prostoru budeme vyšetřovat?}] Budeme vyšetřovat tři
           oblasti, uvnitř \(a\), mezi \(a\) a \(b\), vně \(b\).
  \end{description}    

  {\centering
  \captionsetup{type=figure}
  \luafigure[0.7]{fyz_fig0212.pdf}
  \captionof{figure}{K příkladu \ref{fyz:exam001}. Převzato z překladu
    \cite[s.~16]{Dourmashkin2006}
  \label{fyz:fig0212}}
  \par}
    
    \begin{description}[leftmargin=0em,labelindent=0em, style=nextline]
      \item[\emph{Pro každou oblast v prostoru zvolme Gaussovu plochu. jakou proměnnou zvolíme pro
            parametrizaci těchto ploch? Jaké jsou obory hodnot této proměnné?}] V každé oblasti
            budeme používat válcové plochy o poloměru \(r\) a výšce \(h\), které jsou souosé se
            skutečnými válci. Plochu budeme popisovat poloměrem \(r\), jehož obor hodnot je interval
            \(\left\langle 0,\infty\right)\).
      \item[\emph{Pro oblast r < a spočítejme tok Gaussovou plochou. Ve vyjádření bychom měli
            mít i neznámou intenzitu elektrického pole.}] Obě podstavy Gaussova válce nebudou
            přispívat k celkovému toku plochou (ze symetrie je pole rovnoběžné s touto plochou). Tok
            pláštěm válce tak je
            \begin{equation}\label{fyz:eq1004}
              \Phi = \int_SE_n\dd{S} = 2\pi r h E(r).
            \end{equation}
      \item[\emph{E(r) pro oblast r < a?}] Pro oblast \(r < a\) nejdříve spočítáme náboj uzavřený ve
            zvolené Gaussově ploše a poté vyjádříme velikost intenzity elektrického pole. Víme, že
            ve vnitřním \uv{červeném} válci je náboj homogenně rozložen a můžeme jej vyjádřit dvěma
            způsoby:
            \begin{enumerate}[noitemsep]
              \item objemový podíl uzavřený v ploše \[Q_{in} = \dfrac{V_{in}}{V_{total}}Q =
                    \dfrac{\cancel{\pi}r^2h}{\cancel{\pi}a^2L}Q\]
              \item využitím objemové nábojové hustoty \(\varrho  = \frac{Q}{\pi a^2L}\).
                    \[Q_{in} = \varrho\pi r^2h = \dfrac{r^2h}{a^2L}Q\]
            \end{enumerate}
            A nyní jsme připraveni vyjádřit velikost intenzity elektrického pole. Použijeme k tomu
            rovnici \eqref{fyz:eq1002} a \eqref{fyz:eq1004}:
            \begin{equation*}
              2\pi r h E(r) = \dfrac{Q_{in}}{\varepsilon_0}        \Rightarrow 
                       E(r) = \dfrac{Qr}{2\pi\varepsilon_0a^2L}
            \end{equation*}
      \item[\emph{E(r) pro oblast pro oblast a < r < b?}] V této oblasti není žádný další náboj.
            Jediný náboj, který je uzavřen v gaussovské ploše je ten, který je pochází z
            \uv{červeného} válce. Tudíž náboj je konstantní, ačkoliv vzrůstá poloměr \(r\). Pouze
            se mění celková plocha válce (srovnej s ilustračním obrázkem \ref{fyz:fig0212}).
            \begin{equation*}
              2\pi rhE(r) = \frac{1}{\varepsilon_0}\dfrac{h}{L}Q 
              \Rightarrow E(r) = \frac{Q}{2\pi\varepsilon_0rL}
            \end{equation*}
      \item[\emph{E(r) pro oblast pro oblast r > b?}].  V této oblasti nenaměříme žádné pole,
            protože se kladný a záporný náboj dokonale vyruší. Bude tedy 
            \begin{equation*}
              2\pi r h E(r) = \dfrac{Q^+ + Q^-}{\varepsilon_0}        \Rightarrow 
                       E(r) = 0
            \end{equation*}
    \end{description}      
    
  \tcbsubtitle[before skip=\baselineskip]{Graf intenzity elektrického pole:} 
    Zakresleme intenzitu elektrického pole \(E\) jako graf funkce v závislosti na parametru \(r\),
    který popisoval Gaussovu plochu. Graf nakreslíme pro celý prostor. 
            
    {\centering
      \captionsetup{type=figure}
      \luafigure[0.9]{fyz_fig0213_1.pdf}
      \captionof{figure}{Průběh intenzity elektrického pole \(E\) v průřezu válce na 
        obr. \ref{fyz:fig0212}
        \label{fyz:fig0213}
        }
    \par}

    \begin{description}[leftmargin=0em,labelindent=0em, style=nextline] 
      \item[\emph{Jaký je rozdíl potenciálů mezi body r = a a r = 0?}] Tedy, kolik je
            \(\Delta V = V(a) − V(0)\)? 
            \begin{align*}
              \Delta V  &= V(a) − V(0) = -\int_0^aE(r)\dd{r}   \\
                        &= -\int_{0}^{a}\frac{Q}{2\pi\varepsilon_0a^2L}r\dd{r}     \\
                        &= -\frac{Q}{2\pi\varepsilon_0a^2L}\cdot\left[\frac{r^2}{2}\right]_0^a 
                         = -\frac{Q}{4\pi\varepsilon_0L}
            \end{align*}
            Všimněme si, že rozdíl potenciálů je záporný, tedy vyšší potenciál je v místě,
            kde \(r=a\).
      \item[\emph{Jaký je rozdíl potenciálů mezi body r = b a r = a?}]
            \begin{align*}
              \Delta V  &= V(b) − V(a) = -\int_a^bE(r)\dd{r} \\   
                        &= -\int_{a}^{b}\frac{Q}{2\pi\varepsilon_0L}\frac{\dd{r}}{r}     \\
                        &= -\frac{Q}{2\pi\varepsilon_0L}\cdot\left[\ln{r}\right]_a^b 
                         = -\frac{Q}{2\pi\varepsilon_0L}\ln\frac{b}{a}
            \end{align*}
            Opět je rozdíl potenciálů záporný, neboť potenciál v místě \(r = a\) je vyšší než v
            místě \(r = b\).
    \end{description}
\end{fyzexam}