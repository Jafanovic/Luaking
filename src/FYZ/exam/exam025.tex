% !TeX spellcheck = cs_CZ
\begin{mdframed}[style=mdexam]
  \begin{example}\label{fyz:exam025}
    Nalezněte takové kombinace konstant \(c\), \(G\), \(\hslash\) (rychlosti světla, gravitační
    konstanty a Planckovy konstanty), které dají přirozenou jednotku pro délku, čas, hmotnost a
    energii.
    \begin{subequations}\label{fyz:eq750} 
      \begin{align}
        c       &= \SI{3e8}{\m\per\s}                              \label{fyz:eq750a}  \\
        G       &= \SI{6,67e-11}{\per\kg\cubic\m\per\square\s},    \label{fyz:eq750b}  \\
        \hslash &= \SI{1,05e-34}{\kg\square\m\per\s} .             \label{fyz:eq750c}
      \end{align}
    \end{subequations}

    \textbf{Řešení}: Pokusíme se vytvořit výraz pro délku \(l_0\), čas \(t_0\), hmotnost \(m_0\) a
    energii \(E_0\). Začneme délkou tak, že napíšeme součin výše uvedených tří konstant, s neznámými
    exponenty \(α\), \(β\), \(γ\): 
    \begin{equation*}
      l_0 = c^αG^β\hslash^γ.
    \end{equation*}
    Tato rovnice ve skutečnosti představuje čtyřnásobnou rovnost: rovnost číselnou a rovnost
    rozměrovou v metrech, kilogramech a sekundách. Napíšeme nyní rozměrové části vytvořeného výrazu:
    \begin{equation*}
      \mathrm{m^1kg^0s^0} = \si{\m}^α\si{\s}^{-α}                   %c
                            \si{\kg}^{β}\si{\m}^{3β}\si{\s}^{-2β}   %G
                            \si{\kg}^γ\si{\m}^{-2γ}\si{\s}^{-γ}.     %hslash
    \end{equation*}
    Nyní zapíšeme soustavu rovnic pro exponenty u metru, kilogramu a sekundy:
    \begin{align*}
      1 &=   α + 3β + 2γ    \\
      0 &=     -  β +  γ    \\
      0 &= - α - 2β + -γ    
    \end{align*}
    Řešením této soustavy získáme jednoznačné řešení pro exponenty
    \begin{equation*}
      α =−3/2;\quad β=1/2;\quad γ=1/2. 
    \end{equation*}
    Tyto exponenty jednoznačně až na násobící číselný faktor určují velikost Planckovy délky. Zcela
    analogickým způsobem můžeme odvodit vztahy pro ostatní Planckovy veličiny. Výsledky jsou:
    \begin{subequations}\label{fyz:eq750} 
      \begin{align}
        l_0&=\sqrt{\dfrac{\hslash G}{c^3}}\approx \SI{e-35}{\m},               \label{fyz:eq750a}\\
        t_0&=\sqrt{\dfrac{\hslash G}{c^5}}\approx \SI{e-43}{\s},               \label{fyz:eq750b}\\
        m_0&=\sqrt{\dfrac{\hslash c}{G}}  \approx \SI{e-8}{\kg},               \label{fyz:eq750c}\\
        E_0&=\sqrt{\dfrac{\hslash c^5}{G}}\approx \SI{e19}{\giga\electronvolt},\label{fyz:eq750d}
      \end{align}
    \end{subequations} 
  \end{example}
\end{mdframed}