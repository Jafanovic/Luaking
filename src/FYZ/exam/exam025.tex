% !TeX spellcheck = cs_CZ
\begin{fyzexam}{Nalezněte takové kombinace konstant \(c\), \(G\), \(\hslash\) (rychlosti světla,
  gravitační konstanty a Planckovy konstanty), které dají přirozenou jednotku pro délku, čas,
  hmotnost a energii.}{exam025}  
    \begin{subequations}\label{fyz:eq750} 
      \begin{align}
        c       &= \qty{3e8}{\m\per\s}                              \label{fyz:eq750a}  \\
        G       &= \qty{6,67e-11}{\per\kg\cubic\m\per\square\s},    \label{fyz:eq750b}  \\
        \hslash &= \qty{1,05e-34}{\kg\square\m\per\s} .             \label{fyz:eq750c}
      \end{align}
    \end{subequations}

    Pokusíme se vytvořit výraz pro délku \(l_0\), čas \(t_0\), hmotnost \(m_0\) a energii \(E_0\).
    Začneme délkou tak, že napíšeme součin výše uvedených tří konstant, s neznámými exponenty \(α\),
    \(β\), \(γ\): 
    \begin{equation*}
      l_0 = c^αG^β\hslash^γ.
    \end{equation*}
    Tato rovnice ve skutečnosti představuje čtyřnásobnou rovnost: rovnost číselnou a rovnost
    rozměrovou v metrech, kilogramech a sekundách. Napíšeme nyní rozměrové části vytvořeného výrazu:
    \begin{equation*}
      \mathrm{m^1kg^0s^0} = \unit{\m}^α\unit{\s}^{-α}                    %c
                            \unit{\kg}^{-β}\unit{\m}^{3β}\unit{\s}^{-2β}   %G
                            \unit{\kg}^γ\unit{\m}^{2γ}\unit{\s}^{-γ}.      %hslash
    \end{equation*}
    Nyní zapíšeme soustavu rovnic pro exponenty u metru, kilogramu a sekundy:
    \begin{alignat*}{9}
      &\;1 &=&   &α &+ &3β &\;+ &2γ    \\
      &\;0 &=&   &  &- & β &\;+ & γ    \\
      &\;0 &=& - &α &- &2β &\;- & γ    
    \end{alignat*}
    Řešením této soustavy získáme jednoznačné řešení pro exponenty
    \begin{equation*}
      α =−3/2;\quad β=1/2;\quad γ=1/2. 
    \end{equation*}
    Tyto exponenty jednoznačně až na násobící číselný faktor určují velikost Planckovy délky. Zcela
    analogickým způsobem můžeme odvodit vztahy pro ostatní Planckovy veličiny. Výsledky jsou:
    \begin{subequations}\label{fyz:eq751} 
      \begin{align}
        l_0&=\sqrt{\dfrac{\hslash G}{c^3}}\approx \qty{e-35}{\m},               \label{fyz:eq751a}\\
        t_0&=\sqrt{\dfrac{\hslash G}{c^5}}\approx \qty{e-43}{\s},               \label{fyz:eq751b}\\
        m_0&=\sqrt{\dfrac{\hslash c}{G}}  \approx \qty{e-8}{\kg},               \label{fyz:eq751c}\\
        E_0&=\sqrt{\dfrac{\hslash c^5}{G}}\approx \qty{e19}{\giga\electronvolt},\label{fyz:eq751d}
      \end{align}
    \end{subequations} 
\end{fyzexam}