\begin{fyzexam}{Na obr.38.8 se Sylva (v bodě \(A\)) a Slávkova kosmická loď (o vlastní délce \(L_0 =
  \SI{230}{\m}\)) míjejí konstantní relativní rychlostí \(v\). Sylva měří časový
  interval \SI{3.57}{\micro\s}, po který ji loď míjí (od průchodu bodu \(B\) do průchodu
  bodu \(C\)). Jaký je rychlostní parametr \(β\) mezi Sylvou a lodí?
  \hfill\cite[s.~1015]{Halliday2001}}{exam018} 

  {\centering
  \captionsetup{type=figure}
  \luafigure[0.8]{fyz_fig0958.pdf}
  \captionof{figure}{Příklad \ref{fyz:exam018}. Sylva měří, jak dlouho trvá lodi, když ji
    v bodě A míjí. (\cite[s.~1016]{Halliday2001})
    \label{fyz:fig0958}
  } 
  \par} 

\end{fyzexam}