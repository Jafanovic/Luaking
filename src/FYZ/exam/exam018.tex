% !TeX spellcheck = cs_CZ
\begin{mdframed}[style=mdexam]
  \begin{example}\label{fyz:fey_exam018}
    Dvě částice se pohybují vysokou rychlostí v urychlovači částic. Pomalejší částice má náskok před
    rychlejší měřenou v soustavě této částice. Rychlost obou částic je měřená ve vztažné soustavě
    laboratoře. Za jakou dobu vzhledem k pozorovateli v laboratoři rychlejší částice dožene
    pomalejší?


    Proč je situace na obr. \ref{fyz:fig906} neobvyklá? Uvažujme známější analogickou situaci. Je
    jasný den, čluny jsou na moři a někdo z posádky jednoho člunu hodí do vody kámen právě v
    okamžiku, kdy jsou bok po boku. Na vodě se začnou šířit známé kruhy jako na obr.
    \ref{fyz:fig906b}, které se však jeví dvěma pozorovatelům na dvou různých člunech různě. Už jen
    na základě zjištění, zda je nebo není ve středu těchto kruhů, může každý pozorovatel říci,
    pohybuje-li se vůči vodě či nikoliv. Voda je sama o sobě vztažnou soustavou a pozorovatel na
    člunu, který se vůči ní pohybuje, měří rychlost kruhů vzhledem k sobě samotnému; ta je v různých
    směrech různá, na rozdíl od rychlosti kruhů, měřené pozorovatelem na nehybném člunu. Je důležité
    si uvědomit, že pohyb a vlny ve vodě jsou úplně jiné než pohyb a vlny v (prázdném) prostoru;
    zatímco samotná voda je vztažnou soustavou, prostor takovou soustavou být nemůže, a rychlost
    šíření vln na vodě se na rozdíl od rychlosti šíření světelných vln v prostoru mění s pohybem
    pozorovatele.

    {\centering
    \captionsetup{type=figure}
    \luafigure[1]{fyz_fig906b.png}
    \captionof{figure}{Analogická situace, šíření kruhů na vodní hladině.
    \cite[s.~25]{Beiser1975}
    \label{fyz:fig906}}
    \par}
    \vspace{1em}

  \end{example}
\end{mdframed}