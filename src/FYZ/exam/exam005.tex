% !TeX spellcheck = cs_CZ
\begin{mdframed}[style=mdexam]
  \begin{example}\label{FYZ:exam005}
    Spočtěte, jakou vzdálenost v metrech vyjadřuje jeden parsek \cite[s.~3]{Kulhanek2009}.
    
    \textbf{Řešení}: \(\SI{1}{\parsec}\) (paralaktická sekunda) je vzdálenost, ze které vidíme 
    velkou poloosu oběžné dráhy Země kolem Slunce pod uhlem \(\varphi = \ang{;;1}\). Úhel 
    \(\ang{;;1}\) je tak malý, že strany \(VS\) a \(VZ\) na obrázku prakticky splývají a místo 
    pravého trojúhelníka $VSZ$ můžeme použít definiční vztah úhlu v obloukové míře (\emph{velkost 
    úhlu je možné určit jako poměr délky oblouku vymezeného rameny na kružnici opsané kolem 
    vrcholu k poloměru této kružnice}). Proto 
    
    {\centering
      \captionsetup{type=figure}
      \luafigure[0.9]{fyz_fig224.pdf}
      \captionof{figure}[Parsek]{K příkladu \ref{FYZ:exam005}: Odvození velikosti Parseku}
      \label{fyz:fig224}
      \par}
    \begin{equation*}
      \varphi = \frac{R_{SZ}}{l} \rightarrow l = \frac{R_{SZ}}{\varphi},
    \end{equation*}
    kde $l$ je vzdálenost \SI{1}{\parsec} v metrech, $R_{SZ}$ je vzdálenost země od Slunce a 
    $\varphi$ je úhel jedné vteřiny vyjádřený v radiánech. 
    \begin{equation*}
        l = \frac{\SI{1.5e11}{\meter}}{\dfrac{1}{60\cdot60} 
            \cdot\dfrac{2\pi}{360}}\cong \SI{3e16}{\meter}.
    \end{equation*}
  \end{example}
\end{mdframed}