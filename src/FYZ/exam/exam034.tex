% !TeX spellcheck = cs_CZ
%---------- Driftová rychlost elektroknů ve vodiči: 
\begin{fyzexam}{Driftová rychlost elektronů ve vodiči}{exam034}
  Vodičem z jednomocné mědi o průřezu $S_0 = \qty{1}{\mm^2}$ prochází elektrický proud $I =
  \qty{5}{\A}$. Vypočtěte:
  \begin{itemize}[noitemsep, leftmargin=2em]
    \item počet volných elektronů v jednotkovém objemu \ce{Cu},
    \item úhrnný náboj volných elektronů v jednotkovém objemu,
    \item driftovou rychlost volných elektronů při proudu \(I\).
  \end{itemize}

\tcbsubtitle[before skip=\baselineskip]{Řešení:} 
  Měď má poměrnou atomovou hmotnost $A_r = 63,54$ a hustotu\footnote{Pro hustotu budeme používat 
  alternativní značku $s$, s ohledem na kolizi značky $\rho$, jež označuje hustotu náboje.} $s = 
  \qty{8.93e3}{\kg.\m^{-3}}$. 

  \begin{itemize}[leftmargin=2em]
    \item Jeden mol mědi o molové hmotnosti $M = \qty{0.06354}{\kg\per\mol}$ a o molovém
          objemu 
          \begin{align*}
            V_m &= \frac{M}{s} 
                 = \frac{\qty{63.54e-3}{\kg.\mol^{-1}}}{\qty{8.93e3}{\kg.\m^{-3}}}      \\
                &= \qty{7.12e-6}{\m^3.\mol^{-1}}
          \end{align*}
          obsahuje $N_A = 6,0221\cdot10^{23}$ jednoatomových molekul \emph{Cu} na jeden mol,
          z nichž každý má volný jeden (valenční) elektron. Tedy počet volných elektronů v
          jednotkovém objemu je 
          \begin{align*}
            n_0 &= \frac{N_A}{V_m} = \frac{sN_A}{M}                                           
                 = \frac{\qty{6.0221e23}{\mol^{-1}}}{\qty{7.12e-6}{\m^{3}.\mol^{-1}}}    \\
                &= \qty{8.46e28}{\per\cubic\m}.
          \end{align*}  
    \item Úhrnný náboj volných elektronů v jednotkovém objemu mědi je 
          \begin{equation}
            Q_v = -e\cdot n_0 = \qty{-1.36e10}{\coulomb.m^{-3}}.
          \end{equation}
    \item Velikost driftové rychlosti určíme ze vztahu $I = -en_0v_dS_0 = - Q_v v_d S_0$ tj.
    \begin{align*}
      v_d &= \left\lvert\frac{I}{Q_v\cdot S_0}\right\rvert                       
           = \frac{\qty{5}{\coulomb\per\s}}{\qty{1.36e10}{\coulomb.m^{-3}}\cdot\qty{1e-6}{\m^2}}  \\
          &= \qty{3676e-4}{\m\per\s} = \qty{0.3676}{\mm\per\s}.  
    \end{align*}
  \end{itemize}
  Z provedených výpočtů si můžeme udělat názor o mikroskopických poměrech v kovových vodičích: počet
  volných nositelů náboje - elektronů a jejich úhrný náboj v jednotkovém objemu je značný a proto
  driftová rychlost elektronů potřebná k vyvolání proudu běžné velikosti v drátových vodičích je
  nesmírně malá (doslova hlemýždí).
\end{fyzexam}
