\begin{fyzexam}{Stojíme u železničních kolejí, když nás náhle vyděsí relativistický nákladní vagon
  řítící se kolem nás, jak je ukázáno na obrázku. Ve vagonu je dobře vybavený hobo (americký
  železniční tulák) Jack, který posílá laserový pulz od přední k zadní stěně vagonu. (a) Dá naše
  měření rychlosti pulzu větší, menší, nebo stejný výsledek jako měření Jackovo? (b) Je Jackovo
  měření doby letu pulzu měřením vlastního času? (c) Jsou Jackova a naše měření doby letu pulzu
  spojena rov. (\ref{fyz:eq584})? \cite[s.~1013]{Halliday2001}}{exam020} 

  {\centering
  \captionsetup{type=figure}
  \luafigure[1]{fyz_fig0914.pdf}
  \captionof{figure}{K příkladu \ref{fyz:exam020}.
  \label{fyz:fig0914}}
  \par}  
\end{fyzexam}