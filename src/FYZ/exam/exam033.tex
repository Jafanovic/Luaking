% !TeX spellcheck = cs_CZ
\begin{mdframed}[style=mdexam]
  \begin{example}\label{FYZ:exam033}
    \emph{Do vody o hmotnosti \SI{800}{\g} a teplotě \SI{12}{\degreeCelsius} byla ponořena platinová
    koule o hmotnosti \SI{150}{\g}, která byla předtím ponechána v žáru pece. Po dosažení
    rovnovážného stavu byla výsledná teplota soustavy \SI{19}{\degreeCelsius}. Určete teplotu pece.
    Měrná tepelná kapacita platiny je \SI{133}{\joule\per\kg\per\kelvin}. Předpokládejme, že tepelná
    výměna nastala jen mezi platinovou koulí a vodou. (\cite[s.~34]{Bartuska1997})}
    
    \textbf{Řešení:}\newline 
    Podle kalorimetrické rovnice se teplo, které odevzdá platinová koule, rovná teplu, které přijme
    voda a výsledná teplota soustavy \(t=\SI{19}{\degreeCelsius}\). Platí proto
    \begin{equation*}
      m_Pc_P(t_2 - t) = m_Vc_V(t-t_1), \;\rightarrow\; t_2 = \dfrac{m_Vc_V(t - t_1)}{m_Pc_P} + t_1.
    \end{equation*}
    číselně
    \begin{align*}
      t_2 &= \dfrac{\SI{800}{\g}\cdot\SI{4180}{\joule\per\kg\per\kelvin}          %
              \cdot(\SI{ 19}{\degreeCelsius} - \SI{12}{\degreeCelsius})}
                   {\SI{150}{\g}\cdot\SI{133}{\joule\per\kg\per\kelvin}}          \\
      t_2 &=\SI{1192}{\degreeCelsius}
    \end{align*}
    Teplota pece je asi \SI{1200}{\degreeCelsius}. Při ponoření zahřáté platinové koule se část vody
    odpaří. Teplo potřebné k tomuto odpaření jsme neuvažovali.  
  \end{example} 
\end{mdframed}