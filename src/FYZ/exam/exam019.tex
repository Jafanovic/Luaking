% !TeX spellcheck = cs_CZ
\begin{mdframed}[style=mdexam]
  \begin{example}\label{fyz:fey_exam019}
    Lze ukázat, že elektron s kinetickou energií \SI{20}{\giga\electronvolt} (mluvívá se o
    \SI{20}{\giga\electronvolt}-elektronu) má rychlost \(v = \num{0.999 999 999 67}c\). Zúčastní-li
    se takový elektron závodu se světelným pulzem se startem v okolí Slunce a s cílem u nejbližší
    hvězdy (Proxima Centauri, vzdálenost \num{4.3} světelné roky čili \SI{4.0e16}{\meter}), s jakým
    časovým náskokem světelný pulz zvítězí?

    \begin{equation*}
      \Delta t = \dfrac{L}{v} - \dfrac{L}{c} = L\cdot\dfrac{c-v}{vc}.
    \end{equation*} 
    Zde \(v\) je natolik blízké \(c\), že můžeme ve jmenovateli výrazu (ne však v čitateli!) položit
    \(v = c\). Pak dostaneme
    \begin{align*}
      \Delta t &=\dfrac{L}{c}\cdot\left(1-\dfrac{v}{c}\right)                                     \\
               &=\dfrac{\SI{4.0e16}{\meter}}{\SI{3.0e8}{\meter\per\sec}}(1-\num{0.999 999 999 67})\\
               &=\SI{0.044}{\sec} = \SI{44}{\milli\sec}.
    \end{align*} 
  \end{example}
\end{mdframed}