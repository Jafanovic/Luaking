\begin{fyzexam}{Lze ukázat, že elektron s kinetickou energií \SI{20}{\giga\electronvolt} (mluvívá se
  o \protect\SI{20}{\protect\giga\protect\electronvolt}-elektronu) má rychlost \(v = \num{0.999 999
  999 67}c\). Zúčastní-li se takový elektron závodu se světelným pulzem se startem v okolí Slunce a
  s cílem u nejbližší hvězdy (Proxima Centauri, vzdálenost \num{4.3} světelné roky čili
  \protect\SI{4.0e16}{\protect\meter}), s jakým časovým náskokem světelný pulz zvítězí?
  \hfill\cite[s.~1008]{Halliday2001}}{exam019} 

  \begin{equation*}
    \Delta t = \dfrac{L}{v} - \dfrac{L}{c} = L\cdot\dfrac{c-v}{vc}.
  \end{equation*} 
  Zde \(v\) je natolik blízké \(c\), že můžeme ve jmenovateli výrazu (ne však v čitateli!) položit
  \(v = c\). Pak dostaneme
  \begin{align*}
    \Delta t &=\dfrac{L}{c}\cdot\left(1-\dfrac{v}{c}\right)                                      \\
             &=\dfrac{\SI{4.0e16}{\meter}}{\SI{3.0e8}{\meter\per\s}}(1-\num{0.999 999 999 67}) \\
             &=\SI{0.044}{\s} = \SI{44}{\milli\s}.
  \end{align*} 
\end{fyzexam}