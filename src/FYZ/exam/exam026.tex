% !TeX spellcheck = cs_CZ
\begin{mdframed}[style=mdexam]
  \begin{example}\label{fyz:exam026}
    Nalezněme velikost a vzájemný úhel vektorů \(\vec{a} = (1, 3, 0)\), \(\vec{b} = (2, 2, 0)\).
    \newline
    \textbf{Řešení}:  Nejprve nalezneme velikosti obou vektorů:
    \begin{align*}
      a &=\sqrt{\vec{a}\cdot\vec{a}} = \sqrt{1^2 + 3^2 + 0^0} = \sqrt{10}, \\
      b &=\sqrt{\vec{b}\cdot\vec{b}} = \sqrt{2^2 + 2^2 + 0^0} = \sqrt{8}.
    \end{align*}
    Nyní již snadno nalezneme úhel mezi oběma vektory: 
    \begin{align*}
      \cos\vartheta &= \frac{\vec{a}\vec{b}}{ab} 
                     = \frac{a_xb_x + a_yb_y + a_zb_z}{\sqrt{10}\sqrt{8}},            \\
                    &= \frac{1\cdot2 + 3\cdot2 + 0\cdot0}{\sqrt{10}\sqrt{8}} = \frac{8}{\sqrt{80}}
                       \approx\num{0.89} 
    \end{align*}
    Odpovídající úhel je přibližně \ang{27}.
  \end{example}
\end{mdframed}