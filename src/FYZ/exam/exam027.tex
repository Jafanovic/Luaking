% !TeX spellcheck = cs_CZ
\begin{mdframed}[style=mdexam]
  \begin{example}\label{fyz:exam027}
    Nalezněme vektorový součin vektorůů \(\vec{a} = (1, 2, 3)\), \(\vec{b} = (4, 5, 6)\).
    \newline
    \textbf{Řešení}: Vyjdeme přímo z definice:
    \begin{align*}
      \vec{a}\times\vec{b} 
        &\equiv  (a_2b_3−a_3b_2, a_3b_1−a_1b_3, a_1b_2−a_2b_1)                      \\
        &= (2\cdot6-3\cdot5, 3\cdot4-1\cdot6, 1\cdot5-2\cdot4)                      \\
        &= (3, 6, 3). 
    \end{align*}
  \end{example}
\end{mdframed}