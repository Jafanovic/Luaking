% !TeX spellcheck = cs_CZ
\begin{fyzexam}{Náš Hvězdolet míjí Zemi relativní rychlostí \(\num{0.999 0}c\). Po uplynutí
  \num{10.0} roků se zastavíte na pozorovatelně LP13 a pak cestujeme zpět k Zemi stejnou relativní
  rychlostí. Cesta zabere dalších \num{10.0} roků (našeho času). Jak dlouho trvá cesta podle měření
  vykonaných na Zemi? (Zanedbejme jakýkoli vliv způsobený zrychlením během zastavování a rozletu.)
  \cite[s.~1013]{Halliday2001}}{exam021} 
   
  \vspace{1em}
  Začátek i konec cesty tam (Země - LP13) nastávají ve naší vztažné soustavě, tj. ve vašem
  hvězdoletu, na stejném místě. Měříme tedy vlastní čas \(\Delta t_0\) cesty, který je určen jako
  \(\num{10.0} y\). Rov. (\ref{fyz:eq582}) dává odpovídající čas \(\Delta t\), jak je naměřen v
  pozemské vztažné soustavě
  \begin{align*}
    \Delta t &= \dfrac{\Delta t_0}{\sqrt{1  - \left(\dfrac{v}{c}\right)^2}}
              = \dfrac{\num{10.0} y}{\sqrt{1  - (\num{0.999 0}c/c)^2}}      \\
              &= \num{22.37}\cdot\num{10.0} y = \num{224} y
  \end{align*}
  Při cestě zpět (LP13 - Země) máme stejnou situaci a stejné údaje. Celková cesta tedy zabere 20
  roků vašeho času, ale
  \begin{equation*}
    \Delta t_{celk}  = 2\cdot\num{224} y = \num{448} y
  \end{equation*}
  pozemského času. Jinými slovy, zestárli jsme o \num{20} let, zatímco Země zestárla o \num{448}
  let. Ačkoli nelze (pokud dnes víme) cestovat do minulosti, není vyloučeno cestování např. do
  budoucnosti Země pomocí velmi rychlého relativního pohybu, který změní tempo plynutí času.
\end{fyzexam}