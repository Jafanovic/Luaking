% !TeX spellcheck = cs_CZ
\begin{mdframed}[style=mdexam]
  \begin{example}\label{aes:exam004}
    Běžné OZ mají zesílení \(A = \numrange{20000}{2000000}\). Znamená to, že pro výstupní napětí
    \qty{10}{\V} je mezi kladným a záporným vstupem napětí \(u_D
    =\frac{\qty{10}{\V}}{\numrange{2000000}{20000}}\) = \SIrange{5}{500}{\uV}. V praxi to většinou
    znamená, že rozdílové napětí \(u_D\) považujeme za nulové pro jakékoliv výstupní napětí \(u_0\).
    Jak se později ukáže, tato úvaha je velmi důležitá. Podmínku \(u_D = 0\) se snažíme zajistit za
    všech okolností. Vede to k požadavku, aby zesílení ideálního OZ bylo nekonečně velké (u reálného
    co největší) (\cite[s.~14]{Puncochar1996}).    
  \end{example}
\end{mdframed}