\documentclass{standalone}
  \usepackage{xltxtra}
  \usepackage{tikz}
  \usetikzlibrary{babel}
  \usetikzlibrary{intersections, calc, positioning, mindmap, trees}
  \usetikzlibrary{decorations.pathmorphing, decorations.pathreplacing} 
  \usetikzlibrary{decorations.markings}
  \usetikzlibrary{matrix, arrows}
  \usetikzlibrary{topaths, automata, scopes, chains}
  \usetikzlibrary{quotes, shapes}
  \usetikzlibrary{fit}
  \usepackage{pgfplots}
  % ~~~~~~~~~~~ XY Coordinate ~~~~~~~~~~~~~~~~~~~~~
% umožňuje zobrazit souřadnice daného bodu - x-ová složka proměnná \XCoord
% http://tex.stackexchange.com/questions/66125/extract-x-value-from-coordinate-in-tikz
\newdimen\XCoord
\newdimen\YCoord
\newcommand*{\ExtractCoordinate}[1]{\path (#1); \pgfgetlastxy{\XCoord}{\YCoord};}%
% příklad použití:
%         \ExtractCoordinate{p5};
%         \node [below] at (1cm,-1.5cm) {p5 $(\XCoord,\YCoord)$}; 
\newcommand{\XYcross}{% 
  \draw[name path=axeX,->] 
    (\xmin,\zero) -- (\xmax,\zero)   node[right] {\xlabel} coordinate(x axis);
  \draw[name path=axeY,->] 
    (\phase,\ymin) -- (\phase,\ymax) node[left]  {\ylabel} coordinate(y axis);
  \path[name intersections={of=axeX and axeY, name=pocatek}];
  \node[below left] at (pocatek-1) {$0$};
  \draw[fill=white] (pocatek-1) circle(2pt);
}

\def\xmin{-50}
\def\xmax{70}
\def\ymin{-5}
\def\ymax{+120}
\def\zero{0}
\def\phase{-50}
\def\xlabel{$\omega$}
\def\ylabel{$A_u$}

% Macro for drawing one frame of the F-distribution.
%\newcommand{\distpic}{
%    % Draw the normal distribution curve
%    \draw[blue!50!black,smooth,thick] plot[id=f1,domain=-1:1,samples=50]
%    function {exp(-x*x*3)};
%}
\begin{document}
  \begin{tikzpicture} 
    \begin{scope}[domain=0:3.1415, line join=round, miter limit=4.00,line width=0.5pt, 
                  x=25pt,y=25pt, xshift=0, yshift=0]    
      \node at(1.8,4.4) [left, fill=white] {$Q=\frac{1}{\omega_0\cdot L_p\cdot G_p}$}; %  
      \node at(2.6,3) [left, fill=white] {$Q_1=\frac{\omega_0}{B_1}$}; %
      \node at(2.6,2) [left, fill=white] {$Q_2=\frac{\omega_0}{B_2}$}; %
      \draw[name path=func1, blue!50!black,smooth,thick,] plot[id=f1,domain=-2:2,samples=50]
        function {4*exp(-x*x*5)}; 
      \draw[name path=func2, blue!50!black,smooth,thick] plot[id=f2,domain=-2:2,samples=50]
        function {4*exp(-x*x*2)};  
      \draw[very thin, dashed] (0,4)   -- +(-2.1,0) node[left] {1};
      \draw[very thin, dashed] (0,4.1) -- +(0,-4.2) node[below] {$\omega_0$};
      
      \path [name path=line1](-2,{4/sqrt(2)}) -- (2,{4/sqrt(2)});
      \path [name intersections={of=line1 and func2, name=cross}];
      \coordinate (A) at (cross-1);
      \coordinate (B) at (cross-2);
      
      \ExtractCoordinate{A};
      \draw[very thin, dashed] (B) -- (A) -- +(0,{-(\YCoord+50)}) -- +(0,{-(\YCoord+50)}) coordinate(p1);
      \draw[very thin, dashed] (B) -- +(0,{-(\YCoord+50)}) coordinate(p2);
      \draw[very thin, dashed] (A) -- +({(\phase - 3 - \XCoord)},0) node[left] {$\frac{1}{\sqrt{2}}$};
      \draw[<->] (p1) -- (p2);
      \node at ($ (p1)!0.5!(p2) $) [above] {$B_2$};
      
      \path [name intersections={of=line1 and func1, name=cross}];
      \coordinate (A) at (cross-1);
      \coordinate (B) at (cross-2);
      \draw[very thin, dashed] (B) -- (A) -- +(0,{-(\YCoord+30)}) -- +(0,{-(\YCoord+30)}) coordinate(p1);
      \draw[very thin, dashed] (B) -- +(0,{-(\YCoord+30)}) coordinate(p2);
      \draw[<->] (p1) -- (p2);
      \node at ($ (p1)!0.5!(p2) $) [above] {$B_1$};
      
      \path [name path=line1](-2,0.2) -- (2,0.2);
      \path [name intersections={of=line1 and func2, name=cross}];
      \coordinate (A) at (cross-2);
      \draw[very thin, <-] (A) -- +(0.5,0.5) node[above] {$Q_2<Q_1$};

      \path [name path=line1](-2,0.2) -- (2,0.2);
      \path [name intersections={of=line1 and func1, name=cross}];
      \coordinate (A) at (cross-1);
      \draw[very thin, <-] (A) -- (-1.5,0.7) node[above] {$Q_1$};
      
    \end{scope}  

    \begin{scope}[draw=black,line join=round, miter limit=4.00,line width=0.5pt,y=1pt,x=1pt] 
      \XYcross;
    \end{scope}
  \end{tikzpicture}
\end{document}


