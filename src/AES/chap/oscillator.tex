% !TeX spellcheck = cs_CZ
%---------------------------------------------------------------------------------------------------
% file oscillator.tex
%---------------------------------------------------------------------------------------------------
%======================== Kapitola: Generátory signálů==============================================
\setchaptertoc
\chapter{Generátory signálů}


  \textbf{Generátory signálů} jsou elektronické obvody, které za určitých podmínek generují 
  jednorázový nebo periodický signál. Na rozdíl od zesilovačů nejsou buzeny z vnějších zdrojů 
  signálů, patří tedy mezi \emph{autonomní obvody}.
  
  \begin{itemize}[noitemsep]
    \item \emph{Podle druhu vyráběných signálů:}
      \begin{itemize}
        \item neperiodických průběhů (jednorázových)
        \item periodických průběhů.
      \end{itemize}
    \item  \emph{Podle tvaru generovaného signálů:}
    \begin{itemize}
      \item harmonických signálů - \emph{oscilátory}
      \item impulzních průběhů - \emph{klopné obvody}
      \item funkční generátory - vytvářejí různé periodické průběhy jako trojúhelníkové, pilovité, 
            můžeme sem zařadit také generátory speciálních průběhů, jako jsou generátory, televizního signálů
            apod.
    \end{itemize} 
  \end{itemize}
  
  Mezi hlavní vlastnosti generátorů zahrnujeme tyto parametry:
  \begin{itemize}[noitemsep]
    \item kmitočet (perioda) kmitů
    \item stabilita kmitočtu a amplitudy
    \item doba náběžné a sestupné hrany impulzů - v \emph{klopných obvodech} 
    \item zkreslení harmonického průběhu - u \emph{oscilátorů}.
  \end{itemize}

  \section{Generátory harmonických signálů - oscilátory}
    Základními typy oscilátorů jsou:
    \begin{itemize}[noitemsep]
      \item Jednobranové LC oscilátory, též nazývané oscilátory se záporným diferenciálním odporem 
            (\emph{Negative Resistance Oscillators});
      \item Zpětnovazební oscilátory;
      \item Oscilátory RC;
      \item LC oscilátory;
      \item Krystalem řízené oscilátory.
    \end{itemize}

    Pro pásmo nízkých kmitočtů do několika set kHz může být sinusový signál generován oscilátorem s 
    RC selektivním zpětnovazebním obvodem, pro vyšší kmitočty jsou používány LC oscilátory. Pro 
    velmi přesné a stabilní oscilátory jsou využívány oscilátory řízené krystaly.
    
    \subsection{LC oscilátory se záporným diferenciálním odporem}
      Připojíme-li k paralelnímu rezonančnímu LC obvodu zdroj proudu o proměnném kmitočtu podle 
      obr. \ref{AES:fig_MUE6_311b}, na kterém jsou sériové ztrátové odpory přepočítány na paralelní 
      vodivost \(G_p\), můžeme získat modulové charakteristiky podle obr. \ref{AES:fig_MUE6_311a}. 
      Kmitočtová závislost přenosu je závislá na velikosti činitele jakosti obvodu \(Q\) (na 
      velikosti činných ztrát v obvodu). Čím užší je rezonanční křivka a stabilnější kmitočet 
      \(\omega_0\), tím přesnější a stabilnější je kmitočet oscilátoru.

      \begin{figure}[ht!]
        \centering     
        \subcaptionbox{\label{AES:fig_MUE6_311a}}{\luafigure[0.5]{MUE6_311a.pdf}} 
        \subcaptionbox{\label{AES:fig_MUE6_311b}}{\luafigure[0.4]{MUE6_311b.png}}  \newline
        \subcaptionbox{\label{AES:fig_MUE6_311c}}{\luafigure[0.4]{MUE6_311c.png}} 
        \subcaptionbox{\label{AES:fig_MUE6_311d}}{\luafigure[0.5]{MUE6_311d.pdf}}  \newline
        \subcaptionbox{\label{AES:fig_MUE6_311e}}{\luafigure[0.5]{MUE6_311e.pdf}} 
        \caption{a) Modulové charakteristiky paralelního rezonančního obvodu pro napěťový přenos.
                 b) model paralelního rezonančního obvodu a jeho napájení ze zdroje o proměnném 
                 kmitočtu, c) přeměna rezonančního obvodu v oscilátor, d) časová závislost napětí 
                 na svorkách ztrátového LC obvodu po rozkmitání, d) výstupní napětí bezeztrátového 
                 LC obvodu 
                 \cite[s.~135]{Dolecek2009}.}
        \label{AES:fig_MUE6_311}
      \end{figure}
      
      Na obr. \ref{AES:fig_MUE6_311c} je nakreslen paralelní rezonanční obvod, u kterého do obvodu 
      po přepnutí spínače z polohy \(1\) do polohy \(2\) připojujeme předem nabitý kondenzátor. V 
      obvodu vzniknou vlastní tlumené kmity o kmitočtu \(\omega_0\). Obvod lze popsat diferenciální 
      rovnicí
      \begin{equation}\label{AES:eq_osc04}
        \dder{i}{t} + \frac{R}{L}\cdot\der{i}{t} + \omega^2 i = 0
      \end{equation}
      Rovnice má řešení
      \begin{equation}
        i(t) = I_{0}e^{-\delta t}\sin\omega_v t  \qquad u(t) = U_{0}e^{-\delta t}\cos\omega_v t
      \end{equation}
      kde \(I_0\) je \emph{počáteční amplituda proudu}, \(U_0\) je \emph{počáteční amplituda napětí 
      na kapacitoru}, \(\delta = \frac{R}{2L}\) je \textbf{činitel tlumeni obvodu}, \(\omega_0 
      =\frac{1}{\sqrt{LC}}\) je \textbf{rezonanční kmitočet} a \(\omega_v = \sqrt{\omega^2_0 - 
      \delta^2}\) je \textbf{vlastní kmitočet volných kmitů} obvodu. 
      
      Ze vztahu (\ref{AES:eq_osc04}) je zřejmé, Že je-li činitel tlumeni a \textbf{kladný} je 
      kmitání \textbf{tlumené}, je-li a \textbf{záporný} je kmitání \textbf{netlumené}, je-li 
      \(\delta = 0\) mají kmity konstantní amplitudu. Velikost činitele tlumení je závislá na 
      velikosti rezistoru \(R\), který modeluje ztráty v rezonančním obvodu viz obr. 
      \ref{AES:fig_MUE6_311d}, \ref{AES:fig_MUE6_311e}.
      
      V reálném obvodu je vždy \(\delta > 0\) a proto pozorujeme tlumené kmity, jejichž amplituda 
      klesá exponenciálně podle vztahu 
      \begin{equation}\label{AES:eq_osc02}
        u(t) = U_{ss}e^{-\delta t},
      \end{equation}
      kde \(u(t)\) představuje časovou závislost amplitudy kmitů na svorkách paralelního LC obvodu. 
      Dosáhnout nulového tlumení je možné sériovým zapojením rezistoru \(-R_N = R\) - tedy 
      rezistoru se \textbf{záporným odporem}. K tomu by bylo zapotřebí najít takovou součástku, 
      jejíž voltampérová charakteristika vykazuje úsek se zápornou derivací. Hledáme tedy součástky 
      které mají voltampérovou charakteristiku jako je na obr. \ref{AES:fig_AES62a} nebo 
      \ref{AES:fig_AES62b}.
      
      \begin{figure}[ht!]
        \centering  
        \subcaptionbox{nelinearita typu \textbf{S} \label{AES:fig_AES62a}}{\luafigure[0.5]{AES_62a.pdf}}
        \subcaptionbox{nelinearita typu \textbf{N} \label{AES:fig_AES62b}}{\luafigure[0.5]{AES_62b.pdf}}                                          
        \caption{Voltampérové charakteristiky s úseky vykazující záporný diferenciální odpor 
        \cite[s.~93]{Koucky1997}}
        \label{MIT:fig_AES_62}
      \end{figure}
      
      Pro sériový model kmitavého obvodu obr. \ref{AES:fig_AES63a} je použitelná nelinearita 
      \texttt{typu S} (např. lavinová dioda), pro paralelní model kmitavého obvodu obr. 
      \ref{AES:fig_AES63b} je použitelná nelinearita \texttt{typu N} (např. tunelová dioda) 
      \cite[s.~93]{Koucky1997}.

      \begin{figure}[ht!]
        \centering  
        \subcaptionbox{sériový rezonanční obvod   \label{AES:fig_AES63a}}{\luafigure[0.5]{AES_63a.pdf}} 
        \subcaptionbox{paralelní rezonanční obvod \label{AES:fig_AES63b}}{\luafigure[0.5]{AES_63b.pdf}}                                           
        \caption{ }
        \label{MIT:fig_AES_63}
      \end{figure}
      
      Na velikosti ztrátového odporu \(R\) závisí velikost činitele tlumení \emph{(damping factor)} 
      obvodu podle vztahu:
      \begin{equation}\label{AES:eq_osc01}
        \delta = \frac{R}{2L} \qquad [rad/s]
      \end{equation}
      
      LC oscilátory obsahují klasický rezonanční obvod, jehož rezonanční kmitočet je určen 
      \textbf{Thomsonovým vztahem}:
      \begin{equation}\label{AES:eq_osc03}
        f_0 = \frac{1}{2\pi\cdot\sqrt{LC}} \quad [Hz], \;\text{nebo}\; 
        \omega_0 = \frac{1}{\sqrt{LC}} \quad [rad/s].
      \end{equation}
      
%---------------------------------------------------------------------------------------------------