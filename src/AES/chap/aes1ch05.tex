% !TeX spellcheck = cs_CZ
%---------------------------------------------------------------------------------------------------
% file opamp.tex
%---------------------------------------------------------------------------------------------------
%================================ Kapitola: Zesilovače==============================================
\setchaptertoc
\chapter{Napěťové komparátory}\label{aesIchV}  
  
  Komparátor je podobný operačnímu zesilovači. Má dva vstupy (invertující, neinvertující) a jeden
  výstup (viz \ref{aes:fig066}). Je však speciálně navržen pro porovnání napětí mezi oběma vstupy.
  Komparátor pracuje s otevřenou smyčkou a poskytuje dvoustavové logické výstupní napětí. Tyto dva
  stavy odpovídají znaménku rozdílu\footnote{včetně účinků vstupního ofsetového napětí komparátoru}
  mezi oběma vstupy.
  
  Na výstupu komparátoru bude vysoká úroveň napětí, pokud vstupní signál na neinvertujícím vstup
  převyšuje signál na invertujícím vstupu (plus ofsetové napětí, \(V_{os}\)) a nízkou úroveň pro
  opačný případ. Z hlediska terminologie používané v oblasti logických obvodů nabývá komparátor
  logickou "\texttt{1}" nebo "\texttt{0}", případně \texttt{H} (high) nebo \texttt{L} (low).
  
  Komparátor je běžně používán v aplikacích, kde je nějaká proměnná úroveň signálu porovnána s
  pevnou úrovní (obvykle referenční napětí). Protože je to ve skutečnosti \emph{1-bitový}
  analogově-digitální převodník (ADC), je komparátor základním prvkem ve všech ADC.
  (\cite[s.~72]{Vrba2000} a \cite[s.~2]{MT083}).

  \luagraphic[1]{aes_fig066.png}{Napěťový komparátor \cite[s.~1]{MT083}}{aes:fig066}

  \section{Funkce komparátoru}\label{aesIchVsecI}
    Zapojíme-li operační zesilovač podle obr. \ref{aes:fig067a}, potom se bude projevovat jeho velké
    zesílení. Je-li napájecí napětí zesilovače \SI{15}{\volt}, potom můžeme předpokládat, že
    maximální rozkmit výstupního signálu bude asi \SI{13.5}{\volt}. Meze, mezi kterými se může
    výstupní signál pohybovat jsou na převodní charakteristice uvedené na obr. \ref{aes:fig067b}
    označeny \(U_{sat}\) a \(-U_{sat}\). Velmi velká hodnota zesílení otevřené smyčky \(A_{os}\)
    činí operační zesilovač velmi citlivý na velikost rozdílu vstupních signálů.

    \begin{figure}[ht!]  %\ref{aes:fig067}
      \centering
      \subcaptionbox{\label{aes:fig067a}}{\luafigure[0.9]{aes_fig067a.png}} \\    
      \subcaptionbox{\label{aes:fig067b}}{\luafigure[0.9]{aes_fig067b.png}}                                             
      \caption{a) Zapojení operačního zesilovače bez obvodu zpětné vazby, b) převodní
              charakteristika operačního zesilovače (\cite[s.~177]{Dolecek2007})}
      \label{aes:fig067}
    \end{figure}

    Na charakteristice na obr. \ref{aes:fig067b} jsou uvedené meze výstupního signálu dosaženy již
    při velmi malých hodnotách diferenčního napětí ud na vstupu zesilovače (v příkladu uvedeném na
    obr. \ref{aes:fig067b} je výstup zesilovače v saturaci při rozdílu vstup ních napětí \(U_d =
    u_2 - u_1 = \SI{67.5}{\micro\volt}\).

    Prakticky můžeme říci, že když je \(u_2 > u_1\). bude mít napětí na výstupu zesilovače velikost
    \(U_{sat}\), když \(u_1 > u_2\), bude mít výstupní napětí velikost \(-U_{sat}\).

    \begin{tcnote}
      Doba přepnutí výstupního napětí OZ z jedné meze na druhou je omezena rychlostí přeběhu
      použitého zesilovače.
    \end{tcnote}

    Napěťový komparátor je, stejně jako operační zesilovač, \textbf{diferenční zesilovač} s velmi
    vysokým zesílením, jehož výstup nabývá pouze dva stavy. K výstupům komparátorů je možné připojit
    různá zařízení včetně různých typů logických obvodů. Z toho důvodu mají různé komparátory různá
    zapojení výstupních obvodů. Nebudeme-li uvažovat velikosti výstupních napětí a rychlosti
    spínání, můžeme uvést dvě základní zapojení výstupních obvodů:
    \begin{itemize}[noitemsep]
      \item zapojení typu \textbf{otevřený kolektor} - označení OC \emph{(open collector)};
      \item dvojčinné zapojení s \textbf{komplementárními tranzistory} - \emph{(push-pull)}.
    \end{itemize}

    \begin{figure}[ht!]  %\ref{aes:fig068}
      \centering
      \subcaptionbox{\label{aes:fig068a}}{\luafigure[0.5]{aes_fig068a.png}} \\    
      \subcaptionbox{\label{aes:fig068b}}{\luafigure[0.6]{aes_fig068b.png}} \\
      \subcaptionbox{\label{aes:fig068c}}{\luafigure[0.6]{aes_fig068c.png}}
      \caption{a) Znázornění výstupu komparátoru s otevřeným kolektorem, c) sepnutí tranzistoru, 
              když  \(u_1 > u_2\), c) tranzistor je rozepnut, když \(u_2 > u_1\) 
              (\cite[s.~177]{Dolecek2007})}
      \label{aes:fig068}
    \end{figure}

  \section{OZ ve funkci komparátoru}
    OZ jsou určeny k přesnému zesilování signálů s vysokou linearitou a stabilitou, samozřejmostí u
    nich je \emph{záporná zpětná vazba}.
    
    Stejně jako OZ se komparátory vyznačují malou vstupní napěťovou nesymetrií, velkým zesílením
    diferenčního signálu a velkým potlačením souhlasného napětí. Hlavní rozdíly mezi nimi
    spočívají v tom, že:
    \begin{itemize}[noitemsep]
      \item komparátor má logický výstup, který po překlopení do jedné ze dvou logických úrovní
            (log 1 nebo log 0) vyjadřuje, které ze dvou vstupních napětí je větší, přechod z jedné
            do druhé úrovně je velmi rychlý, ideálně skokový;
      \item OZč má analogový výstup, jeho výstupní napětí zpravidla nedosahuje mezí daných
            velikostmi napájecích napětí;
      \item OZ jsou konstruovány pro aplikace se zápornou zpětnou vazbou, nemělo by dojít k jejich
            přebuzení, které způsobí saturaci tranzistorů.
    \end{itemize}

    Komparátory jsou většinou rychlé, některé OZ rovněž. Komparátory jsou konstruovány pro velké
    vstupní diferenční signály, zatímco záporná zpětná vazba u OZ velikost diferenčních signálů
    minimalizuje. Když je OZ přebuzený, někdy i o několik \si{\mV}, některé tranzistory ve vnitřních
    zesilovacích stupních se mohou dostat do saturace. Doba zotavení nutná k přechodu ze saturace
    může trvat relativně dlouhou dobu a zesilovač je potom mnohem pomalejší než tehdy, kdy k
    saturaci nedošlo (\ref{aes:fig070}).

    \luagraphic[1]{aes_fig070.pdf}{Doba desaturace přebuzeného OZ bývá podstatně delší než jeho
                normální skupinové zpoždění (efektivně doba průchodu signálu ze vstupu na výstup)
                a často závisí na úrovni přebuzení (\emph{overdrive}).
                \cite[s.~2]{AN849}}{aes:fig070}

    \begin{tcnote}    
      Protože pouze malá skupina OZ na trhu mají uvedenou dobu zotavení pro různé úrovně přebuzení,
      je pro uživatele nezbytné stanovit tuto dobu experimentem.  Naměřené zpoždění bude záviset na
      úrovní přebuzení v dané aplikaci. Hodnota použitá při výpočtech, by měla být nejméně dvakrát
      větší než nejhorší hodnota pozorovaná testech, protože testované vzorky nemusí být
      reprezentativním vzorkem.
    \end{tcnote} 
    
    \luagraphic[0.5]{aes_fig071.pdf}{Stejná napájecí napětí umožňuje přímé buzení logického obvodu
                výstupem OZ. \cite[s.~2]{AN849}}{aes:fig071}

    Moderní OZ často mají výstupy typu rail to rail. Jejich největší kladná hodnota výstupního
    napětí se blíží k velikosti kladné hodnoty napájecího napětí, záporná hodnota se blíží k
    velikosti záporné hodnoty napájecího napětí. Když OZ a logické obvody mají stejná napájecí
    napětí, např. +\SI{5}{\volt}, je možné z výstupů OZ přímo budit následující logické obvody
    \ref{aes:fig071}. V opačném případě je nutné mezi výstupem OZ a vstupem logického obvodu vložit
    přizpůsobovací člen \cite[s.~2]{AN849} jak je tomu na obrázku \ref{aes:fig069}.
    

    \begin{figure}[ht!]  %\ref{aes:fig069}
      \centering
        \subcaptionbox{\label{aes:fig069a}}{\luafigure[0.5]{aes_fig069a.pdf}}  
        \subcaptionbox{\label{aes:fig069b}}{\luafigure[0.5]{aes_fig069b.pdf}} \\
        \subcaptionbox{\label{aes:fig069c}}{\luafigure[0.6]{aes_fig069c.pdf}}
      \caption{Přizpůsobení výstupu OZ ke vstupu logického obvodu: a) NPN, c) MOSFET N-channel, c)
                Complementary MOS inverter (\cite[s.~2]{AN849})}
      \label{aes:fig069}
    \end{figure}

    Nejjednodušší obvody rozhraní jsou \textbf{invertory}. Mohou být postaveny s NPN tranzistory,
    avšak za cenu vyšší spotřeby dané proudem báze, nebo výhodněji MOSFET s kanálem typu N.
    vytvořeny invertory s tranzistory NPN, případné s tranzistory MOSFET s kanálem typu N. 
    
    Rezistor \(R_B\) na obr. \ref{aes:fig069a} slouží k nastavení proudu báze, rezistor \(R_L\) k
    nastavení kolektorového proudu a je-li tranzistor uzavřen, zabezpečuje vstupní proud logického
    členu ve stavu log 1. Čím menší jsou velikosti odporů, tím rychlejší je činnost zapojení. NMOS
    tranzistor na obr. \ref{aes:fig069b} by měl mít malé prahové napětí \(V_{GS(th)}
    <\SI{2}{\volt}\) a také průrazné napětí hradla \(V_{GS_{max}}\) musí být větší než \(V_A\).
    Obecně musí platit, že napájecí napětí OZ \(+V_A > +V_L\) a zároveň \(-V_A < -V_L\).

    Invertor vytvořený pomocí komplementárních tranzistorů MOSFET je znázorněn na obr.
    \ref{aes:fig068c}. Jeho výhodou je malý klidový proud. Nevýhoda spočívá ve vzniku proudových
    špiček v časovém intervalu přepínání tranzistorů, kdy jsou oba tranzis tory krátkou dobu
    současně otevřeny.

    \section{Komparační úroveň a hystereze}
      \subsection{Komparační úroveň}
        
    \section{Shrnutí}
      \begin{itemize}[noitemsep]
        \item Operační zesilovač může být použit s velkou přesností jako komparátor na nízkých
              kmitočtech\footnote{Nemá smysl použít operační zesilovač jako komparátor, pokud je
              důležitá vysoká rychlost.}. Použití přesných OZ je důležité hlavně pn porovnávání
              signálů na úrovni mikrovoltů.
        \item Použití OZ může být výhodné též z cenových důvodů, když využíváme např. tři operační
              zesilovače ze čtyř umístěných v jednom pouzdru. Použití čtvrtého zesilovače jako
              komparátoru potom může být velmi ekonomické. 
        \item Pro jejich použití jako komparátorů je nutné znát rozdíly a podobnosti mezi
              komparátory a OZ, podrobně se seznámit s jejich vlastnostmi z technické dokumentace
              a to hlavně s jejich dobami zotavení, rychlostmi přeběhu, spotřebou a cenou.
        \item Komparátory jsou určeny pro rychlé rychlé spínání a v důsledku toho často mají horší
              DC parametry než mnoho operačních zesilovačů. Proto může být vhodné použít operační
              zesilovač jako komparátor v aplikacích vyžadující nízký \(V_{OS}\), nízký \(I_B\) a
              široký \(CMR\).
      \end{itemize}        

  
  \section{Aplikace}  


%---------------------------------------------------------------------------------------------------
