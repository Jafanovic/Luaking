% !TeX spellcheck = cs_CZ
{\tikzset{external/prefix={tikz/AES/}}
 \tikzset{external/figure name/.add={ch07_}{}}
%---------------------------------------------------------------------------------------------------
% file batt.tex
%---------------------------------------------------------------------------------------------------
%========= Kapitola: Autonomní zdroje elektrické energie ===========================================
\chapter{Autonomní zdroje elektrické energie}
\minitoc
  Většina přenosných elektronických zařízení potřebují ke své činnosti zdroj elektrické energie a 
  to nejčastěji ve formě stejnosměrného DC výkonu. Každý napájecí zdroj lze podle Theveninovy věty 
  nahradit sériovým spojením ideálního zdroje napětí a jeho vnitřního odporu. Vlastní zátěž lze 
  často nahradit lineárním rezistorem. Skutečná povaha napájecího zdroje bývá často složitější, 
  mívají charakter setrvačný, nelineární,pa\-ra\-me\-tri\-cký apod. Dokonce se někdy k malé radosti 
  setkáme 
  i se zdroji, které mají vnitřní odpor záporný. Definiční vztahy pro vnitřní a zatěžovací odpor 
  napájecího zdroje jsou následující
  \begin{itemize}
    \item \textbf{Vnitřní (výstupní) odpor zdroje}:
          \begin{equation}\label{aes:eq010}
            R_i = \frac{U_{2a}-U_{2b}}{I_{2a} - I{2b}} = -\der{U_2}{I_2}
          \end{equation}
    \item \textbf{Zatěžovací odpor}:
          \begin{equation}\label{aes:eq011}
            R_z = \frac{U_2}{I_2}
          \end{equation}
  \end{itemize}
  Přitom záporné znaménko ve vztahu (\ref{aes:eq010}) vyjadřuje skutečnost, že v obvyklém případě 
  zvýšení výstupního odebíraného proudu \(I_2\) způsobí snížení výstupního napětí \(U_2\). Pomocí 
  těchto dvou jednoduchých vztahů lze také rozlišit \textbf{zdroj napětí} \(– R_i \ll R_L\) od 
  zdroje proudu \(- R_i\gg R_L\). U elektronických napájecích zdrojů je běžné, že do určitého a 
  často nastavitelného zatěžovacího proudu se obvod chová jako zdroj napětí, po jeho překročení 
  jako zdroj proudu. Tomuto opatření říkáme \emph{nadproudová ochrana, omezení proudu, elektronická 
  pojistka}. Situace je na obr. 1-1.
  
  \section{Bateriové způsoby napájení}
    Bateriové napájení je výhodné svou nezávislostí na napájecí síti a tedy drátovém přívodu. 
    Vzhledem ke stále klesající spotřebě energie u napájených zařízení a rostoucí kvalitě 
    chemických zdrojů je tento způsob dnes velmi oblíbený.
   
    Je všeobecně známo, že bateriové (chemické) zdroje lze rozdělit na \textbf{primární} 
    (\emph{nenabíjitelné}) a \textbf{sekundární} (\textbf{a\-ku\-mu\-lá\-to\-ry}) – 
    \emph{nabíjitelné}. Rozdíly se dnes už stírají, existují např. baterie typu RAM,  které patří 
    mezi primární s možností nabíjení.
    
    \subsection{Primární (galvanické) články}
      Primární články přeměňují přímo chemickou energii v energii elektrickou a patří k velmi 
      starým zdrojům.
    \subsection{Sekundární články, akumulátory}
    \subsection{Palivové články}
    \subsection{Termoemisní generátory}
    \subsection{Termoelektrické články}
    \subsection{Sluneční (solární) , fotovoltaické články}
    
} % tikzset
%---------------------------------------------------------------------------------------------------
\printbibliography[title={Seznam literatury}, heading=subbibliography]
\addcontentsline{toc}{section}{Seznam literatury}
