% !TeX spellcheck = cs_CZ
{\tikzset{external/prefix={tikz/AES/}}
 \tikzset{external/figure name/.add={ch02_}{}}
%---------------------------------------------------------------------------------------------------
% file amplifier.tex
%---------------------------------------------------------------------------------------------------
%============================ Kapitola: Zesilovače==================================================
\chapter{Zesilovače}
\minitoc

  V této kapitole se budeme zabývat rozbory vlastností základních obvodů a jejich účelným
  spojováním do funkčních bloků určených pro zesilování signálů. \cite[p.~101]{Neumann}
  
  \section{Zjednodušení výpočet tranzistorového ze\-si\-lo\-va\-če}
    Přesný výpočet tranzistorového zesilovače vychází z určení dvojbranových pa\-ra\-me\-trů
    tranzistoru a pokračuje sestavením matice obvodu a řešením této matice. Při použití vybraných
    rovnic matematických modelů pro programy SPICE lze dojít ke zjednodušenému řešení, ve kterém se
    některé parametry zanedbají a sestavené náhradní schema pak řešit libovolnou metodou. Přesto
    dostaneme výsledky s přesností, která pro obvyklé technické řešení postačuje.

    \subsection{Obecná převodní charakteristika bipolární tranzistoru}
      Převodní charakteristika udává závislost výstupního proudu na vstupním napětí. Pro zapojení
      \texttt{SE} představuje převodní charakteristiku exponenciální závislost kolektorového proudu
      na napětí mezi bází a emitorem. Strmost je dána derivací funkce (tečnou) v daném pracovním
      bodě a odpovídá parametru $\mathrm{y_{21}}$.

} % tikzset
%---------------------------------------------------------------------------------------------------
\printbibliography[title={Seznam literatury}, heading=subbibliography]
\addcontentsline{toc}{section}{Seznam literatury}
