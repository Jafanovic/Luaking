% This animation helps explain the geometric interpretation of the 
% definite integral.
% http://texample.net/tikz/examples/animated-definite-integral/

\documentclass[tikz]{standalone}



\usepackage{tikz, pgf}
\usepackage{multido}


\begin{document}
%
\multido{\n=1+1}{20}{%
    \begin{tikzpicture}[scale=1.25]
    \draw[red,thick,<->] (-1,1) parabola bend (0,0) (2.1,4.41)
        node[below right] {\(y=x^2\)};
    \draw[loosely dotted] (-1,0) grid (4,4);
    \draw[->] (-0.2,0) -- (4.25,0) node[right] {\(x\)};
    \draw[->] (0,-0.25) -- (0,4.25) node[above] {\(y\)};
    \foreach \x/\xtext in {1/1, 2/2, 3/3}
    \draw[shift={(\x,0)}] (0pt,2pt) -- (0pt,-2pt) node[below] {\(\xtext\)};
    \foreach \y/\ytext in {1/1, 2/2, 3/3, 4/4}
    \draw[shift={(0,\y)}] (2pt,0pt) -- (-2pt,0pt) node[left] {\(\ytext\)};
%
    \def\mc{\n*\n}
    \pgfmathsetmacro\result{int(\n*\n*\n)}
    \shade[top color=blue,bottom color=gray!50]
        (0,0) parabola (0.1*\n,0.01*\mc) |- (0,0);
    \node[draw, fill=white] at (3cm,2cm) 
      {\(\int\limits_0^{\frac{\n}{10}}\!\!x^2\mathrm{d}x = \frac{\result}{3000}\)};
    \draw[fill=black,color=black] (0.1*\n,0.01*\mc) circle (1.2pt);
    \end{tikzpicture}
}
\end{document}