% https://users.math.cas.cz/~rehak/soubory/urc_int.pdf
\begin{mathexam}{Ukažme, že funkce definovaná předpisem
  \begin{equation*}
    f(x) = 
    \begin{cases}
      2x\sin\frac{1}{x} - \cos\frac{1}{x} & \text{pro} x\neq0, \\
      0                                   & \text{pro} x=0.
    \end{cases}
  \end{equation*}
  má primitivní funkcí na celém \(\realset\).
  }{exam155} 

  {\centering
    \captionsetup{type=figure} 
  % \luafigure[0.7]{example-image-a}
    \luafigure[1]{mai_fig081}
    \captionof{figure}{Chaotický průběh funkce \(f(x)\) v okolí \(0\) 
                       a její primitivní funkce \(F(x)\).}
    \label{mai:fig102}
  \par}
  
  Pomocí symbolického toolboxu \texttt{Matlabu} a následujícího skriptu
  \begin{lstlisting}[style=luaMatlabText,gobble=4]
    syms x;
    expr = 2*x*sin(1/x)-cos(1/x);
    F = int(expr)
  \end{lstlisting}
  snadno získáme hledanou primitvní funkci: 
  \begin{equation*}
    F(x) = 
      \begin{cases}
        x^2\sin\frac{1}{x} & \text{pro} x\neq0, \\
        0                  & \text{pro} x=0.
      \end{cases}
  \end{equation*}
\end{mathexam}