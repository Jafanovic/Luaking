% !TeX spellcheck = cs_CZ
% 
% \wikitextrule
\begin{mdframed}[style=mdexam]
  \begin{example}\label{mai:exam089}
    \begingroup
    \renewcommand\arraystretch{1.0}
    \renewcommand\arraycolsep{3pt}
    Předpokládejme, že přechod od báze \((\vec{e}_1, \vec{e}_2, \cdots, \vec{e}_3)\) v
    trojrozměrném prostoru k bázi \((\vec{e}'_1,\vec{e}'_2, \cdots, \vec{e}'_3)\) je popsán
    rovnicemi vyjadřujícími vektory čárkované báze jako lineární kombinace vektorů báze
    nečárkované takto:
    \begin{alignat*}{5}
      \vec{e}'_1&=-&& \vec{e}_1 && +\vec{e}_2 &&       \\
      \vec{e}'_2&= && \vec{e}_1 &&      && +\vec{e}_3, \\
      \vec{e}'_3&= && \vec{e}_1 && +\vec{e}_2 && +\vec{e}_3.    
    \end{alignat*}  
    Matici \(\matr{T}\) z těchto rovnic snadno „přečteme“. Vidíme, že se jedná o matici z
    příkladu \ref{mai:exam090}, k níž máme inverzál matici již spočtenu. Platí tedy
    \begin{equation*}
      T = 
        \begin{pmatrix*}[r]
         -1 & 1 & 0 \\
          1 & 0 & 1 \\
          1 & 1 & 1 
        \end{pmatrix*} 
      \qquad
      S = 
      \begin{pmatrix*}[r]
        -1 & -1 &  1 \\
         0 & -1 &  1 \\
         1 &  2 & -1
      \end{pmatrix*}     
    \end{equation*}
    Vektor \(\vec{u}\) má, dejme tomu, složky \(u = (2, -1, -1)\) v nečárkované bázi. Platí
    (rovnice \ref{mai:eq84})
    \begin{gather*}    
      \begin{aligned}
        (\vec{u}'_1\; \vec{u}'_2\;  \vec{u}'_3) &= (2\; -1\; -1)
        \begin{pmatrix*}[r]
          -1 & -1 &  1 \\
           0 & -1 &  1 \\
           1 &  2 & -1
        \end{pmatrix*} 
        (-3\; -3\;\; 2).                            \\
        \shortintertext{Pro kontrolu převedeme získané složky vektoru \(\vec{u}\) v čárkované bázi 
                        zpět do báze původní:}   \\
        (\vec{u}_1\; \vec{u}_2\;  \vec{u}_3) &= (-3\; -3\;\; 2)
        \begin{pmatrix*}[r]
          -1 & -1 &  0 \\
           1 &  0 &  1 \\
           1 &  1 &  1
        \end{pmatrix*} 
        (-2\; -1\; -1).   
      \end{aligned}  
    \end{gather*}  
    Vyšly původně zadané složky - počítali jsme zřejmě správně.
    \endgroup
  \end{example}
\end{mdframed}