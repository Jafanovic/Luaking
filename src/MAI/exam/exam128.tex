\begin{mdframed}[style=mdexam]
  \begin{example}\label{MAI:exam128}
    $\displaystyle\int{\frac{8x-31}{x^2-9x+14}}\dd{x}$. Převzato z \cite[s.~90]{Knichal}\newline
    Kořeny polynomu ve jmenovateli $\alpha_1 = 2$, $\alpha_2 = 7$ jsou jednoduché - každému z
    nich bude v rozkladu odpovídat jen jeden člen $$\frac{8x-31}{x^2-9x+14} = \frac{A}{x-2}
    + \frac{B}{x-7}.$$ Členy mnohočlenu na pravé straně seřadíme podle mocnin $x$ $$8x-31 =
     x(A+B)+(7A-2B).$$ Porovnáním odpovídajících si koeficientů dostaneme
    \begin{align*}
      8   &=   \; A + \, B \\
      -31 &= -7A - 2B
    \end{align*}
    Řešením této soustavy je $A = 3, B = 5$. Platí tedy (pro všechna $x \neq 2$ a $x \neq 7$)
    $$\frac{8x-31}{x^2-9x+14} = \frac{3}{x-2} + \frac{5}{x-7}.$$
    \begin{align*}
        &= \int{\frac{3}{x-2}}dx + \int{\frac{5}{x-7}}dx      \\
        &= 3\ln\abs{x-2} + 3\ln\abs{x-7} + C.
    \end{align*}
    Výsledek platí v každém intervalu, který neobsahuje body \(x = 2\), \(x = 7\).
  \end{example}
\end{mdframed}