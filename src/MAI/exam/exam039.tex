% !TeX spellcheck = cs_CZ
\wikitextrule
\begin{example}\label{mai:exam039}
  \textbf{Ještě jednou Gaussova eliminační metoda}\newline\small
  Je zadána soustava rovnic:
  \begin{alignat*}{7}
      x_1 &+ 2x_2 &-  x_3 &+  x_4 &- 5x_5 &=  &&0, \\
    -2x_1 &- 4x_2 &+ 2x_3 &+ 4x_4 &+ 4x_5 &= -&&6, \\
     -x_1 &- 2x_2 &+  x_3 &+ 5x_4 &-  x_5 &= -&&6.
  \end{alignat*}
  Rozšířená matice soustavy má nyní tvar 
  \begin{equation*}
    \matr{B} = (\matr{A}|\overline{\matr{B}}) =
    \left(
      \begin{array}{rrrrr|r}
         1 &  2 & -1 & 1 & -5 &  0    \\
        -2 & -4 &  2 & 4 &  4 & -6    \\
        -1 & -2 &  1 & 5 & -1 & -6
      \end{array}
    \right).
  \end{equation*}
  Stejné ekvivalentní úpravy jako v příkladu \ref{mai:exam038} vedou nyní k výsledku
  \begin{gather*}
    \matr{B} \sim
    \left(
      \begin{array}{rrrrr|r}
         1 &  2 & -1 & 1 & -5 &  0         \\
         \bm{0} &  0 &  0 & 6 & -6 & -6    \\
         \bm{0} &  0 &  0 & 6 & -6 & -6
      \end{array}
    \right) \sim
    \left(
      \begin{array}{rrrrr|r}
              1 &  2 & -1 & 1 & -5 &  0    \\
              0 &  0 &  0 & 1 & -1 & -1    \\
              0 &  0 &  0 & 1 & -1 & -1
      \end{array}
    \right) \sim
    \left(
      \begin{array}{rrrrr|r}
              1 &  2 & -1 & 1      & -5 &  0    \\
              0 &  0 &  0 & 1      & -1 & -1    \\
              0 &  0 &  0 & \bm{0} &  0 &  0
      \end{array}
    \right).
  \end{gather*}
  Nyní platí \(h(\matr{A}) = h(\matr{B}) = 2\). Podle Frobeniovy věty \ref{mai:lemma001} tedy 
  soustava určitě má řešení. Ekvivalentní soustava má tvar
  \begin{alignat*}{5}
         x_1 + 2x_2 - x_3 &+  x_4 &- 5x_5 &=  &&0, \\
                          &+  x_4 &-  x_5 &= -&&1, \\
                          &       &     0 &=  &&0.
  \end{alignat*}
  Poslední rovnice je identitou a můžeme ji vypustit. Máme pět neznámých a jen dvě nezávislé 
  rovnice. Dvě z neznámých tedy můžeme vyjádřit pomocí zbývajících. Postupujeme „odzadu“ , začínáme 
  druhou, jednodušší, rovnicí:
  \begin{align*}
                                                x_4 &= -1 + x_5,                \\
    x_1 = - 2x_2 + x_3 - x_4 + 5x_5 \Rightarrow x_1 &= -2x_2 + x_3 + 4x_5 + 1.
  \end{align*}
  Za neznámé vystupující na pravé straně, tj. \(x_2\), \(x_3\) a \(x_5\), můžeme dosazovat cokoli a 
  vždycky se k nějakým hodnotám \(x_1\) a \(x_4\) dopočítáme. Všechna řešení soustavy \(S\) proto 
  můžeme zapsat v obecném tvaru
  \begin{equation}\label{mai:eq040}
    (-2x_2 + x_3 + 4x_5 + 1, x_2, x_3, x_5 - 1, x_5).
  \end{equation}
  \normalsize
\end{example}