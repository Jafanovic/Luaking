% !TeX spellcheck = cs_CZ
\wikitextrule
\begin{example}\label{mai:exam078}
  \textbf{Počítání s komplexními čísly v algebraickém tvaru}\newline\small
    Zvolme \(z_1 = 1-2i\) a \(z_2 =-4+3i\). Vypočteme jejich součet, součin, komplexně sdružená 
    čísla, opačné prvky a inverzní prvky. Platí
    \begin{align*}
      z_1 + z_2 &= (1-2i) +(-4+3i) = (1 + (-4)) + i(-2+3) = -3 + i,                           \\
      z_1z_2    &= (1-2i)(-4+3i) = 1\cdot(-4) +1\cdot(3i) + (-2i)\cdot(-4) + (-2i)\cdot(3i)   \\
                &= -4 + 3i + 8i + 6 = 2 + 11i,                                                \\
      z_1^*     &= 1+2i, \quad z_2 =-4-3i, \quad -z_1 = -1+2i, \quad z_2 =4-3i,               \\
      z_1^{-1}  &= \dfrac{1}{1-2i} = \dfrac{1}{1-2i}\cdot \dfrac{1+2i}{1+2i} = 
                 = \dfrac{1+2i}{5} = \dfrac{1}{5} + \dfrac{2}{5}i,                            \\
      z_2^{-1}  &= \dfrac{1}{-4+3i} = \dfrac{1}{-4+3i}\cdot\dfrac{-4-3i}{-4-3i} = 
                 = \dfrac{-4-3i}{16+9} = -\dfrac{4}{25} - \dfrac{3}{25}i.
    \end{align*}
  \normalsize
\end{example}