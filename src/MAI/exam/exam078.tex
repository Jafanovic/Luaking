% !TeX spellcheck = cs_CZ
% \wikitextrule
\begin{mdframed}[style=mdexam]
  \begin{example}\label{mai:exam078}
    \textbf{Počítání s komplexními čísly v algebraickém tvaru}\newline
      Zvolme \(z_1 = 1-2\imath\) a \(z_2 =-4+3\imath\). Vypočteme jejich součet, součin, komplexně sdružená 
      čísla, opačné prvky a inverzní prvky. Platí
      \begin{gather*}
        \begin{aligned}
          z_1 + z_2 &= (1-2\imath) +(-4+3\imath) = 1 - 4 + \imath(-2+3) = -3 + \imath,                             \\
          z_1z_2    &= (1-2\imath)(-4+3\imath) = 1\cdot(-4) +1\cdot3\imath -2\imath\cdot(-4) -2\imath\cdot3\imath  \\
                    &= -4 + 3\imath + 8\imath + 6 = 2 + 11\imath,                                                  \\
          z_1^*     &= 1+2\imath, \quad z_2 =-4-3\imath, \quad -z_1 = -1+2\imath, \quad z_2 =4-3\imath,            \\
          z_1^{-1}  &= \dfrac{1}{1-2\imath} = \dfrac{1}{1-2\imath}\cdot \dfrac{1+2\imath}{1+2\imath}                 
                     = \dfrac{1+2\imath}{5} = \dfrac{1}{5} + \dfrac{2}{5}\imath,                                   \\
          z_2^{-1}  &= \dfrac{1}{-4+3\imath} = \dfrac{1}{-4+3\imath}\cdot\dfrac{-4-3\imath}{-4-3\imath}               
                     = \dfrac{-4-3\imath}{16+9} = -\dfrac{4}{25} - \dfrac{3}{25}\imath.
        \end{aligned}
    \end{gather*}
  \end{example}
\end{mdframed}