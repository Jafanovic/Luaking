% !TeX spellcheck = cs_CZ
\begin{mdframed}[style=mdexam]
  \begin{example}\label{mai:exam074}
    \textbf{Jak přesně lze změřit čínského císaře?}\newline
      V Číně se kdysi rozhodli změřit co nejpřesněji výšku svého císaře. Hlavní ideologický poradce
      pro problematiku přesnosti císařovy výšky se dočetl, že aritmetický průměr \(n\) měření je
      \(\sqrt{n}\)-krát přesnější než jednotlivé měření. Uvažoval takto: Každý věci oddaný Číňan
      dokáže na běžném lékařském zařízení pro měření výšky odečíst nastavenou hodnotu s přesností
      jednoho milimetru. Všichni Číňané jsou věci oddáni. Pozveme z nich tedy všechny, kteří umějí
      číst. Těch je zhruba jedna miliarda. Každý jednou změří výšku císaře. Poté se určí aritmetický
      průměr všech údajů. Jestliže je jedno měření provedeno s přesností jednoho milimetru, bude
      aritmetický průměr určen s přesností 
      \begin{equation*}
        \dfrac{\qty{1}{\mm}}{\sqrt{10^9}}\doteq\qty{3e-5}{\mm} = \qty{30}{\nm}
      \end{equation*}
      Třicet nanometrů, to už je nějaká přesnost!! Co myslíte? Je úvaha hlavního poradce správná?
      Rozhodně ne! Základní chybou je, že měření v principu nejsou prováděna za stejných podmínek,
      neboť do hry vstupují různí experimentátoři. Kdyby totiž každý (obecně \(j\)-tý) Číňan měřil
      císaře mnohokrát a mohl tak dostat rozdělení náhodné veličiny \(X_j\), která vstupuje do
      výpočtu průměru, byly by výsledky jednotlivých Číňanů obecně různé. Veličiny \(X_i\) až
      \(X_n\) by nesplňovaly výchozí předpoklad, že mají stejnou střední hodnotu a směrodatnou
      odchylku. (Je to zcela obdobné, jako kdybychom v příkladu se střelbou sledovali více střelců.
      Rozdělení náhodné veličiny představující bodový zisk při pěti výstřelech je u střeleckého
      mistra světa jistě výrazně jiné než u roztržitého profesora matematiky.) Navíc by směrodatné
      odchylky \(\sigma_j\) veličin \(X_j\) byly určitě větší než 1 milimetr. I když Číňan může při
      jednotlivém měření císaře přečíst údaj na přístroji tak, že jej odhadne na milimetry, budou se
      jednotlivé údaje odhadnuté na milimetry skoro jistě lišit o několik milimetrů a možná i o
      centimetry. Ani císař by po dobu tolika měření nedokázal neměnit svou výšku. Určitě by nestál
      celou dobu rovně.Jistě by se během doby všech měření měnila teplota a části přístroje by
      podléhaly teplotní roztažnosti. A uplatnila by se řada dalších vlivů. Je vidět, že myšlenka
      zpřesnit výsledky při zjišťování hodnot veličin tím, že budeme provádět více měření a po čítat
      aritmetický průměr, je sice velmilákavá, její správné použití však vyžaduje splnění poměrně
      přísných předpokladů.
      
      Nepoužitelnost tohoto způsobu „zpřesnění“ výsledku si můžeme uvědomit také pomocí následující
      jednoduché úvahy: Představme si, že Číňané svého císaře nikdy neviděli, ale vědí, že výška
      jeho postavy se nijak nevymyká běžné situaci. Budou ji namísto měření pouze odhadovat v
      rozmezí od \qty{150}{\cm} do \qty{180}{\cm}. Přesnost určení výšky císaře by se pak pohybovala v
      jednotkách mikrometrů
  \end{example}
\end{mdframed}