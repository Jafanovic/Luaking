\begin{mathexam}{Řešme \(\protect\scalerel{\int}{\frac{2x^2+34x+14}{x^3-4x^2-x-4}}\dd{x}\)
  \hfill\cite[s.~90]{Knichal}}{exam115}
    
    Polynom $Q(x)=x^3-4x^2-x-4$ má kořeny $\alpha_{1,2}=\pm1$, $\alpha_{3}=-4$, které jsou
    jednoduché tj. $Q(x)=(x-1)(x+1)(x+4)$ $$\frac{2x^2+34x+14}{x^3-4x^2-x-4} =
    \frac{A}{x-1}+\frac{B}{x+1}+\frac{C}{x+4}$$ Vynásobíme-li tuto rovnici společným jmenovatelem
    zlomků pravé strany (polynomem $Q(x)$), dostaneme
    \begin{gather*}
        \begin{align*}
          &= A(x+1)(x+4) + B(x-1)(x+4) + C(x-1)(x+1) \\
          &= A(x^2+5x+4) + B(x^2+3x-4) + C(x^2-1)    \\
          &= (A+B+C)x^2  + (5A+3B)x    + (4A-4B-C)
        \end{align*}
    \end{gather*}
    Porovnáním odpovídajících si koeficientů u stejných mocnin $x$ polynomu \(2x^2+34x+14\)
    dostaneme pro nez\-ná\-mé koeficienty $A, B, C$ soustavu rovnic
    \begin{align*}
    % \nonumber to remove numbering (before each equation)
       A+   B + C &= 2 \\
      5A + 3B     &= 34 \\
      4A - 4B - C &= 14
    \end{align*}
    Řešením této soustavy je $A = 5, B = 3, C = -6$ a tedy
    $$\frac{2x^2+34x+14}{x^3-4x^2-x-4} = \frac{5}{x-1}+\frac{3}{x+1}-\frac{6}{x+4}$$
    Dostáváme tři jednoduché integrály
    \begin{equation*}
      \int{\frac{5}{x-1}}\dd{x} + \int{\frac{3}{x+1}}\dd{x} + \int{\frac{6}{x+4}}\dd{x}            
    \end{equation*}
    jejichž řešení je 
    \begin{equation*}
      5\ln\abs{x-1} +  3\ln\abs{x+1} - 6\ln\abs{x+4} +c.
    \end{equation*}
\end{mathexam}