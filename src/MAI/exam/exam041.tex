% !TeX spellcheck = cs_CZ
\wikitextrule
\begin{example}\label{mai:exam041}
  \textbf{Obecná rovnice přímky}\newline\small
  Přímku \(p\) si snadno představíme jako průsečnici dvou nerovnoběžných rovin \(\varrho\) a 
  \(\sigma\). Jejich rovnice tvoří soustavu, která představuje obecné rovnice přímky
  \begin{subequations}\label{mai:eq042}
    \begin{align}
      \varrho &= \{(x, y, z)\in\mathbb{R}^3\mid a_1x + b_1y + c_1z + d_1 = 0 \},  \\ 
      \sigma  &= \{(x, y, z)\in\mathbb{R}^3\mid a_2x + b_2y + c_2z + d_2 = 0 \}, 
    \end{align}
  \end{subequations}
  Zkusme přijít na to, co musí platit pro koeficienty v rovnicích rovin, aby byly nerovnoběžné. 
  Jedna a táž přímka může být zadána různými dvojicemi nerovnoběžných rovin. Všechny roviny, které 
  přímkou \(p\) procházejí, tvoří geometrický útvar zvaný \textbf{svazek rovin prvního druhu}, 
  přímka sama je \textbf{osou} svazku.  
  \normalsize
\end{example}