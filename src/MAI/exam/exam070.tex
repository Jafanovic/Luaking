% !TeX spellcheck = cs_CZ
\begin{mdframed}[style=mdexam]
  \begin{example}\label{mai:exam070}
    \textbf{Kolik rychlostí má molekula plynu}\newline
    Tato otázka se zdá na první pohled zcela nesmyslná. Každý student fyziky ví, že molekuly plynu
    lze popisovat jako klasické částice, jejichž mechanický stav je jednoznačně určen polohovým
    vektorem a vektorem rychlosti. Molekula má tedy vždy určitou hodnotu rychlosti. Představte si
    ale takový plyn ve skutečnosti. Jeden mol jeho látkového množství (např. pro kyslík to
    představuje hmotnost \num{32} gramů) obsahuje asi \num{6.623e23} molekul! Kdybychom chtěli plyn
    popisovat jako soustavu klasických částic v mechanice, museli bychom v daném okamžiku znát
    polohu a rychlost každé molekuly z tohoto obrovského počtu. A to je principiálně nemožné,
    protože do chování takové soustavy zasahuje velmi podstatným způsobem „náhoda“. Nemůžeme určit,
    ve kterém bodě prostoru právě daná molekula je a jak rychle se pohybuje. Dokážeme pouze určit, s
    jakou pravděpodobností se nachází v elementárním objemu \(\Delta V = \Delta x \Delta y \Delta
    z\) v okolí daného bodu o polohovém vektoru \(\vec{r}\) a s jakou pravděpodobností \(\Delta P\)
    leží koncový bod vektoru její rychlosti v elementárním objemu \(\Delta\Omega = \Delta v_x \Delta
    v_y \Delta v_z\) „rychlostního“ prostoru v okolí zadané rychlosti \(\vec{v}\). Uvažujme o
    nejjednodušším modelu plynového tělesa, takzvaném ideálním plynu, jehož molekuly jsou stejné a
    navzájem neinteragují s výjimkou kratičkých náhodných srážek. Molekuly takového plynu jsou z
    hlediska pravděpodobnostního popisu navzájem ekvivalentní. Pravděpodobnost \(\Delta P\) bude pro
    všechny stejná a pro velmi malé elementární objemy bude dána vztahem
    \begin{equation*}
      \Delta P(\vec{r},\vec{v}) = \varrho(\vec{r},\vec{v})\Delta V \Delta\Omega,
    \end{equation*}
    kde \(\varrho(\vec{r},\vec{v}) = \varrho(x, y, z, v_x, v_y, v_z)\) je odpovídající hustota
    pravděpodobnosti. Jak ale hustota konkrétně závisí na polohách a rychlostech molekul? Tento
    fyzikální zákon, zvaný \textbf{Gibbsovo rozdělení}, se řídí exponenciální funkcí
    \begin{equation*}
      \varrho(\vec{r},\vec{v}) = K\exp\left(-\dfrac{E(\vec{r},\vec{v})}{kT}\right)
    \end{equation*}
    kde \(E\) je \textbf{mechanická energie molekuly} (kinetická plus potenciální v případném
    silovém poli), \(T\) je \textbf{absolutní teplota plynu} udávaná v kelvinech a \(k =
    \SI{1.38e-23}{\joule\per\kelvin}\) je \textbf{Boltzmannova konstanta}.
    
    Zajímá-li vás, proč si příroda v tomto případě vybrala zrovna exponenciální funkci, sledujte
    následující orientační úvahu: Rozdělme si v myšlenkách plynové těleso na dvě části, jimž
    odpovídají energie \(E_1\) a \(E_2\). Celková energie soustavy je \(E = E_1 + E_2\). Označme
    \(P(E)\) pravděpodobnost, že, se soustava nachází ve stavu s energií \(E\), pravděpodobnosti, že
    se jednotlivé části nachází nezávisle ve stavech s energiemi \(E_1\) a \(E_2\), pak jako
    \(P(E_1)\) a \(P(E_2)\). Pravděpodobnost, že se první část soustavy nachází ve stavu s energií
    \(E_1\) a \textbf{současně} druhá část ve stavu s energií \(E_2\), je rovna součinu
    pravděpodobností těchto nezávislých jevů. Proto \(P(E_1 + E_2) = P(E_1) \cdot P(E_2)\). Tuto
    vlastnost mají ovšem právě exponenciální funkce. Platí tedy \(\varrho \approx\exp(\beta E)\).
    Konstantu \(\beta\) určí jen experiment, z něhož vychází \(\beta = - (kT)^{-1}\).
    
    Vrátíme se nyní k výchozímu problému, neboť úvodní otázka nabyla smyslu: Molekula může mít
    libovolnou rychlost s větší či menší pravděpodobností. Nebude-li ideální plyn umístěn v žádném
    silovém poli, bude mechanická energie molekuly dána pouze energií kinetickou. Elementární
    pravděpodobnost, že koncový bod rychlosti molekuly leží v elementárním objemu \(\Delta\Omega\) v
    okolí bodu \(\vec{v}\) „rychlostního“ prostoru, bez ohledu na to, v jaké části „obyčejného“, tj.
    \textbf{konfiguračního prostoru} se vyskytuje, je
    \begin{align*}
      \Delta P(\vec{v}) &= \varrho(\vec{v})\Delta\Omega                                          \\
                        &= C\exp\left(-\dfrac{m(v_x^2 + v_y^2 + v_z^2)}{2kT}\right)\Delta\Omega.
    \end{align*}
    Tato pravděpodobnost, jak je vidět, nezávisí na směru rychlosti, pouze na její velikosti,
    \(\varrho(\vec{v}) = \varrho(v)\). Konstantu \(C\) určíme snadno. Pravděpodobnost, že molekula
    má vůbec nějakou rychlost, je rovna jedné (jistý jev). Matematický zápis této skutečnosti
    vyžaduje znalost takzvaného trojného integrálu (integrujeme podle tří proměnných  - složek
    vektoru rychlosti). V našem případě se však výpočet redukuje na součin tří integrálů
    jednoduchých,
    \begin{equation*}
      \int_{\Omega}\varrho(v_x, v_y, v_z)\dd{v_x}\dd{v_y}\dd{v_z} = 1
    \end{equation*}
    \begin{align*}
      \Rightarrow C &\cdot
                      \int_{-\infty}^{\infty}\exp\left(\dfrac{mv_x^2}{2kT}\right)\dd{v_x} 
                 \cdot\int_{-\infty}^{\infty}\exp\left(\dfrac{mv_y^2}{2kT}\right)\dd{v_y}   \\
                &\cdot\int_{-\infty}^{\infty}\exp\left(\dfrac{mv_z^2}{2kT}\right)\dd{v_z} =1.
    \end{align*}
    Po substitucích \(mv_i^2/2kT = u^2,\, i = x, y, z\) vede výpočet na
    \textbf{Laplaceův integrál}
    \begin{equation*}
      \int_{-\infty}^{\infty}\exp(-u^2)\dd{u} = \sqrt{\pi}.
    \end{equation*}
    Dostáváme
    \begin{align*}
      C &= \left(\dfrac{m}{2\pi kT}\right)^{\frac{3}{2}} \Rightarrow  \\
      \Delta P(\vec{v}) &= \left(\dfrac{m}{2\pi kT}\right)^{\frac{3}{2}}
                           \exp\left(- \dfrac{mv^2}{2 kT}\right)\dd{v_x}\dd{v_y}\dd{v_z}
    \end{align*}
    Hustota pravděpodobnosti je stejná pro všechny koncové body vektoru rychlosti \(\vec{v}\) ležící
    v rychlostním prostoru na kulové ploše o poloměru rovném velikosti rychlosti \(v\). Jaká bude
    elementární pravděpodobnost \(\Delta P(v)\), že molekula má velikost rychlosti v intervalu \((v,
    v + \Delta v)\) bez ohledu na směr pohybu? Tuto pravděpodobnost dostaneme, vezmeme-li za
    \(\Delta\Omega\) objem tenké kulové slupky o poloměru \(v\) a tloušťce \(\Delta v\), v níž končí
    všechny vektory rychlosti, jejichž velikost leží v požadovaném intervalu. Tento objem je
    \(\Delta\Omega = 4\pi v^2\Delta v\) a
    \begin{gather*}
      P(v) = 4\pi\left(\dfrac{m}{2\pi kT}\right)^{\frac{3}{2}}v^2
                \exp\left(- \dfrac{mv^2}{2 kT}\right)\Delta v = f_M(v)\Delta v. 
    \end{gather*}

    {\centering
      \captionsetup{type=figure}
      \luafigure[1]{mai_fig048.pdf}
      \captionof{figure}{Maxwellovo rozdělení rychlostí molekul dusíku pro teploty \(T_1 =
                          \SI{300}{\kelvin}\) a \(T_2 = \SI{500}{\kelvin}\).
      \cite[s.~243]{Musilova2009MA1}
      \label{mai:fig048}}
    \par}
    
    Dokážete vyložit, proč jsme zvolili za \(\Delta\Omega\) celý objem slupky? Počítáme totiž
    pravděpodobnost, že koncový bod vektoru rychlosti molekuly leží, zhruba řečeno, v kterémkoli
    elementárním kvádříku \(\Delta v_x\Delta v_z\Delta v_z\) obsaženém ve slupce. A ta je součtem
    pravděpodobností odpovídajících všem kvádříkům vytvářejícím slupku. Jedná se o pravděpodobnosti
    navzájem neslučitelných jevů (pohybuje-li se molekula v jednom směru, nepohybuje se v jiném).
    Hustota této pravděpodobnosti se nazývá \textbf{Maxwellovo rozdělení rychlostí}. Na rozdíl od
    Gaussova rozdělení, popisujícího hustotu pravděpodobnosti pro jednotlivé složky rychlosti, je
    nesymetrická vlivem faktoru \(v^2\). Obrázek \ref{mai:fig048} ukazuje funkci \(f_M(v)\) pro dvě
    různé teploty \(T_2 > T_1\). Důležité hodnoty spjaté s tímto rozdělením jsou
    \textbf{nejpravděpodobnější rychlost} \(v_p\), \textbf{střední rychlost} \(\langle v \rangle\) a
    \textbf{střední kvadratická rychlost} \(\langle v^2 \rangle\). Platí
    \begin{align*}
      \der{f_M}{v}        &= 0\, \longrightarrow v_P = \sqrt{\dfrac{2kT}{m}},                    \\
      \langle v \rangle   &= \int_{-\infty}^{\infty}vf_M(v)\dd{v} = \sqrt{\dfrac{8kT}{\pi m}},   \\
      \langle v \rangle^2 &= \int_{-\infty}^{\infty}v^2f_M(v)\dd{v} = \dfrac{3kT}{m}
    \end{align*}
  \end{example}
\end{mdframed}