  % !TeX spellcheck = cs_CZ
% Musilova2009MA2
\begin{mdframed}[style=mdexam]
  \begin{example}\label{mai:exam086}
    \textbf{Příklad o sáňkování}\newline
    Kdo bude rychlejší na sáňkách? Tatínek o hmotnosti \(M\), nebo Pepíček s Mařenkou o hmotnosti
    \(m\)? Každý fyzik hned namítne, že tíhové zrychlení, které určuje rozjezd sáněk, na hmotnosti
    nezávisí. Z praxe však víme, že tatínkové bývají rychlejší. Jak to? Zřejmě proto, že sáňkování
    ve vakuu není obvyklé. Kromě průmětu tíhové síly do nakloněné roviny (\(F_1 = Mg\sin\alpha\)),
    který „nás urychluje", jsme brzděni silou odporu prostředí. V jednoduchém přiblížení ji můžeme
    předpokládat ve tvaru \(F_2 = -Cv^2\). Konstanta \(C\) závisí na tvaru a rozměrech pohybujícího
    se objektu a na hustotě vzduchu. Pro jednoduchost předpokládáme, že je stejná u tatínka i
    Pepíčka. Fyzikální zákon \(Ma= F_1 + F_2 = Mg\sin\alpha - Cv^2\)

    {\centering
    \captionsetup{type=figure}
    \luafigure[1]{mai_fig057.jpg}
    \captionof{figure}{Ladův obrázek dětí na sáňkách.}
    \label{mai:fig057}
    \par}

    například pro tatínka můžeme přepsat na diferenciální rovnici takto:
    \begin{equation}\label{mai:eq079}
      M\der{v}{t} = Mg\sin\alpha - Cv^2
    \end{equation}
    
    Hledanou funkcí je nyní časová závislost rychlosti, jako počáteční podmínku budeme uvažovat, že
    rychlost v čase \(t = 0\) byla nulová. Řešením této počáteční úlohy je funkce
    \begin{align*}
      v(t) &= \sqrt{\dfrac{Mg\sin\alpha}{C}}
              \left[\dfrac{e^{2t\sqrt{\dfrac{Cg\sin\alpha}{M}}}-1}
                          {e^{2t\sqrt{\dfrac{Cg\sin\alpha}{M}}}+1}
              \right]                                                                           \\
          &= \sqrt{\dfrac{Mg\sin\alpha}{C}}\tanh\left(t\sqrt{\dfrac{Cg\sin\alpha}{M}}\right).
    \end{align*}
    Graf takovéto závislosti pro dvě různé hmotnosti \(M = \qty{100}{\kg}\) (červená) a \(m =\qty{10}
    {\kg}\) (modrá) vidíme na obrázku \ref{mai:fig057}. (Poměr hmotností byl takto zvolen pro
    zvýraznění rozdílnosti výsledků - každému je zřejmé, že mimino samo sáňkovat nemůže.) Další
    hodnoty: \(\alpha = \ang{20}\), \(g = \qty{10}{\m\per\square\s}\), \(C =
    \qty{1.00}{N\m^2s^{-2}}\). Na obrázku \ref{mai:fig058} si všimněme, že rychlost se nejprve

    {\centering
    \captionsetup{type=figure}
    \luafigure[1]{mai_fig058.png}
    \captionof{figure}{Graf řešení úlohy o sáňkování. \cite[s.~221]{Musilova2012MA2}}
    \label{mai:fig058}
    \par}

    poměrně prudce zvyšuje, ale poté se asymptoticky blíží k tzv. \textbf{mezní rychlosti} \(v_{max}
    = Mg\sin\alpha\). Mezní rychlost odpovídá situaci, kdy se síly \(F_1\) a \(F_2\) „vyrovnají“.
    (Taková situace však nenastane, je pouze limitním případem pro \(t \rightarrow \infty\).) 
  \end{example}
\end{mdframed}