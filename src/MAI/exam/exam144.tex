\begin{mathexam}{\(\scalerel{\int}{\left(\dfrac{7}{x}+ \dfrac{1}{x^2} -
    \dfrac{1}{4}x^2\right)\dd{x}}\)}{exam144} 
    %
    Protože v intervalu \(x\in(0, +\infty)\) existují integrály
    \begin{align*}
      \scalerel{\int}{\dfrac{1}{x}\dd{x}}   &= \ln x + c_1  \\
      \scalerel{\int}{\dfrac{1}{x^2}\dd{x}} &= \int x^{-2}\dd{x} = -\dfrac{1}{x} + c_2  \\
      \mathlarger{\int}{x^2\dd{x}}            &= \dfrac{x^3}{3} + c_3 ,
    \end{align*}  
    platí v tomto intervalu
    \begin{multline*}
                  7\scalerel{\int}{\dfrac{1}{x}\dd{x}}   + 
                   \scalerel{\int}{\dfrac{1}{x^2}\dd{x}} - 
       \dfrac{1}{4}\mathlarger{\int}{x^2\dd{x}} =                   \\
      =7\ln x - \dfrac{1}{x} - \dfrac{1}{12}x^3 + c.
    \end{multline*}
\end{mathexam}