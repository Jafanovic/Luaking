% !TeX spellcheck = cs_CZ
\begin{mdframed}[style=mdexam]
  \begin{example}\label{mai:exam073}
    \textbf{Rozptyl aritmetického průměru}\newline
    Již v úvodu odstavce o náhodných veličinách jsme konstatovali, že opakujeme-li v nezměněných
    podmínkách měření jisté fyzikální veličiny (délka závěsu kyvadla, proud procházející vodičem,
    napětí na vodiči, atd.), budeme díky náhodným vlivům dostávat pokaždé poněkud jiný výsledek.
    Říkáme, že měření je zatíženo náhodnými chybami. Výsledek získaný při každém opakování lze
    interpretovat jako hodnotu náhodné veličiny. Dejme tomu, že jsme provedli uměření fyzikální
    veličiny \(X\) a získali hodnoty \(x_1\) až \(x_n\) . V praktické situaci budou tyto hodnoty
    většinou navzájem různé, nemusí tomu tak však nutně být. Fyzikální veličinu chceme ovšem
    reprezentovat jediným údajem, a tím bude její střední hodnota, tj.
    \textbf{aritmetický průměr}
    \begin{equation*}
      \langle x \rangle = \dfrac{x_1 + x_2 + \cdots + x_n}{n}.
    \end{equation*}
    Rozptyl veličiny \(X\) je dán vztahem
    \begin{align*}
      D &= D(X)                                                             \\
        &= \dfrac{\left(x_1 - \langle x \rangle\right)^2 + 
                  \left(x_2 - \langle x \rangle\right)^2 + \cdots +
                  \left(x_n - \langle x \rangle\right)^2}{n}.
    \end{align*}
    Víme, že směrodatná odchylka \(\sigma( x ) = \sqrt{D(X)}\) určuje, nakolik jsou jednotlivé
    výsledky měření v průměru odchýleny od střední hodnoty, charakterizuje tedy přesnost každého
    opakování měření. Podívejme se na celou úlohu z jiné strany: Představme si, že sledujeme \(n\)
    po dvou nezávislých náhodných veličin \(X_1\) až \(X_n\) se shodnou střední hodnotou \(\langle
    x_j \rangle = \langle x \rangle\) a shodnou směrodatnou odchylkou \(\sigma(x_j) = \sqrt{D},\, 1
    \leq j \leq n\). Aritmetický průměr těchto veličin,
    \begin{equation*}
      \langle \Xi \rangle = \dfrac{X_1 + X_2 + \cdots + X_n}{n}.
    \end{equation*}
    je tedy rovněž náhodnou veličinou. Pro jeho střední hodnotu, rozptyl a směrodatnou odchylku
    platí
    \begin{align*}
      \langle \xi \rangle 
                  &= \dfrac{n\langle x \rangle}{n}, \\
      D(\Xi)      &= \dfrac{1}{n^2}\cdot D(X_1 + \cdots + X_n)=\dfrac{nD^2}{n^2}=\dfrac{D}{n}, \\
      \sigma(\xi) &= \dfrac{\sigma(x)}{\sqrt{n}}.
    \end{align*}
  \end{example}
\end{mdframed}