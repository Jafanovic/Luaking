% !TeX spellcheck = cs_CZ
%====================== Sbírka řešených příkladů ==================================================
\begin{mdframed}[style=mdexam]
  \begin{example}\label{mai:exam118}
    Najdi primitivní funkce k funkcím \cite[s.~254]{Brabec1989}:
    \begin{enumerate}
      \item \(f(x) = 2x+5\)
      \item \(f(x) = \sin x\) na libovolném intervalu \(J\subset(-\infty,+\infty)\),
      \item \(f(x) = \dfrac{1}{\sqrt{1-x^2}}\) na intervalu \((-1, 1)\),
      \item \(f(x) = -\dfrac{1}{x^2} + \dfrac{1}{\sin^2x} + \sin x\).
    \end{enumerate}
    \vspace{1em}
    \textbf{Řešení}:\newline

    Funkce \(F(x) = x^2+5x+3\) je v intervalu \(-\infty, \infty\) primitivní funckí k funkci \(f(x)
    = 2x+5\), neboť pro každé \(x\in(-\infty,+\infty)\) platí \(F'(x)=2x+5\).

    K funkci \(\sin x\) je primitivní funkcí na libovolném intervalu \(J\subset(-\infty,+\infty)\) 
    funkce \(-\cos x\), protože \((-\cos x)' = \sin x\). Ale též funkce \(3-\cos x\) je primitivní 
    funkcí k funkci \(\sin x\), protože \((3 - \cos x)' = \sin x\) pro všechna \(x\in(-\infty, 
    \infty)\).

    K funkci  \(\dfrac{1}{\sqrt{1-x^2}}\) je primitivní funkcí na intervalu \((-1, 1)\) funkce
    \(\arcsin x\), ale též \(-\arccos x\), \(1 + \arcsin x\), \(\arcsin x + c\), \(c\in\realset\),
    protože platí \((\arcsin x)' = (-\arccos x)' = (1+\arcsin x)' = (\arcsin x + c)' =
    \dfrac{1}{\sqrt{1-x^2}}, x\in(-1, 1)\).

    V intervalu \((0,\pi)\) je primitivní funkcí každá funkce \(F(x) = \dfrac{1}{x}-\cotg x - \cos x
    +C\), kde \(C\) je libovolná konstanta, neboť pro všechna \(x\in(0,\pi)\) a pro každé \(C\)
    platí \(F'(x) = \left(\dfrac{1}{x}-\cotg x -\cos x +C\right)'= -\dfrac{1}{x^2} +
    \dfrac{1}{\sin^2x} + \sin x\).
  \end{example}
\end{mdframed}