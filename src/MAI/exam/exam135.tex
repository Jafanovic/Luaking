\begin{mathexam}{Polynom \(Q(x) = x^4-x^3-x^2-x-2\) má zřejmě kořen \(\alpha_1=-1\).}{exam135} 
    \begin{equation*}
      \rotatebox{90}{$
        {\polylongdiv[style=C,div=:]{x^4-x^3-x^2-x-2}{x+1}}
      $}
    \end{equation*}
    Výsledný polynom má kořen \(\alpha_2=2\). Vydělením kořenovým činitelem \((x-2)\) dostaneme
    polynom \((x^2+1)\), který je již nerozložitelný v reálné kořenové činitele. Tedy
    \begin{equation*}
      Q(x) \equiv x^4-x^3-x^2-x-2 = (x+1)(x-2)(x^2+1), 
    \end{equation*}
    což je již tvar \eqref{mai:eq159}.
\end{mathexam}