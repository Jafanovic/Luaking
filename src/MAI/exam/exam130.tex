\begin{mdframed}[style=mdexam]
  \begin{example}\label{MAI:exam130}
    \begin{equation}\label{MA:int_ex_02}
      \int\frac{3}{(1+x^2)x^2}\dd{x}
    \end{equation}
    Převzato z Zdroj \cite[s.~29]{Knichal}. Integrand upravíme přičtením a odečtením výrazu $3x^2$ v
    čitateli zlomku takto:
    \begin{align*}
      \frac{3}{(1+x^2)x^2} 
        &= \frac{3+3x^2-3x^2}{(1+x^2)x^2} = \frac{3}{x^2}-\frac{3}{1+x^2}                      \\  
      \intertext{Tedy v každém otevřeném intervalu, který neobsahuje bod \(x=0\), platí}
      \int{\frac{3}{(1+x^2)x^2}\dd{x}} 
        &= 3\int{\frac{1}{x^2}dx} - 3\int{\frac{1}{1+x^2}dx}                                   \\
        &= -\frac{3}{x}-3\arctan x + c. 
    \end{align*}
  \end{example}
\end{mdframed}