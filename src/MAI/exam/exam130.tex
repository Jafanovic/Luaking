% \int\frac{3}{(1+x^2)x^2}\dd{x}
  [\ref{mai:eq146}]: Převzato z \cite[s.~29]{Knichal}. Integrand upravíme přičtením a
  odečtením výrazu $3x^2$ v čitateli zlomku takto:
  \begin{equation*}
    \frac{3}{(1+x^2)x^2} = \frac{3+3x^2-3x^2}{(1+x^2)x^2} = \frac{3}{x^2}-\frac{3}{1+x^2}  
  \end{equation*}
  Tedy v každém otevřeném intervalu, který neobsahuje bod \(x=0\), platí
  \begin{gather*}
    \int{\frac{3}{x^2}\dd{x}} - \int{\frac{3}{1+x^2}\dd{x}} = -\frac{3}{x}-3\arctan x + c. 
  \end{gather*}