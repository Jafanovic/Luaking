% https://users.math.cas.cz/~rehak/soubory/urc_int.pdf
\begin{mathexam}{Vezměme funkci 
  \begin{equation*}
    f(x) = 
    \begin{cases}
       0 & \text{pro} x\in\realset\backslash \{0\}, \\
       1 & \text{pro} x=0.
   \end{cases}
  \end{equation*}
  viz \ref{mai:fig080}. Dokažte, že k takto definované funkci neexistuje primitivní funkce na
  \(\realset\).
  }{exam154} 

  {\centering
    \captionsetup{type=figure} 
    \luafigure[0.7]{mai_fig080.pdf}
    \captionof{figure}{Funkce, která nemá primitivní funkci na žádném okolí \(0\)}
    \label{mai:fig080}
  \par}
  
  Kdyby \(F(x)\) byla primitivní funkcí k \(f(x)\) na \(\realset\), pak by \(F(x)\) byla na
  \(\realset\) spojitá. Ze zadání má být \(F'(x) = f(x) = 0\) na \(\realset\backslash \{0\}\), tj.
  funkce \(F(x)\) by musela být konstantní jak na \((−\infty, 0)\), tak na \((0, ∞)\). Tyto dva
  postřehy znamenají, že \(F(x)\) by byla konstantní na celém \(\realset\). Potom by však \(F'(0) =
  0\), což nesouhlasí s \(f(0) = 1\). Proto primitivní funkce k \(f(x)\) na celém \(\realset\)
  existovat nemůže. Stejně tak neexistuje na žádném intervalu obsahujícím \(0\), na ostatních
  intervalech však jistě existuje (a může to být jakákoli konstantní funkce).
\end{mathexam}