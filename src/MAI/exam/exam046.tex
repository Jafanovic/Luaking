% !TeX spellcheck = cs_CZ
\wikitextrule
\begin{example}\label{mai:exam046}
  \textbf{Kolik nul má (aditivní) grupa?}\newline\small
  Jak dokážeme, že má grupa právě jednu nulu? Co když předchozímu tvrzení nebudeme věřit? Můžeme se 
  o jeho pravdivosti přesvědčit? Ten, kdo mu nevěří, si může třeba představit, že grupa má nuly 
  dvě. Označme je \(0_G\) a \(\overline{0}_G\). Vztah \ref{mai:eq049} platí pro libovolný prvek 
  grupy, proto \(0_G + \overline{0}_G = 0_G\) (to jsme brali \(a = 0_G\) a \(\overline{0}_G\) 
  považovali za nulu) a současně \(0_G + \overline{0}_G = \overline{0}_G\) (nyní zase byl prvek 
  \(0_G\) v roli nuly a \(a = \overline{0}_G\)). Je vidět, že \(0_G = \overline{0}_G\). Nula je 
  tedy skutečně jen jedna. Podobnou úvahu můžeme provést pro jedničku multiplikativní grupy. Platí 
  tedy tvrzení:
  
  Neutrální prvek grupy je určen jednoznačně.
  
  Stejně tak bychom mohli mít pochybnosti o tom, že opačný prvek k danému \(a\in G\) je jen jeden. 
  Předpokládejme, že \(b\) a \(c\) jsou dva opačné prvky k \(a\). Platí \(a + b = 0_G\). Přičteme k 
  této rovnosti \(c\) zleva, tj. \(c + a + b = c\). Protože však \(c + a = 0_G\), dostáváme \(b=c\) 
  a tedy:
  
  Opačný (resp. inverzní) prvek k libovolně zvolenému prvku grupy je určen jednoznačně. 
  \normalsize
\end{example}