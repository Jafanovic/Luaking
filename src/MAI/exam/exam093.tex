% !TeX spellcheck = cs_CZ 
\begin{mathexam}{Světelný zdroj \(B\) (např. pouliční svítilna) má vzdálenost \SI{36}{\m} od
  světelného zdroje \(A\). Zdroj \(B\) má osmkrát větší intenzitu než zdroj \(A\). Který bod na
  spojnici obou zdrojů bude nejméně osvětlený? (Přitom intenzita osvětlení světelným zdrojem je
  přímo úměrná intenzitě zdroje a klesá s druhou mocninou vzdálenosti od uvažovaného
  zdroje.)}{exam093} 
  
  {\centering
    \captionsetup{type=figure}
    \luafigure[1]{mai_fig063.png} 
    \captionof{figure}{K příkladu \ref{mai:exam093}. Kredit: \cite[s.~48]{rektorys2011}}
    \label{mai:fig063}
  \par}      

  Matematická formulace: Označme \(a\) intenzitu zdroje \(A\). Pak intenzita zdrojo \(8\) je \(8a\).
  Označme dále \(x\) vzdálenost bodu \(P\) na spojnici bodů \(A\) a \(B\) meřenou od zdroje \(A\).
  Pak intenzita osvětlení v bodě \(P\) od zdroje \(A\) bude úměrná číslu \(\frac{a}{x^2}\) zdroje B
  číslu \(\frac{8a}{(36 - x)^2}\) (se stejnou konstantou úměrnosti). Máme tedy najít v intervalu
  \((0,36)\) takové \(x\), pro které bude součet obou intenzit minimalni, a tj.
  \begin{equation}\label{mai:eq089}
    y = \dfrac{a}{x^2} + \dfrac{8a}{(36 - x)^2} = \text{min}
  \end{equation}
  Ale funkce \ref{mai:eq089} má všude v otevřeném intervalu  \((0,36)\) první a druhou derivaci.
  Nastává chvíle si procvičit derivování podílu:
  \begin{gather*}
    \begin{align*} 
      \left(\dfrac{a}{x^2}\right)'          
        &= (ax^{-2})' = -\dfrac{2a}{x^3},   \\ 
      \left( \dfrac{8a}{(36 - x)^2}\right)' 
        &= \dfrac{(8a)'\cdot(36 - x)^2 - (8a)\cdot(36^2 - 72x + x^2)'}{(36 - x)^4} \\
        &= \dfrac{-8a(-72 + 2x)}{(36 - x)^4} 
         = +\dfrac{8a\cdot2\cdot\cancel{(36 - x)}}{(36 - x)^{\cancel{4}3}}
         =  \dfrac{16a}{(36 - x)^3}
    \end{align*}
  \end{gather*}
  (neboť \(8a = \text{konst}\), takže \((8a)' = 0\); k derivování funkce \(8a/(36 - x)^2\) jsme také
  mohli místo věty o derivování podílu použít větu o derivování složených funkcí a psát
  \begin{equation*}
    \dfrac{8a}{(36 - x)^2} = \dfrac{8a}{z}, \quad\text{kde}\quad z = 36 - x
  \end{equation*}  
  Tedy 
  \begin{align*}
    y'  &= -\dfrac{2a}{x^3} + \dfrac{16a}{(36 - x)^3} \\
    y'' &=  \dfrac{6a}{x^4} + \dfrac{48a}{(36 - x)^4}. 
  \end{align*}
  Jak víme, v bodě lokálního minima bude \(y' = 0\)
  \begin{equation}\label{mai:eq090}
    -\dfrac{2a}{x^3} + \dfrac{16a}{(36 - x)^3} = 0.
  \end{equation}
  Rovnici můžeme řešit tak, že zlomky na levé straně dáme na společného jmenovatele. (Zkuste to,
  nebude to hezká rovnice!) Ale má být \(x > 0\), \(36 - x > 0\), takže (\ref{mai:eq090}) můžeme
  zapsat ve tvaru
  \begin{align*}
    \dfrac{(36 - x)^3}{x^3} &= -\dfrac{16a}{2a} = 8 \qquad\text{takže}      \\
    \dfrac{36 - x}{x}       &= \sqrt[3]{8} \Rightarrow 36 - x = 2x.
  \end{align*}
  Zároveň \(y''>0\), takže v bodě \(x=12\) bude ostré lokální minimum. Nejméně osvětlený bod bude
  tedy ve vzdálenosti \SI{12}{\m} od zdroje \(A\).

  {\centering
  \captionsetup{type=figure}
  \luafigure[1]{mai_fig064.pdf} 
  \captionof{figure}{K příkladu \ref{mai:exam093}.}
  \label{mai:fig064}
  \par}   
\end{mathexam}