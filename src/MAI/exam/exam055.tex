% !TeX spellcheck = cs_CZ
\wikitextrule
\begin{example}\label{mai:exam055}
  \textbf{Hody kostkou a mincí - jev \(C\)}\newline\small
  Označme jako jev \(A\) „Při náhodném hodu kostkou padne šestka.“ a jako jev \(B\) \uv{Při	
  náhodném hodu mincí padne hlava.} Platí
  \begin{alignat*}{3}
    N(A) &= 6\qquad   M(A) &&=1 \qquad \Rightarrow \qquad p(A) &&= \frac{1}{6}  \\
    N(B) &= 2\qquad   M(B) &&=1 \qquad \Rightarrow \qquad p(B) &&= \frac{1}{2}  \\
  \end{alignat*}
  Jev \(C\) je definován jako \(A\) \textbf{a} \(B\), tj. „Při náhodném provedení současného hodu 
  kostkou a mincí padne na kostce šestka a na minci hlava.“ Počítejme pravděpodobnost \(p(C)\). 
  Jevy \(A\) a \(B\) jsou \textbf{nezávislé}, to znamená, že výsledek hodu kostkou neovlivní 
  výsledek hodu mincí a naopak. Počet možných výsledků současného hodu kostkou a mincí je
  \begin{equation*}
    N(C) = N(A \text{ a } B) = N(A)N(B) = 12.
  \end{equation*}
  Každý možný výsledek hodu kostkou je totiž možno kombinovat s každým možným výsledkem hodu mincí.
  Označme výsledky hodu mincí jako \(\mathcal{A}\) (hlava neboli avers) a opačný výsledek jako 
  \(\mathcal{R}\). (orel neboli revers). Výčet možných výsledků současného hodu kostkou a mincí je

  \begin{table}[h]
    \centering
    \begin{tabular}{c|rrrrrrrrrrrr}
      \textbf{kostka} & 1 & 2 & 3 & 4 & 5 & 6 & 1 & 2 & 3 & 4 & 5 & 6 \\ \hline
      \textbf{mince}  & \(\mathcal{A}\) & \(\mathcal{A}\) & \(\mathcal{A}\) & \(\mathcal{A}\) & 
                        \(\mathcal{A}\) & \(\mathcal{A}\) & \(\mathcal{R}\) & \(\mathcal{R}\) & 
                        \(\mathcal{R}\) & \(\mathcal{R}\) & \(\mathcal{R}\) & \(\mathcal{R}\) 
    \end{tabular}
    % \caption{ }
  \end{table}
  
  Příznivý případ je pouze jeden, tj. situace, kdy se výsledek \num{6} na kostce kombinuje s 
  výsledkem \(\mathcal{A}\) na minci
  
  \begin{equation*}
    M(C) = M(A)M(B) = 1.
  \end{equation*}
  \begin{equation*}
    p(C) = \dfrac{M(C)}{N(C)} = \dfrac{M(A)M(B)}{N(A)N(B)} 
         = \dfrac{M(A)}{N(A)}\cdot\dfrac{M(B)}{N(B)} = p(A)p(B).
  \end{equation*}
  \normalsize
\end{example}