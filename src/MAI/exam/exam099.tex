% !TeX spellcheck = cs_CZ
\begin{mdframed}[style=mdexam]
  \begin{example}\label{mai:exam099}
    K sestavení vlajky, která má být složena ze tří různobarevných vodorovných pruhů, jsou k
    dispozici látky barvy bílé, červené, modré, zelené a žlůté. Kredit:
    \cite[s.~14]{calda2008matematika} \newline
    \begin{enumerate}[noitemsep]
      \item Určete počet vlajek, které lze z látek těchto barev sestavit.
      \item Kolik z nich má modrý pruh?
      \item Kolik jich má modrý pruh uprostřed?
      \item Kolik jich nemá uprostřed červený pruh?  
    \end{enumerate}
    \textbf{Řesení}
    \begin{enumerate}[noitemsep]
      \item Vzhledem k tomu, že každé dva pruhy mají být různé barvy a že záleží na pořadí těchto
            pruhů, jde o tříčlenné variace z pěti prvk. Z látek daných barev lze sestavit \(V(3,5) =
            5\cdot4\cdot3=60\) různých vlajek.
      \item Vlajku s modrým pruhem dostaneme tak, že vybereme uspořádanou dvojici pruhů z látek
            barvy bílé, červené, zelené a žlůté (to lze provést \(V(2,4)=4\cdot3=12\) způsoby) a
            přidáme pruh modrý (což lze provést třemi způsoby: nahoru, doprostřed, dolů). Vlajek s
            modrým pruhem je tedy \(3\cdot V(2,4)=36\).
      \item Vlajek s modrým pruhem uprostřed je zřejmě \(V(2,4)=12\), neboť pro zařazení modrého
            pruhu už nemáme tři možnosti jako v případě předchozím, ale jedinou. 
      \item Počet vlajek, které nemají uprostřed červený pruh, je stejný jako počet vlajek, které
            nemají uprostřed modrý pruh. Tento počet je roven počtu všech vlajek zmenšenému o počet
            vlajek, které mají uprostřed modrý pruh, tj. číslu \(60-12=48\).
    \end{enumerate}


  \end{example}
\end{mdframed}