% !TeX spellcheck = cs_CZ
\wikitextrule
\begin{example}\label{mai:exam059}
  \textbf{Kolika let se dožijeme?}\newline\small
  V rámci evidence obyvatelstva se často sledují různé údaje, které slouží k odhadům vývoje 
  mohutnosti populace. Dejme tomu, že v jihomoravském regionu zjistili, že ze statisíce dětí, které 
  se dožily pěti let, se v průměru dožije dvaceti let \num{93} tisíc a osmdesáti let \num{36} 
  tisíc. Jaká je pravděpodobnost, že vy, kteří jste se již dvaceti let dožili, se dožijete 
  osmdesátky? Označme jako jev \(A\) „Pětileté dítě se dožije osmdesáti let.“ a jako jev \(B\) 
  „Pětileté dítě se dožije dvaceti let.“ Je zřejmé, že v tomto případě platí \(p(A) = p(A\text{ a 
  }B)\) (jestliže se někdo dožil osmdesáti let, s jistotou se předtím dožil dvaceti let). My 
  posuzujeme pravděpodobnost nastoupení jevu \(A\) za podmínky, že nastal jev \(B\), tj. podmíněnou 
  pravděpodobnost \(p_B(A)\). Platí
  \begin{equation*}
    p(A)   = p(A\text{ a }B) = \num{0.36}, p(B) = \num{0.93}\Rightarrow 
    p_B(A) = \dfrac{p(A\text{ a }B)}{p(B)} = \dfrac{\num{0.36}}{\num{0.93}} \simeq \num{0.39},
  \end{equation*}
  Že ta pravděpodobnost není velká? Nezoufejte. Čísla byla fiktivní a předpovědi říkají, že již v 
  roce \num{2015} bude u nás průměrný věk žen \num{83} let, u mužů, bohužel, o něco nižší. Ještě 
  zpřesněme úvahu, která nás vede k závěru \(p(A\text{ a }B) = p(A)\). Pravděpodobnost nastoupení 
  jevu \(B\) za podmínky, že nastal jev \(A\), je v našem případě rovna jedné. Jak jsme totiž již 
  konstatovali, každý, kdo se dožil osmdesátky, se s jistotou dožil i dvacítky. Platí
  pravděpodobnost \(p_B(A)\). Platí
  \begin{equation*}
    p_A(B) = \dfrac{p(A\text{ a }B)}{p(A)} = 1 \Rightarrow p(A\text{ a }B) = p(A),
  \end{equation*}
\normalsize
\end{example}