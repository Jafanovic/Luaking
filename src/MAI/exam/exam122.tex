\begin{mathexam}{Náhrada proměnně \(x\) funkcí. Typické jsou neurčité integrály, které vedou na
  goniometrické substituce, například \[\int\sqrt{1-x^2}\dd{x}\]}{exam122}
    
  Označme \(x=\psi(t)=\sin(t)  \Rightarrow \psi'(t)=\cos(t)\). Budeme potřebovat také základní
  goniometrické vzorce (\ref{MA1:eq_sincos},\ref{MA1:eq_cos2x} a \ref{MA1:eq_sin2x}). Můžeme psát
  \begin{gather*}
    \int\sqrt{1-\sin^2t}\cos t\dd{t} 
      = \int\cos^2 t \dd{t}  = \int\frac{1+\cos2t}{2}\dd{t}                         
  \end{gather*}
  Dostáváme
  \begin{align*}
      &= \frac{1}{2}t+\frac{\sin2t}{4}+c                       \\
      &= \frac{1}{2}\arcsin x + \frac{2\sin t\cos t}{4}        \\
      &= \frac{1}{2}\arcsin x + \frac{x\sqrt{1-x^2}}{2} + c.
  \end{align*}
  Správně bychom měli místo \(\sqrt{1 - \sin^2x}\) psát \(\abs{\cos x}\). Vzhledem k tomu, že jde o
  neurčitý integrál, je možné hledat primitivní funkci na intervalu, kde platí \(\cos x = \abs{\cos
  x}\).
\end{mathexam}