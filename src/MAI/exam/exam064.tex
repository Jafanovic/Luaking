% !TeX spellcheck = cs_CZ
\wikitextrule
\begin{example}\label{mai:exam064}
  \textbf{Ještě jednou střelba, tentokrát přesněji}\newline\small
  Názvem pochopitelně nemyslíme přesnější střelbu, ale přesnější komentář, který již bude založen 
  na našich znalostech o pravděpodobnosti. Dejme tomu, že podmínky střelby jsou pevně dány a nemění 
  se. Patří k nim zcela jistě typ zbraně, typ terče, vzdálenost stanoviště střelce od terče, 
  základní povětrnostní podmínky. Výsledky jsou pak závislé na zručnosti střelce, avšak jsou 
  ovlivněny náhodným i vlivy (foukne nenadálý vítr, střelec se lekne, zatřese se mu ruka, náhodně 
  se mírně pozmění vzdálenost ústí hlavně od terče nebo její sklon, apod.). Počty dosažených bodů 
  daného střelce při jednom výstřelu, nebo při sérii deseti výstřelů, atd., jsou tedy náhodnými
  veličinami. Pokusme se posoudit zručnost střelce přesněji. Označme jako náhodnou veličinu \(X\) 
  počet bodů dosažených při jednom výstřelu. Nejprve určeme, jakých hodnot může nabývat. Všichni 
  víme, jak vypadá běžný střelecký terč. Aby však naše počty nebyly příliš komplikované a 
  zdlouhavé, uvažujme o terči mnohem jednodušším. Bude tvořen vnitřním černým kruhem s hodnotou 
  \(3\) body, dále středním šedivým mezikružím s hodnotou \(2\) body a vnějším bílým mezikružím s 
  hodnotou \(1\) bod. Střelba do terče mimo vnější kružnici nebo zcela mimo terč představuje 
  bodovou hodnotu \(0\). Při jednom výstřelu tedy může střelec docílit v principu jakékoli z 
  možných hodnot
  \begin{equation*}
    X\in\{x_1, x_2, x_3,x_4\} = \{0,1,2,3\}.
  \end{equation*}
  Informace, jakých hodnot může náhodná veličina nabývat, je jistě nejen cenná, ale je pro jakékoli 
  další úvahy nezbytná. Sama o sobě je však nepostačující. O střelcově zručnosti se na základě 
  konstrukce terče nic nedovídáme. Kvalitativně jinou informaci získáme, víme-li, že možných hodnot 
  zásahu dociluje střelec s následujícími pravděpodobnostmi:
  \begin{table}[h]
    \centering
    \begin{tabular}{c|rrrr}
      \(x_i\)    &      0     &      1     &      2     &      3       \\ \hline
      \(p_i\)    & \num{0.03} & \num{0.28} & \num{0.52} & \num{0.17} 
    \end{tabular}
    % \caption{ }
  \end{table}
  Můžeme tak třeba zjistit, kolika bodů střelec zhruba docílí s vysokou pravděpodobností při pěti 
  výstřelech. Tento počet je
  \begin{equation*}
    5\cdot(\num{0.03}\cdot0 + \num{0.28}\cdot1 + \num{0.52}\cdot2 + \num{0.17}\cdot3) = 
    5\cdot\num{1.83} = \num{9.15} \simeq 9.
  \end{equation*}
  Pro každou pětici výstřelů může být počet dosažených bodů samozřejmě poněkud odlišný. Veličina 
  \(Y\) představující počet dosažených bodů na pět výstřelů je rovněž veličinou náhodnou. Je nám 
  však jasné, že hodnota dosažených bodů v každé pětici výstřelů je s vysokou pravděpodobností 
  blízká číslu 9. Co to znamená „s vysokou pravděpodobností“? Dokážeme ji spočítat? Pokusme se o 
  to. Především bychom museli určit, o kolik bodů se smí dosažený počet lišit od hodnoty 9, abychom 
  jej ještě považovali za „blízký číslu 9“ . Tato volba závisí čistě na naší vůli a bude jí 
  odpovídat i vypočtená pravděpodobnost. Dejme tomu, že zvolíme tento interval od 7 do 11 bodů 
  včetně. Jev, jehož pravděpodobnost hledáme, je tedy
  \begin{itemize}
    \item \(A\) : P ři pěti výstřelech získá střelec 7 nebo 8 nebo 9 nebo 10 nebo 11 bodů.
  Jevy
  A j : Střelec získá při pěti výstřelech j bodů.
    jsou po dvou neslučitelné, pravděpodobnost jevu A tedy bude rovna součtu pravděpodobností
  \end{itemize}
\normalsize
\end{example}