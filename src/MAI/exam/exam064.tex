% !TeX spellcheck = cs_CZ
\wikitextrule
\begin{example}\label{mai:exam064}
  \textbf{Ještě jednou střelba, tentokrát přesněji}\newline\small
  Názvem pochopitelně nemyslíme přesnější střelbu, ale přesnější komentář, který již bude založen 
  na našich znalostech o pravděpodobnosti. Dejme tomu, že podmínky střelby jsou pevně dány a nemění 
  se. Patří k nim zcela jistě typ zbraně, typ terče, vzdálenost stanoviště střelce od terče, 
  základní povětrnostní podmínky. Výsledky jsou pak závislé na zručnosti střelce, avšak jsou 
  ovlivněny náhodným i vlivy (foukne nenadálý vítr, střelec se lekne, zatřese se mu ruka, náhodně 
  se mírně pozmění vzdálenost ústí hlavně od terče nebo její sklon, apod.). Počty dosažených bodů 
  daného střelce při jednom výstřelu, nebo při sérii deseti výstřelů, atd., jsou tedy náhodnými
  veličinami. Pokusme se posoudit zručnost střelce přesněji. Označme jako náhodnou veličinu \(X\) 
  počet bodů dosažených při jednom výstřelu. Nejprve určeme, jakých hodnot může nabývat. Všichni 
  víme, jak vypadá běžný střelecký terč. Aby však naše počty nebyly příliš komplikované a 
  zdlouhavé, uvažujme o terči mnohem jednodušším. Bude tvořen vnitřním černým kruhem s hodnotou 
  \(3\) body, dále středním šedivým mezikružím s hodnotou \(2\) body a vnějším bílým mezikružím s 
  hodnotou \(1\) bod. Střelba do terče mimo vnější kružnici nebo zcela mimo terč představuje 
  bodovou hodnotu \(0\). Při jednom výstřelu tedy může střelec docílit v principu jakékoli z 
  možných hodnot
  \begin{equation*}
    X\in\{x_1, x_2, x_3,x_4\} = \{0,1,2,3\}.
  \end{equation*}
  Informace, jakých hodnot může náhodná veličina nabývat, je jistě nejen cenná, ale je pro jakékoli 
  další úvahy nezbytná. Sama o sobě je však nepostačující. O střelcově zručnosti se na základě 
  konstrukce terče nic nedovídáme. Kvalitativně jinou informaci získáme, víme-li, že možných hodnot 
  zásahu dociluje střelec s následujícími pravděpodobnostmi:
  \begin{table}[ht!]
    \centering
    \begin{tabular}{c|rrrr}
      \(x_i\)    &      0     &      1     &      2     &      3       \\ \hline
      \(p_i\)    & \num{0.03} & \num{0.28} & \num{0.52} & \num{0.17} 
    \end{tabular}
    % \caption{ }
  \end{table}
  Můžeme tak třeba zjistit, kolika bodů střelec zhruba docílí s vysokou pravděpodobností při pěti 
  výstřelech. Tento počet je
  \begin{equation*}
    5\cdot(\num{0.03}\cdot0 + \num{0.28}\cdot1 + \num{0.52}\cdot2 + \num{0.17}\cdot3) = 
    5\cdot\num{1.83} = \num{9.15} \simeq 9.
  \end{equation*}
  Pro každou pětici výstřelů může být počet dosažených bodů samozřejmě poněkud odlišný. Veličina 
  \(Y\) představující počet dosažených bodů na pět výstřelů je rovněž veličinou náhodnou. Je nám 
  však jasné, že hodnota dosažených bodů v každé pětici výstřelů je s vysokou pravděpodobností 
  blízká číslu 9. Co to znamená „s vysokou pravděpodobností“? Dokážeme ji spočítat? Pokusme se o 
  to. Především bychom museli určit, o kolik bodů se smí dosažený počet lišit od hodnoty 9, abychom 
  jej ještě považovali za „blízký číslu 9“ . Tato volba závisí čistě na naší vůli a bude jí 
  odpovídat i vypočtená pravděpodobnost. Dejme tomu, že zvolíme tento interval od 7 do 11 bodů 
  včetně. Jev, jehož pravděpodobnost hledáme, je tedy
  \begin{itemize}
    \item \(A\) : Při pěti výstřelech získá střelec \num{7} nebo \num{8} nebo \num{9} nebo \num{10} 
          nebo \num{11} bodů.
    \item[] Jevy
    \item \(A_j\) : Střelec získá při pěti výstřelech \(j\) bodů.
  \end{itemize}
  jsou po dvou neslučitelné, pravděpodobnost jevu \(A\) tedy bude rovna součtu pravděpodobností
  \begin{equation*}
    p(A) = \sum_{j=7}^{11} p(A_j).
  \end{equation*}
  Pozor! Pravděpodobnosti \(p(Aj)\) jsou odlišné od pravděpodobností zadaných v tabulce. Týkají se 
  totiž pěti výstřelů, zatímco tabulka je pro jeden výstřel. Musíme tedy určit pravděpodobnosti 
  \(p(A_j)\). K tomu je třeba zjistit všechny možnosti, jak docílit součtu bodů \(j\) pomocí pěti 
  sčítanců nabývajících hodnot \num{0} až \num{3}. Soupis je v následující tabulce. Tabulka uvádí 
  různé rozklady součtu na sčítance, počet případů, jak se tento rozklad realizuje (různé pořadí 
  dosažených bodů při jednotlivých výstřelech) a pravděpodobnosti jednotlivých rozkladů
  zaokrouhlené na tři platná místa.
  
  \begin{table}[ht!]
    \centering
    \resizebox{0.5\textwidth}{!}{%
    \begin{tabular}{c|rrrr}
      součet \(j\) & sčítance   &    počet   & ppst rozkladu & \(p_(A_j)\)    \\ \hline
            7      & 3+3+1+0+0  & \num{30}   & \num{2.18e-4} & \num{0.102}    \\
                   & 3+2+2+0+0  & \num{30}   & \num{1.24e-3} &                \\
                   & 3+2+1+1+0  & \num{60}   & \num{1.25e-2} &                \\
                   & 3+1+1+1+1  & \num{5}    & \num{5.22e-3} &                \\
                   & 2+2+2+1+0  & \num{20}   & \num{2.36e-2} &                \\
                   & 2+2+1+1+1  & \num{10}   & \num{5.94e-2} &                \\ \hline
            8      & 3+3+2+0+0  & \num{30}   & \num{4.06e-4} & \num{0.185}    \\
                   & 3+3+1+1+0  & \num{30}   & \num{2.04e-3} &                \\
                   & 3+2+2+1+0  & \num{60}   & \num{2.32e-2} &                \\
                   & 3+2+1+1+1  & \num{20}   & \num{3.88e-2} &                \\
                   & 2+2+2+2+0  & \num{5}    & \num{1.10e-2} &                \\
                   & 2+2+2+1+1  & \num{10}   & \num{1.10e-1} &                \\ \hline
            9      & 3+3+3+0+0  & \num{10}   & \num{4.42e-5} & \num{0.238}    \\
                   & 3+3+2+1+0  & \num{60}   & \num{7.57e-3} &                \\
                   & 3+3+1+1+1  & \num{10}   & \num{6.34e-3} &                \\
                   & 3+2+2+2+0  & \num{20}   & \num{1.43e-2} &                \\
                   & 3+2+2+1+1  & \num{30}   & \num{1.08e-1} &                \\
                   & 2+2+2+2+1  & \num{5}    & \num{1.02e-1} &                \\ \hline
           10      & 3+3+3+1+0  & \num{20}   & \num{8.25e-4} & \num{0.215}    \\
                   & 3+3+2+2+0  & \num{30}   & \num{7.03e-3} &                \\
                   & 3+3+2+1+1  & \num{30}   & \num{3.53e-2} &                \\
                   & 3+2+2+2+1  & \num{20}   & \num{1.34e-1} &                \\
                   & 2+2+2+2+2  & \num{1}    & \num{3.80e-2} &                \\ \hline
           11      & 3+3+3+2+0  & \num{20}   & \num{1.53e-3} & \num{0.133}    \\
                   & 3+3+3+1+1  & \num{10}   & \num{3.85e-3} &                \\
                   & 3+3+2+2+1  & \num{30}   & \num{6.56e-2} &                \\
                   & 3+2+2+2+2  & \num{5}    & \num{6.21e-2} &                \\ \hline
    \end{tabular}}
    % \caption{ }
  \end{table}
  
  Jak jsme spočítali pravděpodobnosti jednotlivých rozkladů? Dejme tomu, že počet bodů dosažených 
  při pěti výstřelech je \(j\). Nechť \(j = j_1 + j_2 + j_3 + j_4 + j_5\) je rozklad hodnoty \(j\) 
  na součet pěti sčítanců. Čísla \(j_1\) až \(j_5\) se pohybují od \num{0} do \num{3}, mohou být i 
  shodná. Například jeden z možných rozkladů bodového součtu \(j = 9\) je \(9 = 3 + 2 + 2 + 1 + 
  1\). Pravděpodobnost, že při pěti nezávislých výstřelech budou jednotlivé zásahy činit \(j_1\), 
  \(j_2\), \(j_3\) , \(j_4\)a \(j_5\) bodů, je
  \begin{equation*}
    p_{j1}p_{j2}p_{j3}p_{j4}p_{j5}.
  \end{equation*}
  Pro případ zvoleného konkrétního rozkladu čísla \num{9} je to \(p_3p_2^2p_1^2\) - Zásahy s 
  bodovým ziskem \(j_1\) až \(j_5\) mohou být realizovány v různých pořadích (například \(\num{3} + 
  \num{2} + \num{2} + \num{1} + \num{1} = \num{2} + \num{3} + \num{2} + \num{1} + \num{1} 
  =\ldots\), atd.). Označme počet všech takových pořadí \(n_{j_1\ldots j_5}\). (Pro \(j = \num{3} + 
  \num{2} + \num{2} + \num{1} + \num{1}\) je \(n_{32211} = \begin{pmatrix} 5 \\ 2 
  \end{pmatrix}\begin{pmatrix} 3 \\ 2 \end{pmatrix} = 30\): Dvě z pěti pozic pro umístění dvou
  dvojek lze vybrat \(\begin{pmatrix} 5 \\ 2 \end{pmatrix}\)  způsoby, ze zbývajících tří pozic lze 
  dvě pro umístění jedniček vybrat \(\begin{pmatrix} 3 \\ 2 \end{pmatrix}\) způsoby a na
  trojku zbude poslední pozice.) Pravděpodobnost \(p_{j_1\ldots j_5}\) daného rozkladu čísla \(s\) 
  pak je
  \begin{equation*}
    p_{j_1\ldots j_5} = n_{j_1\ldots j_5}p_{j_1}p_{j_2}p_{j_3}p_{j_4}p_{j_5}
  \end{equation*}
  Pro rozklad \(\num{9} = \num{3} + \num{2} + \num{2} + \num{1} + \num{1}\) je (výsledek viz také v 
  tabulce)
  \begin{equation*}
    p_{32211} =30\cdot p_3p_2^2p_1^2 = \num{30}\cdot\num{0.17}\cdot\num{0.52}^2\cdot\num{0,282} 
              \simeq \num{0.108}.
  \end{equation*}
  Pravděpodobnost \(p(A_j)\) je součtem pravděpodobností jednotlivých rozkladů čísla \(j\). Konečně 
  pravděpodobnost, že střelec dosáhne bodového výsledku \(y \in [7, 11]\), je rovna součtu 
  pravděpodobností \(p(A_j)\) pro \(j = 7, 8, 9, 10, 11\). Tato hodnota je \num{0.873}, tedy 
  opravdu poměrně vysoká, jak jsme očekávali. Také bychom mohli říci, že střelec dosahuje při každé 
  pětici výstřelů „v průměru“ \num{9} bodů, a tedy při jednom výstřelu „v průměru“ \num{1.8} bodů. 
  Pokud bychom „průměrnou hodnotu“ jednoho výstřelu počítali i pro jiný počet výstřelů než pro pět, 
  budeme dostávat čísla, která budou hodnotě \num{1.8} velmi blízká.
\normalsize
\end{example}