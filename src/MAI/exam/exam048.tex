% !TeX spellcheck = cs_CZ
\begin{mdframed}[style=mdexam]
  \begin{example}\label{mai:exam048}
    \textbf{Algebraická struktura na množině komplexních čísel}\newline
      Množina \(\cmplxset\) komplexních čísel je kartézským součinem reálných os, \(\cmplxset = 
      \mathbb{R} \times \mathbb{R}\), tedy množinou uspořádaných dvojic \([a, b]\) čísel reálných. 
      Značíme \(z = [a, b]\). Reálné číslo \(a = \operatorname{Re}(z)\) je \emph{reálnou} částí 
      komplexního čísla \(z\) a reálné číslo \(b = \operatorname{Im}(z)\) částí \emph{imaginární}. 
      Operace sčítání a násobení komplexních čísel jsou definovány takto:
      \begin{align*}
        \cmplxset\times\cmplxset      &\ni\left[[a_1, b_1],[a_2, b_2]\right] \longrightarrow   \\
          [a_1, b_1] + [a_2, b_2]     &= [a_1 + a_2, b_1 + b_2]\in\cmplxset,                   \\
        \cmplxset\times\cmplxset      &\ni\left[[a_1, b_1],[a_2, b_2]\right] \longrightarrow   \\
          [a_1, b_1] \cdot [a_2, b_2] &= 
          \begin{array}{c}
            \left[a_1 \cdot a_2 - b_1 \cdot b_2\right], \\
            \left[a_1 \cdot b_2 + a_2 \cdot b_1\right]
          \end{array}
          \in\cmplxset.
      \end{align*}
      
      \begin{mdframed}[style=highlight]
        Množina komplexních čísel s operací součtu je komutativní grupou.
      \end{mdframed}
      
      Jejím neutrálním prvkem je číslo \(0_{\cmplxset} = [0,0]\), opačným prvkem k číslu \(z = [a, 
      b]\) je \(-z = [-a, - b]\). Při operaci násobení je neutrálním prvkem číslo [1,0], prvkem 
      inverzním k číslu \(z = [a, b] \neq 0_{\cmplxset}\) je
      \begin{equation*}
        z^{-1} = \left[\dfrac{a}{a^2 + b^2}, \dfrac{-b}{a^2 + b^2}\right].
      \end{equation*}
      
      K číslu \(0_{\cmplxset} = [0,0]\) však inverzní prvek opět neexistuje. Množina komplexních 
      čísel \textbf{není grupou vzhledem k násobení}.
  \end{example}
\end{mdframed}