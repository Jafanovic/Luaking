\begin{mathexam}{\(\int x\sin x\dd{x}\)}{exam111} 
  
  Součin v zadání je zřejmý. Můžeme si zvolit buď \(u=x\) a \(v'=\sin x\), nebo naopak \(u=\sin x\)
  a \(v'= x\).
  
  Zkusíme nejprve první volbu. Je-li \(u=x\) bude \(u=1\). Dále \(v'=\sin x\), tedy \(v=\int\sin
  x\dd{x} = -\cos x\) (integrační konstantu volíme rovnou nule, stačí nám jedna konkrétní primitivní
  funkce). Ze vzorce \eqref{mai:eq147} dostaneme
  \begin{align*}
    \int x\sin x\dd{x} &= x(-\cos x) - \int1\cdot(-\cos x)\dd{x} \\
                       &= -x\cos x + \sin x + c.
  \end{align*}  
  Tato volba tedy vedla k cíli. Výpočet obvykle zapisujeme do jakési tabulky, takže zápis vypadá
  následovně:
  \begin{align*}
    \int x\sin x\dd{x} &= %
      \left\lvert
        \begin{array}{ll} 
          u = x     & u' =1        \\
          v'=\sin x & v  = -\cos x 
        \end{array}  
      \right\rvert =                                              \\
                       & = x(-\cos x) - \int1\cdot(-\cos x)\dd{x} \\
                       &= -x\cos x + \sin x + c.
  \end{align*}  
\end{mathexam}