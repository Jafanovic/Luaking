% !TeX spellcheck = cs_CZ
\begin{mathexam}{Fyzika - speciální typy pohybů}{exam036}
  Při rovnoměrném pohybu tělesa (ať již přímočarém či křivočarém) je dráha, kterou těleso urazí za
  dobu \(t\), přímo úměrná velikosti jeho rychlosti \(v\), tj. \(s(t) = s_0 + vt\). Při pohybu
  rovnoměrně zrychleném (zpožděném) je lineární závislost velikosti rychlosti na čase, tj. \(v(t) =
  v_0 \pm at\) při pohybu přímočarém (\(a\) je velikost zrychlení), nebo \(v(t) = v_0 \pm a_\tau t\)
  při pohybu křivočarém (\(a_\tau\) je velikost průmětu zrychlení do směru tečny k trajektorii
  tělesa - tečného zrychlení).
\end{mathexam}