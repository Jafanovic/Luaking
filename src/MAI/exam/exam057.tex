% !TeX spellcheck = cs_CZ
\wikitextrule
\begin{example}\label{mai:exam052}
  \textbf{Bernoulliův pokus}\newline\small
  Bernoulliův pokus spočívá v tom, že \(n\)-krát nezávisle provedeme určitý pokus, například hod 
  mincí. (V terminologii teorie pravděpodobnosti nazýváme každé takové provedení opakováním 
  pokusu.) Sledujeme, v kolika případech z těchto \(n\) opakování nastal daný jev (například jev 
  \(A\) — padne hlava). Výsledek opakování pokusu, při kterém daný jev nastal, nazveme zdarem, 
  výsledek, kdy nastal jev opačný, nezdarem. Dejme tomu, že pravděpodobnost zdaru je \(p\). (Pro 
  případ padnutí hlavy na minci je \(p = 1/2\).) Pravděpodobnost nezdaru je pak \((l - p)\).
  (V případě hodů mincí je \((1 — p) = 1/2\).) Zajímáme se o to, jaká je pravděpodobnost \(P(x)\), 
  že při \(n\) opakováních pokusu docílíme \(x\)-krát zdaru, \(x\) přitom můžeme předem volit 
  libovolně v rozmezí \(0 \leq x \leq n\). V případě hodů mincí jistě dokážeme předem odhadnout, 
  že pravděpodobnosti \(P(0)\) a \(P(n)\), tj. pravděpodobnosti toho, že nepadne hlava vůbec nebo 
  že padne hlava vždy, budou při větším počtu opakování pokusu malé a budou se blížit nule tím 
  více, čím větší bude \(n\). Naopak bychom se mohli domnívat, že pravděpodobnost \(P(n/2)\), 
  tj. že padne hlava v polovině opakování pokusu, by měla být při velkém počtu \(n\) blízká 
  \SI{100}{\percent}. Správnost tohoto našeho předběžného odhadu však posoudíme teprve poté, co si 
  odvodíme obecný vzorec pro \(P(x)\). Budeme možná překvapeni. Zvolme nejprve pevně, při kterých 
  konkrétních opakováních pokusu má dojít ke zdaru  (například při prvních \(x\)). Při ostatních 
  pak požadujeme nezdar. Protože jevy
  \begin{align*}
    A_1                &: \text{Při prvním opakování dojde ke zdaru.}                  \\
    A_2                &: \text{Při druhém opakování dojde ke zdaru.}                  \\
    \ldots             &: \ldots\ldots\ldots\ldots\ldots\ldots\ldots\ldots\ldots\ldots \\
    A_x                &: \text{Při \(x\)-tém opakování dojde ke zdaru.}               \\
    \overline{A}_{x+1} &: \text{Při \((x + 1)\)-tém opakování dojde k nezdaru.}        \\
    \ldots             &: \ldots\ldots\ldots\ldots\ldots\ldots\ldots\ldots\ldots\ldots \\
    \overline{A}_n     &: \text{Při posledním \(n\)-tém opakování dojde k nezdaru,}
  \normalsize
  \end{align*}
  jsou nezávislé, je pravděpodobnost jevu
  \begin{itemize}
    \item \(B_1\): Při každém z prvních \(x\) opakování dojde ke zdaru a současně při každém z 
          dalších \((n — x)\) opakování dojde k nezdaru, rovna součinu pravděpodobností
          \begin{equation*}
            p(B_1) = p(A_1)p(A_2)\cdots p(A_x)p(\overline{A}_{x+1})\cdots p(\overline{A}_n) 
                   = p^x (1 - p)^{n-x}.
          \end{equation*}
          Nám však jde o pravděpodobnost následujícího jevu
    \item \(B\): Právě při \(x\) opakováních pokusu (bez ohledu na to, kterých) dojde ke zdaru a 
          současně při každém ze zbývajících opakování pokusu dojde k nezdaru.
  \end{itemize}
\end{example}