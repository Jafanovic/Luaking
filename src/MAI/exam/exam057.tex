% !TeX spellcheck = cs_CZ
\wikitextrule
\begin{example}\label{mai:exam052}
  \textbf{Bernoulliův pokus}\newline\small
  Bernoulliův pokus spočívá v tom, že \(n\)-krát nezávisle provedeme určitý pokus, například hod 
  mincí. (V terminologii teorie pravděpodobnosti nazýváme každé takové provedení opakováním 
  pokusu.) Sledujeme, v kolika případech z těchto \(n\) opakování nastal daný jev (například jev 
  \(A\) — padne hlava). Výsledek opakování pokusu, při kterém daný jev nastal, nazveme zdarem, 
  výsledek, kdy nastal jev opačný, nezdarem. Dejme tomu, že pravděpodobnost zdaru je \(p\). (Pro 
  případ padnutí hlavy na minci je \(p = 1/2\).) Pravděpodobnost nezdaru je pak \((l - p)\).
  (V případě hodů mincí je \((1 — p) = 1/2\).) Zajímáme se o to, jaká je pravděpodobnost \(P(x)\), 
  že při \(n\) opakováních pokusu docílíme \(x\)-krát zdaru, \(x\) přitom můžeme předem volit 
  libovolně v rozmezí \(0 < x < n\). V 
  případě hodů
  mincí jistě dokážeme předem odhadnout, že pravděpodobnosti P (0) a P{n), tj. pravděpodobnosti 
  toho, že nepadne hlava vůbec nebo že padne hlava vždy, budou při větším počtu opakování pokusu 
  malé a budou se blížit
  nule tím více, čím větší bude n. Naopak bychom se mohli domnívat, že pravděpodobnost P (n /2), 
  tj. že padne
  hlava v polovině opakování pokusu, by měla být při velkém počtu n blízká 100%. Správnost tohoto 
  %našeho
  předběžného odhadu však posoudíme teprve poté, co si odvodíme obecný vzorec pro P(x). Budeme možná
  překvapeni. Zvolme nejprve pevně, při kterých konkrétních opakováních pokusu m á dojít ke zdaru 
  (například
  při prvních x). Při ostatních pak požadujeme nezdar. Protože jevy
  \normalsize
\end{example}