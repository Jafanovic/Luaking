% !TeX spellcheck = cs_CZ
\wikitextrule
\begin{example}\label{mai:exam057}
  \textbf{Bernoulliův pokus}\newline\small
  Bernoulliův pokus spočívá v tom, že \(n\)-krát nezávisle provedeme určitý pokus, například hod 
  mincí. (V terminologii teorie pravděpodobnosti nazýváme každé takové provedení opakováním 
  pokusu.) Sledujeme, v kolika případech z těchto \(n\) opakování nastal daný jev (například jev 
  \(A\) — padne hlava). Výsledek opakování pokusu, při kterém daný jev nastal, nazveme zdarem, 
  výsledek, kdy nastal jev opačný, nezdarem. Dejme tomu, že pravděpodobnost zdaru je \(p\). (Pro 
  případ padnutí hlavy na minci je \(p = 1/2\).) Pravděpodobnost nezdaru je pak \((l - p)\).
  (V případě hodů mincí je \((1 — p) = 1/2\).) Zajímáme se o to, jaká je pravděpodobnost \(P(x)\), 
  že při \(n\) opakováních pokusu docílíme \(x\)-krát zdaru, \(x\) přitom můžeme předem volit 
  libovolně v rozmezí \(0 \leq x \leq n\). V případě hodů mincí jistě dokážeme předem odhadnout, 
  že pravděpodobnosti \(P(0)\) a \(P(n)\), tj. pravděpodobnosti toho, že nepadne hlava vůbec nebo 
  že padne hlava vždy, budou při větším počtu opakování pokusu malé a budou se blížit nule tím 
  více, čím větší bude \(n\). Naopak bychom se mohli domnívat, že pravděpodobnost \(P(n/2)\), 
  tj. že padne hlava v polovině opakování pokusu, by měla být při velkém počtu \(n\) blízká 
  \SI{100}{\percent}. Správnost tohoto našeho předběžného odhadu však posoudíme teprve poté, co si 
  odvodíme obecný vzorec pro \(P(x)\). Budeme možná překvapeni. Zvolme nejprve pevně, při kterých 
  konkrétních opakováních pokusu má dojít ke zdaru  (například při prvních \(x\)). Při ostatních 
  pak požadujeme nezdar. Protože jevy
  \begin{align*}
    A_1                &: \text{Při prvním opakování dojde ke zdaru.}                  \\
    A_2                &: \text{Při druhém opakování dojde ke zdaru.}                  \\
    \ldots             &: \ldots\ldots\ldots\ldots\ldots\ldots\ldots\ldots\ldots\ldots \\
    A_x                &: \text{Při \(x\)-tém opakování dojde ke zdaru.}               \\
    \overline{A}_{x+1} &: \text{Při \((x + 1)\)-tém opakování dojde k nezdaru.}        \\
    \ldots             &: \ldots\ldots\ldots\ldots\ldots\ldots\ldots\ldots\ldots\ldots \\
    \overline{A}_n     &: \text{Při posledním \(n\)-tém opakování dojde k nezdaru,}
  \normalsize
  \end{align*}
  jsou nezávislé, je pravděpodobnost jevu
  \begin{itemize}
    \item \(B_1\): Při každém z prvních \(x\) opakování dojde ke zdaru a současně při každém z 
          dalších \((n — x)\) opakování dojde k nezdaru, rovna součinu pravděpodobností
          \begin{equation*}
            p(B_1) = p(A_1)p(A_2)\cdots p(A_x)p(\overline{A}_{x+1})\cdots p(\overline{A}_n) 
                   = p^x (1 - p)^{n-x}.
          \end{equation*}
          Nám však jde o pravděpodobnost následujícího jevu
    \item \(B\): Právě při \(x\) opakováních pokusu (bez ohledu na to, kterých) dojde ke zdaru a 
          současně při každém ze zbývajících opakování pokusu dojde k nezdaru.
  \end{itemize}
  
  Možností výběru \(x\) opakování, při kterých dojde ke zdaru, je \(N(x) = \begin{pmatrix} n \\ 
  x\end{pmatrix}\). Pokud bychom očíslovali jednotlivé výběry \(j = 1, 2, \cdots, N(x)\), dostaneme 
  odpovídající jevy \(B_1, \cdots, B_{N(x)}\) Pravděpodobnost každého z nich je stejná a rovna 
  pravděpodobnosti jevu \(B_1\), který jsme popsali před chvílí. Tyto jevy jsou po dvou 
  neslučitelné a jev \(B\) znamená, že nastane právě jeden (kterýkoli) z nich. Pro jeho 
  pravděpodobnost tedy platí, podle pravidla pro součet pravděpodobností po dvou neslučitelných 
  jevů,
  \adjustbox{}{%
    \begin{equation}\label{mai:eq055}
      p(B) = P(x) = \begin{pmatrix} n \\ x\end{pmatrix}p^x (1 - p)^{n-x}.
    \end{equation}
    }
   
  Zkusme nyní prověřit správnost našeho odhadu týkajícího se hodů mincí:
  \begin{equation*}
    P(0) = \begin{pmatrix} n \\ 0\end{pmatrix} 
           \left(\dfrac{1}{2}\right)^0\left(\dfrac{1}{2}\right)^{n-0} 
         = \dfrac{1}{2^n}            \qquad
    P(n) = \begin{pmatrix} n \\ n\end{pmatrix} 
           \left(\dfrac{1}{2}\right)^n\left(\dfrac{1}{2}\right)^{n-n} 
         = \dfrac{1}{2^n}     
  \end{equation*}
  Vidíme, že náš odhad byl správný. Obě pravděpodobnosti klesají s rostoucím počtem opakování 
  pokusu k nule.  Pro jediné opakování pokusu, tj. \(n = 1\), jsou obě rovny jedné polovině, a to 
  bychom jistě také měli očekávat. 
  
  Pro \(n\) sudé nyní počítejme \(P(n/2)\). Položme \(n = 2m\):
  \begin{equation*}
    P(m) = \begin{pmatrix} 2m \\ m\end{pmatrix} 
           \left(\dfrac{1}{2}\right)^m\left(\dfrac{1}{2}\right)^{2m-m} 
         = \dfrac{(2m)!}{m!m!}\left(\dfrac{1}{2}\right)^{2m}     
  \end{equation*}

  \begin{table}[h]
    \centering
    \begin{tabular}{c|rrrrr}
      \(m\)    & 1 & 2 & 3 & 5 & 10  \\ \hline
      \(P(m)\) & \num{0.500} & \num{0.375} & \num{0.313} & \num{0.246} & \num{0.176}
    \end{tabular}
    % \caption{ }
  \end{table}
  
  Tady se zdá, že nás naše intuice při odhadu pravděpodobnosti \(P(n/2)\) zklamala. Tendence hodnot 
  \(P(n/2)\) je pro rostoucí \(n\) klesající. Pravděpodobnost je největší pro \(n = 2\), a to právě 
  padesátiprocentní! Zkusme ještě odhad pro velká \(n\) pomocí \textbf{Stirlingova vzorce}. Podle 
  něj pro velká \(n\) platí
  
  \begin{equation}\label{mai:eq056}
    n! = \left(\dfrac{n}{e}\right)^n\sqrt{2\pi n}
  \end{equation}
  Použijeme-li jej pro výpočet P(m), dostáváme
  \begin{equation*}
    P(m)\simeq \dfrac{\left(\dfrac{2m}{e}\right)^{2m}\sqrt{4\pi m}}
     {\left(\dfrac{m}{e}\right)^m\left(\dfrac{m}{e}\right)^m\left(\sqrt{2\pi m}\right)^2}
     \left(\dfrac{1}{2}\right)^{2m} = \dfrac{1}{\sqrt{\pi m}} \longrightarrow 0
  \end{equation*}
  pro velká \(m\). Kde jsme se tedy zmýlili? Ze zkušenosti víme, že budeme-li házet mincí 
  mnohokrát, je prakticky jisté, že hlava skutečně padne zhruba v polovině případů! Problém spočívá 
  ve slovíčku zhruba. Pravděpodobnost \(P(m)\) pro \(n = 2m\) se však týká jevu, kdy hlava padne 
  přesně v polovině případů. A ta samozřejmě bude tím menší, čím větší je počet posuzovaných hodů 
  mincí. Při zvyšujícím se počtu \(n\) opakování pokusu totiž roste i počet jednotlivých možností 
  volby \(x\) a \(n\) a každou z nich tak „připadne“ menší pravděpodobnost. (Součet  
  pravděpodobností přes všechna přípustná \(x\) musí být roven jedné.) Později, v odstavci 
  \ref{mai:IchapIIIsecII}, uvidíme, že jsme nevědomky místo pravděpodobnosti odhadovali střední 
  hodnotu náhodné veličiny.
  
  Položme si ještě poslední otázku v souvislosti s Bernoulliovým pokusem: Jaká je pravděpodobnost, 
  že alespoň při jednom z \(n\) opakování pokusu nastane zdar? Pokud si po předchozím neúspěchu s 
  intuitivními odhady ještě trochu věříme, můžeme předpovídat, že tato pravděpodobnost poroste s 
  počtem opakování pokusu \(n\) a pro velmi velká \(n\) se bude blížit jedné. Musíme ji ale 
  spočítat. Někdo, kdo nečetl předchozí text příliš pečlivě, by mohl navrhnout jednoduchou úvahu: 
  Pravděpodobnost zdaru při každém opakování pokusu je \(p\), pravděpodobnost, že nastane zdar při 
  alespoň jednom z nich tedy musí být, podle pravidla pro sčítání pravděpodobností, \(np\).
  Úvaha je sice jednoduchá, ale zcela chybná. Vidíme to již ze skutečnosti, že při pevné hodnotě 
  \(p\) a dostatečně velkém \(n\) může hodnota \(np\) překročit jedničku, a to nemůže žádná 
  pravděpodobnost udělat. Kde se málo pozorný čtenář dopustil chyby, když chtěl sčítat 
  pravděpodobnosti zdaru při jednotlivých opakováních? Neuvědomil si, že pravidlo součtu 
  pravděpodobností jednotlivých jevů \(A_1\) až \(A_k\) při výpočtu pravděpodobnosti jevu 
  (\(A_1\) nebo \(A_2\) nebo \(\cdots\) nebo \(A_k\)) může použít jedině pro jevy po dvou 
  neslučitelné. Zdar při některém z opakování pokusu však nevylučuje možnost zdaru při jiném 
  pokusu. Pravidlo tedy bylo použito nesprávně. Pravděpodobnost zdaru při alespoň jednom opakování 
  pokusu snadno vypočteme pomocí jevu opačného. Opačný jev znamená, že nenastane zdar ani při 
  jednom opakování pokusu. Jednotlivá opakování jsou nezávislá, proto je pravděpodobnost nezdarů
  při všech opakováních rovna součinu pravděpodobností při jednotlivých z nich, tj. \((1 - p)^n\) . 
  Pravděpodobnost zdaru při alespoň jednom opakováni je pak doplňkem do jedničky, tedy \(1 - (1 - 
  p)^n\). Je vidět, že je tím větší, čím je větší \(n\), a její limita pro \(n\rightarrow \infty\) 
  je rovna jedné. A to je výsledek, který jsme předpověděli. 

\normalsize
\end{example}