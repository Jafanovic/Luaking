% !TeX spellcheck = cs_CZ
\begin{mdframed}[style=mdexam]
  \begin{example}\label{mai:exam057}
    \textbf{Návrh zákona}\newline
    S připomínkami k navrhovanému zákonu chce v parlamentě vystoupit šest poslanců \(A\), \(B\),
    \(C\), \(D\), \(E\), \(F\). Určete počet: 
    \begin{enumerate}[leftmargin=2em,rightmargin=1em, label=\emph{\alph*}),noitemsep]
      \item všech možných pořadí jejich vystoupení;
      \item všech pořadí, v nichž vystupuje \(A\) po \(E\);
      \item všech peřadí, v nichž vystupuje \(A\) ihned po \(E\).
    \end{enumerate} \vspace{0.5em}
    \textbf{Řešení} \vspace{0.5em}

    \begin{enumerate}[leftmargin=2em, label=\emph{\alph*}),noitemsep]
      \item Jde o permutace ze sešti prvků, takže všech možných pořadí vystoupení jednotlivých
            poslanců je \(P(6) = 6! = 720\).
      \item Každému pořadí, v němž vystupuje \(A\) po \(E\), lze přiřadit jediné pořadí, v němž
            vystupuje \(E\) po \(A\), a. také obráceně. Pořadí těchto dvou druhů je tedy stejný
            počet, a protože jiná neexistují, je počet všech pořadí, v nichž vystupuje \(A\) po
            \(E\), roven
            \begin{equation*}
              \dfrac{1}{2}P(6) = \dfrac{1}{2}6! = 360
            \end{equation*}      
      \item Tento případ se od b) odlišuje tím, že po projevu poslance \(E\) násleleduje projev
            poslanece \(A\) ihned, takže si oba projevy poslanců \(E\), \(A\) můžeme myslet jako
            jediný projev hypotetického poslance \(EA\). Počet všech pořadí, v nichž vystupuje \(A\)
            ihned po \(E\), je tedy roven počtu všech permutací z pěti prvků \(B\), \(C\), \(D\),
            \(F\), \(EA\), takže máme výsledek: \(P(5) = 5! = 120\).
    \end{enumerate}
  \end{example}
\end{mdframed}