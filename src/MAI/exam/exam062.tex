% !TeX spellcheck = cs_CZ
\wikitextrule
\begin{example}\label{mai:exam062}
  \textbf{Pohádka o Honzovi}\newline\small
  Honza se vydal ke zlému černokněžníkovi vysvobodit princeznu. Černokněžník se zachechtal a 
  pravil: „Princezna je za jedním z těchto tří závěsů. Uhodneš-li, za kterým, můžeš si ji odvést. 
  Ne-li, zkameníš.“ Honza tentokrát neměl na pomoc hodné mravenečky ani mazanou lišku Ryšku, a tak 
  se rozhodl použít svých znalostí o pravděpodobnosti. Když zjistil, že mu nepomohou, protože 
  pravděpodobnosti, že princezna je za jednotlivými závěsy, jsou stejné a rovny jedné třetině, 
  zvolil závěs A. Černokněžníka napadlo, že se na Honzovy znalosti o pravděpodobnostech podívá 
  lépe, a aniž závěs A odhrnul, řekl: „Mám už na zahradě kamení dost, a proto ti dám jednu 
  nápovědu. Ze dvou zbývajících závěsů, B a C, odhrnu ten, za kterým princezna není.“ A odhrnul 
  závěs B. Princezna tam opravdu nebyla. Černokněžník řekl: „Teď se teprve rozhodni, budeš-li trvat 
  na závěsu A, nebo změníš své rozhodnutí a označíš C.“ Honza přemýšlel, až se mu z hlavy kouřilo, 
  a snažil se určit pravděpodobnosti, že princezna je za závěsem A, resp. C. Napadly ho dvě úvahy, 
  které vypadaly docela dobře, ale vedly k různým výsledkům:
  \begin{itemize}
  \item Úvaha prvá: Za závěsem B princezna není a závěsy A a C jsou rovnocenné. Pravděpodobnost, že 
        je moje vyvolená ze kterýmkoli z nich, je proto \num{0.5}. Černokněžník mi tedy nijak 
        nepomohl. (Nic jiného se od něj taky čekat nedalo.)
  \item Úvaha druhá: Pravděpodobnost, že princezna je za závěsem A, který jsem předem zvolil, byla 
        rovna jedné třetině. Tím, že černokněžník odkryl závěs B, se však nemohla změnit. Proto 
        pravděpodobnost, že najdu princeznu za závěsem C, je nyní rovna dvěma třetinám.
  \end{itemize}
  
  Poradíme Honzovi, který ze závěsů A a C má zvolit? Vzpomeneme-li si na to, co jsme před chvílí 
  zjistili o podmíněných pravděpodobnostech, určitě mu poradit můžeme: Označme jako \(A\), \(B\), 
  \(C\) jevy
  \begin{itemize}
    \item Jev \(A\) : Princezna je za závěsem A.
    \item Jev \(B\) : Princezna je za závěsem B.
    \item Jev \(C\) : Princezna je za závěsem C.
  \end{itemize}
  Jejich pravděpodobnosti jsou
  \begin{equation*}
    p(A) = \dfrac{1}{3}, \qquad p(B) = \dfrac{1}{3}, \qquad p(C) = \dfrac{1}{3}.
  \end{equation*}
  Víme, že Honza vybral závěs A, černokněžník tedy může odhrnout závěs B nebo C. Označme jako 
  \(B'\) a \(C'\) tyto jevy.
  \begin{itemize}
    \item Jev \(B'\) : Černokněžník odhrne závěs B.
    \item Jev \(C'\) : Černokněžník odhrne závěs C.
  \end{itemize}
  Pravděpodobnosti jevů \(B'\) i \(C'\) jsou shodné a rovny jedné polovině, tj. \(p(B') = p(C') = 
  1/2\) . Podle vztahu (\ref{mai:eq057}) platí
  \begin{align*}
    p(A\text{ a }B') &= p(B')\cdot p_{B'}(A) = p(A)\cdot p_{A}(B') 
                      = \dfrac{1}{3}\cdot\dfrac{1}{2} = \dfrac{1}{6},  \\
    p(A\text{ a }C') &= p(C')\cdot p_{C'}(A) = p(A)\cdot p_{A}(C')
                      = \dfrac{1}{3}\cdot\dfrac{1}{2} = \dfrac{1}{6},  \\
    \shortintertext{Obdodně je}
    p(C\text{ a }B') &= p(B')\cdot p_{B'}(C) = p(C)\cdot p_{C}(B')
                      = \dfrac{1}{3}\cdot1 = \dfrac{1}{3},              \\
    p(C\text{ a }C') &= p(C')\cdot p_{C'}(C) = p(C)\cdot p_{C}(C') = 0  \\
  \end{align*}
  Použijeme-li těchto výsledků, dostaneme podmíněné pravděpodobnosti, že princezna je za závěsem A, 
  resp. C za podmínky, že černokněžník odhrnul B:
  \begin{equation*}
    p_{B'}(A) = \dfrac{p(A\text{ a }B')}{p_{B'}} = \dfrac{1}{6}:\dfrac{1}{2} = \dfrac{1}{3},  \qquad
    p_{B'}(C) = \dfrac{p(C\text{ a }B')}{p_{B'}} = \dfrac{1}{6}:\dfrac{1}{2} = \dfrac{1}{3}    
  \end{equation*}
  Správná je tedy druhá Honzova úvaha. Můžeme ještě provést její kontrolu pomocí podmíněných 
  pravděpodobností: Pro jev (\(A\) a \(B'\)) nebo (\(A\) a \(C'\)), utvořený ze dvou neslučitelných 
  jevů pomocí spojky „nebo“, je tedy
  \begin{equation*}
    p\left((A\text{ a }B')\text{ nebo }(A\text{ a }C')\right) = \dfrac{1}{6} + \dfrac{1}{6} 
      = \dfrac{1}{3}.
  \end{equation*}
  Poslední výsledek představuje skutečnost, že princezna je za závěsem A stále s pravděpodobností 
  rovnou jedné třetině, ať se černokněžník chystá odhrnout kterýkoli ze zbývajících závěsů.
\normalsize
\end{example}