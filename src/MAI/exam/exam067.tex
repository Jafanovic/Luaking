% !TeX spellcheck = cs_CZ
\wikitextrule
\begin{example}\label{mai:exam067}
  \textbf{Ještě střelba}\newline\small
  Problém s definici \(P\)-kvantilu u veličiny s diskrétním rozdělením snadno vidíme na příkladu 
  střelby (příklad \ref{mai:exam064}). Pro \(s\) postupně 1, 2, 3, 4 nabývá součet na levé straně 
  rovnice (\ref{mai:eq064}) hodnot
  
  \begin{equation*}
    p_1 = \num{0.003}, p_1 + p_2 = \num{0.31},\qquad p_1 + p_2 + p_3 = \num{0.83},\qquad 
    p_1 + p_2 + p_3 + p_4 = 1.
  \end{equation*}
  Pojem \(P\)-kvantil je tedy definován jen pro \(P = 0\), \(P = \num{0.03}\), \(P = \num{0.31}\), 
  \(P = \num{0.83}\) a \(P = 1\). (Pro \(P = 0\) a \(P = 1\) nemá žádný praktický význam.) 
  Nenabývá-li \(P\) žádné z přípustných hodnot, tj. některé hodnoty z množiny \(\{\num{0.03}, 
  \num{0.31}, \num{0.83}, 1\}\), nemá rovnice pro s řešení a \(P\)-kvantil není vůbec definován. 
  Je-li hodnotou \(P\) některý prvek této množiny, dostaneme z rovnice (\ref{mai:eq064}) sice 
  jediné řešení \(s\), avšak která hodnota bude \(P\)-kvantilem? Z grafu je vidět, že pro každou 
  přípustnou hodnotu \(P\) vyhovuje podmínce celý interval proměnné \(x\). Konkrétní výsledky 
  shrnuje následující tabulka:
  
  \begin{table}[h]
    \centering
    \begin{tabular}{c|crrrr}
      pravděpodobnost \(P\)  & \num{0} & \num{0.03} &  \num{0.31} &  \num{0.83} & \num{1}  \\ \hline
      interval řešení rovnice \(F(x) = P\) & \((-\infty,\num{0})\) & \(\left[0, 1\right)\) & 
      \(\left[1, 2\right)\) & \(\left[2, 3\right)\) & \(\left[3, \infty\right)\)
    \end{tabular}
    % \caption{ }
  \end{table}
  
  Význam pojmu \(P\)-kvantil je tedy pro náhodnou veličinu s diskrétním rozdělením poněkud sporný. 
  Uplatní se však velmi dobře u veličin s rozdělením spojitým, jak uvidíme později. Než však 
  opustíme příklad se střelbou definitivně, spočtěme si ještě střední hodnotu a rozptyl veličiny 
  \(X\), kterou jsme definovali jako počet dosažených bodů při jednom výstřelu:
  \begin{equation*}
    \langle x \rangle = \sum_{j=1}^{4}x_jp_j = 0\cdot\num{0.03} + 1\cdot\num{0.28} 
     + 2\cdot\num{0.52} + 3\cdot\num{0.17} = \num{1.83}, 
  \end{equation*}
  \begin{align*}
    D(X)  = \sum_{j=1}^{4}\left(x_j - \langle x \rangle \right)^2p_j 
         &= (0 - \num{1.83})^2\cdot\num{0.03} + (1 - \num{1.83})^2\cdot\num{0.28}           \\
         &+ (2 - \num{1.83})^2\cdot\num{0.52} + (3 - \num{1.83})^2\cdot\num{0.17} \simeq\num{0.541},
  \end{align*}
  \begin{equation*}
    \sigma(x) \simeq \num{0.736}.
  \end{equation*}
  V příkladu \ref{mai:exam064} jsme odhadovali, kolika bodů dosáhne střelec při pěti výstřelech. 
  Tato hodnota nám vyšla \(\num{5}\cdot\num{1.83} = \num{9.15} = 9\). Nyní vidíme je jí souvislost 
  se střední hodnotou náhodné veličiny \(X\). Pokud totiž definujeme veličinu \(Y\) jako počet bodů 
  dosažených při pěti výstřelech, je \(Y = 5X\) a \(\langle y \rangle = 5\langle x \rangle\). 
  Uvažujme nyní o významu směrodatné odchylky. Zřejmě \(\sigma(y) = 5\sigma(x) = \num{3.68}\). 
  Směrodatná odchylka \(\sigma(y)\) určuje interval \((\langle y \rangle - \sigma(y), \langle y 
  \rangle + \sigma(y)) = (\num{5.32}, \num{12.68})\). Možnosti bodového zisku ležící v tomto 
  intervalu jsou \num{6} až \num{12} bodů včetně. Pokud bychom doplnili tabulku z příkladu 
  \ref{mai:exam064} ještě o rozklady a jejich pravděpodobnosti pro bodový součet při pěti
  výstřelech \(j = \num{6}\) a \(j = \num{12}\), dostaneme \(p(A_6) = \num{0.0400}\), \(p(A_{12}) = 
  \num{0.0550}\). Pravděpodobnost, že výsledek střelce leží při pěti výstřelech v intervalu 
  \((\langle y \rangle - \sigma(y), \langle y \rangle + \sigma(y)) = (\num{5.32}, \num{12.68})\), 
  je tedy
  \begin{equation*}
    \sum_{j=6}^{12} = p(A_j) = p(A_6) + \sum_{j=7}^{11}p(A_j) + p(A_{12}) 
                    = \num{0.0400} + \num{0.873} + \num{0.0550} \simeq \num{0.97}.
  \end{equation*}
  Při výpočtu jsme využili výsledku z příkladu \ref{mai:exam064}, kde jsme počítali 
  pravděpodobnost, že střelec dosáhne bodového výsledku v rozmezí \num{7} až \num{11} bodů. 
  Směrodatná odchylka \(\sigma(y)\) veličiny \(Y\) určuje tedy v tomto případě interval okolo 
  střední hodnoty \(\langle y \rangle\), v němž leží střelcův bodový zisk s velmi vysokou 
  pravděpodobností \SI{97}{\percent}. Tento výsledek lze velmi názorně interpretovat také takto: 
  Vystřelí-li střelec pětkrát na terč, bude téměř s jistotou jeho bodový zisk ležet v intervalu 
  určeném směrodatnou odchylkou, tj. bude ležet mezi šesti a dvanácti body. Není vyloučeno, že 
  bodový zisk bude třeba pět bodů, nebo i nula, nebo naopak dokonce maximálních možných patnáct 
  bodů. Všechny ty to možnosti dohromady jsou však vysoce nepravděpodobné, připadá na ně
  pravděpodobnost pouhé \SI{3}{\percent}!. Anebo ještě trochu jinak: Kdyby střelec při tréninku 
  uskutečnil třeba sto sérií po pěti výstřelech, pak by skoro jistě bylo sedmadevadesát z nich v 
  rozmezí bodového zisku \num{6} až \num{12} bodů a tři mimo. Toto konstatování ovšem opět 
  nevylučuje možnost, že v rozmezí \num{6} až \num{12} bodů bude ležet jiný počet sérií
  než \num{97}. Může dokonce v principu dojít k tomu, že do této kategorie padnou série všechny 
  nebo žádná. Takový výsledek je však opět vysoce nepravděpodobný.
  
  Není vyloučeno, že i po prostudování tohoto příkladu bude někdo stále nespokojen s tím, že je 
  naše vyjadřování „málo přesné“. Vzhledem k pravděpodobnostnímu charakteru posuzovaných jevů však, 
  bohužel, přesnější být nemůže.
\normalsize
\end{example}