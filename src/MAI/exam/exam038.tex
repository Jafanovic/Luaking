% !TeX spellcheck = cs_CZ
% \wikitextrule
\begin{mdframed}[style=mdexam]
  \begin{example}\label{mai:exam038}
    \textbf{Gaussova eliminační metoda}\newline
    Je zadána soustava tří (\(m = 3\)) rovnic o pěti (\(n = 5\)) neznámých:
    \begin{alignat*}{7}
        x_1 &+ 2x_2 &-  x_3 &+  x_4 &- 5x_5 &=  &&0, \\
      -2x_1 &- 4x_2 &+ 2x_3 &+ 4x_4 &+ 4x_5 &= -&&6, \\
       -x_1 &- 2x_2 &+  x_3 &+ 5x_4 &-  x_5 &=  &&6.
    \end{alignat*}
    Budeme provádět ekvivalentní úpravy matice \(\matr{B}\) tak, abychom ji zjednodušovali.
    Ekvivalenci matic budeme označovat znakem \(\sim\). 
    \begingroup
      \renewcommand\arraystretch{1.0}
      \renewcommand\arraycolsep{3pt}
      \begin{equation*}
        \matr{B} = (\matr{A}\lvert\overline{\matr{B}}) =
        \left(
          \begin{array}{rrrrr|r}
             1 &  2 & -1 & 1 & -5 &  0    \\
            -2 & -4 &  2 & 4 &  4 & -6    \\
            -1 & -2 &  1 & 5 & -1 &  6
          \end{array}
        \right).
      \end{equation*}
    \endgroup
    V prvním řádku je na první pozici jednička. Toho využijeme k snadné „likvidaci“, tedy eliminaci,
    prvních prvků v druhém a třetím řádku pomocí elementárních úprav. Provedeme tyto dvě úpravy:
    první řádek vynásobený číslem \num{2} přičteme k druhému řádku, první řádek přičteme ke třetímu
    řádku. Dostaneme
    \begingroup
      \renewcommand\arraystretch{1.0}
      \renewcommand\arraycolsep{3pt}
      \begin{equation*}
        \matr{B} \sim
        \left(
          \begin{array}{rrrrr|r}
            1 &  2 & -1 & 1 & -5 &  0         \\
            \bm{0} &  0 &  0 & 6 & -6 & -6    \\
            \bm{0} &  0 &  0 & 6 & -6 &  6
          \end{array}
        \right).
      \end{equation*}
    \endgroup  
    Vidíme, že jsme v druhém i třetím řádku dostali na první sloupcové pozici nulu (tučná). (Nuly na
    dalších  pozicích vznikly náhodou, vlivem jednoduchosti zadání.) Nyní vynásobíme druhý i třetí
    řádek číslem (\num{1/6}):
    \begingroup
      \renewcommand\arraystretch{1.0}
      \renewcommand\arraycolsep{3pt}    
      \begin{equation*}
        \matr{B} \sim
        \left(
          \begin{array}{rrrrr|r}
                  1 &  2 & -1 & 1 & -5 &  0    \\
                  0 &  0 &  0 & 1 & -1 & -1    \\
                  0 &  0 &  0 & 1 & -1 &  1
          \end{array}
        \right).
      \end{equation*}
    \endgroup  
    Přestože již nyní vidíme, že soustava nemá řešení (rovnice druhého a třetího řádku znějí \(x_4 -
    x_5 = - 1\) a \(x_4 -x_5 = 1\), takže jim nevyhovuje žádná dvojice čísel \((x_4, x_5)\)), budeme
    v eliminaci pokračovat. Další úpravy se již týkají pouze druhého a třetího řádku. Druhý řádek
    vynásobený (\num{-1}) přičteme ke třetímu. Pak

    \begingroup
    \renewcommand\arraystretch{1.0}
    \renewcommand\arraycolsep{3pt}
      \begin{equation*}
        \matr{B} \sim
        \left(
          \begin{array}{rrrrr|r}
                  1 &  2 & -1 & 1      & -5 &  0    \\
                  0 &  0 &  0 & 1      & -1 & -1    \\
                  0 &  0 &  0 & \bm{0} &  0 &  2
          \end{array}
        \right).
      \end{equation*}
    \endgroup  
    Získáváme tak ekvivalentní soustavu rovnic
    \begin{alignat*}{5}
          x_1 + 2x_2 - x_3 &+  x_4 &- 5x_5 &=  &&0, \\
                            &+  x_4 &-  x_5 &= -&&1, \\
                            &       &     0 &=  &&2.
    \end{alignat*}
    V poslední rovnici je ihned vidět rozpor - soustava nemá řešení
  \end{example}
\end{mdframed}