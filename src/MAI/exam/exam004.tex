% !TeX spellcheck = cs_CZ
% \wikitextrule
\begin{mdframed}[style=mdexam]
  \begin{example}\label{mai:exam004}
    \textbf{Parametrická vyjádření roviny}:\newline Rovina v trojrozměrném prostoru \(\mathbb{R}^3\)
    je zadána třemi body \(A\), \(B\) a \(C\), které nesmějí ležet v jedné přímce, popřípadě dvěma
    body \(A\) a \(B\) a vektorem \(\vec{v}\) nerovnoběžným s \(\overrightarrow{AB}\), anebo bodem
    \(A\) a dvěma nerovnoběžnými směrovými vektory \(\vec{u}\) a \(\vec{v}\) (obr.
    \ref{MAI:FIG002}). Všechny tyto typy zadání jsou ekvivalentní. Lze volit například \(\vec{u} =
    \overrightarrow{AB}\), \(\vec{v} = \overrightarrow{AC}\). Je-li \(X\) libovolným bodem roviny
    \(\varrho\), jsou vektory \(\overrightarrow{AX}\), \(\vec{u}\) a \(\vec{v}\) \textbf{lineárně
    závislé}. To znamená, že existují taková reálná čísla \(r\) a \(s\), že vektor
    \(\overrightarrow{AX}\) lze zapsat jako lineární kombinaci

    \begin{equation*}
      \overrightarrow{AX} = r\cdot\vec{u} + s\cdot\vec{v}, \qquad r,s \in\mathbb{R}
    \end{equation*}
    Při obdobném zápisu kartézských souřadnic bodů a složek vektorů jako u vyjádření přímky
    dostaneme parametrické vyjádření roviny \(\varrho\)
    \begin{equation*}
      \varrho = \left\{
      \begin{matrix}  
        (x,y,z)\in\mathbb{R}^3  \\
        r, s \in\mathbb{R}
      \end{matrix}
      \,\left\lvert\,
      \begin{matrix}
        x = x_A + ru_1 + sv_1,        \\
        y = y_A + ru_2 + sv_2,        \\
        z = z_A + ru_3 + sv_3,
      \end{matrix}\right.          
      \right\}.
    \end{equation*}

    { \centering
      \captionsetup{type=figure}
      \luafigure[1]{mai_fig026.pdf}
      \captionof{figure}{Zadání roviny. \cite[s.~3]{Musilova2009MA1}
      \label{MAI:FIG002}}
    \par}

    Toto vyjádření obsahuje opět lineární závislost: Souřadnice \(x\), \(y\) a \(z\) se vůči
    souřadnicím bodu \(A\) mění v závislosti na prvních mocninách parametrů \(r\) a \(s\). Můžeme
    tak hovořit o jakési „vícerozměrné úměře“.
  \end{example}
\end{mdframed}
  