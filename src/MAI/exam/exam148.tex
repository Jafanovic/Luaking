\begin{mathexam}{\(\int(x^3+2)^3\dd{x}\)}{exam148} 
  Umocněním výrazu v závorce převedeme funkci \((x^3+2)^3\) na součet čtyř funkcí, které mají
  známe primitivní funkce \[(x^3+2)^3 = x^9+6x^6+12x^3+8,\]. Dostaneme tak sadu integrálů
  \begin{multline*}
    \int x^9\dd{x} + 6\int x^6\dd{x} + 12\int x^3\dd{x} + 8\int\dd{x} =     \\
      = \dfrac{1}{10}x^{10} + \dfrac{6}{7}x^{7} + 3x^4 + 8x + c  
  \end{multline*}
\end{mathexam}