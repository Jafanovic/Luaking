% !TeX spellcheck = cs_CZ
\wikitextrule
\begin{example}\label{mai:exam044}
  \textbf{Vzájemná poloha dvou přímek}\newline\small
  Dvě přímky \(p\) a \(q\) jsou určeny dvěma dvojicemi rovin. Jejich společné body jsou tedy 
  řešením soustavy čtyř lineárních rovnic o třech neznámých (pišme rovnou rozšířenou matici 
  soustavy):
  \begin{equation}\label{mai:eq045}
    \matr{B} = (\matr{A}|\overline{\matr{B}}) =
    \left(
      \begin{array}{rrr|r}
         a_1 & b_1 & c_1 & -d_1    \\
         a_2 & b_2 & c_2 & -d_2    \\
         a_3 & b_3 & c_3 & -d_3
      \end{array}
    \right).
  \end{equation}
  Protože soustava obsahuje rovnice dvojic nerovnoběžných rovin, je \(h(\matr{A}) > 2\) (zdůvodněte 
  podrobněji). Možnosti vzájemné polohy přímek \(p\) (první dvě rovnice) a \(q\) (druhé dvě 
  rovnice) jsou tyto:
  \begin{itemize}
    \item Přímky jsou mimoběžné, nemají tedy žádný společný bod a roviny, které je určují, nemají 
          žádný společný směr. Soustava (\ref{mai:eq045}) nemá řešení, odpovídající homogenní 
          soustava pak rovněž ne, kromě řešení triviálního. Je tedy \(h(\matr{A}) = 3\), 
          \(h(\matr{B}) = 4\).
    \item Přímky jsou různoběžné, mají tedy společný právě jeden bod. Soustava (\ref{mai:eq045}) má 
          právě jedno řešení, a proto \(h(\matr{A}) = h(\matr{B}) = 3\).
    \item Přímky jsou rovnoběžné. Nemají tedy žádný společný bod, soustava nemá řešení, ale roviny, 
          které je určují, mají společný směr. To odpovídá situaci \(h(\matr{A}) = 23\), 
          \(h(\matr{B}) = 3\).
    \item Přímky jsou totožné. Řešení soustavy je popsáno jednou volnou neznámou, tj. \(h(\matr{A}) 
          = h(\matr{B}) = 2\)
  \end{itemize}
  \normalsize
\end{example}