% !TeX spellcheck = cs_CZ
\wikitextrule
\begin{example}\label{mai:exam060}
  \textbf{Ještě jednou bomba v letadle}\newline\small
  V úvodu odstavce jsme uvažovali o pravděpodobnosti dvou bomb v letadle jako o pravděpodobnosti 
  současného nastoupení dvou nezávislých jevů s komentářem, že tato úvaha není tak docela v 
  pořádku. Někdo je možná zvědavý, proč, a tak se tomuto problému budeme ještě chvíli věnovat. (Kdo 
  zvědavý není, může příklad přeskočit.)
  
  Nebudeme nyní posuzovat situaci, kdy jsme do letadla přinesli bombu my sami. Zabývejme se 
  přesnější odpovědí na otázku, jaká je pravděpodobnost, že v letadle budou bomby dvě, aniž bychom 
  tomu sami napomáhali. Taková situace odpovídá Bernoulliovu pokusu. Samozřejmě, je třeba udělat 
  jisté předpoklady, které nemusejí být zcela realistické, ale v průměru budou fungovat. 
  Předpokládejme, že v letadle je \(n\) pasažérů, kteří se nijak neliší pokud jde o sklon „vzít 
  bombu do letadla“. Pravděpodobnost, že daný pasažér vezme s sebou bombu, je tedy u všech stejná a 
  označme ji \(p\). (To je právě ten předpoklad, který u jednotlivce není příliš realistický, neboť 
  venkovská tetička jistě nemá takové nutkání vzít si spolu s husou do košíku bombu, jako 
  fanatický terorista.) Hodnota \(p\) zde tedy představuje jistou „zprůměrovanou zkušenost“. Co 
  přesně znamená otázka, jaká je pravděpodobnost, že v letadle je bomba? Je tím myšlena 
  pravděpodobnost jevu „V letadle je alespoň jedna bomba.“ Pravděpodobnost \(P\) tohoto jevu jsme 
  zadali jako jednu tisícinu (dejme tomu, že je to zase údaj odpovídající „zprůměrované zkušenosti“ 
  u letadel s velkým počtem cestujících). Také jsme již \(P\) počítali v závěru příkladu 
  \ref{mai:exam056}. Platí pro ni
  \begin{equation*}
    P = 1 - (1 - p)^n,
  \end{equation*}
  kde \(n\) je počet opakování pokusu. V našem případě nastoupení jednotlivého pasažéra do letadla 
  představuje jedno opakování pokusu, takže je tento počet roven počtu pasažérů v letadle. Můžeme 
  tedy určit pravděpodobnost \(p\) týkající se jednotlivého pasažéra,
  \begin{equation*}
    p = 1 - \sqrt[n]{1 - P}
  \end{equation*}
  Nyní potřebujeme znát pravděpodobnost, že v letadle jsou dvě bomby, myšleno alespoň dvě bomby. 
  Označme tento jev jako \(B\). Znamená, že alespoň při dvou opakováních Bernoulliova pokusu 
  nastane zdar. Jev \(\overline{B}\) k němu opačný znamená, že nastanou buď samé nezdary 
  (pravděpodobnost je \((1 — p)^n\) a odpovídá hodnotě \(x = 0\) ve vzorci (\ref{mai:eq055})), nebo 
  nastane právě \((n — 1)\) nezdarů a jeden zdar (pravděpodobnost je \(np(l — p)^{n-1}\) a odpovídá 
  hodnotě \(x = 1\) ve vzorci (\ref{mai:eq055})). Výsledky Bernoulliova pokusu pro různá \(x\) se 
  ovšem navzájem vylučují, takže pravděpodobnost jevu \(\overline{B}\) je
  \begin{equation*}
    p(\overline{B}) = (1 - p)^n + np(1 - p)^{n-1}.
  \end{equation*}
  Pravděpodobnost alespoň dvou bomb v letadle (posuzovaný jev \(B\)) je pak
  \begin{equation*}
    p(B) = 1 - (1 - p)^n - np(1 - p)^{n-1}.
  \end{equation*}
  Zbývá dosadit za \(p\) pomocí známé hodnoty \(P\). Je-li \(P\) velmi malé, lze získat přibližný 
  výsledek pomocí odhadů. Vzpomeneme-li si na odhady pomocí diferenciálu v odstavci 2.2.3, 
  zjistíme, že pro hodnoty \(P\) mnohonásobně menší než \(1\) (a to je i náš případ) dostaneme
  \begin{equation*}
    p \simeq 1 - \left(1 - \dfrac{1}{n}P\right) = \dfrac{P}{n}.
  \end{equation*}
  Obdobně provedeme odhad pro \(p(B)\),
  \begin{equation*}
    p(B) = 1 - (1 - np) - np\left[1 - (n - 1)p\right] = n(n - 1)p^2 \simeq \dfrac{n-1}{n}P^2 
         \simeq P^2 = \num{e-6}.
  \end{equation*}
\normalsize
\end{example}