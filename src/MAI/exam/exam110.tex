\begin{mdframed}[style=mdexam]
  \begin{example}\label{MAI:exam110}
    Jak poznat kandidáta na substituční metodu I. Počítejme neurčitý integrál 
    \begin{equation*}
      \int\frac{x}{\sqrt{x^2+1}}.
    \end{equation*} 
    Vidíme, že čitatel funkce za integrálem je až na násobení konstantou derivací výrazu pod
    odmocninou. Při označení \(u=\varphi(x) = x^2 + 1\) dostáváme \(\varphi'(x) = x\):
    \begin{align*}
      \int\frac{x}{\sqrt{x^2+1}}\dd{x} 
        &= \frac{1}{2}\int\frac{2x}{\sqrt{x^2+1}}\dd{x} \\
        &= \frac{1}{2}\int\frac{1}{\sqrt{u}}\dd{u}         
        = \sqrt{u} + c = \sqrt{x^2 + 1} + c  
    \end{align*}
  \end{example}
\end{mdframed}