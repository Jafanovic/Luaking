\begin{mathexam}{\(\protect\scalerel{\int}{\dfrac{19x+15}{x^2-x-2}\dd{x}} \quad x\in
  \realset-\{1,2\}\)}{exam129} 
  Kořeny polynomu ve jmenovateli $\alpha_1 = -1$, $\alpha_2 = 2$ jsou jednoduché - každému z nich
  bude v rozkladu odpovídat jen jeden člen: 
    \begin{align*}
      \frac{19x+15}{x^2-x-2}      &= \frac{A}{x+1} + \frac{B}{x-2} \\
                        19x +15   &= A(x-2) + B(x+1)               \\
                        19x +15   &= x(A+B) - 2A + B               \\
                        19        &= A + B                         \\
                             15   &=        - 2A + B
    \end{align*}              
    Řešením této soustavy je \(A = \frac{4}{3}\), \(B = \frac{53}{3}\).
    \begin{align*}
      &= \frac{4}{3}\int{\frac{1}{x+1}}\dd{x} + \frac{53}{3}\int{\frac{1}{x-2}}\dd{x}  \\
      &= \frac{4}{3}\ln\abs{x+1} - \frac{53}{3}\ln\abs{x-2} +  C
    \end{align*}   
\end{mathexam}