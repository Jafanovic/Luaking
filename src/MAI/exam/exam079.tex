% !TeX spellcheck = cs_CZ
\wikitextrule
\begin{example}\label{mai:exam079}
  \textbf{Počítání s komplexními čísly v goniometrickém tvaru}\newline\small
    Zvolme čísla \(z_1 = x_1 + iy_1\) a \(z_2 = x_2 + iy_2\) a počítejme jejich součin. Nejprve je 
    však vyjádříme v goniometrickém tvaru, \(z_1 =\abs{z_1}(\cos\varphi_1 + i\sin\varphi_1)\), 
    \(z_2 =\abs{z_2}(\cos\varphi_2 + i\sin\varphi_2)\). Pak
    \begin{align*}
      z_1\cdot z_2 &= \abs{z_2}\cdot\abs{z_2}(\cos\varphi_1 + i\sin\varphi_1)
                                             (\cos\varphi_2 + i\sin\varphi_2)                 \\
                   &= \abs{z_2}\cdot\abs{z_2}
                       [(\cos\varphi_1\cos\varphi_2 - \sin\varphi_1\sin\varphi_2) + 
                       i(\cos\varphi_1\sin\varphi_2 + \sin\varphi_1\cos\varphi_2)]            \\
                   &= \abs{z_2}\cdot\abs{z_2}
                       [\cos(\varphi_1 +\varphi_2) + i\sin(\varphi_1 +\varphi_2)]
    \end{align*}
  \normalsize
\end{example}