% !TeX spellcheck = cs_CZ
% \wikitextrule
\begin{mdframed}[style=mdexam]
  \begin{example}\label{mai:exam079}
    \textbf{Počítání s komplexními čísly v goniometrickém tvaru}\newline
      Zvolme čísla \(z_1 = x_1 + \imath y_1\) a \(z_2 = x_2 + \imath y_2\) a počítejme jejich
      součin. Nejprve je však vyjádříme v goniometrickém tvaru, \(z_1 =\abs{z_1}(\cos\varphi_1 +
      \imath\sin\varphi_1)\), \(z_2 =\abs{z_2}(\cos\varphi_2 + \imath\sin\varphi_2)\). Pak
      \begin{gather*}
        \begin{aligned}
          z_1\cdot z_2 &= \abs{z_2}\cdot\abs{z_2}(\cos\varphi_1 + \imath\sin\varphi_1)
                                                 (\cos\varphi_2 + \imath\sin\varphi_2)           \\
                      &= \abs{z_2}\cdot\abs{z_2}
                          [(\cos\varphi_1\cos\varphi_2 - \sin\varphi_1\sin\varphi_2)             \\ 
                      &+ \imath(\cos\varphi_1\sin\varphi_2 + \sin\varphi_1\cos\varphi_2)]        \\
                      &= \abs{z_2}\cdot\abs{z_2}
                          [\cos(\varphi_1 +\varphi_2) + \imath\sin(\varphi_1 +\varphi_2)].
        \end{aligned}
      \end{gather*}  
  \end{example}
\end{mdframed}