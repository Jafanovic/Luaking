% !TeX spellcheck = cs_CZ
\wikitextrule
\begin{example}\label{mai:exam040}
  \textbf{Obecná rovnice roviny}\newline\small
  Parametrické rovnice roviny z příkladu \ref{mai:exam004} můžeme chápat jako soustavu tří rovnic o 
  dvou neznámých:
  \begin{align*}
      ru_1 + sv_1 &= x - x_A, \\
      ru_2 + sv_2 &= y - y_A, \\
      ru_3 + sv_3 &= z - z_A, 
  \end{align*}
  kde neznámými jsou parametry \(r\) a \(s\). Z geometrického významu této soustavy je zřejmé, že 
  pro každý bod \(X = (x, y, z)\), který leží v rovině \(\varrho\), bude soustava mít jako řešení 
  právě jednu dvojici parametrů \((r, s)\) (pro body, které v rovině neleží, soustava řešení nemá). 
  Vypočteme parametry \(r\) a \(s\) například z prvních dvou rovnic. Předpokládejme, že \(u_1 \neq 
  0\), a upravujme matici soustavy:
  
  \begin{equation*}
    \left(
      \begin{array}{rr|r}
         u_1 &  v_1  &  x-x_A         \\
         u_2 &  v_2  &  y-y_A
      \end{array}
    \right) \sim
    \left(
      \begin{array}{cc|c}
              u_1 &  v_1               & x - x_A     \\
              0   &  u_1v_2 - u_2v_1   & (y-y_A)u_1 - (x-x_A)u_2
      \end{array}
    \right).
  \end{equation*}
  odkud pro \((u_1v_2 — u_2v_1) \neq 0\) dostaneme
  \begin{equation*}
    r = - \dfrac{(y-y_A)v_1 - (x-x_A)v_2}{u_1v_2 - u_2v_1}, \qquad 
    s =   \dfrac{(y-y_A)u_1 - (x-x_A)u_2}{u_1v_2 - u_2v_1}
  \end{equation*}
  Dosadíme-li získané hodnoty do třetí rovnice (dá to trochu práce), dostáváme obecnou  rovnici roviny 
  \(\varrho\)
  \begin{subequations}\label{mai:eq041}
    \begin{equation}
      ax + by + cz + d= 0,
    \end{equation}
    \begin{equation}
      a = u_2v_3 - u_3v_2, \qquad b = U_3v_1 - u_1v_3, \qquad c = u_1v_2 - u_2v_1,
    \end{equation}
    \begin{equation}
      d = (u_2v_3 - u_3v_2)x_A - (u_3v_1 - u_1v_3)y_A - (u_1v_2 - u_2v_1)z_A.
    \end{equation}
  \end{subequations}
  Při tomto výpočtu vyvstaly některé problémy. Pokusme se je vyřešit:
  \begin{itemize}
    \item Aby získaná rovnice opravdu představovala nějakou rovinu, musí v ní zůstat alespoň jedna 
          ze souřadnic \(x, y, z\). Alespoň jedno z čísel \(a, b, c\) by tedy mělo být nenulové. 
          Dokažte, že tomu tak opravdu je, a využijte při tom skutečnosti, že vektory \(\vec{u}\) a 
          \(\vec{v}\) nesmí být rovnoběžné. Co znamená předpoklad \((u_1v_2 - u_2v_1) \neq 0\)?
    \item Předpokládali jsme, že \(u_1 \neq 0\). Jak budeme postupovat, nebude-li tento předpoklad  
          splněn? Lze v tomto případě použít obecné výrazy získané pro \(r\) a \(s\)?
  \end{itemize}
  \normalsize
\end{example}