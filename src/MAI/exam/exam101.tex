% !TeX spellcheck = cs_CZ
\begin{mdframed}[style=mdexam]
  \begin{example}\label{MAI:exam101}
    Efektivní hodnota $i_{ef}$ střídavého proudu $$i(t) = I_0\sin\omega t$$ (viz
    předchozí příklad) je definována jako odmocnina ze střední hodnoty funkce $i^2(t)$ v průběhu
    jedné periody $T = \frac{2\pi}{\omega}$. Tedy
    \begin{align*}
      i_{ef}^2 &= \frac{1}{T}\int_0^T I_0^2\sin^2\omega t\dd{t} = 
                  \frac{1}{T}\int_0^T \frac{I_0^2}{2}(1- \cos2\omega t)\dd{t}           \\
               &= \frac{I_0^2}{2T}
                  \left[
                    t-\frac{\sin2\omega t}{2\omega}
                  \right]_0^T = \frac{I_0^2}{2}
    \end{align*}
    neboť $\sin2\omega T=\sin4\pi = 0.$ Odtud $$i_{ef} = \frac{I_0}{\sqrt{2}}.$$ Střídavý proud
    $i(t) = I_0\sin\omega t$ má na témže odporu stejný výkon jako stejnosměrný proud o intenzitě
    $i = 0,707I_0$.
  \end{example}
\end{mdframed}















