\begin{mathexam}{\(\int\dfrac{1}{x^2 - x + 1}\dd{x}, \qquad x\in\realset\)}{exam120}
  Kvadratický polynom ve jmenovateli upravíme na čtverec \(f(x) = (x + m)^2 + n\) a dostaneme
  integrál
  \begin{equation*}
    \int\dfrac{1}{\left(x-\dfrac{1}{2}\right)^2+\dfrac{3}{4}}\dd{x},
  \end{equation*}
  na který lze aplikovat vzorec \ref{mai:eq121} z tabulky neurčitých integrálů: 
  \begin{multline*}
    \dfrac{1}{\sqrt{1-\left(\dfrac{1}{2}\right)^2}}\arctan
    \dfrac{x-\dfrac{1}{2}}{\sqrt{1-\left(\dfrac{1}{2}\right)^2}} = \\
    \dfrac{2}{\sqrt{3}}\arctan\dfrac{2x-1}{\sqrt{3}}  =
    \dfrac{2\sqrt{3}}{3}\arctan\dfrac{\sqrt{3}(2x-1)}{3} + c
  \end{multline*}
\end{mathexam}