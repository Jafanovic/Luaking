\begin{mdframed}[style=mdexam]
  \begin{example}\label{MAI:exam124} 
    Řešme: 
    \begin{equation*}
      \int\sin^3t\cos t\dd{t}.
    \end{equation*}

    \noindent\textbf{Řešení:}

    Položme \(\sin t = x\), \(\cos t\dd{t} =\dd{x}\). Pak získáme triviální integrál
    \begin{equation*}
      \int x^3\dd{x} = \frac{1}{4}x^4 + c = \frac{1}{4}\sin^4t + c,
    \end{equation*}
    Podmínky věty jsou splněny: Funkce \(\sin t\) je na intervalu \(-\infty, +\infty\) spojitá i se
    svou derivací \(\cos t\), její hodnoty leží v intervalu \(\langle-1,1\rangle\). V tomto
    intervalu je funkce \(x^3\) spojitá \cite[s.~261]{Brabec1989}. 
  \end{example}
\end{mdframed}