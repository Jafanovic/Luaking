% !TeX spellcheck = cs_CZ
\begin{mdframed}[style=mdexam]
  \begin{example}\label{mai:exam108}
    \textbf{Výpočet úhlu dvou vektorů užitím skalárního součinu}\newline
    V ortonormální bázi jsou zadány vektory: \(\vec{u}=(3,5,4)\), \(\vec{v}=(-2,7,1)\). Vypočteme
    nejprve skalární součin těchto vektorů:
    \begin{equation*}
      \vec{u}\vec{v} = 3\cdot(-2) + 5\cdot7 + 4\cdot1 = 33.
    \end{equation*}
    Z definice skalárního součinu můžeme určit úhel vektorů, známe-li jejich velikosti. V
    ortonormální bázi (všimněme si souhlasu vzorce pro výpočet velikosti s definicí skalárního
    součinu) \cite[s.~27]{Musilova2012MA2}.
    \begin{equation*}
      \abs{\vec{u}} = \sqrt{50} = 5\sqrt{2},\; \abs{\vec{v}} = \sqrt{54} = 3\sqrt{6},
    \end{equation*}    
    \begin{equation*}
      \cos\varphi = \dfrac{33}{5\sqrt{2}\cdot3\sqrt{6}} = \dfrac{11}{10\sqrt{3}}\approx\num{0.635}.
    \end{equation*}
    {\centering
      \captionsetup{type=figure}
      \luafigure[1]{mai_fig077.png}
      \captionof{figure}{K příkladu \ref{mai:exam108}.}
      \label{fyz:fig225}
    \par}
  \end{example}
\end{mdframed}