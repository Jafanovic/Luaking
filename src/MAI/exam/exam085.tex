  % !TeX spellcheck = cs_CZ
% Musilova2009MA2
\begin{mdframed}[style=mdexam]
  \begin{example}\label{mai:exam085}
    \textbf{Příklad s koupelnou}\newline
      V koupelně o celkovém objemu \(V = \num{12000}\) litrů byl nainstalován větrák, jehož výkon je
      \(P = \num{400}\) litrů za minutu. Položme si otázku: Jaká je optimální doba, na kterou je
      třeba nastavit časový spínač, aby větrák neběžel příliš dlouho a přitom se vyměnil všechen
      vzduch v místnosti? A je to vůbec možné? Může se opravdu vzduch vyměnit všechen? Přemýšlejme o
      této situaci důkladněji. Označme \(c(t)\) funkci popisující okamžitou objemovou koncentraci
      „původního vzduchu“ v místnosti, tj. poměr objemu původního vzduchu ku objemu místnosti. (V
      okamžiku zapnutí větráku, tj. pro \(t = 0\), je \(c(0) = c_0 = 1\), v okamžiku, kdy bude
      původní vzduch zcela vyčerpán, pokud to vůbec nastane, bude \(c = 0\).) V intervalu \([t,t +
      \Delta t]\) větrák odčerpá \(P\Delta t\) litrů vzduchu celkem, z toho množství starého vzduchu
      činí \(c(t)P\Delta t\) a jeho podíl na celkovém množství vzduchu je \(c(t)P\Delta t/V\). Tato
      hodnota představuje pro velmi malé \(\Delta t\) \textbf{úbytek} koncentrace starého vzduchu v
      koupelně v časovém intervalu \([t, t + \Delta t]\), tj.
      \begin{equation*}
        \Delta c(t) = - \frac{c(t)P\Delta t}{V} \Rightarrow 
        \dfrac{\Delta c(t)}{\Delta t} = - \dfrac{P}{V}c(t)
      \end{equation*}
      
      (uměli bychom vysvětlit záporné znaménko?). Získáváme rovnici
      \begin{equation}\label{mai:eq078}
        \der{c(t)}{t} = - \dfrac{P}{V}c(t)
      \end{equation}
      Dosazením snadno ověříme, že řešením rovnice (\ref{mai:eq078}) je každá funkce
      \begin{equation*}
        c(t) = Ke^{-\frac{Pt}{V}},
      \end{equation*}

      {\centering
      \captionsetup{type=figure}
      \luafigure[1]{mai_fig056.png}
      \captionof{figure}{Graf řešení úlohy s koupelnou.}
      \label{mai:fig056}
      \par}
      
      kde \(K\) je libovolné číslo. Jeho konkrétní hodnotu pro náš případ určíme z počáteční
      podmínky \(c(0) = 1\), tj. \(K = 1\). A vida, pokud jsme počítali správně, můžeme usoudit, že
      koncentrace původního vzduchu neklesne na nulovou hodnotu nikdy. Naše řešení samozřejmě
      nevylučuje použití časového spínače - rozumně bychom mohli například požadovat, aby
      koncentrace původního vzduchu klesla na hodnotu \(c(\tau) = \num{0.1}\). Hledaná doba bude \(t
      =\frac{P}{V}\ln(1/c(\tau)) =\qty{69}{\minute}\). Řešení naší počáteční úlohy je v grafu na
      obrázku \ref{mai:fig056} vyznačeno červeně. Jakým počátečním úlohám odpovídají modré křivky?
  \end{example}
\end{mdframed}