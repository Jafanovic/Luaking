  % !TeX spellcheck = cs_CZ
% Musilova2009MA2
\begin{mdframed}[style=mdexam]
  \begin{example}\label{mai:exam084}
    \textbf{Pohyb po přímce}\newline
    Hlemýžď se pohybuje po přímce od kopretiny k pampelišce stálou rychlostí \(v_0 =
    \qty{2}{\mm\per\s}\). V počátečním okamžiku \(t = 0\) byl ve vzdálenosti \(s_0 = \qty{10}{\mm}\)
    od kopretiny. Jaká bude jeho vzdálenost od kopretiny v libovolném okamžiku \(t\geq0\)? Na tuto
    otázku by jistě snadno odpověděl i žák první třídy. Ukažme si však, že úlohu lze také vyjádřit
    pomocí diferenciální rovnice. Vzdálenost \(s(t)\) je hledanou funkcí jedné proměnné, a to času
    \(t\). Rychlost \(v_0\) rovnoměrného přímočarého pohybu je časovou derivací vzdálenosti.
    Získáváme tedy rovnici

    {\centering
     \captionsetup{type=figure}
     \luafigure[0.7]{mai_fig055.png}
     \captionof{figure}{Hlemýžď pohybující se po přímce od kopretiny k pampelišce 
                       \cite[s.~217]{Musilova2012MA2}}
     \label{mai:fig055}
    \par}
    
    \begin{equation}\label{mai:eq076}
      \der{s(t)}{t} = v_0.
    \end{equation}

    Již jsme se zmínili, že rovnice obsahující neznámou reálnou funkci jedné reálné proměnné a její
    derivace obecně i vyššího řádu se nazývá obyčejnou diferenciální rovnicí. Každá z funkcí, které
    rovnici splňují, se nazývá jejím řešením. Řád rovnice je určen nejvyšší derivací, která se v
    rovnici vyskytuje, v našem příkladu jde tedy o rovnici prvního řádu.

    {\centering
     \captionsetup{type=figure}
     \luafigure[1]{mai_fig054.png}
     \captionof{figure}{Graf řešení počáteční úlohy (\ref{mai:eq077}).}
     \label{mai:fig054}
    \par}
    
    Řešení rovnice (\ref{mai:eq076}) snadno \uv{uhodneme}. Bude jím každá funkce
    \begin{equation*}
      s(t) = v_0t +C,
    \end{equation*}
    kde \(C\) je libovolné reálné číslo. Funkce, které jsou řešením rovnice, mají stejný charakter a
    jsou odlišeny pouze číselnou hodnotou \(C\), tvoří soubor, který se nazývá \textbf{obecné řešení
    rovnice}. Ze všech funkcí, které vyhovují rovnici (\ref{mai:eq076}), však skutečný pohyb
    hlemýždě popisuje právě jedna. Abychom ji našli, potřebujeme určit správnou hodnotu \(C\). K
    jejímu zjištění stačí, abychom věděli, jaká byla poloha hlemýždě v jediném okamžiku. Jestliže
    jsme například začali měřit čas ve chvíli, kdy byl hlemýžď ve vzdálenosti \(s_0\) od kopretiny,
    máme tzv. \textbf{počáteční podmínku} \(s(0) = s_0\). Pomocí ní můžeme z nekonečně mnoha funkcí
    obecného řešení vybrat jediné \textbf{partikulární řešení}. V našem případě to bude funkce
    \(s(t) = v_0t + s_0\). Naše rovnice společně s počáteční podmínkou, tj.
    \begin{equation}\label{mai:eq077}
      \der{s(t)}{t} = v_0, \qquad s(0) = s_0,
    \end{equation}
    představuje tzv. \textbf{počáteční úlohu}. Její řešení je v grafu na obrázku \ref{mai:fig054}
    vyznačeno červeně, modře jsou vyznačena některá další partikulární řešení. Dokážete určit, jaké
    počáteční úloze odpovídají?
  \end{example}
\end{mdframed}