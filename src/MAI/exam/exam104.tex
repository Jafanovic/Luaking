% !TeX spellcheck = cs_CZ
\begin{mdframed}[style=mdexam]
  \begin{example}\label{mai:exam104}
    \textbf{Tramvaj (dilema nerozhodného zamilovaného}:\newline
    Stalo se v jednom velkém městě. Důležitým prvkem příběhu, který vám budu vyprávět, je schéma
    městské hromadné dopravy. Ve městě byla jediná tramvajová trať ve tvaru osmičky, a po ní stále
    stejným směrem jezdila tramvaj (obr. \ref{mai:fig073}).

    {\centering
    \captionsetup{type=figure}
    \luafigure[1]{mai_fig073.png}
    \captionof{figure}{Schéma tramvajové“ trati. \cite[s.~13]{Rogalewicz2007}
    \label{mai:fig073}}
    \par}

    Hlavní stanice je v místě, kde se trať k sobě v obou směrech přibližuje. Nástupní ostrůvek je
    zde mezi kolejemi a cestující může nastoupit do tramvaje jedoucí oběma směry. Ostatní stanice se
    nacházejí v určitém odstupu na celé tramvajové trati. Dvě z těchto stanic jsou znázorněny na
    grafu; mají totiž v našem příběhu svou úlohu.

    V příběhu vystupují tři postavy. Pan \(X\), jehož kancelář se nachází nedaleko hlavní stanice, a
    jeho dvě přítelkyně, slečna \(A\) a slečna \(B\). Ty žily na opačných koncích města, jak je
    znázorněno na grafu.

    Pan \(X\) řeší problém: obě dívky se mu líbí stejně a nemůže se rozhodnout, kterou z nich by měl
    požádat o ruku. Každý den se s jednou z nich schází, velice těžko však vybírá, za kterou z nich
    by měl jet právě dnes. Proto se rozhodl, že se spolehne na náhodu. Měl klouzavou pracovní dobu;
    když měl vše hotovo, odcházel z práce kdykoli během dne. Nasedl do první tramvaje, která
    přijela; pokud jela na východ, odjel za slečnou \(A\), pokud jela na západ, jel za slečnou
    \(B\). Za předpokladu, že se pravděpodobnost, že tramvaj pojede na východ, přibližně rovná
    pravděpodobnosti, že pojede na západ, musí být četnost setkání pana \(X\) se slečnou \(A\) i se
    slečnou \(B\) přibližně stejná.

    V praxi to však dopadlo úplně jinak. Jednoho krásného dne si slečna \(B\) panu \(X\)
    postěžovala, že se setkávají málo a že pan \(X\) tedy určitě má ještě nějakou jinou. Pan \(X\)
    znejistěl a rozhodl se, že bude počítat, kolikrát se sejde se slečnou \(XA\) a kolikrát se
    slečnou \(B\). S překvapením zjistil, že slečna \(B\) asi měla k podezření důvod, během měsíce
    navštívil jednadvacetkrát slečnu \(A\) a pouze devětkrát slečnu \(B\).

    {\centering
    \captionsetup{type=figure}
     \subcaptionbox{Východní smyčka kratší než západní\label{mai:fig074a}}
      {\luafigure[1]{mai_fig074a.png}}                                                    \\
     \subcaptionbox{Západní část má více stanic než východní\label{mai:fig074b}}
      {\luafigure[1]{mai_fig074b.png}}
     \captionof{figure}{Možné důvody neplatnosti nulové hypotézy \cite[s.~14]{Rogalewicz2007}
    \label{mai:fig074}}
    \par}

    Náhoda nebo znamení osudu? Jaká je příčina této nespravedlnosti ke Slečně \(B\)? Nejdříve si
    všimněme, na základě čeho se pan \(X\) rozhodl svěřit volbu směru své cesty náhodě. Za
    dostatečně dlouhý časový interval se musí počty setkání se slečnou \(A\) a slečnou \(B\) rovnat,
    protože pravděpodobnost nastoupení do tramvaje jedoucí oběma směry je stejná. Lze tedy říct, že
    neexistuje rozdíl v počtu setkání se slečnou \(A\) a \(B\). To se ve statistice \textbf{nazývá
    nulovou hypotézou}. V našem konkrétním případě bude nulová hypotéza odpovídat předpokladu, že se
    za měsíc sejde patnáctkrát se slečnou \(A\) a patnáctkrát se slečnou \(B\).

    Přesně patnáctkrát? Nejspíš ne. Intuice nám říká, že ve skutečnosti budou setkání s oběma
    dívkami v poměru blízkém ideálnímu 50 : 50. Co pak lze říct o našem poměru 21 : 9? Většina
    čtenářů bude asi mít sklon zamítnout názor, že by podobná nerovnováha mohla nastat za podmínek,
    kdy teoreticky platí 15 : 15. Řečeno odborně, čtenáři \textbf{zamítnou nulovou hypotézu} a
    předpokládají, že se předpoklady, na nichž byla založena, ukázaly nereálnými. 

    Abychom si vše představili ještě jasněji, předpokládejme, že poměr návštěv byl 17 : 13. Co lze
    říct o takovém případu? A jak by to bylo s poměry 18 : 12, nebo 19 : 11, nebo 20 : 10? Je
    Zřejmě, že tady někde leží hranice, po jejímž překročení řekněte: „Ne, to je už příliš! Něco
    takového se nemohlo stát náhodou. Takové tvrzení nemohu přijmout. Existuje nějaký skrytý důvod,
    který způsobuje, že pan \(X\) jezdí častěji za slečnou \(A\)“

    Poměr 21 : 9 mohl skutečně nastat i za těch podmínek, s kterými kalkuloval pan \(X\). Jestliže
    hodíme třicetkrát mincí, pak se může stát, že padne jednadvacetkrát panna a devětkrát orel.
    Podobně jevy jsou řídké, ale setkáme se s nimi. Lze spočítat, že z 30 hodů padne panna
    jednadvacetkrát průměrně v pěti případech ze sta. 

    Pokud nám tato četnost připadá malá, vzniká důvod pro zamítnutí nulové hypotézy. Pak je třeba
    hledat důvody, proč předpoklad neplatí. Čím to mohlo být způsobeno v našem případě? Obecně
    řečeno, čímkoli, co vnáší nerovnováhu do jednotlivých polovin tramvajové trati. Pokud je
    východní smyčka kratší než západní (obr. \ref{mai:fig074a}), pak se tramvaj v danou chvíli
    nachází s větší pravděpodobností v západní části města. Nemusí jít pouze o délku trati, ale o
    jakýkoli důvod, který způsobí, že tramvaj projíždí západní část trati déle než východní (obr.
    \ref{mai:fig074b}).
  \end{example}
\end{mdframed}