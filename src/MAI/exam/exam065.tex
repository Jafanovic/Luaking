% !TeX spellcheck = cs_CZ
\begin{mdframed}[style=mdexam]
  \begin{example}\label{mai:exam065}
    \textbf{Bernoulliovo (binomické) rozdělení}\newline
    Představme si opět Bernoulliův pokus o \(n\) opakováních a pravděpodobností zdaru při jednom
    opakování rovnou \(p\) (kapitola \ref{mai:IchapIVsecIIssecIV}). Náhodnou veličinu \(X\)
    definujme jako počet zdarů při tomto pokusu. Tato veličina nabývá všech celočíselných hodnot
    \(x_j = j, 0 \leq j \leq n\), přitom hodnoty \(j\) nabývá s pravděpodobností určenou vztahem
    (\ref{mai:eq055}), v němž za \(x\) dosadíme \(j\). 
    
    {\centering
      \captionsetup{type=figure}
      \luafigure[1]{mai_fig044.pdf}
      \captionof{figure}{Bernoulliovo rozdělení
      \cite[s.~229]{Musilova2009MA1}
      \label{mai:fig044}}
    \par}
    
    Získané rozdělení je tedy
    \begin{equation*}
      \lbrace j,p_j\rbrace, \quad\text{kde}\quad p_j = \binom{n}{j}p^j(1 - p)^{n-j}.
    \end{equation*}
    Graf Bernoulliova rozdělení, které je často nazýváno také binomickým, je na obrázku 
    \ref{mai:fig044} pro \(n = 15\) a \(p = 1/2\) (červený asterisk).
    
    Z grafu je názorně vidět, co to znamená, že některé hodnoty veličiny \(X\) jsou více a jiné méně 
    pravděpodobné. Hodnota \(x_i\), veličiny \(X\), které odpovídá největší pravděpodobnost \(p_i\), 
    se nazývá nejpravděpodobnější hodnota. V případě Bernoulliova rozdělení na obrázku 
    \ref{mai:fig044} jsou takové hodnoty dvě, konkrétně \(x_7 = 7\) a \(x_8 = 8\).
    
      \begin{lstlisting}[style=luaCPPStyle, caption={PPST001.m}]
        citatel= factorial(n);
        jmenovatel= factorial(n-j).*factorial(j);
        binom = citatel./jmenovatel;
        f = binom.*p.^j.*(1-p).^(n-j);
      \end{lstlisting}
  \end{example}
\end{mdframed}