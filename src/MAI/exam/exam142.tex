\begin{mathexam}{\(\protect\scalerel{\int}{(7x^2+2x^3+5\cos x)\dd{x}}\).}{exam142}
  V tomto případě \(f_1(x) = x^5\), \(f_2(x) = x^3\), \(f_3(x) = 5\cos x\), \(c_1=7\), \(c_1=2\) ,
  \(c_3=5\). Protože v intervalu \((-\infty, \infty)\) existují podle
  \eqref{mai:IchapVIIsecIIssecI} integrály
  \begin{align*}
    \int x^5\dd{x}   &= \frac{x^6}{6} + C_1, \\
    \int x^3\dd{x}   &= \frac{x^4}{4} + C_2, \\
    \int\cos x\dd{x} &= \sin x +C_3,
  \end{align*}
  platí podle věty \eqref{mai:lemma013} \[7\int x^5\dd{x} + 2\int x^3\dd{x} + 5\int\cos x\dd{x},\]
  a tedy (integrační konstanty shrneme v jedinou)
  \begin{fleqn}[0pt]
    \begin{multline*}
      \int(7x^2+2x^3+5\cos x)\dd{x} = \\
        = \frac{7}{6}x^6 + \frac{1}{2}x^4 + 5\sin x + c.
    \end{multline*}
  \end{fleqn}
\end{mathexam}