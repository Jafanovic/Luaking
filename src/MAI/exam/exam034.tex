% !TeX spellcheck = cs_CZ
% 
\wikitextrule
\begin{example}\label{MAI:exam034}
  Vypočteme z definice \ref{mai:eq038} skalární součiny vektorů ortonormální báze \(\vec{e_1}\), 
  \(\vec{e_2}\) a \(\vec{e_3}\), spjaté s kartézskou soustavou souřadnic. Připomeňme, že tyto 
  vektory jsou jednotkové a navzájem kolmé.
  \begin{itemize}
    \item pro \(i\neq j\) \(\vec{e_i}\vec{e_j}=0\), 
              \(\sphericalangle\vec{e_i}\vec{e_j} =\dfrac{\pi}{2}\), vektory jsou kolmé,
    \item pro \(i = j\) \(\vec{e_i}\vec{e_j}=0\), 
              \(\sphericalangle\vec{e_i}\vec{e_j} =0, \abs{\vec{e_i}}=1\), vektory jsou 
              jednotkové.
  \end{itemize}

  Pro skalární součiny vektorů ortonormální báze použijeme zkrácené značení
  \begin{equation}\label{MAI:exam035}
    \vec{e_i}\vec{e_j} = \delta_{ij},
  \end{equation}
  kde \(\delta_{ij}\) nabývá hodnoty \num{1} pro \(i = j\) a hodnoty \num{0} pro \(i \neq j\). 
  Nazývá se \textbf{Kroneckerovo delta} \cite[s.~40]{Musilova2009MA1}.
\end{example}