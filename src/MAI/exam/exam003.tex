% !TeX spellcheck = cs_CZ
\begin{mathexam}{Vyšetřete průběh funkce \[f(x):y=\frac{1+x^2}{1-x^2}\]}{exam003}
  \begin{enumerate}[noitemsep,leftmargin=12pt,rightmargin=2pt, label=\emph{\alph*})]
    \item Definiční obor $D_f=\realset-\{±1\}=(-\infty,-1)\cup(-1,1)\cup(1,+\infty)$
    \item Funkce je sudá 
        \begin{equation*}
          f(-x)=f(x): \frac{1+x^2}{1-x^2}=\frac{1+(-x)^2}{1-(-x)^2}.
        \end{equation*}
        Funkce není periodická.
    \item Stanovíme funkční hodnoty v krajních bodech definičního obor $1, -1$ a v nevlastních
        bodech \(-\infty,+\infty\). Protože je funkce \textbf{sudá}, omezíme se jen na vyšetřování
        nezáporné části. Nejprve vlastnosti funkce v okolí bodu \num{1}. Ten nepatří do \(D_f\) a
        proto určíme limity funkce v pravém a levém okolí tohoto bodu. 
        \begin{equation*}
          \lim_{x\to1_{-}}=\frac{1+x^2}{1-x^2}.
        \end{equation*}
        Pro výpočet limity použijeme substituci \(y=1-x^2\):
        \begin{align*}
          \lim_{y\to0+}\frac{2-y}{y}&=+\infty \\
          \shortintertext{proto \footnote{$\lim_{x\to0_+}\frac{1}{x}=\infty$}}
          \lim_{x\to1_{-}}\frac{1+x^2}{1-x^2}&=+\infty. \\
          \shortintertext{Obdobně dojdeme k}
          \lim_{x\to1_+}\frac{1+x^2}{1-x^2}&=-\infty.
        \end{align*}
        A konečně v nevlastních bodech \(±\infty\) je limita 
        \begin{align*}
          \lim_{x\to±\infty}\frac{1+x^2}{1-x^2} 
            &= \lim_{x\to\pm\infty}\frac{1}{1-x^2} +
               \lim_{x\to\pm\infty}\frac{x^2}{1-x^2}     \\
            &=0-1=-1.  
        \end{align*}
        Výpočtem limit jsme zároveň určili dva absolutní (globální) extrémy a jeden lokální:
        \begin{itemize}
          \item v intervalu \((-1,1)\) má funkce maximum \(\infty\) a minimum $1$,
          \item v intervalech $(-1,1)\cup(1,+\infty)$ má funkce minimum $-\infty$ a maximum $-1$.
        \end{itemize}
    \item Nyní vyšetříme zda, případně kolik a jaké, má funkce \(f(x)\) průsečíky s osami souřadnic.
        S osou \(x\) nemá funkce žádné průsečíky, protože pro $y=0$ není definována
        $H_f=\realset-\{-1,1\rangle$. Pro \(x=0\) je $y=\frac{1+0^2}{1-0^2}=1$, proto má \(f(x)\)
        právě jeden průsečík s osou \(y\) a to \([0,1]\).
    \item Zatím jsme zjistili, že naše funkce není definována v bodech $1$ a $-1$ a proto není
        spojitá v  \(\realset\). Nevíme však, jaký je její průběh v jednotlivých intervalech
        definičního oboru.  Abychom získali názornější představu o průběhu funkce, zjistíme má-li
        derivaci.
        \begin{align*}
          y' &= \frac{(1+x^2)'(1-x^2 )-(1+x^2)(1-x^2 )'}{(1-x^2)^2} \\
          y' &= \frac{2x(1-x^2 )-(1+x^2 )(-2x)}{(1-x^2 )^2}         \\
          y' &= \frac{4x}{(1-x^2 )^2}
        \end{align*}
        Protože má vlastní derivaci\footnote{\(f(x)\) je spojitá v intervalech $(-\infty,-1),
        (-1,1),(1,\infty)$  věta s spojité funkci}, můžeme určit její vlastnosti v intervalech
        \(\langle0,1)\) a \((1,\infty)\). V těchto intervalech je $y'>0$ a proto jde o funkci ryze
        monotónní, rostoucí \footnote{Plyne z věty o postačujících podmínkách ryzí monotónnosti
        funkce na intervalu} v daných intervalech \footnote{V intervalech
        \((-\infty,-1),(-1,0\rangle\) je funkce klesající.}. Výpočtem zjistíme druhou derivaci funkce.
        Ta nám pomůže určit další extrém v intervalu \(\langle0,1)\) a zároveň vyšetřit
        \textbf{konkávnost} a \textbf{konvexnost}.
        \begin{align*}
          y'' &= \frac{(4x)' (1-x^2 )^2-(4x)(1-2x^2+x^4 )'}{(1-x^2 )^4}  \\
          y'' &= \frac{4(1-2x^2+x^4 )-4x(-4x+4x^3 )}{(1-x^2 )^4}         \\
          y'' &= \frac{4(1-x^2 )(3x^2+1)}{(1-x^2 )^4}                    \\
          y'' &= \frac{4(3x^2+1)}{(1-x^2 )^3}
        \end{align*}
        Abychom mohli určit lokální extrém funkce \(f(x)\) v intervalu \(\langle0,1)\), pomocí druhé
        derivace, musíme najít kořeny rovnice \(f' (x)=0\). V našem případě
        $$y'=\frac{4x}{(1-x^2)^2}\Rightarrow\frac{4x}{(1-x^2)^2}=0\rightarrow x_0=0,$$ tento kořen
        \footnote{stacionární bod}  pak dosadíme do druhé derivace, tj. 
        $$y''(0)=\frac{4(3\cdot0^2+1)}{(1-0^2 )^3}=4,$$ protože je \(f''(x)>0\), má v bodě \(x_0\)
        lokální minimum. Můžeme rovněž konstatovat, že funkce nemá inflexní body \footnote{Pro
        existenci inflexního bodu je nutné splnění jedné z podmínek a to buď \(f''(x_0)=0\), nebo
        \(f''(x_0)\) neexistuje.}. Konkávnost a konvexnost funkce v intervalech \(\langle0,1)\) a
        \((1,\infty)\) vyšetříme pomocí vlastností druhé derivace funkce. Tedy
        \begin{itemize}
          \item \(\langle0,1): y''=\frac{4(3x^2+1)}{(1-x^2 )^3} >0 \Rightarrow\) funkce je v tomto
                intervalu \textbf{konvexní},
          \item \((1,\infty): y''=\frac{4(3x^2+1)}{(1-x^2 )^3} <0 \Rightarrow\) funkce je v tomto
                intervalu \textbf{konkáv\-ní}.
        \end{itemize}
    \item Z předchozích výpočtů plyne, že křivka má asymptoty \(y=-1,x=\pm1\).
  \end{enumerate}
  {\centering \captionsetup{type=figure}          % %\ref{mai:fig_028}
    \includegraphics[width=1\linewidth]{mai_fig028.pdf}
    \captionof{figure}{Graf funkce \(f(x):y=\dfrac{1+x^2}{1-x^2}\)}
    \label{mai:fig_028}
  \par}
\end{mathexam}  