% !TeX spellcheck = cs_CZ
\begin{mdframed}[style=mdexam]
  \begin{example}\label{mai:exam096}
    Z místa \(A\) do místa \(B\) vedou čtyři turistické cesty, z místa \(B\) do místa \(C\) tři.
    Určete, kolika zpåsoby lze vybrat trasu z \(A\) do \(C\) a zpět tak, že z těchto sedmi cest je
    právě jedna použita dvakrát. \newline
    \textbf{Řešení}: Nejprve určíme, kolika způsoby lze vybrat trasu z \(A\) do \(C\): ke každému ze
    čtyř způsobů, jak dojít z \(A\) do \(B\), existují tři způsoby, jak dojít z \(B\) do \(C\).
    Trasu z \(A\) do \(C\) lze tedy vybrat \(4\cdot3\), tj. dvanácti způsoby (obr.
    \ref{mai:fig066a}).

    {\centering
      \captionsetup{type=figure}
      \captionsetup[subfigure]{justification=centering}
      \subcaptionbox{\label{mai:fig066a}}{\luafigure[0.5]{mai_fig066a.png}}  
      \subcaptionbox{\label{mai:fig066b}}{\luafigure[0.5]{mai_fig066b.png}} \newline
      \subcaptionbox{\label{mai:fig066c}}{\luafigure[0.5]{mai_fig066c.png}}
      \captionof{figure}{K příkladu \ref{mai:exam096} \cite[s.~11]{calda2008matematika}}
      \label{mai:fig066}
    \par}
    
    Nyní jde o to, kolika způsoby lze vybrat zpáteční trasu z \(C\) do \(A\) tak, aby v ní byla
    použita právě jedna cesta z těch, po kterých jsme už šli z \(A\) do \(C\). Máme tedy dvě
    možnosti:
    \begin{itemize}[noitemsep]
      \item Po stejné cestě se budeme vracet z \(C\) do \(B\). Potom z \(B\) do \(A\) půjdeme jinou
            cestou, než kterou jsme šli z \(A\) do \(B\). V tomto případě lze vybrat zpáteční trasu
            z \(C\) do \(A\) třemi způsoby (obr. \ref{mai:fig066b}).
      \item Z \(C\) do \(B\) půjdeme jinou cestou, než kterou jsme přišli, a z \(B\) do \(A\)
            půjdeme po stejné cestě, jako z \(A\) do \(B\). V tomto případě lze vybrat zpáteční
            trasu z \(C\) do \(A\) dvěma způsoby. (obr. \ref{mai:fig066c})
    \end{itemize}
    Protože obě uvedené možnosti se navzájem vylučují a jiné nejsou, dostáváme (podle
    kombinatorického pravidla součtu), že celkový počet tras z \(C\) do \(A\), které splňují dané
    podmínky, je roven pěti. Ke každé z dvanácti tras z \(A\) do \(C\) existuje tedy pět tras z
    \(C\) do \(A\), které splňují požadovanou podmínku. Pomocí kombinatorického pravidla součinu
    získáme výsledek úlohy: 
    
    Počet všech způsobů, kterými lze vybrat trasu z \(A\) do \(C\) a zpět tak, že z daných cest je
    právě jedna použita dvakrát, je \(12\cdot5 = 60\).

    \textbf{Podobné úlohy:} Určete počet způsobů, jimiž lze vybrat trasu
    \vspace*{-0.5\baselineskip}
    \begin{itemize}[noitemsep]
      \item z \(A\) do \(C\) a zpět: \(4\cdot3\cdot3\cdot4=144\);
      \item z \(A\) do \(C\) a zpět tak, že z těchto cest není žádná použita dvakrát:
            \(4\cdot3\cdot3\cdot2=72\);
      \item z A do C a zpět tak, že z těchto cest jsou právě dvě použity dvakrát:
            \(4\cdot3\cdot1\cdot1 = 12\).
    \end{itemize}
  \end{example}
\end{mdframed}