% !TeX spellcheck = cs_CZ
\begin{mdframed}[style=mdexam]
  \begin{example}\label{MAI:exam032}
    Určete střední hodnotu $i_s$ střídavého proudu $$i(t) = I_0\sin\omega t$$ v časovém intervalu
    $\langle 0, \frac{T}{2}\rangle$ (v průběhu jedné poloviny periody). $I_0$ je maximální hodnota
    proudu (obr. \ref{MAI:exam032}), perioda $T$ je dána vztahem $T = \frac{2\pi}{\omega}$
    
    {\centering
    \captionsetup{type=figure}
    \luafigure[1]{mai_fig030.pdf}
    \captionof{figure}{K příkladu \ref{MAI:exam032}
    \cite[s.~119]{Brabec1989}
    \label{mai:fig030}}
    \par}

      Podle \ref{MA:eq_av2} bude
      \begin{align*}
      i_s &=  \frac{2}{T}
              \int_0^{\frac{T}{2}}I_0\sin\omega t\dd{t} =
              \frac{2I_0}{T}\left[-\frac{\cos\omega t}{\omega}\right]_0^{\frac{T}{2}}        \\
          &=  \frac{2I_0}{T}\frac{1}{\omega}\left(-\cos\frac{\omega T}{2}+ \cos 0\right)     \\
          &=  \frac{2I_0}{2\pi}(-\cos\pi + \cos 0) = \frac{2}{\pi}I_0 \doteq 0,637 I_0.
    \end{align*}

    Tato hodnota se rovná intenzitě elektrického proudu, při kterém by vodičem v průběhu uvažované
    poloviny periody prošel stejný elektrický náboj jako při proudu střídavém.
  \end{example}
\end{mdframed}















