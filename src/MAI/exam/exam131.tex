% \int{\sqrt{1+\cos2x}\dd{x}} 
\begin{mathexam}{\(\mathlarger{\int}{\sqrt{1+\cos2x}\dd{x}}\) \hfill\cite[s.~30]{Knichal}}{exam131} 
  Funkci $\sqrt{1+\cos2x}$ upravíme na základě goniometrické identity \ref{MA1:eq_cos2x}: \(1+\cos2x
  = 1+\cos^2x-\sin^2x=2\cos^2x\) takto
  \begin{equation*}
    \sqrt{1+\cos2x} =\sqrt{2\cos^2x} = \sqrt{2}\abs{\cos x} = \varepsilon\sqrt{2}\cos x, 
  \end{equation*}
  \begin{equation*}
    \text{kde}\,\varepsilon =
      \begin{cases} 
        +1, &  x\in \left(-\frac{\pi}{2}+2n\pi,\frac{\pi}{2}+2n\pi\right), \\
        -1, &  x\in \left(\frac{\pi}{2}+2n\pi,\frac{3\pi}{2}+2n\pi\right),
      \end{cases}
  \end{equation*}
  $n$ je přirozené číslo. Proto pro \(x\) ležící v uvedených intervalech je
  \begin{equation*}
    \varepsilon\sqrt{2}\int\cos x\dd{x} = \varepsilon\sqrt{2}\sin x + c.
  \end{equation*}
\end{mathexam}