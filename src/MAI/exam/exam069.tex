% !TeX spellcheck = cs_CZ
\begin{mdframed}[style=mdexam]
\begin{example}\label{mai:exam069} \textbf{Normální rozdělení}\newline
  Veličinou s normálním rozdělením rozumíme takovou náhodnou veličinu \(X\), jejíž hustota 
  pravděpodobnosti má tvar
  \begin{mdframed}[style=highlight]
    \begin{equation}\label{mai:eq069}
      w(x) = \dfrac{1}{\sigma\sqrt{2\pi}}\exp\left[-\dfrac{(x-\mu)^2}{2\sigma^2}\right],
             \qquad x\in(-\infty, \infty).
    \end{equation}
  \end{mdframed}
  Grafem této funkce je \textbf{Gaussova křivka}. Distribuční funkce má tvar
  \begin{mdframed}[style=highlight]
    \begin{equation}\label{mai:eq70}
      F(x) = \int_{-\infty}^{x}\dfrac{1}{\sigma\sqrt{2\pi}}
               \exp\left[-\dfrac{(t-\mu)^2}{2\sigma^2}\right]\dd{t}.
    \end{equation}
  \end{mdframed}
  udává pravděpodobnost, že hodnota náhodné proměnné je menší než zadaná hodnota (nerovnost může 
  být i neostrá). Přitom \(F(\infty) = 1\) (pravděpodobnost jistého jevu). Skutečně, platí
  \begin{equation*}
    \begin{multlined}
      \int_{-\infty}^{\infty}\dfrac{1}{\sigma\sqrt{2\pi}}
      \exp\left[-\dfrac{(t-\mu)^2}{2\sigma^2}\right]\dd{t}   \\
      \shoveleft[1cm]= \dfrac{\sigma\sqrt{2}}{\sigma\sqrt{2\pi}}
              \int_{-\infty}^{\infty}\exp\left(-u^2\right)\dd{u} =1.
    \end{multlined}
  \end{equation*}
  Takzvaný Laplaceův integrál \(\int_{-\infty}^{\infty}\exp(-u^2)\dd{u} = \sqrt{\pi}\) 
  sice můžeme najít v tabulkách a v dalším dílu jej i odvodíme, v tu to chvíli se však budeme řídit 
  výrokem lorda Kelvina: „Matematik je ten, komu je toto zřejmé jako je zřejmé vám, že dvakrát dvě 
  jsou čtyři.“ Příklady normálního rozdělení pro různé hodnoty \(\sigma,\,\mu\) a odpovídající 
  distribuční funkce vidíme na obrázku \ref{mai:fig046a} a \ref{mai:fig046b}.

  {\centering
    \captionsetup{type=figure}
     \subcaptionbox{\label{mai:fig046a}}{\luafigure[1]{mai_fig046a.pdf}}              \newline
     \subcaptionbox{\label{mai:fig046b}}{\luafigure[1]{mai_fig046b.pdf}}
     \captionof{figure}{Hustota pravděpodobnosti normálních rozdělení a jejich distribuční funkce 
              s různými charakteristikami \(\sigma\) a \(\mu\). Červenou čárou je vyznačeno 
              normované normální rozdělení. \cite[s.~240]{Musilova2009MA1}
    \label{mai:fig046}}
  \par}
  \vspace*{10px} Určíme střední hodnotu a rozptyl veličin s tímto rozdělením:
  \begin{align*}
    \langle x \rangle 
      &= \int_{-\infty}^{x}\dfrac{1}{\sigma\sqrt{2\pi}}x\cdot
         \exp\left[-\dfrac{(x-\mu)^2}{2\sigma^2}\right]\dd{x}                                     \\
      &= \dfrac{\sigma\sqrt{2}}{\sigma\sqrt{2\pi}}
         \int_{-\infty}^{\infty}\left(\mu+t\sigma\sqrt{2}\right)\cdot\exp\left(-t^2\right)\dd{t}  \\
      &= \dfrac{\mu}{\sqrt{\pi}}\int_{-\infty}^{\infty}\exp\left(-t^2\right)\dd{t}                \\
      &+ \dfrac{1}{\sqrt{\pi}}\int_{-\infty}^{\infty}t\sigma\sqrt{2}\cdot\exp\left(-t^2\right)\dd{t}
       =\mu.
  \end{align*}
  Druhý z integrálů je totiž roven nule, neboť integrand je lichá funkce.
  \begin{align*}
    D(X)  &= \dfrac{1}{\sigma\sqrt{2\pi}}\int_{-\infty}^{\infty}\left(x - \mu\right)^2 
             \exp\left[-\dfrac{(x-\mu)^2}{2\sigma^2}\right]\dd{x}                                \\
          &= \dfrac{2\sqrt{2}\sigma^3}{\sigma\sqrt{2\pi}}
             \int_{-\infty}^{\infty}t^2\exp\left(-t^2\right)\dd{t}
           = \sigma^2.
  \end{align*}
  Integrál \(\int_{-\infty}^{\infty}t^2\exp\left(-t^2\right)\dd{t} = \frac{\sqrt{\pi}}{2}\) lze buď 
  opět najít v tabulkách, nebo jej metodou per partes převést na výpočet Laplaceova integrálu:
  \begin{align*}
    I &= \int_{-\infty}^{\infty}t\cdot t\exp\left(-t^2\right)\dd{t}         \\
      &= \left[-\dfrac{t}{2}\exp\left(-t^2\right)\right]_{-\infty}^{\infty}
       + \dfrac{1}{2}\int_{-\infty}^{\infty}\exp\left(-t^2\right)\dd{t}.
  \end{align*}
  
  Distribuční funkce normálního rozdělení, zvaná \(errorfunkce\), je běžnou součástí různých 
  počítačových programů, takže poměrně snadno zjistíme pravděpodobnostní obsah intervalu určeného 
  směrodatnou odchylkou \(\sigma(x) = \sqrt{d(X)} = \sigma\). Pravděpodobnost, že hodnota náhodné 
  veličiny \(X\) s normálním rozdělením leží v intervalu \((\mu - \sigma, \mu + \sigma)\), je 
  zhruba \SI{68.3}{\percent}. V souvislosti s normálním rozdělením se často užívají další 
  dva druhy odchylek. \textbf{Pravděpodobná chyba} \(\theta\) určuje interval \((\mu - \theta, \mu 
  + \theta)\), v  němž leží hodnota veličiny \(X\) s pravděpodobností \SI{50}{\percent}. 
  \textbf{Krajní chyba} \(\kappa\) určuje interval \((\mu - \kappa, \mu + \kappa)\), v němž leží 
  hodnota veličiny \(X\) s pravděpodobností \SI{99.7}{\percent}. Z tabelovaných hodnot 
  \(errorfunkce\) zjistíme, že platí
  \begin{equation}\label{mai:eq71}
   \theta \simeq \dfrac{2}{3}\sigma, \qquad \kappa = 3\sigma
  \end{equation}
  Poznamenejme, že normálním rozdělením \(w(x)\) (\ref{mai:eq069}) lze přibližně nahradit 
  Bernoulliovo rozdělení
  \begin{align*}
    w_{Ber}(x)        &= \binom{n}{x}p^x(1 - p)^{n-x},  \\
    \langle x \rangle &= np, \;  D(x) = np(1-p)
  \end{align*}
  pro velké hodnoty \(n\) a také Poissonovo rozdělení
  \begin{equation*}
    w_{Pois}(x) = e^{-\langle x \rangle}\dfrac{\langle x \rangle^x}{x!}, \qquad 
    D(x) = \langle x \rangle
  \end{equation*}
  s velkou střední hodnotou \(\langle x \rangle\)
\end{example}
\end{mdframed}