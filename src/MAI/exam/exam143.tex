\begin{mathexam}{Určeme k funkci \(f(x)=2x+5\), \(x\in(-\infty, +\infty)\) primitivní funkci
  \(F(x)\) tak, aby její hodnota pro \(x=2\) byla rovna deseti, tj. \(F(2)=10\).}{exam143}
  %
  Podle věty \eqref{mai:lemma013} a podle \eqref{mai:IchapVIIsecIIssecI} bude \[F(x)=x^2 + 5x +
  c\qquad x\in(-\infty, +\infty)\], kde \(c\) je určitá konstanta. Má-li být \(F(2) = 10\), musí
  platit \[2^2+5\cdot2 + c = 10\], čili \(c=-4\). 

    Hledaná primitivní funkce, splňující uvednou podmínku, je \(F(x)=x^2 + 5x -4\).
\end{mathexam}