\begin{mathexam}{\(\int xe^x\dd{x}\quad x\in\realset\) \hfill\cite[s.~33]{Knichal}}{exam157} 
  Vyjdeme z rovnice \eqref{mai:eq147}, kde položíme  
  \begin{equation*}
    \left\lvert
      \begin{array}{llll} 
        u(x)  &= x   &  u'(x) &= 1  \\
        v'(x) &= e^x &  v(x)  &= e^x   
      \end{array}
    \right\rvert
  \end{equation*}
  Protože funkce \(u'(x)=1\) a \(v'(x) = e^x\) jsou spojité na \(\realset\), a tedy podmínky věty
  \ref{mai:lemma014} jsou splněny, můžeme přejít k samotnému výpočtu:    
  \begin{flalign*}
    &\int u(x)'v(x)\dd{x} = u(x)v(x) - \int u(x)v'(x)\dd{x}        &    \\
    &\int{xe^x\dd{x}}     = xe^x-\int{e^x\dd{x}} = xe^x - e^x+ c   &
  \end{flalign*}
  Derivováním se můžeme snadno přesvědčit, že jsme správně počítali:
  \[(xe^x-e^x)'=e^x+xe^x-e^x = xe^x.\]
\end{mathexam}