% !TeX spellcheck = cs_CZ
\begin{mdframed}[style=mdexam]
  \begin{example}\label{mai:exam094}
    Pokusme se určit počet všech přirozených dvojciferných čísel, v jejichž dekadickém zápisu se
    každá číslice vyskytuje nejvýše jednou. Kredit: \cite[s.~8]{calda2008matematika} \newline
    \textbf{Řešení}:
    \begin{enumerate}[noitemsep]
      \item způsob: na místě desítek může stát libovolná z devíti číslic \(1, 2, \ldots, 9\), neboť
            zde nesmí být číslice 0. Ke každé z těchto devíti možností pro výběr číslice na místě
            desítek existuje devět možností, jak vybrat číslici pro místo jednotek: může zde totiž
            být číslice 0 a jakákoliv z osmi číslic, která je různá od číslice stojící na místě
            desítek. Počet uvažovaných dvojciferných čísel je tedy \(9\cdot9 = 81\).
      \item způsob: všechna přirozená dvojciferná čísla lze rozložit do dvou disjunktních skupin
            tak, že v první jsou dvojciferná čísla s různými a ve druhé dvojciferná čísla s týmiž
            číslicemi. Je zřejmé, že všech dvojciferných čísel je 90 a dvojciferných čísel s týmiž
            číslicemi 9 ( jde o čísla \(11, 22, \ldots, 99)\); označíme-li \(p\) hledaný počet
            dvojciferných čísel s různými číslicemi, platí \(p+9 = 90\). Odtud dostáváme, že je
            \(p=81\).      
    \end{enumerate}
  \end{example}
\end{mdframed}