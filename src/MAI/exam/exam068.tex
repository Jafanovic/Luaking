% !TeX spellcheck = cs_CZ
\wikitextrule
\begin{example}\label{mai:exam068}
  \textbf{Poissonovo rozdělení}\newline\small
  Limitním případem Bernoulliova (binomického) rozdělení pro velké hodnoty \(n\), tj. \(n 
  \rightarrow \infty\), a pro \(j \ll n\) je \textbf{rozdělení Poissonovo}. Odvodíme je. Pro velká 
  \(n\) a \(j \ll n\) platí přibližný vzorec
  \begin{equation*}
    n! \simeq n^j(n-j)!,
  \end{equation*}
  a tedy
  \begin{equation*}
    P_j \simeq \lim\limits_{n\rightarrow\infty}\dfrac{n^j}{j!}p^j(1-p)^{n - j}
        \simeq \lim\limits_{n\rightarrow\infty}\dfrac{(np)^j}{j!}\left(1-\dfrac{np}{n}\right)^{n}.
  \end{equation*}
  Při vzpomínce na kapitolu o počítání i limitami a na \textbf{l’Hospitalovo pravidlo} snadno 
  provedeme následující výpočet. O testujte své předchozí znalosti a jednotlivé kroky výpočtu 
  proveďte:
  \begin{equation*}
    \lim\limits_{n\rightarrow\infty}\left(1 - \dfrac{A}{n}\right)^n = 
    \lim\limits_{x\rightarrow0}\left(1 - Ax\right)^\dfrac{1}{x} =
    \exp\left(\lim\limits_{x\rightarrow0}\dfrac{\ln(1 - Ax)}{x}\right) = e^{-A}.
  \end{equation*}
  V našem případě je \(A = np = \langle x \rangle\) (střední hodnota veličiny \(X\) při 
  Bernoulliově rozdělení), takže
  \begin{equation}\label{mai:eq065}
    p_j = \langle x \rangle^j\dfrac{e^{-\langle x \rangle}}{j!}.
  \end{equation}
  Možnost použití Bernoulliova rozdělení si již představit dokážeme. Přinejmenším jsou hody mincemi 
  a kostkami pěkné hříčky. K čemu však je dobré rozdělení Poissonovo? Ja k může vypadat praktická 
  situace, kdy provádíme obrovské množství opakování pokusu a zajímá nás pravděpodobnost pouze 
  malého počtu zdarů? Typickým příkladem takové situace je registrace částic vznikajících při 
  radioaktivním rozpadu. Taková měření jsou potřebná nejen ve fyzikálním výzkumu, ale i v 
  aplikovaných oborech, například v lékařství. Počet \(n\) radioaktivních rozpadů za jednotku času, 
  například za sekundu, je u běžných zdrojů obrovský, zatímco počet těch z nich, které jsou 
  zachycovány detektorem, může být při určitých experimentech malý. Částice se registrují
  pomocí Geigerova-Mullerova počítače, kterým je ionizační komora pracující ve vhodném režimu. 
  Pokud je počet částic \(j\) dopadajících za jednu sekundu do detektoru velmi malý, vyvolá každá z 
  nich měřitelný a dokonce slyšitelný pulz (v obvodu to „praská“). Po registraci částice potřebuje 
  detektor jistou \textbf{mrtvou dobu}, aby se vrátil do výchozího stavu, v němž je schopen 
  registrovat další částici. Tato doba se pohybuje kolem \SI{e-4}{s}. Aby byly jednotlivé pulzy 
  dobře odlišeny, je však třeba, aby do detektoru dopadlo za jednu sekundu mnohem a mnohem méně 
  částic, než jak by odpovídalo převrácené hodnotě mrtvé doby. Zejména pokud bychom chtěli pulzy 
  počítat sluchem, nemělo by jich být více než zhruba jeden až dva každou sekundu. Uvažujme o 
  radioaktivním rozpadu jader cesia \ce{Cs^137}. Jedná se o takzvaný \textbf{beta-rozpad}, který 
  probíhá následovně:
  \begin{itemize}
    \item \ce{Cs^137} \(\longrightarrow\) \ce{Ba^137} + elektron + neutrino, asi \SI{8}{\percent} 
          všech rozpadů
    \item \ce{Cs^137} \(\longrightarrow\) \ce{Ba^137}* + elektron + neutrino, asi \SI{92}{\percent} 
          všech rozpadů
  \end{itemize}
  Excitované baryum \ce{Ba^137}* (jádro má vyšší energii než atom \ce{Ba^137} v základním stavu 
  asi o \SI{0.66}{\mega\electronvolt}  - odpovídá energii \SI{1.1e-13}{\joule}) se pak dále rozpadá 
  podle vzorce
  \begin{itemize}
    \item \ce{Ba137}* \(\longrightarrow\) \ce{Ba137} + částice gama.
  \end{itemize}

  Ze všech částic, které při reakci vznikají, se v Geigerově-Můllerově detektoru registrují 
  elektrony a částice gama, nelze je však od sebe odlišit. Uvažujme o cesiovém zdroji s běžnou 
  hodnotou aktivity, například \SI{10}{\pico\coulomb} (mikrocurie). Jednotka aktivity 
  radioaktivních preparátů \num{1} curie představuje situaci, kdy se za jednu sekundu rozpadá 
  \num{3.7e10} jader. Počet rozpadů, které v průměru nastanou třeba za deset sekund v našem vzorku, 
  je \(n = \num{3700000}\). V Bernoulliově pokusu to odpovídá počtu jeho opakování \(n\). Ja k jsme 
  již řekli, je to obrovský počet. Nastavíme-li experiment tak, abychom registrovali každou sekundu 
  zhruba jednu částici (uslyšíme jeden „prásk“ ), bude počet zdarů \(j\) v Bernoulliově pokusu 
  velmi malý ve srovnání s \(n\). Jsou tedy splněny podmínky pro použití Poissonova rozdělení. 
  Zvolíme například desetisekundový interval měření a počítáme pulzy. Počet registrovaných pulzů 
  \(j\) v tomto intervalu je roven počtu zdarů. Takové měření provedeme třeba dvěstěkrát.
  Označme počet intervalů, v nichž jsme naměřili právě \(j\) pulzů, jako \(\nu(j)\). Celkem je v 
  \(\nu(1) + \cdots + \nu(j_{max}) = \num{200}\). Získáme tak tabulku nebo graf, z nichž pak lze 
  usuzovat na parametry Poissonova rozdělení:
  \begin{table}[ht!]
    \centering
    \resizebox{0.8\textwidth}{!}{%
    \begin{tabular}{c|crrrrrrrrrrrrrrrrr}
      \(j\)      & 0 & 1 & 2 & 3 & 4 & 5 & 6 & 7 & 8 & 9 & 10 & 11& 12& 13 & 14 & 15 & 16 & 17   \\
      \hline
      \(\nu(j)\) & 0 & 4 & 10 & 19 & 28 & 33 & 34 & 26 & 19 & 10 & 6 & 3 & 2 & 2 & 2 & 0 & 1 & 1 \\
    \end{tabular}}
    % \caption{ }
  \end{table}
  Budeme-li předpokládat, že větší počet pulzů v desetisekundovém intervalu je již velmi málo 
  pravděpodobný, můžeme četnosti \(\nu(j)\) považovat za úměrné pravděpodobnostem \(p_j\) 
  Poissonova rozdělení. Všimněme si formule pro Poissonovo rozdělení podrobněji. Je vidět, že platí
  \begin{equation*}
    \dfrac{\nu(j + 1)}{\nu(j)} = \dfrac{p_{j+1}}{p{j}} = \dfrac{\langle x \rangle}{j + 1},
    \qquad p_0 = e^{-\langle x \rangle}.
  \end{equation*}
  Pro hodnotu \(j\), pro kterou jsou si četnosti \(\nu(j)\) a \(\nu(j + 1)\) „nejblíže“, je 
  \(\langle x \rangle \simeq j + 1\). Z tabulky vidíme, že v případě našeho experimentu je 
  \(\langle x \rangle = 6\). Hodnota \(p_0 = e^{-6} \simeq \num{0.002}\) je tedy tak malá, že se 
  ani nedivíme, že jsme mezi dvěma stovkami měření nezaznamenali ani jeden případ, kdy v 
  desetisekundovém intervalu nebyla zaregistrována žádná částice.
  
  Můžeme ještě určit podíly sousedních hodnot \(\nu(j + 1)\) a \(\nu(j)\) a zjistit, zda výsledky 
  našeho experimentu odpovídají vlastnostem Poissonova rozdělení:
  \begin{table}[ht!]
    \centering
    \resizebox{0.7\textwidth}{!}{%
    \begin{tabular}{c|crrrrrrrr}
      \(j\)               & 0 & 1 & 2  & 3  & 4  & 5  & 6  & 7 & 8    \\ \hline
      \(\nu(j+1)/\nu(j)\) & - & \num{2.50} & \num{1.90} & \num{1.47} & \num{1.18} 
                          & \num{1.03} & \num{0.76} & \num{0.73} & \num{0.53}   \\
      \(\langle x \rangle / j + 1\) & \num{6.00} & \num{3.00} & \num{2.00} & \num{1.50} & \num{1.20}
                          & \num{1.00} & \num{0.86} & \num{0.75} & \num{0.67}
    \end{tabular}}
    % \caption{ }
  \end{table}
  \begin{table}[ht!]
    \centering
    \resizebox{0.7\textwidth}{!}{%
    \begin{tabular}{c|crrrrrrrr}
      \(j\)               & 9 & 10 & 11  & 12  & 13  & 14  & 15  & 16 & 17    \\ \hline
      \(\nu(j+1)/\nu(j)\) & \num{0.60} & \num{0.50} & \num{0.67} & - & - & - & - & - & -   \\
      \(\langle x \rangle / j + 1\) & \num{0.60} & \num{0.55} & \num{0.50} & \num{0.46}
                          & \num{0.43} & \num{0.40} & \num{0.38} & \num{0.35} & \num{0.33}
    \end{tabular}}
    % \caption{ }
  \end{table}
  Vidíme, že hodnoty podílů sousedních četností celkem dobře odpovídají vlastnostem Poissonova 
  rozdělení pro \(0 \leq j \leq 12\). Pro \(j > 12\) jsou již četnosti \(\nu(j)\) tak malé, že 
  vytvářet jejich podíly nemá smysl. Tato skutečnost je v tabulce vyznačena pomlčkou.
  
  Poissonovým rozdělením se řídí také například četnost červených krvinek, které se v daném časovém 
  intervalu objeví ve vymezené části zorného pole mikroskopu, četnost zmetků v dodávce zboží, 
  četnost překlepů písařky, apod.  
\normalsize
\end{example}