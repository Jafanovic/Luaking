% !TeX spellcheck = cs_CZ
\begin{mathexam}{Určete vlastní čísla a vlastní vektory matice \(\mathbf{B} = \mathbf{A}^2 -
  4\mathbf{A} + 9\mathbf{A}^{-1} - \mathbf{I}\), kde \(\mathbf{A}\) je matice \(\mathbf{A} = 
  \begin{pmatrix}1&\num{0.5}\\\num{3.5}&4\end{pmatrix}\).}{exam014}

  \textbf{Řešení}: (z předchozího příkladu víme, že \(\lambda_1=4.5, \lambda_2=0.5\)) a
  \(\mathbf{I}\) jednotková matice. Označme symbolem \(\lambda\) vlastní číslo matice \(\mathbf{A}\)
  a nechť \(\mathbf{x}\) je příslušný vlastní vektor. Pak platí:
  \begin{itemize}
    \item Matice \(\mathbf{A}^2\) má vlastní čísla rovna \(\lambda^2\).
    \item Matice \(4\mathbf{A}\) má vlastní čísla rovna \(4\lambda\).
    \item Matice \(9\mathbf{A}^{-1}\) má vlastní čísla rovna \(\frac{9}{\lambda}\).
  \end{itemize}
  Matice \(\mathbf{B}=\mathbf{A}^2-4\mathbf{A}+9\mathbf{A}^{-1}-\mathbf{I}\) má vlastní čísla ve
  tvaru  \(\lambda^2-4\lambda+\frac{9}{\lambda}-1\), vlastní vektory jsou stejné jako vlastní
  vektory odpovídající vlastním číslům matice \(\mathbf{A}\). Tedy:
  \begin{multline*}
      \sigma(B)=\{\num{4.5}^2-4\cdot\num{4.5}+\frac{9}{\num{4.5}}-1,\\
      \num{0.5}^2-4\cdot\num{0.5}+\frac{9}{\num{0.5}}-1\}=\{\num{3.25}, \num{15.25}\}
  \end{multline*}
\end{mathexam}