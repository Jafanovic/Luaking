% !TeX spellcheck = cs_CZ
%====================== Sbírka řešených příkladů ==================================================
\begin{mathexam}{Je dána okamžitá rychlost \(v\) pohybu bodu po přímce (ose) \(x\) rovnicí \(v(t) =
  2t + 1\), \(t\in\langle -\infty,+\infty \rangle\). Najděte zákon dráhy pohybu, je-li známo, že v
  čase \(t = 0\) měl bod polohu \(x = x_0\) \cite[p.~253]{Brabec1989}.}{exam117}

  Označíme-li \(x(t)\) polohu bodu v okamžiku \(t\), pak \(v(t) = \frac{dx}{dt}\). Hledáme tedy
  funkci \(x = x(t)\), pro níž platí \[\frac{dx}{dt} = 2t + 1 \qquad x(0) = x_0.\] Je ihned patrné,
  že první podmínce vyhovuje nekonečně mnoho funkcí
  \begin{equation}\label{MA:int_ex_09}
    x(t) = t^2 + t + c, 
  \end{equation}
  kde \(c\) je libovolná konstanta. Funkce, která splňuje i druhou podmínku (říkáme ji též počáteční
  podmínka), najdeme z rovnice \ref{MA:int_ex_09} dosazením dané podmínky \(t = 0\), \(x = x_0\).
  Dostaneme \(x_0 = c\). Dosazením do \ref{MA:int_ex_09} za \(c\) plyne hledaný zákon dráhy \(x(t) =
  t^2+t+x_0\).                 

  Jednoduchou zkouškou se přesvědčíme, že tato funkce splňuje obě dané podmínky a zároveň vidíme, že
  hledaná primitivní funkce daných vlastností je jediná.
\end{mathexam}