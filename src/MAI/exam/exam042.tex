% !TeX spellcheck = cs_CZ
\wikitextrule
\begin{example}\label{mai:exam042}
  \textbf{Vektor rovnoběžný s rovinou}\newline\small
   Jak poznáme, zda je vektor \(\vec{u} = (u_1, u_2, u_3)\) rovnoběžný s rovinou \(ax + by + cz + 
   d = 0\)? Pokud vektor \(\vec{u}\) s rovinou rovnoběžný je, pak zcela jistě existují v této 
   rovině dva body \(A = (x_A, y_A, z_A)\) a \(B = (x_B, y_B, z_B)\) tak, že \(\overrightarrow{AB} 
   = \vec{u} = (x_B - x_A, y_B - y_A, z_B - z_A)\). Tyto body splňují rovnici roviny, tj.
   \begin{equation*}
     ax_A + by_A + cz_A + d = 0,\qquad ax_B + by_B + cz_B + d = 0.
   \end{equation*}
   Odečtením rovnic dostaneme kritérium rovnoběžnosti vektoru s rovinou \(au_1 + bu_2+ cu_3 = 0\).
   \normalsize
\end{example}