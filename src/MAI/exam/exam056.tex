% !TeX spellcheck = cs_CZ
\wikitextrule
\begin{example}\label{mai:exam056}
  \textbf{Hody kostkou trochu jinak - jev \(D\)}\newline\small
  Označme nyní jako jev \(A\) „Při hodu kostkou padne šestka.“ a jako jev \(B\) „Při hodu kostkou 
  padne pětka.“ Jev \(D\) nechť je definován jako \(D = (A\text{ nebo }B)\), tj. „Při hodu kostkou 
  padne šestka nebo pětka.“ Pravděpodobnosti jevů \(A\) a \(B\) jsou \(p(A) = p(B) = 1/6\). Jevy 
  \(A\) a \(B\) jsou přitom \textbf{neslučitelné} (též vylučující se ). Nemůže
  totiž padnout pětka a šestka současně. Platí
  \begin{align*}
    N(A) &= N(B) = N(D) = N = 6                                                              \\
    M(A) &= M(B) = 1 \qquad M(D) = M(A) + M(B) = 2,                                          \\
    p(D) &= \dfrac{M(D)}{N(D)} = \dfrac{M(A) + M(B)}{N} = \dfrac{M(A)}{N} + \dfrac{M(B)}{N}
          = p(A) + p(B) = \dfrac{2}{6} = \dfrac{1}{3}.
  \end{align*}
  \normalsize
\end{example}