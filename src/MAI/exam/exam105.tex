% !TeX spellcheck = cs_CZ
\begin{mdframed}[style=mdexam]
  \begin{example}\label{mai:exam105a}
    Zjednodušte výrazy \cite[s.~22]{calda2008matematika} :
    \begin{enumerate}[label=\emph{\alph*}),noitemsep]
      \item \(\dfrac{(n+1)!}{n!} - \dfrac{(2n)!}{(2n+1)!} + \dfrac{(3n-1)!}{(3n-2)!}\)
      \item \(\dfrac{(n+1)!}{(n!)^2} - \dfrac{n!}{[(n-1)!]^2}\)
    \end{enumerate}
    V obou případech zkrátíme jednotlivé zlomky a výsledný výraz upravíme:
    \begin{equation*}
      \frac{(n+1)\cancel{n!}}{\cancel{n!}} - \frac{\cancel{(2n)!}}{(2n+1)\cancel{(2n)!}} +
      \frac{(3n-1)\cancel{(3n-2)!}}{\cancel{(3n-2)!}}
    \end{equation*}
    \begin{equation*}
      (n+1) - \frac{1}{(2n+1)} + (3n-1) = \frac{8n^2 + 4n -1}{2n + 1}
    \end{equation*}
    Podobně pro druhý výraz
    \begin{equation*}
      \frac{(n+1)\cancel{n!}}{n!\cancel{n!}} + \frac{n\cancel{(n-1)!}}{(n-1)!\cancel{(n-1)!}} =
      \frac{(n+1)+n\cdot n}{n!}
    \end{equation*}
  \end{example}

  \begin{example}\label{mai:exam105b}
    Dokažte, že pro všechna přirozená čísla \(n\) platí
    \begin{equation*}
      n!(n+3)! > (n+1)!(n+2)!
    \end{equation*}
    Výraz na levé straně nerovnice upravíme:
    \begin{equation*}
      \frac{(n+1)!}{n+1}\cdot(n+3)(n+2)! = (n+1)!(n+2)!\frac{n+3}{n+1}
    \end{equation*}   
    Protože pro všechna \(n\in\naturalset\) je zlomek \(\frac{n+3}{n+1}>1\), dokazovaný vztah
    \textbf{platí}  
  \end{example}
\end{mdframed}