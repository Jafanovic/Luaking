\begin{mdframed}[style=mdexam]
  \begin{example}\label{MAI:exam136}
    \begin{equation*}
      R(x) = \dfrac{x + 2}{x^3 - x}
    \end{equation*}
    \noindent\textbf{Řešení:}

    Racionální funkce je ryze lomená. Jmenovatel \(x^3 + x = x(x^2-1) = x(x+1)(x-1)\) má jedoduché
    reálné různé kořeny \(0,-1,1\). Podle věty o rozkladu platí
    \begin{equation*}
      \dfrac{x+2}{x(x+1)(x-1)} = \dfrac{A}{x} + \dfrac{B}{x+1} + \dfrac{C}{x-1}
    \end{equation*}
    Odtud
    \begin{align}
      x+2 &= A(x+1)(x-1) + Bx(x-1)            \nonumber          \\ 
          &+ Cx(x+1),                         \label{mai:eq162}  \\
      x+2 &= Ax^2 - A +Bx^2 - Bx + Cx^2 + Cx  \nonumber          \\
      x+2 &= (A+B+C)x^2 + (-B+C)x - A         \nonumber
    \end{align}
    Porovnáním koeficientů u jednotlivých mocnin proměnné \(x\) na levé a pravé straně dostaneme
    rovnice 
    \begin{equation*}
      \begin{array}{rcrcrcl}
        A &+& B &+& C &=&0, \\
          &-& B &+& C &=&1, \\
       -A & &   & &   &=&2. 
      \end{array}   
    \end{equation*}
    Řešením této lineární soustavy dostáváme \(A = -2\), \(B = \frac{1}{2}\), \(C = \frac{3}{2}\).
    Je tedy 
    \begin{equation}\label{mai:eq163}
      \dfrac{x+2}{x^3-x} = -\dfrac{2}{x} + \dfrac{1}{2}\dfrac{1}{x+1} + \dfrac{3}{2}\dfrac{1}{x-1}
    \end{equation} 
    Sloučením zlomků vpravo se můžeme přesvědčit o správnosti vysledku. Převzato z
    \cite[p.~268]{Brabec1989}
  \end{example}
\end{mdframed}