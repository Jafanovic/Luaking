% !TeX spellcheck = cs_CZ
\wikitextrule
\begin{example}\label{mai:exam037}
  \textbf{Jeníček a Mařenka kradli ježibabě perník:}\newline\small
  Dohromady snědli \num{11} perníkových srdíček. Jeníček jich přitom zkonzumoval o \num{3} více než 
  Mařenka. Otázka je tradiční — kolik srdíček snědl každý z nich? Označíme-li \(M\) počet kousků, 
  které snědla Mařenka a \(J\) počet srdíček, na nichž si pochutnal Jenda, můžeme informace zadané 
  v úloze zapsat takto:
  \begin{equation*}
    M + J = 11, \qquad J = M + 3.
  \end{equation*}
  
  Řešení není problémem, snadno vidíme, že \(M = 4\) a \(J = 7\).
  \normalsize
\end{example}