  % !TeX spellcheck = cs_CZ
% Musilova2009MA1
% \wikitextrule
\begin{mdframed}[style=mdexam]
\begin{example}\label{mai:exam001}
  \textbf{Parametrické vyjádření přímky}\newline\emph{Přímka} - jednorozměrný lineární útvar
  v jednorozměrném prostoru \(\mathbb{R}^1\), dvojrozměrném prostoru \(\mathbb{R}^2\), trojrozměrném
  prostoru \(\mathbb{R}^3\) (nebo i n-rozměrném prostoru \(\mathbb{R}^n\)), je určena dvěma body,
  třeba \(A\) a \(B\), nebo ekvivalentně, bodem \(A\) a \emph{směrovým} vektorem \(\vec{u}\) (obr.
  \ref{mai:fig000}). Je-li \(X\) obecným bodem na této přímce, je vektor \(\overrightarrow{AX}\)
  rovnoběžný, tj. \emph{kolineární}, se směrovým vektorem \(\vec{u}\). (Jako směrový můžeme
  samozřejmě použít i vektor \(\overrightarrow{AB}\).) Vektor \(\overrightarrow{AX}\) má tedy s
  vektorem \(\vec{u}\) stejný směr, lišit se může velikostí nebo orientací. Tuto skutečnost zapíšeme
  tak, že \(\overrightarrow{AX}\) je \(t\)-násobkem vektorů \(\vec{u}\),
  
  \begin{equation*}
    \overrightarrow{AX} = t \cdot \vec{u}.
  \end{equation*}
  Veličinou \(t\), takzvaným \emph{parametrem}, který může nabývat všech reálných hodnot,
  \(t\in\mathbb{R}\), dokážeme popsat všechny vektory \(\overrightarrow{AX}\), jejichž koncový
  bod \(X\) leží na přímce \(p\). Naopak, žádné jiné body \(X\) než ty, které leží na přímce
  \(p\), tuto vlastnost nemají. S označením bodů \(A\), \(X\), resp. vektorů \(\vec{u}\),
  \(\overrightarrow{AX}\) kartézskými souřadnicemi, resp. složkami

  \begin{align*}
    A &= (x_A,y_A, z_A), \; x =(x,y,z),\; \vec{u} = (u_1,u_2,u_3),  \\ 
    \overrightarrow{AX} &= (x - x_A, y - y_A, z-z_A),
  \end{align*}

  { \centering
    \captionsetup{type=figure}
    \luafigure[1]{mai_fig000.pdf}
    \captionof{figure}{Zadáni přímky. \cite[s.~1]{Musilova2009MA1}
    \label{mai:fig000}}
  \par}

  dostáváme \textbf{parametrické vyjádřeni přímky} \(p\) ve tvaru
  \begin{equation*}
    p = \left\{
      \begin{matrix}  
        (x,y,z)\in\mathbb{R}^3  \\
              t \in\mathbb{R}
      \end{matrix}
    \,\left\lvert\,
      \begin{matrix}
        x = x_A + tu_1,        \\
        y = y_A + tu_2,        \\
        z = z_A + tu_3,
      \end{matrix}\right.
    \right\}. 
  \end{equation*}
  \normalsize
\end{example}
\end{mdframed}