% !TeX spellcheck = cs_CZ
\begin{mdframed}[style=mdexam]
  \begin{example}\label{mai:exam107}
    \textbf{Uspořádané \(n\)-tice komplexních čísel jako vektory}\newline
    Na množině uspořádaných \(n\)-tic komplexních čísel
    \begin{equation*}
      a=(\alpha^1,\alpha^2,\ldots,\alpha^n)\in\cmplxset\times\cmplxset\times\cdots\times\cmplxset
    \end{equation*}
    zavedeme operace sčítání
    \begin{align}
      a + b &= (\alpha^1,\alpha^2,\ldots,\alpha^n) + (\beta^1,\beta^2,\ldots,\beta^n)   \nonumber\\
            &= (\alpha^1+ \beta^1,\alpha^2 +\beta^2,\ldots,\alpha^n + \beta^n)    \label{mai:eq098}
    \end{align}
    a násobení skalárem (komplexním číslem)
    \begin{equation}\label{mai:eq099}
      \gamma a = \gamma(\alpha^1,\alpha^2,\ldots,\alpha^n) 
               = (\gamma\alpha^1,\gamma\alpha^2,\ldots,\gamma\alpha^n). 
    \end{equation}
    O tom, že takto konkrétně definované operace splňují všechny axiomy vektorového prostoru nad
    polem komplexních čísel, se snadno přesvědčíme sami. Jde nyní o to, zda se jedná o prostor
    konečné dimenze a jaká tato dimenze je. Označme
    \begingroup
    \setlength{\arraycolsep}{0pt}
    \begin{equation*}
    \begin{array}{rl}
      % https://tex.stackexchange.com/questions/114959/putting-row-of-dots-in-an-equation
      e_1     &= (1, 0, 0, \ldots, 0, 0),  \\
      e_2     &= (0, 1, 0, \ldots, 0, 0),  \\
      \hdotsfor{2}                         \\
      e_{n-1} &= (0, 0, 0, \ldots, 1, 0),  \\
      e_n     &= (0, 0, 0, \ldots, 0, 1).
    \end{array}
    \end{equation*}
    Tyto \(n\)-tice jsou lineárně nezávislé. Zkusme z nich udělat nulovou lineární kombinaci!
    \begin{align*}
      \gamma^1e_1 + \gamma^2e_2 + \cdots + \gamma^ne_n 
        &= (\gamma^1, \gamma^2, \ldots, \gamma^n)              \\
        &= (0,0, \ldots, 0).
    \end{align*}
    Je zřejmé, že všechny koeficienty \(\gamma^j, j\in\{1,2,\ldots,n\}\), jsou nulové \(n\)-tice
    \(e_1\) až \(e_n\) jsou tedy lineárně nezávislé. Zvolíme-li libovolnou \(n\)-tici
    \(\alpha=(\alpha^1, \alpha^2,\ldots, \alpha^n\), můžeme ji zapsat ve tvaru
    \begin{equation*}
      a = \alpha^1e_1 + \alpha^2e_2 + \ldots + \alpha^ne_n.
    \end{equation*}
    Soubor \(e_1, e_2, \ldots, e_n\) je tedy bází ve vektorovém prostoru uspořádáných \(n\)-tic nad
    polem komplexních čísel. Tento prostor je \(n\)-rozměrný. Báze, kterou jsme nalezli, je
    obzvlášťě jednoduchá, a proto je čsto používána. Říkáme jí báze \emph{standardní}. Podobně
    bychom mohli zkonstruovat standrardní bázi vektorovém prostoru \(n\)-tic reálných čísel nad
    polem \(\realset\).
    \endgroup
  \end{example}
\end{mdframed}