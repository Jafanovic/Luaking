% !TeX spellcheck = cs_CZ
\wikitextrule
\begin{example}\label{mai:exam053}
  \textbf{Losování karet}\newline\small
    Máme karetní hru mariáš, která obsahuje celkem \num{32} karet osmi hodnot \num{7}, \num{8}, 
    \num{9}, \num{10}, J (kluk), Q (dáma), K (král), A (eso), každá hodnota je ve čtyřech barvách, 
    červené barvy jsou \(\heartsuit\) (srdce) a \(\lozenge\) (kára), černé barvy jsou \(\spadesuit\)
    (piky) a \(\clubsuit\) (kříže). Jaká je pravděpodobnost, že při náhodném vylosování deseti 
    karet budou 
    mezi nimi:
    \begin{enumerate}[label=\Alph*]
      \item právě dvě esa,
      \item alespoň dvě esa,
      \item nejvýše dvě esa,
      \item alespoň šest karet stejné barvy,
      \item právě dvě dámy a alespoň jeden kluk,
      \item právě dvě dámy nebo alespoň jeden kluk?
    \end{enumerate}

    Písmena (A) až (F) představují různé části úlohy a také zároveň definují jevy, jejichž 
    pravděpodobnost počítáme. Jedná se opět o kombinace. Nezáleží totiž na pořadí, v jakém karty 
    vytahujeme. Důležité je jen to, zda jsou vyjmenované karty ve výběru obsaženy. Počet možných 
    výsledků náhodného vylosování deseti karet z dvaatřiceti, tj. počet případů možných, je pro 
    všechny části úlohy stejný,
    \begin{equation*}
      N = \begin{pmatrix} 32 \\ 10\end{pmatrix} 
        = \dfrac{32\cdot31\cdots24\cdot23}{10\cdot9\cdots2\cdot1} = \num{64512240}.
    \end{equation*}
    
    Počítejme nyní případy příznivé pro jednotlivé jevy \(A\) až \(F\) a pravděpodobnosti těchto 
    jevů:
    \begin{equation*}
      M(A) = \begin{pmatrix} 4 \\ 2\end{pmatrix}\begin{pmatrix} 32 - 4 \\ 10 - 2\end{pmatrix}
           = \begin{pmatrix} 4 \\ 2\end{pmatrix}\begin{pmatrix} 28 \\ 8\end{pmatrix}
           = 6\cdot\dfrac{28\cdot27\cdots1}{8\cdot7\cdots2\cdot1} = \num{18648630}.
    \end{equation*}
    Jak jsme k tomuto výsledku došli? Příznivý pro daný jev je každý výběr, v němž jsou obsažena 
    právě dvě esa (libovolných barev) a žádná další esa (význam slova „právě“). Počet výběrů dvou 
    es z celkového počtu čtyř es je \(C_2(4)\), počet výběrů dalších libovolných osmi karet ze 
    zbývající části hry, která vznikne po odstranění es (nechceme, aby v příznivém výběru byla 
    další esa), je \(C_{(10-2)}(32 - 4) = C_8(28)\). Každý výběr dvojice es lze kombinovat s každým 
    výběrem zbývajících osmi karet ze zbytku hry, tj. \(M(A) = C_2(4)\cdot C_8(28)\). A to je právě
    náš předchozí výsledek. Potom:
    \begin{equation*}
      p(A) = \dfrac{M(A)}{N} 
           = \dfrac{\begin{pmatrix} 4 \\ 2\end{pmatrix}\begin{pmatrix} 28 \\ 8\end{pmatrix}}
                   {\begin{pmatrix} 32 \\ 10\end{pmatrix}}
           = \dfrac{\num{18648630}}{\num{64512240}} \simeq \num{0.29}
    \end{equation*}
    
    Aby nastal jev \(B\), požadujeme, aby v náhodném výběru deseti karet z dvaatřiceti byla alespoň 
    dvě esa. To znamená, že výběr považujeme za příznivý, obsahuje-li dvě esa libovolné barvy a osm 
    libovolných karet jiné hodnoty, nebo obsahuje tři esa libovolné barvy a sedm libovolných karet 
    jiné hodnoty, nebo obsahuje všechna čtyři esa a šest libovolných karet jiné hodnoty, \(k\) es 
    (pro \(k = 2, 3, 4\)) můžeme ze čtyř es vybrat \(\begin{pmatrix} 4 \\ k\end{pmatrix}\) způsoby.
    \(10 - k\) karet jiné hodnoty pak musíme vybírat pouze z \num{28} karet (esa je nutno 
    odstranit, aby bylo zaručeno, že „doplňkové“ karty budou mít jinou hodnotu než eso). Výběr 
    zbývajících karet lze učinit \(\begin{pmatrix} 28 \\ 10 - k\end{pmatrix}\) způsoby. Nakonec 
    tedy dostáváme
    \begin{align*}
      M(B) &=  \begin{pmatrix} 4 \\ 2\end{pmatrix}\begin{pmatrix} 28 \\ 8\end{pmatrix}
              +\begin{pmatrix} 4 \\ 3\end{pmatrix}\begin{pmatrix} 28 \\ 7\end{pmatrix}
              +\begin{pmatrix} 4 \\ 4\end{pmatrix}\begin{pmatrix} 28 \\ 6\end{pmatrix}         \\
           &=  6\cdot\dfrac{28\cdot27\cdots22\cdot21}{8\cdot7\cdots2\cdot1}
              +4\cdot\dfrac{28\cdot27\cdots23\cdot22}{7\cdot6\cdots2\cdot1}
              +1\cdot\dfrac{28\cdot27\cdots24\cdot23}{6\cdot5\cdots2\cdot1} = \num{23761530},  \\
      P(B) &= \dfrac{M(B)}{N} = \dfrac{\num{23761530}}{\num{64512240}} \simeq\num{0.37}.
    \end{align*}
    Pozn.: Někomu se předchozí výpočet může zdát příliš složitý. Nelze jej nějak zjednodušit? Co 
    kdybychom uvažovali třeba takto: Výběr dvou es již zajistí splnění požadavku. Doplňkové karty 
    tedy již pak můžeme vybírat ze třiceti karet - nebudeme tedy odstraňovat esa, protože budou-li 
    vybrána mezi doplňkovými kartami, požadavek „alespoň dvou es ve výběru“ to nenaruší. Při takové 
    interpretaci bychom dostali
    \begin{equation*}
      M(B) = \begin{pmatrix} 4 \\ 2\end{pmatrix}\begin{pmatrix} 30 \\ 8\end{pmatrix}   
           = 6\cdot\dfrac{30\cdot29\cdots24\cdot23}{8\cdot7\cdots2\cdot1}
           = \num{35117550}.
    \end{equation*}
    Vidíme, že vyšlo číslo vyšší než při předchozí úvaze. Co je tedy správně? Správně je první 
    úvaha vedoucí k nižšímu počtu příznivých případů. Při druhé úvaze jsme některé případy 
    započetli vícekrát. Zkuste přijít na to, jak se to stalo. V každém případě vidíme, že 
    kombinatorické úvahy, ať již vypadají jakkoli jednoduše, mohou být zrádné a je třeba dát si na 
    ně pozor. 
    
    Jev \(C\) podle zadání nastane, obsahuje-li náhodný výběr deseti karet nejvýše dvě esa. Znamená 
    to, že výběr je příznivý, neobsahuje-li žádné eso a obsahuje deset karet jiné hodnoty, nebo 
    obsahuje-li jedno eso a devět karet jiné hodnoty, nebo obsahuje-li dvě esa a osm karet jiné 
    hodnoty. Počet \(M(C)\) určíme analogicky jako \(M(B)\), ale pro \(k= 0, 1, 2\):
    \begin{align*}
      M(C) &=  \begin{pmatrix} 4 \\ 0\end{pmatrix}\begin{pmatrix} 28 \\ 10\end{pmatrix}
              +\begin{pmatrix} 4 \\ 1\end{pmatrix}\begin{pmatrix} 28 \\ 9\end{pmatrix}
              +\begin{pmatrix} 4 \\ 2\end{pmatrix}\begin{pmatrix} 28 \\ 8\end{pmatrix}         \\
           &=   \cdot\dfrac{28\cdot27\cdots20\cdot19}{10\cdot9\cdots2\cdot1}
              +4\cdot\dfrac{28\cdot27\cdots21\cdot20}{ 9\cdot8\cdots2\cdot1}
              +6\cdot\dfrac{28\cdot27\cdots22\cdot21}{ 8\cdot7\cdots2\cdot1} = \num{59399340}, \\
      P(C) &= \dfrac{M(C)}{N} = \dfrac{\num{59399340}}{\num{64512240}} \simeq\num{0.92}.
    \end{align*}
    
    Jev \(D\) znamená alespoň šest karet stejné barvy (připomeňme, že barvou rozumíme jednu z 
    možností \(\heartsuit\), \(\lozenge\), \(\spadesuit\), \(\clubsuit\)). Hra obsahuje osm karet 
    od každé barvy. Současně je tedy zřejmé, že karet stejné barvy může být ve výběru nejvýše osm. 
    Výběr je příznivý pro \(k = 6, 7, 8\). Obdobnou úvahou jako v předchozích případech dostáváme
    \begin{equation*}
      M = 4\cdot\sum^{8}_{k=6}
          \begin{pmatrix} 8 \\ k \end{pmatrix}\begin{pmatrix} 32 - 8 \\ 10 - k\end{pmatrix}
        = \num{1255984}.
    \end{equation*}
    
    Faktor \num{4} před celou sumou se objevuje proto, že nebylo specifikováno, která ze čtyř barev 
    má být zastoupena alespoň šesti kartami. Všechny čtyři možnosti volby barvy jsou tedy příznivé. 
    Pravděpodobnost jevu \(D\) je
    \begin{equation*}
      p(D)  = \dfrac{M(D)}{N}
            = \dfrac{4\cdot\sum^{8}_{k=6}\dfrac{8!}{k!(8-k)!}\dfrac{24!}{(10-k)!(14+k)!}}
                    {\begin{pmatrix} 32 \\ 10 \end{pmatrix}}
            = \dfrac{\num{1255984}}{\num{64512240}} \simeq \num{0.019}.
    \end{equation*}
    
    Případy (\(E\)) a (\(F\)) v zadání se liší pouze slůvkem „a“ a „nebo“. Uvidíme, že nejde o 
    slovíčka, ale o podstatný rozdíl. 
    
    Aby nastal jev \(E\), požadujeme, aby náhodný výběr deseti karet obsahoval právě dvě dámy a 
    alespoň jednoho kluka. Znamená to, že výběr je příznivý, obsahuje-li dvě dámy libovolné barvy a 
    současně alespoň jednoho kluka libovolné barvy. Příznivé možnosti tedy jsou:
    \begin{enumerate}
    \item  dvě dámy libovolné barvy, jeden kluk libovolné barvy, \num{7} libovolných karet, které 
           nemají hodnotu dámy ani kluka, celkem 
           \(\begin{pmatrix} 4 \\ 2 \end{pmatrix}
             \begin{pmatrix} 4 \\ 1\end{pmatrix}
             \begin{pmatrix} 32-2\cdot4 \\ 7 \end{pmatrix} = \num{8306496}\) možností,
    \item dvě dámy libovolné barvy, dva kluci libovolné barvy, \num{6} libovolných karet, které 
          nemají hodnotu dámy ani kluka, celkem 
          \(\begin{pmatrix} 4 \\ 2 \end{pmatrix}
            \begin{pmatrix} 4 \\ 2\end{pmatrix}
            \begin{pmatrix} 32-2\cdot4 \\ 6 \end{pmatrix} = \num{4845456}\) možností,
    \item dvě dámy libovolné barvy, tři kluci libovolné barvy, \num{5} libovolných karet, které 
          nemají hodnotu dámy ani kluka, celkem 
          \(\begin{pmatrix} 4 \\ 2 \end{pmatrix}
            \begin{pmatrix} 4 \\ 3\end{pmatrix}
            \begin{pmatrix} 32-2\cdot4 \\ 5 \end{pmatrix} = \num{1020096}\) možností,
    \item dvě dámy libovolné barvy, všichni čtyři kluci, \num{4} libovolné karty, které nemají 
          hodnotu dámy ani kluka, celkem 
          \(\begin{pmatrix} 4 \\ 2 \end{pmatrix}
            \begin{pmatrix} 4 \\ 4\end{pmatrix}
            \begin{pmatrix} 32-2\cdot4 \\ 4 \end{pmatrix} = \num{63756}\) možností.
    \end{enumerate}
    \begin{align*}
      M(E) &= \begin{pmatrix} 4 \\ 2 \end{pmatrix}\cdot\sum^{4}_{k=1}
              \begin{pmatrix} 4 \\ k \end{pmatrix}\begin{pmatrix} 24 \\ 8 - k \end{pmatrix}
            = 6\left[ 
                  4\begin{pmatrix} 24 \\ 7 \end{pmatrix} +
                  6\begin{pmatrix} 24 \\ 6 \end{pmatrix} +
                  4\begin{pmatrix} 24 \\ 5 \end{pmatrix} +
                   \begin{pmatrix} 24 \\ 4 \end{pmatrix}
                \right]                                                       \\ 
      M(E) &= \num{14235804},                                                 \\
      p(E) &= \dfrac{M(E)}{N} = \dfrac{\num{14235804}}{\num{64512240}} \simeq \num{0.22}.
    \end{align*}
    
    Aby nastal jev \(F\), požadujeme, aby náhodný výběr deseti karet obsahoval právě dvě dámy nebo 
    alespoň jednoho kluka. Znamená to, že výběr je příznivý, obsahuje-li dvě dámy libovolné barvy a 
    jakékoli další karty, nebo obsahuje alespoň jednoho kluka a jakékoli další karty. Nyní je nutno 
    o všech možnostech pečlivě rozvažovat, abychom některé nezapočítali vícekrát. Pozor, slůvko 
    \uv{nebo} zde nemá vylučovací význam, připouští se, že mohou být splněny obě podmínky jevu 
    \(F\), tj. jak právě dvě dámy, tak alespoň jeden kluk. Příznivé možnosti jsou
    \begin{enumerate}
      \item dvě dámy libovolné barvy, žádný kluk, \num{8} libovolných karet, které nemají 
            hodnotu dámy ani kluka, celkem
            \begin{equation*}
              \begin{pmatrix} 4  \\ 2 \end{pmatrix}
              \begin{pmatrix} 4  \\ 0 \end{pmatrix}
              \begin{pmatrix} 32 - 2\cdot4 \\ 8 \end{pmatrix} = 6
              \begin{pmatrix} 24 \\ 8 \end{pmatrix},
            \end{equation*}
      \item žádná dáma, \(k\) kluků libovolné barvy pro \(k = 1, 2, 3, 4\) (alespoň jeden kluk), 
            \(10 - k\) karet, které nemají hodnotu dám y ani kluka, celkem
            \begin{equation*}
              \begin{pmatrix} 4  \\ 0 \end{pmatrix}
              \sum^{4}_{k=1}\begin{pmatrix} 4  \\ k \end{pmatrix}
                            \begin{pmatrix} 32 - 2\cdot4 \\ 10 - k \end{pmatrix} =
              \sum^{4}_{k=1}\begin{pmatrix} 4  \\ k \end{pmatrix}
                            \begin{pmatrix} 24 \\ 10 - k \end{pmatrix},
            \end{equation*}
      \item jedna dáma libovolné barvy, \(k\) kluků libovolné barvy pro \(k = 1,2, 3, 4\) (alespoň 
            jeden kluk), \(10 - k - 1\) karet, které nemají hodnotu dámy ani kluka, celkem
            \begin{equation*}
              \begin{pmatrix} 4  \\ 1 \end{pmatrix}
              \sum^{4}_{k=1}\begin{pmatrix} 4  \\ k \end{pmatrix}
                            \begin{pmatrix} 32 - 2\cdot4 \\ 10 - k - 1 \end{pmatrix} =
              4\sum^{4}_{k=1}\begin{pmatrix} 4  \\ k \end{pmatrix}
                            \begin{pmatrix} 24 \\ 9 - k \end{pmatrix},
            \end{equation*}
      \item dvě dámy libovolné barvy, \(k\) kluků libovolné barvy pro \(k = 1, 2, 3, 4\) (alespoň 
            jeden kluk), \(10 - 2 - k = 8 - k\) karet, které nemají hodnotu dámy ani kluka, celkem
            \begin{equation*}
              \begin{pmatrix} 4  \\ 2 \end{pmatrix}
              \sum^{4}_{k=1}\begin{pmatrix} 4  \\ k \end{pmatrix}
                            \begin{pmatrix} 32 - 2\cdot4 \\ 10 - k - 2 \end{pmatrix} =
              6\sum^{4}_{k=1}\begin{pmatrix} 4  \\ k \end{pmatrix}
                            \begin{pmatrix} 24 \\ 8 - k \end{pmatrix},
            \end{equation*}
      \item tři dámy libovolné barvy, k kluků libovolné barvy pro \(k = 1, 2, 3, 4\) (alespoň jeden 
            kluk), \(10 - k - 3\) karet, které nemají hodnotu dámy ani kluka, celkem
            \begin{equation*}
              \begin{pmatrix} 4  \\ 3 \end{pmatrix}
              \sum^{4}_{k=1}\begin{pmatrix} 4  \\ k \end{pmatrix}
                            \begin{pmatrix} 32 - 2\cdot4 \\ 10 - k - 3 \end{pmatrix} =
              4\sum^{4}_{k=1}\begin{pmatrix} 4  \\ k \end{pmatrix}
                            \begin{pmatrix} 24 \\ 7 - k \end{pmatrix},
            \end{equation*}
      \item všechny čtyři dámy, k kluků libovolné barvy pro \(k = 1, 2, 3, 4\) (alespoň jeden 
            kluk), \(10 - k - 4\) karet, které nemají hodnotu dámy ani kluka, celkem
            \begin{equation*}
              \begin{pmatrix} 4  \\ 4 \end{pmatrix}
              \sum^{4}_{k=1}\begin{pmatrix} 4  \\ k \end{pmatrix}
                            \begin{pmatrix} 32 - 2\cdot4 \\ 10 - k - 4 \end{pmatrix} =
              \sum^{4}_{k=1}\begin{pmatrix} 4  \\ k \end{pmatrix}
                            \begin{pmatrix} 24 \\ 6 - k \end{pmatrix},
            \end{equation*}
    \end{enumerate}
    
    Počet příznivých případů \(M(F)\) je dán součtem všech těchto možností, tedy
    \begin{equation*}
      M(F) = \begin{pmatrix}  4 \\ 2 \end{pmatrix}\begin{pmatrix} 4  \\ 0 \end{pmatrix}
             \begin{pmatrix} 24 \\ 8 \end{pmatrix} + 
              \sum^{4}_{s=0}\begin{pmatrix} 4  \\ s \end{pmatrix}
              \left[\sum^{4}_{k=1}\begin{pmatrix}  4 \\ k \end{pmatrix}
                    \begin{pmatrix} 24 \\ 10 - k - s \end{pmatrix}
              \right].
    \end{equation*}
    Všimněme si nyní výsledku. Výraz s dvojitou sumou můžeme přepsat jak
    \begin{equation*}
      \sum^{4}_{k=1}\begin{pmatrix} 4  \\ k \end{pmatrix}
        \left[\sum^{4}_{s=0}\begin{pmatrix}  4 \\ s \end{pmatrix}
              \begin{pmatrix} 24 \\ 10 - k - s \end{pmatrix}
        \right].
    \end{equation*}
    V učebnicích můžeme najít různé vzorce pro kombinační čísla, mezi nimi i vzorec
    \begin{equation}\label{mai:eq076}
      \sum^{p}_{s=0}\begin{pmatrix} p \\ s \end{pmatrix}\begin{pmatrix} r \\ q - s \end{pmatrix}
        = \begin{pmatrix} r + p \\ q \end{pmatrix} \qquad\text{pro}\qquad r\geq q,\,q \geq p.
    \end{equation}
    (Nebo si jej můžeme sami odvodit — pokuste se o to!) Pro \(p = 4\), \(r = 24\), \(q = 10 - k\), 
    \(1 \leq k \leq 4\) máme právě náš případ, takže
    \begin{equation*}
      \sum^{4}_{k=1}\begin{pmatrix} 4  \\ k \end{pmatrix}
        \left[\sum^{4}_{s=0}\begin{pmatrix}  4 \\ s \end{pmatrix}
              \begin{pmatrix} 24 \\ 10 - k - s \end{pmatrix}
        \right] = 
        \sum^{4}_{k=1}\begin{pmatrix}  4 \\ k \end{pmatrix}
                      \begin{pmatrix} 28 \\ 10 - k \end{pmatrix}.
    \end{equation*}
    Jak můžeme tento výsledek interpretovat? Jedná se o počet případů, kdy náhodný výběr deseti 
    karet z mariášové hry dvaatřiceti karet obsahuje alespoň jednu kartu pevně zvolené hodnoty (v 
    našem případě kluka), bez ohledu na to, kolik obsahuje karet ostatních hodnot. Přidáme-li počet 
    případů, kdy výběr neobsahuje žádného kluka a právě dvě dámy, dostaneme právě počet případů 
    příznivých pro jev \(F\). Pň úpravě použijeme ještě jednou vzorce
    \begin{equation*}
      \sum^{4}_{k=1}\begin{pmatrix} 4 \\ k \end{pmatrix}\begin{pmatrix} 28 \\ 10 - k\end{pmatrix}
        = \sum^{4}_{k=0}\begin{pmatrix}  4 \\ k     \end{pmatrix}
                        \begin{pmatrix} 28 \\ 10 - k\end{pmatrix} -
                        \begin{pmatrix}  4 \\ 0     \end{pmatrix}
                        \begin{pmatrix} 28 \\ 10    \end{pmatrix} =
                        \begin{pmatrix} 32 \\ 10    \end{pmatrix} -
                        \begin{pmatrix} 28 \\ 10    \end{pmatrix},
    \end{equation*}
    kde podle vzorce \ref{mai:eq076} je \(p=4\), \(q=10\), \(s=k\) a \(r=28\).  
    \begin{align*}
      M(F) &= \begin{pmatrix}  4 \\ 2 \end{pmatrix}\begin{pmatrix}  4 \\ 0 \end{pmatrix} 
              \begin{pmatrix} 24 \\ 8 \end{pmatrix} + 
              \sum^{4}_{k=1}\begin{pmatrix} 4  \\ k \end{pmatrix}
                      \left[\sum^{4}_{s=0}\begin{pmatrix}  4 \\ s \end{pmatrix}
                            \begin{pmatrix} 24 \\ 10 - k - s \end{pmatrix}
                      \right]                                                                   \\
           &= \begin{pmatrix}  4 \\ 2 \end{pmatrix}\begin{pmatrix} 24 \\ 8 \end{pmatrix} +
              \sum^{4}_{k=1}\begin{pmatrix}  4 \\ k \end{pmatrix}
                            \begin{pmatrix} 28 \\ 10 - k \end{pmatrix}                          \\
           &= \begin{pmatrix}  4 \\ 2 \end{pmatrix}\begin{pmatrix} 24 \\ 8 \end{pmatrix} + 
              \left[\begin{pmatrix}  32 \\ 10 \end{pmatrix} - 
                    \begin{pmatrix}  28 \\ 10 \end{pmatrix}
              \right]
            = \begin{pmatrix} 32 \\ 10 \end{pmatrix} - 
              \left[\begin{pmatrix} 28 \\ 10 \end{pmatrix} - 
                    \begin{pmatrix}  4 \\  2 \end{pmatrix}
                    \begin{pmatrix} 24 \\  8 \end{pmatrix}
              \right],                                                                          \\
      p(F) &= \dfrac{M(F)}{N}
            = 1 - \dfrac{\begin{pmatrix} 28 \\ 10 \end{pmatrix} -
                         \begin{pmatrix}  4 \\  2 \end{pmatrix}
                         \begin{pmatrix} 24 \\  8 \end{pmatrix}
                        }
                        {\begin{pmatrix} 32 \\ 10 \end{pmatrix}}
            = 1 - \dfrac{\num{8710284}}{\num{64512240}} \simeq \num{0.86}.
    \end{align*}
    Zamysleme se ještě nad interpretací posledního výrazu pro \(M(F)\). Od počtu všech možných 
    případů se odečítá hodnota \(\begin{pmatrix} 28 \\ 10 \end{pmatrix}\) představující počet 
    situací, kdy ve výběru nebude žádný kluk, zmenšená o hodnotu \(\begin{pmatrix} 4 \\ 2 
    \end{pmatrix}\begin{pmatrix} 24 \\ 8 \end{pmatrix}\), která představuje počet situací, kdy ve 
    výběru budou právě dvě dámy a žádný kluk.
  \normalsize
\end{example}