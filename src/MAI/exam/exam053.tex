% !TeX spellcheck = cs_CZ
\begin{mdframed}[style=mdexam]
  \begin{example}\label{mai:exam053}
    \textbf{Losování karet}\newline
      Máme karetní hru mariáš, která obsahuje celkem \num{32} karet osmi hodnot \num{7}, \num{8},
      \num{9}, \num{10}, J (kluk), Q (dáma), K (král), A (eso), každá hodnota je ve čtyřech barvách,
      červené barvy jsou \(\heartsuit\) (srdce) a \(\lozenge\) (kára), černé barvy jsou
      \(\spadesuit\) (piky) a \(\clubsuit\) (kříže). Jaká je pravděpodobnost, že při náhodném
      vylosování deseti karet budou mezi nimi:
      \begin{enumerate}[leftmargin=1cm,rightmargin=1cm, label=\emph{\Alph*}),noitemsep]
        \item právě dvě esa,
        \item alespoň dvě esa,
        \item nejvýše dvě esa,
        \item alespoň šest karet stejné barvy,
        \item právě dvě dámy a alespoň jeden kluk,
        \item právě dvě dámy nebo alespoň jeden kluk?
      \end{enumerate}

      Písmena (A) až (F) představují různé části úlohy a také zároveň definují jevy, jejichž
      pravděpodobnost počítáme. Jedná se opět o kombinace. Nezáleží totiž na pořadí, v jakém karty
      vytahujeme. Důležité je jen to, zda jsou vyjmenované karty ve výběru obsaženy. Počet možných
      výsledků náhodného vylosování deseti karet z dvaatřiceti, tj. počet případů možných, je pro
      všechny části úlohy stejný,
      \begin{align*}
        N&=\binom{32}{10} = \dfrac{32!}{(32-10)!\cdot10!}                                   \\
		 &= \dfrac{32\cdot31\cdots24\cdot23}{10\cdot9\cdots2\cdot1} = \num{64512240}.
      \end{align*}
      
      Počítejme nyní případy příznivé pro jednotlivé jevy \(A\) až \(F\) a pravděpodobnosti těchto
      jevů:
      \begin{align*}
        M(A) &= \binom{4}{2}\binom{32 - 4}{10 - 2}
              = \binom{4}{2}\binom{28}{8}                                     \\
             &= 6\cdot\dfrac{28\cdot27\cdots21}{8\cdot7\cdots2\cdot1} = \num{18648630}.
      \end{align*}
      Jak jsme k tomuto výsledku došli? Příznivý pro daný jev je každý výběr, v němž jsou obsažena
      právě dvě esa (libovolných barev) a žádná další esa (význam slova „právě“). Počet výběrů dvou
      es z celkového počtu čtyř es je \(C_2(4)\), počet výběrů dalších libovolných osmi karet ze
      zbývající části hry, která vznikne po odstranění es (nechceme, aby v příznivém výběru byla
      další esa), je \(C_{(10-2)}(32 - 4) = C_8(28)\). Každý výběr dvojice es lze kombinovat s
      každým výběrem zbývajících osmi karet ze zbytku hry, tj. \(M(A) = C_2(4)\cdot C_8(28)\). A to
      je právě náš předchozí výsledek. Potom:
      \begin{equation*}
        p(A) = \dfrac{M(A)}{N} 
             = \dfrac{\binom{4}{2}\binom{28}{8}}{\binom{32}{10}}
             = \dfrac{\num{18648630}}{\num{64512240}} \simeq \num{0.29}
      \end{equation*}
      
      Aby nastal jev \(B\), požadujeme, aby v náhodném výběru deseti karet z dvaatřiceti byla
      alespoň dvě esa. To znamená, že výběr považujeme za příznivý, obsahuje-li dvě esa libovolné
      barvy a osm libovolných karet jiné hodnoty, nebo obsahuje tři esa libovolné barvy a sedm
      libovolných karet jiné hodnoty, nebo obsahuje všechna čtyři esa a šest libovolných karet jiné
      hodnoty, \(k\) es (pro \(k = 2, 3, 4\)) můžeme ze čtyř es vybrat \(\binom{4}{k}\) způsoby.
      \(10 - k\) karet jiné hodnoty pak musíme vybírat pouze z \num{28} karet (esa je nutno
      odstranit, aby bylo zaručeno, že „doplňkové“ karty budou mít jinou hodnotu než eso). Výběr
      zbývajících karet lze učinit \(\binom{28}{10-k}\) způsoby. Nakonec
      tedy dostáváme
      \begin{gather*}
        \begin{align*}
          M(B) &= \binom{4}{2}\binom{28}{8} +\binom{4}{3}\binom{28}{7} +\binom{4}{4}\binom{28}{6} \\
               &=  6\cdot\dfrac{28\cdot27\cdots22\cdot21}{8\cdot7\cdots2\cdot1}
                  +4\cdot\dfrac{28\cdot27\cdots23\cdot22}{7\cdot6\cdots2\cdot1}+                  \\
               &  +1\cdot\dfrac{28\cdot27\cdots24\cdot23}{6\cdot5\cdots2\cdot1} = \num{23761530}, \\
          P(B) &= \dfrac{M(B)}{N} = \dfrac{\num{23761530}}{\num{64512240}} \simeq\num{0.37}.
        \end{align*}
      \end{gather*}
      Pozn.: Někomu se předchozí výpočet může zdát příliš složitý. Nelze jej nějak zjednodušit? Co
      kdybychom uvažovali třeba takto: Výběr dvou es již zajistí splnění požadavku. Doplňkové karty
      tedy již pak můžeme vybírat ze třiceti karet - nebudeme tedy odstraňovat esa, protože budou-li
      vybrána mezi doplňkovými kartami, požadavek „alespoň dvou es ve výběru“ to nenaruší. Při
      takové interpretaci bychom dostali
      \begin{align*}
        M(B) &= \binom{4}{2}\binom{30}{8}   
              = 6\cdot\dfrac{30\cdot29\cdots24\cdot23}{8\cdot7\cdots2\cdot1}     \\
             &= \num{35117550}.
      \end{align*}
      Vidíme, že vyšlo číslo vyšší než při předchozí úvaze. Co je tedy správně? Správně je první
      úvaha vedoucí k nižšímu počtu příznivých případů. Při druhé úvaze jsme některé případy
      započetli vícekrát. Zkuste přijít na to, jak se to stalo. V každém případě vidíme, že
      kombinatorické úvahy, ať již vypadají jakkoli jednoduše, mohou být zrádné a je třeba dát si na
      ně pozor. 
      
      Jev \(C\) podle zadání nastane, obsahuje-li náhodný výběr deseti karet nejvýše dvě esa.
      Znamená to, že výběr je příznivý, neobsahuje-li žádné eso a obsahuje deset karet jiné hodnoty,
      nebo obsahuje-li jedno eso a devět karet jiné hodnoty, nebo obsahuje-li dvě esa a osm karet
      jiné hodnoty. Počet \(M(C)\) určíme analogicky jako \(M(B)\), ale pro \(k= 0, 1, 2\):
      \begin{align*}
        M(C) &= \binom{4}{0}\binom{28}{10} + \binom{4}{1}\binom{28}{9} +\binom{4}{2}\binom{28}{8} \\
             &=        \dfrac{28\cdot27\cdots20\cdot19}{10\cdot9\cdots2\cdot1}
                +4\cdot\dfrac{28\cdot27\cdots21\cdot20}{ 9\cdot8\cdots2\cdot1}                    \\
             &  +6\cdot\dfrac{28\cdot27\cdots22\cdot21}{ 8\cdot7\cdots2\cdot1} = \num{59399340},  \\
        P(C) &= \dfrac{M(C)}{N} = \dfrac{\num{59399340}}{\num{64512240}} \simeq\num{0.92}.
      \end{align*}
      
      Jev \(D\) znamená alespoň šest karet stejné barvy (připomeňme, že barvou rozumíme jednu z
      možností \(\heartsuit\), \(\lozenge\), \(\spadesuit\), \(\clubsuit\)). Hra obsahuje osm karet
      od každé barvy. Současně je tedy zřejmé, že karet stejné barvy může být ve výběru nejvýše osm.
      Výběr je příznivý pro \(k = 6, 7, 8\). Obdobnou úvahou jako v předchozích případech dostáváme
      \begin{align*}
        M(D) &= 4\cdot\sum^{8}_{k=6}\binom{8}{k}\binom{32 - 8}{10 - k}                    \\
             &= 4\cdot\sum^{8}_{k=6}\dfrac{8!}{(8-k)!k!}\cdot\dfrac{24!}{(14+k)!(10-k)!}  \\
             &= \num{1255984}.
      \end{align*}
      
      Faktor \num{4} před celou sumou se objevuje proto, že nebylo specifikováno, která ze čtyř
      barev má být zastoupena alespoň šesti kartami. Všechny čtyři možnosti volby barvy jsou tedy
      příznivé. Pravděpodobnost jevu \(D\) je
      \begin{equation*}
          p(D) = \dfrac{M(D)}{N} = \dfrac{\num{1255984}}{\num{64512240}} \simeq \num{0.019}.
      \end{equation*}
      Případy (\(E\)) a (\(F\)) v zadání se liší pouze slůvkem „a“ a „nebo“. Uvidíme, že nejde o
      slovíčka, ale o podstatný rozdíl. 
      
      Aby nastal jev \(E\), požadujeme, aby náhodný výběr deseti karet obsahoval právě dvě dámy a
      alespoň jednoho kluka. Znamená to, že výběr je příznivý, obsahuje-li dvě dámy libovolné barvy
      a současně alespoň jednoho kluka libovolné barvy. Příznivé možnosti tedy jsou:
      \begin{enumerate}[leftmargin=10pt,noitemsep]
        \item  dvě dámy libovolné barvy, jeden kluk libovolné barvy, \num{7}
              libovolných karet, které nemají hodnotu dámy ani kluka, celkem 
              \(\binom{4}{2}\binom{4}{1}\binom{32-2\cdot4}{7} = \num{8306496}\) možností,
        \item dvě dámy libovolné barvy, dva kluci libovolné barvy, \num{6}
              libovolných karet, které nemají hodnotu dámy ani kluka, celkem 
              \(\binom{4}{2}\binom{4}{2}\binom{32-2\cdot4}{6} = \num{4845456}\) možností,
        \item dvě dámy libovolné barvy, tři kluci libovolné barvy, \num{5}
              libovolných karet, které nemají hodnotu dámy ani kluka, celkem 
              \(\binom{4}{2}\binom{4}{3}\binom{32-2\cdot4}{5} = \num{1020096}\) možností,
        \item dvě dámy libovolné barvy, všichni čtyři kluci, \num{4} libovolné karty, které nemají
              hodnotu dámy ani kluka, celkem 
              \(\binom{4}{2}\binom{4}{4}\binom{32-2\cdot4}{4} = \num{63756}\) možností.
      \end{enumerate}
      \begin{gather*}     
        \begin{align*}
          M(E) &= \binom{4}{2}\cdot\sum^{4}_{k=1}\binom{4}{k}\binom{24}{8 - k}                    \\
               &= 6\left[4\binom{24}{7} + 6\binom{24}{6} + 4\binom{24}{5} + \binom{24}{4}\right]  \\ 
          M(E) &= \num{14235804}, 
        \end{align*}
      \end{gather*}
      Pravděpodobnost jevu \(E\) je
      \begin{equation*}
        p(E) = \dfrac{M(E)}{N} = \dfrac{\num{14235804}}{\num{64512240}} \simeq \num{0.22}.
      \end{equation*}

      Aby nastal jev \(F\), požadujeme, aby náhodný výběr deseti karet obsahoval právě dvě dámy nebo
      alespoň jednoho kluka. Znamená to, že výběr je příznivý, obsahuje-li dvě dámy libovolné barvy
      a jakékoli další karty, nebo obsahuje alespoň jednoho kluka a jakékoli další karty. Nyní je
      nutno o všech možnostech pečlivě rozvažovat, abychom některé nezapočítali vícekrát. Pozor,
      slůvko \uv{nebo} zde nemá vylučovací význam, připouští se, že mohou být splněny obě podmínky
      jevu \(F\), tj. jak právě dvě dámy, tak alespoň jeden kluk. Příznivé možnosti jsou:
      \begin{enumerate}[leftmargin=10pt,noitemsep]
        \item dvě dámy libovolné barvy, žádný kluk, \num{8} libovolných karet, které nemají hodnotu
              dámy ani kluka, celkem
              \begin{gather*}
                \binom{4}{2}\binom{4}{0}\binom{32 - 2\cdot4}{8} = 6\binom{24}{8},
              \end{gather*}
        \item žádná dáma, \(k\) kluků libovolné barvy pro \(k = 1, 2, 3, 4\) (alespoň jeden kluk),
              \(10 - k\) karet, které nemají hodnotu dám y ani kluka, celkem
              \begin{gather*}
                \binom{4}{0}\sum^{4}_{k=1}\binom{4}{k}\binom{32 - 2\cdot4}{10 - k} =
                            \sum^{4}_{k=1}\binom{4}{k}\binom{24}{10 - k},
              \end{gather*}
        \item jedna dáma libovolné barvy, \(k\) kluků libovolné barvy pro \(k = 1,2, 3, 4\) (alespoň
              jeden kluk), \(10 - k - 1\) karet, které nemají hodnotu dámy ani kluka, celkem
              \begin{gather*}
                \binom{4}{1}\sum^{4}_{k=1}\binom{4}{k}\binom{32 - 2\cdot4}{10 - k - 1} =
                           4\sum^{4}_{k=1}\binom{4}{k}\binom{24}{9 - k},
              \end{gather*}
        \item dvě dámy libovolné barvy, \(k\) kluků libovolné barvy pro \(k = 1, 2, 3, 4\) (alespoň
              jeden kluk), \(10 - 2 - k = 8 - k\) karet, které nemají hodnotu dámy ani kluka, celkem
              \begin{gather*}
                \binom{4}{2}\sum^{4}_{k=1}\binom{4}{k}\binom{32 - 2\cdot4}{10 - k - 2} =
                           6\sum^{4}_{k=1}\binom{4}{k}\binom{24}{8 - k},
              \end{gather*}
        \item tři dámy libovolné barvy, k kluků libovolné barvy pro \(k = 1, 2, 3, 4\) (alespoň
              jeden kluk), \(10 - k - 3\) karet, které nemají hodnotu dámy ani kluka, celkem
              \begin{gather*}
                \binom{4}{3}\sum^{4}_{k=1}\binom{4}{k}\binom{32 - 2\cdot4}{10 - k - 3} =
                           4\sum^{4}_{k=1}\binom{4}{k}\binom{24}{7 - k},
              \end{gather*}
        \item všechny čtyři dámy, k kluků libovolné barvy pro \(k = 1, 2, 3, 4\) (alespoň jeden
              kluk), \(10 - k - 4\) karet, které nemají hodnotu dámy ani kluka, celkem
              \begin{gather*}
                \binom{4}{4}\sum^{4}_{k=1}\binom{4}{k}\binom{32 - 2\cdot4}{10 - k - 4} =
                            \sum^{4}_{k=1}\binom{4}{k}\binom{24}{6 - k},
              \end{gather*}
      \end{enumerate}
      
      Počet příznivých případů \(M(F)\) je dán součtem všech těchto možností, tedy
      \begin{gather*}
        \binom{4}{2}\binom{4}{0}\binom{24}{8} + \sum^{4}_{s=0}\binom{4}{s}
        \left[\sum^{4}_{k=1}\binom{4}{k}\binom{24}{10 - k - s}\right].
      \end{gather*}
      Všimněme si nyní výsledku. Výraz s dvojitou sumou můžeme přepsat jak
      \begin{equation*}
        \sum^{4}_{k=1}\binom{4}{k}\left[\sum^{4}_{s=0}\binom{4}{s}\binom{24}{10 - k - s}
          \right].
      \end{equation*}
      V učebnicích můžeme najít různé vzorce pro kombinační čísla, mezi nimi i vzorec
      \begin{mdframed}[style=highlight]
        \begin{equation}\label{mai:eq084}
          \sum^{p}_{s=0}\binom{p}{s}\binom{r}{q - s} = \binom{r + p}{q} 
          \quad\text{pro } r\geq q,\,q \geq p.
        \end{equation}
      \end{mdframed}
      (Nebo si jej můžeme sami odvodit - pokuste se o to!) Pro \(p = 4\), \(r = 24\), \(q = 10 -
      k\), \(1 \leq k \leq 4\) máme právě náš případ, takže
      \begin{gather*} 
        \begin{align*}
          \sum^{4}_{k=1}\binom{4}{k}\left[\sum^{4}_{s=0}\binom{4}{s}\binom{24}{10 - k - s}\right] = 
          \sum^{4}_{k=1}\binom{4}{k}\binom{28}{10 - k}.
        \end{align*}
      \end{gather*}
      Jak můžeme tento výsledek interpretovat? Jedná se o počet případů, kdy náhodný výběr deseti
      karet z mariášové hry dvaatřiceti karet obsahuje alespoň jednu kartu pevně zvolené hodnoty (v
      našem případě kluka), bez ohledu na to, kolik obsahuje karet ostatních hodnot. Přidáme-li
      počet případů, kdy výběr neobsahuje žádného kluka a právě dvě dámy, dostaneme právě počet
      případů příznivých pro jev \(F\). Pň úpravě použijeme ještě jednou vzorce
      \begin{gather*} 
        \begin{align*}
        \sum^{4}_{k=1}\binom{4}{k}\binom{28}{10 - k}  &= 
        \sum^{4}_{k=0}\binom{4}{k}\binom{28}{10 - k} - \binom{4}{0}\binom{28}{10}                 \\
                                                      &=\binom{32}{10} - \binom{28}{10},
      \end{align*}
      \end{gather*}
      kde podle vzorce \ref{mai:eq084} je \(p=4\), \(q=10\), \(s=k\) a \(r=28\).   
      \begin{gather*} 
      \begin{align*}
        M(F)&= \binom{4}{2}\binom{4}{0}\binom{24}{8}                                             \\
            &+ \sum^{4}_{k=1}\binom{4}{k}
               \left[\sum^{4}_{s=0}\binom{4}{s}\binom{24}{10 - k - s}\right]                     \\
            &= \binom{4}{2}\binom{24}{8} + \sum^{4}_{k=1}\binom{4}{k}\binom{28}{10 - k}          \\
            &= \binom{ 4}{2}\binom{24}{8} + \left[\binom{32}{10} - \binom{28}{10}\right]         \\
            &= \binom{32}{10} - \left[\binom{28}{10} - \binom{4}{2}\binom{24}{8}\right],         \\
        p(F)&= \dfrac{M(F)}{N}=1-\dfrac{\binom{28}{10}-\binom{4}{2}\binom{24}{8}}{\binom{32}{10}}\\
            &= 1 - \dfrac{\num{8710284}}{\num{64512240}} \simeq \num{0.86}.
      \end{align*}
      \end{gather*}  
      Zamysleme se ještě nad interpretací posledního výrazu pro \(M(F)\). Od počtu všech možných
      případů se odečítá hodnota \(\binom{28}{10}\) představující počet situací, kdy ve výběru
      nebude žádný kluk, zmenšená o hodnotu \(\binom{4}{2}\binom{24}{8}\), která představuje počet
      situací, kdy ve výběru budou právě dvě dámy a žádný kluk.
  \end{example}
\end{mdframed}  