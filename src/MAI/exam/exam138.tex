\begin{mathexam}{Vypočtěte neurčité integrály \(\protect\scalerel{\int}{\dfrac{2}{x+5}\dd{x}}\),
  \(\quad\protect\scalerel{\int}{\dfrac{\sqrt{5}}{(x+7)^4}\dd{x}}\) \hfill\cite[s.~71]{Knichal}}{exam138}

  První integrál odpovídá vzorci \ref{mai:eq166}:
  \begin{equation*}
    \scalerel{\int}{\dfrac{2}{x+5}\dd{x}} = 2\ln\abs{x+5} + c = \ln(x+5)^2 + c
  \end{equation*}

  U druhého integrálu položíme \(x+7 = t\), takže \(\dd{x} = \dd{t}\) a dostaneme
  \begin{align*}
    \scalerel{\int}{\dfrac{\sqrt{5}}{(x+7)^4}\dd{x}} &= 
    \sqrt{5}\scalerel{\int}{\dfrac{\dd{t}}{t^4}} = \dfrac{\sqrt{5}}{-3t^3} + c.   \\
    \shortintertext{a tedy}
    \scalerel{\int}{\dfrac{\sqrt{5}}{(x+7)^4}\dd{x}} &=-\dfrac{\sqrt{5}}{3}\dfrac{1}{(x+7)^3} + c.
  \end{align*}
\end{mathexam}