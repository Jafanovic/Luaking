% !TeX spellcheck = cs_CZ
\begin{mdframed}[style=mdexam]
  \begin{example}\label{mai:exam015}
    Rozložte na parciální zlomky lomenou racionální funkci \((x):y=\frac{7x+8}{x^2+x-2}\).
    \newline\textbf{Řešení:} Nejprve vypočteme nulové body jmenovatele:
    \begin{align*} 
      x^2+px+q &=(x-u)(x-v) = x^2-(u+v)x+uv            \\
                &\rightarrow p=-(u+v),\quad q=uv
    \end{align*}
    Kořenové činitele  \(x^2+x-2\rightarrow x_1=1, x_2=-2\) zvolíme za jmenovatele parciálních
    zlomků a rozklad hledáme ve tvaru \(\frac{7x+8}{x^2+x-2}=\frac{A}{x-1}+\frac{B}{x+2}\)
    kde \(A\), \(B\) jsou neznámé konstanty. Tyto konstanty určíme tak, aby rozklad platil pro 
    každé \(x\in\mathcal{R}-\{1,-2\}\). Po jednoduché úpravě dostaneme rovnost dvou polynomů
    \(7x+8=(A+B)x+2A-B\). Podle \ref{la:eq_eqv_poly} se musí rovnat koeficienty u \(x\) a absolutní 
    členy obou stran poslední rovnice \(\Rightarrow\) dostaneme soustavu rovnic pro určení \(A\) a 
    \(B\) ve tvaru:
    \begin{align}
      % \nonumber to remove numbering (before each equation)
      7 &= A+B  \nonumber \\ 
      8 &= 2A+B \label{la:eq_parc_example}   
    \end{align}
    dostáváme \(A=5,\quad B=2\). Postup, který jsme užili, nazýváme \textbf{Metodou neurčitých 
    koeficientů}.
    
    Pro určení koeficientů \(A\), \(B\) se užívají také jiné postupy, např. dosazování
    kořenů jmenovatele, která je výhodná zejména v případech, kdy jmenovatel lomené racionální
    funkce má jednoduché kořeny. Postupujeme tak, že rov. \ref{la:eq_parc_example} násobíme
    součinem kořenových činitelů \((x-1)(x+2)=x^2+x-2\) a dostaneme rovnici 
    \(7x+8=A(x+2)+B(x-1)\) pro určení koeficientů \(A\), \(B\) dosazováním kořenů.
      \begin{align*}
        % \nonumber to remove numbering (before each equation)
        x=-2 &\rightarrow       -14+8=B(-2-1)      \rightarrow B=2\\
        x=+1 &\rightarrow  \,\,\,+7+8=A(1+2)\quad  \rightarrow A=5
      \end{align*}
  \end{example}
\end{mdframed}