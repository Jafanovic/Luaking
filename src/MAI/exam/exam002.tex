% !TeX spellcheck = cs_CZ
\begin{mahtexam}{Určete vlastní čísla a odpovídající vlastní vektory následují\-cích matic:
  \begin{equation*}
    \mathbf{A}=
      \begin{pmatrix}
        1   & 0.5\\
        3.5 & 4
      \end{pmatrix}, \quad
    \mathbf{B}=
      \begin{pmatrix}
        3   & -1 \\
        2.5 &  4 
      \end{pmatrix}
  \end{equation*}
  }{exam002}

  Vlastní čísla určíme z charakteristické rovnice: \(\det(\mathbf{A} - \lambda\mathbf{I}) = 0\).
  Vlastní vektory \(\mathbf{x_i}\) odpovídající vlastním číslům \(\lambda_i\), jsou řešením
  homogenní soustavy rovnic \((\mathbf{A} - \lambda_i\mathbf{I})\mathbf{x_i} = 0\).
  \begin{itemize}
    \item Vlastní čísla matice \textbf{A}:
      \begin{equation*}
          \textbf{A} - \lambda\textbf{I} =
            \begin{pmatrix}
                1-\lambda  &  0.5          \\
              -3.5         &  4-\lambda
            \end{pmatrix}
      \end{equation*}
      \begin{align*}
        \det(\mathbf{A}-\lambda\mathbf{I}) &= 0 \\
        (1-\lambda)(4-\lambda)-\frac{7}{4} &= 0 \\
        \lambda^2-5\lambda+\frac{9}{4}     &= 0
      \end{align*}
      \begin{equation*}
        \lambda_1 = 4.5,\quad \lambda_2 = 0.5
      \end{equation*}
  \end{itemize}

  \begin{itemize}
    \item Vlastní čísla matice \textbf{B}:
      \begin{equation*}
          \textbf{B} - \lambda\textbf{I}=
            \begin{pmatrix}
              3-\lambda  & -1             \\
              2.5        &  4-\lambda
            \end{pmatrix}
      \end{equation*}
      \begin{align*}
        \det(\mathbf{B}-\lambda\mathbf{I}) &= 0 \\
        (3-\lambda)(4-\lambda)+\frac{5}{2} &= 0 \\
        \lambda^2-7\lambda+\frac{29}{2}    &= 0
      \end{align*}
      \begin{equation*}
        \lambda_1 = \frac{7+3i}{2},\quad \lambda_2 = \frac{7-3i}{2}
      \end{equation*}
  \end{itemize}
  % matice A
  Vlastní vektor matice \(\mathbf{A}\) pro \(\lambda_1=4.5: (\mathbf{A} -
  \lambda_1\mathbf{I})\mathbf{x_1} = 0 \Rightarrow\)
  \begin{equation*}
    \begin{pmatrix}
      1  -4.5  &  0.5     \\
      -3.5     &  4-4.5
    \end{pmatrix}
    \sim
    \begin{pmatrix}
      -3.5  &  0.5         \\
      -3.5  & -0.5
    \end{pmatrix}
  \end{equation*}
  \begin{equation*}
    \Rightarrow\mathbf{x_1} =
    \begin{pmatrix}
      1 \\ 7
    \end{pmatrix}
    \, r, r\in\mathbb{R}, r\neq0
  \end{equation*}
  Vlastní vektor matice \(\mathbf{A}\) pro \(\lambda_2=0.5: (\mathbf{A} -
  \lambda_1\mathbf{I})\mathbf{x_2}=0 \Rightarrow\)
  \begin{equation*}
    \begin{pmatrix}
      1  -0.5  &  0.5   \\
      -3.5      &  4-0.5
    \end{pmatrix}
    \sim
    \begin{pmatrix}
      0.5  &  0.5       \\
      3.5  &  3.5
    \end{pmatrix}
  \end{equation*}
  \begin{equation*}
    \Rightarrow\mathbf{x_2} =
    \begin{pmatrix}
      -1 \\ 1
    \end{pmatrix}
    \, r, r\in\mathbb{R}, r\neq0
  \end{equation*}
  % matice B
  Vlastní vektor matice \(\mathbf{A}\) pro \(\lambda_1=\frac{7+3i}{2}: (\mathbf{B} -
  \lambda_1\mathbf{I})\mathbf{x_1}=0 \Rightarrow\)
  \begin{align*}
    \begin{pmatrix}
      3 - \frac{7+3i}{2}            & -1                                       \\
      \frac{5}{2}                   &  4 - \frac{7+3i}{2}
    \end{pmatrix}
    &\sim
    \begin{pmatrix}
      -\frac{1}{2}-\frac{3}{2}i      &  -1                                     \\
      \frac{5}{2}                    & \frac{1}{2}-\frac{3}{2}i
    \end{pmatrix}
    \sim                                                                            \\
    \begin{pmatrix}
      -\frac{10}{4}                  &-\left(\frac{1}{2} -\frac{3}{2}i\right)  \\
      \frac{5}{2}                    & \frac{1}{2}-\frac{3}{2}i
    \end{pmatrix}
    &\sim
    \begin{pmatrix}
      -5                           &-\left(1-3i\right)                         \\
        5                           & \left(1-3i\right)
    \end{pmatrix}
  \end{align*} 
  \begin{align*} 
    \Rightarrow \mathbf{x_1}=
    \begin{pmatrix}
      -1+3i \\ 5
    \end{pmatrix}
    \, r, r\in\mathbb{C}, r\neq0
  \end{align*}
  Vlastní vektor matice \(\mathbf{B}\) pro \(\lambda_2=\frac{7-3i}{2}: (\mathbf{B} -
  \lambda_1\mathbf{I})\mathbf{x_2}=0 \Rightarrow\)
  \begin{align*}
    \begin{pmatrix}
      3  - \frac{7-3i}{2}       &  -1                                     \\
      \frac{5}{2}               &  4 - \frac{7-3i}{2}
    \end{pmatrix}
    &\sim
    \begin{pmatrix}
      -\frac{1}{2}+\frac{3}{2}i  &  -1                                     \\
      \frac{5}{2}                & \frac{1}{2}+\frac{3}{2}i
    \end{pmatrix}                                 
    \sim                                                                          \\
    \begin{pmatrix}
      -\frac{10}{4}              &-\left(\frac{1}{2} +\frac{3}{2}i\right)  \\
      \frac{5}{2}                & \quad\frac{1}{2}+\frac{3}{2}i
    \end{pmatrix}
    &\sim                                                                   
    \begin{pmatrix}
      -5                         &-\left(1+3i\right)                       \\
      5                          & \quad\left(1+3i\right)
    \end{pmatrix}
  \end{align*} 
  \begin{equation*} 
    \Rightarrow \mathbf{x_2}=
    \begin{pmatrix}
      -1-3i \\ 5
    \end{pmatrix}
    \, r, r\in\mathbb{C}, r\neq0
  \end{equation*}
  %---------------------------------------------------------------
  \lstinputlisting[%
    style=luaMatlabStyle,
    caption={Výpis programu pro ověření výpočtu vlastních čísel matic programem Matlab.}
    ]{../src/MAI/matlab/LA001.m}
  %--------------------------------------------------------------- 
\end{mathexam}