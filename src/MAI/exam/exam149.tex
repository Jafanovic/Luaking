\begin{mathexam}{\(\scalerel{\int}{\dfrac{3}{(1+x^2)x^2}\dd{x}}.\)}{exam149} 
  Integrand upravíme přičtením a odečtením výrazu \(3x^2\) v čitateli zlomku takto:
  \begin{align*}
    \dfrac{3}{(1+x^2)x^2} 
      &= \dfrac{3+3x^2-3x^2}{(1+x^2)x^2} = \dfrac{3(1+x^2)-3x^2}{(1+x^2)x^2}      \\
      &= \dfrac{3}{x^2} - \dfrac{3}{1+x^2}
  \end{align*}
  Později se budeme integrováním racionálních lomených funkcí zabývat systematicky.
  \[3\scalerel{\int}{\dfrac{\dd{x}}{x^2}} - 3\scalerel{\int}{\dfrac{\dd{x}}{1+x^2}}= -\dfrac{3}{x} -
    3\arctan x + c\]
\end{mathexam}