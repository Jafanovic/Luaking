% !TeX spellcheck = cs_CZ
\wikitextrule
\begin{example}\label{mai:exam071}
  \textbf{Jak číst výsledky studentské ankety aneb není průměr jako průměr}\newline\small
  Každý semestr na Masarykově univerzitě se uzavírá vyhodnocením velmi užitečné studentské ankety v 
  Informačním systému MU. Studenti hodnotí na jedenáctihodnotové stupnici (nula až deset bodů) 
  několik položek pro každý studijní předmět (obtížnost, zajímavost, srozumitelnost výkladu, 
  přístup učitele, rozmanitost literatury) a mohou doplnit i slovní komentáře. Na ty se učitelé 
  těší nejvíce, neboť díky anonymitě pisatelů se tak o sobě mohou dovědět leccos zajímavého. 
  Všimneme si však statistického zpracování ankety. U každého předmětu je pro danou položku 
  vypočtena průměrná bodová hodnota odpovědí a vyznačena na téže jedenáctihodnotové stupnici. Pro 
  porovnání je na stupnici vyznačen i takzvaný „fakultní průměr“. Každý přednášející může vidět svá 
  hodnocení a hodnocení svých kolegů, kteří mu vedou cvičení. Děkan má přístupové právo k celé
  statistice, a tak může porovnávat. Jednoho deštivého večera přestalo děkana bavit vyplňování 
  rektorátních formulářů a začal si výsledky ankety prohlížet. Zajímala jej zejména položka 
  „srozumitelnost výkladu“. Řekl si, že všem učitelům, kteří v této položce budou hodnoceni 
  nadprůměrně, zvýší osobní ohodnocení. Soubor předmětů je veliký, a tak děkan klikal a klikal. 
  Zjišťoval, že u veliké většiny předmětů leží průměrné hodnocení srozumitelnosti nad fakultním 
  průměrem. Jeho pocity byly smíšené. Na jedné straně se radoval, jakými jsou jeho podřízení 
  dobrými pedagogy, na druhé straně trnul, kolik to bude stát. Snad aby se raději vrátil 
  k protivným formulářům. Najednou v něm zahlodalo podezření i naděje, že není všechno v pořádku. 
  Jak je možné, že většina hodnocení leží nad průměrem? Kladné a záporné odchylky by se přece měly 
  kompenzovat. Zavolal proto na koberec proděkana pro informační technologie, aby se jej zeptal, co 
  je to „fakultní průměr“. Proděkan odpověděl takto: Máme soubor \(K\) předmětů \(\lbrace 
  X_\alpha\rbrace\), \(\alpha = 1, \ldots, K\). V předmětu \(X_\alpha\) vyplnilo anketu 
  \(N_\alpha\)  studentů, jednotlivé hodnoty odpovědí pro danou položku (srozumitelnost výkladu) 
  byly označeny \(\lbrace x_{\alpha,j}\rbrace\), \(j = 1, \ldots, N_\alpha\). Celkem přirozeně 
  předpokládáme, že váha odpovědi každého studenta je stejná, nezávisle na předmětu. Tato váha je
  rovna převrácené hodnotě celkového počtu studentů, kteří vyplnili anketu, tj. \(w = N^{-1}\), \(N 
  = N_1 + \ldots + N_k\). Fakultní průměr je proto dán vzorcem
  \begin{equation*}
    \langle x \rangle 
      = \sum_{\alpha=1}^{K}\sum_{j=1}^{N_\alpha}wx_{\alpha,j} 
      = \dfrac{1}{N}\sum_{\alpha=1}^{K}\left(\sum_{j=1}^{N_\alpha}x_{\alpha,j}\right)
      = \dfrac{1}{N}\sum_{\alpha=1}^{K}A_\alpha,
  \end{equation*}
  kde jsme označili \(A_\alpha = \sum_{j=1}^{N_\alpha}x_{\alpha,j}\). Děkan chvíli přemýšlel a 
  pravil: To vypadá docela logicky. Neměli bychom však počítat fakultní průměr tak, že vezmeme 
  průměrné hodnoty pro každý předmět a vypočteme jejich aritmetický průměr? Pak bychom dostali
  \begin{equation*}
    \langle \overline{x} \rangle
      = \dfrac{1}{K}\sum_{\alpha=1}^{K}\langle x_{\alpha}\rangle
      = \dfrac{1}{K}\sum_{\alpha=1}^{K}
        \left(\dfrac{1}{N_\alpha}\sum_{j=1}^{N_\alpha}x_{\alpha,j}\right)
      = \dfrac{1}{K}\sum_{\alpha=1}^{K}\dfrac{A_\alpha}{N_\alpha}.
  \end{equation*}
  Tento závěr se akademickým funkcionářům na první pohled nijak zvlášť nelíbil. Bylo totiž jasné, 
  že náhodné veličiny \(X_1, \ldots, X_k\) mají odlišná rozdělení. No jo, řekli si oba, musíme 
  počítat. My už ale počítat nemusíme, neboť jsme takový problém před chvílí vyřešili obecně. 
  Zjistili jsme totiž, že střední hodnota součtu náhodných veličin je rovna součtu středních 
  hodnot, bez ohledu na konkrétní rozdělení každé z veličin. Definujeme-li tedy náhodnou veličinu 
  \(Y\) jako aritmetický průměr veličin \(X_\alpha\), tj.
  \begin{align*}
    Y &= \dfrac{1}{K}\left(X_1 + \ldots + X_K\right),  \\
    \shortintertext{dostaneme}
    \langle y \rangle &= \dfrac{1}{K}\left(\langle x_1 \rangle +\ldots+\langle x_K \rangle\right).
  \end{align*}
  Tento výsledek se shoduje s hodnotou \(\langle \overline{x} \rangle\), kterou pro výpočet 
  „fakultního průměru“ navrhl děkan. Vypočteme-li součet odchylek hodnot \(\langle x_\beta 
  \rangle\) od \(\langle y \rangle\), dostaneme skutečně nulu:
  \begin{equation*}
    \sum_{\beta =1}^{K}\left(\langle x_\beta \rangle - \langle y \rangle\right)
      = \sum_{\beta =1}^{K}\left(\dfrac{A_\beta}{N_\beta} 
      - \dfrac{1}{K}\sum_{\alpha=1}^{K}\langle x_\alpha \rangle\right)
      = 0.
  \end{equation*}
  Zkusme se ještě zamyslet nad tím , jak použití „špatného“ fakultního průměru zkreslilo výsledky a 
  proč. Vypočtěme si rozdíl \(\Delta = \langle \overline{x} \rangle - \langle x \rangle\):
  \begin{equation*}
    \Delta = \langle \overline{x} \rangle - \langle x \rangle 
      = \dfrac{1}{K}\sum_{\alpha=1}^{K}\langle x_\alpha \rangle
      - \dfrac{1}{N}\sum_{\alpha=1}^{K}A_\alpha
      = \dfrac{1}{K}\sum_{\alpha=1}^{K}\langle x_\alpha \rangle
      - \dfrac{1}{N}\sum_{\alpha=1}^{K}N_\alpha\langle x_\alpha \rangle
      = \dfrac{1}{K}\sum_{\alpha=1}^{K}\langle x_\alpha\rangle\left(1 - K\dfrac{N_\alpha}{N}\right).
  \end{equation*}
  Platí přitom
  \begin{equation*}
    \sum_{\alpha=1}^{K}\left(1 - K\dfrac{N_\alpha}{N}\right) = 0.
  \end{equation*}
  Pokud by byl počet studentů, kteří vyplnili anketu, ve všech předmětech stejný, tj. \(N_\alpha = 
  \dfrac{N}{K}\) , byla by odchylka \(\Delta\) podle očekávání nulová. Stejná situace by nastala, 
  kdyby byly shodné všechny průměrné hodnoty \(\langle x_\alpha \rangle\). Je-li odchylka 
  \(\Delta\) kladná, je „nesprávný“ fakultní průměr \(\langle x \rangle\) nižší než \(\langle 
  \overline{x} \rangle\). Proto hodnocení jednotlivých předmětů vypadají příznivěji, právě tak, jak 
  to zjistil děkan. Odchylku \(\Delta\) posouvají do kladných hodnot předměty, které hodnotilo málo 
  studentů, a předměty, které měly vysoké hodnocení. Dobře je to vidět na příkladu dvou předmětů, 
  tj. pro \(K = 2\), kde vychází
  \begin{equation*}
    \Delta = \langle \overline{x} \rangle - \langle x \rangle 
           = \dfrac{N_2 - N_1}{2(N_2+N_1)}\left(\langle\overline{x}\rangle-\langle x\rangle\right).
  \end{equation*}
  
  Pro \(N_1\ll N_2\) a \(\langle x_1 \rangle  \gg \langle x_2 \rangle \) bude rozdíl \(\Delta\) 
  skoro polovina hodnoty \(\langle x_1 \rangle\)! U volitelných specializovaných předmětů, které si 
  vybírají jen poměrně malé počty studentů, kteří navíc mají o předmět opravdový zájem a hodnotí 
  jej proto většinou vyšším počtem bodů, je splněno obojí (malý počet hodnotících a vysoké bodové
  hodnocení). Je vidět, že při nesprávně zvoleném výpočtu srovnávací hodnoty (fakultního průměru) 
  mohou právě předměty, jejichž statistický význam je spíše okrajový, ovlivnit celkové hodnocení.
\normalsize
\end{example}