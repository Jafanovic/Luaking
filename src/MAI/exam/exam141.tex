\begin{mdframed}[style=mdexam]
  \begin{example}\label{MAI:exam141}
    Vypočtěme
    \begin{equation*}
      \int\dfrac{4x + 1}{(x^2 + 9)^3}\dd{x}.
    \end{equation*}
    \noindent\textbf{Řešení:}

    Převzato z \cite[s.~76]{Knichal}. Diskriminant kvadratického výrazu \(x^2+9\) je zřejmě záporný,
    takže jde o integrál typu 4. Nejprve rozložíme integrand na součet,

    \begin{align*}
      \dfrac{4x+1}{(x^2+9)^3} 
        &= 2\cdot\dfrac{2x}{(x^2+9)^3} + \dfrac{1}{(x^2+9)^3},             \\
      \shortintertext{takže}
      \int\dfrac{4x+1}{(x^2+9)^3}\dd{x} 
        &= 2\int\dfrac{2x}{(x^2+9)^3}\dd{x} + \underbrace{\int\dfrac{1}{(x^2+9)^3}\dd{x}}_{I_3},
    \end{align*}
    V prvním integrálu na pravé straně použijeme substituce \(x^2+9=z\), \(2x\dd{x} = \dd{z}\), z
    čehož 
    \begin{align*}
      2\int\dfrac{2x}{(x^2+9)^3}\dd{x} 
        &= 2\int z^{-3}\dd{z} = -z^{-2} + c  \\
        &= -\dfrac{1}{(x^2+9)^2} + c.
    \end{align*}  
    Pro druhý integrál zavedeme označení \(I_3\) a použijeme rekurentní vztah \ref{mai:eq175}
    \begin{align*}
      I_3 &= \dfrac{x}{2\cdot(3-1)\cdot9\cdot(x^2+9)^2}+\dfrac{6-3}{6-2}\cdot\dfrac{1}{9}\cdot I_2\\
      I_2 &= \dfrac{x}{2\cdot(2-1)\cdot9\cdot(x^2+9)  }+\dfrac{4-3}{4-2}\cdot\dfrac{1}{9}\cdot I_1\\
      \shortintertext{\(I_1\) je však tabulkový integrál}
      I_1 &= \int\dfrac{1}{x^2+9}\dd{x} = \dfrac{1}{3}\arctan\dfrac{x}{3} + c
    \end{align*}
    takže
    \begin{multline*}
      I_3 = 
        \dfrac{x}{36(x^2+9)^2} +        \\
          \left(
            \dfrac{3}{4}\cdot\dfrac{1}{9}\dfrac{x}{18(x^2+9)^2} + 
            \dfrac{1}{2}\cdot\dfrac{1}{9}\cdot\dfrac{1}{3}\arctan\dfrac{x}{3}
          \right) + c
    \end{multline*}
    Je tedy
    \begin{multline*}
      \int\dfrac{4x + 1}{(x^2 + 9)^3}\dd{x} = -\dfrac{1}{(x^2+9)^2} + \dfrac{x}{36(x^2+9)^2} +  \\
      + \dfrac{3}{4}\cdot\dfrac{1}{9}\cdot\dfrac{x}{18(x^2+9)^2}
      + \dfrac{3}{4}\cdot\dfrac{1}{9}\cdot\dfrac{1}{2}\cdot
        \dfrac{1}{9}\cdot\dfrac{1}{3}\arctan\dfrac{x}{3} + c
    \end{multline*}
    Po úpravě dostaneme výsledek
    \begin{equation*}
      \dfrac{x^3+15x-216}{216(x^2+9)^2} + \dfrac{1}{648}\arctan\dfrac{x}{3} + c.
    \end{equation*}
  \end{example}
\end{mdframed}