% !TeX spellcheck = cs_CZ
\begin{mathexam}{{Šance milion}}{exam007}
  „Znáte nějakou jinou hru, kde můžete denně vyhrát milion?“ Tento nebo jiný, obdobně nepříliš
  vtipný reklamní slogan propaguje v televizi hru, jejímž cílem je uhodnout šestici tažených cifer
  ve správném pořadí. (Hru raději nehrajte, pravděpodobnost výhry je mizivá.) Tah se provádí
  následovně: V každém ze šesti bubnů, očíslovaných pořadovými čísly \num{1} až \num{6}, je
  připraveno deset míčků opatřených ciframi \(0, 1, \ldots, 9\). Z prvního bubnu se náhodně vylosuje
  jedna cifra (deset možností). Poté se náhodně vylosuje jedna cifra z druhého bubnu (opět deset
  možností). Možností vzniku uspořádané dvojice cifer (jedna cifra z prvního a druhá z druhého
  bubnu) je již sto (každou možnost výsledku u prvního bubnu lze kombinovat s každou možností
  výsledku z druhého bubnu). Losování pokračuje u třetího, čtvrtého, pátého a šestého bubnu. Celkový
  počet možností je \num{1e6}, tedy \textbf{milion}. (Šance získat výhru, tedy vyhrát milion, je
  ovšem pouze jedna milióntina, neboť z milionu možností je pouze jedna skutečně tažena.) 
\end{mathexam}