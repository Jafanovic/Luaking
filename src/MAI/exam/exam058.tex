% !TeX spellcheck = cs_CZ
\wikitextrule
\begin{example}\label{mai:exam058}
  \textbf{Jak nekoupit zmetek}\newline\small
    Do finále soutěže o „šmejd roku“ se dostaly dva podniky, „Hvizd, s.r.o.“ a „Svist, a.s.“, které 
    zásobují trh zábavnou pyrotechnikou. První z nich kryje požadavky trhu ze \SI{70}{\percent}, 
    druhý ze zbývajících \SI{30}{\percent}. “Zjistilo se, že \SI{83}{\percent} ze všech výrobků 
    Hvizdu je vadných (nebouchají, když je to třeba, zejména však bouchají, když se to
    nejméně očekává). V případě Svistu je zmetků pouze \SI{63}{\percent}. Porota soutěže rozhodla, 
    že cenu dostane ten z obou podniků, jehož ředitel zodpoví správně následující otázky:
    \begin{enumerate}
     \item Jaká je pravděpodobnost, že náhodně zakoupená rachejtle bude fungovat tak, jak má?
     \item Jaká je pravděpodobnost, že náhodně zakoupená rachejtle, o níž se na obalu píše, že byla 
     vyrobena podnikem Hvizd, není zmetek?
     \item Jaká je pravděpodobnost, že náhodně zakoupená rachejtle, kterou se podařilo úspěšně 
     odpálit, byla vyrobena podnikem Svist?
    \end{enumerate} 
    Postupně jednotlivé úkoly vyřešíme.
    V případě a) posuzujeme pravděpodobnost jevu \(A\): „Náhodně zakoupený výrobek je funkční.“ bez 
    dalších podmínek. Jedná se o nepodmíněnou pravděpodobnost. Dejme tomu, že na trhu je v dané 
    chvíli ke koupi \(n\) rachejtlí. Z nich \SI{70}{\percent}, tj. \(\num{0.7}n\), bylo vyrobeno v 
    Hvizdu a zbytek, \(\num{0.3}n\), ve Svistu. Víme, že \SI{17}{\percent} výrobků Hvizdu je 
    funkčních, v případě Svistu je to \SI{37}{\percent}. Na trhu je tedy v tuto chvíli
    \begin{equation*}
      m = \num{0.17} - \num{0.7}n + \num{0.37}\cdot\num{0.3}n = \num{0.23}n
    \end{equation*}
    funkčních raket. Pravděpodobnost zakoupení funkční rakety je tedy
    \begin{equation*}
      p(A) = \dfrac{m}{n} = \num{0.23}.
    \end{equation*}
    Výsledek lze snadno zobecnit. Označíme-li \(p_1\) pravděpodobnost zmetku ve firmě Hvizd a 
    \(p_2\) pravděpodobnost zmetku ve firmě Svist, je pravděpodobnost funkčního výrobku ve Hvizdu 
    \((1 - p_1)\) a ve Svistu \((1 - p_2)\). Označme \(q\) podíl Hvizdu na celkové produkci. Podíl 
    Svistu je pak \((1 - q)\). Pravděpodobnost koupě funkčního výrobku je
    \begin{equation*}
      p(A) = (1 - p_1)q + (1 - p_2)(l - q) = (1 - p_2) + q(p_2 - p_1).
    \end{equation*}
    
    V úloze b) se již jedná o \textbf{podmíněnou pravděpodobnost}. Koupíme v obchodě raketu, 
    podíváme se na obal a zjistíme, že byla vyrobena ve Hvizdu. S touto dodatečnou informací chceme 
    zjistit pravděpodobnost, že až raketu rozbalíme a odpálíme, bude skutečně fungovat. Označme 
    jako jev \(B\) „Raketa byla vyrobena ve Hvizdu.“ Naším úkolem je tedy zjistit pravděpodobnost 
    jevu \(A\) (raketa bude funkční) za podmínky, že nastal jev \(B\) (byla vyrobena ve Hvizdu). 
    Tuto pravděpodobnost značíme \(p_B(A)\). Víme, že na trhu je \(qn = \num{0.7}n\) raket 
    vyrobených ve Hvizdu. To představuje pro náš další výpočet počet případů možných. Pouze \((1 - 
    p_1) = \num{0.17}\) z nich je však funkčních, počet případů příznivých je tedy \((1 - p_1)qn = 
    \num{0.17}\cdot\num{0.7}n = \num{0.119}n\). Hledaná pravděpodobnost je
    \begin{equation*}
      p_B(A) = \dfrac{(1 - p_1)qn}{qn} = 1 - p_1 = \num{0.17}
    \end{equation*}
    Všimněme si výpočtu podrobněji. \((1 - p_1)q\) představuje pravděpodobnost jevu \((A\text{ a 
    }B)\), že náhodně zakoupená raketa bude funkční a zároveň bude vyrobena ve Hvizdu. Skutečně, na 
    trhu je v dané chvíli \(n\) raket, z nich \(qn\) bylo vyrobeno ve Hvizdu a z těchto \(qn\) 
    výrobků Hvizdu je \((1 - p_1)qn\) funkčních. Proto 
    \begin{equation*}
      p(A\text{ a }B) = \dfrac{(1 - p_1)qn}{n} = (1 - p_1)q = \num{0.119}
    \end{equation*}
    (Divíte se, že tato pravděpodobnost není součinem \(p(A)p(B)\)? Nedivte se, jevy \(A\) a \(B\) 
    nejsou totiž nezávislé!). Vidíme, že platí
    \begin{equation*}
      p(A\text{ a }B) = p(B)p_B(A),
    \end{equation*}
    neboť \(q = p(B)\). Získáváme tedy vztah pro výpočet podmíněné pravděpodobnosti:
    \adjustbox{minipage=[c][32pt][c]{406pt}}{%
      \begin{equation}\label{mai:eq057}
        p_B(A) = \dfrac{p(A\text{ a }B)}{p(B)} \qquad\text{a obdobně}\qquad
        p_A(B) = \dfrac{p(A\text{ a }B)}{p(A)}.
      \end{equation}
      }
    Vzorec (\ref{mai:eq057}) jsme získali pro konkrétní příklad. Abychom byli korektní, odvoďme jej 
    obecně. Označme \(p(A)\) a \(p(B)\) pravděpodobnost jevu \(A\) a pravděpodobnost jevu \(B\), 
    \(p_B(A)\) podmíněnou pravděpodobnost jevu \(A\) za podmínky, že nastal jev \(B\), a \(p_A(B)\) 
    podmíněnou pravděpodobnost jevu \(B\) za podmínky, že nastal jev \(A\). \(p(A\text{ a }B)\) je 
    pravděpodobnost současného nastoupení jevů \(A\) a \(B\). Dejme tomu, že v celkovém počtu \(n\) 
    opakování pokusu nastal jev \(B\) \(s\)-krát. Nechť v \(t\) případech z těch, kdy nastal jev 
    \(A\), nastal také jev \(B\). Pro pravděpodobnosti pak platí
    
\normalsize
\end{example}