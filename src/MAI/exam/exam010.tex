% !TeX spellcheck = cs_CZ
\begin{mathexam}{Obsazování kvantových stavů}{exam010}
  Úloha o kuličkách a přihrádkách má přímou aplikaci v \textbf{kvantové fyzice}. Představme si, že
  fyzikální soustava je tvořena \(k\) částicemi. Každá částice se nachází v určitém stavu, v němž jí
  můžeme přisoudit fyzikální charakteristiky, které jsou s tímto stavem spojeny (třeba energii,
  moment hybnosti, apod.). Jednotlivé stavy jsou pak rozlišitelné právě pomocí těchto
  charakteristik. Dejme tomu, že přípustných stavů je \(n \geqq 1\). Problémem kvantové fyziky je
  to, že kvantové částice jsou nerozlišitelné. Nepoznáme jednu od druhé. Je to stejné, jako bychom
  měli \(k\) naprosto stejně vypadajících kuliček, které nemáme nijak očíslovány. Záměna dvou částic
  (nerozlišitelných kuliček) se nepozná, nevede tedy ke změně stavu fyzikální soustavy. Pro hodnoty
  fyzikálních charakteristik soustavy jako celku je tedy důležité jen to, kolik částic je v každém z
  přípustných stavů. Musíme se tedy zajímat o to, kolika způsoby lze našich \(k\)
  \textbf{nerozlišitelných částic} (kuliček) umístit do \(n\) \textbf{stavů} (přihrádek). Kvantové
  částice jsou však dvojího druhu, \textbf{fermiony} (například elektrony, neutrony, protony, jádra
  s lichým počtem nukleonů) a \textbf{bosony} (například fotony, mezony, jádra se sudým počtem
  nukleonů). Rozdíl mezi nimi je ten, že bosony se „dobře snášejí“, a proto jich může být v jednom
  stavu i více. 
  \begin{itemize}[noitemsep]
    \item Počet možností, jak rozmístit \(k\) \textbf{bosonů} po \(n\) stavech je tedy
          \begin{equation*}
            N_{\text{boson}} = \binom{ n + k - 1}{k}
          \end{equation*}
    \item S \textbf{fermiony} je tomu jinak. \textbf{Pauliho vylučovací princip} jim zakazuje, aby v
          daném stavu byl více než jeden fermion. Stav může být buď prázdný, nebo obsazen jedním
          fermionem. V takovém případě musí být \(n \geqq k\) a v každé přihrádce může být nejvýše
          jedna kulička. Situace tak odpovídá \textbf{kombinacím bez opakování \(k\)-té třídy z
          \(n\) prvků}, tj.
          \begin{equation*}
            N_{\text{fermion}} = \binom{n}{k}
          \end{equation*}
  \end{itemize}
\end{mathexam}