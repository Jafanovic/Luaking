% !TeX spellcheck = cs_CZ
% \wikitextrule
\begin{mdframed}[style=mdexam]
  \begin{example}\label{mai:exam088}
    \textbf{Determinant matice třetího řádu}
    \begin{gather*}
      \det{\matr{A}} = \det
      \begin{pmatrix}
          a_{11} & a_{12} &  a_{13} \\
          a_{21} & a_{22} &  a_{23} \\
          a_{31} & a_{32} &  a_{33} 
      \end{pmatrix} =  \\
      \begin{aligned}
        = &+a_{11}a_{22}a_{33} + a_{13}a_{21}a_{32} + a_{12}a_{23}a_{31} \\
          &-a_{12}a_{21}a_{33} - a_{13}a_{22}a_{31} - a_{11}a_{23}a_{32}
      \end{aligned}    \displaybreak\\
      \det{\matr{A}} = \det
      \begin{pmatrix}
          -1 & 3 &  2 \\
          -2 & 0 & -4 \\
           5 & 1 & -3
      \end{pmatrix} =   \\   
      \begin{aligned}
        &= (-1)\cdot0\cdot(-3) + 2\cdot(-2)\cdot1 + 3\cdot4\cdot5   \\
        &- 3\cdot(-2)\cdot(-3)-2\cdot0\cdot5 - 1\cdot4\cdot1        \\
        &= 0 + (-4) + 60 - 18 + 0 + 4 = 42. 
      \end{aligned}
    \end{gather*}   
    Způsob výpočtu si snadno zapamatujeme ve tvaru \emph{Sarrusova pravidla} (1.37).
    \todo[inline]{Příklad exam088: "Determinant matice třetího řádu" odkaz na vzorec Sarrusova pravidla}
  \end{example}
\end{mdframed}