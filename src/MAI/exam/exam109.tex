\begin{mathexam}{\(\int\ln x\dd{x}\) \hfill\cite[s.~31]{Knichal}}{exam109} 
  \textbf{Umělý rozklad na součin}: někdy zadaná funkce \(f(x)\) jako součin vůbec nevypadá, a
  přesto je použití metody per partes vhodné. Například pro elementární funkci \(f(x) = \ln x\) sice
  najdeme primitivní funkci \ref{mai:eq110} v tabulce základních neurčitých integrálů z odstavce
  \ref{mai:IchapVIIsecIIssecI}, ale je možné postupovat i jinak. Představme si \(f(x)\) jako součin
  \(f(x) = 1\cdot\ln x\) a zvolme 
  \begin{equation*}
    \left\lvert
      \begin{array}{ll} 
        u  = \ln x  &  u'= \frac{1}{x}  \\
        v' = 1      &  v =  x   
      \end{array}
    \right\rvert
  \end{equation*}
  Výpočet integrálu bude tedy vypadat takto:
  \begin{equation*}
    \int\ln x\dd{x} = x\ln x - \int x\cdot\frac{1}{x}\dd{x}= x\ln x - x + c.
  \end{equation*}
\end{mathexam}