\begin{mathexam}{Umělý rozklad na součin}{exam109}
  Někdy zadaná funkce \(f(x)\) jako součin vůbec nevypadá, a přesto je použití metody per partes
  vhodné. Například pro elementární funkci \(f(x) = \ln x\) sice najdeme primitivní funkci
  \ref{mai:eq110} v tabulce základních neurčitých integrálů z odstavce \ref{mai:IchapVIIsecIIssecI},
  ale je možné postupovat i jinak. Představme si \(f(x)\) jako součin \(f(x) = 1\cdot\ln x\) a
  zvolme \[u'(x) = 1 ⇒ u(x) = x, \quad v(x) = lnx ⇒ v'(x) = \frac{1}{x}\] Pak 
  \begin{equation*}
    \int\ln\dd{x} = x\ln x - \int x\cdot\frac{1}{x}\dd{x}= x\ln x - x.
  \end{equation*}
\end{mathexam}