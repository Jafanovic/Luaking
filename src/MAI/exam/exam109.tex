\begin{mdframed}[style=mdexam]
  \begin{example}\label{mai:exam109}
    (\emph{Umělý rozklad na součin}): Někdy zadaná funkce \(f(x)\) jako součin vůbec nevypadá, a
    přesto je použití metody per partes vhodné. Například pro elementární funkci \(f(x) = \ln x\)
    sice najdeme primitivní funkci \ref{MA:baseInt06} v tabulce základních neurčitých integrálů z
    odstavce \ref{MA:chap_tabINT}, ale je možné postupovat i jinak. Představme si \(f(x)\) jako
    součin \(f(x) = 1\cdot\ln x\) a zvolme \[u'(x) = 1 ⇒ u(x) = x, \quad v(x) = lnx ⇒ v'(x) =
    \frac{1}{x}\] Pak 
    \begin{align*}
      \int\ln\dd{x} &= x\ln x - \int x\cdot\frac{1}{x}\dd{x}  \\ 
                    &= x\ln x - x.
    \end{align*}
  \end{example}
\end{mdframed}