% !TeX spellcheck = cs_CZ
\wikitextrule
\begin{example}\label{mai:exam066}
  \textbf{Bernoulliovo (binomické) rozdělení}\newline\small
  Pro střední hodnotu Bernoulliova rozdělení platí
  \begin{equation*}
    \langle j \rangle = \sum_{j=0}^{n}j\cdot p_j
      = \sum_{j=1}^{n}j\begin{pmatrix} n \\ j \end{pmatrix}p^j(1-p)^{n-j} = np.
  \end{equation*}
  
  Tento vztah lze dokázat matematickou indukcí vzhledem k proměnné \(n\). Na tomto místě nebudeme 
  důkaz provádět pro jeho poměrnou zdlouhavost. Každý jej však může zvládnout. Vzpomeňme si v tuto 
  chvíli na náš chybný intuitivní odhad v příkladu \ref{mai:exam057}, v němž jsme poprvé hovořili o 
  Bernoulliově pokusu v souvislosti s hody mincí. Chybně jsme tam odhadli pravděpodobnost, že při 
  \(n\) opakováních pokusu nastane v polovině z nich zdar. Tato chyba vznikla v důsledku naší 
  zkušenosti, že když budeme mincí vícekrát házet, padne hlava (zdar) skutečně zhruba v polovině 
  případů. Vidíme nyní, že \(n/2\) reprezentuje střední hodnotu náhodné veličiny \(X =\) 
  \textbf{počet zdarů při \(n\) hodech mincí}. A to naší zkušenosti již odpovídá.
\normalsize
\end{example}