% !TeX spellcheck = cs_CZ
\begin{mdframed}[style=mdexam]
  \begin{example}\label{mai:exam054}
    \textbf{Sestavování čísel z cifer}\newline
    Máme k dispozici libovolný počet cifer \(0, 1, \ldots, 9\). Kolik \(k\)-ciferných čísel z nich
    můžeme sestavit? Odpověď na tuto otázku každý zná. Dvojciferná jsou čísla od \num{10} do
    \num{99} včetně, je jich tedy \((99 - 10 + 1) = 90\). Trojciferná jsou od \(100\) do \(999\)
    včetně, jejich počet je \((999 - 100 + 1) = 900\), \(k\)-ciferná jsou čísla od \(100\ldots0 =
    1\cdot10^{k-1}\) do \(999\ldots9\) včetně (\(k\) devítek), jejich počet je \(9\cdot10^{k-1}\).
    Tento výsledek bychom však měli získat i kombinatorickými úvahami. Čísla totiž dostáváme tak, že
    z deseti cifer \(0, 1,\ldots, 9\) vytváříme variace \(k\)-té třídy s opakováním, musíme však
    vyjmout ty možnosti, které začínají nulami. Dostáváme
    \begin{gather*}
      \begin{aligned}
        10^k &- 
          \left(
            9\cdot10^{k-2} + 9\cdot10^{k-3} + \cdots + 9\cdot10^1 + + 9\cdot10^0 + 1 
          \right)                                                                     \\
            &= 10^k - 9\cdot\dfrac{10^{k-1} - 1}{10 - 1} - 1                          \\
            &= 10^k - 10^{k-1} = 9\cdot10^{k-1}
      \end{aligned}
    \end{gather*}
    Jak jsme dostali odečítaný výraz v závorce? Hodnota \(9\cdot10^{k-2}\) představuje počet těch
    výběrů cifer (s opakováním), které mají na první pozici pevnou nulu, na druhé pozici kteroukoli
    nenulovou cifru (\num{9} možností) a na dalších \((k - 2)\) pozicích kteroukoli cifru
    (\(10^{k-2}\) možností). Hodnota \(9\cdot10^{k-3}\) je počet těch výběrů cifer (s opakováním),
    které mají na prvních dvou pozicích pevné nuly, na třetí pozici kteroukoli nenulovou cifru
    (\num{9} možností) a na dalších \((k - 3)\) pozicích kteroukoli cifru (\(10^{k-2}\) možností). A
    tak dále. Nakonec odečítáme ještě jedničku, která reprezentuje jediný výběr \(k\) cifer tvořený
    samými nulami. Kdybychom se nyní zeptali, jaká je pravděpodobnost, že při náhodném výběru ze
    souboru jednociferných až \(n\)-ciferných čísel vylosujeme třeba \(k\)-ciferné číslo, odpovíme
    si již snadno: Počet případů možných je
    \begin{equation*}
      N(n) =9 + 90 + \cdots + 9\cdot10^{n-1} = 9\dfrac{10^n-1}{10 - 1} = 10^n - 1
    \end{equation*}
    počet případů příznivých je  \(M(n,k) = 9\cdot10^{k-1}\). Hledaná pravděpodobnost je tedy
    \begin{equation*}
      p(n,k) = \dfrac{9\cdot10^{k-1}}{10^n-1}.
    \end{equation*}
    Zkontrolujme si platnost získaného vzorce pro jednoduché případy, kdy ji snadno určíme přímo.
    Pro \(n = 1\) a \(k = 1\) je vylosování jednociferného čísla jevem jistým. A skutečně, náš
    vzorec dává
    \begin{equation*}
      p(1,1) = \dfrac{9\cdot10^0}{10^1-1} = 1.
    \end{equation*}
    Pro \(n = 2\) máme celkem \num{99} jednociferných a dvojciferných čísel, z nich jednociferných
    je devět a dvojciferných \num{90}. Pravděpodobnost vylosování jednociferného čísla by tedy měla
    vyjít \(9/99=1/11\) a pravděpodobnost vylosování čísla dvojciferného \(90/99=10/11\). Z našeho
    obecného vzorce dostáváme
    \begin{align*}
      p(2,1) &= \dfrac{9\cdot10^0}{10^2-1} = \frac{9}{99} = \frac{1}{11}. \\
      p(2,2) &= \dfrac{9\cdot10^1}{10^2-1} = \frac{90}{99} = \frac{10}{11}.
    \end{align*}
    Jistým jevem je, že vylosujeme nějaké číslo. Skutečně také
    \begin{equation*}
      \sum_{k=1}^{n}p(n,k) = \dfrac{9}{10^n - 1}\dfrac{10^n - 1}{10 - 1} =1.
    \end{equation*}
  \end{example}
\end{mdframed}