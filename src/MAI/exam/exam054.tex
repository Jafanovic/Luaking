% !TeX spellcheck = cs_CZ
\wikitextrule
\begin{example}\label{mai:exam052}
  \textbf{Sestavování čísel z cifer}\newline\small
   Máme k dispozici libovolný počet cifer \(0, 1, \ldots, 9\). Kolik \(k\)-ciferných čísel z nich 
   můžeme sestavit? Odpověď na tuto otázku každý zná. Dvojciferná jsou čísla od \num{10} do 
   \num{99} včetně, je jich tedy \((99 - 10 + 1) = 90\). Trojciferná jsou od \(100\) do \(999\) 
   včetně, jejich počet je \((999 - 100 + 1) = 900\), \(k\)-ciferná jsou čísla od \(100\ldots0 = 
   1\cdot10^{k-1}\) do \(999\ldots9\) včetně (\(k\) devítek), jejich počet je \(9\cdot10^{k-1}\). 
   Tento výsledek bychom však měli získat i kombinatorickými úvahami. Čísla totiž dostáváme tak, že 
   z deseti cifer \(0, 1,\ldots, 9\) vytváříme variace \(k\)-té třídy s opakováním, musíme však 
   vyjmout ty možnosti, které začínají nulami. Dostáváme
   \begin{equation*}
     10^k - 
   \end{equation*}
   Jak jsme dostali odečítaný výraz v závorce? Hodnota \(9\cdot10^{k-2}\) představuje počet těch 
   výběrů cifer (s opakováním), které mají na první pozici pevnou nulu, na druhé pozici kteroukoli 
   nenulovou cifru (\num{9} možností) a na dalších \((k - 2)\) pozicích kteroukoli cifru 
   (\(10^{k-2}\) možností). Hodnota \(9\cdot10^{k-3}\) je počet těch výběrů cifer (s opakováním), 
   které mají na prvních dvou pozicích pevné nuly, na třetí pozici kteroukoli nenulovou cifru 
   (\num{9} možností) a na dalších \((k - 3)\) pozicích kteroukoli cifru (\(10^{k-2}\) možností). A 
   tak dále. Nakonec odečítáme ještě jedničku, která reprezentuje jediný výběr \(k\) cifer tvořený 
   samými nulami. Kdybychom se nyní zeptali, jaká je pravděpodobnost, že při náhodném výběru ze 
   souboru jednociferných až \(n\)-ciferných čísel vylosujeme třeba \(k\)-ciferné číslo, odpovíme 
   si již snadno: Počet případů možných je
   \begin{equation*}
     N(n) =
   \end{equation*}
   počet případů příznivých je  \(M(n,k) = 9\cdot10^{k-1}\). Hledaná pravděpodobnost je tedy
   \begin{equation*}
     p(n,k)
   \end{equation*}
   Zkontrolujme si platnost získaného vzorce pro jednoduché případy, kdy ji snadno určíme přímo. 
   Pro \(n = 1\) a \(k = 1\) je vylosování jednociferného čísla jevem jistým. A skutečně, náš 
   vzorec dává
  \normalsize
\end{example}