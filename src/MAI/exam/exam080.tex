% !TeX spellcheck = cs_CZ
% \wikitextrule
\begin{mdframed}[style=mdexam]
  \begin{example}\label{mai:exam080}
    \textbf{Počítání s komplexními čísly v exponenciálním tvaru}\newline
      Vraťme se nyní k rovnici
      \begin{equation*}
        z^4 = -1,
      \end{equation*}
      na jejíž řešení v algebraickém tvaru jsme museli rezignovat, a pokusme se ji řešit ve tvaru 
      goniometrickém nebo exponenciálním, 
      \begin{equation*}
        z = \abs{z}(\cos\varphi + i\sin\varphi) = \abs{z}e^{i\varphi}.
      \end{equation*}
      Porovnáváme dvě čísla, na levé straně je výraz
      \begin{equation*}
        z^4 = \abs{z}^4(\cos4\varphi + i\sin4\varphi) = \abs{z}e^{i\cdot4\varphi},
      \end{equation*}
      na pravé straně číslo \(-1 = \cos\pi + i\cdot\sin0 = e^{i\pi}\). Porovnáním modulů zjistíme,
      že \(\abs{z} = 1\). Porovnáním argumentů pak
      \begin{equation*}
        \cos4\varphi = \cos\pi,\quad\sin4\varphi = 0, \quad\text{nebo}\quad e^{i4\varphi}= e^{i\pi}.
      \end{equation*}
      Odtud dostáváme nejednoznačný výsledek
      \begin{equation*}
        4\varphi = \pi + 2k\pi \Rightarrow \varphi = \dfrac{\pi}{4} + k\dfrac{\pi}{2},
      \end{equation*}
      kde \(k\) je libovolné celé číslo. Hodnoty \(k = 0, 1, 2, 3\) vedou k různým řešením rovnice:
      \begin{alignat*}{4}
        k &=0:\qquad\varphi_0=\dfrac{ \pi}{4}\qquad
              &&z&&= \dfrac{\sqrt{2}}{2}+i\dfrac{\sqrt{2}}{2},\\
        k &=1:\qquad\varphi_1=\dfrac{3\pi}{4}\qquad 
              &&z&&=-\dfrac{\sqrt{2}}{2}+i\dfrac{\sqrt{2}}{2},\displaybreak\\
        k &=2:\qquad\varphi_2=\dfrac{5\pi}{4}\qquad 
              &&z&&=-\dfrac{\sqrt{2}}{2}-i\dfrac{\sqrt{2}}{2},\\
        k &=3:\qquad\varphi_1=\dfrac{7\pi}{4}\qquad 
              &&z&&= \dfrac{\sqrt{2}}{2}-i\dfrac{\sqrt{2}}{2},
      \end{alignat*} 
      Pro další hodnoty \(k\) se řešení začnou opakovat (vyzkoušejte).
  \end{example}
\end{mdframed}