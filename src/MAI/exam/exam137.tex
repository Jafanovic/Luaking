\begin{mdframed}[style=mdexam]
  \begin{example}\label{MAI:exam137}
    \begin{equation*}
      R(x) = \dfrac{2x^3 + x + 2}{x^4 + x^3 + x^2}
    \end{equation*}
    \noindent\textbf{Řešení:}

    Rozložíme jmenovatele v reálném oboru: \(x^4+x^3+x^2 = x^2(x^2 + x + 1)\), trojčlen v závorce
    má diskriminant rovný \(-3<0\), tedy dále jej rozkládáat nebudme. Potom
    \begin{gather*}
      \dfrac{2x^3+x+2}{x^2(x^2+x+1)} = \dfrac{A}{x^2} + \dfrac{B}{x} + \dfrac{Cx+D}{x^2+x+1}
    \end{gather*}
    Úpravou dostáváme
    \begin{align*}
      2x^3+x+2 &= A(x^2+x+1) + Bx(x^2+x+1)     \\ 
               &+ x^2(Cx+D),                   \\
      2x^3+x+2 &= Ax^2 + Ax + A + Bx^3 + Bx^2  \\
               &+ Bx + Cx^3 + Dx^2             \\
      2x^3+x+2 &= (B+C)x^3 + (A+B+D)x^2        \\
               &+ (A+B)x +A.
    \end{align*}
    Odtud porovnáním koeficientů dostaneme soustavu rovnic
    \begin{equation*}
      \begin{array}{rcrcrcrcl}
          & & B &+& C & &   &=& 2,  \\
        A &+& B & &   &+& D &=& 0,  \\
        A &+& B & &   & &   &=& 1,  \\
        A & &   & &   & &   &=& 2.
      \end{array}
    \end{equation*}
    Řešením soustavy dostaneme \(A=2\),\(B = -1\), \(C=3\), \(D=-1\), tj.  
    \begin{gather*}
      \dfrac{2x^3+x+2}{x^2(x^2+x+1)} = \dfrac{2}{x^2} - \dfrac{1}{x} + \dfrac{3x-1}{x^2+x+1}
    \end{gather*}
  \end{example}
\end{mdframed}