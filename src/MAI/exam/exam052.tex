% !TeX spellcheck = cs_CZ
\wikitextrule
\begin{example}\label{mai:exam052}
  \textbf{Grupy nemusí být tvořeny jen čísly}\newline\small
  Jaká je pravděpodobnost hlavní výhry ve sportce? Všichni víme, že malá, ale máme představu, jak malé
  toto číslo je? Při sportce se losuje \(k = 6\) čísel a jedno dodatkové z celkového počtu \(n = 49\) 
  čísel. (Dříve byla čísla spojena s názvy sportů, odtud název „sportka“ .) Na pořadí čísel ve výběru 
  nezáleží, vytažené číslo se do hry nevrací. Jde tedy o \textbf{kombinace bez opakování}. Hlavní výhra 
  požaduje uhodnout všech \num{6} tažených čísel. Jev \(A\) je tedy definován takto:
  
  \begin{itemize}
  \item Jev \(A\): Bude taženo právě oněch \num{6} čísel, která jsem vsadil.
        Počet možností, které při tah u sportky mohou nastat (počet případů možných), je \(N = 
        \begin{pmatrix} n \\ k\end{pmatrix} =  \begin{pmatrix} 49 \\ 6 \end{pmatrix} \). Hlavní výhru 
        představuje jediná kombinace, počet příznivých případů je proto \(M = 1\). Pravděpodobnost hlavní 
        výhry ve sportce, tj. pravděpodobnost jevu \(A\), je
        \begin{equation*}
          p(A) = \dfrac{M}{N} = \dfrac{1}{\begin{pmatrix} 49 \\ 6 \end{pmatrix}} 
               = \dfrac{43!6!}{49!} = \dfrac{720}{49\cdot48\cdot47\cdot46\cdot45\cdot44} \simeq \num{7e-8}
               = \SI{7e-6}{\percent}
        \end{equation*}
      \end{itemize}
      Pravděpodobnost hlavní výhry je velmi malá, sedm milióntin procenta. 
  \item A o kolik lepší to bude s pravděpodobností některé z vedlejších výher? Tak třeba pátá cena znamená, 
        že je nutné ze šesti tažených čísel uhodnout libovolné tři. Jev \(A\) je tedy: Ze šesti čísel, 
        která jsme vsadili, budou v tažené kombinaci obsažena právě tři libovolná z nich. Počet \(N\) 
        zůstává stejný jako v předchozí části úlohy. Je třeba jen určit \(M\). Každý příznivý případ     
        vzniká tak, že trojice správných čísel (výběry tří ze šesti) je doplněna trojicí chybných čísel 
        (výběry tří ze čtyřiceti tří). Tedy \(M = \begin{pmatrix} 6 \\ 3\end{pmatrix}\begin{pmatrix} 49 - 6 
        \\ 3\end{pmatrix} =  \begin{pmatrix} 6 \\ 3\end{pmatrix}\begin{pmatrix} 43 \\ 6 \end{pmatrix}\),
        \begin{align*}
          p(A) &= \dfrac{M}{N} 
                = \dfrac{\begin{pmatrix} 6 \\ 3\end{pmatrix}\begin{pmatrix} 43 \\ 6 
                \end{pmatrix}\)}{\dfrac{\begin{pmatrix} 49 \\ 6\end{pmatrix}}                \\
               &=\left(\dfrac{6!}{3!\cdot3!}\right)
        \end{align*}
      
  \normalsize
\end{example}