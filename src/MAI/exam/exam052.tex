% !TeX spellcheck = cs_CZ
\wikitextrule
\begin{example}\label{mai:exam052}
  \textbf{Výhra ve sportce}\newline\small
  Jaká je pravděpodobnost hlavní výhry ve sportce? Všichni víme, že malá, ale máme představu, jak 
  malé toto číslo je? Při sportce se losuje \(k = 6\) čísel a jedno dodatkové z celkového počtu \(n 
  = 49\) čísel. (Dříve byla čísla spojena s názvy sportů, odtud název „sportka“ .) Na pořadí čísel 
  ve výběru nezáleží, vytažené číslo se do hry nevrací. Jde tedy o \textbf{kombinace bez 
  opakování}. Hlavní výhra požaduje uhodnout všech \num{6} tažených čísel. Jev \(A\) je tedy 
  definován takto:
  
  \begin{itemize}
    \item Jev \(A\): Bude taženo právě oněch \num{6} čísel, která jsem vsadil.
          Počet možností, které při tahu sportky mohou nastat (počet případů možných), je \(N = 
          \begin{pmatrix} n \\ k\end{pmatrix} =  \begin{pmatrix} 49 \\ 6 \end{pmatrix} \). Hlavní 
          výhru představuje jediná kombinace, počet příznivých případů je proto \(M = 1\). 
          Pravděpodobnost hlavní výhry ve sportce, tj. pravděpodobnost jevu \(A\), je
          \begin{equation*}
            p(A) = \dfrac{M}{N} = \dfrac{1}{\begin{pmatrix} 49 \\ 6 \end{pmatrix}} 
                 = \dfrac{43!6!}{49!} = \dfrac{720}{49\cdot48\cdot47\cdot46\cdot45\cdot44} \simeq 
                 \num{7e-8}
                 = \SI{7e-6}{\percent}
          \end{equation*}
          Pravděpodobnost hlavní výhry je velmi malá, sedm milióntin procenta. 
    \item A o kolik lepší to bude s pravděpodobností některé z vedlejších výher? Tak třeba pátá 
          cena znamená, že je nutné ze šesti tažených čísel uhodnout libovolné tři. Jev \(A\) je 
          tedy: Ze šesti čísel, která jsme vsadili, budou v tažené kombinaci obsažena právě tři 
          libovolná z nich. Počet \(N\) zůstává stejný jako v předchozí části úlohy. Je třeba jen 
          určit \(M\). Každý příznivý případ vzniká tak, že trojice správných čísel (výběry tří ze 
          šesti) je doplněna trojicí chybných čísel (výběry tří ze čtyřiceti tří). Tedy \(M = 
          \begin{pmatrix}6 \\ 3\end{pmatrix}\begin{pmatrix} 49 - 6 \\ 3\end{pmatrix} =  
          \begin{pmatrix} 6 \\ 3\end{pmatrix}\begin{pmatrix} 43 \\ 3 \end{pmatrix}\),
          \begin{align*}
            p(A) &= \dfrac{M}{N} 
                  = \dfrac{\begin{pmatrix} 6 \\ 3\end{pmatrix}
                           \begin{pmatrix} 43 \\ 3 \end{pmatrix}}
                          {\begin{pmatrix} 49 \\ 6\end{pmatrix} }                
                  =\left(\dfrac{6!}{3!\cdot3!}\right)\left(\dfrac{43!}{40!\cdot3!}\right)
                   \left(\dfrac{43!6!}{49!}\right)                                            \\
                 &=\dfrac{120\cdot43\cdot42\cdot41\cdot720}
                         {49\cdot48\cdot47\cdot46\cdot45\cdot44\cdot36} 
                  \simeq\num{0.018}.
          \end{align*}
          Tato pravděpodobnost již zanedbatelná není, na rozdíl od finanční částky, jíž bývá 
          ohodnocena pátá cena. Sázení sportky může domácímu rozpočtu spíše ublížit.
    \item Třetí, resp. čtvrtá cena jsou, podobně jako první a pátá, definovány velmi jednoduše. Je 
          třeba uhodnout pět, resp. čtyři ze šesti tažených čísel. V případě druhé ceny hraje roli
          dodatkové číslo. Druhou cenu získává ten, kdo uhodl pět ze šesti čísel vylosovaných v 
          prvním tahu a ještě navíc číslo dodatkové, které se losuje ze zbylých \num{43} čísel, jež 
          zůstala po prvním tahu v osudí. Jev \(A\) je tedy definován takto:
          
          Ze šesti čísel, která jsem vsadil, bude při prvním tahu vylosováno libovolných pět a v 
          druhém tahu bude vylosováno právě to dodatkové číslo, které jsem vsadil. Počet případů 
          příznivých je pouze \(M = \begin{pmatrix}6 \\ 5\end{pmatrix}\), neboť šestým číslem 
          nemůže být libovolné ze \num{43} čísel, která nebyla v prvním tahu vylosována, ale 
          musí to být právě číslo dodatkové. Pravděpodobnost jevu \(A\) je
          \begin{equation*}
            p(A)  = \dfrac{M}{N} 
                  = \dfrac{\begin{pmatrix} 6  \\ 5\end{pmatrix}}
                          {\begin{pmatrix} 49 \\ 6\end{pmatrix}}
                  \simeq\num{4.2e-17}.
          \end{equation*}
          
          Pokud bychom jako jev \(A\) označili výhru jakékoliv ceny, dostaneme
          \begin{align*}
            M &= \sum_{6}^{k=3}\begin{pmatrix} 6  \\   k  \end{pmatrix}
                               \begin{pmatrix} 43 \\ 6 - k\end{pmatrix}  +
                               \begin{pmatrix} 6  \\   5  \end{pmatrix}                      \\
              &= \begin{pmatrix} 6 \\ 3\end{pmatrix}\begin{pmatrix} 43 \\ 3\end{pmatrix} +
                 \begin{pmatrix} 6 \\ 4\end{pmatrix}\begin{pmatrix} 43 \\ 2\end{pmatrix} +
                 \begin{pmatrix} 6 \\ 5\end{pmatrix}\begin{pmatrix} 43 \\ 1\end{pmatrix} +
                 \begin{pmatrix} 6 \\ 6\end{pmatrix}\begin{pmatrix} 43 \\ 0\end{pmatrix} +
                 \begin{pmatrix} 6 \\ 5\end{pmatrix},                                        \\
           p(A) &= \dfrac{\begin{pmatrix}  6 \\ 3\end{pmatrix}
                          \begin{pmatrix} 43 \\ 3\end{pmatrix}  +
                          \begin{pmatrix}  6 \\ 4\end{pmatrix}
                          \begin{pmatrix} 43 \\ 2\end{pmatrix}  +
                          \begin{pmatrix}  6 \\ 5\end{pmatrix}
                          \begin{pmatrix} 43 \\ 1\end{pmatrix}  +
                          \begin{pmatrix}  6 \\ 6\end{pmatrix}
                          \begin{pmatrix} 43 \\ 0\end{pmatrix}  +
                          \begin{pmatrix}  6 \\ 5\end{pmatrix}}
                         {\begin{pmatrix} 49 \\ 6\end{pmatrix}} =\simeq\num{0.019}.
          \end{align*}
          Všimněte si, že tento výsledek je roven součtu pravděpodobností výhry páté, čtvrté, 
          třetí, druhé a hlavní ceny. Později si tento závěr ještě připomeneme.
  \end{itemize}  
  \normalsize
\end{example}