% !TeX spellcheck = cs_CZ
% Musilova2009MA1@p27
% \wikitextrule
\begin{mdframed}[style=mdexam]
  \begin{example}\label{MAI:exam083} 
    \textbf{Určení hodnosti matice úpravou na schodovitý tvar}\newline
    V matici
    \begin{gather*} 
      \matr{A} = 
        \begin{pmatrix*}[r]
             1 & -2 &  0 &  2  \\
             0 & -3 &  1 & -2  \\
             2 &  5 & -3 & 10  \\
            -2 &  4 &  0 & -4  \\
             0 &  9 & -3 &  6
        \end{pmatrix*}. \sim
        \begin{pmatrix*}[r]
            1 & -2 &  0 &  2  \\
            0 & -3 &  1 & -2  \\
            0 &  9 & -3 &  6  \\
            0 &  0 &  0 &  0  \\
            0 &  9 & -3 &  6
        \end{pmatrix*}. \sim
        \begin{pmatrix*}[r]
            1 & -2 &  0 &  2  \\
            0 & -3 &  1 & -2  \\
            0 &  0 &  0 &  0  \\
            0 &  0 &  0 &  0  \\
            0 &  0 &  0 &  0
        \end{pmatrix*}.
    \end{gather*}
    Při první sérii úprav jsme odečetli dvojnásobek prvního řádku od třetího řádku a dvojnásobek
    prvního řádku přičetli k řádku čtvrtému. Při druhé sérii jsme třetí řádek odečetli od pátého a
    poté trojnásobek druhého řádku přičetli k třetímu. Počet nenulových řádků schodovitého tvaru
    matice je \(\num{2}\), její hodnost je tedy . Již matice sama je tak jednoduchá, že její hodnost
    můžeme stanovit i přímo. Vidíme, že třetí řádek je lineární kombinací prvního a druhého, s
    koeficienty \num{2} a (\num{-3}). Čtvrtý řádek je (\num{-2})-násobkem prvního, je tedy opět
    lineární kombinací prvních dvou, s koeficienty (\num{-2}) a \num{0}. Pátý řádek je součtem
    třetího a čtvrtého, je tedy lineární kombinací prvních dvou řádků s koeficienty \num{0} a
    (\num{-3}). Nezávislé jsou pouze první dva řádky zadané matice, zbývající jsou jejich lineárními
    kombinacemi. Hodnost matice určená přímo tedy opět vychází \(h(\matr{A}) = 2\).
  \end{example}
\end{mdframed}