% !TeX spellcheck = cs_CZ
\wikitextrule
\begin{example}\label{mai:exam063}
  \textbf{Může se člověk živit sázením?}\newline\small
  Sázením sportky nebo návštěvami kasina jistě ne! To snad každý po přečtení předchozích odstavců 
  pochopil. Je však možné docela dobře „vydělat“ sázením se s lidmi. Při odhadu pravděpodobností 
  některých jevů nás často intuice zklame a výpočtem získáme hodnoty, které bychom vůbec 
  neočekávali. Tak například odhadněte bez výpočtu, jaká je pravděpodobnost, že alespoň dva lidé ve 
  vaší třídě (čítající například \(k = 23\) studentů) mají narozeniny ve stejný den. Až tento odhad 
  učiníte, zkuste počítat: Uvažme, že rok má \(n = 365\) dní. Spočítejme nejprve pravděpodobnost 
  \(P'\), že každý ze třídy má narozeniny v jiný den. Počet možných případů odpovídá variacím s 
  opakováním \(n^k\), počet případů příznivých odpovídá variacím bez opakování 
  \(\dfrac{n!}{(n-k)!}\) Máme tedy \(P'=\dfrac{n!}{(n-k)!n^k}\). Jev, který nás zajímá (tj., že 
  alespoň dva mají narozeniny ve stejný den), je jevem opačným. Hledanou pravděpodobností bude
  \begin{equation*}
    P  = 1 - P' = 1 - \dfrac{n!}{(n - k)!n^k} \simeq \num{0.5}.
  \end{equation*}
  Vsadíte-li se na večírku s třiceti a více lidmi, že se mezi vámi najdou dva s narozeninami ve 
  stejný den, je vaše vítězství již téměř zaručeno.
\normalsize
\end{example}