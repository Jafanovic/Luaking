% !TeX spellcheck = cs_CZ
\begin{mathexam}{Určete spektrum matice a její spektrální poloměr následující matice
  \begin{equation*}\label{pr:spektrum_matice}
    \mathbf{A} =
      \begin{pmatrix}
        2  &        2    & 0 \\
       -3  &       -3    & 5 \\
        0  & -\num{0.25} & 2
      \end{pmatrix}
  \end{equation*}
  }{exam013} 
  \textbf{Řešení}: Spektrum matice je množina všech jejích vlastních čísel. Spektrální poloměr je
  maximum z absolutních hodnot vlastních čísel. Vlastní čísla určíme z charakteristické rovnice
  \(\det(\mathbf{A}-\lambda \mathbf{I})=0\).
      \begin{equation*}
        \textbf{A} - \lambda\textbf{I}=
          \begin{pmatrix}
            2-\lambda  &  2          & 0 \\
           -3          & -3-\lambda  & 5 \\
            0          & -0.25       & 2-\lambda
        \end{pmatrix}
      \end{equation*}
      \begin{align}
        \det(\mathbf{A}-\lambda \mathbf{I})                    &= 0           \nonumber\\
        (2-\lambda)
          \begin{pmatrix}
            -3-\lambda  &  5\\
              -0.25    &  2 - \lambda
          \end{pmatrix} -2\cdot
          \begin{pmatrix}
            -3       &  5\\
            0       &  2 - \lambda
          \end{pmatrix}                                        &= 0           \nonumber\\
        (2-\lambda)^2(-3-\lambda)+1.25(2-\lambda)+6(2-\lambda) &= 0           \nonumber\\
        (2-\lambda)[(2-\lambda)(-3-\lambda)+1.25+6]            &= 0           \nonumber\\
        (2-\lambda)(\lambda^2+\lambda+1.25)                    &= 0           \nonumber
      \end{align}
      \begin{equation*}
        \lambda_1 = 2, \quad\lambda_2 = -0.5+i, \quad\lambda_3=-0.5-i
      \end{equation*}
      \begin{itemize}
        \item Spektrum matice \(\mathbf{A}\) je \(\sigma(\mathbf{A})=\{2,-0.5+i,-0.5-i\}\).
        \item Spektrální poloměr \(\rho(\mathbf{A})=\max_i\abs{\lambda_i}=2\).
      \end{itemize}

  %    \attachfile[icon=Paperclip, description=Matlab Determine the spectrum of a matrix 
  %      and its spectral radius]{../SRC/MAI/matlab/LA001.m}
\end{mathexam}