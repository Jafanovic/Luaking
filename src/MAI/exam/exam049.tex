% !TeX spellcheck = cs_CZ
\begin{mdframed}[style=mdexam]
  \begin{example}\label{mai:exam049}
    \textbf{Struktura aditivní grupy na podmnožinách reálné osy}\newline
      Některé významné podmnožiny reálné osy mají svá zavedená označení. \(\naturalset\) je množina 
      přirozených čísel. \(\intset\) množina celých čísel a \(\ratioset\) množina racionálních 
      čísel. Můžeme se zajímat o to, zda při zúžení definičního oboru operace sčítání na 
      \(\naturalset \times \naturalset\), \(\intset \times \intset\), popřípadě \(\ratioset \times 
      \ratioset\), budou tyto množiny stále ještě grupami. Ihned vidíme, že množina přirozených 
      čísel grupou nebude, neboť neobsahuje nulu ani záporná čísla, která jsou v grupě \(\realset\). 
      opačnými prvky k číslům kladným. Tímto nedostatkem netrpí množiny celých a racionálních čísel, 
      které tedy budou grupami.
  \end{example}
\end{mdframed}