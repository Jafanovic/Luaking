% !TeX spellcheck = cs_CZ
\begin{mdframed}[style=mdexam]
  \begin{example}\label{mai:exam095}
    Je dán čtverec \(ABCD\) a na jeho každé straně n vnitřních bodů. Určete počet všech trojúhelníků
    \(XYZ\), jejichž vrcholy leží v daných bodech a na různých stranách čtverce \(ABCD\). Kredit
    \cite[s.~9]{calda2008matematika}.\newline
    \textbf{Řešení}:
    \begin{itemize}[noitemsep]
      \item Vrchol \(X\) je možno zvolit v libovolném z daných bodů, takže pro něj máme \(4n\)
            způsobů výběru. Po výběru bodu \(X\) lze bod \(Y\) vybrat už jen \(3n\) způsoby, neboť
            nesmí ležet na téže straně čtverce jako bod \(X\).
      \item Po výběru bodu \(Y\) je možné bod \(Z\) vybrat pouze \(2n\) způsoby, neboť nesmí ležet
            na těch stranách čtverce \(ABCD\), na nichž leží body \(X\), \(Y\). Existuje tedy
            \(4n\cdot3n \cdot2n = 24n^3\) uspořádaných trojic utvořených z bodů \(X\), \(Y\), \(Z\).
      \item Uvědomme si však, že šest uspořádáných trojic takto sestavených určuje stejný
            trojúhelník. Tak např. každá z uspořádaných trojic \((X, Y, Z)\), \((X, Z, Y)\), \((Y,
            X, Z)\), \((Y, Z, X)\), \((Z, X, Y),\) \((Z, Y, X)\) představuje trojúhelník, a to
            \(\Delta XYZ\). 
      \item Abychom dostali počet všech trojúhelníků požadované vlastnosti, musíme získaný počet
            uspořádaných trojic dělit šesti. Trojúhelníků dané vlastnosti je tedy:
            \begin{equation*} 
              \dfrac{24n^3}{6}
            \end{equation*}
    \end{itemize}

    {\centering
    \captionsetup{type=figure} 
    \luafigure[0.8]{mai_fig069}
    \captionof{figure}{Ilustace k příkladu \ref{mai:exam095}}
    \label{mai:fig069}
    \par}

  \end{example}
\end{mdframed}