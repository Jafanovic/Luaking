\begin{mathexam}{\(\protect\scalerel{\int}{\dfrac{5x+8}{x^2+4x+7}\dd{x}}\)
  \hfill\cite[s.~73]{Knichal}}{mai:exam140} 
  Kvadratický trojčlen ve jmenovateli má záporný diskriminatn, \(16-4\cdot7 = -12<0\), a proto jde
  skutečně o integrál typu 3.
  
  Postup, který jsme ukázali obecně, provedeme nyní na speciálním případě. Nejdříve rozložíme
  integrand na dva zlomky tak, aby první měl v čitateli derivaci jmenovatele: 
  \begin{equation*}
    \frac{5}{2}\frac{2x+4}{x^2+4x+7} + \left(8-\frac{5}{2}\cdot4\right)\frac{1}{x^2+4x+7}.
  \end{equation*}
  Tedy
  \begin{align*}
    & \int\dfrac{5x+8}{x^2+4x+7}\dd{x}  = \dfrac{5}{2}\int\frac{2x+4}{x^2+4x+7}\dd{x}  \\
    &-2\int\frac{1}{x^2+4x+7}\dd{x}     = \dfrac{5}{2}\ln(x^2+4x+7) - 2I
  \end{align*}
  Abychom určili \(I\) budeme postupovat takto:
  \begin{equation*}
    I = \int\dfrac{1}{x^2+4x+7}\dd{x} = \int\dfrac{1}{(x+2)^2+3}\dd{x} 
  \end{equation*}
  Nyní položíme \(x+2=t\), takže \(\dd{x}= \dd{t}\). Dostaneme
  \begin{align*}
    I &= \int\dfrac{1}{t^2+3}\dd{t} = \dfrac{1}{\sqrt{3}}\arctan\dfrac{t}{\sqrt{3}} + c \\
      &= \dfrac{1}{\sqrt{3}}\arctan\dfrac{x+2}{\sqrt{3}} + c
  \end{align*}
  Výsledek integrálu \(\int\dfrac{5x+8}{x62+4x+7}\dd{x}\) tedy je 
  \begin{equation*}
    \dfrac{5}{2}\ln(x^2+4x+7) - \dfrac{2}{\sqrt{3}}\arctan\dfrac{x+2}{\sqrt{3}} + c
  \end{equation*}
\end{mathexam}