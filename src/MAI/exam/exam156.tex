% https://users.math.cas.cz/~rehak/soubory/urc_int.pdf
\begin{mathexam}{Prozkoumejme, jak je to s (ne)existencí primitivní funkce k funkci
  \begin{equation*}
    \sgn = 
      \begin{cases}
         -1 & \text{pro} x<0, \\
          0 & \text{pro} x=0, \\
          1 & \text{pro} x>0.
      \end{cases}
  \end{equation*}
  na intervalech obsahujících a neobsahujících \(0\)}{exam156} 

  {\centering
    \captionsetup{type=figure} 
    \luafigure[0.5]{mai_fig079.pdf}
    \captionof{figure}{Graf funkce \(\sgn x\)}
    \label{mai:fig079}
  \par}
  
  Podobně jako v příkladu \ref{mai:exam154} snadno zdůvodníme, že k funkci \(\sgn x\) neexistuje
  primitivní funkce na žádném intervalu obsahujícím \(0\). Na jakémkoli intervalu neobsahujícím
  \(0\) primitivní funkce podle věty \eqref{mai:lemma011} existuje (z cvičných důvodů určete
  nějakou, příp. všechny). 
\end{mathexam}