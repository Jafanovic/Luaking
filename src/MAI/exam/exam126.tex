\begin{mathexam}{Nechť \(\int{f(x)}\dd{x} = F(x) + c, \quad a,b\in\realset, a\neq0\)
  \hfill\cite[s.~261]{Brabec1989}.}{exam126}
  Pak platí
  \begin{equation}\label{mai:eq138}
    \int{f(ax + b)}\dd{x} = \frac{1}{a}F(ax + b) + c.
  \end{equation}
  Položíme \(ax + b =z\). Odtud \(a\dd{x} = \dd{z}\), \(\dd{x} = \frac{1}{a}\dd{z}\); 
  \begin{equation*}
    \boxed{\frac{1}{a}\int{f(z)}\dd{z} = \frac{1}{a}F(z) + c = \frac{1}{a}F(ax+b) + c.}
  \end{equation*}
  Například \(\int{\sin(2x+1)}\dd{x} = -\frac{1}{2}\cos(2x + 1) + c\) nebo \(\int{e^{-x}}\dd{x} =
  -e^{-x} + c\). 
\end{mathexam}