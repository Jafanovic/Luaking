% !TeX spellcheck = cs_CZ
\wikitextrule
\begin{example}\label{mai:exam047}
  \textbf{Algebraická struktura a kupecké počty}\newline\small
    Uvažujme o množině reálných čísel \(G = \mathbb{R}\) tak jako bychom je uměli jen 
    sčítat. Násobení si zatím nevšímejme. Sčítání reálných čísel je zobrazení typu prvního 
    vztahu v \ref{mai:eq047}, které bezpochyby splňuje požadavky \ref{mai:eq048} a 
    \ref{mai:eq049}. Neutrálním prvkem \(0_\mathbb{R}\) je \uv{obyčejná} nula, opačným prvkem k 
    číslu \(a\) je \(-a\), položené na reálné ose symetricky k \(a\) vzhledem k nule. Množina 
    reálných čísel s operací sčítání je tedy aditivní grupou. Pro operaci sčítání dokonce platí 
    něco navíc - komutativní zákon
    \begin{equation}\label{mai:eq051}
      a + b = b + a \qquad\text{pro libovolné} a,b\in\mathbb{R}.
    \end{equation} 
    Grupu s komutativním zákonem nazýváme grupou \emph{komutativní} nebo také \textbf{abelovskou}.
    
    Množina reálných čísel s operací sčítání je komutativní grupou.
    
    Nyní se místo na sčítání zaměřme na násobení reálných čísel a znovu posuďme vlastnosti grupy. 
    Násobení reálných čísel je zobrazením typu druhého vztahu v \ref{mai:eq047} a splňuje požadavek 
    asociativnosti \ref{mai:eq048}. Dále je zřejmé, že číslo \(e_\mathbb{R}\) (\uv{obyčejná} 
    jednička) vyhovuje prvnímu požadavku ve vztazích \ref{mai:eq050}. Potíž je s požadavkem druhým. 
    Inverzní prvek najdeme jen k nenulovým číslům. Nula inverzní prvek nemá. Tato zdánlivá drobnost 
    je příčinou toho, že \textbf{množina reálných čísel s operací násobení není grupou}.
  \normalsize
\end{example}