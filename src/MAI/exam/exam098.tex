  % !TeX spellcheck = cs_CZ
% Musilova2009MA2
\begin{mdframed}[style=mdexam]
  \begin{example}\label{mai:exam098}
    \textbf{Směrové pole}\newline
      Nakreslete směrové pole, izokliny a integrální křivky rovnice
      \begin{equation*}
        \dot{x} = 2t, \quad\text{tj.}\quad f(t,x) = 2t
      \end{equation*}
      Izoklinami jsou množiny bodů \(t=\text{konst.}\), tj. přímky rovnoběžné s osou \(x\). Řešením
      je každá funkce \(x(t) = t^2 + c\), kde \(c\) je libovolná konstanta. Integrálními křivkami
      rovnice jsou paraboly odlišené konstantou \(c\), určující polohu vrcholu paraboly. Tyto
      paraboly tvoří jednoparametrickou soustavu křivek s parametrem \(c\). Situace je na obrázku
      \ref{mai:fig068}.
      
      {\centering
      \captionsetup{type=figure}
      \luafigure[1]{mai_fig068.png}
      \captionof{figure}{Směrové pole, izokliny a integrální křivky rovnice \(\dot{x} = 2t\). 
                         \cite[s.~231]{Musilova2012MA2}}
      \label{mai:fig068}
      \par}
  \end{example}
\end{mdframed}