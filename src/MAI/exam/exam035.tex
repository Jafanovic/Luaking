% !TeX spellcheck = cs_CZ
% \wikitextrule
\begin{mdframed}[style=mdexam]
  \begin{example}\label{mai:exam035}
    \textbf{Fyzika - Ohmův zákon:}\newline
    Z elektřiny víme, že některé vodiče či elektrické prvky se při průchodu elektrického proudu 
    chovají podle zákona linearity: Proud, který jimi protéká, závisí přímo úměrně na přiloženém 
    napětí (obr. \ref{mai:fig036}). Platí \(I(U) = R^{-1 }U\) s konstantou úměrnosti \(R^{-1}\), kde 
    \(R\) je elektrický odpor vodiče (prvku).

    Pozn. 1: Předpokládáme, že elektrický odpor voltmetru je tak velký, že proud jím procházející je 
    z hlediska přesnosti měření zanedbatelný.
    
    Pozn. 2: Graf závislosti proudu na napětí na obrázku \ref{mai:fig036} může pro vyšší hodnoty 
    napětí vykazovat odchylku od linearity (přímkové závislosti), neboť se prvek při vyšším proudu 
    zahřívá a jeho odpor roste.
    
    {\centering
      \captionsetup{type=figure}
      \luafigure[1]{mai_fig036.png}
      \captionof{figure}{Chování lineárního vodiče (Ohmův zákon). \cite[s.~15]{Musilova2009MA1}
      \label{mai:fig036}}
      \par}  
  \end{example}
\end{mdframed}