% !TeX spellcheck = cs_CZ
\begin{mdframed}[style=mdexam]
  \begin{example}\label{mai:exam050}
    \textbf{Grupy některých číselných objektů}\newline
      Komutativní grupou je například množina číselných matic typu \(m/m\) (osazených reálnými nebo
      komplexními čísly) s obvyklým sčítáním matic. Není divu sčítání matic se děje po jednotlivých
      prvcích, které jsou čísly. Komutativní grupou je i množina všech funkcí reálné proměnné
      \(\mathcal{F}(\realset)\), v níž je sčítání definováno \uv{bod po bodu}
      \begin{equation*}
        \mathcal{F}(\realset)\times\mathcal{F}(\realset)\ni[f,g]\longrightarrow
        f + g \in\mathcal{F}(\realset), \qquad\text{kde } (f+g)(x) = f(x) + g(x)
      \end{equation*}
      pro libovolné \(x \in\realset\). Ve výsledku jde totiž opět jen o počítaní s čísly. Je zřejmé,
      že proměnná by mohla být i komplexní. Na množině všech reálných funkci jedné reálné proměnné s
      definičním oborem \(\realset\) máme z kapitoly \ref{mai:IchapII} definovanou operaci skládání
      funkci. Tato operace je asociativní, funkce \(f(x) = x\) (identita) hraje roli neutrálního
      prvku. Radost nám kazí pouze ty funkce, ke kterým neexistuje funkce inverzní s definičním
      oborem \(\realset\), tj. funkce, které nejsou prosté nebo jejich oborem hodnot není celé
      \(\realset\). Vezmeme-li v úvahu pouze podmnožinu prostých funkci s oborem hodnot
      \(\realset\), vytváří na ni operace skládání strukturu (nekomutativní) grupy. 
      
      Posuďme nyní algebraickou strukturu množiny čtvercových matic typu \(n/n\) s operaci
      maticového násobení. Operace je zobrazením, které má tvar druhého vztahu v (\ref{mai:eq047}),
      asociativní zákon (\ref{mai:eq048}) rovněž platí. Jednotková matice hraje roli prvku \(e_G\),
      avšak inverzní prvek existuje pouze k regulárním maticím. Množina matic typu \(n/n\) s operaci
      maticového násobení tedy není grupou. Pokud bychom se však omezili pouze na podmnožinu matic
      regulárních, potíž s inverzním prvkem odpadne. 
      
      \begin{mdframed}[style=highlight]
        Množina regulárních čtvercových matic typu n/n s operaci násobení je (nekomutativní) grupou.
      \end{mdframed}
      
      Tato grupa, ke které se později ještě vrátíme, je velmi důležitá ve fyzikálních teoriích. 
  \end{example}
\end{mdframed}