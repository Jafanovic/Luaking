% !TeX spellcheck = cs_CZ
\begin{mdframed}[style=mdexam]
  \begin{example}\label{mai:exam075}
    \textbf{Záhada přijímací zkoušky aneb k čemu může posloužit distribuční funkce?}\newline
    Mohlo by se zdát, že distribuční funkce je jen teoretický pojem a že v praktických situacích ji
    těžko využijeme. Podstatné je přece pravděpodobnostní rozdělení náhodné veličiny a distribuční
    funkce je z něj jen jaksi odvozena sčítáním pravděpodobností (u diskrétního rozdělení) nebo
    integrací (u rozdělení spojitého). Přesvědčíme se, že existují velmi realistické případy, kdy
    distribuční funkce přináší věrohodnější informaci o náhodné veličině než samotné rozdělení.
    
    Na Masarykově univerzitě musí každý uchazeč o studium, ať již se hlásí na přírodovědeckou,
    právnickou, lékařskou či jinou fakultu, absolvovat Test studijních předpokladů. Jedná se o
    všeobecný test, zaměřený na zjišťování úrovně všech schopností uchazeče, které jsou potřebné pro
    univerzitní studium, například analytického myšlení, verbálních schopností, numerického myšlení,
    geometrické představivosti, atd. Pro nás však v tu to chvíli není podstatný obsah testu, ale
    způsob zpracování jeho výsledků a vyhodnocení pořadí uchazečů. Test skládá kolem třiceti tisíc
    studentů. Není tedy možné technicky zajistit, aby proběhl v jediné variantě v jednom dni. K
    dispozici je proto osm variant testu, každou variantu řeší tři až čtyři tisíce studentů. Test má
    \num{80} otázek, základním údajem pro zpracování jeho výsledků je počet správných odpovědí
    každého studenta. Pokud bychom označili jako \(i\) počet správných odpovědí (\(i \in\lbrace0, 1,
    2, \ldots, 80\rbrace\)) v kterékoli variantě a \(\mathcal{N}_i\) počet studentů, kteří dosáhli
    právě \(i\) správných odpovědí, dostaneme náhodnou veličinu \(X_i\), kterou bychom mohli nazvat
    „počet správných odpovědí“, pro celou univerzitu. Její rozdělení by mělo tvar
    \begin{equation*}
      \lbrace(i,p_i)\rbrace,\quad\text{kde}\qquad p_i = \dfrac{\mathcal{N}_i}{\mathcal{N}}, \quad
      \mathcal{N} = \sum_{i=0}^{80}\mathcal{N}_i
    \end{equation*}
    A zde je malý „kámen úrazu“. A by bylo možné sestavit opravdu „univerzální pořadí“, musely by
    být všechny varianty testu ekvivalentní z hlediska obtížnosti. To znamená, že kdyby kterýkoli
    student vyplnil za stejných podmínek všechny varianty, dosáhl by v každé z nich stejného počtu
    správných odpovědí s pravděpodobností velmi blízkou jedné. Skutečnost je však principiálně
    taková, že u sebelépe promyšleného a sestaveného testu se jednotlivé varianty budou mírně, v
    rámci statistických, a tedy již neodstranitelných, odchylek lišit. Tato odlišnost se nepozná
    předem, ale až po zpracování výsledků všech variant. Použít pro stanovení pořadí uchazečů
    rozdělení náhodné veličiny \(X\) \emph{= počet správných odpovědí je tedy nespravedlivé}.
    Student, který řešil variantu „statisticky obtížnější“, by v pořadí skončil s horším umístěním,
    než student, který je stejně schopný, avšak měl to štěstí, že na něj připadla varianta
    „statisticky méně obtížná“. Skutečně, kdybychom sestavili grafy rozdělení náhodných veličin
    \(X^{(\alpha)}\) \emph{= počet správných odpovědí v \(\alpha\)-té variantě},
    \begin{equation*}
      \lbrace(i,p_i^{(\alpha)})\rbrace,\quad\text{kde}\; 
      p_i^{(\alpha)} = \dfrac{\mathcal{N}_i^{(\alpha)}}{\mathcal{N}^{(\alpha)}}, \quad
      \mathcal{N}^{(\alpha)} = \sum_{i=0}^{80}\mathcal{N}_i^{(\alpha)}
    \end{equation*}
    zjistili bychom, že se mírně liší. (V předchozím zápisu značí \(\mathcal{N}_i^{(\alpha)}\) počet
    studentů, kteří odpověděli správně na \(i\) otázek \(\alpha\)-té varianty
    \(\mathcal{N}^{(\alpha)}\) je počet všech studentů, kteří tuto variantu řešili.) Střední hodnoty
    i mediány náhodných veličin se i při vynikající shodě obtížnosti všech variant mohou lišit v
    rozmezí jedné až dvou správných odpovědí. A s ohledem na skutečnost, že každou variantu řeší
    obrovský počet studentů, až čtyři tisíce, je zřejmé, že tento rozdíl může poněkud „zamíchat
    “pořadím, zejména v blízkosti mediánu, kde se týká třeba i tří stovek studentů v každé variantě.
    Situaci dokládá obrázek \ref{mai:fig049}. Jak tedy zařídit, abychom dostali spravedlivé pořadí?
    Jediný rozumný způsob, jak minimalizovat vliv statistických odchylek obtížnosti jednotlivých
    variant, je nehodnotit studenty podle absolutního počtu správných odpovědí, ale nějak je
    porovnat mezi sebou. Budeme při tom předpokládat, že rozložení schopností studentů je ve všech
    osmi skupinách, které řeší osm daných variant, stejné. Řeknete si - zase nějaké další
    předpoklady. To je jako z bláta do louže. Předpoklad o stejném rozložení schopností studentů v
    tak velkých skupinách, jako jsou ty naše, je však mnohem realističtější než předpoklad o
    dokonalé shodě obtížnosti variant testu. Budeme se jej proto držet. Každému studentovi
    přisoudíme číslo, které informuje o tom, kolik řešitelů dané varianty bylo horších nebo stejně
    dobrých jako on, tj. mělo nižší nebo stejný počet správných odpovědí. Z matematického hlediska
    to znamená přejít v každé variantě od rozdělení k distribuční funkci. Věnujme se nyní tomuto
    přepočtu podrobněji jak pro diskrétní rozdělení náhodné veličiny \(X^{(\alpha)}\), které
    odpovídá skutečné situaci, tak pro zajímavost i pro rozdělení spojité. V dalším budeme vždy
    zpracovávat výsledky jedné varianty, upustíme proto od vyznačování indexu \(\alpha\).

    {\centering
    \captionsetup{type=figure}
    \luafigure[0.8]{mai_fig049.png}
    \captionof{figure}{Rozdělení pro dvě varianty testu,
    \cite[s.~252]{Musilova2009MA1}
    \label{mai:fig049}}
    \par}
    
    \textbf{Diskrétní rozdělení}
      \begin{itemize}
        \item \emph{Zadání:} Skupina \(N\) studentů řeší jednu variantu testu. Test má \(Q\) otázek.
              Za každou správnou odpověď je přidělen jeden výchozí bod. Získáváme rozdělení
              \begin{gather*}
                \left\lbrace\left(i, \dfrac{N_i}{N} \right)\right\rbrace, \;
                i\in\lbrace0, 1, 2, \ldots, Q\rbrace, \;
                \sum_{i=0}^{Q}N_i = N 
              \end{gather*}
              kde \(i\) je počet výchozích bodů a \(N_i\) počet studentů, kteří získali \(i\) bodů.
              Distribuční funkce tohoto rozdělení
              \begin{gather*}
                F(x) = \dfrac{1}{N}\sum_{i=0}^{j}N_i\;\text{ pro }\;
                j\leq x < j+1, \; x\in\left[0,\infty\right)
              \end{gather*}
              Pro uchazeče, který získal \(j\) bodů, mají význam následující hodnoty:
              \begin{itemize}
                \item \(F(x)    x \in \left[j, j+1\right)\): poměrný počet uchazečů, kteří získali
                      počet výchozích bodů nižší nebo shodný s daným uchazečem,
                \item \(NF(x)   x \in \left[j, j+1\right)\): absolutní počet uchazečů, kteří získali
                      počet výchozích bodů nižší nebo shodný s daným uchazečem,
                \item \(100F(x) x \in \left[j, j+1\right)\): absolutní počet uchazečů, kteří získali
                      počet výchozích bodů nižší nebo shodný s daným uchazečem,
              \end{itemize}
        \item Hodnoty distribuční funkce můžeme získat z následující tabulky:
        
              {\centering
                \resizebox{0.8\textwidth}{!}{%
                \begin{tabular}{c|c}
                          interval \(x\)     &  \(NF(x)\)         \\ \hline
                  \(\left[0,1\right)\)       &  \(N_0\)           \\ 
                  \(\left[1,2\right)\)       &  \(N_0 + N_1\)     \\ 
                  \(\cdots\)                 &  \(\cdots\)        \\
                  \(\left[j,j+1\right)\)     &  \(N_0 + N_1 + \cdots + N_j\) \\ 
                  \(\cdots\)                 &  \(\cdots\) \\
                  \(\left[Q-1,Q\right)\)     &  \(N_0 + N_1 + \cdots + N_{Q-1}\) \\ 
                  \(\left[Q,\infty\right)\)  &  \(N_0 + N_1 + \cdots + N_{Q} = N\) 
                \end{tabular}}
              \par}
        \item Přepočet hodnocení uchazečů tak, aby nová stupnice byla opět v rozsahu mezi nulou a
              \(Q\) a aby nové hodnocení bylo opět celočíselné, je následující:
              \begin{equation*}
                y =QF(x), \quad 0\leq F(x) \leq 1, \Rightarrow y \in[0,Q].
              \end{equation*}
              Uchazeči se ziskem \(i\) výchozích bodů náleží hodnota \(y = QF(x)\) právě když i\(
              \in \left[i, i + 1\right)\), tj. \(y_i = Q F(i)\). Tato hodnota není obecně
              celočíselná. Zaokrouhlení se provede ve prospěch uchazeče, tedy vždy nahoru. Výsledný
              převodní vzorec je
              \begin{equation*}
                \begin{multlined}
                  \text{výchozí body } i \longrightarrow\text{ nové body }  \\ 
                  \shoveleft[1cm]Y_i: Y_i = [y_i] + 1 = [QF(i)] + 1,
                \end{multlined}
              \end{equation*} 
              kde \([a]\) značí celočíselnou část čísla \(a\), tedy například \([\num{23.05}] =
              [\num{23.48}] = [23,89] = 23\)
        \item Zaveďme novou náhodnou veličinu \(Z\) s rozdělením \(\left\lbrace(z_\alpha,
              M_\alpha)\right\rbrace\): Označme \(z_1, z_2, \ldots, z_\alpha, ...,\) \(z_S\)
              navzájem různé hodnoty ze souboru \(\lbrace Y_i\rbrace, i = 0, 1, 2, \ldots, Q\)
              řazené vzestupně. Její rozdělení udává kterákoli z následujících tabulek:

              {\centering
                \resizebox{0.9\textwidth}{!}{%
                \begin{tabular}{c|c}
                          hodnota                       &  četnost         \\
                          \hline
                  \(z_1 = Y_0 = Y_1 = \cdots Y_{i_1}\)  & \(M_1 = N_0 + N_1 + \cdots + N_{i_1}\)  \\ 
                  \(Z_2 = Y_{i_1+1} = \cdots Y_{i_2}\)  & \(M_2 = N_{i_1+1} + \cdots + N_{i_2}\)  \\ 
                  \(\cdots\)                            & \(\cdots\)                              \\
                  \(Z_S = Y_{i_{S-1}+1}=\cdots Y_{i_S}\)& \(M_S = N_{i_{S_1}+1}+\cdots+N_{i_S}\)   
                \end{tabular}}
              \par}

              {\centering
                \resizebox{0.9\textwidth}{!}{%
                \begin{tabular}{c|c}
                          hodnota                        &  četnost         \\
                          \hline
                  \(z_1 = Y_0 = Y_1 = \cdots Y_{i_1}\)   &  \(M_1 = NF(i_1)\)              \\
                  
                  \(Z_2 = Y_{i_1+1} = \cdots Y_{i_2}\)   &  \(M_2 = N[F(i_2) - F(i_1)]\)  \\ 
                  \(\cdots\)                             &  \(\cdots\)                     \\
                  \(Z_S = Y_{i_{S-1}+1}=\cdots Y_{i_S}\) & \(M_S = N[F(i_S) - F(i_{S-1})]\)   
                \end{tabular}}
              \par}
              kde \(i_1 < i_2 < \ldots < i_{S-1} < i_S, i_S = Q\) (Vzhledem k zaokrouhlování nahoru
              není žádná bodová hodnota \(Y_i\) nulová.) I když skutečné rozdělení při zpracování
              výsledků testů je diskrétní, ukažme si, jak by vypadal analogický postup u rozdělení
              spojitého, kde je početní zpracování názornější.
      \end{itemize}
    
    \textbf{Spojité rozdělení}
      \begin{itemize}
        \item \emph{Zadání:} Je dáno rozdělení četností \(n(x) \leq 0,\, x \in [0, Q]\).
        \item Normovací podmínka a distribuční funkce jsou
              \begin{equation*}
                \int_{0}^{Q}n(x)\dd{x} = N, \quad 
                F(x) = \dfrac{1}{n}\int_{0}^{x}n(\xi)\dd{\xi},                
              \end{equation*}
              kde \(0 \leq F(x) \leq 1.\)
        \item Označme \(z = QF(x)\), tedy \(z \in [0, Q]\), novou náhodnou veličinu. (Uvědomme si,
              že \(z\) je rostoucí funkcí proměnné \(x\)). Označme její rozdělení \(\nu(z)\). Její
              distribuční funkce je
              \begin{align*}
                \Phi(z) &= \int_{0}^{z}\nu(\zeta)\dd{\zeta} 
                         = \int_{0}^{x(z)}\dfrac{n(\xi)}{N}\dd{\xi}          \\
                        &= F\left(F^{-1}(z/Q)\right) 
                        = \dfrac{z}{Q}, \quad
                \nu(z) = \dfrac{1}{Q}.
              \end{align*}
        \item Rozdělení je konstantní s mediánem i střední hodnotou \(Q/2\). Takové rozdělení se
              nazývá rovnoměrné.
      \end{itemize}
  \end{example}
\end{mdframed}