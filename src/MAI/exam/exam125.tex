\begin{mdframed}[style=mdexam]
  \begin{example}\label{MAI:exam125} 
    Řešme: 
    \begin{equation*}
      \int(1+x^2)^5x\dd{x}.
    \end{equation*}

    \noindent\textbf{Řešení:}

    Zavedeme substituci \(1+x^2 = u\); odftud \(2x\dd{x} = \dd{u}\), tj.
    \(x\dd{x}=\frac{1}{2}\dd{u}\) (používáme jiné označení proměnných než je v uvedené větě). Potom
    platí 
    \begin{equation*}
      \frac{1}{2}\int u^5\dd{u} = \frac{1}{12}u^6 + c = \frac{1}{12}(1+x^2)^6 + c,
    \end{equation*}
    kde \(x\in(-\infty, +\infty)\), \(u\in\langle 1,+\infty)\); podmínky věty o substituci jsou
    zřejmě splněny. \cite[s.~261]{Brabec1989}. 
  \end{example}
\end{mdframed}