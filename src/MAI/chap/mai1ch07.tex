% !TeX spellcheck = cs_CZ
%---------------------------------------------------------------------------------------------------
% file: mai1ch07.tex
%---------------------------------------------------------------------------------------------------
%============================ Primitivní funkce ====================================================
% \setcounter{chapter}{1000}
% \setcounter{page}{43210}
\setchaptertoc
\chapter{Poznáváme funkci z její derivace - neurčitý integrál}\label{mai:IchapVII}
  %----------------------------Neurčitý integrál----------------------------------------------------
  \section{Motivace}
    Problém \emph{neurčitého integrálu}, neboli \textbf{primitivní funkce}, lze vyložit velmi 
    jednoduše: Máme podezření, že zadaná funkce \(f(x)\) vznikla derivováním jisté, zatím neznámé, 
    funkce \(F(x)\). Dokážeme ji najít? 
  
    K danému problému můžeme přistupovat také fyzikálně: Zavedením pojmu derivace funkce jsme 
    motivovali důležitým požadavkem definovat okamžitou rychlost pohybu bodu po přímce. Existuje 
    přirozeně i požadavek opačný, tj. nalézt zákon dráhy pohybu bodu po přímce, je-li dána jeho 
    okamžitá rychlost jako funkce času \cite[s.~253]{Brabec1989}. Vše si ukážeme na následujícím 
    příkladu \eqref{mai:exam117}.      

    Každé takové funkci, jejíž derivací je daná funkce, budeme říkat \emph{primitivní funkce} k 
    dané funkci. Na uvedeném příkladě je patrné, že k dané funkci může existovat nekonečně mnoho 
    primitivních funkcí. Množinu všech primitivních funkcí se často nazývá \textbf{neurčitým 
    integrálem}. Po tomto názorném uvedení do problému přejděme k přesné formulaci základních pojmů.
    %-------------------------------------
    % !TeX spellcheck = cs_CZ
%====================== Sbírka řešených příkladů ==================================================
\begin{mathexam}{Je dána okamžitá rychlost \(v\) pohybu bodu po přímce (ose) \(x\) rovnicí \(v(t) =
  2t + 1\), \(t\in\langle -\infty,+\infty \rangle\). Najděte zákon dráhy pohybu, je-li známo, že v
  čase \(t = 0\) měl bod polohu \(x = x_0\) \cite[p.~253]{Brabec1989}.}{exam117}

  Označíme-li \(x(t)\) polohu bodu v okamžiku \(t\), pak \(v(t) = \frac{dx}{dt}\). Hledáme tedy
  funkci \(x = x(t)\), pro níž platí \[\frac{dx}{dt} = 2t + 1 \qquad x(0) = x_0.\] Je ihned patrné,
  že první podmínce vyhovuje nekonečně mnoho funkcí
  \begin{equation}\label{MA:int_ex_09}
    x(t) = t^2 + t + c, 
  \end{equation}
  kde \(c\) je libovolná konstanta. Funkce, která splňuje i druhou podmínku (říkáme ji též počáteční
  podmínka), najdeme z rovnice \ref{MA:int_ex_09} dosazením dané podmínky \(t = 0\), \(x = x_0\).
  Dostaneme \(x_0 = c\). Dosazením do \ref{MA:int_ex_09} za \(c\) plyne hledaný zákon dráhy \(x(t) =
  t^2+t+x_0\).                 

  Jednoduchou zkouškou se přesvědčíme, že tato funkce splňuje obě dané podmínky a zároveň vidíme, že
  hledaná primitivní funkce daných vlastností je jediná.
\end{mathexam}
    %-------------------------------------
    
    \subsection{Definice primitivní funkce}\label{mai:IchapVIIsecI}    
      \begin{mathdef}{Primitivní funkce}{def002}
        Funkci \(F(x): J\rightarrow \realset\), kde \(J\subset \realset\) je interval, nazveme
        \textbf{primitivní} funkcí k funkci \(f(x)\) na intervalu \(J\) právě když, pro všechna
        \(x\in J\) je \(F'(x) = f(x)\). Nebo což je totéž, je-li \(f(x)\dd{x}\) pro každé \(x\in J\)
        diferenciálem funkce \(F(x)\),
        \begin{equation*}
          \dd{F}(x) = f(x)\dd{x}.
        \end{equation*}        
        V případě uzavřeného intervalu \(J=[a,b]\) požadujeme ještě, aby obě jednostranné derivace
        splňovaly \(F_+'(a)=f(a)\) a \(F_-'(b)=f(b)\). 
      \end{mathdef}

      Zřejmě, je-li \(F(x)\) primitivní funkcí k \(f(x)\) v intervalu \(a,b\) (kde může být
      popřípadě \(a=-\infty\), \(b=+\infty)\), je také primitivní funkcí k \(f(x)\) v každém
      intervalu \(\alpha, \beta\), tj. pro nejž platí \(a\leq\alpha<\beta\leq b\). 

      \begin{figure}[ht!]
        \centering
        \animategraphics[controls,autoplay,loop]{2}{mai_fig022}{}{}
        \caption{Graf funkce \(f(x) = x^2\)}
        \label{mai:fig099}
      \end{figure}
      V dalším textu poznáme, že primitivní funkce mají velmi důležitou úlohu v matematice a v
      jejich aplikacích. Řešení mnoha úloh z integrálního počtu a příbuzných partií vede k určení
      primitvní funkce. Uvidíme například, že najdeme-li k funkci \(f(x)= x^2\) primitivní funkci
      \(F(x) = x^3\), vypočítáme velmi snadno obsah \(P\) obrazce, který je vyšrafován na obr.
      \ref{mai:fig099}.
      \begin{equation*}
        P = F(2) - F(0) = \frac{2^3}{3} = \frac{8}{3}.
      \end{equation*}
      Užitečnost získání primitivní funkce vysvitne zejména v kapitole \ref{mai:IchapVIII}. Metodám
      hledání funkce je věnován následující text. Uvidíme, že určení primitivní funkce k dané
      elementární funkci je v obecnémpřípadě úloha podstatně obtíšžnější než určení derivace.  
 
      Uvedeme nyní některé jednoduché příklady.
      %-------------------------------------
      % !TeX spellcheck = cs_CZ
%====================== Sbírka řešených příkladů ==================================================
\begin{mdframed}[style=mdexam]
  \begin{example}\label{mai:exam118}
    K funkci $\sin x$ je primitivní funkcí na libovolném intervalu $J\subset(-\infty,+\infty)$ 
    funkce $-\cos x$, protože $(-\cos x)' = \sin x$. Ale též funkce $3-\cos x$ je primitivní 
    funkcí k funkci $\sin x$, protože $(3 - \cos x)' = \sin x$ pro všechna $x\in(-\infty, 
    \infty)$.
  \end{example}
\end{mdframed}
      %-------------------------------------

      Často používáme pro nezávisle proměnnou jiného oznčení než \(x\). Funkcí primitivní k funkci
      \(f(u)\) rozumíme pak ovšem funkci \(F(u)\), pro kterou Platí
      \begin{equation*}
        F'(u) = \diff{F}{u} = f(u),
      \end{equation*}
      a podobně při jiném označení. Například funkce \(F(u) = \tan u + 3\) je pro \(u\in(-\pi/2,
      \pi/2)\) primitivní funkcí k funkci \(f(u)=\dfrac{1}{\cos^2u}\), neboť pro každé \(u\in(-\pi/2,
      \pi/2)\) platí \(F'(u)=(\tan u + 3)' = \dfrac{1}{\cos^2u}\).

      Primitivní funkce k funkci \(f(x)=3x^2\), \(x\in(-\infty, \infty)\) je funkce \(F(x)=x^3\),
      neboť pro všechna \(x\) platí \((x^3)'=3x^2\). Avšak také funkce \(G(x) = x^3+5\) nebo funkce
      \(H(x)=x^3-2\pi\sqrt{3}\) a vůbec každá funkce \(R(x)=x^3+C\), kde \(C\) je libovolná (reálná)
      konstanta, je v intervalu \((-\infty, +\infty)\) primitivní funkcí k funkci \(f(x)=3x^2\),
      neboť \(G'(x)=H'(x)=R'(x)=3x^2\) pro každé \(x\in(-\infty, +\infty)\). 

      Z posledních příkladů je zřejmé, že známe-li (v uvažovaném intervalu) k dané funkci \(f(x)\)
      jednu primitivní funkci, známe jich nekonečně mnoho; je-li totiž \(F(x)\) jedna z nich, pak
      rovněž každá funkce tvaru \(F(x)+c\), kde \(c\) je libovolné reálné číslo, je primitivní
      funkcí k dané funkci \(f(x)\), neboť
      \begin{equation*}
        \bigl[F(x) + c\bigl] = F'(x) = f(x).
      \end{equation*}
      Vzniká nyní otázka: Najdeme-li k dané funkci \(f(x)\) primitivní funkci \(F(x)\), neexistují
      pak kromě funkcí tvaru \(F(x) + c\) ještě jiné primitivní funkce? Nebo jinak řečeno:
      Neexistují k dané funkci dvě různé primitvní funkce, lišící se jinak než o konstantu? Odpověď
      je záporná: Výrazem \(F(x) + c\) jsou vyčerpány všechny primitivní funkce k dané funkci
      \(f(x)\). Platí věta: 

      \begin{mathlemma}{Věta o množině všech primitivních funkcí}{lemma008}
        Je-li \(F(x)\) v intervalu \(J\) primitivní funkcí k funkci \(f(x)\), pak \textbf{každá}
        primitivní funkce k funkci \(f(x)\) v intervalu \(J\) je tvaru \[F(x) + c,\] kde \(c\) je
        (reálná) konstanta.
        \tcblower
        \begin{proof}
          Tvrzení a) Nechť \(F(x)\) a \(G(x)\) jsou dvě libovolné primitivní funkce k funkci
          \(f(x)\) v intervalu \(J\). Máme dokázat, že se liší nejvýše o konstantu, tj. že platí 
          \[G(x) - F(x) = c,  \qquad x\in J.\] Podle definice primitivní funkce platí:
          \begin{align*}
            F'(x) &= f(x), \qquad G'(x) = f(x)   \\
            \intertext{Funkce \(F(x)\) a \(G(x)\) mají tedy pro všechna \( x\in J\) stejná derivace, 
              a tedy funkce}
            R(x)  &= G(x) - F(x), \qquad x\in J.  \\
            \shortintertext{má derivea rovnou nule.}
            R'(x) &= G'(x) - F'(x) = f(x) - f(x) = 0
          \end{align*}
          pro všechna \( x\in J\). Je-li však \(R'(x)=0\) pro všechna \(x\) z daného intervalu je
          \(R(x)=c\) v \(J\). Tedy \(R(x) = G(x) - F(x)=c\).
        \end{proof}
      \end{mathlemma}

      I když všechny primitivní funkce k funkci \(f(x)\) mají až na konstantu stejný tvar, může se
      stát že použijeme-li k hledání primitivní funkce různých Integračních metod, dostaneme pokaždé
      \uv{trochu jiný} výsledek. V tom případě je vždy možné \uv{převést} jeden na druhý
      \cite[p.~58]{rektorys2011}. Dostaneme-li například jednou metodou \(1/\cos^2x\) a druhou nám
      vyjde \(1+\tan^2x\), jsou oba výsledky správné, neboť \[1+\tan^2x = 1+\dfrac{\sin^2x}{\cos^2x}
      = \dfrac{\sin^2x + \cos^2x}{\cos^2x} = \dfrac{1}{\cos^2x}\].

      Geometrická interpretace věty \ref{mai:lemma008}: Grafy dvou libovolných primitivních funkcí k
      funkci \(f(x)\) (v uvažovaném intervalu \(J\)) lze ztotožnit posunutím ve směru osy \(y\)
      (obr. \ref{mai:fig100})

      \begin{figure}[ht!]
        \centering
        \luafigure[1]{mai_fig078.pdf}
        \caption{Geometrická interpretace primitivní funkce \(F(x) +c\)}
        \label{mai:fig100}
      \end{figure}

      Všechny primitivní funkce k funkci \(f(x)\) se tedy liší navzájem jen o konstanu. Je-li známá
      jedna primitivní funkce \(F(x)\) k funkci \(f(x)\), lze všechny ostatní vyjádřit ve tvaru
      \(F(x) + c\), kde \(c\) je libovolné reálné číslo. Množina funkcí tvaru \(F(x) + c\) zahrnuje
      tedy všechny funkce, které mají derivaci \(f(x)\) (a tedy diferenciál \(f(x)\dd{x}\)). Pro
      tuto množinu funkcí, s níž se budeme v daším textu neustále setkávat, je vhodné zavést
      zvláštní název:  

      \begin{mathdef}{Neurčitý integrál}{def004} 
        Množinu všech primitivních funkcí k funkci \(f(x)\) v intervalu \(J\) nazýváme
        \textbf{neručitým integrálem} k funkci \(f(x)\) a označujeme ji \(\int f(x)\dd{x}\). Píšeme 
        \[\int f(x)\dd{x} = F(x) + c \qquad x\in J.\]
      \end{mathdef}

      Funkci \(f(x)\) nazveme \emph{integrandem}, symbol \(\int\) (který vznikl protažením písmena
      S) \emph{integračním znakem}. Výkon, kterým určujeme neurčitý integrál \(F(x) + c\) k dané
      primitivní funkci \(f(x)\), resp. k diferenciálu \(f(x)\dd{x}\), nazýváme \emph{integrováním}
      funkce \(f(x)\). Odůvodnění názvu integrál a jeho označení vysvitne později, až poznáme vztah
      mezi primitivní funkcí a tzv. určitým integrálem v kapitole \ref{mai:IchapVIII}. Slovem
      \uv{neurčitý} vyznačujeme, že uvažujeme všechny funkce tvaru \(F(x)+ c\), nikoli jednu
      primitivní funkci. Místo \[\int\dfrac{1}{x}\dd{x}\] je zvykem psát \[\int\dfrac{\dd{x}}{x}\]
      apod. 

      \subsubsection{Poznámky k definici}  
        Z definice \ref{mai:def002} vyplývá, že primitivní funkce je spojitou funkcí, neboť má
        derivaci a věta \ref{mai:lemma008} nám říká, že pokud k dané funkci existuje funkce
        primitivní, existuje jich nekonečně mnoho. Je-li např. \(F(x)\) primitivní funkce k funkci
        \(f(x)\) na intervalu \(J\), pak množina všech primitivních funkcí je množina \(\{F(x) + c;
        c\in\realset\}\). Tuto množinu označujeme symbolem \(\int f(x)\dd{x}\) a čteme \uv{integrál
        \(f(x)\dd{x})\)} Symbolu \(\int f(x)\dd{x}\) říkáme \textbf{neurčitý integrál}. Zpravidla
        píšeme

        \begin{equation}\label{mai:eq102}
          \int f(x)\dd{x} = F(x) + c
        \end{equation}
        nebo jen \(\int f(x)\dd{x} = F(x)\) (tj. aditivní konstantu vynecháváme) a máme tím na mysli
        že \(F(x) \in \int f(x)\dd{x}\). Rovnost \ref{mai:eq102} je tedy rovnost mezi množinami,
        nikoliv rovnost funkcí. 
        
        V tomto smyslu tedy \(\int f(x)\dd{x}\) znamená, že \[\diff{}{x}\int f(x)\dd{x} = f(x)\], a
        teprve ve výsledku výpočtu připsat integrační konstantu \(c\), abychom vyznačili, že jde o
        množinu funkcí.

        Jedna z primitivní funkcí k funkci \(f(x)=x^3\), \(x\in(-\infty, +\infty)\) je \(F(x)=
        \frac{1}{4}x^4\). Je proto
        \begin{fleqn}[\parindent]
          \begin{equation*}
            \int x^3\dd{x} = \frac{x^4}{4} + c \qquad x\in(-\infty, +\infty)
          \end{equation*}
        \end{fleqn} 

        Jedna z funkcí, jejichž diferenciál je \(f(x) = \frac{1}{1+x^2}\dd{x}\), \(x\in(-\infty,
        +\infty)\) je \(F(x) = \arctan x\), neboť v tomto intervalu je \(\dd{(\arctan x)} =
        \frac{1}{1+x^2}\dd{x}\). Je proto
        \begin{fleqn}[\parindent]
          \begin{equation*}
            \int\frac{1}{1+x^2}\dd{x} = \arctan x + c \qquad x\in(-\infty, +\infty).
          \end{equation*}
        \end{fleqn}
        Často používáme pro argument jiného označení než \(x\).

        Nechť \(f(z) = \frac{1}{z+3}\), \(x\in(-3, +\infty)\). Pak jedna z primitivních funkcí je
        \(F(z) = \ln(z+3)\), a tedy
        \begin{fleqn}[\parindent]
          \begin{equation*}
            \int\frac{1}{z+3}\dd{x} = \ln(z+3) + c \qquad x\in(-3, +\infty).
          \end{equation*}
        \end{fleqn}
    
    \subsection{Existuje k dané funkci vždy funkce primitivní?}  
      Ukázali jsme, že všechny primitivní funkce k dané funkci se mohou lišit navzájem až na
      konstantu. Nyní si všimneme otázky, zda k dané funkci \(f\) vůbec nějaká primitivní funkce
      \emph{existuje}. Následující příklad ukazuje, že primitivní funkce nemusí existovat vždy.
      %-------------------------------------  
      % https://users.math.cas.cz/~rehak/soubory/urc_int.pdf
\begin{mathexam}{Vezměme funkci 
  \begin{equation*}
    f(x) = 
    \begin{cases}
       0 & \text{pro} x\in\realset\backslash \{0\}, \\
       1 & \text{pro} x=0.
   \end{cases}
  \end{equation*}
  viz \ref{mai:fig080}. Dokažte, že k takto definované funkci neexistuje primitivní funkce na
  \(\realset\).
  }{exam154} 

  {\centering
    \captionsetup{type=figure} 
    \luafigure[0.7]{mai_fig080.pdf}
    \captionof{figure}{Funkce, která nemá primitivní funkci na žádném okolí \(0\)}
    \label{mai:fig080}
  \par}
  
  Kdyby \(F(x)\) byla primitivní funkcí k \(f(x)\) na \(\realset\), pak by \(F(x)\) byla na
  \(\realset\) spojitá. Ze zadání má být \(F'(x) = f(x) = 0\) na \(\realset\backslash \{0\}\), tj.
  funkce \(F(x)\) by musela být konstantní jak na \((−\infty, 0)\), tak na \((0, ∞)\). Tyto dva
  postřehy znamenají, že \(F(x)\) by byla konstantní na celém \(\realset\). Potom by však \(F'(0) =
  0\), což nesouhlasí s \(f(0) = 1\). Proto primitivní funkce k \(f(x)\) na celém \(\realset\)
  existovat nemůže. Stejně tak neexistuje na žádném intervalu obsahujícím \(0\), na ostatních
  intervalech však jistě existuje (a může to být jakákoli konstantní funkce).
\end{mathexam}
      %-------------------------------------      
      Viděli jsme, že nemusí existovat primitivní funkce k funkci na intervalu obsahujícím
      „problematický“ bod. Pro „dostatečně rozumné“ funkce však máme existenci zaručenou následující
      větou
      \begin{mathlemma}{o existenci primitivní funkce \(F(x)\)}{lemma011}         
        Je-li funkce \(f(x)\) \emph{spojitá} na intervalu \(J\), pak k ní existuje primitivní funkce
        na tomto intervalu.
      \end{mathlemma}
      Věta \eqref{mai:lemma011} je typickým příkladem tzv. \emph{existenční věty}. Říká, že něco
      existuje, ale neříká, jak se to najde. Větu nebudeme dokazovat, protože k tomu zatím nemáme
      potřebné nástroje, což nevadí, protože samotný důkaz není v tomto smyslu konstruktivní.
      
      Mohlo by se zdát, že spojitost je i nutnou podmínkou pro existenci primitivní funkce. Tak tomu
      však není, jak ukazuje následující příklad:
      %-------------------------------------  
      % https://users.math.cas.cz/~rehak/soubory/urc_int.pdf
\begin{mathexam}{Ukažme, že funkce definovaná předpisem
  \begin{equation*}
    f(x) = 
    \begin{cases}
      2x\sin\frac{1}{x} - \cos\frac{1}{x} & \text{pro} x\neq0, \\
      0                                   & \text{pro} x=0.
    \end{cases}
  \end{equation*}
  má primitivní funkcí na celém \(\realset\).
  }{exam155} 

  {\centering
    \captionsetup{type=figure} 
  % \luafigure[0.7]{example-image-a}
    \luafigure[1]{mai_fig081}
    \captionof{figure}{Chaotický průběh funkce \(f(x)\) v okolí \(0\) a její primitivní 
      funkce \(F(x)\).}
    \label{mai:fig081}
  \par}
  
  Pomocí symbolického toolboxu \texttt{Matlabu} a následujícího skriptu
  \begin{lstlisting}[style=luaMatlabText,gobble=4]
    syms x;
    expr = 2*x*sin(1/x)-cos(1/x);
    F = int(expr)
  \end{lstlisting}
  snadno získáme hledanou primitvní funkci: 
  \begin{equation*}
    F(x) = 
      \begin{cases}
        x^2\sin\frac{1}{x} & \text{pro} x\neq0, \\
        0                  & \text{pro} x=0.
      \end{cases}
  \end{equation*}
\end{mathexam}
      %-------------------------------------       
      Všimněme si, že zatímco poměrně „příjemná“ nespojitost v příkladu \eqref{mai:exam154}
      znemožnila existenci primitivní funkce, tak o dost „nepříjemnější“ nespojitost ve druhém
      příkladu \ref{mai:exam155} existenci připouští. V následujícím příkladu je zastoupen další
      typ nespojitosti.
      %-------------------------------------  
      % https://users.math.cas.cz/~rehak/soubory/urc_int.pdf
\begin{mathexam}{Prozkoumejme, jak je to s (ne)existencí primitivní funkce k funkci
  \begin{equation*}
    \sgn = 
      \begin{cases}
         -1 & \text{pro} x<0, \\
          0 & \text{pro} x=0, \\
          1 & \text{pro} x>0.
      \end{cases}
  \end{equation*}
  na intervalech obsahujících a neobsahujících \(0\)}{exam156} 

  {\centering
    \captionsetup{type=figure} 
    \luafigure[0.7]{mai_fig079.pdf}
    \captionof{figure}{Graf funkce \(\sgn x\)}
    \label{mai:fig079}
  \par}
  
  Podobně jako v příkladu \ref{mai:exam154} snadno zdůvodníme, že k funkci \(\sgn x\) neexistuje
  primitivní funkce na žádném intervalu obsahujícím \(0\). Na jakémkoli intervalu neobsahujícím
  \(0\) primitivní funkce podle věty \eqref{mai:lemma011} existuje (z cvičných důvodů určete
  nějakou, příp. všechny). 
\end{mathexam}
      %-------------------------------------       

      Uvedli jsme příklady se třemi typy nespojitosti, ve dvou z nich neexistovala primitivní
      funkce. Nabízí se tedy přirozená otázka, jak vlastně musí vypadat funkce, k níž existuje
      funkce primitivní. Zde nám pomůže pojem tzv. \textbf{darbouxovské funkce} \(f(x)\) na
      intervalu \(J\), tj. takové funkce \(f(x)\), že pro každé \(x_1, x_2 ∈ J\) takové, že \(f(x_1)
      < f(x_2)\), a každé \(y_0 ∈ \realset\) takové, že \(f(x_1) < y_0 < f(x_2)\) existuje \(x_0 ∈
      \langle x_1, x_2\rangle\) i tak, že \(f(x_0) = y_0\). Z diferenciálního počtu je známo, že
      má-li \(F(x)\) derivaci na intervalu \(J\), pak \(F'(x)\) je darbouxovská; důležitou roli v
      důkazu hraje Weierstrassova věta. Odtud již snadno plyne následující nutná podmínka existence
      primitivní funkce.
      \begin{mathlemma}{o darbouxovské funkci}{lemma015}         
        Existuje-li k funkci \(f(x)\) na intervalu \(J\) funkce primitivní, pak je \(f(x)\)
        darbouxovská na \(J\).
      \end{mathlemma}

      Má-li darbouxovská funkce bod nespojitosti, pak nutně alespoň jedna z jednostranných limit v
      tomto bodě neexistuje. Je evidentní, že jak funkce z příkladu \ref{mai:exam154}, tak i funkce
      signum nejsou darbouxovské na žádném intervalu obsahujícím \(0\) — vadí nám totiž „skok“ ve
      funkčních hodnotách. Funkce z příkladu \ref{mai:exam155} je nespojitá, avšak darbouxovská na
      \(\realset\). Mohlo by se zdát, že každá darbouxovská funkce je derivací nějaké funkce. Tak
      tomu však není; příklady takových funkcí jdou ovšem za rámec této kapitoly. Darbouxovské
      funkce však tvoří velmi velkou třídu funkcí, např. se ví, že libovolnou funkci lze vyjádřit
      jako součet dvou darbouxovských funkcí. To však mimo jiné implikuje, že třída darbouxovských
      funkcí není uzavřená vzhledem ke sčítání. 
      
      Výše jsme tedy uvedli jednoduchou postačující (avšak nikoli nutnou) podmínku a jednoduchou
      nutnou (avšak nikoliv postačující) podmínku pro existenci primitivní funkce.

    \subsection{Některé vlastnosti integrálů}  
      Platnost několika následujících vztahů lze prověřit přímo užitím definic derivace a neurčitého
      integrálu. Tyto vztahy jsou při výpočtech často používány. Nejprve však uvedeme základní
      pravidla pro primitivní funkce, která plynou z pravidel pro derivování:
      \begin{subequations}
        \begin{flalign}
          \int\dd{f(x)}                    
            &= f(x) + c,                                  &\label{mai:eq134} \\
          \dxd\left[\int f(x)\dd{x}\right] 
            &= f(x)\dd{x},                                &\label{mai:eq135} \\
          \int[f(x)\pm g(x)]\dd{x}
            &= \int f(x)\dd{x} + \int g(x)\dd{x},         &\label{mai:eq103} \\
          \int kf(x)\dd{x}      
            &=k\int f(x)\dd{x},\,\text{\(k=\) konst.},    &\label{mai:eq104}\raisetag{14pt} \\
          \left[\int f(x)\dd{x}\right]'    
            &= f(x),                                      &\label{mai:eq136}
        \end{flalign}
      \end{subequations}
      
  \section{Základní integrály}\label{mai:IchapVIIsecII} 
    Neurčitý integrál k některým elementárním funkcím najdeme velmi snadno na základě známých
    výsledků z diferenciálního počtu.
    
    \begin{enumerate}
      \item Je-li \(n\neq-1\) a \(x\in(0,+\infty)\), platí \(\left(\dfrac{x^{n+1}}{n+1}\right)'=
            x^n\) proto
            % https://tex.stackexchange.com/questions/145657/align-equation-left
            \begin{fleqn}[\parindent]
              \begin{equation*}
                \int x^n\dd{x} = \dfrac{x^{n+1}}{n+1}, \hfill x\in(0,+\infty) 
              \end{equation*}
            \end{fleqn}  
            Pro některá \(n\) je možno platnost vztahu rozšířit. Například pro přirozená \(n\) platí
            vztah v celém intervalu \((-\infty,+\infty)\). Například: 
            \begin{flalign*}
              & \int x^6\dd{x}          
                =  \frac{1}{7}x^7 + c,                                 &                        \\
              & \int\frac{1}{x^4}\dd{x} 
                =   \int x^{-4}\dd{x} = -\frac{1}{3}x^{-3} + c 
                = - \frac{1}{3x^3} + c,                                &                        \\
              & \int\sqrt[3]{x^2}\dd{x} 
                =   \int x^{\frac{2}{3}}\dd{x} 
                =   \frac{3}{5}\sqrt[3]{x^5} + c,                      &                        \\
              & \int\frac{1}{\sqrt{t^7}}\dd{x} 
                =   \int t^{-\frac{7}{2}}\dd{x} 
                = - \frac{2}{5}\frac{1}{\sqrt{t^5}} + c,               & 
            \end{flalign*}
      \item Je-li \(F(x)=C\), kde \(C\) je konstanta, platí \(F'(x)=0\) pro všechna \(x\), a tedy
            \begin{fleqn}[\parindent]
              \begin{equation*}
                \int 0\cdot\dd{x} = c, \hfill x\in(-\infty,+\infty).
              \end{equation*}
            \end{fleqn}
      \item Je-li \(F(x)=x\), kde \(x\in(-\infty,+\infty)\), platí \(F'(x)=1\) pro všechna \(x\), a
            proto
            \begin{fleqn}[\parindent]
              \begin{equation*}
                \int 1\cdot\dd{x} = \int\dd{x} = x + c, \hfill x\in(-\infty,+\infty).
              \end{equation*}
            \end{fleqn}
            (speciální případ prvního integrálu pro \(n=1\)).
      \item Funkce \(F(x)=\ln x\), \(x\in(0,+\infty)\) má derivaci \(F'(x) = \frac{1}{x}\), a proto
            \begin{fleqn}[\parindent]  
              \begin{equation*}
                \int\frac{1}{x}\dd{x} = \ln x + c, \hfill x\in(0,+\infty).
              \end{equation*} 
            \end{fleqn}
            Funkce \(G(x) = \ln(-x)\), \(x\in(-\infty,0)\) má derivaci \(G'(x) = \frac{-1}{-x} =
            \frac{1}{x}\), a platí tedy
            \begin{fleqn}[\parindent]
              \begin{equation*}
                \int\frac{1}{x}\dd{x} = \ln(-x) + c, \hfill x\in(-\infty,0).
              \end{equation*}
            \end{fleqn}
            Oba vzorce můžeme spojit zápisem
            \begin{fleqn}[\parindent]
              \begin{equation*}
                \int\frac{1}{x}\dd{x} = \ln\abs{x} + c
              \end{equation*}
            \end{fleqn}
            pro \(x\) z intervalu \((-\infty,0)\) nebo pro \(x\) z intervalu \((0,+\infty)\).
            Protože také
            \begin{fleqn}[\parindent]
              \begin{equation*}
                (\ln kx)' = \frac{k}{kx} = \frac{1}{x} \hfill (kx>0),
              \end{equation*}
            \end{fleqn}  
            můžeme psát
            \begin{fleqn}[\parindent]
              \begin{equation*}
                \int\frac{1}{x}\dd{x} = \ln kx \hfill (kx>0).
              \end{equation*}
            \end{fleqn}  
            Přitom je tedy \(k\) libovolná kladná konstanta, uvažujeme-li \(x\) z intervalu
            \((0,+\infty)\), nebo libovolná záporná konstanta, uvažujeme-li \(x\) z intervalu
            \((-\infty,0)\) (aby bylo vždy \(kx>0\)). Integrační konstanta je již v tomto zápisu
            obsažena. Například pro kladná \(x\) je \(\ln kx = \ln k + \ln x = \ln x + c\).
            
            Podobně pro \(y=\ln\abs{x+a}\), \(x\neq-a\) dostaneme \(y'=1/(x+a)\), a proto
            \begin{fleqn}[\parindent]
              \begin{equation*}
                \int\frac{1}{x+a}\dd{x} = \ln\abs{x+a} + c
              \end{equation*}
            \end{fleqn} 
            v každém otevřeném intervalu, který neobsahuje bod \(x=-a\), resp. 
            \begin{fleqn}[\parindent]
              \begin{equation*}
                \int\frac{1}{x+a}\dd{x} = \ln k(x+a), \hfill k(x+a)>0.
              \end{equation*}
            \end{fleqn} 
      \item Derivace funkce \(y=\dfrac{1}{2a}\ln\abs{\dfrac{x-a}{x+a}}\), \(a\neq0\), \(x\neq a\),
            \(x\neq -a\) je
            \begin{fleqn}[\parindent]
              \begin{equation*}
                y'=\frac{1}{2a}\cdot\frac{x+a}{x-a}\cdot\frac{x+a-(x-a)}{(x+a)^2} 
                  = \frac{1}{x^2-a^2}.
              \end{equation*}
            \end{fleqn} 
            Tedy pro \(a\neq0\) je
            \begin{fleqn}[\parindent]
              \begin{equation*}
                \int\frac{1}{x^2-a^2}\dd{x} = \frac{1}{2a}\ln\abs{\dfrac{x-a}{x+a}} + c
              \end{equation*}
            \end{fleqn} 
            v každém otevřeném intervalu, kerý neobsahuje bod \(x\neq a\) nebo \(x\neq -a\), resp.
            \begin{fleqn}[\parindent] 
              \begin{equation*}
                \int\frac{1}{x^2-a^2}\dd{x} = \frac{1}{2a}\ln\abs{\dfrac{x-a}{x+a}} + c,
                \hfill \left(k\dfrac{x-a}{x+a}>0\right).
              \end{equation*}
            \end{fleqn} 
      \item Pro všechna reálná \(x\) platí
            \begin{fleqn}[\parindent]      
              \begin{equation*}
                (e^x)' = e^x, \,\, \left(\frac{1}{k}e^{kx}\right)' = e^{kx}, \,\,
                \left(a^x\frac{1}{\ln a}\right) = a^x
              \end{equation*}
            \end{fleqn} 
            kde \(a>0, a\neq1\). Proto je
            \begin{flalign*}
              & \int{e^x}\dd{x} = e^x +c, \quad \int{e^{kx}}\dd{x} = \frac{1}{k}e^{kx} + c      & \\
              & \int{a^x}\dd{x} =\frac{a^x}{\ln{a}} + c                                         & 
            \end{flalign*}
            kde \(a>0, a\neq1, x\in(-\infty,+\infty)\)
      \item Pro funkce \(F(x) = \sin x\), \(F(x) = -\cos x\) platí v celém intervalu
            \(x\in(-\infty,+\infty)\) \(F'(x)=\cos x\) a \(G'(x) = \sin x\). Odtud plyne
            \begin{flalign*}
              &\int\cos x\dd{x} =  \sin x + c, & \\
              &\int\sin x\dd{x} = -\cos x + c, &  
            \end{flalign*}           
            (Všimněme si rozdílu ve znaménkách při derivování a integrování funkcí \(\sin x\) a
            \(\cos x\).) 
      \item Pro funkci \(F(x) = \tan x\), \(x\neq\pm\frac{\pi}{2}, \pm\frac{3\pi}{2}, \ldots\),
            platí \(F'(x)=\frac{1}{cos^2x}\). Platí proto v každém otevřeném intervalu, který
            neobsahuje liché násobky čísla \(\frac{\pi}{2}\),
            \begin{fleqn}[\parindent]
              \begin{equation*}
                \int\dfrac{1}{\cos^2x}\dd{x} = \tan x + c.
              \end{equation*}
            \end{fleqn}
            Obdobně v každém otevřeném intervalu, který neobsahuje celé násobky čísla \(\pi\), platí
            \begin{fleqn}[\parindent]
              \begin{equation*}
                \int\dfrac{1}{\sin^2x}\dd{x} = -\cotg x + c.
              \end{equation*}
            \end{fleqn}
      \item Pro funkce \(F(x) = \arcsin x\), \(G(x) = -\arccos x\) platí v intervalu \((-1,1)\)
            \begin{fleqn}[\parindent]
              \begin{equation*}
                F'(x) = \dfrac{1}{\sqrt{1-x^2}}, \quad G'(x) = \dfrac{1}{\sqrt{1-x^2}},
              \end{equation*} 
            \end{fleqn}
            tedy
            \begin{fleqn}[\parindent]
              \begin{equation*}
                \begin{rcases}
                  \int\dfrac{1}{\sqrt{1-x^2}}\dd{x} &=  \arcsin x + c \\
                  \int\dfrac{1}{\sqrt{1-x^2}}\dd{x} &= -\arccos x + k
                \end{rcases}x\in(-1,1).
              \end{equation*}
            \end{fleqn}
            Podle právě uvedených vzorců existují k funkci \(f(x)=1/\sqrt{1-x^2}\) dvě různé
            primitivní funkce. Výsledek však není ve sporu s větou \ref{mai:lemma008}, neboť platí
            \(\arcsin x - (-\arccos x) = \arcsin x + \arccos x = \frac{\pi}{2}\). Z této rovnice je
            zároveň vidět, že rozhodneme-li se v prvním vzorci pro určitou primitivní funkci tím, že
            zvolíme za \(c\) určité číslo, dostaneme z druhého vzorce touž funkci pro \(k = c
            +\frac{\pi}{2}\).
      \item Funkce
            \begin{fleqn}[\parindent]
              \begin{equation*}
                F(x) = \arcsin\frac{\pi}{2}\,\text{a}\, G(x) = \arcsin\frac{\pi}{2}\, a>0
              \end{equation*}
            \end{fleqn}  
            mají v intervalu \((-a,a)\) derivace
            \begin{flalign*}
              & F'(x) = \frac{1}{\sqrt{1-\frac{x^2}{a^2}}}\cdot\frac{1}{a}
                      = \dfrac{1}{\sqrt{a^2-x^2}},                           &               \\
              & G'(x) = \frac{1}{\sqrt{1-\frac{x^2}{a^2}}}\cdot\frac{1}{a}
                      = \dfrac{1}{\sqrt{a^2-x^2}}.                           &
            \end{flalign*}
            Proto je 
            \begin{fleqn}[\parindent]
              \begin{equation*}
                \begin{rcases}
                  \int\dfrac{1}{\sqrt{a^2-x^2}}\dd{x} &=  \arcsin\frac{x}{a} + c \\
                  \int\dfrac{1}{\sqrt{a^2-x^2}}\dd{x} &= -\arccos\frac{x}{a} + k
                \end{rcases}
                \begin{array}{c}
                  a>0,      \\
                  x\in(-a,a).
                \end{array}
              \end{equation*}
            \end{fleqn} 
            \begin{fleqn}[\parindent]
              \begin{equation*}
                \int\dfrac{1}{\sqrt{16-x^2}} = \arcsin\dfrac{x}{4} + c \hfill x\in(-4,4).
              \end{equation*}
            \end{fleqn} 
      \item Derivace funkcí \(F(x) = \arctan x\) a \(G(x) = \arccotg x\), \(x\in(-\infty, \infty)\)
            jsou 
            \begin{fleqn}[\parindent]
              \begin{equation*}
                F'(x) = \dfrac{1}{1+x^2} \quad\text{a}\quad G'(x) = \dfrac{1}{1+x^2}
              \end{equation*}
            \end{fleqn} 
            proto
            \begin{fleqn}[\parindent]
              \begin{equation*}
                \begin{rcases}
                  \int\dfrac{1}{1+x^2}\dd{x} &= \arctan x + c  \\
                  \int\dfrac{1}{1+x^2}\dd{x} &= \arccotg x + k
                \end{rcases}
                x\in(-\infty, \infty).
              \end{equation*}
            \end{fleqn} 
            Zde se také primitivní funkce \(\arctan x\), \(-\arccotg x\) liší jen o konstantu, neboť
            \(\arctan x - (-\arccotg x) = \frac{\pi}{2}\), ve shodě s větou
            \ref{mai:lemma008}. Derivování funkcí 
            \begin{fleqn}[\parindent]
              \begin{equation*}
                F(x) = \dfrac{1}{a}\arctan\dfrac{x}{a} \quad\text{a}\quad 
                G(x) = \dfrac{1}{a}\arccotg\dfrac{x}{a}
              \end{equation*}
            \end{fleqn} 
            dostaneme
            \begin{fleqn}[\parindent]
              \begin{equation*}
                F'(x) = G'(x) = \frac{1}{a}\cdot\dfrac{1}{1+\frac{x^2}{a^2}}\cdot\frac{1}{a} 
                      = \dfrac{1}{a^2+x^2}
              \end{equation*}
            \end{fleqn} 
            Proto
            \begin{fleqn}[\parindent]
              \begin{equation*}
                \begin{rcases}
                  \int\dfrac{1}{a^2+x^2}\dd{x} &= \dfrac{1}{a}\arctan\dfrac{x}{a}  + c  \\
                  \int\dfrac{1}{a^2+x^2}\dd{x} &= \dfrac{1}{a}\arccotg\dfrac{x}{a} + k
                \end{rcases}
                x\in(-\infty, \infty).
              \end{equation*}
            \end{fleqn} 
            \begin{fleqn}[\parindent]
              \begin{equation*}
                \int\dfrac{1}{5+x^2}\dd{x} = \dfrac{1}{\sqrt{5}}\arctan\dfrac{x}{\sqrt{5}}  + c,
                \hfill x\in(-\infty, \infty)
              \end{equation*}
            \end{fleqn}
      \item Funkce \(F(x)=\ln k(x+\sqrt{x^2+a})\) má pro taková \(x, k, a\), pro která \(x^2+a>0\),
            \(k(x+\sqrt{x^2+a})>0\), derivaci
            \begin{fleqn}[\parindent]
              \begin{equation*}
                F'(x) = \dfrac{k\left(1+\dfrac{2x}{2\sqrt{x^2+a}}\right)}{k(x+\sqrt{x^2+a})}
                      = \dfrac{\dfrac{\sqrt{x^2+a}+x}{\sqrt{x^2+a}}}{x+\sqrt{x^2+a}}
              \end{equation*}
            \end{fleqn}
            Tedy
            \begin{fleqn}[\parindent]
              \begin{equation*}
                \int\dfrac{1}{\sqrt{x^2+a}}\dd{x} = \ln k(x+\sqrt{x^2+a}),
              \end{equation*}
            \end{fleqn}
            kde \(x^2+a>0\), \(k(x+\sqrt{x^2+a})>0\). Popřípadě podobně jako v bodě 4:
            \begin{fleqn}[\parindent]
              \begin{equation*}
                \int\dfrac{1}{\sqrt{x^2+a}}\dd{x} = \ln\abs{x+\sqrt{x^2+a}} + c,
              \end{equation*}
            \end{fleqn}
            v každém otevřeném intervalu, v němž je \(x^2+a>0\), \(k(x+\sqrt{x^2+a})\neq0\). Je-li
            \(a>0\), je pro všechna \(x\) (kladná, záporná i nulu) \(x^2+a>0\),
            \(x+\sqrt{x^2+a}>0\), a tedy
            \begin{fleqn}[\parindent]
              \begin{equation*}
                \int\dfrac{1}{\sqrt{x^2+a}}\dd{x} = \ln \bigl(x+\sqrt{x^2+a}\bigr) + c,
              \end{equation*}
            \end{fleqn}
            kde \(a>0\), \(x\in(-\infty, \infty)\). Znak absolutní hodnoty je zde zbytečný.
            Integrály, které jsme v tomto článku odvodili, se nazývají \emph{základní} a jsou
            přehledně sestaveny v následující tabulce.
    \end{enumerate}
  
    \subsection{Tabulka základních integrálů}\label{mai:IchapVIIsecIIssecI}
      Jak ale primitivní funkce hledat? V jednoduchých příkladech poslouží tabulka derivací, již
      čteme „zprava doleva“. (Je dobré si ji uložit do paměti.) Tabulka však pokryje jen velmi málo
      případů, pouze elementární funkce. Je tedy třeba najít metody, jak při hledání primitivních
      funkcí postupovat.

      Uveďme nyní některé základní integrály. Poznamenejme, že touto tabulkou nejsou zdaleka
      vyčerpány všechny funkce, ke kterým umíme primitivní funkce najít. Existují celé knihy
      obsahující tabulky integrálů a programy výrazně ulehčující hledání primitivních funkcí.
      Literatura: \cite{Rektorys1963}, \cite{Brabec1989}, \cite{diblik2002}. 

      Pokud není nic uvedeno, platí vzorce pro všechna \(x\) a pro všechny hodnoty uvedených
      konstant. Místo platí pro \(x\) z intervalu \((-\infty,0),(0,+\infty)\) píšeme stručně
      \(x\neq0\) apod. Obory platnosti uvádíme jen tam, kde nejsou evidentní.  

      \begin{flalign}
        \midrule
        & \int 0\dd{x} = c                                              \label{mai:eq105}&   \\
        & \int  \dd{x} = x + c                                          \label{mai:eq187}&   \\
        & \int a\dd{x} = ax + c                                         \label{mai:eq106}&   \\
        & \int x^n\dd{x} = \frac{x^{n+1}}{n+1}+c,                       \label{mai:eq107}&   \\              
          \shortintertext{\hspace{2em} kde \(
            \begin{cases}
              \forall x\in\realset,\,n\in\naturalset, n>0,         \\
              \forall x\in\realset-\{0\},\,n\in\naturalset, n<-1,  \\
              \forall x>0,\,n\in\realset\,\,n\neq-1
            \end{cases}\)
          }                                          
        & \int\frac{\dd{x}}{x} = 
            \ln\abs{x}+c \hspace{1ex}\forall x\neq0              &       \label{mai:eq108}   \\
        & \int\frac{\dd{x}}{x+a} = 
            \begin{cases}
               \ln\abs{x+a} + c,                             \\
               \text{pro } x\in(-a, +\infty)\cup(-\infty,+a) \\ 
              -\ln k(x+a),\, k(x+a)>0 
            \end{cases}                                  &\raisetag{14pt}\label{mai:eq181}   \\
        & \int\frac{\dd{x}}{x^2-a^2} = 
            \begin{cases}
               \dfrac{1}{2a}\ln\abs{\dfrac{x-a}{x+a}} + c,\, x\neq\pm a   \\[10pt]
               \dfrac{1}{2a}\ln k\dfrac{x-a}{x+a} + c,\, k\dfrac{x-a}{x+a} >0 
            \end{cases}                                  &\raisetag{14pt}\label{mai:eq182}   \\
        & \int e^x \dd{x}       = e^x+c, \,\,
          \int e^{kx} \dd{x}    = \frac{1}{k}e^{kx}+c            &       \label{mai:eq109}   \\
        & \int\ln x\dd{x}       = 
            x\ln x - x + c \hspace{1ex}\forall x>0               &       \label{mai:eq110}   \\
        & \int a^x \dd{x}     =
          \frac{a^x}{\ln a}+c 
          \hspace{1ex}\forall a>0,\,a\neq1                       &       \label{mai:eq111}   \\
        & \int \sin x \dd{x}  = -\cos x + c                      &       \label{mai:eq112}   \\
        & \int \cos x \dd{x}  =  \sin x + c                      &       \label{mai:eq113}   \\
        & \int\frac{\dd{x}}{\cos^2x} =  \tan x + c               &       \label{mai:eq114}   \\
          \shortintertext{\hspace{2em}\(\forall x\neq\frac{1}{2}\pi+k\pi,\,k\in\naturalset\)}       
        & \int\frac{\dd{x}}{\sin^2x} =  -\cotg x+c               &       \label{mai:eq115}   \\
          \shortintertext{\hspace{2em}\(\forall x\neq k\pi,\,k\in\naturalset\)}
        & \int\frac{\dd{x}}{\sqrt{1-x^2}} =
            \begin{cases}
              +\arcsin x + c,        \\
              -\arccos x + c 
            \end{cases}                                           &       \label{mai:eq116}  \\
          \shortintertext{\hspace{2em}\(\forall x\in(-1,1) \)}  
        & \int\frac{\dd{x}}{\sqrt{a^2-x^2}} =
          \begin{cases}
            +\arcsin\dfrac{x}{a} + c,         \\[10pt]
            -\arccos\dfrac{x}{a} + c 
          \end{cases}                                             &       \label{mai:eq179}  \\
        \shortintertext{\hspace{2em}\(\forall a>0, x\in(-a,a) \)}  
        & \int\sinh x\dd{x} = \cosh x + c                         &       \label{mai:eq117}  \\
        & \int\dfrac{\dd{x}}{\sinh^2 x} = -\cotgh x + c\, x\neq0  &       \label{mai:eq118}  \\
        & \int\cosh x\dd{x} = \sinh x + c                         &       \label{mai:eq119}  \\
        & \int\dfrac{\dd{x}}{\cosh^2 x} = \tanh x + c             &       \label{mai:eq120}  \\
        & \int\frac{\dd{x}}{1+x^2} =              
        \begin{cases}
           \arctan x + c,           \\
          -\arccotg x + c 
        \end{cases}                                               &       \label{mai:eq121}  \\
        & \int\frac{\dd{x}}{1-x^2} =              
        \begin{cases}
           \argtanh x + c,                        \, &\abs{x}<1          \\
           \arccotg x + c                         \, &\abs{x}>1          \\
           \dfrac{1}{2}\ln\abs{\dfrac{1+x}{1-x}}  \, &\abs{x}\neq1        
        \end{cases}                                               &       \label{mai:eq122}  \\
        & \int\frac{\dd{x}}{a^2+x^2} =              
        \begin{cases}
           \dfrac{1}{a}\arctan\dfrac{x}{a}  + c,  \, a\neq0 \\[10pt]
          -\dfrac{1}{a}\arccotg\dfrac{x}{a} + c 
        \end{cases}                                               &       \label{mai:eq180}  \\
        & \int\frac{\dd{x}}{\sqrt{x^2 + 1}} =
            \begin{cases}
                \ln(x + \sqrt{x^2+1}) + c,         \\
                \argsinh x            + c 
            \end{cases}                                           &       \label{mai:eq192}  \\ 
        & \int \frac{\dd{x}}{\sqrt{x^2 - 1}} =
            \begin{cases}
                \ln(x + \sqrt{x^2-1}) + c,\, &\abs{x}>1  \\
                \argcosh x            + c \, &x>1  
            \end{cases}                                           &       \label{mai:eq123}  \\    
        & \int\frac{\dd{x}}{\sqrt{x^2+a^2}} 
            = \begin{cases}
                  \argsinh\dfrac{x}{a}   + c  \\[10pt]
                  \ln(x+\sqrt{x^2+a^2}) + c     
              \end{cases}                                          &      \label{mai:eq124}  \\
        & \int \frac{\dd{x}}{\sqrt{x^2-a^2}} 
            = \begin{cases}
                  \argcosh\dfrac{x}{a}   + c   \\[10pt]
                  \ln(x+\sqrt{x^2-a^2}) + c
              \end{cases}                                          &      \label{mai:eq125}  \\
        & \int\tan x \dd{x}   = \ln\abs{\sec x} + c                &      \label{mai:eq126}  \\
        & \int\sec x \dd{x}   = \ln\abs{\sec x + \tan x} + c       &      \label{mai:eq127}  \\
        & \int\sec^2 x \dd{x} = \tan x + c                         &      \label{mai:eq128}  \\
        & \int\sec x\tan x \dd{x} = \sec x + c                     &      \label{mai:eq129}  \\
        & \int\frac{a}{a^2+x^2}\dd{x} = \tan^{-1}\frac{x}{a} + c   &      \label{mai:eq130}  \\
        & \int\frac{a}{a^2-x^2}\dd{x} = 
          \frac{1}{2}\ln\left\lvert\frac{x+a}{x-a}\right\rvert     &      \label{mai:eq131}  \\
        & \int\frac{\dd{x}}{\sqrt{a^2-x^2}} = 
          \sin^{-1} \frac{x}{a}                                    &      \label{mai:eq132}  \\
        & \int\frac{a}{x\sqrt{x^2-a^2}}\dd{x} = 
          \sec^{-1} \frac{x}{a}                                    &      \label{mai:eq133}     
      \end{flalign}
    
      \begin{mdframed}[style=mdnote]
        \begin{note}  
          \(\argsinh x\), \(\argcosh x\), \(\argtanh x\), \(\argcoth x\) jsou funkce inverzní k
          hyperbolickým funkcím. Značení těchto funkcí není jednotné. Správné zkratky jsou ty, které
          specifikuje norma \texttt{ISO 80000-2}. Skládají se z \uv{ar} následovaných zkratkou
          odpovídající hyperbolické funkce (např. arsinh, arcosh).

          Nicméně, předpona \uv{arc} - následovaná odpovídající hyperbolickou funkcí (např. arcsinh,
          arccosh) je také běžně vidět, analogicky s nomenklaturou pro inverzní trigonometrické
          funkce. Není to však správné pojmenování, protože prefix \uv{arc} je zkratka pro
          \textbf{arcus}, zatímco prefix \uv{ar} znamená \textbf{area}. 

          Autoři jako prof. K. Rektorys upřednostňují použití označení \(\argsinh\), \(\argcosh\),
          \(\argtanh\) atd., kde předpona \uv{arg} je zkratkou latinského \textbf{argumentu}. V
          počítačové vědě se to často zkracuje na asinh.

          Používá se také zápis \(\sinh^{-1}(x)\), \(\cosh^{-1}(x)\) atd., a to navzdory
          skutečnosti, že je třeba dbát na to, aby nedošlo k nesprávným interpretacím horního indexu
          \(-1\) jako mocniny, na rozdíl od zkratky pro označení inverzní funkce
          (např. \(\cosh^{-1}(x)\) versus \(\cosh(x)^{-1}\)).
        \end{note}
      \end{mdframed}

    \subsection{Elementárnost primitivních funkcí}\label{mai:IchapVIIsecIIssecII} 
      Třebaže ke každé spojité funkci existuje primitivní funkce, nelze v mnoha případech tuto
      primitivní funkci vyjádřit pomocí elementárních funkcí. Jinými slovy, nedovedeme použitím
      uvedených tří metod (metody integrování úpravou a rozkladem, metody integrování per partes a
      metody substituční) vypočítat daný integrál v tom smyslu, že jej převedeme na jeden nebo více
      tabulkových integrálů, čímž je zároveň zaručeno, že také primitivní funkce je elementární
      funkcí. Příkladem takové funkce je funkce
      \(e^{x^2}\). Integrálem
      \begin{equation*}
        \int e^{x^2}\dd{x}
      \end{equation*}
      je tedy definována nová transendentní funkce. Podobně jsou nové transendentní funkce
      definovány integrály
      \begin{align*}
        \text{Si}(x) 
          &= \int\limits_0^x     \frac{\sin t}{t}\dd{t},&&\ldots\,\text{Integrálsinus}           \\
        \text{Ci}(x) 
          &=-\int\limits_x^\infty\frac{\cos t}{t}\dd{t},&&\ldots\,\text{Integrálcosinus}         \\
        \text{Li }x 
          &= \int\limits_0^x     \frac{\dd{t}}{\ln t},  &&\ldots\,\text{Logaritmusintegrál}      \\
        \text{Ei}(x) 
          &= \int\limits_{-\infty}^x \frac{e^t}{t}\dd{t}&&\ldots\,\text{Exponenciální integrál}  \\
        \Gamma(x) 
          &= \int\limits^\infty_0 t^{x-1}{e^{-t}}\dd{t}&&\ldots\,\text{Gamma funkce}             \\
        Y
          &= \int\limits^\infty_{-\infty}{e^{-x^2}}\dd{x}&&\ldots\,\text{Gausův inegrál}     
      \end{align*}

      Vidíme, že se jedná na první pohled o velmi jednoduché funkce. Tedy odhadnout podle
      \uv{složitosti} zadané funkce, zda je její primitivní funkce elementární nebo ne, je nemožné.
      Například \(\int\cos^2x\dd{x}\) lehce spočítáme, ale \(\int\cos x^2\dd{x}\), už není
      elementární funkce. Přitom v obou případech jde o složenou funkci se složkami \uv{druhá
      mocnina} a \uv{kosinus}. Liší se jen pořadím složek. Jedná se totiž o: 
      \[\int\sin{x^2}\dd{x},\quad\int\sin{x^2}\dd{x}\quad\ldots\text{Fresnelovy integrály}\]


  \section{Metody určení primitivní funkce}\label{mai:IchapVIIsecIII}
    Procesu hledání primitivní funkce se často říká integrování nebo integrace (od slova“integrál”),
    což z matematického hlediska znamená provést inverzní operaci k operaci derivování. Smutnou
    zprávou je, že na rozdíl od derivování neexistuje obecný vzorec pro integrování součinu či
    podílu, ani obecný vzorec pro integrování složených funkcí. Při hledání integrálů složitějších
    funkcí se využívá např. \emph{linearita, metoda per partes, substituční metoda}, popř. některé
    další speciální metody. Řešitel v mnoha případech musí projevit důvtip a intuici, která mu
    pomůže nalézt primitivní funkci k dané funkci.

    % --------------------------Integrování součtu, úprava integrandu-------------------------------
    \subsection{Integrování součtu, úprava integrandu}\label{mai:IchapVIIsecIIIssecI}  
      V tomto článku ukážeme metody, jež nám umožní najít neurčitý integrál některých funkcí
      jednoduchého tvaru, jež nejsou uvedeny v tabulce základních integrálů. Příslušné věty jsou
      důsledkem známých vět z diferenciálního počtu o derivování součtu a součinu funkcí a o
      derivování složené funkce. 

      \begin{mathlemma}{}{lemma012}        
        Existují-li k funkcím \(f_1(x)\), \(f_2(x)\), \(\ldots\), \(f_n(x)\) v otevřeném
        intervalu \(J\) primitivní funkce \[F_1(x), F_2(x), \ldots, F_n(x)\] pak funkce \(F(x)\)
        se rovná
        \begin{fleqn}[0pt]
          \begin{equation}\label{mai:eq185}
            c_1F_1(x) + c_2F_2(x) + \ldots + c_nF_n(x),
          \end{equation}
        \end{fleqn}
          kde \(c_1, c_2, \ldots, c_n\) jsou libovolné (reálné) konstanty, je v intervalu \(J\)
          primitivní funkci k funkci 
          \begin{fleqn}[0pt]
            \begin{equation}\label{mai:eq186}
              f(x) = c_1f_1(x) + c_2f_2(x) + \ldots + c_nf_n(x),
            \end{equation}
          \end{fleqn}
        \useshortskip
        \tcblower
        \begin{proof}
          Protože podle předpokladu \(F'_1(x) = f_1(x)\), \(F'_2(x) = f_2(x)\), \(\ldots\),
          \(F'_n(x) = f_n(x)\), je funkce \eqref{mai:eq186} v \(J\) derivací funkce
          \eqref{mai:eq185}.
        \end{proof}
      \end{mathlemma}

      Věta \eqref{mai:lemma012} se vyslovuje obvykle v tomto tvaru, který je pro účely integrálního
      počtu vhodnější:
      \begin{mathlemma}{}{lemma013}       
        Existují-li v otevřeném intervalu \(J\) neurčité integrály funkcí \(f_1(x)\), \(f_2(x)\),
        \(\ldots\) \(f_n(x)\) a jsou-li \(c_1, c_2, \ldots, c_n\) libovolné konstanty, existuje
        neurčitý integrál funkce  
        \begin{fleqn}[0pt]
          \begin{equation}\label{mai:eq188}
            f(x) = c_1f_1(x) + c_2f_2(x) + \ldots + c_nf_n(x),
          \end{equation}
        \end{fleqn}
        a platí
        \begin{multline}\label{mai:eq189}
          \int[c_1f_1(x) + c_2f_2(x) + \ldots + c_nf_n(x)]\dd{x}  =    \\   
          \int{c_1f_1(x)}\dd{x} + \int{c_2f_2(x)}\dd{x} +              \\
          + \ldots + \int{c_nf_n(x)}\dd{x}.
        \end{multline}
      \end{mathlemma}
      Přitom rovnici \eqref{mai:eq189} rozumíme takto: Na její pravé straně jsou součty funkcí tvaru
      \(c_1[F_1(x) + C_1]\), \(c_2[F_2(x) + C_2]\), \(\ldots\), \(c_n[F_n(x) + C_n]\). Na její levé
      straně je funkce tvaru \(F(x)+C\), kde \(F(x)\) je dána rovnicí \eqref{mai:eq186}. Zvolíme-li
      konstanty \(C_1, C_2, \ldots C_n\) libovolně, pak v obecném případě rovnice 
      \begin{multline}\label{mai:eq190}
        F(x) + c = c_1[F_1(x) + C_1] + c_2[F_2(x) + C_2] +\\ 
             + \ldots + c_n[F_n(x) + C_n]
      \end{multline}
      neplatí. Zvolíme-li \(C_1, C_2, \ldots C_n\) libovolně, lze konstantu \(c\) určit tak, aby
      \eqref{mai:eq190} bylo splněno v celém intervalu \(J\), a naopak, zvolíme-li \(c\) libovolně,
      je možno konstanty \(C_1, C_2, \ldots C_n\) zvolit tak, aby byl splněn vztah
      \eqref{mai:eq190}. 

      V tomto smyslu je tedy třeba rozumět rovnici \eqref{mai:eq189}. S rovnicemi tohoto typu se v
      textu ještě často setkáme. Uveďme nyní několik příkladů na použití věty \eqref{mai:lemma013}:
      %-------------------------------------
      \begin{mathexam}{\(\scalerel{\int}{(7x^2+2x^3+5\cos x)\dd{x}}\).}{exam142}
  V tomto případě \(f_1(x) = x^5\), \(f_2(x) = x^3\), \(f_3(x) = 5\cos x\), \(c_1=7\), \(c_1=2\) ,
  \(c_3=5\). Protože v intervalu \((-\infty, \infty)\) existují podle
  \eqref{mai:IchapVIIsecIIssecI} integrály
  \begin{align*}
    \int x^5\dd{x}   &= \frac{x^6}{6} + C_1, \\
    \int x^3\dd{x}   &= \frac{x^4}{4} + C_2, \\
    \int\cos x\dd{x} &= \sin x +C_3,
  \end{align*}
  platí podle věty \eqref{mai:lemma013} 
  \begin{fleqn}[0pt]
    \begin{multline*}
      \int(7x^2+2x^3+5\cos x)\dd{x} = \\
        = 7\int x^5\dd{x} + 2\int x^3\dd{x} + 5\int\cos x\dd{x},
    \end{multline*}
  \end{fleqn}
    a tedy (integrační konstanty shneme v jedinou)
    \begin{fleqn}[0pt]
      \begin{multline*}
        \int(7x^2+2x^3+5\cos x)\dd{x} = \\
          = \frac{7}{6}x^6 + \frac{1}{2}x^4 + 5\sin x + c.
      \end{multline*}
    \end{fleqn}
\end{mathexam}
      %-------------------------------------
      \begin{mdframed}[style=mdexam]
  \begin{example}\label{MAI:exam143}
    Určeme k funkci \(f(x)=2x+5\), \(x\in(-\infty, +\infty)\) primitivní funkci \(F(x)\) tak, aby
    její hodnota pro \(x=2\) byla rovna deseti, tj. \(F(2)=10\).  

    \noindent\textbf{Řešení:}
    Podle věty \eqref{mai:lemma013} a podle \eqref{mai:IchapVIIsecIIssecI} bude \[F(x)=x^2 + 5x +
    c\qquad x\in(-\infty, +\infty)\], kde \(c\) je určitá konstanta. Má-li být \(F(2) = 10\), musí
    platit \[2^2+5\cdot2 + c = 10\], čili \(c=-4\). 

    Hledaná primitivní funkce, splňující uvednou podmínku, je \(F(x)=x^2 + 5x -4\).
  \end{example}
\end{mdframed}
      %------------------------------------- 
      \begin{mathexam}{\(\scalerel{\int}{\left(\dfrac{7}{x}+ \dfrac{1}{x^2} -
    \dfrac{1}{4}x^2\right)\dd{x}}\)}{exam144} 
    %
    Protože v intervalu \(x\in(0, +\infty)\) existují integrály
    \begin{align*}
      \scalerel{\int}{\dfrac{1}{x}\dd{x}}   &= \ln x + c_1  \\
      \scalerel{\int}{\dfrac{1}{x^2}\dd{x}} &= \int x^{-2}\dd{x} = -\dfrac{1}{x} + c_2  \\
      \mathlarger{\int}{x^2\dd{x}}            &= \dfrac{x^3}{3} + c_3 ,
    \end{align*}  
    platí v tomto intervalu
    \begin{multline*}
                    7\scalerel{\int}{\dfrac{1}{x}\dd{x}}   + 
                     \scalerel{\int}{\dfrac{1}{x^2}\dd{x}} - 
       \dfrac{1}{4}\mathlarger{\int}{x^2\dd{x}} =                   \\
         = 7\ln x - \dfrac{1}{x} - \dfrac{1}{12}x^3 + c.
    \end{multline*}
\end{mathexam}
      %-------------------------------------
      \begin{mdframed}[style=mdexam]
  \begin{example}\label{MAI:exam145}
    Určeme \[\int\bigl(5^u + r^u + u^r\bigr)\dd{u}\qquad u\in(0, +\infty)\], kde \(r\) je dané
    kladné číslo. 
    
    Pro \(u\in(0, +\infty)\) (a v případě prvních dvou z následujících integrálů pro všechna \(u\))
    platí
    \begin{align*}
      \int 5^u\dd{u}  &= \dfrac{5^u}{\ln u}   + c_1,  \\
      \int r^u\dd{u}  &= \dfrac{r^u}{\ln r}   + c_2,  \\
      \int u^r\dd{u}  &= \dfrac{u^{r+1}}{r+1} + c_3,
    \end{align*}  
    a tedy
    \begin{multline*}
      \int\bigl(5^u + r^u + u^r\bigr)\dd{u} = \int 5^u\dd{u}+\int r^u\dd{u}+\int u^r\dd{u}        \\
          =\dfrac{5^u}{\ln u} + \dfrac{r^u}{\ln r} + \dfrac{u^{r+1}}{r+1}
    \end{multline*}
  \end{example}
\end{mdframed}
      %-------------------------------------
      \begin{mathexam}{\(\protect\scalerel{\int}{\left(5\sqrt{x} + \dfrac{4}{x+2} +
  \dfrac{3}{\cos^2x}\right)\dd{x}} \quad\) pro \(x\in\left(\dfrac{\pi}{2},
  \dfrac{3\pi}{2}\right)\).}{exam146} 
%   
  K funkcím \(\sqrt{x}\), \(\frac{1}{(x+2)}\), \(\frac{1}{\cos^2x}\) existují v intervalu
  \(x\in(\frac{\pi}{2}, \frac{3\pi}{2})\) primitvní funkce, tedy  
  \begin{multline*}
    \int\left(5\sqrt{x} + \dfrac{4}{x+2} + \dfrac{3}{\cos^2x}\right)\dd{x} =       \\
        = 5\int\sqrt{x}\dd{x} + \int\dfrac{4}{x+2}\dd{x} + \int\dfrac{3}{\cos^2x}\dd{x} = \\
        = \dfrac{10}{3}\sqrt{x^3} + 4\ln(x+2) + 3\tan x + c, 
  \end{multline*}
  kde \(x\in\left(\dfrac{\pi}{2}, \dfrac{3\pi}{2}\right)\)
\end{mathexam}
      %-------------------------------------

      Věty \ref{mai:lemma013} lze použít i k integrování funkcí,které nejsou přímo dány jako součet
      jiných funkcí, které dovedeme integrova, lze již použitím věty \ref{mai:lemma013} snadno
      vypočítat daný integrál. Posup ukážeme na několika příkladech. Obory platnosti uvádíme jen
      tam, kde nejsou zřejmé. 
      %-------------------------------------
      \begin{mathexam}{\(\scalerel{\int}{\dfrac{6^{x-1} + 4^{x+1}}{2^x}}\dd{x}\)}{exam147} 
  Integrand postupně převedeme na součet dvou funkcí:
  \begin{multline*}
    \dfrac{6^{x-1} + 4^{x+1}}{2^x} = \dfrac{6^{-1}\cdot6^x + 4\cdot4^x}{2^x} =   \\
      = \dfrac{6^{-1}\cdot\cancel{2^x}\cdot3^x + 4\cdot\cancel{2^x}\cdot2^x}{\cancel{2^x}} 
      = 6^{-1}\cdot3^x + 4\cdot2^x. 
  \end{multline*}
  Použitím věty \ref{mai:lemma013} dostaneme
  \begin{multline*}
    \int\left(\dfrac{6^{x-1}+4^{x+1}}{2^x}\right)\dd{x} = \int(6^{-1}\cdot3^x + 4\cdot2^x)\dd{x} \\
      = \dfrac{1}{6\ln3}3^x + \dfrac{4}{\ln2}2^x + c
  \end{multline*}
\end{mathexam}
      %-------------------------------------
      \begin{mathexam}{\(\int(x^3+2)^3\dd{x}\)}{exam148} 
  Umocněním výrazu v závorce převedeme funkci \((x^3+2)^3\) na součet čtyř funkcí, které mají
  známe primitivní funkce \[(x^3+2)^3 = x^9+6x^6+12x^3+8,\]. Dostaneme tak sadu integrálů
  \begin{multline*}
    \int x^9\dd{x} + 6\int x^6\dd{x} + 12\int x^3\dd{x} + 8\int\dd{x} =     \\
      = \dfrac{1}{10}x^{10} + \dfrac{6}{7}x^{7} + 3x^4 + 8x + c  
  \end{multline*}
\end{mathexam}
      %-------------------------------------   
      \begin{mathexam}{\(\scalerel{\int}{\dfrac{3}{(1+x^2)x^2}\dd{x}}.\)
  \hfill\cite[s.~29]{Knichal}}{exam149} 
  Integrand upravíme přičtením a odečtením výrazu \(3x^2\) v čitateli zlomku takto:
  \begin{align*}
    \dfrac{3}{(1+x^2)x^2} 
      &= \dfrac{3+3x^2-3x^2}{(1+x^2)x^2} = \dfrac{3(1+x^2)-3x^2}{(1+x^2)x^2}      \\
      &= \dfrac{3}{x^2} - \dfrac{3}{1+x^2}
  \end{align*}
  Později se budeme integrováním racionálních lomených funkcí zabývat systematicky.
  \[3\scalerel{\int}{\dfrac{\dd{x}}{x^2}} - 3\scalerel{\int}{\dfrac{\dd{x}}{1+x^2}}= -\dfrac{3}{x} -
    3\arctan x + c\]
\end{mathexam}
      %-------------------------------------
      \begin{mdframed}[style=mdexam]
  \begin{example}\label{MAI:exam121}
    Zdroj \cite[s.~29]{Knichal}.
    \begin{equation}\label{MA:int_ex_01}
      \int{\frac{x^4+3x^3-3x^2+3x}{x^2+1}\dd{x}}
    \end{equation}
    Dělením čitatele integrandu jmenovatelem  dostaneme rozklad integrandu na součet funkcí,
    jejich integrály najdeme snadno:
    \begin{equation*}
      \rotatebox{90}{$
        {\polylongdiv[style=C,div=:]{x^4+3x^3-3x^2+3x}{x^2+1}}
      $}
    \end{equation*}

    Tedy
    \begin{equation*}
      \frac{x^4+3x^3-3x^2+3x}{x^2+1} = x^2+3x-4+\frac{4}{x^2+1}  
    \end{equation*}
    Pro uvedený integrál dostaneme
    \begin{align*}
      \int{x^2}\dd{x} &+\int{3x}\dd{x}-4\int\dd{x}+\int{\frac{4}{x^2+1}\dd{x}} \\
                      &= \frac{x^3}{3}+\frac{3x^2}{2}-4x+4\arctan x + c.
    \end{align*}
  \end{example}
\end{mdframed}
      %-------------------------------------  
      \begin{mathexam}{\(\protect\scalerel{\int}{\dfrac{1}{\sin^2x\cos^2x}\dd{x}}.\)
  \hfill\cite[s.~31]{Knichal}}{exam150} 
  V čitateli integrandu položíme \(1=\sin^2x +\cos^2x\) čímž dostaneme
  \[\dfrac{1}{\sin^2x\cos^2x} = \dfrac{\sin^2x +\cos^2x}{\sin^2x\cos^2x} 
                              = \dfrac{1}{\sin^2x} + \dfrac{1}{\cos^2x}.\]
  Tedy v libovolném otevřeném intervalu, neobsahujícím celý násobek čísla \(\pi/2\), platí
  \begin{equation*}
    \int\dfrac{\dd{x}}{\sin^2x} + \int\dfrac{\dd{x}}{\cos^2x} = \tan x - \cotg x + c.
  \end{equation*}
\end{mathexam}
      %-------------------------------------  
      \begin{mathexam}{\(\scalerel{\int}{\cos^2\frac{x}{2}\dd{x}}.\)\hfill Převzato z
  \cite[s.~30]{Knichal}.}{exam132} 
  
  Použitím vzorce \(\cos^2\frac{x}{2} = \frac{1}{2}(1+\cos x)\) dostaneme součet dvou tabulkových
  integrálů:
  \begin{equation*}
    \frac{1}{2}\int{(1+\cos x)}\dd{x} = \frac{1}{2}(x+\sin x) + c.
  \end{equation*}
\end{mathexam}
      %-------------------------------------
      % 
\begin{mathexam}{\(\scalerel{\int}{\frac{\cos2x}{\cos^2x\cdot\sin^2x}\dd{x}}\)}{exam133} 
  Je-li \(\sin^2x\cos^2x\neq0;\, x\neq k\frac{\pi}{2};\, k\in Z\). Integrand upravíme pomocí vzorce
  pro dvojnásobný úhel \ref{MA1:eq_cosx2}:
  \begin{equation*}
    \scalerel{\int}{\frac{\cos^2x-\sin^2x}{\cos^2x\cdot\sin^2x}\dd{x}}
  \end{equation*}
  Dostaneme dva tabulkové integrály (\ref{mai:eq114}) a (\ref{mai:eq115}):
  \begin{gather*}
    \scalerel{\int}{\frac{1}{\sin^2x}\dd{x} -\int\frac{1}{\cos^2x}\dd{x} = -\cotg x - \tan x + c.}
  \end{gather*}
\end{mathexam}
      %-------------------------------------        
      \begin{mathexam}{\(\int\dfrac{\cos2x}{\cos x - \sin x}\dd{x}.\)}{exam151} 
  Integrand rozložíme na součet funkcí, které umíme integrovat, použitím vztahu \(\cos2x=\cos^2x -
  \sin^2x\): 
  \begin{multline*}
    \dfrac{\cos2}{\sin^2x\cos^2x} 
      = \dfrac{\cos^2x - \sin^2x}{\cos x - \sin x} = \dfrac{1}{\sin^2x} + \dfrac{1}{\cos^2x}. \\
      = \dfrac{(\cancel{\cos x - \sin x})(\cos x + \sin x)}{\cancel{\cos x - \sin x}} 
      = \cos x + \sin x.  
  \end{multline*}
  Platí tedy v libovolném otevřeném intervalu, neobsaujícím body, pro něž \(\cos x - \sin x = 0\)
  (tj. pro něž \(\tan x = 1\)),
  \begin{equation*}
    \int(\cos x + \sin x)\dd{x} = -\sin x + \cos x + c 
  \end{equation*}
  \vspace{-4mm}
\end{mathexam}
      %-------------------------------------  
      \begin{mathexam}{\(\protect\scalerel{\int}{\dfrac{\sqrt{x^3}+1}{\sqrt{x}+1}}\dd{x}.\)}{exam152} 
  Integrand upravíme takto: 
  \begin{equation*}
    \dfrac{\sqrt{x^3}+1}{\sqrt{x}+1} = \dfrac{(\sqrt{x})^3+1^3}{\sqrt{x}+1}
      = \dfrac{(\cancel{\sqrt{x} + 1})(x-\sqrt{x}+1)}{\cancel{\sqrt{x}+1}}
  \end{equation*}
  Dostaneme
  \[\int(x-\sqrt{x}+1)\dd{x} = \dfrac{1}{2}x^2 - \dfrac{2}{3}x\sqrt{x} + x +c\] pro
  \(x\in(0,+\infty)\). 
\end{mathexam}
      %-------------------------------------
      \begin{mathexam}{\(\protect\scalerel{\int}{\sqrt{x\sqrt{x\sqrt{x}}}}\dd{x}.\)}{exam153} 

  Zřejmě je: 
  \[
    \sqrt{\sqrt{\sqrt{x}}} = \sqrt{{x}\sqrt{x\cdot x^{\frac{1}{2}}}} = \sqrt{x\cdot x^{\frac{3}{4}}}
      = \sqrt{x^\frac{7}{4}} = x^{\frac{7}{8}},
  \]
  a proto
  \[
    \int x^{\frac{7}{8}}\dd{x} = \frac{8}{15}x^{\frac{15}{8}} + c 
      = \frac{8}{15}x\cdot\sqrt[8]{x^7} + c
  \]
  pro \(x\in(0,+\infty)\).
\end{mathexam}
      %-------------------------------------  
      \begin{mdframed}[style=mdexam]
  \begin{example}\label{MAI:exam131}
    Zdroj \cite[s.~30]{Knichal}.
    \begin{equation}\label{MA:int_ex_04}
      \int{\sqrt{1+\cos2x}\dd{x}}
    \end{equation}
    Funkci $\sqrt{1+\cos2x}$ upravíme na základě goniometrické identity \ref{MA1:eq_cos2x}:
    \(1+\cos2x = 1+\cos^2x-\sin^2x=2\cos^2x\) takto
    \begin{equation*}
      \sqrt{1+\cos2x} =\sqrt{2\cos^2x} = \sqrt{2}\abs{\cos x} = \varepsilon\sqrt{2}\cos x, 
    \end{equation*}
    \begin{equation*}
      \text{kde}\,\varepsilon =
        \begin{cases} 
         +1, &  x\in \left(-\frac{\pi}{2}+2n\pi,\frac{\pi}{2}+2n\pi\right), \\
         -1, &  x\in \left(\frac{\pi}{2}+2n\pi,\frac{3\pi}{2}+2n\pi\right),
        \end{cases}
    \end{equation*}
    $n$ je přirozené číslo. Proto pro $x$ ležící v uvedených intervalech je
    \begin{equation*}
      \varepsilon\sqrt{2}\int\cos x\dd{x} = \varepsilon\sqrt{2}\sin x + c.
    \end{equation*}
  \end{example}
\end{mdframed}
      %-------------------------------------  
      % \int{\tan^2x}\dd{x}
\begin{mathexam}{\(\mathlarger{\int}{\tan^2x}\dd{x}\)\hfill\cite[s.~30]{Knichal}}{exam127}
  Funkci \(\tan^2x\) napíšeme ve tvaru 
  \begin{align*}
    \tan^2x &= \frac{\sin^2x}{\cos^2x}=\frac{1-\cos^2x}{\cos^2x} = \frac{1}{\cos^2x}-1   \\
    \shortintertext{takže}
    \int{\tan^2x}\dd{x} &= \int\left(\frac{1}{\cos^2x}-1\right)\dd{x} = \tan x - x + c.  
  \end{align*}  
  Platí pro \(\forall x\in\left(-\frac{\pi}{2}+k\pi, \frac{\pi}{2}+k\pi\right)\),
  \(k\in\naturalset\).
\end{mathexam}
      %------------------------------------- 
      % \int\frac{1}{\cos x\cdot\sin x}\dd{x}, 
\begin{mdframed}[style=mdmathsolution] 
  [\ref{mai:eq151}]: \((\sin x\cos x\neq0; x\neq k\frac{\pi}{2}; k\in Z)\). Integrand rozšíříme o
  funkci \(\frac{1}{\cos^2x}\)
  \begin{equation*}
    \dfrac{\dfrac{1}{\cos^2x}}{\dfrac{\sin x\cdot\cos x}{\cos^2x}}  = 
    \dfrac{\dfrac{1}{\cos^2x}}{\dfrac{\sin x\cdot\cancel{\cos x}}{\cos x\cdot\cancel{\cos x}}} = 
    \dfrac{\dfrac{1}{\cos^2x}}{\tan x} = 
  \end{equation*}  
  Získali jsme výraz typu \(\dfrac{f'(x)}{f(x)}\), pro který platí (rov. \ref{mai:eq139}):  
  \begin{equation*}
    \int\dfrac{\dfrac{1}{\cos^2x}}{\tan x}\dd{x} = \ln\abs{\tan x} + c.
  \end{equation*}    
\end{mdframed}
    % --------------------------Integrace po částech - per partes-----------------------------------
    \newpage
    \subsection{Integrace po částech - per partes}\label{mai:IchapVIIsecIIIssecII} 
      V kapitole \ref{mai:IchapVIIsecIIIssecI} jsme ukázali, jak integrujeme součet konečného počtu
      funkcí. Příklady jsme řešili podle věty \ref{mai:lemma013}, která, zhruba řečeno, říká že
      integrál ze součtu je roven součtu integrálů. Analogická věta pro součin neplatí. (Proč?) Tato
      skutečnost značně komplikuje praktický výpočet inegrálů i tehdy, jde-li o součin poměrně
      jednoduchých funkcí. V některých případech (příklady \ref{mai:exam132}, \ref{mai:exam151}) je
      možno jednoduchou formální úpravou převést součin na součet funkcí, které dovedme integrovat.
      To se ovšem podaří jen ve velmi speciálních případech. Jednou z nejúčinějších metod pro
      integrování součinu je tzv. \emph{metoda intergrování per partes (po částech)}, jejímž
      použitím můžeme často integrál z daného součinu funkcí převést na integrál z jednodušších
      funkcí, který již dovedeme snadno integrovat. Literatura: \cite[p.~137]{Musilova2009MA1},
      \cite[s.~33]{Knichal}, \cite[s.~20]{Hoskova}.
      
      Metoda integrování per partes je založena na této větě:
      \begin{mathlemma}{o metodě integrování per partes}{lemma014}       
        Nechť funkce \(u(x)\) a \(v(x)\) mají derivaci na intervalu \(J\). Pak platí 
        \begin{equation}\label{mai:eq147}
          \int u(x)v'(x)\dd{x} = u(x)v(x) - \int u'(x)v(x)\dd{x} 
        \end{equation}
        pokud alespoň jeden z integrálů v předchozím vztahu existuje. 
        \tcblower
        \begin{proof}
          Pro funkce \(u(x)\) a \(v(x)\) mající derivaci platí vztah \((u(x)v(x))' = u(x)'v(x) +
          u(x)v'(x)\). Jeho integrací dostaneme
          \begin{equation*}
            \begin{multlined}
              \int(u(x)v(x))'\dd{x} = u(x)v(x) + c  =  \\
              = \int\left(u'(x)v(x)+u(x)v'(x)\right)\dd{x}.
            \end{multlined}
          \end{equation*}
          Integrál \(\int(u(x)v'(x) + u'(x)v(x))\dd{x}\) tedy existuje. Pokud existuje aspoň jeden z
          integrálů \(\int u(x)v'(x)\dd{x}\), \(\int u'(x)v(x)\dd{x}\), nechť je to např. \(\int
          u(x)v'(x)\dd{x}\) musí podle vzorce \eqref{mai:eq103} existovat i integrál z rozdílu
          \(\int[(u(x)v'(x) + u'(x)v(x)) - u(x)v'(x)]\dd{x} = \int u'(x)v(x)\dd{x}\), což je druhý
          uvažovaný integrál, takže \[u(x)v(x) + c  = \int\left(u'(x)v(x)+u(x)v'(x)\right)\dd{x}\]
          a odtud již dostáváme vztah \eqref{mai:eq147}.
        \end{proof}
      \end{mathlemma}

      V příkladech, které budeme řešit, mají funkce spojité derivace, takže existence integrálů bude
      zaručena větou \eqref{mai:lemma011}. Stručně ji zapisujeme
      \begin{equation}\label{mai:eq149}
        \int uv'\dd{x} = uv - \int u'v\dd{x}.
      \end{equation}

      Není vždy jednoduché rozpoznat, jak máme rozložit funkci \(f(x)\) na součin funkcí \(u'(x)\) 
      a \(v(x)\). Takový rozklad není určen jednoznačně a požadavek na něj bychom mohli (dosti 
      nepřesně) formulovat tak, aby funkce \(v'(x)\) byla jednodušší než v \(v(x)\) (například 
      derivováním polynomu se snižuje jeho stupeň) a funkce \(u'(x)\) a \(u(x)\) aby byly zhruba 
      „stejně složité“ (například \(u'(x) =e^x\), \(u(x) = e^x\), nebo \(u'(x) = \cos x\), \(u(x) = 
      \sin x\), apod.). Spolehlivě používat metodu per partes se však můžeme naučit pouze studiem 
      vyřešených příkladů z literatury a praktickým procvičováním \cite[p.~138]{Musilova2009MA1}.

      %-------------------------------------
      \begin{mathexam}{\(\int x\sin x\dd{x}\)}{exam111} 
  
  Součin v zadání je zřejmý. Můžeme si zvolit buď \(u=x\) a \(v'=\sin x\), nebo naopak \(u=\sin x\)
  a \(v'= x\).
  
  Zkusíme nejprve první volbu. Je-li \(u=x\) bude \(u=1\). Dále \(v'=\sin x\), tedy \(v=\int\sin
  x\dd{x} = -\cos x\) (integrační konstantu volíme rovnou nule, stačí nám jedna konkrétní primitivní
  funkce). Ze vzorce \eqref{mai:eq147} dostaneme
  \begin{align*}
    \int x\sin x\dd{x} &= x(-\cos x) - \int1\cdot(-\cos x)\dd{x} \\
                       &= -x\cos x + \sin x + c.
  \end{align*}  
  Tato volba tedy vedla k cíli. Výpočet obvykle zapisujeme do jakési tabulky, takže zápis vypadá
  následovně:
  \begin{align*}
    \int x\sin x\dd{x} &= %
      \left\lvert
        \begin{matrix} 
          u = x     & u' =1        \\
          v'=\sin x & v  = -\cos x 
        \end{matrix}  
      \right\rvert =                                              \\
                       & = x(-\cos x) - \int1\cdot(-\cos x)\dd{x} \\
                       &= -x\cos x + \sin x + c.
  \end{align*}  
\end{mathexam}
      %-------------------------------------
      \begin{mdframed}[style=mdexam]
  \begin{example}\label{mai:exam109}
    (\emph{Umělý rozklad na součin}): Někdy zadaná funkce \(f(x)\) jako součin vůbec nevypadá, a
    přesto je použití metody per partes vhodné. Například pro elementární funkci \(f(x) = \ln x\)
    sice najdeme primitivní funkci \ref{MA:baseInt06} v tabulce základních neurčitých integrálů z
    odstavce \ref{MA:chap_tabINT}, ale je možné postupovat i jinak. Představme si \(f(x)\) jako
    součin \(f(x) = 1\cdot\ln x\) a zvolme \[u'(x) = 1 ⇒ u(x) = x, \quad v(x) = lnx ⇒ v'(x) =
    \frac{1}{x}\] Pak 
    \begin{align*}
      \int\ln\dd{x} &= x\ln x - \int x\cdot\frac{1}{x}\dd{x}  \\ 
                    &= x\ln x - x.
    \end{align*}
  \end{example}
\end{mdframed}
      %-------------------------------------
  
    %--------------------------- Substituční metoda ----------------------------------------------
    \newpage
    \subsection{Substituční metoda typu \(\varphi(x) = t\)}\label{mai:IchapVIIsecIIIssecIII}
      Tato metoda \emph{substituce} neboli \emph{náhrady} spočívá v tom, že vhodně zvolenou funkci
      obsaženou v předpisu \(f(x)\) označíme jako novou jednoduchou proměnnou. Čeho tím dosáhneme?
      Předpokládejme například, že \[f(x)=\varphi'(x)g[\varphi(x)]\] a označme jako novou proměnnou
      \(u = f(x)\). Že to vypadá, jako bychom se chystali použít vzorec pro derivaci složené funkce?
      Správně! Dejme tomu, že známe primitivní funkci \(G(u)\) k funkci \(g(u)\). Pak platí
      \begin{equation*}
        \left[G\left(\varphi(x)\right)\right]' = G'\left[\varphi(x)\right]\cdot\varphi'(x) 
        = g\left[\varphi(x)\right]\cdot\varphi'(x),     
      \end{equation*}
      a tedy
      \begin{equation*}
        \int \varphi'(x) g\left[\varphi(x)\right]\dd{x} =  G\left[\varphi(x)\right]. 
      \end{equation*}      
      Na základě těchto úvah formulujeme následující větu (viz. \cite[p.~142]{diblik2002}):
      \begin{mdframed}[style=mdmathlemma]
        \begin{lemma}\label{mai:lemma009}          
          (\textbf{o substituci}). Nechť funkce \(F(t)\) je primitivní funkcí k funkci \(f(t)\) v
          intervalu \(\alpha, \beta\). Nechť funkce \(t = \varphi(x)\) v intervalu \((a,b)\).
          [Intervaly \(\alpha, \beta\) a \((a,b)\) mohou být popřípadě nekonečné.] Pro každé
          \(x\in(a,b)\) nechť číslo \(\varphi(x)\) patří do intervalu \(\alpha, \beta\). Pak v
          intervalu \((a,b)\) je funkce \(F(\varphi(x))\) primitivní funkcí k funkci
          \(f(\varphi(x))\varphi'(x)\), tj. 
          \begin{fleqn}[0pt]
            \begin{equation}\label{mai:eq176}
              \int{f(\varphi(x))\varphi'(x)\dd{x}} = F(\varphi(x)) + c.
            \end{equation}
          \end{fleqn}
          Tento výsledek je zvykem zapisovat v integrálním počtu takto:
          \begin{fleqn}[0pt]
            \begin{equation}\label{mai:eq177}
              \int{f[\varphi(x)]\varphi'(x)\dd{x}}= \int f(t)\dd{t}= F(t)+c
              \raisetag{14pt}
            \end{equation}
          \end{fleqn}
          kde \(t=\varphi(x)\).
        \end{lemma}
      \end{mdframed}
  
      Základem úspěchu při aplikací věty je správný výběr funkce $\varphi(x)$. Praxe je totiž
      taková, že výpočet konkrétních příkladů je schématicky veden od rov. \eqref{mai:eq177} ke
      vzorci rov. \eqref{mai:eq176}.

      Vzorec \ref{mai:eq177} se sandno pamatuje. Dosadíme-li v integrálu \(\int f(t)\dd{t}\)
      formálně \(t=\varphi(x)\) a (diferencování tohoto vztahu) \(\dd{t} = \varphi'(x)\), dostaneme
      integrál \(\int{f[\varphi(x)]\varphi'(x)\dd{x}}\).

      %-------------------------------------
      \begin{mdframed}[style=mdexam]
  \begin{example}\label{MAI:exam110}
    Jak poznat kandidáta na substituční metodu I. Počítejme neurčitý integrál 
    \begin{equation*}
      \int\frac{x}{\sqrt{x^2+1}}\dd{x}.
    \end{equation*} 

    \noindent\textbf{Řešení:}

    Vidíme, že čitatel funkce za integrálem je až na násobení konstantou \((2)\) derivací výrazu pod
    odmocninou. Při označení \(u=\varphi(x) = x^2 + 1\) dostáváme \(\varphi'(x) = x\) a řešíme
    následující integrál:
    \begin{gather*}
      \frac{1}{2}\int\frac{2x}{\sqrt{x^2+1}}\dd{x} 
        = \frac{1}{2}\int\frac{1}{\sqrt{u}}\dd{u} = \sqrt{u} + c    
        = \sqrt{x^2 + 1} + c.  
    \end{gather*}
  \end{example}
\end{mdframed}
      %-------------------------------------
      \begin{mdframed}[style=mdexam]
  \begin{example}\label{MAI:exam124} 
    Řešme: 
    \begin{equation*}
      \int\sin^3t\cos t\dd{t}.
    \end{equation*}

    \noindent\textbf{Řešení:}

    Položme \(\sin t = x\), \(\cos t\dd{t} =\dd{x}\). Pak získáme triviální integrál
    \begin{equation*}
      \int x^3\dd{x} = \frac{1}{4}x^4 + c = \frac{1}{4}\sin^4t + c,
    \end{equation*}
    Podmínky věty jsou splněny: Funkce \(\sin t\) je na intervalu \(-\infty, +\infty\) spojitá i se
    svou derivací \(\cos t\), její hodnoty leží v intervalu \(\langle-1,1\rangle\). V tomto
    intervalu je funkce \(x^3\) spojitá \cite[s.~261]{Brabec1989}. 
  \end{example}
\end{mdframed}
      %-------------------------------------
      \begin{mathexam}{\(\protect\scalerel{\int}{(1+x^2)^5x\dd{x}}\)
  \hfill\cite[s.~261]{Brabec1989}.}{exam125} 
  
  Zavedeme substituci \(1+x^2 = u\); odftud \(2x\dd{x} = \dd{u}\), tj. \(x\dd{x}=\frac{1}{2}\dd{u}\)
  (používáme jiné označení proměnných než je v uvedené větě). Potom platí 
  \begin{equation*}
    \frac{1}{2}\int u^5\dd{u} = \frac{1}{12}u^6 + c = \frac{1}{12}(1+x^2)^6 + c,
  \end{equation*}
  kde \(x\in(-\infty, +\infty)\), \(u\in\langle 1,+\infty)\); podmínky věty o substituci jsou
  zřejmě splněny.  
\end{mathexam}
      %-------------------------------------      
      \begin{mathexam}{Nechť \(\int{f(x)}\dd{x} = F(x) + c, \quad a,b\in\realset, a\neq0\)
  \hfill\cite[s.~261]{Brabec1989}.}{exam126}
  Pak platí
  \begin{equation}\label{mai:eq138}
    \int{f(ax + b)}\dd{x} = \frac{1}{a}F(ax + b) + c.
  \end{equation}
  Položíme \(ax + b =z\). Odtud \(a\dd{x} = \dd{z}\), \(\dd{x} = \frac{1}{a}\dd{z}\); 
  \begin{equation*}
    \boxed{\frac{1}{a}\int{f(z)}\dd{z} = \frac{1}{a}F(z) + c = \frac{1}{a}F(ax+b) + c.}
  \end{equation*}
  Například \(\int{\sin(2x+1)}\dd{x} = -\frac{1}{2}\cos(2x + 1) + c\) nebo \(\int{e^{-x}}\dd{x} =
  -e^{-x} + c\). 
\end{mathexam}
      %------------------------------------- 
      \begin{mdframed}[style=mdexam]
  \begin{example}\label{MAI:exam119}
    Cvičení:
    \begin{enumerate}[label=\alph*)]
      \item \(\displaystyle\int x\cdot e^{x^2}\dd{x}\)
      \item \(\displaystyle\int x^3\cdot e^{x^4}\dd{x}\)
    \end{enumerate}

    \noindent\textbf{Řešení:}

    \begin{enumerate}[label=\alph*)]
      \item Položme \(u=x^2\) potom dostaneme diferenciál \(\dd{u}=2x\dd{x}\). Podmínky věty jsou
            splněny. Funkce \(x^2\) je spojitá, včetně derivace \(x\). Dostáváme
            \(\frac{1}{2}\int{e^u\dd{u}}=\frac{1}{2}e^u=\frac{1}{2}e^{x^2} + c\). 
      \item Podobně \(u=x^4 \Rightarrow \dd{u}=4x^3\dd{x}\). Dostáváme 
            \(\frac{1}{4}\int{e^u}\dd{u} = \frac{e^u}{4} = \frac{e^{x^4}}{4} + c \).
    \end{enumerate}
  \end{example}
\end{mdframed}
      %-------------------------------------

      Ukažme, že platí 
      \begin{equation}\label{mai:eq139}
        \boxed{\int{\dfrac{f'(x)}{f(x)}}\dd{x} = \ln\abs{f(x)} + c.}
      \end{equation}
      Zavedeme substituci \(f(x) =t\), pak \(f'(x)\dd{x} = \dd{t}\) a 
      \begin{equation*}
        \int{\dfrac{\dd{t}}{t}} = \ln\abs{t} + c = \ln\abs{f(x)} + c
      \end{equation*}
      na každém intervalu, v němž \(f(x)\neq 0\) a existuje \(f'\). Připomeňme, že výraz
      \(\dfrac{f'(x)}{f(x)}\) se nazývá \emph{logaritmická derivace funkce} \(f\). Viz příklad
      \ref{mai:eq141}.
   

    % -------------------Substituční metoda II------------------------------------------------------
    \newpage
    \subsection{Substituční metoda II}
      Druhý typ substituční metody spočívá naopak v tom, že na místo původní proměnné \(x\) 
      dosadíme vhodnou funkci \(x = \psi(t)\). Místo primitivní funkce k funkci \(f(x)\) pak 
      hledáme primitivní funkci k funkci \(g(t) = f[\psi(t)]\psi'(t)\). Skutečně, je-li \(F(x)\) 
      primitivní funkcí k \(f(x)\), pak derivací složené funkce \(G(t) = F[\psi(t)]\) dostaneme
      \begin{equation*}
       G'(t) = F'[\psi(t)]\psi'(t) = f[\psi(t)]\psi'(t) = g(t).
      \end{equation*}

      %-------------------------------------
      \begin{mathexam}{Náhrada proměnně \(x\) funkcí. Typické jsou neurčité integrály, které vedou na
  goniometrické substituce, například \[\int\sqrt{1-x^2}\dd{x}\]}{exam122}
    
  Označme \(x=\psi(t)=\sin(t)  \Rightarrow \psi'(t)=\cos(t)\). Budeme potřebovat také základní
  goniometrické vzorce (\ref{MA1:eq_sincos},\ref{MA1:eq_cos2x} a \ref{MA1:eq_sin2x}). Můžeme psát
  \begin{gather*}
    \int\sqrt{1-\sin^2t}\cos t\dd{t} 
      = \int\cos^2 t \dd{t}  = \int\frac{1+\cos2t}{2}\dd{t}                         
  \end{gather*}
  Dostáváme
  \begin{align*}
      &= \frac{1}{2}t+\frac{\sin2t}{4}+c                       \\
      &= \frac{1}{2}\arcsin x + \frac{2\sin t\cos t}{4}        \\
      &= \frac{1}{2}\arcsin x + \frac{x\sqrt{1-x^2}}{2} + c.
  \end{align*}
  Správně bychom měli místo \(\sqrt{1 - \sin^2x}\) psát \(\abs{\cos x}\). Vzhledem k tomu, že jde o
  neurčitý integrál, je možné hledat primitivní funkci na intervalu, kde platí \(\cos x = \abs{\cos
  x}\).
\end{mathexam}
      %-------------------------------------

      Jistě nám neuniklo, že princip substitučních metod I a II je stejný. Jsou totiž obě založeny 
      na použití pravidla pro derivaci složené funkce.
       
    %--------------------------- Integrace racionální funkce--------------------------------------
    \newpage
    \subsection{Integrace racionální funkce}
      Jde o integrály 
      \begin{equation}\label{mai:eq157}
        \int\dfrac{P(x)}{Q(x)}\dd{x}
      \end{equation}
      kde \(P(x)\) a \(Q(x)\) jsou mnohočleny (budeme všude předpokládat, že \(P(x)\) a \(Q(x)\)
      mají \emph{reálné koeficienty}). Základní myšlenka: Funkci za znakem integrálu rozložíme v
      součet jednoduchých funkcí, které dovedeme integrovat:
      
      Je-li stupeň \(m\) mnohočlenu \(P(x)\) větší nebo roven stupni \(n\) mnohočlenu \(Q(x)\), pak
      dělením převedeme výraz \(P(x):Q(x)\) na součet mnohočlenu (stupně \(m-n\)) a \textbf{ryze}
      lomené funkce
      \begin{equation}\label{mai:eq158}
        \dfrac{R(x)}{Q(x)},
      \end{equation}
      kde stupeň \(r\) mnohočlenu \(R(x)\) je menší než stupeň \(n\) mnohočlenu \(Q(x)\).
      Mechanismus dělení:
      \begin{equation*}
        \rotatebox{90}{$
          {\polylongdiv[style=C,div=:]{x^3+4x^2-x+2}{x^2+x-3}}
        $}
      \end{equation*}

      \begin{mdframed}[style=mdmathlemma] 
        \begin{lemma}\label{mai:lemma010}
          Každý mnohočlen
          \begin{equation*}
            Q(x) = a_nx^n + a_{n-1}x^{n-1} + \cdots + a_1x + a_0
          \end{equation*}
          s reálnými koeficienty lze (až na pořadí faktorů jednoznačně) rozložit takto (srov.
          příklad \ref{mai:exam135}):
          \begin{equation}\label{mai:eq159}
            Q(x) = a_n(x-\alpha_1)^{k_1}(x-\alpha_2)^{k_2}\ldots                   \notag \\
            \ldots(x-\alpha_i)^{k_i}(x^2+p_1x+q_1)^{l_1}(x^2+p_2x+q_2)^{l_2}\ldots \notag \\ 
            \ldots(x^2+p_jx+q_j)^{l_j}.
          \end{equation}
          Přitom čísla \(\alpha_1, \alpha_2, \ldots \alpha_i\), \(p_1, q_1, p_2, q_2, \ldots \) jsou
          reálná a kvadratické výrazy v \ref{mai:eq157} nejsou rozložitelné v reálné lineární
          činitele - mají tedy záporný diskriminant tj. platí
          \begin{equation}\label{mai:eq160}
            \frac{p^2}{4}-q<0.
          \end{equation} 
        \end{lemma}
      \end{mdframed}

      Tvar rozkladu \ref{mai:eq160} vznikne ze známého rozkladu polynomu \(Q(x)\) v součin
      kořenových činitelů takto: Pokud \(\alpha\) je reálný kořen, zůstává činitel \(x-\alpha\) v
      rozkladu nezměněn. Je-li \(\beta\) komplexní kořen \(\beta = \beta_1-\imath\beta_2\), pak má
      \(Q(x)\) také komplexně sdružený kořen \(\bar{\beta} = \beta_1-\imath\beta_2\). Součin
      příslušných kořenových činitelů
      \begin{equation}\label{mai:eq161}
        \left[x-(\beta_1+\imath\beta_2)\right]  
        \left[x-(\beta_1-\imath\beta_2)\right]                                     \notag \\
        = (x-\beta_1)^2+\beta^2_2 = x^2-2\beta_1x + \beta^2_1+\beta^2_2. 
      \end{equation}
      dává kvadratický výraz s reálnými koeficienty. Je-li \(l\) násobnost kořene \(\beta\), pak
      touž násobnost má i kořen \(\bar{\beta}\) a v (\ref{mai:eq159}) dostaneme výraz tvaru
      \begin{equation*}
        (x^2+px+q)^l.
      \end{equation*}

      Uvedené příklady byly jednoduché, polynomy nízkého stupně a vzláštního tvaru, takže rozklad
      jsme mohli provést jednoduchými úpravami. Často se podaří zjistit kořen \(\alpha\) daného
      mnohočlenu zkusmo, zejména má-li tento mnohočlen celočíselné koeficienty. V aplikacích se však
      vyskytují mnohem složitější případy. Otázky nalézt kořeny polynomu a řešit soustavy lineárních
      rovnic v obecném případě patří do speciálních partií algebry a matematické analýzy, zejména
      numerických metod.
      
      Například funkci \(\frac{1}{x^2-4}\) integrujeme takto: Zřejmě v každém intervalu, který
      neobsahuje body \(x=2\), \(x=-2\), platí
      \begin{equation*}
        \dfrac{1}{x^2-4} = \dfrac{1}{4}\left(\dfrac{1}{x-2} - \dfrac{1}{x+2}\right)
      \end{equation*}
      takže v každém takovém intervalu jedině
      \begin{align*}
        \int\dfrac{1}{x^2-4}\dd{x}  
          &= \dfrac{1}{4}\int\left(\dfrac{1}{x-2} - \dfrac{1}{x+2}\right)\dd{x}             \\
          &= \dfrac{1}{4}\left(\ln\abs{x-2} - \ln\abs{x+2}\right) + c.        
      \end{align*}

      V tomto jednoduchém případě jsme našli rozklad velmi snadno. Je otázka, lze-li vždy racionální
      funkci rozložit na takové jednoduché funkce, které dovedeme integrovat. V této kapitole
      uvidíme, že odpověď je kladná a že nakonec půjde vždy jen o integrování mnohočlenů a o
      integrování několika typů racionálních funkcí (tzv. \emph{parciálních zlomků}). 
      %-------------------------------------
      \begin{mdframed}[style=mdexam]
  \begin{example}\label{MAI:exam135} 
    Polynom \(Q(x) = x^4-x^3-x^2-x-2\) má zřejmě kořen \(\alpha_1=-1\). Je 
    \begin{equation*}
      \rotatebox{90}{$
        {\polylongdiv[style=C,div=:]{x^4-x^3-x^2-x-2}{x+1}}
      $}
    \end{equation*}
    Výsledný mnohočlen má kořen \(\alpha_2=2\). Vydělením kořenovým činitelem \((x-2)\) dostaneme
    mnohočlen \((x^2+1)\), který je již nerozložitelný v reálné kořenové činitele. Tedy
    \begin{equation*}
      Q(x) \equiv x^4-x^3-x^2-x-2 = (x+1)(x-2)(x^2+1), 
    \end{equation*}
    což je již tvar \ref{mai:eq159}.
  \end{example}
\end{mdframed}
      %-------------------------------------
      
    %--------------------------- Integrace ryze lomené racionální funkce----------------------------
    \newpage
    \twocolumn[\subsection{Integrace ryze lomené racionální funkce rozkladem na parciální zlomky}]
      Některé příklady v předchozím odstavci, (viz např. \ref{mai:exam135}) jsme dělením čitatele
      integrandu jmenovatelem dostali rozklad integrandu na součet racionální funkce (polynomu) a
      ryze lomené racionální funkce. Integrování polynomu je snadné, neboť jde o součet integrálů
      tvaru \(\int c_kx^k dx\), kde \(k\) je celé nezáporné číslo. Omezíme se tedy na integrování
      \emph{ryze lomené racionální funkce},  tj. funkce ve tvaru \(P(x)/Q(x)\), kde \(P(x)\),
      \(Q(x)\) jsou polynomy, přičemž stupeň polynomu \(P(x)\) je menší než stupeň polynomu \(Q(x)\).
      Taková funkce může vzniknout součtem několika jednoduchých zlomků (viz. následující příklad
      \ref{mai:exam116}).      
      %-------------------------------------
      \begin{mdframed}[style=mdexam]
  \begin{example}\label{MAI:exam116}
    Upravte
    \begin{align*}
      \frac{1}{x-1}+\frac{x+2}{x^2+x+3} 
        &= \frac{x^2+x+3+x^2+x-2}{(x-1)(x^2+x+3)}                \\
        &= \frac{2x^2+2x+1}{x^3+2x-3}
    \end{align*}  
  \end{example}
\end{mdframed}
      %-------------------------------------
      
      Jsme tedy vedeni myšlenkou, zda naopak každá ryze lomená racionální funkce se dá rozložit
      na součet jednoduchých zlomků určitého tvaru - budeme jim říkat \textbf{parciální zlomky},
      které umíme integrovat. Tím se budeme zabývat v dalších odstavcích. 
    
      %-------------------------------------
      \begin{mdframed}[style=mdexam]
  \begin{example}\label{MAI:exam120}
    Řešme integrál 
    \begin{equation*}
      \int\frac{1}{x^2 - x + 1}\dd{x}, \qquad x\in\realset:
    \end{equation*}
    \noindent\textbf{Řešení:}

    Kvadratický polynom ve jmenovateli upravíme na čtverec \(f(x) = (x + m)^2 + n\) a dostaneme
    integrál
    \begin{equation*}
      \int\dfrac{1}{\left(x-\dfrac{1}{2}\right)^2+\dfrac{3}{4}}\dd{x},
    \end{equation*}
    na který lze aplikovat vzorec \ref{mai:eq121} z tabulky neurčitých integrálů: 
    \begin{equation*}
      \dfrac{1}{\sqrt{1-\left(\dfrac{1}{2}\right)^2}}\arctan
      \dfrac{x-\dfrac{1}{2}}{\sqrt{1-\left(\dfrac{1}{2}\right)^2}} 
    \end{equation*}
    \begin{equation*}
      \dfrac{2}{\sqrt{3}}\arctan\dfrac{2x-1}{\sqrt{3}}  =
      \dfrac{2\sqrt{3}}{3}\arctan\dfrac{\sqrt{3}(2x-1)}{3} + c
    \end{equation*}
  \end{example}
\end{mdframed}
      %-------------------------------------
      
      \begin{mathdef}{Parciální (částečný) zlomek}{def003}
        Parciálním (částečným) zlomkem, budeme nazývat zlomek tvaru
        \begin{equation}\label{mai:eq168}
          \frac{A}{(x-\alpha)^k} \qquad\text{nebo}\qquad\frac{Bx + D}{(x^2 + px + q)^k}
        \end{equation}  
        kde \(A,\ B,\ D,\ \alpha,\ p,\ q\) jsou reálné konstanty, trojčlen \(x^2 + px + q\) nemá
        reálné kořeny (tj. má záporný diskriminant $p^2-4q < 0$), $k$ celé nezáporné číslo.          
      \end{mathdef}    
      
      Jak ukazuje definice \ref{mai:def003}, jsou to funkce těchto čtyř typů:
      \begin{equation}\label{mai:eq166}
        \boxed{f_1(x) = \dfrac{A}{x -\alpha}.} \hfill k = 1
      \end{equation}
      Ten již dovedeme snadno integrovat
      \begin{equation*}
        \int\dfrac{A}{x -\alpha} = A\ln\abs{x-\alpha} + c, 
      \end{equation*}
      
      Druhý typ zlomku dostáváme pro případ \( k>1\)
      \begin{equation}\label{mai:eq167}
        \boxed{f_2(x) = \dfrac{A}{(x -\alpha)^k}.} \hfill k>1
      \end{equation}
      Integrál druhého zlomku, vypočteme substitucí $x-\alpha=t$ a \(\dd{x} = \dd{t}\), čímž snadno
      dostaneme 
      \begin{equation}\label{mai:eq169}
        \int\frac{A}{(x-\alpha)^k}\dd{x} = \int\frac{A}{t^k}\dd{t}.
      \end{equation}
      Tento integrál se rovná
      % https://tex.stackexchange.com/questions/299407/does-raisetag-actually-do-anything
      \begin{flalign}\label{mai:eq170}
        \boxed{\int\frac{A}{(x-\alpha)^k}\dd{x} = -\frac{A}{k-1}\frac{1}{(x-\alpha)^{k-1}} + c.} &&
        \raisetag{20pt}
      \end{flalign}    
      Výsledek platí na každém intervalu neobsahujícím bod \(\alpha\).
      
      %-------------------------------------
      \begin{mathexam}{Vypočtěte neurčité integrály \(\protect\scalerel{\int}{\dfrac{2}{x+5}\dd{x}}\),
  \(\quad\protect\scalerel{\int}{\dfrac{\sqrt{5}}{(x+7)^4}\dd{x}}\) \hfill\cite[s.~71]{Knichal}}{exam138}

  První integrál odpovídá vzorci \ref{mai:eq166}:
  \begin{equation*}
    \scalerel{\int}{\dfrac{2}{x+5}\dd{x}} = 2\ln\abs{x+5} + c = \ln(x+5)^2 + c
  \end{equation*}

  U druhého integrálu položíme \(x+7 = t\), takže \(\dd{x} = \dd{t}\) a dostaneme
  \begin{align*}
    \scalerel{\int}{\dfrac{\sqrt{5}}{(x+7)^4}\dd{x}} &= 
    \sqrt{5}\scalerel{\int}{\dfrac{\dd{t}}{t^4}} = \dfrac{\sqrt{5}}{-3t^3} + c.   \\
    \shortintertext{a tedy}
    \scalerel{\int}{\dfrac{\sqrt{5}}{(x+7)^4}\dd{x}} &=-\dfrac{\sqrt{5}}{3}\dfrac{1}{(x+7)^3} + c.
  \end{align*}
\end{mathexam}
      %-------------------------------------

      U integrálu třetího zlomku se trochu zapotíme
      \begin{equation}\label{mai:eq137}
        \boxed{f_3(x) = \frac{Bx + D}{x^2+px+q}}
      \end{equation} 
       předpokládáme samozřejmě, že kvadratický trojčlen \(x^2+px+q\) nemá kořeny v reálném oboru,
       takže jej nelze rozložit do tvaru \((x - \alpha)(x - \beta)\) s reálnými \(\alpha\),
       \(\beta\) (\cite[p.~140]{Musilova2009MA1}). Platí 
      \begin{gather*}
        \frac{B}{2}\underbrace{\frac{2x + p}{x^2+px+q}}_{I_1} + 
        \left(D - \frac{Bp}{2}\right)\underbrace{\frac{1}{x^2+px+q}}_{I_2}
      \end{gather*} 
      K čemu byla předchozí úprava dobrá? Původní parciální zlomek jsme napsali jako součet dvou
      zlomků, z nich je v čitateli derivace funkce \(u(x)=x^2+px+q\). První zlomek se tak stává
      kandidátem na použití substituční metody I.
      \begin{equation*}
        I_1=\frac{B}{2}\int{\frac{(2x+p)}{x^2+px+q}\dd{x}}=\frac{B}{2}\int\frac{1}{u^k}\dd{u} ,                   
      \end{equation*}  
      \begin{equation*}
        \boxed{\frac{B}{2}\int{\frac{(2x+p)}{x^2+px+q}\dd{x}} = \frac{B}{2}\ln(x^2+px+q) + c}
      \end{equation*}

      Nemuseli jsme zde psát \(\abs{x^2+px+q}\), neboť kvadratický mnohočlen se záporným
      diskriminantem nabývá pro všechna \(x\) hodnot téhož znaménka, rovného znaménku koeficientu u
      \(x^2\). V našem případě je tento koeficient rovný jedné, takže výraz \(x^2+px+q\) je pro
      všechna \(x\) kladný.   


      Při výpočtu durhého integrálu použijeme procedury \uv{doplnění na čtverec}
      (\cite[p.~407]{Rektorys1963}, \cite[s.~264]{Brabec1989}) takto:
      \begin{equation*}
        x^2 + px + q = \left(x + \dfrac{p}{2}\right)^2 + q -  \left(\dfrac{p}{2}\right)^2.
      \end{equation*}
      Výraz \(q-(p/2)^2\) je jistě kladný, neboť diskriminant \(p^2-4q\) trojčlenu \(x^2+px+q\) je
      podle předpokladu záporný. Položíme-li tedy \(a = \sqrt{q-(p/2)^2}\), můžeme psát 

      \begin{equation}
        I_2 =\int\dfrac{1}{x^2+px+q}\dd{x} 
            =\int{\dfrac{1}{\left(x+\frac{p}{2}\right)^2 + a^2}}\dd{x}
      \end{equation}
      substitucí \(x + \frac{p}{2} = t\), \(\dd{x} = \dd{t}\) dostaneme 
      \begin{gather*}
        \int{\dfrac{1}{\left(x+\frac{p}{2}\right)^2 + a^2}}\dd{x} = 
        \int{\dfrac{1}{t^2+a^2}}\dd{t}
      \end{gather*}
      takže
      \begin{equation*}
        \dfrac{1}{a}\arctan\dfrac{t}{a} + c = \dfrac{1}{a}\arctan\dfrac{x + \frac{p}{2}}{a} + c 
      \end{equation*}
      Je tedy
      \begin{align}
          &\int\dfrac{1}{x^2+px+q}\dd{x} = \frac{B}{2}\ln(x^2+px+q) +              \notag     \\ 
        + &\frac{1}{\sqrt{q-\left(\frac{p}{2}\right)^2}}
                \arctan\frac{x+\frac{p}{2}}{\sqrt{q-\left(\frac{p}{2}\right)^2}}.  \label{mai:eq171}
      \end{align}
     
      %-------------------------------------
      \begin{mdframed}[style=mdexam]
  \begin{example}\label{mai:exam140}
    Vypočtěme 
    \begin{equation*}
      \int\dfrac{5x+8}{x62+4x+7}\dd{x}
    \end{equation*}
    Čerpáno z \cite[s.~73]{Knichal}.  Kvadratický trojčlen ve jmenovateli má záporný diskriminatn,
    \(16-4\cdot7 = -12<0\), a proto jde skutečně o integrál typu 3.
    
    Postup, který jsme ukázali obecně, provedeme nyní na speciálním případě. Nejdříve rozložíme
    integrand na dva zlomky tak, aby první měl v čitateli derivaci jmenovatele: 
    \begin{equation*}
      \frac{5}{2}\frac{2x+4}{x^2+4x+7} + \left(8-\frac{5}{2}\cdot4\right)\frac{1}{x^2+4x+7}.
    \end{equation*}
    Tedy
    \begin{align*}
      & \int\dfrac{5x+8}{x^2+4x+7}\dd{x}  = \dfrac{5}{2}\int\frac{2x+4}{x^2+4x+7}\dd{x}  \\
      &-2\int\frac{1}{x^2+4x+7}\dd{x}     = \dfrac{5}{2}\ln(x^2+4x+7) - 2I
    \end{align*}
    Abychom určili \(I\) budeme postupovat takto:
    \begin{equation*}
      I = \int\dfrac{1}{x^2+4x+7}\dd{x} = \int\dfrac{1}{(x+2)^2+3}\dd{x} 
    \end{equation*}
    Nyní položíme \(x+2=t\), takže \(\dd{x}= \dd{t}\). Dostaneme
    \begin{align*}
      I &= \int\dfrac{1}{t^2+3}\dd{t} = \dfrac{1}{\sqrt{3}}\arctan\dfrac{t}{\sqrt{3}} + c \\
        &= \dfrac{1}{\sqrt{3}}\arctan\dfrac{x+2}{\sqrt{3}} + c
    \end{align*}
    Výsledek integrálu \(\int\dfrac{5x+8}{x62+4x+7}\dd{x}\) tedy je 
    \begin{equation*}
      \dfrac{5}{2}\ln(x^2+4x+7) - \dfrac{2}{\sqrt{3}}\arctan\dfrac{x+2}{\sqrt{3}} + c
    \end{equation*}
  \end{example}
\end{mdframed}
      %------------------------------------- 

      Zbývá vyřešit nejpracnější problém, určit primitivní funkci ke zlomku 4. typu 
        \begin{equation}\label{mai:eq172}
          \boxed{f_4(x) = \frac{Bx + D}{(x^2+px+q)^k}} \qquad k>1
        \end{equation}
      
      Nejdříve prozkoumáme integrál, se kterým jsme se setkali v předchozím příkladu pro případ
      \(k = 2\) (viz \cite[p.~141]{Musilova2009MA1}). Jedná se o integrál typu 
      \begin{equation*}
        \int{1\cdot\dfrac{1}{(t^2 + 1)^{2}}\dd{t}}.
      \end{equation*}
      Integrál vyjádříme nikoli jako arkustangentu, ale pomocí metody per partes, pro kterou zvolíme
      \(u'(t)=1\rightarrow u(t)=t\), \(v(t) = \dfrac{1}{t^2+1}\rightarrow v'(t) =
      \dfrac{2t}{(t^2+1)^2}\),
      \begin{gather*} 
        \begin{align*}
          \arctan u &=  \int{1\cdot\dfrac{1}{(t^2 + 1)^{2}}\dd{t}}              \\
                    &=  \dfrac{u}{1+u^2} + \int\dfrac{2u^2}{(1+u^2)^2}\dd{u},   \\
                    &=  \dfrac{u}{1+u^2} + \int\dfrac{2(1+u^2)}{(1+u^2)^2}\dd{u}
                                        - \int\dfrac{2\dd{u}}{(1+u^2)^2}      \\
          \arctan u &=  \dfrac{u}{1+u^2} + 2\arctan u - 2\int\dfrac{\dd{u}}{(1+u^2)^2}  
        \end{align*}
      \end{gather*}  
      Odtud již můžeme požadovaný integrál vyjádřit jako
      \begin{equation*}
        \int\dfrac{1}{(1+u^2)^2}\dd{u} = \dfrac{1}{2}\left(\dfrac{u}{1+u^2} + \arctan u\right) + c. 
      \end{equation*}

      Integrál pro \(k=2\) jsme vyjádřili pomocí arkustangenty, tj. pomocí integrálu pro \(k=1\).
      Metodou per partes aplikovanou na integrál
      \begin{equation*}
        I_{k-1} = \int1\cdot\dfrac{1}{(1+u^2)^{k-1}}\dd{u}
      \end{equation*}
      můžeme vyjádřit integrál \(I_k\) pomocí \(I_{k-1}\) a získat \emph{rekurentní vzorec}
      \begin{align}
        I_k &= \int\dfrac{1}{(1+u^2)^k}\dd{u}                                   \label{mai:eq165} \\
            &= \dfrac{1}{2(k-1)}\left[\dfrac{u}{(1+u^2)^{k-1}}+(2k-3)I_{k-1}\right]  \notag
      \end{align}

      Nyní jsme připraveni řešit \ref{mai:eq172} obecně. I zde, dříve než přistoupíme k integrování,
      rozložíme zlomek obdobným způsobem jako v předcházejícím případě:
      \begin{gather*}
        \frac{B}{2}\underbrace{\frac{2x + p}{(x^2+px+q)^k}}_{I_3} + 
        \left(D - \frac{Bp}{2}\right)\underbrace{\frac{1}{(x^2+px+q)^k}}_{I_4}
      \end{gather*} 
      Integrál \(I_3\) určíme snadno substitucí \(x^2+px+q = z\), \((2x+p)\dd{x} = \dd{z}\),
      \begin{align}
        I_3 &= \frac{B}{2}\int\dfrac{1}{(x^2+px+q)^k}\dd{x} 
             = \frac{B}{2}\int\dfrac{\dd{z}}{z^k} =                                \notag   \\
            &=-\frac{B}{2}\dfrac{1}{k-1}\dfrac{1}{z^{k-1}} + c                     \notag   \\
            &=-\frac{B}{2(k-1)}\dfrac{1}{(x^2+px+q)^{k-1}} + c                     \label{mai:eq173} 
      \end{align}      

      Druhý integrál \(I_4\) převedeme opět úpravami obdobnými těm, kterých jsme použili u zlomku
      \ref{mai:eq137} na tvar \(1/(t^2+a^2)^k\):
      \begin{align*}
        I_3 &= \left(D - \frac{Bp}{2}\right)\int\dfrac{1}{(x^2+px+q)^k}\dd{x}      \notag   \\  
            &= \left(D - \frac{Bp}{2}\right)\int\dfrac{1}{
                  \left[
                    \left(
                      x+\frac{p}{2}
                    \right)^2+q-
                    \left(
                      \frac{p}{2}
                    \right)^2
                  \right]^k
                }\dd{x}.
      \end{align*}    

      V integrálu na pravé straně provedeme substituci \(x + \frac{p}{2} = t\), \(\dd{x} = \dd{t}\),
      a položíme 
      \begin{equation*}
        \sqrt{q - \left(\dfrac{p}{2}\right)^2} = a.
      \end{equation*}
      dostaneme
      \begin{equation*}
        \left(D - \frac{Bp}{2}\right)\int\dfrac{1}{(t^2+a^2)^k}\dd{t}.
      \end{equation*}
      Pro integrál na pravé straně zavedeme označení
      \begin{equation*}
        \int\dfrac{1}{(t^2+a^2)^k}\dd{t} = I_k
      \end{equation*}
      a odvodíme pro jeho výpočet rekurentní vzorec. Integrál \(I_k\) nejprve upravíme:
      \begin{align*}
        I_k &= \int\dfrac{1}{(t^2+a^2)^k}\dd{t} 
             = \dfrac{1}{a^2}\int\dfrac{t^2+a^2-t^2}{(t^2+a^2)^k}\dd{t}                       \\
            &= \dfrac{1}{a^2}\underbrace{\int\dfrac{1}{(t^2+a^2)^{k-1}}\dd{t}}_{I_{k-1}}          
             - \dfrac{1}{a^2}\underbrace{\int\dfrac{t^2}{(t^2+a^2)^k}\dd{t}}_{I_5}                
      \end{align*}

      První integrál na pravé straně, jsme ve shodě se zvoleným označením pro \(I_k\), označili
      \(I_{k-1}\). Na druhý integrál použijeme metody integrování per partes. Položíme 
      \begin{alignat*}{2}
        u  &= t, \quad v' &&=   \dfrac{t}{(t^2+a^2)^k} = \dfrac{1}{2}\dfrac{2t}{(t^2+a^2)^k}   \\
        \shortintertext{a dostaneme}
        u' &= 1, \quad v  &&= - \dfrac{1}{2(k-1)}\dfrac{1}{(t^2+a^2)^{k-1}}.
      \end{alignat*}
      Bude
      \begin{alignat}{2}
        I_5 &=  &&{} \int\dfrac{t^2}{(t^2+a^2)^k}\dd{t} =
                     \int\dfrac{t\cdot t}{(t^2+a^2)^k}\dd{t}                 \notag \\
            &=  &&-  \dfrac{1}{2(k-1)}\cdot\dfrac{t}{(t^2+a^2)^{k-1}} +      \notag \\
            &{} &&+  \dfrac{1}{2(k-1)}\int\dfrac{1}{(t^2+a^2)^{k-1}}\dd{t}.  \label{mai:eq174}
      \end{alignat}
      Dosadíme-li tento výsledek do rovnice pro \(I_k\), dostaneme
      \begin{gather}
        \begin{aligned}
        I_k &= \dfrac{1}{a^2}I_{k-1} - \dfrac{1}{a^2}
              \left[
                -\dfrac{1}{2(k-1)}\dfrac{1}{(t^2+a^2)^{k-1}} + \dfrac{1}{2(k-1)}I_{k-1}                
              \right]                                                                            \\
        \shortintertext{čili}     
        I_k &= \dfrac{t}{2(k-1)a^2(t^2+a^2)^{k-1}} + 
               \dfrac{1}{a^2}\left(1-\dfrac{1}{2(k-1)}\right)I_{k-1}                      \notag \\               
        \shortintertext{a konečně}
        I_k &= \dfrac{t}{2(k-1)a^2(t^2+a^2)^{k-1}} + 
              \dfrac{2k-3}{2k-2}\dfrac{1}{a^2}I_{k-1}                              \label{mai:eq175}
      \end{aligned}  
    \end{gather}

      Pomocí vzorce \ref{mai:eq175} se redukuje výpočet integrálu \(I_k\) na výpočet integrálu
      \(I_{k-1}\) je tabulkový. Přejdeme-li zpět k proměnné \(x\) a uvážíme \ref{mai:eq173},
      dostaneme hledaný integrál funkce
      \begin{equation*}
        \dfrac{Bx + D}{(x^2+px+q)^k}
      \end{equation*}

      %-------------------------------------
      \begin{mathexam}{\(\scalerel{\int}{\dfrac{4x + 1}{(x^2 + 9)^3}\dd{x}}.\)
  \hfill\cite[s.~76]{Knichal}}{exam141} 
  Diskriminant kvadratického výrazu \(x^2+9\) je zřejmě záporný, takže jde o integrál typu 4.
  Nejprve rozložíme integrand na součet,
  \begin{align*}
    \dfrac{4x+1}{(x^2+9)^3} 
      &= 2\cdot\dfrac{2x}{(x^2+9)^3} + \dfrac{1}{(x^2+9)^3},             \\
    \shortintertext{takže}
    \int\dfrac{4x+1}{(x^2+9)^3}\dd{x} 
      &= 2\int\dfrac{2x}{(x^2+9)^3}\dd{x} + \underbrace{\int\dfrac{1}{(x^2+9)^3}\dd{x}}_{I_3},
  \end{align*}
  V prvním integrálu na pravé straně použijeme substituce \(x^2+9=z\), \(2x\dd{x} = \dd{z}\), z
  čehož 
  \begin{align*}
    2\int\dfrac{2x}{(x^2+9)^3}\dd{x} 
      &= 2\int z^{-3}\dd{z} = -z^{-2} + c  \\
      &= -\dfrac{1}{(x^2+9)^2} + c.
  \end{align*}  
  Pro druhý integrál zavedeme označení \(I_3\) a použijeme rekurentní vztah \ref{mai:eq175}
  \begin{align*}
    I_3 &= \dfrac{x}{2\cdot(3-1)\cdot9\cdot(x^2+9)^2}+\dfrac{6-3}{6-2}\cdot\dfrac{1}{9}\cdot I_2\\
    I_2 &= \dfrac{x}{2\cdot(2-1)\cdot9\cdot(x^2+9)  }+\dfrac{4-3}{4-2}\cdot\dfrac{1}{9}\cdot I_1\\
    \shortintertext{\(I_1\) je však tabulkový integrál}
    I_1 &= \scalerel{\int}{\dfrac{1}{x^2+9}\dd{x}} = \dfrac{1}{3}\arctan\dfrac{x}{3} + c
  \end{align*}
  takže
  \begin{multline*}
    I_3 = 
      \dfrac{x}{36(x^2+9)^2} +        \\
        \left(
          \dfrac{3}{4}\cdot\dfrac{1}{9}\dfrac{x}{18(x^2+9)^2} + 
          \dfrac{1}{2}\cdot\dfrac{1}{9}\cdot\dfrac{1}{3}\arctan\dfrac{x}{3}
        \right) + c
  \end{multline*}
  Je tedy
  \begin{multline*}
    \scalerel{\int}{\dfrac{4x + 1}{(x^2 + 9)^3}\dd{x}} 
      = - \dfrac{1}{(x^2+9)^2} + \dfrac{x}{36(x^2+9)^2} +            \\
        + \dfrac{3}{4}\cdot\dfrac{1}{9}\cdot\dfrac{x}{18(x^2+9)^2}
        + \dfrac{3}{4}\cdot\dfrac{1}{9}\cdot\dfrac{1}{2}\cdot
          \dfrac{1}{9}\cdot\dfrac{1}{3}\arctan\dfrac{x}{3} + c
  \end{multline*}
  Po úpravě dostaneme výsledek
  \begin{equation*}
    \dfrac{x^3+15x-216}{216(x^2+9)^2} + \dfrac{1}{648}\arctan\dfrac{x}{3} + c.
  \end{equation*}
\end{mathexam}
      %-------------------------------------  
      
    % ---------------- Rozklad ryze lomené funkce v parciální zlomky -------------------------------
    \subsubsection{Rozklad ryze lomené funkce v parciální zlomky}
    Nechť je dána racionální funkce $R = \frac{P}{Q}$ s reálnými koeficienty. Můžeme
    předpokládat, že je \emph{ryze lomená}\footnote{tj. stupeň polynomu $P$ je menší než
    stupeň polynomu $Q$}. Pokud by tomu tak nebylo, dostaneme dělením čitatele jmenovatelem
    zlomku součet polynomu a ryze lomené racionální funkce.

    %-------------------------------------
    \begin{mdframed}[style=mdexam]
  \begin{example}\label{MAI:exam136}
    \begin{equation*}
      R(x) = \dfrac{x + 2}{x^3 - x}
    \end{equation*}
    \noindent\textbf{Řešení:}

    Racionální funkce je ryze lomená. Jmenovatel \(x^3 + x = x(x^2-1) = x(x+1)(x-1)\) má jedoduché
    reálné různé kořeny \(0,-1,1\). Podle věty o rozkladu platí
    \begin{equation*}
      \dfrac{x+2}{x(x+1)(x-1)} = \dfrac{A}{x} + \dfrac{B}{x+1} + \dfrac{C}{x-1}
    \end{equation*}
    Odtud
    \begin{align}
      x+2 &= A(x+1)(x-1) + Bx(x-1)            \nonumber          \\ 
          &+ Cx(x+1),                         \label{mai:eq162}  \\
      x+2 &= Ax^2 - A +Bx^2 - Bx + Cx^2 + Cx  \nonumber          \\
      x+2 &= (A+B+C)x^2 + (-B+C)x - A         \nonumber
    \end{align}
    Porovnáním koeficientů u jednotlivých mocnin proměnné \(x\) na levé a pravé straně dostaneme
    rovnice 
    \begin{equation*}
      \begin{array}{rcrcrcl}
        A &+& B &+& C &=&0, \\
          &-& B &+& C &=&1, \\
       -A & &   & &   &=&2. 
      \end{array}   
    \end{equation*}
    Řešením této lineární soustavy dostáváme \(A = -2\), \(B = \frac{1}{2}\), \(C = \frac{3}{2}\).
    Je tedy 
    \begin{equation}\label{mai:eq163}
      \dfrac{x+2}{x^3-x} = -\dfrac{2}{x} + \dfrac{1}{2}\dfrac{1}{x+1} + \dfrac{3}{2}\dfrac{1}{x-1}
    \end{equation} 
    Sloučením zlomků vpravo se můžeme přesvědčit o správnosti vysledku. Převzato z
    \cite[p.~268]{Brabec1989}
  \end{example}
\end{mdframed}
    %-------------------------------------

    Soustavu rovnic v příkladu \eqref{mai:exam136} pro výpočet koeficientů \(A\), \(B\), \(C\)
    dostaneme též dosazením tří libovolných čísel do rovnice \ref{mai:eq162} (polynomy jsou si
    rovné, tedy mají stejné hodnoty pro libovolné \(x\)), aniž na pravé straně uspořádáme podle
    mocnin \(x\). Dosadíme-li postupně \(x=0\), \(x=-1\), \(x=1\) (tedy právě kořeny jmenovatele),
    dostaneme ihned \(2=-A\), \(1=2B\), \(3=2C\) a odtud už snadno hledané koeficienty. Rovnost
    \ref{mai:eq163} platí ovšem pro všechna \(x\) s výjimkou právě kořenů  \(x=0\), \(x=-1\) a
    \(x=1\).

    %-------------------------------------
    \begin{mathexam}{\(R(x) = \dfrac{2x^3 + x + 2}{x^4 + x^3 + x^2}\)}{exam137} 
  Rozložíme jmenovatele v reálném oboru: \(x^4+x^3+x^2 = x^2(x^2 + x + 1)\), trojčlen v závorce má
  diskriminant rovný \(-3<0\), tedy dále jej rozkládáat nebudme. Potom
  \begin{gather*}
    \dfrac{2x^3+x+2}{x^2(x^2+x+1)} = \dfrac{A}{x^2} + \dfrac{B}{x} + \dfrac{Cx+D}{x^2+x+1}
  \end{gather*}
  Úpravou dostáváme
  \begin{align*}
    2x^3+x+2 &= A(x^2+x+1) + Bx(x^2+x+1)     \\ 
             &+ x^2(Cx+D),                   \\
    2x^3+x+2 &= Ax^2 + Ax + A + Bx^3 + Bx^2  \\
             &+ Bx + Cx^3 + Dx^2             \\
    2x^3+x+2 &= (B+C)x^3 + (A+B+D)x^2        \\
             &+ (A+B)x +A.
  \end{align*}
  Odtud porovnáním koeficientů dostaneme soustavu rovnic
  \begin{equation*}
    \begin{array}{rcrcrcrcl}
        & & B &+& C & &   &=& 2,  \\
      A &+& B & &   &+& D &=& 0,  \\
      A &+& B & &   & &   &=& 1,  \\
      A & &   & &   & &   &=& 2.
    \end{array}
  \end{equation*}
  Řešením soustavy dostaneme \(A=2\),\(B = -1\), \(C=3\), \(D=-1\), tj.  
  \begin{gather*}
    \dfrac{2x^3+x+2}{x^2(x^2+x+1)} = \dfrac{2}{x^2} - \dfrac{1}{x} + \dfrac{3x-1}{x^2+x+1}
  \end{gather*}
\end{mathexam}
    %-------------------------------------

    Obdobný postup při určování koeficientů \(A\), \(B\), \(C\) a \(D\) zde není tak výhodný jako
    v příkladu \eqref{mai:exam136}, protože jmenovatel nemá reálné různé kořeny. 

    %-------------------------------------
    \begin{mathexam}{\(\scalerel{\int}{\dfrac{8x-31}{x^2-9x+14}\dd{x}}\) \hfill
  \cite[s.~90]{Knichal}}{exam128} 
  Převzato z \cite[s.~90]{Knichal} Kořeny polynomu ve jmenovateli $\alpha_1 = 2$, $\alpha_2 = 7$
  jsou jednoduché - každému z nich bude v rozkladu odpovídat jen jeden člen
  $$\frac{8x-31}{x^2-9x+14} = \frac{A}{x-2} + \frac{B}{x-7}.$$ Členy mnohočlenu na pravé straně
  seřadíme podle mocnin $x$ $$8x-31 = x(A+B)+(7A-2B).$$ Porovnáním odpovídajících si koeficientů
  dostaneme
  \begin{align*}
    8   &=   \; A + \, B \\
    -31 &= -7A - 2B
  \end{align*}
  Řešením této soustavy je $A = 3, B = 5$. Platí tedy (pro všechna $x \neq 2$ a $x \neq 7$)
  $$\frac{8x-31}{x^2-9x+14} = \frac{3}{x-2} + \frac{5}{x-7}.$$
  \begin{align*}
      &= \int{\frac{3}{x-2}}dx + \int{\frac{5}{x-7}}dx      \\
      &= 3\ln\abs{x-2} + 3\ln\abs{x-7} + C.
  \end{align*}
  Výsledek platí v každém intervalu, který neobsahuje body \(x = 2\), \(x = 7\).
\end{mathexam}
    %-------------------------------------

    %-------------------------------------
    \begin{mathexam}{\(\scalerel{\int}{\dfrac{19x+15}{x^2-x-2}\dd{x}} \quad x\in
  \realset-\{1,2\}\)}{exam129} 
  Kořeny polynomu ve jmenovateli $\alpha_1 = -1$, $\alpha_2 = 2$ jsou jednoduché - každému z nich
  bude v rozkladu odpovídat jen jeden člen: 
    \begin{align*}
      \frac{19x+15}{x^2-x-2}      &= \frac{A}{x+1} + \frac{B}{x-2} \\
                        19x +15   &= A(x-2) + B(x+1)               \\
                        19x +15   &= x(A+B) - 2A + B               \\
                        19        &= A + B                         \\
                             15   &=        - 2A + B
    \end{align*}              
    Řešením této soustavy je \(A = \frac{4}{3}\), \(B = \frac{53}{3}\).
    \begin{align*}
      &= \frac{4}{3}\int{\frac{1}{x+1}}\dd{x} + \frac{53}{3}\int{\frac{1}{x-2}}\dd{x}  \\
      &= \frac{4}{3}\ln\abs{x+1} - \frac{53}{3}\ln\abs{x-2} +  C
    \end{align*}   
\end{mathexam}
    %-------------------------------------
    
    %-------------------------------------
    \begin{mathexam}{Řešme \(\protect\scalerel{\int}{\frac{2x^2+34x+14}{x^3-4x^2-x-4}}\dd{x}\)
  \hfill\cite[s.~90]{Knichal}}{exam115}
    
    Polynom $Q(x)=x^3-4x^2-x-4$ má kořeny $\alpha_{1,2}=\pm1$, $\alpha_{3}=-4$, které jsou
    jednoduché tj. $Q(x)=(x-1)(x+1)(x+4)$ $$\frac{2x^2+34x+14}{x^3-4x^2-x-4} =
    \frac{A}{x-1}+\frac{B}{x+1}+\frac{C}{x+4}$$ Vynásobíme-li tuto rovnici společným jmenovatelem
    zlomků pravé strany (polynomem $Q(x)$), dostaneme
    \begin{gather*}
        \begin{align*}
          &= A(x+1)(x+4) + B(x-1)(x+4) + C(x-1)(x+1) \\
          &= A(x^2+5x+4) + B(x^2+3x-4) + C(x^2-1)    \\
          &= (A+B+C)x^2  + (5A+3B)x    + (4A-4B-C)
        \end{align*}
    \end{gather*}
    Porovnáním odpovídajících si koeficientů u stejných mocnin \(x\) polynomu \(2x^2+34x+14\)
    dostaneme pro nez\-ná\-mé koeficienty $A, B, C$ soustavu rovnic
    \begin{align*}
    % \nonumber to remove numbering (before each equation)
       A+   B + C &= 2 \\
      5A + 3B     &= 34 \\
      4A - 4B - C &= 14
    \end{align*}
    Řešením této soustavy je $A = 5, B = 3, C = -6$ a tedy
    $$\frac{2x^2+34x+14}{x^3-4x^2-x-4} = \frac{5}{x-1}+\frac{3}{x+1}-\frac{6}{x+4}$$
    Dostáváme tři jednoduché integrály
    \begin{equation*}
      \int{\frac{5}{x-1}}\dd{x} + \int{\frac{3}{x+1}}\dd{x} + \int{\frac{6}{x+4}}\dd{x}            
    \end{equation*}
    jejichž řešení je 
    \begin{equation*}
      5\ln\abs{x-1} +  3\ln\abs{x+1} - 6\ln\abs{x+4} +c.
    \end{equation*}
\end{mathexam}
    %-------------------------------------
      
  % -----------------------Funkce typu {f(x)=\sqrt{ax+b}} ------------------------------------
  \twocolumn[\subsection{Integrály iracionálních funkcí}]
    \subsubsection*{Funkce typu $\boxed{f(x)=\sqrt{ax+b}}$ :}
      Funkci, jež je dána rovnicí, jež obsahuje polynomy proměnné x  ve výrazu $\sqrt{ax+b}$,
      v němž $ax+b>0$, $a>0$, integrujeme pomocí substituce:
      \begin{equation}\label{ma:eq_sub_fce1}
          u=\sqrt{ax+b},\quad du=\frac{1}{2}\frac{a}{u}dx,\quad dx=2\frac{u}{a}du
      \end{equation}
      Je-li potřeba dosadit do integrované funkce také za $x$, vyjádříme ze substituční
      rovnice $x=\frac{u^2-b}{a}$.
    % ----------------------Funkce typuf(x)=\frac{1}{\sqrt{x^2+a}}, a\neq0 -------------------- 
    \subsubsection*{Integrál typu $\boxed{\int\frac{1}{\sqrt{x^2+a}}\dd{x}}, a\neq0$ :}
      \(\int\frac{1}{\sqrt{x^2+a}}\dd{x}\):\vskip0.5mm
      Užijeme \textbf{Eulerovu substituci}: \(u=x+\sqrt{x^2+a}\), a dostáváme
      \(\dd{u}=\dfrac{u}{\sqrt{x^2+a}}\dd{x}\), \(\dfrac{\dd{u}}{u}=\dfrac{1}{\sqrt{x^2+a}}\dd{x}\).
      \begin{equation}
        \int{\frac{\dd{u}}{u}}=\ln\abs{u} =\ln\left\lvert x+\sqrt{x^2+a}\right\rvert+c
      \end{equation}
  % --------------------------Integrály goniometrických funkcí------------------------------------
  \newpage
  \subsection{Integrace goniometrických funkcí}
    \subsubsection{Integrály tvaru \(\int R(\cos x, \sin x)\dd{x}\) kde \(R(u, v)\) je racionální
    funkce}
      Zavedeme substituci \(\tan(x/2)\). Užitím základních vztahů mezi goniometrickými funkcemi
      dostáváme 
      \begin{equation}\label{mai:eq164}
        \cos^2\dfrac{x}{2} = \dfrac{1}{1+ \tan^2\dfrac{x}{2}} = \dfrac{1}{1+t^2},
      \end{equation}
  % ---------------- Sbírka řešených příkladů ------------------------------------------------------
  \section{Sbírka řešených příkladů}
    \begin{excercise}\label{mai:cviko001}
      Hledejme primitivní funkce \(F(x)\) k následujícím funkcím
      \begin{flalign}
        &\int{xe^x\dd{x}},                                                      &\label{mai:eq140}\\
        &\int\frac{x}{x^2+1}\dd{x}                                              &\label{mai:eq141}\\
        &\int{\arctan x\dd{x}},                                                 &\label{mai:eq142}\\
        &\int{\frac{2x^4-5x^2+14x+13}{x^2-x-2}\dd{x}}                           &\label{mai:eq144}\\
        &\int{\sqrt{x^2+a}\dd{x}}, \quad a\neq0, x^2+a>0,                       &\label{mai:eq143}
      \end{flalign}
    \end{excercise}

    I když jsme již zdůraznili, že hbitému hledání a nalézání primitivních funkcí se lze naučit
    jedině praxí, tj. tak, že budeme „počítat, počítat, počítat“, přece jen si ještě ukážeme
    složitější příklady, v nichž se kombinují substituční metody s metodou per partes. Uvidíme však,
    že triky“, které při výpočtu nejvíce pomohou, nespočívají často v těchto metodách samotných, ale
    jsou spíše založeny na představivosti počtáře při úpravě obyčejných středoškolských vzorců.

    % !TeX spellcheck = cs_CZ
%====================== Sbírka řešených příkladů ==================================================
% \int{xe^x\dd{x}}, \quad x\in\realset,
\begin{mdframed}[style=mdmathsolution]
  [\ref{mai:eq140}]: Užijeme metodu per partes: \(u(x)=x \rightarrow u'(x)=1\) a \(v(x)=e^x
  \rightarrow v'(x)=e^x\). Tedy
  \begin{flalign*}
    &\int u'(x)v(x)\dd{x} = u(x)v(x) - \int u(x)v'(x)\dd{x}        &\\
    &\int{xe^xdx}         = xe^x-\int{e^x\dd{x}} = xe^x - e^x+ c   &
  \end{flalign*}
\end{mdframed}
    % !TeX spellcheck = cs_CZ
%====================== Sbírka řešených příkladů ==================================================
\begin{mdframed}[style=mdexam]
  \begin{example}\label{mai:exam123}
    \begin{equation*}
      \int\frac{x}{x^2+1}\dd{x}
    \end{equation*}
    Použijeme substituci
    \begin{equation*}
      \left[
        \begin{array}{c} 
          x^2 + 1 = t  \Rightarrow 2x\dd{x} = \dd{t}        \\ 
              x\dd{x} = \displaystyle{\frac{\dd{t}}{2}}
        \end{array} 
      \right]        
    \end{equation*}
    \begin{equation*}
      \frac{1}{2}\int\frac{dt}{t} = \frac{1}{2}\ln\abs{t} =\frac{1}{2}\ln\abs{1+x^2}+ c
    \end{equation*}
  \end{example}
\end{mdframed}
    % !TeX spellcheck = cs_CZ
%====================== Sbírka řešených příkladů ==================================================
% \int{\arctan x\dd{x}}\qquad x\in R
  [\ref{mai:eq142}]: Metodu per partes \(u(x) =\arctan x \rightarrow u'(x) =\frac{1}{x^2+1}\),
  \(v(x)= x \rightarrow v'(x) = 1\)      
  \begin{gather*}
    x\arctan x-\underbrace{\int\frac{x}{x^2+1}\dd{x}}_{I_1} =
    x\arctan x-\frac{1}{2}\ln\abs{1+x^2}+ c 
  \end{gather*}
  Integrál \(I_1\) jsme již řešili v příkladu \ref{mai:eq141}.
    % !TeX spellcheck = cs_CZ
%====================== Sbírka řešených příkladů ==================================================
\begin{mdframed}[style=mdexam]
  \begin{example}\label{mai:exam016}
    \begin{equation}\label{mai:int_ex_02}
      \int{\frac{2x^4-5x^2+14x+13}{x^2-x-2}\dd{x}} \qquad x\in R - \{1,2\}
    \end{equation}
    Dělením čitatele integrandu jmenovatelem dostaneme rozklad integrandu na součet funkcí, jejich 
    integrály najdeme snadno:
    \begin{equation*}
      \polylongdiv[style=C,div=:]{2x^4-5x^2+14x+13}{x^2-x-2}
    \end{equation*}

    Zbytek po dělení představuje integrál, jež je počítán v příkladu \ref{MA:eq_ex1} a proto ho 
    vynecháme. 
    \begin{align*}
       &= 2\int x^2\dd{x} + 2\int x\dd{x} + \int\dd{x} + \int\frac{19x+15}{x^2-x-2}\dd{x}     \\
       &= \frac{2}{3}x^3 + x^2 + x + \frac{4}{3}\ln\abs{x+1} - \frac{53}{3}\ln\abs{x-2} + C 
    \end{align*}
  \end{example}
\end{mdframed}
    \newpage
    % !TeX spellcheck = cs_CZ
%====================== Sbírka řešených příkladů ==================================================
\begin{mdframed}[style=mdexam]
  \begin{example}\label{mai:exam114}
    \begin{equation}\label{mai:exam016_003}
      \boxed{\int\sqrt{x^2+a}\dd{x}}
    \end{equation}  
    Použijeme metodu per partes
    \begin{equation*}
      \left[
        \begin{array}{cc} 
           u =\sqrt{x^2+a}              & dv = 1 \\ 
          du =\displaystyle
                \frac{x}{\sqrt{x^2+a}}  &  v = x
        \end{array}
      \right]   
    \end{equation*}
    Dostáváme
    \begin{equation*}
      \int{\sqrt{x^2+a}\dd{x}} = x\sqrt{x^2+a}-\int{\frac{x^2}{\sqrt{x^2+a}}\dd{x}}
    \end{equation*}
    Integrand rozšíříme, abychom dostali zadaný integrál 
    \begin{equation*}
        \int\frac{x^2+a-a}{\sqrt{x^2+a}}\dd{x} 
          = \int{\sqrt{x^2+a}\dd{x}} - \int{\frac{a}{\sqrt{x^2+a}}\dd{x}}                   
    \end{equation*}
    \begin{align*}
      2\!\int{\sqrt{x^2+a}\dd{x}} 
        = x\sqrt{x^2+a}+a\underbrace{\int{\frac{1}{\sqrt{x^2+a}}}\dd{x}}_{J_1}
    \end{align*}    
    Integrál \(J_1\) na pravé straně vyjádříme podle příkladu \ref{ma:ex_sub_metoda1} a výsledek
    do\-sta\-ne\-me ve tvaru
    \begin{gather*}
      \sqrt{x^2+a}\dd{x}
          =\frac{1}{2}\left[x\sqrt{x^2+a}+a\ln{\abs{x + \sqrt{x^2+a}}}\right]
    \end{gather*}
  \end{example}
\end{mdframed}
    %-------------------------------------
    \newpage
    \begin{excercise}\label{mai:cviko002}
      Z toho, co jsme uvedli, vyplývá, jak postupujeme při rozkladu racionální lomené funkce na
      parciální zlomky:
      \begin{enumerate}[label=\roman*]
        \item Není-li funkce ryze lomená, převedeme ji dělením čitatele jmenovatelem na součet
              polynomu a ryze lomené racionální funkce. Dále upravujeme ryze lomenou racionální
              funkci.
        \item Rozložíme jmenovatele v reálném oboru.
        \item Napíšeme rozklad na parciální zlomky podle věty o rozkladu.
        \item Najdeme koeficienty parciálních zlomků metodou neurčitých součinitelů.
      \end{enumerate}
      Příklady integrace racionálních funkcí. Čerpáno z \cite[p.~271]{Brabec1989}
      \begin{flalign}
        &\int\dfrac{x+2}{x^3-x}\dd{x},                                          &\label{mai:eq152}\\
        &\int\dfrac{2x^3 + x + 2}{x^4 + x^3 + x^2}\dd{x}                        &\label{mai:eq153}\\
        &\int\dfrac{x^2 + x}{x^2 + x}\dd{x},                                    &\label{mai:eq154}\\
        &\int\dfrac{2x - 1}{x^2 - 5x - 6}\dd{x}                                 &\label{mai:eq155}\\
        &\int\sqrt{x^2+a}\dd{x}, \quad a\neq0, x^2+a>0,                         &\label{mai:eq156}
      \end{flalign}
    \end{excercise}
    
%---------------------------------------------------------------------------------------------------