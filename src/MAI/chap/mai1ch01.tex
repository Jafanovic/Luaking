% !TeX spellcheck = cs_CZ
%{\tikzset{external/prefix={tikz/MAI/}}
% \tikzset{external/figure name/.add={ch10_}{}}
%---------------------------------------------------------------------------------------------------
% file mai1ch01.tex
%---------------------------------------------------------------------------------------------------
\chapter{Logika a teorie množin}\label{mai:IchapI}
\minitoc
  \section{Matematická logika}\label{mai:IchapIsecI}
    Vyjadřovacími prostředky matematiky, s nimiž se setkáváme v učebnicích i dalších matematických 
    textech, jsou především běžný spisovný jazyk, speciální jazyk (terminologie a symbolika) logiky 
    a matematiky, grafy, diagramy, schémata a tabulky \cite[s.~13]{polak1991matematika}.
    
    \subsection{Symboly, konstanty a proměnné}
      Velmi významné je pro matematiku užití symbolů (znaků), jež umožňuje stručné vyjadřování 
      matematických poznatků ve formě symbolických zápisů, vzorců apod. Tak např. uvažovaným 
      objektům (prvkům množin) se přiřazuje kromě názvu (jména) objektu také symbol objektu, který 
      ho zastupuje ve stručném vyjádření (v symbolických zápisech). Symbol objektu může být dvojího 
      druhu:
      \begin{enumerate}[label=\alph*)]
        \item \textbf{Konstanta} je symbol, který označuje určitý (jediný) objekt. Z dané množiny 
              objektů. Např. 4 (označuje číslo čtyři), \(\pi\) (označuje \emph{Ludolfovo číslo}). 
        \item \textbf{Proměnná} je symbol, který označuje kterýkoli objekt z dané množiny 
              objektů. Zpravidla je to písmeno \(x\), \(y\) apod.
      \end{enumerate}
      
      Množina konstant, které zastupuje proměnná, se nazývá \textbf{obor proměnné}. Prvky oboru 
      proměnné se nazývají \emph{hodnoty proměnné}. Nahrazení proměnné konstantou z oboru 
      proměnné v nějakém výraze nazýváme \emph{dosazení konstanty za proměnnou} do daného výrazu.
    
    \subsection{Výroky}
      Ze složitějších jazykových výrazů mají základní význam v logice a v matematice výroky, což 
      jsou takové jazykové výrazy (sdělení), o nichž má po obsahové stránce smysl tvrdit, že jsou 
      buď pravdivé, anebo nepravdivé (nastává právě jedna z těchto možností). Je-li \textbf{výrok 
      pravdivý}, říkáme též, že výrok platí. Je-li \textbf{výrok nepravdivý}, říká se také, že 
      výrok neplatí.
      
      Výroky, o nichž jsme (v daném okamžiku) dosud neurčili jednoznačně zda jsou pravdivé, anebo 
      nepravdivé, avšak principiálně jedna z těchto možností musí nastat, se nazývají 
      \textbf{hypotézy (domněnky)}. Gramaticky má výrok vždy formu oznamovací věty vyjádřené slovně 
      anebo symbolicky (pomocí matematických, logických, event. dalších symbolů.
      
      K označení výroků se obvykle používají malá písmena \(p\), \(q\) aj. (V literatuře se lze 
      setkat též s označením velkými písmeny.) 
      
      Výrokům se přiřazují \textbf{pravdivostní hodnoty} takto: pravdivostní hodnota \num{1} 
      (pravda), je-li výrok  pravdivý, pravdivostní hodnota \num{0} (nepravda), je-li výrok 
      nepravdivy.
      
      Příklady výroků:
      \begin{enumerate}[label=\alph*)]
        \item \textbf{Pravdivé výroky}: Bedřich Smetana je autorem hudby k opeře Prodaná 
              nevěsta. Číslo \num{10} je sudé. 
        \item \textbf{Nepravdivé výroky}: Král Karel IV. nebyl korunován císařem. Každé prvočíslo 
              je liché. 
        \item \textbf{Hypotézy}: Rakovina není virového původu. Každé sudé číslo větší než dvě lze 
              rozložit na součet dvou prvočísel.
      \end{enumerate}
      
      
  \section{Množiny}\label{mai:IchapIsecII}

%} %tikzset
%---------------------------------------------------------------------------------------------------
\printbibliography[heading=subbibliography]
\addcontentsline{toc}{section}{Seznam literatury}