% !TeX spellcheck = cs_CZ
%{\tikzset{external/prefix={tikz/MAI/}}
% \tikzset{external/figure name/.add={ch10_}{}}
%---------------------------------------------------------------------------------------------------
% file mai1ch01.tex
%---------------------------------------------------------------------------------------------------
\setchaptertoc
\chapter{Logika a teorie množin}\label{mai:IchapI}
  \MakeUppercase{Matematika} pochází z řeckého slova \emph{Mathema}, což znamená vědění a poznání.
  Matematika nejsou počty - ty jsou jen jedním z nástrojů, které navíc může za nás vykonat počítač.
  Je prostředkem k popisu a formalizaci jevů v okolním světě, umožňuje odhadnout důsledky těchto
  jevů a najít souvislosti mezi nimi.


  \section{Matematická logika}\label{mai:IchapIsecI}
    Vyjadřovacími prostředky matematiky, s nimiž se setkáváme v učebnicích i dalších matematických 
    textech, jsou především běžný spisovný jazyk, speciální jazyk (terminologie a symbolika) logiky 
    a matematiky, grafy, diagramy, schémata a tabulky \cite[s.~13]{polak1991matematika}.
    
    \subsection{Symboly, konstanty a proměnné}
      Velmi významné je pro matematiku užití symbolů (znaků), jež umožňuje stručné vyjadřování 
      matematických poznatků ve formě symbolických zápisů, vzorců apod. Tak např. uvažovaným 
      objektům (prvkům množin) se přiřazuje kromě názvu (jména) objektu také symbol objektu, který 
      ho zastupuje ve stručném vyjádření (v symbolických zápisech). Symbol objektu může být dvojího 
      druhu:
      \begin{enumerate}[label=\alph*), noitemsep]
        \item \textbf{Konstanta} je symbol, který označuje určitý (jediný) objekt. Z dané množiny 
              objektů. Např. 4 (označuje číslo čtyři), \(\pi\) (označuje \emph{Ludolfovo číslo}). 
        \item \textbf{Proměnná} je symbol, který označuje kterýkoli objekt z dané množiny 
              objektů. Zpravidla je to písmeno \(x\), \(y\) apod.
      \end{enumerate}
      
      Množina konstant, které zastupuje proměnná, se nazývá \textbf{obor proměnné}. Prvky oboru 
      proměnné se nazývají \emph{hodnoty proměnné}. Nahrazení proměnné konstantou z oboru 
      proměnné v nějakém výraze nazýváme \emph{dosazení konstanty za proměnnou} do daného výrazu.
    
    \subsection{Výroky a jejich pravdivostní hodnoty}
      Ze složitějších jazykových výrazů mají základní význam v logice a v matematice výroky, což 
      jsou takové jazykové výrazy (sdělení), o nichž má po obsahové stránce smysl tvrdit, že jsou 
      buď pravdivé, anebo nepravdivé (nastává právě jedna z těchto možností). Je-li \textbf{výrok 
      pravdivý}, říkáme též, že výrok platí. Je-li \textbf{výrok nepravdivý}, říká se také, že 
      výrok neplatí.
      
      Výroky, o nichž jsme (v daném okamžiku) dosud neurčili jednoznačně zda jsou pravdivé, anebo 
      nepravdivé, avšak principiálně jedna z těchto možností musí nastat, se nazývají 
      \textbf{hypotézy (domněnky)}. Gramaticky má výrok vždy formu oznamovací věty vyjádřené slovně 
      anebo symbolicky (pomocí matematických, logických, event. dalších symbolů.
      
      K označení výroků se obvykle používají malá písmena \(p\), \(q\) aj. (V literatuře se lze 
      setkat též s označením velkými písmeny.) 
      
      Výrokům se přiřazují \textbf{pravdivostní hodnoty} takto: pravdivostní hodnota \num{1} 
      (pravda), je-li výrok  pravdivý, pravdivostní hodnota \num{0} (nepravda), je-li výrok 
      nepravdivy.
      
      Příklady výroků:
      \begin{enumerate}[label=\alph*), noitemsep]
        \item \emph{Pravdivé výroky:} Bedřich Smetana je autorem hudby k opeře Prodaná 
              nevěsta. Číslo \num{10} je sudé. 
        \item \emph{Nepravdivé výroky:} Král Karel IV. nebyl korunován císařem. Každé prvočíslo 
              je liché. 
        \item \emph{Hypotézy:} Nekteré druhy rakoviny jsou vírového puvodu. Každé sudé číslo větší
              než dvě lze rozložit na součet dvou prvočísel.
      \end{enumerate}

      Mezi výroky se zařazují i taková sdělení, o jejichž pravdivosti, resp. nepravdivosti nemůžeme
      v současnosti rozhodnout, ale principiálně právě jedna z těchto možností musí nastat. Jde
      zejména o výroky, jejichž pravdivost, resp. nepravdivost je časově nebo místně podmíněná, a
      výroky vztahující se k budoucnosti:
      \begin{enumerate}[label=\alph*), noitemsep]
        \item \emph{Výroky s časove nebo místne podmínenou pravdivostí:} Prší. Jsme ve škole.
              Vyšetrujeme vlastnosti daných posloupností. 
        \item \emph{Výroky vztahující se k budoucnosti:} Zítra bude pršet. Půjdeme do kina. Daným
              bodem \(Q\) sestrojíme přímku \(q\) kolmou k dané přímce \(p\).
      \end{enumerate}

      Příklady jazykových výrazů, jež nejsou výroky:
      \begin{enumerate}[label=\alph*), noitemsep]
        \item \emph{Názvy:} stredoškolská matematika. Praha.
        \item \emph{Výrazy typu:} 2+3. 
        \item \emph{Rozkazovací a tázací věty}. 
        \item \emph{Výrazy obsahující proměnné:} Číslo \(x\) je sudé
      \end{enumerate}

      \begin{mdframed}[style=mdnote]
        \begin{note}
          Výrokům se přiřazují pravdivostní hodnoty výroku takto: Je-li výrok pravdivý, přiřazuje se
          mu pravdivostní hodnota pravda označovaná symbolem (číslicí) \num{1}\protect\footnotemark[2].
          Je-li výrok nepravdivý, přiřazuje se mu pravdivostní hodnota nepravda označovaná symbolem
          (číslicí) \num{0}. O daném výroku se pak stručně říká, že má pravdivostní hodnotu
          \num{1}\protect\footnotemark[1], resp. \num{0}. 
        \end{note}
        \footnotetext[1]{v literatuře se lze setkat též s jejím označením \(P\)}
        \footnotetext[2]{v literatuře se lze setkat též s jejím označením \(N\)}
      \end{mdframed}
    
    \subsection{Složené výroky}  
      
      
  \section{Množiny}\label{mai:IchapIsecII}

%} %tikzset
%---------------------------------------------------------------------------------------------------