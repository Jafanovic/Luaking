% !TeX spellcheck = cs_CZ
%     An overview of high school mathematics
% HTML
% Dějiny matematiky I (NMUM305, ZS, 2/0, Z, 2 kredity)
% https://www2.karlin.mff.cuni.cz/~halas/Historie_MFF/Historie.htm

% Fuchs, E. Přehled vývoje matematiky. Praha, 1993.
% https://dml.cz/bitstream/handle/10338.dmlcz/400583/DejinyMat_01-1994-1_2.pdf

% Bečvář, J., Bečvářová, M., Vymazalová, H. Matematika ve starověku. Egypt a Mezopotámie. Praha,
% 2003.
% https://dml.cz/handle/10338.dmlcz/401848

% Vymazalová, H. Staroegyptská matematika. Hieratické matematické texty. Praha, 2006.

% Bečvář, J. Hrdinský věk řecké matematiky I, II. Praha, 1993, 1995. pdf: I. díl a II. díl
% https://dml.cz/bitstream/handle/10338.dmlcz/400590/DejinyMat_01-1994-1_3.pdf
% https://dml.cz/bitstream/handle/10338.dmlcz/401035/DejinyMat_07-1997-1_4.pdf

% Bečvář, J., Štoll, I. Archimédés. Největší vědec starověku. Prometheus, Praha, 2005. (info)
% Halas, Z. (ed.) Archimédés. Několik pohledů do jeho života a díla. Praha, 2012. pdf
% https://dml.cz/handle/10338.dmlcz/402371

% Halas, Z. Výpočty hodnot goniometrických funkcí. Praha, 2010.
% https://www2.karlin.mff.cuni.cz/~halas/Historie_MFF/Historie.htm

% Šír, Z. Řecké matematické texty. OIKOYMENH, Praha, 2011. (info)
% https://dml.cz/handle/10338.dmlcz/400831

% Hudeček, J. Matematika v devíti kapitolách. Praha, 2008.

% Sýkorová, I. Matematika ve staré Indii. Praha, 2016.
% https://dml.cz/handle/10338.dmlcz/404297

% Bečvář, J. (ed.) Matematika ve středověké Evropě. Praha, 2001.
% https://dml.cz/handle/10338.dmlcz/401778

% Mačák, K. Tři středověké sbírky matematických úloh. Praha, 2001.
% https://dml.cz/handle/10338.dmlcz/401217

% dějinách elementární matematiky (Fr. Balada)
% https://www2.karlin.mff.cuni.cz/~halas/Historie_MFF/Balada.pdf

% Eukleidovy Základy: překlad Fr. Servít, 1907
% https://www2.karlin.mff.cuni.cz/~halas/Historie_MFF/Eukleides.pdf

% stručný přehled dějin M a Hrdinský věk řecké M I a II 
% https://www2.karlin.mff.cuni.cz/~halas/Historie_MFF/Dejiny1_prehled_Recko.pdf

%---------------------------------------------------------------------------------------------------
% history_MA.tex
%---------------------------------------------------------------------------------------------------
\setchaptertoc
\chapter{Průvodce historií matematiky}\label{mai:OchapI}
  \section{Výpočty obsahů a objemů ve starověké matematice}\label{mai:OchapIsecI}
    \subsection{Egypt a Mezopotámie}\label{mai:OchapIsecIssecI}
    \subsection{Řecko}\label{mai:OchapIsecIssecII}
  \section{Vznik a vývoj infinitezimálního počtu (16.-18. století)}\label{mai:OchapIsecII}
    \subsection{Období přechodu od starověku k renesanci}
      Roku 476 n. l. se rozpadla římská říše. Na jejím území vznikly nové feudální státy, jejichž
      hospodářství však bylo na velmi nízké úrovni. Především západní část bývalé římské říše
      (Evropa) byla velmi zaostalá. Její zemědělství bylo extenzívní, neopíralo se o důmyslné
      systémy Východu (zavodňování, ...) ani o jejich zkušenosti a znalosti. Západ vystačil s
      minimem astronomie a trochou praktické aritmetiky pro obchod a zeměměřičství.

      Úpadku kultury v mnohém napomáhalo ulpívání na nevědeckých dogmatech křesťanské církve, která
      se zpočátku snažila úplně vymýtit „pohanskouÿ řeckou a římskou kulturu. I přesto zůstaly po
      mnoho staletí křesťanské kláštery jedinými místy, kde se alespoň částečně udržovala
      vzdělanost. Úroveň klášterní matematiky však byla velmi nízká a sloužila hlavně k sestavování
      kalendáře a výpočtu dat církevních svátků. Téměř tisíc let zůstávaly největší autoritou
      matematické spisy, které sepsal filozof A. M. T. S. Boetius (asi 480 - 524), a to i přesto, že
      jsou z vědeckého hlediska poměrně chudé.

      Situace se začala měnit až v 11. a 12. století, kdy došlo k všeobecnému rozvoji výroby,
      obchodu, k rozkvětu měst a posílení úlohy měšťanů, což vytvářelo základ pro celkový rozvoj
      kultury a vědy. Obnovení obchodního styku s Východem vedlo k dovozu a rozšíření řecké
      literatury. Současně s rozšiřováním obchodu se začaly navazovat i vědecké styky s arabskou
      kulturou a to především přes Španělsko a Sicílii. A tak se pomalu Evropanům začalo otevírat
      bohatství arabské a antické vědy a kultury.
      
      Ve 12. a 13. století bylo přeloženo z arabštiny do latiny obrovské množství vědeckých prací.
      Vznikla tak skutečně rozsáhlá vědecká a filozofická literatura psaná latinsky. Překládána byla
      jak původní arabská díla, tak řecká literatura existující v arabštině. Přeložená díla
      Eukleida, Archiméda, Apollónia, Ptolemaia a dalších se brzy stala základem pro rozvoj evropské
      matematiky. Důležitou úlohu v rozvoji matematiky sehrály univerzity. Nejstarší evropská
      univerzita byla založena v Salernu v první polovině 11. století. Po ní následovaly univerzity
      v Bologni (1119), Paříži (1160), Oxfordu (1167), Cambridge (1209) a další.

      Ve 12. a 13. století zaujala v Evropě v rozvoji řemesel, obchodu a peněžnictví první místo
      italská města Janov, Pisa, Benátky, Milán a Florencie. Obchodníci z těchto měst podnikali,
      podobně jako Marco Polo, daleké cesty, při nichž se snažili poznat umění a vědu jiných národů.
      Jedním z obchodníků, který se zapsal do dějin matematiky, byl Leonardo Pisánský, zvaný
      Fibonacci (asi 1170 - 1250). Aritmetické a algebraické znalosti, které nashromáždil na svých
      cestách, shrnul do knihy Liber Abaci (Kniha o abaku), geometrické znalosti do knihy Practica
      Geometriae.

      S rozšiřováním obchodu se šířila i do severnějších měst potřeba matematiky. Zájem o ni byl
      především praktický, a proto po několik staletí učili aritmetiku a algebru mimo tehdy
      vznikající univerzity řemeslní počtáři. Teoretickou matematiku pěstovali pouze scholastičtí
      filozofové a matematici. Ti spekulovali např. o podstatě pohybu, o kontinuu a o nekonečnosti
      (např. Tomáš Akvinský) a jejich úvahy ovlivnily v 17. století objevitele infinitezimálního
      počtu a v 19. století filozofy zabývající se nekonečnem.

      V druhé polovině 15. století začíná období renesance. Hlavními středisky kultury a vědy nadále
      zůstávají italská města. V této době dochází hlavně k rozvoji trigonometrie a algebry.
      Rozšíření matematiky (a obecněji vědy i kultury) velmi ovlivnil vynález knihtisku, který
      zpřístupnil literaturu širší vrstvě obyvatelstva. Také bouřlivý rozvoj architektury a rozkvět
      výtvarného umění pomohl rozvoji a šíření matematiky. Jedním z malířů, jenž byl zároveň
      matematikem, byl Leonardo da Vinci (1452 - 1519). Zachovaly se nám jeho poznámkové sešity,
      které obsahují matematické a filozofické úvahy. Je například pozoruhodné, že při zkoumání
      těžišť obrazců a těles a také při určování obsahu elipsy Leonardo používal Archimédovu metodu,
      kterou matematikové při řešení podobných úloh začali užívat až v 17. století.

      Počátkem 16. století evropská matematika překračuje rámec znalostí, které tvořilo dědictví
      antického Řecka a orientálních národů. Postupně převládá poziční desetinná aritmetika, rozvíjí
      se trigonometrie, v algebře se krok za krokem vytváří vyhovující symbolika, začínají se
      studovat Apolloniovy kuželosečky a Archimédovy kvadratury a kubatury. Dlouhé období matematiky
      konstantních veličin se blížilo ke svému konci; začínalo období matematiky proměnných veličin,
      symbolické algebry, analytické geometrie a diferenciálního a integrálního počtu.

      16. století bylo renesancí kultury a vědy, tedy i matematiky. Studium a porozumění základním
      dílům starého Řecka, zejména Archiméda, brzy dosáhlo takové úrovně, že byl možný rozvoj
      nových, jednodušších metod výpočtů obsahů a objemů. Avšak na rozdíl od Archiméda, jehož důkazy
      byly naprosto přesné, renesanční matematikové se více zajímali o rychlý výsledek, než o přesný
      důkaz. Mezi těmi, kdo studovali Archiméda, vynikli Simon Stevin, který psal o těžištích, a
      Luca Valerio - autor prací o těžištích a kvadratuře paraboly. Bezprostředně po těchto prvních
      průkopnících vznikly významné práce Keplera, Cavalieriho a Torricelliho, v nichž byly
      rozvinuty metody, které vedly později k objevu infinitezimálního počtu.

      K novému rozvoji matematiky přispěla i revoluce v astronomii, kterou představují Mikuláš
      Koperník (1473 - 1543), Tycho Brahe (1546 - 1601) a Johann Kepler. Přístrojové vybavení se
      zlepšilo, byla k dispozici přesnější pozorování a ta bylo třeba interpretovat, tj. mimo jiné i
      matematicky zpracovat.

      Tehdejší horlivá aktivita matematiků vedla ke vzniku diskusních kroužků a k jejich rozsáhlé a
      soustavné korespondenci. Z těchto skupin učenců postupně vyrůstaly akademie. Vznikaly v
      určitém ohledu jako opozice k univerzitám, které se většinou vyvinuly již v období scholastiky
      a byly z hlediska schopnosti řešit aktuální problémy příliš strnulé. Nové akademie, jichž bylo
      v tomto období vytvořeno přes pět set, naopak ztělesňovaly nového ducha bádání.
    
    \subsection{Kepler a Cavalieri a jejich výpočty obsahů a objemů}    
      V díle Johanna Keplera (1571 - 1630) je zvlášť zřetelný podnětný vliv nové astronomie, která
      vyžadovala jak obsáhlé počtářské práce, tak i infinitezimální úvahy. Ve svém díle Nová
      stereometrie vinných sudů (1615) počítal objemy těles, které vznikly rotací částí kuželoseček
      kolem osy ležící v jejich rovině.
      
      Podle legendy byla roku 1612 v Linci v Horním Rakousku, kde tehdy Kepler pobýval, mimořádně
      dobrá úroda vína. Víno se prodávalo téměř všude. Na jednom takovém trhu byl Kepler velice
      udiven, když prodavač, aby zjistil objem sudu, prostě strčil do otvoru po zátce měřící proutek
      a ze vzdálenosti tohoto otvoru od protější stěny vypočítal objem sudu. Tři dny hloubal Kepler
      nad problémem objemu vinných sudů, které pojal jako rotační tělesa, a úlohu rozřešil. Doufal,
      že vynalezne tvar sudu, který by při stejném povrchu měl větší objem, než sudy skutečně
      užívané. Přesvědčil se však, že rakouské sudy mají téměř ideálně účelný tvar, což ho přivedlo
      k výroku: Kdo může popřít, že lidská přirozenost sama, beze všeho hloubavého rozumového
      uvažování, učí základním pravdám geometrie?

      Kepler při svých výpočtech postupoval metodou rozdělení tělesa na nekonečně mnoho nekonečně
      malých „kusů, jejichž objem lze jednoduše výpočtem určit. Použil tedy úvahu, které se říká
      infinitezimální. Např. při určování objemu koule při známém povrchu rozdělil kouli na
      nekonečně mnoho jehlanů s vrcholy ve středu koule a základnou na povrchu koule a výškou rovnou
      poloměru koule. Sečetl objemy těchto jehlanů a dostal \(V = \sfrac{1}{3}Ar\), kde \(A =
      4πr^2\) je povrch koule. Odtud získal objem koule \(V = \sfrac{4}{3}πr^3\). 
      
      
      Ještě známější je jeho určování obsahu kruhu. Každou z (nekonečně malých) částí ohraničující
      kružnice považuje za základnu rovnoramenného trojúhelníka s vrcholem ve středu kruhu. Obsah
      kruhu je pak roven součtu obsahů všech takových trojúhelníků. Představme si, že kružnice se
      středem \(S\) je rozvinuta do úsečky \(AC\) (její délka je rovna délce obvodu \(O\) kruhu)
      tak, že poloměr \(SA\) je k ní kolmý. Nekonečně malému \(XY\) na kružnici odpovídá dílek
      \(X′Y′\) na úsečce \(AC\). Trojúhelníky \(XYS\), \(X′Y′S′\) mají výšku i základnu stejné
      délky, a tedy mají stejný obsah (Kepler zde považuje délku oblouku \(XY\) a délku jemu
      odpovídající úsečky \(X′Y′\) za stejné).

      \begin{figure}[ht!] 
        \centering
        \luafigure[1]{example-image-b}
        \caption{Keplerův výpočet obsahu kruhu}
        \label{mai:fig104}
      \end{figure}

      Tyto trojúhelníky lze zaměnit jinými, se stejnými základnami a výškou, přičemž vrcholy všech
      trojúhelníků se posunou do středu kružnice \(S\). Takto vzniklé trojúhelníky mají stejné
      obsahy jako původní trojúhelníky a dohromady vyplňují trojúhelník \(ACS\).

      \begin{figure}[ht!]
        \centering
        \luafigure[1]{example-image-b}
        \caption{Keplerův výpočet obsahu kruhu}
        \label{mai:fig105}
      \end{figure}

      Obsah kruhu je tedy roven obsahu pravoúhlého trojúhelníka s odvěsnami \(AC\) a \(AS\), kde
      velikost strany \(AC\) je rovna velikosti obvodu \(O\) kruhu. Odtud plyne \[S = \dfrac{1}{2}rO
      = \dfrac{1}{2}r\cdot2πr = πr^2\].

      Kepler podobných úvah použil k výpočtům objemů velkého množství těles používaných v praxi. Z
      hlediska důkazových metod se Kepler rozešel s archimédovským požadavkem přesnosti. Prohlásil,
      že Archimédovy důkazy jsou absolutně přesné, že je však přenechává lidem, kteří si chtějí
      dopřát přesné důkazy.

      Za nepřesnosti tohoto typu bylo Keplerovo dílo ve své době velmi kritizováno. Dnes vidíme, že
      však znamenalo velký krok ke vzniku moderních integračních metod. Kepler pro řešení praktické
      úlohy vedl správné úvahy nového typu, chyběla mu však jejich odpovídající matematická
      formalizace a proto i rigorózní důkazy.

      \begin{figure}[ht!]
        \centering
        \luafigure[1]{example-image-b}
        \caption{Galileo Galilei (15. února 1564, Pisa - 8. ledna 1642, Arcetri)}
        \label{mai:fig100}
      \end{figure}
      
      Bonaventura Cavalieri (1598 - 1647) ve svém díle Geometria indivisibilibus continuorum (1635)
      vyložil jednoduchou formou metodu výpočtu objemu tělesa. Své výsledky shrnul ve formulaci,
      které dnes říkáme \emph{Cavalieriho princip}: \uv{Když dvě tělesa mají stejnou výšku a když
      řezy rovinami, které jsou rovnoběžné s jejich podstavami a mají od nich stejnou vzdálenost,
      jsou takové, že poměr jejich obsahů je vždy stejný, potom objemy těles mají týž poměr.}

      \begin{figure}[ht!]
        \centering
        \luafigure[1]{example-image-b}
        \caption{Bonaventura Francesco Cavalieri}
        \label{mai:fig101}
      \end{figure}

      Když budeme pomocí Cavalieriho principu určovat objem kruhového kužele s poloměrem podstavy
      \(r\) a s výškou \(h\), můžeme jej porovnat s jehlanem o výšce \(h\) se čtvercovou podstavou,
      jejíž strana má délku \(1\) (viz obr. \ref{mai:fig103}). Roviny, které jsou rovnoběžné s
      podstavami obou těles a jsou vedeny ve stejné vzdálenosti od podstav, protínají tato tělesa v
      kruhu, resp. ve čtverci, jejichž obsahy jsou v konstantním poměru \(πr^2 : 1\). Podle
      Cavalieriho principu tedy platí \(\frac{V_k}{V_j} = πr^2\), tedy \(V_k = πr^2V_j\), kde
      \(V_k\) je objem kužele a \(V_j\) objem jehlanu, pro nějž platí \(V_j = \sfrac{1}{3}h\). Odtud
      plyne, že objem kužele je roven \(V_k = \sfrac{1}{3}πr^2h\).

      \begin{figure}[ht!]
        \centering
        \luafigure[1]{example-image-b}
        \caption{Cavalieriho princip}
        \label{mai:fig103}
      \end{figure}

      Cavalieriho metoda se liší od Keplerových postupů ve dvou aspektech. Za prvé, Kepler rozkládal
      těleso dané dimenze na nekonečně mnoho částí téže dimenze, kdežto Cavalieriho vrstvičky mají
      nižší dimenzi, než vyšetřovaný útvar. Za druhé, Kepler rozkládal dané těleso na
      infinitezimální části a sečtením jejich obsahů (resp. objemů) obdržel obsah (resp. objem)
      daného tělesa. Cavalieri potřeboval k výpočtu dvě tělesa a použil metodu porovnávání nekonečně
      malých částí těles, jakýchsi nedělitelných vrstviček.

      Praktický efekt Cavalieriho principu při výpočtu obsahů (resp. objemů) spočívá v tom, že
      odvozuje správné formule, aniž je nucen použít postupu, který dnes nazýváme výpočtem limity. I
      přes některé nedostatky měla Cavalieriho metoda velký vliv na jeho současníky i matematiky
      pozdějšího období. Leibniz například napsal, že Galilei a Cavalieri byli prvními, kdo začali
      odhalovat drahocenné metody a postupy Archiméda. Torricelli prohlásil, že Cavalieriho metoda
      je udivující ve své ekonomičnosti pro nalezení vět a dává možnosti dokázat ohromné množství
      nových tvrzení a to krátkými, přímými důkazy, což je nemožné metodami starověku.

      Kromě této metody pro výpočet objemů dvou těles porovnáváním jejich řezů, Cavalieri objevil i
      metodu pro výpočet obsahů a objemů jednoduchých útvarů pomocí tzv. příčných řezů. Ilustrujme
      tuto metodu na třech příkladech.

%} % tikzset
%---------------------------------------------------------------------------------------------------              