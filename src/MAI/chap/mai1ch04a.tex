% !TeX program = lualatex
% !TeX root = luaking.tex
% !TeX encoding = UTF-8
% !TeX spellcheck = cs_CZ
%---------------------------------------------------------------------------------------------------
\graphicspath{{../src/MAI/img/}}
% file mai1ch04a.tex
%---------------------------------------------------------------------------------------------------
\setchaptertoc
\chapter{Determinismus a náhodnost, míra jistoty}\label{mai:IchapIVa}
  \epigraph{\emph{The true logic of this world is in the calculus of probabilities.}}{James Clerk 
    Maxwell}
  \section{Determinismus a náhodnost, míra jistoty}
    Okolo nás existuje spousta věcí, jevů a událostí, které nelze předvídat - jsou důsledkem náhody.
    Patří mezi ně např. zakoupení losu, který vyhrává, doba čekání na tramvaj, zlomení nohy na
    náledí, výsledky parlamentních voleb, obsažená či neobsazená telefonní linka, zasažení stromu
    bleskem, počet poruch jaderné elektrárny a doba mezi nimi, atd. Běžné matematické prostředky
    nelze k popisu náhodných dějů použít. Např. diferenciální rovnice popisují, jaké vztahy platí za
    daných podmínek mezi jednotlivými veličinami. Jestliže dodržíme všechny předpoklady, výsledek
    bude vždy stejný a přesně definovaný.
    
    Jinak je tomu u náhodných dějů. I když si koupím los vždy u stejného prodavače, zavřu oči a
    vyberu ho levou rukou, mohu někdy vyhrát a jindy ne. Záleží-li výsledek děje na náhodě, nemůžeme
    ho nikdy přesně spočítat a popsat. V takovém případě nemůže žádná exaktní věda pomoci již ze
    samé podstaty náhody. Můžeme se ale snažit popsat, tj. kvantifikovat tuto náhodu.
    
    Nejjednodušším kvantifikacím náhody všichni alespoň částečně rozumíme. Jestliže řekneme, že
    pravděpodobnost šestky při hodu kostkou je \sfrac{1/6}, znamená to, že při velkém počtu hodů i
    padne přibližně jedna šestina šestek. Co je to ale „velký počet hodů “? Jestliže ze šedesáti
    hodů mine pouze jedna šestka, je to v rozporu s tímto tvrzením? Nejlépe si výpovědní hodnotu
    tohotu tvrzení uvědomíme v mezní situaci.

    %--Do nemocnice přichází pacient na operaci---------------------
    % !TeX spellcheck = cs_CZ
\begin{mdframed}[style=mdexam]
  \begin{example}\label{mai:exam102}
    Jako první příklad vám povím poněkud Černý vtip: \newline
    Do nemocnice přichází pacient na operaci. Na chodbě potká sestru a plný obav se jí ptá:
    Sestřičko, není to příliš nebezpečné? Slyšel jsem, že ta operace je úspěšná pouze u
    \qty{10}{\percent} pacientů.“ Sestra ho však uklidní: Vůbec se nebojte. Pan primář zatím operoval
    devět pacientů, a všichni umřeli.“ Přejato z \cite[s.~9]{Rogalewicz2007}.
  \end{example}
\end{mdframed}
    %---------------------------------------------------------------

    Otázkami náhody a náhodných dějů se zabývají dvě matematické disciplíny: teorie pravděpodobnosti
    a matematická statistika.

    \textbf{Teorie pravděpodobnosti} řeší následující problém: jistý jev má řadu různých možných
    následků. Náhoda určuje, který (jediný) z nich skutečně nastane. Jde-li někdo na zkoušku na
    vysoké škole, může dostat jedničku, dvojku, trojku nebo čtyřku. Mám-li jít zítra na zkoušku, lze
    předem říct něco o možnosti, že ji neudělám (dostanu 4)? Teorie pravděpodobnosti dává návod, jak
    mohu své šance na jednotlivé známky předem kvantifikovat.

    \textbf{Matematická Statistika} řeší v jistém smyslu opačnou situaci. Vidím nějaký jev, který
    může mít řadu příčin. Např. můj syn dostal ve škole pětku. To může být způsobeno tím, že se
    neučil, bolela ho hlava, učitele bolela hlava, učitel si něho zasedl atd. Matematická statistika
    přináší metody, jak (na základě zkušenosti) tyto možné příčiny třídit a kvantifikovat, např.
    kterou z nich vybrat jako nejvěrohodnější.

    Schematicky lze tyto disciplíny znázornit takto:
    \luagraphic[1]{mai_fig072.png}{Disciplíny
                     \cite[s.10]{Rogalewicz2007}}{mai:fig072}

    Teorie pravděpodobnosti a matematická statistika jsou založeny na známých matematických teoriích
    - kalkulu, lineární algebře, teorii míry. Nepřinášejí (až na výjimky) nové matematické postupy,
    ale dávají známým metodám nový výklad. Je zde tedy přirozeně věnována mnohem větší pozornost
    tomu, jak budou vstupní údaje i výsledky interpretovány než vlastnímu průběhu výpočtu.

    Jedná se vlastně o porozumění číslům a vztahům mezi nimi. Špatnou interpretaci (ať již nechtěnou
    nebo záměrnou) často dochází k posunutí smyslu nebo minimálně k závěrům, které budí úsměv. 

    Jedním ze základních úkolů teorie pravděpodobnosti je počítání s nepřesnými čísly. Průměr nebo
    relativní chyba jsou speciálními případy obecnějších statistických metod. Opět je třeba získat
    určitý cit, aby nedocházelo k situacím popsaným v následujících příkladech:

    %--Egyptským muzeem prochází výprava turistů--------------------
    % !TeX spellcheck = cs_CZ
\begin{mdframed}[style=mdexam]
  \begin{example}\label{mai:exam103}
    Egyptským muzeem prochází výprava turistů a průvodce vykládá:\newline
    „... a tato mumie po levé straně. je stará \num{3024} let.“ Malá holčička se ptá: „Jak se to
    zjistí tak přesně? “ „To nevím, ale když jsem tu před 24 roky začínal, řekli mi, že je stará
    tři tisíce let.“ Přejato z \cite[s.~10]{Rogalewicz2007}.
  \end{example}
\end{mdframed}
    %---------------------------------------------------------------

    Reichmann ve Své knize \emph{Use and Abuse of Statistics} (Používání a zneužívání statistiky)
    říká: Denní tisk vybírá a zdůrazňuje vždy to, co se zdá zvlášť pozoruhodné, na úkor tabulek a
    poznámek, které často současně neotiskuje. To není kritika tisku, jehož úkolem je na skutečnosti
    upozorňovat, nikoli však je do všech podrobností popisovat. Žádný čtenář novin nechce místo
    svého obvyklého čtení dlouhé statistické tabulky, ale nesmí podlehnout pokušení brát doslovně
    znázornění, která se stala povrchními. Mnohé zprávy jsou ovšem dokonce výslovně nesprávnými
    výklady, dovozují nesprávné závěry a tím neprávem kazí pověst statistiky.

    %--Tramvaj (dilema nerozhodného zamilovaného)-------------------
    % !TeX spellcheck = cs_CZ
\begin{mdframed}[style=mdexam]
  \begin{example}\label{mai:exam104}
    \textbf{Tramvaj (dilema nerozhodného zamilovaného}:\newline
    Stalo se v jednom velkém městě. Důležitým prvkem příběhu, který vám budu vyprávět, je schéma
    městské hromadné dopravy. Ve městě byla jediná tramvajová trať ve tvaru osmičky, a po ní stále
    stejným směrem jezdila tramvaj (obr. \ref{mai:fig073}).

    {\centering
    \captionsetup{type=figure}
    \luafigure[1]{mai_fig073.png}
    \captionof{figure}{Schéma tramvajové“ trati. \cite[s.~13]{Rogalewicz2007}
    \label{mai:fig073}}
    \par}

    Hlavní stanice je v místě, kde se trať k sobě v obou směrech přibližuje. Nástupní ostrůvek je
    zde mezi kolejemi a cestující může nastoupit do tramvaje jedoucí oběma směry. Ostatní stanice se
    nacházejí v určitém odstupu na celé tramvajové trati. Dvě z těchto stanic jsou znázorněny na
    grafu; mají totiž v našem příběhu svou úlohu.

    V příběhu vystupují tři postavy. Pan \(X\), jehož kancelář se nachází nedaleko hlavní stanice, a
    jeho dvě přítelkyně, slečna \(A\) a slečna \(B\). Ty žily na opačných koncích města, jak je
    znázorněno na grafu.

    Pan \(X\) řeší problém: obě dívky se mu líbí stejně a nemůže se rozhodnout, kterou z nich by měl
    požádat o ruku. Každý den se s jednou z nich schází, velice těžko však vybírá, za kterou z nich
    by měl jet právě dnes. Proto se rozhodl, že se spolehne na náhodu. Měl klouzavou pracovní dobu;
    když měl vše hotovo, odcházel z práce kdykoli během dne. Nasedl do první tramvaje, která
    přijela; pokud jela na východ, odjel za slečnou \(A\), pokud jela na západ, jel za slečnou
    \(B\). Za předpokladu, že se pravděpodobnost, že tramvaj pojede na východ, přibližně rovná
    pravděpodobnosti, že pojede na západ, musí být četnost setkání pana \(X\) se slečnou \(A\) i se
    slečnou \(B\) přibližně stejná.

    V praxi to však dopadlo úplně jinak. Jednoho krásného dne si slečna \(B\) panu \(X\)
    postěžovala, že se setkávají málo a že pan \(X\) tedy určitě má ještě nějakou jinou. Pan \(X\)
    znejistěl a rozhodl se, že bude počítat, kolikrát se sejde se slečnou \(XA\) a kolikrát se
    slečnou \(B\). S překvapením zjistil, že slečna \(B\) asi měla k podezření důvod, během měsíce
    navštívil jednadvacetkrát slečnu \(A\) a pouze devětkrát slečnu \(B\).

    {\centering
    \captionsetup{type=figure}
     \subcaptionbox{Východní smyčka kratší než západní\label{mai:fig074a}}
      {\luafigure[1]{mai_fig074a.png}}                                                    \\
     \subcaptionbox{Západní část má více stanic než východní\label{mai:fig074b}}
      {\luafigure[1]{mai_fig074b.png}}
     \captionof{figure}{Možné důvody neplatnosti nulové hypotézy \cite[s.~14]{Rogalewicz2007}
    \label{mai:fig074}}
    \par}

    Náhoda nebo znamení osudu? Jaká je příčina této nespravedlnosti ke Slečně \(B\)? Nejdříve si
    všimněme, na základě čeho se pan \(X\) rozhodl svěřit volbu směru své cesty náhodě. Za
    dostatečně dlouhý časový interval se musí počty setkání se slečnou \(A\) a slečnou \(B\) rovnat,
    protože pravděpodobnost nastoupení do tramvaje jedoucí oběma směry je stejná. Lze tedy říct, že
    neexistuje rozdíl v počtu setkání se slečnou \(A\) a \(B\). To se ve statistice \textbf{nazývá
    nulovou hypotézou}. V našem konkrétním případě bude nulová hypotéza odpovídat předpokladu, že se
    za měsíc sejde patnáctkrát se slečnou \(A\) a patnáctkrát se slečnou \(B\).

    Přesně patnáctkrát? Nejspíš ne. Intuice nám říká, že ve skutečnosti budou setkání s oběma
    dívkami v poměru blízkém ideálnímu 50 : 50. Co pak lze říct o našem poměru 21 : 9? Většina
    čtenářů bude asi mít sklon zamítnout názor, že by podobná nerovnováha mohla nastat za podmínek,
    kdy teoreticky platí 15 : 15. Řečeno odborně, čtenáři \textbf{zamítnou nulovou hypotézu} a
    předpokládají, že se předpoklady, na nichž byla založena, ukázaly nereálnými. 

    Abychom si vše představili ještě jasněji, předpokládejme, že poměr návštěv byl 17 : 13. Co lze
    říct o takovém případu? A jak by to bylo s poměry 18 : 12, nebo 19 : 11, nebo 20 : 10? Je
    Zřejmě, že tady někde leží hranice, po jejímž překročení řekněte: „Ne, to je už příliš! Něco
    takového se nemohlo stát náhodou. Takové tvrzení nemohu přijmout. Existuje nějaký skrytý důvod,
    který způsobuje, že pan \(X\) jezdí častěji za slečnou \(A\)“

    Poměr 21 : 9 mohl skutečně nastat i za těch podmínek, s kterými kalkuloval pan \(X\). Jestliže
    hodíme třicetkrát mincí, pak se může stát, že padne jednadvacetkrát panna a devětkrát orel.
    Podobně jevy jsou řídké, ale setkáme se s nimi. Lze spočítat, že z 30 hodů padne panna
    jednadvacetkrát průměrně v pěti případech ze sta. 

    Pokud nám tato četnost připadá malá, vzniká důvod pro zamítnutí nulové hypotézy. Pak je třeba
    hledat důvody, proč předpoklad neplatí. Čím to mohlo být způsobeno v našem případě? Obecně
    řečeno, čímkoli, co vnáší nerovnováhu do jednotlivých polovin tramvajové trati. Pokud je
    východní smyčka kratší než západní (obr. \ref{mai:fig074a}), pak se tramvaj v danou chvíli
    nachází s větší pravděpodobností v západní části města. Nemusí jít pouze o délku trati, ale o
    jakýkoli důvod, který způsobí, že tramvaj projíždí západní část trati déle než východní (obr.
    \ref{mai:fig074b}).
  \end{example}
\end{mdframed}
    %--------------------------------------------------------------- 
  
  \section{Klasická definice pravděpodobnosti}
    Hry jsou staré jako lidstvo samo. Řada her byla jistě konstruována na základě zkušenosti a
    intuice. U složitějších her se ale autor neobešel bez výpočtu, kterým porovnal možnosti výhry a.
    prohry. Stejně výpočty se snažili provádět i hráči. Jejich snahou bylo, je, a jistě i nadále
    bude nalézt hrací systém, který by jim umožnil pohádkové zisky.
    
    Většina hazardních her má následující schéma: existuje konečný počet možných výsledků. Ten může
    být malý (např. 6 u hodu kostkou) i velmi vysoký (např. 13.983.816 při tahu šesti čísel ze 49
    možných ve Sportce). Některé výsledky přinášejí výhru, jiné ne. Spravedlivá hra je taková, u
    které jsou tyto počty stejné. U hazardních her je počet výsledků, při kterých hráč prohrává vždy
    (alespoň nepatrně) vyšší než počet výsledků, při kterých vyhraje. Tento rozdíl přináší
    organizátorovi hry (držiteli banku, majiteli výherního automatu) při dlouhodobém provozování hry
    zaručený zisk.
    
    Výsledek pravděpodobností v tomto schématu je znám již dlouho. Jeho zobecnění přineslo první
    definici pravděpodobnosti:
    \begin{mathdef}{Klasická definice pravděpodobnosti}{mai:def001}
      Může-li určitý jev vykázat \(n\) (\(n\) je konečné číslo) různých, vzájemně se vylučujících
      výsledků, které jsou stejně možné (např. na základě symetrie a homogenity), a jestliže \(m\) z
      těchto výsledků má nevyhnutelně za následek realizaci jevu \(A\) a zbylých \(n - m\) výsledků
      tuto realizaci vylučuje potom pravděpodobnost jevu \(A\) (píšeme \(P(A)\)) položíme rovnu
      \(\frac{m}{n}\)
    \end{mathdef}

    Tato „definice“ pochází od francouzského matematika Pierra Simona de Laplace (1749- 1829) a celé
    století sloužila za základ pro výpočet pravděpodobností. Přesto vzbuzuje značné pochyby:

    \begin{itemize}[noitemsep]
      \item Jak spočítat pravděpodobnosti na nesymetrické kostce?
      \item Co když mám nekonečně mnoho možností? 
      \item Co to vůbec znamená, že výsledky jsou stejně možné? To vlastně jen jinými slovy říkám,
            že jsou stejně pravděpodobné. Pravděpodobnost (její míru) chci ale touto větou definovat.
            Jedná se tedy o tautologii (v definici pojmu již tento pojem používám).
    \end{itemize}
    
    Ze všech těchto důvodů nemůže být „klasická definice“ považována za definici pravděpodobnosti v
    matematickém smyslu. Může ale v určitých případech sloužit jako algoritmus výpočtu.

% ------------------------------- PRAVDĚPODOBNOST---------------------------------------------------      
  \section{Pravděpodobnost}\label{mai:IchapIVsecII}
    Slovo pravděpodobnost používáme velmi často. Jaký však je jeho přesný význam? Jsme přesvědčeni, 
    že pravděpodobnost výhry ve sportce je velmi malá. Ani pravděpodobnost, že se vyplní předpověď 
    počasí, nepovažujeme mnohdy za výraznou. Přesto je mezi oběma příklady obrovský kvantitativní 
    rozdíl. Zkusme význam pojmu pravděpodobnost ukázat pomocí konkrétních číselných příkladů.
  
    \begin{itemize}[noitemsep]
      \item \textbf{Příklad se střelcem}: Sportovní střelec střílí na terč série \num{100} ran. 
            Předpokládejme, že podmínky při střelbě jsou stále stejné. Stejná je zbraň, terč, 
            vzdálenost, povětrnostní podmínky i momentální zdravotní stav střelce. Při hodnocení 
            střelcova „mistrovství“ někdo řekne, že střelec zasáhne terč s pravděpodobností 
            \num{92}\%. Jak tomu rozumět? Znamená to, že v souboru sérií výstřelů jsou velmi časté 
            ty, v nichž zasáhl střelec terč \num{92}-krát. Samozřejmě, není řídké, že se objeví i 
            série s \num{93} nebo \num{94} zásahy, ale také s \num{91} nebo \num{90}. Vyloučen není 
            ani případ s úspěšností \num{96} či \num{88}, a dokonce i stovku bychom mohli 
            zaznamenat. Situace výrazně odlišné od \num{92} zásahů však budou řídké, a to tím více, 
            čím více se úspěšnost série liší od \num{92} oběma směry.
      \item \textbf{Příklad se zmetky}: Koupíte si výrobek u firmy, o které je známo, že vyrábí 
            zmetky s pravděpodobností 0,16\%? Situaci lze posuzovat obdobně jako úspěšnost střelce. 
            Budeme-li například zkoumat série obsahující 1000 výrobků, bude každá z nich obsahovat 
            „v průměru“ 16\% zmetků. Z příkladu se střelcem již zhruba víme, jak posuzovat slovo v 
            průměru.
    \end{itemize}
    
    V této kapitole se budeme pravděpodobnostmi zabývat podrobněji. Zjistíme, že i když se týkají 
    náhodných jevů, platí i pro ně jisté zákonitosti. V úvodních příkladech jsme si vyložili, jak 
    intuitivně chápat pojem pravděpodobnost. Jednalo se v nich o posouzení průměrné úspěšnosti ve 
    velkém souboru operací či úkonů prováděných za stejných podmínek, šlo tedy o jakousi 
    „průměrnou“ pravděpodobnost. Nyní definujeme pravděpodobnost matematicky.
    
    \twocolumn[\subsection{Co se pravdě podobá - definice pravděpodobnosti}\label{mai:IchapIVsecIIssecI}]
      Pro definici pravděpodobnosti použijeme pojmu \textbf{náhodný pokus}, jehož význam si ukážeme 
      na příkladu. Dobrým příkladem náhodných pokusů je třeba házení mincí, hraní kostkou, výběr 
      karet z balíčku, vidíme-li pouze jejich rub, apod. Budeme třeba házet kostkou. Abychom si 
      situaci zbytečně nekomplikovali, budeme předpokládat, že všechny výsledky hodu kostkou 
      (náhodné pokusy) jsou stejně časté, žádný z nich není nijak preferován\footnote{Kostka by 
      tedy měla být homogenní, plocha, na kterou po hodu dopadne, vodorovná, kvalita povrchu všech 
      stěn kostky stejná (žádná stěna by třeba neměla být natřena lepidlem), apod.}. Počet možných 
      výsledků jednotlivého hodu je \(N = 6\) (kostka má \num{6} stěn, na každé je vyznačen odlišný 
      počet ok, tedy \num{1} až \num{6}). Jednotlivé situace, které mohou nastat, nazýváme 
      náhodnými jevy. Náhodným jevem \(A\) tak může být situace \emph{„padne číslo \num{2}“}, jiným 
      náhodným jevem \(B\) situace \emph{„padne číslo dělitelné třemi“}, apod. Počet situací, kdy 
      výsledek hodu lze hodnotit tak, že určitý jev nastal, označíme \(M\). Například pro jev \(A\) 
      \emph{„padne číslo \num{2}“} je \(M(A)= 1\), pro jev \(B\) \emph{„padne číslo dělitelné 
      třemi“} je \(M(B) = 2\) (počet ok \num{3} nebo \num{6}). Můžeme také definovat jev \(O\) 
      \emph{„nepadne žádné číslo“} (\(M(0) = 0\)) nebo jev \(J\) \emph{„padne jakékoli číslo“} 
      (\(M(J) = 6\)).      
      \begin{mathdef}{Klasická definice pravděpodobnosti}{def005}
        Pravděpodobností jevu rozumíme podíl
        \begin{equation}\label{mai:eq183}
          p = \frac{M}{N} = \frac{\text{počet případů příznivých}}{\text{počet případů možných}}.
        \end{equation}  
        Počtem případů možných jsme zkráceně nazvali počet všech možných výsledků náhodného 
        pokusu, počtem případů příznivých pak počet všech takových výsledků pokusu, při nichž daný 
        jev nastal.
      \end{mathdef}

      Je zřejmé, že hodnota pravděpodobnosti jakéhokoli jevu je nezáporná a může nabývat hodnoty 
      nejvýše \num{1}, tj. \(0 <p< 1\). Jev s \emph{nulovou pravděpodobností} se nazývá 
      \textbf{nemožný}, jev s \emph{jednotkovou pravděpodobností} je \textbf{jistý}. V našem 
      příkladu s kostkou tak dostáváme
      \begin{equation*}
        p(A) = \frac{1}{6}, \; p(B) = \frac{2}{6} = \frac{1}{3}, \; p(O) = 0, \; p(J) = 1.
      \end{equation*}  

      %--Barevné ponožky----------------------------------------------
      % !TeX spellcheck = cs_CZ
\begin{example}
 \label{mai:exam006}
  \textbf{Barevné ponožky}:\newline\small
  V zásuvce jsou ponožky tří barev. Červené (\textbf{Č}), zelené (\textbf{Z}) a modré (\textbf{M}). 
  Je jich tam od každé barvy hodně. Student jde na schůzku a chce si vzít čisté ponožky. Náhle 
  zhasne světlo. Student vytáhne potmě dvě ponožky. Jaká je pravděpodobnost, že ponožky budou mít 
  stejnou barvu? Vyjmenujme případy, které mohou při vytažení dvou ponožek nastat: (\textbf{Č+Č}), 
  (\textbf{Č+Z}), (\textbf{Z+Č}), (\textbf{Č+M}), (\textbf{M+Č}), (\textbf{Z+Z}), (\textbf{Z+M}), 
  (\textbf{M+Z}), (\textbf{M+M}). Je tedy \(n = 9\). Příznivé situace jsou tří, (\textbf{Č+Č}), 
  (\textbf{Z+Z}), (\textbf{M+M}). Pravděpodobnost je tedy 1/3. (Převzato z 
  \cite[s.~200]{Musilova2009MA1}) 
\normalsize
\end{example}
      %---------------------------------------------------------------
    \subsection{Cifry, kostky, karty - kombinatorické opakování}\label{mai:IchapIVsecIIssecII}
      Příklad s ponožkami byl velmi jednoduchý. Podařilo se nám vyjmenovat všechny případy možné i 
      všechny případy příznivé, neboť obojího bylo docela málo. Daleko běžnější jsou však situace, 
      kdy výčet případů není schůdný. A tehdy potřebujeme \textbf{kombinatoriku}.
      
      Nechť \(\mathcal{M}\) je \(n\)-prvková množina, z níž budeme provádět výběry \(k\) prvků 
      podle určitých pravidel. Prvky množiny \(\mathcal{M}\) nemusíme nijak konkretizovat. Abychom 
      si však o výběrech a pravidlech pro jejich tvorbu dokázali udělat nějakou názornou představu, 
      je taková konkretizace vhodná. Prvky množiny \(\mathcal{M}\): mohou být třeba žáci ve třídě, 
      barvy, hrací karty, apod. Výběry mohou představovat třeba družstva pro odbíjenou, signály 
      tvořené barevnými praporky, možnosti rozdání karet při mariáši, apod. Jednotlivé typy výběrů 
      získaly své názvy právě na základě pravidel stanovených pro jejich vytváření. Rozhodující 
      jsou dvě základní kritéria:
      \begin{itemize}[noitemsep]
        \item Je pro tvorbu výběru podstatné pořadí prvků ve výběru či nikoliv?
        \item Mohou se prvky ve výběru opakovat či nikoliv?
      \end{itemize}
      
      Typy výběrů shrnuje následující diagram:
      \luagraphic[1]{mai_fig021.pdf}{Typy výběrů. \cite[s.~201]{Musilova2009MA1}}{mai:fig021}

      Představuje-li daný výběr například volejbalové družstvo osmi děvčat (šest hráček a dvě 
      náhradnice), které bude reprezentovat v soutěži třídu osmou bé, do níž chodí \num{25} děvčat 
      a \num{18} chlapců, jedná se o výběr \(k - 8\) prvků z počtu \(n = 25\) prvků. Chlapce nelze 
      postavit do družstva volejbalistek. Každý výběr možného družstva bude představovat 
      \emph{kombinaci bez opakování}, neboť pořadí hráček nehraje roli a třeba Aničku Novákovou 
      máme ve třídě jen jednu. Budeme-li však chtít vytvářet z deseti cifer \(0, 1, \ldots, 9\) 
      trojciferná čísla, pak tyto výběry tří prvků z deseti (\(k = 3\), \(n = 10\)) jsou 
      \emph{variacemi s opakováním}. Čísla \num{125}, \num{512}, \num{251}, \num{215}, \num{521} a 
      \num{152} jsou totiž různá, a například \num{222} je také trojciferné číslo. Kombinace s 
      opakováním bychom mohli vytvářet třeba i při výběru různobarevných ponožek ze zásuvky a 
      konečně \emph{variacemi bez opakování} by mohly být dejme tomu trojbarevné signály (\(k = 
      3\)) tvořené trojicemi barevných hadříků vybíraných z \(n\) barev (pro \(n = 3\) třeba zrovna 
      z těch ponožek). Nyní bychom však rádi věděli, jak pro zadané hodnoty \(n\) a \(k\) určit 
      počet všech možných výběrů předepsaného typu. Ukážeme si to na příkladech.

      %--Šance milion-------------------------------------------------
      % !TeX spellcheck = cs_CZ
\begin{mdframed}[style=mdexam]
  \begin{example}\label{mai:exam007}
    \textbf{Šance milion}:\newline
    „Znáte nějakou jinou hru, kde můžete denně vyhrát milion?“ Tento nebo jiný, obdobně nepříliš
    vtipný reklamní slogan propaguje v televizi hru, jejímž cílem je uhodnout šestici tažených cifer
    ve správném pořadí. (Hru raději nehrajte, pravděpodobnost výhry je mizivá.) Tah se provádí
    následovně: V každém ze šesti bubnů, očíslovaných pořadovými čísly \num{1} až \num{6}, je
    připraveno deset míčků opatřených ciframi \(0, 1, \ldots, 9\). Z prvního bubnu se náhodně
    vylosuje jedna cifra (deset možností). Poté se náhodně vylosuje jedna cifra z druhého bubnu
    (opět deset možností). Možností vzniku uspořádané dvojice cifer (jedna cifra z prvního a druhá z
    druhého bubnu) je již sto (každou možnost výsledku u prvního bubnu lze kombinovat s každou
    možností výsledku z druhého bubnu). Losování pokračuje u třetího, čtvrtého, pátého a šestého
    bubnu. Celkový počet možností je \num{1e6}, tedy \textbf{milion}. (Šance získat výhru, tedy
    vyhrát milion, je ovšem pouze jedna milióntina, neboť z milionu možností je pouze jedna skutečně
    tažena.) 
  \end{example}
\end{mdframed}
      %---------------------------------------------------------------
      
      Zobecněním předchozího příkladu získáváme vzorec pro počet \textbf{variací s opakováním} 
      \emph{k}-té třídy z \(n\) prvků. Při tahu totiž záleží na pořadí bubnů a každý buben obsahuje 
      všechny cifry. Výsledky tahů z jednotlivých bubnů se tedy mohou opakovat. Pokud by bubnů bylo 
      \(k\) a v každém \(n\) různých cifer, dostali bychom pro \textbf{variace s opakováním 
      \(k\)-té třídy z \(n\) prvků} celkový počet
      \begin{mdframed}[style=highlight]
        \begin{equation}\label{mai:eq007}
          \boxed{V_k' = n^k}\, .
        \end{equation}
      \end{mdframed}

      %--Modifikovaná šance milion------------------------------------
      % !TeX spellcheck = cs_CZ
\begin{example}\label{mai:exam008}
  \textbf{Modifikovaná šance milion}:\newline
  Představme si hru z předchozího příkladu upravenou takto: K dispozici bude jen jeden buben s 
  ciframi \(0, 1, \ldots, 9\), každá cifra je v bubnu obsažena pouze jednou. Opět máme losovat 
  uspořádanou \textbf{šestici cifer}. Nyní se však jedná o \textbf{variace šesté třídy z deseti 
  prvků bez opakování}. S jediným bubnem musíme totiž provést šest losování, přičemž při každém 
  losování ubude z bubnu jedna cifra. Při prvním tahu je deset možností, při druhém již jen devět, 
  atd., při šestém již pouze pět možností. Celkem je tedy \(10 \cdot 9 \cdots 5 = \num{151200}\) 
  možností.
\end{example}
      %---------------------------------------------------------------
      
      Uvážíme-li, že v předchozím příkladu je \(n = 10\) a \(k = 6\), dostáváme pro \textbf{počet 
      variací bez opakování \(k\)-té třídy z \(n\) prvků} obecný vztah
      \begin{align}
        V_k(n) &= n(n-1)(n-2)\cdots(n-k+1)  \nonumber \\
        \shortintertext{neboli}
        V_k(n) &= \frac{n!}{(n-k)!}\, .    \label{mai:eq008}
      \end{align}
      Poznamenejme, že \(n!\) značí \textbf{faktoriál}, \(n! = n(n - 1)\cdots 3 \cdot 2 \cdot 1\). 
      Pro nulu definujeme \(0! = 1\). Je zřejmé, že při vytváření variací bez opakování musí být 
      \(k\leqq n\). Variace bez opakování \(n\)-té třídy z \(n\) prvků se nazývají 
      \textbf{permutace}. Každá z nich představuje určité uspořádání těchto \(n\) prvků. Platí
      \begin{equation}\label{mai:eq009}
        \boxed{P(n) = V_n(n) = n!}\, .
      \end{equation}
      
      Nyní odvodíme vzorec pro počet \textbf{kombinací \(k\)-té třídy z \(n\) prvků bez opakování}. 
      Již jsme si řekli, že \emph{kombinací} rozumíme takový výběr z celkového počtu \(n\) prvků, 
      který obsahuje určitých \(k\) prvků nezávisle na jejich pořadí. Představme si, že máme k 
      dispozici všechny variace bez opakování \(k\)-té třídy ze zmíněných \(n\) prvků. Vezměme 
      kteroukoli z nich. Soubor všech variací \(k\)-té třídy z \(n\) prvků však obsahuje i další 
      variace, lišící se od té naší jen pořadím prvků. Celkem je takových variací (i s tou první) 
      \(k!\) a z hlediska kombinací představují totéž. Soubor variací se tak rozpadá na podsoubory, 
      z nichž každý obsahuje \(k!\) variací lišících se navzájem pouze pořadím prvků. Každý z 
      těchto podsouborů představuje však jedinou kombinaci. Počet kombinací \(k\)-té třídy z \(n\) 
      prvků bez opakování je tedy
      \begin{equation}\label{mai:eq010}
        \boxed{C(k) = \frac{V_k(n)}{P(k)} = \frac{n!}{(n-k)!\,k!} = \binom{n}{k}}\, .
      \end{equation}
      
      Pro odvození vzorce pro \textbf{kombinace s opakováním} použijeme opět příkladu.
      %--Kuličky v přihrádkách----------------------------------------
      % !TeX spellcheck = cs_CZ
\begin{mathexam}{Kuličky v přihrádkách}{exam009}
  Máme kuličky \(n\) různých barev, v každé barvě máme tolik kuliček, kolik bude potřeba. Naším
  úkolem je vytvářet výběry \(k\) kuliček. Na \textbf{pořadí barev nezáleží}, kuliček jedné barvy
  může být ve výběru libovolný počet \(0\leqq s \leqq k\). Výběry budeme vytvářet tak, že budeme
  kuličky dávat do \(n\) přihrádek, z nichž každá bude vyhrazena pro určitou barvu. Pokud tedy v
  daném výběru zrovna nebude třeba modrá kulička, bude přihrádka vyhrazená pro modrou barvu prázdná.
  Budou-li v daném výběru právě tři červené kuličky, budou umístěny v přihrádce vyhrazené pro
  červenou barvu. Vidíme, že pokud konkrétním přihrádkám přisoudíme konkrétní barvy, samotné kuličky
  by již barevné být nemusely, stačily by třeba kuličky skleněné, bezbarvé. Zůstane-li například
  přihrádka pro modrou barvu prázdná, víme, že daný výběr neobsahuje modrou barvu. Budou-li v
  přihrádce pro červenou barvu tři (bezbarvé) kuličky, víme, že daný výběr obsahuje červenou barvu
  třikrát. Příklad takové situace ukazuje následující schéma:
  
  {\centering
    \luafigure[0.9]{mai_fig033.pdf}
    \par}

  Náš úkol můžeme přeformulovat takto: Je třeba rozmístit \(k\) kuliček do \(n\) přihrádek. V každé
  přihrádce může být obecně \(s\) kuliček, kde \(0\leqq s \leqq k\), přitom celkový počet kuliček
  musí být samozřejmě stále \(k\). Můžeme si představit, že \(k\) kuliček máme položených v řadě na
  polici mezi dvěma pevnými stěnami (krajní svislé čáry v předchozím schématu) a různé způsoby
  rozmístění kuliček do přihrádek provádíme přemísťováním pohyblivých přepážek. Kdybychom například
  v předchozím schématu přesunuli druhou svislou čáru, počítáno zleva, až za první kuličku v
  přihrádce na červenou barvu, dostaneme uspořádání, při němž je v přihrádce na modrou barvu jedna
  kulička a v přihrádce na červenou barvu dvě kuličky. Tedy takto:

  {\centering
    \luafigure[0.9]{mai_fig034.pdf}
  \par}

  Mezi dvěma krajními pevnými stěnami máme tedy k dispozici \(k\) pozic pro kuličky a \((n - 1)\)
  pozic pro pohyblivé přepážky. Celkem tedy \((n + k - 1)\) pozic, na které můžeme libovolně
  rozmísťovat \(k\) kuliček a \((n - 1)\) přepážek. Do těchto \((n + k - 1)\) pozic můžeme umístit
  \(k\) kuliček \(C_k'(n)\) způsoby, kde
  \begin{equation}\label{MAI:eq011}
    \boxed{C_k'(n) =  \binom{n + k - 1}{k} = \binom{n + k - 1}{n -1}}\, .
  \end{equation}
  Na zbylé pozice již musíme umístit přepážky. Nebo naopak, nejprve umístíme \((n - 1)\) přepážek a
  potom kuličky. Výsledek je stejný, jak je vidět z předchozího vzorce. Protože jsme vytváření
  kombinací s opakováním \(k\)-té třídy z \(n\) prvků převedli na úlohu o rozmísťování kuliček do
  přihrádek, udává získaný vzorec právě počet takových kombinací. Aby měl vzorec smysl, musí platit
  \(n + k - 1 \geqq k\), tedy \(n \geqq 1\).

  Komu nevyhovuje představa kuliček v přihrádkách a má raději čísla, může uvažovat následovně: Tak
  jako je každé číslo v desítkové soustavě zapsáno pomocí cifer \(0, 1, 2, \ldots , 8, 9\), je k
  jeho zápisu ve dvojkové soustavě potřeba pouze dvou cifer, nuly a jedničky. Představme si nyní
  přepážku jako jedničku a kuličku jako nulu. Náš úkol zjistit počet všech možných způsobů rozdělení
  \(k\) kuliček do \(n\) přihrádek, ohraničených \((n+1)\) přepážkami, můžeme převést na
  ekvivalentní problém: Kolik dokážeme najít čísel, která jsou ve dvojkové soustavě zapsána právě
  \(k\) nulami a \((n + 1)\) jedničkami, požadujeme-li, aby první i poslední cifrou byla jednička?
  Odpověď je jednoduchá. Máme k dispozici \((n+k+1)\) pozic pro cifry. První a poslední pozice jsou
  pevně obsazeny jedničkami, volných pozic je tedy pouze \((n + k - 1)\). Počet všech různých
  způsobů, kterými na \(k\) z těchto pozic můžeme umístit nuly, je roven počtu kombinací \(k\)-té
  třídy z \((n + k - 1)\) prvků. Na zbylé pozice již musíme umístit jedničky. Komplementárně,
  budeme-li hledat počet všech možných způsobů, jak na \((n-1)\) pozic umístit jedničky, dostaneme
  shodný výsledek, v souhlasu se vzorcem (\ref{mai:eq010}).
\end{mathexam}
      %---------------------------------------------------------------
      
      %--Obsazování kvantových stavů----------------------------------
      % !TeX spellcheck = cs_CZ
\begin{example}\label{mai:exam010}
  \textbf{Obsazování kvantových stavů}:\newline\small
  Úloha o kuličkách a přihrádkách má přímou aplikaci v \textbf{kvantové fyzice}. Představme si, že 
  fyzikální soustava je tvořena \(k\) částicemi. Každá částice se nachází v určitém stavu, v němž 
  jí můžeme přisoudit fyzikální charakteristiky, které jsou s tímto stavem spojeny (třeba energii, 
  moment hybnosti, apod.). Jednotlivé stavy jsou pak rozlišitelné právě pomocí těchto 
  charakteristik. Dejme tomu, že přípustných stavů je \(n \geqq 1\). Problémem kvantové fyziky je 
  to, že kvantové částice jsou nerozlišitelné. Nepoznáme jednu od druhé. Je to stejné, jako bychom 
  měli \(k\) naprosto stejně vypadajících kuliček, které nemáme nijak očíslovány. Záměna dvou 
  částic (nerozlišitelných kuliček) se nepozná, nevede tedy ke změně stavu fyzikální soustavy. Pro 
  hodnoty fyzikálních charakteristik soustavy jako celku je tedy důležité jen to, kolik částic je v 
  každém z přípustných stavů. Musíme se tedy zajímat o to, kolika způsoby lze našich \(k\) 
  \textbf{nerozlišitelných částic} (kuliček) umístit do \(n\) \textbf{stavů} (přihrádek). Kvantové 
  částice jsou však dvojího druhu, \textbf{fermiony} (například elektrony, neutrony, protony, jádra 
  s lichým počtem nukleonů) a \textbf{bosony} (například fotony, mezony, jádra se sudým počtem 
  nukleonů). Rozdíl mezi nimi je ten, že bosony se „dobře snášejí“, a proto jich může být v jednom 
  stavu i více. 
  \begin{itemize}\addtolength{\itemsep}{-0.5\baselineskip}
    \item Počet možností, jak rozmístit \(k\) \textbf{bosonů} po \(n\) stavech je tedy
          \begin{equation*}
            N_{\text{boson}} = 
              \begin{pmatrix}
                n + k - 1 \\
                    k
               \end{pmatrix}
          \end{equation*}
    \item S \textbf{fermiony} je tomu jinak. \textbf{Pauliho vylučovací princip} jim zakazuje, 
          aby v daném stavu byl více než jeden fermion. Stav může být buď prázdný, nebo obsazen 
          jedním fermionem. V takovém případě musí být \(n \geqq k\) a v každé přihrádce může být 
          nejvýše jedna kulička. Situace tak odpovídá \textbf{kombinacím bez opakování \(k\)-té 
          třídy z \(n\) prvků}, tj.
          \begin{equation*}
            N_{\text{fermion}} = 
              \begin{pmatrix}
                n  \\
                k
              \end{pmatrix}
          \end{equation*}
  \end{itemize}
\normalsize
\end{example}
      %---------------------------------------------------------------
      
      Získané kombinatorické vzorce nyní použijeme při řešení základních úloh o pravděpodobnostech. 
      V každé úloze bude důležité
      \begin{itemize}
        \item definovat jev \(A\), jehož pravděpodobnost počítáme,
        \item určit počet \(N\) případů možných, tj. počet všech možných výsledků pokusu, při 
              kterém sledujeme, zda jev \(A\) nastal či nenastal,
        \item určit počet \(M\) případů příznivých, tj. počet těch výsledků daného pokusu, při 
              kterých jev \(A\) nastal.
      \end{itemize}

      %--Výhra ve sportce---------------------------------------------
      % !TeX spellcheck = cs_CZ
\wikitextrule
\begin{example}\label{mai:exam052}
  \textbf{Výhra ve sportce}\newline\small
  Jaká je pravděpodobnost hlavní výhry ve sportce? Všichni víme, že malá, ale máme představu, jak 
  malé toto číslo je? Při sportce se losuje \(k = 6\) čísel a jedno dodatkové z celkového počtu \(n 
  = 49\) čísel. (Dříve byla čísla spojena s názvy sportů, odtud název „sportka“ .) Na pořadí čísel 
  ve výběru nezáleží, vytažené číslo se do hry nevrací. Jde tedy o \textbf{kombinace bez 
  opakování}. Hlavní výhra požaduje uhodnout všech \num{6} tažených čísel. Jev \(A\) je tedy 
  definován takto:
  
  \begin{itemize}
    \item Jev \(A\): Bude taženo právě oněch \num{6} čísel, která jsem vsadil.
          Počet možností, které při tahu sportky mohou nastat (počet případů možných), je \(N = 
          \begin{pmatrix} n \\ k\end{pmatrix} =  \begin{pmatrix} 49 \\ 6 \end{pmatrix} \). Hlavní 
          výhru představuje jediná kombinace, počet příznivých případů je proto \(M = 1\). 
          Pravděpodobnost hlavní výhry ve sportce, tj. pravděpodobnost jevu \(A\), je
          \begin{equation*}
            p(A) = \dfrac{M}{N} = \dfrac{1}{\begin{pmatrix} 49 \\ 6 \end{pmatrix}} 
                 = \dfrac{43!6!}{49!} = \dfrac{720}{49\cdot48\cdot47\cdot46\cdot45\cdot44} \simeq 
                 \num{7e-8}
                 = \SI{7e-6}{\percent}
          \end{equation*}
          Pravděpodobnost hlavní výhry je velmi malá, sedm milióntin procenta. 
    \item A o kolik lepší to bude s pravděpodobností některé z vedlejších výher? Tak třeba pátá 
          cena znamená, že je nutné ze šesti tažených čísel uhodnout libovolné tři. Jev \(A\) je 
          tedy: Ze šesti čísel, která jsme vsadili, budou v tažené kombinaci obsažena právě tři 
          libovolná z nich. Počet \(N\) zůstává stejný jako v předchozí části úlohy. Je třeba jen 
          určit \(M\). Každý příznivý případ vzniká tak, že trojice správných čísel (výběry tří ze 
          šesti) je doplněna trojicí chybných čísel (výběry tří ze čtyřiceti tří). Tedy \(M = 
          \begin{pmatrix}6 \\ 3\end{pmatrix}\begin{pmatrix} 49 - 6 \\ 3\end{pmatrix} =  
          \begin{pmatrix} 6 \\ 3\end{pmatrix}\begin{pmatrix} 43 \\ 3 \end{pmatrix}\),
          \begin{align*}
            p(A) &= \dfrac{M}{N} 
                  = \dfrac{\begin{pmatrix} 6 \\ 3\end{pmatrix}
                           \begin{pmatrix} 43 \\ 3 \end{pmatrix}}
                          {\begin{pmatrix} 49 \\ 6\end{pmatrix} }                
                  =\left(\dfrac{6!}{3!\cdot3!}\right)\left(\dfrac{43!}{40!\cdot3!}\right)
                   \left(\dfrac{43!6!}{49!}\right)                                            \\
                 &=\dfrac{120\cdot43\cdot42\cdot41\cdot720}
                         {49\cdot48\cdot47\cdot46\cdot45\cdot44\cdot36} 
                  \simeq\num{0.018}.
          \end{align*}
          Tato pravděpodobnost již zanedbatelná není, na rozdíl od finanční částky, jíž bývá 
          ohodnocena pátá cena. Sázení sportky může domácímu rozpočtu spíše ublížit.
    \item Třetí, resp. čtvrtá cena jsou, podobně jako první a pátá, definovány velmi jednoduše. Je 
          třeba uhodnout pět, resp. čtyři ze šesti tažených čísel. V případě druhé ceny hraje roli
          dodatkové číslo. Druhou cenu získává ten, kdo uhodl pět ze šesti čísel vylosovaných v 
          prvním tahu a ještě navíc číslo dodatkové, které se losuje ze zbylých \num{43} čísel, jež 
          zůstala po prvním tahu v osudí. Jev \(A\) je tedy definován takto:
          
          Ze šesti čísel, která jsem vsadil, bude při prvním tahu vylosováno libovolných pět a v 
          druhém tahu bude vylosováno právě to dodatkové číslo, které jsem vsadil. Počet případů 
          příznivých je pouze \(M = \begin{pmatrix}6 \\ 5\end{pmatrix}\), neboť šestým číslem 
          nemůže být libovolné ze \num{43} čísel, která nebyla v prvním tahu vylosována, ale 
          musí to být právě číslo dodatkové. Pravděpodobnost jevu \(A\) je
          \begin{equation*}
            p(A)  = \dfrac{M}{N} 
                  = \dfrac{\begin{pmatrix} 6  \\ 5\end{pmatrix}}
                          {\begin{pmatrix} 49 \\ 6\end{pmatrix}}
                  \simeq\num{4.2e-17}.
          \end{equation*}
          
          Pokud bychom jako jev \(A\) označili výhru jakékoliv ceny, dostaneme
          \begin{align*}
            M &= \sum_{6}^{k=3}\begin{pmatrix} 6  \\   k  \end{pmatrix}
                               \begin{pmatrix} 43 \\ 6 - k\end{pmatrix}  +
                               \begin{pmatrix} 6  \\   5  \end{pmatrix}                      \\
              &= \begin{pmatrix} 6 \\ 3\end{pmatrix}\begin{pmatrix} 43 \\ 3\end{pmatrix} +
                 \begin{pmatrix} 6 \\ 4\end{pmatrix}\begin{pmatrix} 43 \\ 2\end{pmatrix} +
                 \begin{pmatrix} 6 \\ 5\end{pmatrix}\begin{pmatrix} 43 \\ 1\end{pmatrix} +
                 \begin{pmatrix} 6 \\ 6\end{pmatrix}\begin{pmatrix} 43 \\ 0\end{pmatrix} +
                 \begin{pmatrix} 6 \\ 5\end{pmatrix},                                        \\
           p(A) &= \dfrac{\begin{pmatrix}  6 \\ 3\end{pmatrix}
                          \begin{pmatrix} 43 \\ 3\end{pmatrix}  +
                          \begin{pmatrix}  6 \\ 4\end{pmatrix}
                          \begin{pmatrix} 43 \\ 2\end{pmatrix}  +
                          \begin{pmatrix}  6 \\ 5\end{pmatrix}
                          \begin{pmatrix} 43 \\ 1\end{pmatrix}  +
                          \begin{pmatrix}  6 \\ 6\end{pmatrix}
                          \begin{pmatrix} 43 \\ 0\end{pmatrix}  +
                          \begin{pmatrix}  6 \\ 5\end{pmatrix}}
                         {\begin{pmatrix} 49 \\ 6\end{pmatrix}} =\simeq\num{0.019}.
          \end{align*}
          Všimněte si, že tento výsledek je roven součtu pravděpodobností výhry páté, čtvrté, 
          třetí, druhé a hlavní ceny. Později si tento závěr ještě připomeneme.
  \end{itemize}  
  \normalsize
\end{example}
      %---------------------------------------------------------------

      %--Losování karet-----------------------------------------------
      % !TeX spellcheck = cs_CZ
\wikitextrule
\begin{example}\label{mai:exam053}
  \textbf{Losování karet}\newline\small
    Máme karetní hru mariáš, která obsahuje celkem \num{32} karet osmi hodnot \num{7}, \num{8}, 
    \num{9}, \num{10}, J (kluk), Q (dáma), K (král), A (eso), každá hodnota je ve čtyřech barvách, 
    červené barvy jsou \(\heartsuit\) (srdce) a \(\lozenge\) (kára), černé barvy jsou \(\spadesuit\)
    (piky) a \(\clubsuit\) (kříže). Jaká je pravděpodobnost, že při náhodném vylosování deseti 
    karet budou 
    mezi nimi:
    \begin{enumerate}[label=\Alph*]
      \item právě dvě esa,
      \item alespoň dvě esa,
      \item nejvýše dvě esa,
      \item alespoň šest karet stejné barvy,
      \item právě dvě dámy a alespoň jeden kluk,
      \item právě dvě dámy nebo alespoň jeden kluk?
    \end{enumerate}

    Písmena (A) až (F) představují různé části úlohy a také zároveň definují jevy, jejichž 
    pravděpodobnost počítáme. Jedná se opět o kombinace. Nezáleží totiž na pořadí, v jakém karty 
    vytahujeme. Důležité je jen to, zda jsou vyjmenované karty ve výběru obsaženy. Počet možných 
    výsledků náhodného vylosování deseti karet z dvaatřiceti, tj. počet případů možných, je pro 
    všechny části úlohy stejný,
    \begin{equation*}
      N = \begin{pmatrix} 32 \\ 10\end{pmatrix} 
        = \dfrac{32\cdot31\cdots24\cdot23}{10\cdot9\cdots2\cdot1} = \num{64512240}.
    \end{equation*}
    
    Počítejme nyní případy příznivé pro jednotlivé jevy \(A\) až \(F\) a pravděpodobnosti těchto 
    jevů:
    \begin{equation*}
      M(A) = \begin{pmatrix} 4 \\ 2\end{pmatrix}\begin{pmatrix} 32 - 4 \\ 10 - 2\end{pmatrix}
           = \begin{pmatrix} 4 \\ 2\end{pmatrix}\begin{pmatrix} 28 \\ 8\end{pmatrix}
           = 6\cdot\dfrac{28\cdot27\cdots1}{8\cdot7\cdots2\cdot1} = \num{18648630}.
    \end{equation*}
    Jak jsme k tomuto výsledku došli? Příznivý pro daný jev je každý výběr, v němž jsou obsažena 
    právě dvě esa (libovolných barev) a žádná další esa (význam slova „právě“). Počet výběrů dvou 
    es z celkového počtu čtyř es je \(C_2(4)\), počet výběrů dalších libovolných osmi karet ze 
    zbývající části hry, která vznikne po odstranění es (nechceme, aby v příznivém výběru byla 
    další esa), je \(C_{(10-2)}(32 - 4) = C_8(28)\). Každý výběr dvojice es lze kombinovat s každým 
    výběrem zbývajících osmi karet ze zbytku hry, tj. \(M(A) = C_2(4)\cdot C_8(28)\). A to je právě
    náš předchozí výsledek. Potom:
    \begin{equation*}
      p(A) = \dfrac{M(A)}{N} 
           = \dfrac{\begin{pmatrix} 4 \\ 2\end{pmatrix}\begin{pmatrix} 28 \\ 8\end{pmatrix}}
                   {\begin{pmatrix} 32 \\ 10\end{pmatrix}}
           = \dfrac{\num{18648630}}{\num{64512240}} \simeq \num{0.29}
    \end{equation*}
    
    Aby nastal jev \(B\), požadujeme, aby v náhodném výběru deseti karet z dvaatřiceti byla alespoň 
    dvě esa. To znamená, že výběr považujeme za příznivý, obsahuje-li dvě esa libovolné barvy a osm 
    libovolných karet jiné hodnoty, nebo obsahuje tři esa libovolné barvy a sedm libovolných karet 
    jiné hodnoty, nebo obsahuje všechna čtyři esa a šest libovolných karet jiné hodnoty, \(k\) es 
    (pro \(k = 2, 3, 4\)) můžeme ze čtyř es vybrat \(\begin{pmatrix} 4 \\ k\end{pmatrix}\) způsoby.
    \(10 - k\) karet jiné hodnoty pak musíme vybírat pouze z \num{28} karet (esa je nutno 
    odstranit, aby bylo zaručeno, že „doplňkové“ karty budou mít jinou hodnotu než eso). Výběr 
    zbývajících karet lze učinit \(\begin{pmatrix} 28 \\ 10 - k\end{pmatrix}\) způsoby. Nakonec 
    tedy dostáváme
    \begin{align*}
      M(B) &=  \begin{pmatrix} 4 \\ 2\end{pmatrix}\begin{pmatrix} 28 \\ 8\end{pmatrix}
              +\begin{pmatrix} 4 \\ 3\end{pmatrix}\begin{pmatrix} 28 \\ 7\end{pmatrix}
              +\begin{pmatrix} 4 \\ 4\end{pmatrix}\begin{pmatrix} 28 \\ 6\end{pmatrix}         \\
           &=  6\cdot\dfrac{28\cdot27\cdots22\cdot21}{8\cdot7\cdots2\cdot1}
              +4\cdot\dfrac{28\cdot27\cdots23\cdot22}{7\cdot6\cdots2\cdot1}
              +1\cdot\dfrac{28\cdot27\cdots24\cdot23}{6\cdot5\cdots2\cdot1} = \num{23761530},  \\
      P(B) &= \dfrac{M(B)}{N} = \dfrac{\num{23761530}}{\num{64512240}} \simeq\num{0.37}.
    \end{align*}
    Pozn.: Někomu se předchozí výpočet může zdát příliš složitý. Nelze jej nějak zjednodušit? Co 
    kdybychom uvažovali třeba takto: Výběr dvou es již zajistí splnění požadavku. Doplňkové karty 
    tedy již pak můžeme vybírat ze třiceti karet - nebudeme tedy odstraňovat esa, protože budou-li 
    vybrána mezi doplňkovými kartami, požadavek „alespoň dvou es ve výběru“ to nenaruší. Při takové 
    interpretaci bychom dostali
    \begin{equation*}
      M(B) = \begin{pmatrix} 4 \\ 2\end{pmatrix}\begin{pmatrix} 30 \\ 8\end{pmatrix}   
           = 6\cdot\dfrac{30\cdot29\cdots24\cdot23}{8\cdot7\cdots2\cdot1}
           = \num{35117550}.
    \end{equation*}
    Vidíme, že vyšlo číslo vyšší než při předchozí úvaze. Co je tedy správně? Správně je první 
    úvaha vedoucí k nižšímu počtu příznivých případů. Při druhé úvaze jsme některé případy 
    započetli vícekrát. Zkuste přijít na to, jak se to stalo. V každém případě vidíme, že 
    kombinatorické úvahy, ať již vypadají jakkoli jednoduše, mohou být zrádné a je třeba dát si na 
    ně pozor. 
    
    Jev \(C\) podle zadání nastane, obsahuje-li náhodný výběr deseti karet nejvýše dvě esa. Znamená 
    to, že výběr je příznivý, neobsahuje-li žádné eso a obsahuje deset karet jiné hodnoty, nebo 
    obsahuje-li jedno eso a devět karet jiné hodnoty, nebo obsahuje-li dvě esa a osm karet jiné 
    hodnoty. Počet \(M(C)\) určíme analogicky jako \(M(B)\), ale pro \(k= 0, 1, 2\):
    \begin{align*}
      M(C) &=  \begin{pmatrix} 4 \\ 0\end{pmatrix}\begin{pmatrix} 28 \\ 10\end{pmatrix}
              +\begin{pmatrix} 4 \\ 1\end{pmatrix}\begin{pmatrix} 28 \\ 9\end{pmatrix}
              +\begin{pmatrix} 4 \\ 2\end{pmatrix}\begin{pmatrix} 28 \\ 8\end{pmatrix}         \\
           &=   \cdot\dfrac{28\cdot27\cdots20\cdot19}{10\cdot9\cdots2\cdot1}
              +4\cdot\dfrac{28\cdot27\cdots21\cdot20}{ 9\cdot8\cdots2\cdot1}
              +6\cdot\dfrac{28\cdot27\cdots22\cdot21}{ 8\cdot7\cdots2\cdot1} = \num{59399340}, \\
      P(C) &= \dfrac{M(C)}{N} = \dfrac{\num{59399340}}{\num{64512240}} \simeq\num{0.92}.
    \end{align*}
    
    Jev \(D\) znamená alespoň šest karet stejné barvy (připomeňme, že barvou rozumíme jednu z 
    možností \(\heartsuit\), \(\lozenge\), \(\spadesuit\), \(\clubsuit\)). Hra obsahuje osm karet 
    od každé barvy. Současně je tedy zřejmé, že karet stejné barvy může být ve výběru nejvýše osm. 
    Výběr je příznivý pro \(k = 6, 7, 8\). Obdobnou úvahou jako v předchozích případech dostáváme
    \begin{equation*}
      M = 4\cdot\sum^{8}_{k=6}
          \begin{pmatrix} 8 \\ k \end{pmatrix}\begin{pmatrix} 32 - 8 \\ 10 - k\end{pmatrix}
        = \num{1255984}.
    \end{equation*}
    
    Faktor \num{4} před celou sumou se objevuje proto, že nebylo specifikováno, která ze čtyř barev 
    má být zastoupena alespoň šesti kartami. Všechny čtyři možnosti volby barvy jsou tedy příznivé. 
    Pravděpodobnost jevu \(D\) je
    \begin{equation*}
      p(D)  = \dfrac{M(D)}{N}
            = \dfrac{4\cdot\sum^{8}_{k=6}\dfrac{8!}{k!(8-k)!}\dfrac{24!}{(10-k)!(14+k)!}}
                    {\begin{pmatrix} 32 \\ 10 \end{pmatrix}}                               
            = \dfrac{\num{1255984}}{\num{64512240}} \simeq \num{0.019}.
    \end{equation*}
    
    Případy (\(E\)) a (\(F\)) v zadání se liší pouze slůvkem „a“ a „nebo“. Uvidíme, že nejde o 
    slovíčka, ale o podstatný rozdíl. 
    
    Aby nastal jev \(E\), požadujeme, aby náhodný výběr deseti karet obsahoval právě dvě dámy a 
    alespoň jednoho kluka. Znamená to, že výběr je příznivý, obsahuje-li dvě dámy libovolné barvy a 
    současně alespoň jednoho kluka libovolné barvy. Příznivé možnosti tedy jsou:
    \begin{enumerate}
    \item  dvě dámy libovolné barvy, jeden kluk libovolné barvy, \num{7} libovolných karet, které 
           nemají hodnotu dámy ani kluka, celkem 
           \(\begin{pmatrix} 4 \\ 2 \end{pmatrix}
             \begin{pmatrix} 4 \\ 1\end{pmatrix}
             \begin{pmatrix} 32-2\cdot4 \\ 7 \end{pmatrix} = \num{8306496}\) možností,
    \item dvě dámy libovolné barvy, dva kluci libovolné barvy, \num{6} libovolných karet, které 
          nemají hodnotu dámy ani kluka, celkem 
          \(\begin{pmatrix} 4 \\ 2 \end{pmatrix}
            \begin{pmatrix} 4 \\ 2\end{pmatrix}
            \begin{pmatrix} 32-2\cdot4 \\ 6 \end{pmatrix} = \num{4845456}\) možností,
    \item dvě dámy libovolné barvy, tři kluci libovolné barvy, \num{5} libovolných karet, které 
          nemají hodnotu dámy ani kluka, celkem 
          \(\begin{pmatrix} 4 \\ 2 \end{pmatrix}
            \begin{pmatrix} 4 \\ 3\end{pmatrix}
            \begin{pmatrix} 32-2\cdot4 \\ 5 \end{pmatrix} = \num{1020096}\) možností,
    \item dvě dámy libovolné barvy, všichni čtyři kluci, \num{4} libovolné karty, které nemají 
          hodnotu dámy ani kluka, celkem 
          \(\begin{pmatrix} 4 \\ 2 \end{pmatrix}
            \begin{pmatrix} 4 \\ 4\end{pmatrix}
            \begin{pmatrix} 32-2\cdot4 \\ 4 \end{pmatrix} = \num{63756}\) možností.
    \end{enumerate}
    \begin{align*}
      M(E) &= \begin{pmatrix} 4 \\ 2 \end{pmatrix}\cdot\sum^{4}_{k=1}
              \begin{pmatrix} 4 \\ k \end{pmatrix}\begin{pmatrix} 24 \\ 8 - k \end{pmatrix}
            = 6\left[ 
                  4\begin{pmatrix} 24 \\ 7 \end{pmatrix} +
                  6\begin{pmatrix} 24 \\ 6 \end{pmatrix} +
                  4\begin{pmatrix} 24 \\ 5 \end{pmatrix} +
                   \begin{pmatrix} 24 \\ 4 \end{pmatrix}
                \right]                                                      \\ 
      M(E) &= \num{14235804},                                                \\
      p(E) &= \dfrac{M(E)}{N} = \dfrac{\num{14235804}}{\num{64512240}} \simeq \num{0.22}.
    \end{align*}
    
    Aby nastal jev \(F\), požadujeme, aby náhodný výběr deseti karet obsahoval právě dvě dámy nebo 
    alespoň jednoho kluka. Znamená to, že výběr je příznivý, obsahuje-li dvě dámy libovolné barvy a 
    jakékoli další karty, nebo obsahuje alespoň jednoho kluka a jakékoli další karty. Nyní je nutno 
    o všech možnostech pečlivě rozvažovat, abychom některé nezapočítali vícekrát. Pozor, slůvko 
    \uv{nebo} zde nemá vylučovací význam, připouští se, že mohou být splněny obě podmínky jevu 
    \(F\), tj. jak právě dvě dámy, tak alespoň jeden kluk. Příznivé možnosti jsou
    \begin{enumerate}
      \item dvě dámy libovolné barvy, žádný kluk, \num{8} libovolných karet, které nemají 
            hodnotu dámy ani kluka, celkem
            \begin{equation*}
              \begin{pmatrix} 4  \\ 2 \end{pmatrix}
              \begin{pmatrix} 4  \\ 0 \end{pmatrix}
              \begin{pmatrix} 32 - 2\cdot4 \\ 8 \end{pmatrix} = 6
              \begin{pmatrix} 24 \\ 8 \end{pmatrix},
            \end{equation*}
      \item žádná dáma, \(k\) kluků libovolné barvy pro \(k = 1, 2, 3, 4\) (alespoň jeden kluk), 
            \(10 - k\) karet, které nemají hodnotu dám y ani kluka, celkem
            \begin{equation*}
              \begin{pmatrix} 4  \\ 0 \end{pmatrix}
              \sum^{4}_{k=1}\begin{pmatrix} 4  \\ k \end{pmatrix}
                            \begin{pmatrix} 32 - 2\cdot4 \\ 10 - k \end{pmatrix} =
              \sum^{4}_{k=1}\begin{pmatrix} 4  \\ k \end{pmatrix}
                            \begin{pmatrix} 24 \\ 10 - k \end{pmatrix},
            \end{equation*}
      \item jedna dáma libovolné barvy, \(k\) kluků libovolné barvy pro \(k = 1,2, 3, 4\) (alespoň 
            jeden kluk), \(10 - k - 1\) karet, které nemají hodnotu dámy ani kluka, celkem
            \begin{equation*}
              \begin{pmatrix} 4  \\ 1 \end{pmatrix}
              \sum^{4}_{k=1}\begin{pmatrix} 4  \\ k \end{pmatrix}
                            \begin{pmatrix} 32 - 2\cdot4 \\ 10 - k - 1 \end{pmatrix} =
              4\sum^{4}_{k=1}\begin{pmatrix} 4  \\ k \end{pmatrix}
                            \begin{pmatrix} 24 \\ 9 - k \end{pmatrix},
            \end{equation*}
      \item dvě dámy libovolné barvy, \(k\) kluků libovolné barvy pro \(k = 1, 2, 3, 4\) (alespoň 
            jeden kluk), \(10 - 2 - k = 8 - k\) karet, které nemají hodnotu dámy ani kluka, celkem
            \begin{equation*}
              \begin{pmatrix} 4  \\ 2 \end{pmatrix}
              \sum^{4}_{k=1}\begin{pmatrix} 4  \\ k \end{pmatrix}
                            \begin{pmatrix} 32 - 2\cdot4 \\ 10 - k - 2 \end{pmatrix} =
              6\sum^{4}_{k=1}\begin{pmatrix} 4  \\ k \end{pmatrix}
                            \begin{pmatrix} 24 \\ 8 - k \end{pmatrix},
            \end{equation*}
      \item tři dámy libovolné barvy, k kluků libovolné barvy pro \(k = 1, 2, 3, 4\) (alespoň jeden 
            kluk), \(10 - k - 3\) karet, které nemají hodnotu dámy ani kluka, celkem
            \begin{equation*}
              \begin{pmatrix} 4  \\ 3 \end{pmatrix}
              \sum^{4}_{k=1}\begin{pmatrix} 4  \\ k \end{pmatrix}
                            \begin{pmatrix} 32 - 2\cdot4 \\ 10 - k - 3 \end{pmatrix} =
              4\sum^{4}_{k=1}\begin{pmatrix} 4  \\ k \end{pmatrix}
                            \begin{pmatrix} 24 \\ 7 - k \end{pmatrix},
            \end{equation*}
      \item všechny čtyři dámy, k kluků libovolné barvy pro \(k = 1, 2, 3, 4\) (alespoň jeden 
            kluk), \(10 - k - 4\) karet, které nemají hodnotu dámy ani kluka, celkem
            \begin{equation*}
              \begin{pmatrix} 4  \\ 4 \end{pmatrix}
              \sum^{4}_{k=1}\begin{pmatrix} 4  \\ k \end{pmatrix}
                            \begin{pmatrix} 32 - 2\cdot4 \\ 10 - k - 4 \end{pmatrix} =
              \sum^{4}_{k=1}\begin{pmatrix} 4  \\ k \end{pmatrix}
                            \begin{pmatrix} 24 \\ 6 - k \end{pmatrix},
            \end{equation*}
    \end{enumerate}
    
    Počet příznivých případů \(M(F)\) je dán součtem všech těchto možností, tedy
    \begin{equation*}
      M(F) = \begin{pmatrix}  4 \\ 2 \end{pmatrix}\begin{pmatrix} 4  \\ 0 \end{pmatrix}
             \begin{pmatrix} 24 \\ 8 \end{pmatrix} + 
              \sum^{4}_{s=0}\begin{pmatrix} 4  \\ s \end{pmatrix}
              \left[\sum^{4}_{k=1}\begin{pmatrix}  4 \\ k \end{pmatrix}
                    \begin{pmatrix} 24 \\ 10 - k - s \end{pmatrix}
              \right].
    \end{equation*}
    Všimněme si nyní výsledku. Výraz s dvojitou sumou můžeme přepsat jak
    \begin{equation*}
      \sum^{4}_{k=1}\begin{pmatrix} 4  \\ k \end{pmatrix}
        \left[\sum^{4}_{s=0}\begin{pmatrix}  4 \\ s \end{pmatrix}
              \begin{pmatrix} 24 \\ 10 - k - s \end{pmatrix}
        \right].
    \end{equation*}
    V učebnicích můžeme najít různé vzorce pro kombinační čísla, mezi nimi i vzorec
    \begin{equation*}
      \sum^{p}_{s=0}\begin{pmatrix} p \\ s \end{pmatrix}\begin{pmatrix} r \\ q - s \end{pmatrix}
        = \begin{pmatrix} r + p \\ q \end{pmatrix} \qquad\text{pro}\qquad r\geq q,\,q \geq p.
    \end{equation*}
    (Nebo si jej můžeme sami odvodit — pokuste se o to!) Pro \(p = 4\), \(r = 24\), \(q = 10 - k\), 
    \(1 \leq k \leq 4\) máme právě náš případ, takže
    \begin{equation*}
      \sum^{4}_{k=1}\begin{pmatrix} 4  \\ k \end{pmatrix}
        \left[\sum^{4}_{s=0}\begin{pmatrix}  4 \\ s \end{pmatrix}
              \begin{pmatrix} 24 \\ 10 - k - s \end{pmatrix}
        \right] = 
        \sum^{4}_{k=1}\begin{pmatrix}  4 \\ k \end{pmatrix}
                      \begin{pmatrix} 28 \\ 10 - k \end{pmatrix}.
    \end{equation*}
    Jak můžeme tento výsledek interpretovat? Jedná se o počet případů, kdy náhodný výběr deseti 
    karet z mariášové hry dvaatřiceti karet obsahuje alespoň jednu kartu pevně zvolené hodnoty (v 
    našem případě kluka), bez ohledu na to, kolik obsahuje karet ostatních hodnot. Přidáme-li počet 
    případů, kdy výběr neobsahuje žádného kluka a právě dvě dámy, dostaneme právě počet případů 
    příznivých pro jev \(F\). Pň úpravě použijeme ještě jednou vzorce
    \begin{equation*}
      \sum^{4}_{k=1}\begin{pmatrix} 4 \\ k \end{pmatrix}\begin{pmatrix} 28 \\ 10 - k\end{pmatrix}
        = \sum^{4}_{k=0}\begin{pmatrix}  4 \\ k     \end{pmatrix}
                        \begin{pmatrix} 28 \\ 10 - k\end{pmatrix} -
                        \begin{pmatrix}  4 \\ 0     \end{pmatrix}
                        \begin{pmatrix} 28 \\ 10    \end{pmatrix} =
                        \begin{pmatrix} 32 \\ 10    \end{pmatrix} -
                        \begin{pmatrix} 28 \\ 10    \end{pmatrix},
    \end{equation*}
    \begin{align*}
      M(F) &= \begin{pmatrix}  4 \\ 2 \end{pmatrix}\begin{pmatrix}  4 \\ 0 \end{pmatrix} 
              \begin{pmatrix} 24 \\ 8 \end{pmatrix} + 
              \sum^{4}_{k=1}\begin{pmatrix} 4  \\ k \end{pmatrix}
                      \left[\sum^{4}_{s=0}\begin{pmatrix}  4 \\ s \end{pmatrix}
                            \begin{pmatrix} 24 \\ 10 - k - s \end{pmatrix}
                      \right]                                                                   \\
           &= \begin{pmatrix}  4 \\ 2 \end{pmatrix}\begin{pmatrix} 24 \\ 8 \end{pmatrix} +
              \sum^{4}_{k=1}\begin{pmatrix}  4 \\ k \end{pmatrix}
                            \begin{pmatrix} 28 \\ 10 - k \end{pmatrix}                          \\
           &= \begin{pmatrix}  4 \\ 2 \end{pmatrix}\begin{pmatrix} 24 \\ 8 \end{pmatrix} + 
              \left[\begin{pmatrix}  32 \\ 10 \end{pmatrix} - 
                    \begin{pmatrix}  28 \\ 10 \end{pmatrix}
              \right]
            = \begin{pmatrix} 32 \\ 10 \end{pmatrix} - 
              \left[\begin{pmatrix} 28 \\ 10 \end{pmatrix} - 
                    \begin{pmatrix}  4 \\  2 \end{pmatrix}
                    \begin{pmatrix} 24 \\  8 \end{pmatrix}
              \right],                                                                          \\
      p(F) &= 1 - \dfrac{\begin{pmatrix} 28 \\ 10 \end{pmatrix} -
                         \begin{pmatrix}  4 \\  2 \end{pmatrix}
                         \begin{pmatrix} 24 \\  8 \end{pmatrix}
                        }
                        {\begin{pmatrix} 32 \\ 10 \end{pmatrix}}
            = 1 - \dfrac{\num{8710284}}{\num{64512240}} \simeq \num{0.86}.
    \end{align*}
    Zamysleme se ještě nad interpretací posledního výrazu pro \(M(F)\). Od počtu všech možných 
    případů se odečítá hodnota \(\begin{pmatrix} 28 \\ 10 \end{pmatrix}\) představující počet 
    situací, kdy ve výběru nebude žádný kluk, zmenšená o hodnotu \(\begin{pmatrix} 4 \\ 2 
    \end{pmatrix}\begin{pmatrix} 24 \\ 8 \end{pmatrix}\), která představuje počet situací, kdy ve 
    výběru budou právě dvě dámy a žádný kluk.
  \normalsize
\end{example}
      %---------------------------------------------------------------

      %--Sestavování čísel z cifer------------------------------------
      % !TeX spellcheck = cs_CZ
\wikitextrule
\begin{example}\label{mai:exam054}
  \textbf{Sestavování čísel z cifer}\newline\small
   Máme k dispozici libovolný počet cifer \(0, 1, \ldots, 9\). Kolik \(k\)-ciferných čísel z nich 
   můžeme sestavit? Odpověď na tuto otázku každý zná. Dvojciferná jsou čísla od \num{10} do 
   \num{99} včetně, je jich tedy \((99 - 10 + 1) = 90\). Trojciferná jsou od \(100\) do \(999\) 
   včetně, jejich počet je \((999 - 100 + 1) = 900\), \(k\)-ciferná jsou čísla od \(100\ldots0 = 
   1\cdot10^{k-1}\) do \(999\ldots9\) včetně (\(k\) devítek), jejich počet je \(9\cdot10^{k-1}\). 
   Tento výsledek bychom však měli získat i kombinatorickými úvahami. Čísla totiž dostáváme tak, že 
   z deseti cifer \(0, 1,\ldots, 9\) vytváříme variace \(k\)-té třídy s opakováním, musíme však 
   vyjmout ty možnosti, které začínají nulami. Dostáváme
   \begin{align*}
     10^k - 
       \left(
         9\cdot10^{k-2} + 9\cdot10^{k-3} + \cdots + 9\cdot10^1 + + 9\cdot10^0 + 1 
       \right)
        &= 10^k - 9\cdot\dfrac{10^{k-1} - 1}{10 - 1} - 1     \\
        &= 10^k - 10^{k-1} = 9\cdot10^{k-1}
   \end{align*}
   Jak jsme dostali odečítaný výraz v závorce? Hodnota \(9\cdot10^{k-2}\) představuje počet těch 
   výběrů cifer (s opakováním), které mají na první pozici pevnou nulu, na druhé pozici kteroukoli 
   nenulovou cifru (\num{9} možností) a na dalších \((k - 2)\) pozicích kteroukoli cifru 
   (\(10^{k-2}\) možností). Hodnota \(9\cdot10^{k-3}\) je počet těch výběrů cifer (s opakováním), 
   které mají na prvních dvou pozicích pevné nuly, na třetí pozici kteroukoli nenulovou cifru 
   (\num{9} možností) a na dalších \((k - 3)\) pozicích kteroukoli cifru (\(10^{k-2}\) možností). A 
   tak dále. Nakonec odečítáme ještě jedničku, která reprezentuje jediný výběr \(k\) cifer tvořený 
   samými nulami. Kdybychom se nyní zeptali, jaká je pravděpodobnost, že při náhodném výběru ze 
   souboru jednociferných až \(n\)-ciferných čísel vylosujeme třeba \(k\)-ciferné číslo, odpovíme 
   si již snadno: Počet případů možných je
   \begin{equation*}
     N(n) =9 + 90 + \cdots + 9\cdot10^{n-1} = 9\dfrac{10^n-1}{10 - 1} = 10^n - 1
   \end{equation*}
   počet případů příznivých je  \(M(n,k) = 9\cdot10^{k-1}\). Hledaná pravděpodobnost je tedy
   \begin{equation*}
     p(n,k) = \dfrac{9\cdot10^{k-1}}{10^n-1}.
   \end{equation*}
   Zkontrolujme si platnost získaného vzorce pro jednoduché případy, kdy ji snadno určíme přímo. 
   Pro \(n = 1\) a \(k = 1\) je vylosování jednociferného čísla jevem jistým. A skutečně, náš 
   vzorec dává
   \begin{equation*}
     p(1,1) = \dfrac{9\cdot10^0}{10^1-1} = 1.
   \end{equation*}
   Pro \(n = 2\) máme celkem \num{99} jednociferných a dvojciferných čísel, z nich jednociferných 
   je devět a dvojciferných \num{90}. Pravděpodobnost vylosování jednociferného čísla by tedy měla 
   vyjít \(9/99=1/11\) a pravděpodobnost vylosování čísla dvojciferného \(90/99=10/11\). Z našeho 
   obecného vzorce dostáváme
   \begin{equation*}
     p(2,1) = \dfrac{9\cdot10^0}{10^2-1} = \frac{9}{99} = \frac{1}{11}. \qquad
     p(2,2) = \dfrac{9\cdot10^1}{10^2-1} = \frac{90}{99} = \frac{10}{11}.
   \end{equation*}
   Jistým jevem je, že vylosujeme nějaké číslo. Skutečně také
   \begin{equation*}
     \sum_{k=1}^{n}p(n,k) = \dfrac{9}{10^n - 1}\dfrac{10^n - 1}{10 - 1} =1.
   \end{equation*}
  \normalsize
\end{example}
      %---------------------------------------------------------------

    \subsection{Sčítání a násobení - základní počty s pravděpodobnostmi}\label{mai:IchapIVsecIIssecIII}
      Někdy je třeba určit pravděpodobnosti jevů, které jsou nějakým způsobem „složeny“ z jevů
      jednodušších. Uvažujme například o jevech \(A\) a \(B\), jejichž pravděpodobnosti známe a 
      označíme je \(p(A)\) a \(p(B)\). Definujme nové jevy \(C\) a \(D\) takto:
      \begin{equation*}
        C = A \text{ a } B, \qquad D = A \text{ nebo } B.
      \end{equation*}
      Vzniká přirozená otázka, zda můžeme na základě znalosti pravděpodobností \(p(A)\) a \(p(B)\) 
      určit pravděpodobnosti \(p(C)\) a \(p(D)\). Ukazuje se, že za jistých předpokladů ano. Jako 
      obvykle nám napoví příklady.

      %--Hody kostkou a mincí - jev \(C\)-----------------------------
      % !TeX spellcheck = cs_CZ
\wikitextrule
\begin{example}\label{mai:exam055}
  \textbf{Hody kostkou a mincí - jev \(C\)}\newline\small
  Označme jako jev \(A\) „Při náhodném hodu kostkou padne šestka.“ a jako jev \(B\) \uv{Při	
  náhodném hodu mincí padne hlava.} Platí
  \begin{alignat*}{3}
    N(A) &= 6\qquad   M(A) &&=1 \qquad \Rightarrow \qquad p(A) &&= \frac{1}{6}  \\
    N(B) &= 2\qquad   M(B) &&=1 \qquad \Rightarrow \qquad p(B) &&= \frac{1}{2}  \\
  \end{alignat*}
  Jev \(C\) je definován jako \(A\) \textbf{a} \(B\), tj. „Při náhodném provedení současného hodu 
  kostkou a mincí padne na kostce šestka a na minci hlava.“ Počítejme pravděpodobnost \(p(C)\). 
  Jevy \(A\) a \(B\) jsou \textbf{nezávislé}, to znamená, že výsledek hodu kostkou neovlivní 
  výsledek hodu mincí a naopak. Počet možných výsledků současného hodu kostkou a mincí je
  \begin{equation*}
    N(C) = N(A \text{ a } B) = N(A)N(B) = 12.
  \end{equation*}
  Každý možný výsledek hodu kostkou je totiž možno kombinovat s každým možným výsledkem hodu mincí.
  Označme výsledky hodu mincí jako \(\mathcal{A}\) (hlava neboli avers) a opačný výsledek jako 
  \(\mathcal{R}\). (orel neboli revers). Výčet možných výsledků současného hodu kostkou a mincí je

  \begin{table}[h]
    \centering
    \begin{tabular}{c|rrrrrrrrrrrr}
      \textbf{kostka} & 1 & 2 & 3 & 4 & 5 & 6 & 1 & 2 & 3 & 4 & 5 & 6 \\ \hline
      \textbf{mince}  & \(\mathcal{A}\) & \(\mathcal{A}\) & \(\mathcal{A}\) & \(\mathcal{A}\) & 
                        \(\mathcal{A}\) & \(\mathcal{A}\) & \(\mathcal{R}\) & \(\mathcal{R}\) & 
                        \(\mathcal{R}\) & \(\mathcal{R}\) & \(\mathcal{R}\) & \(\mathcal{R}\) 
    \end{tabular}
    % \caption{ }
  \end{table}
  
  Příznivý případ je pouze jeden, tj. situace, kdy se výsledek \num{6} na kostce kombinuje s 
  výsledkem \(\mathcal{A}\) na minci
  
  \begin{equation*}
    M(C) = M(A)M(B) = 1.
  \end{equation*}
  \begin{equation*}
    p(C) = \dfrac{M(C)}{N(C)} = \dfrac{M(A)M(B)}{N(A)N(B)} 
         = \dfrac{M(A)}{N(A)}\cdot\dfrac{M(B)}{N(B)} = p(A)p(B).
  \end{equation*}
  \normalsize
\end{example}
      %---------------------------------------------------------------
      
      Z příkladu intuitivně chápeme, co jsou to nezávislé jevy, a usuzujeme, že obecně platí
      \begin{mathlemma}{Násobení pravděpodobností}{mai:lemma003}
        Pravděpodobnost současného výskytu dvou nezávislých jevů \(A\) a \(B\) (jev \(C\)) je rovna
        součinu jejich pravděpodobností, tj.
        \begin{equation}\label{mai:eq052}
          p(A \text{ a } B)= p(A)p(B) 
        \end{equation}
        pro \(A\) a \(B\) neslučitelné.
      \end{mathlemma}
      
      Pokusme se o přesnější definici nezávislých jevů a o odvození vztahu (\ref{mai:eq052}). 
      Označme jako \(N_A\) množinu všech možných výsledků pokusu, při němž může nastat jev \(A\), a 
      obdobně \(N_B\) množinu všech možných výsledků pokusu, při němž může nastat jev \(B\). V 
      předchozím příkladu je \(N_A = \{1, 2, 3, 4, 5, 6\}\) a \(N_B = \{\mathcal{A}, 
      \mathcal{B}\}\). Jako \(N_C\) označme množinu všech možných výsledků pokusu, při němž může 
      nastat současně jev \(A\) i jev \(B\). Jevy \(A\) a \(B\) nazveme nezávislé, jestliže
      platí \(N_C = N_A \times N_B\) (kartézský součin množin). Označíme-li obdobně \(M_A \subseteq 
      N_A\) a, \(M_B \subseteq N_B\) a \(M_C \subseteq N_C\) podmnožiny příznivých výsledků pro 
      jednotlivé jevy, je zřejmé, že také \(M_C = M_A \times M_B\). Počty prvků jednotlivých množin 
      označíme \(N(A)\), \(N(B)\), \(N(C)\) (počty možných případů) a \(M(A)\), \(M(B)\), \(M(C)\) 
      (počty příznivých případů). O konečných množinách víme, že mohutnost (počet prvků) 
      kartézského součinu množin je rovna součinu mohutností jednotlivých
      činitelů v tomto kartézském součinu. Proto
      \begin{align*}
        N(C) &= N(A)N(B),\qquad M(C) = M(A)M(B). \\
        \shortintertext{Odtud}
        p(C) &= \dfrac{M(C)}{N(C)} = \dfrac{M(A)M(B)}{N(A)N(B)} = p(A)p(B).
      \end{align*}
      Platnost tohoto vzorce lze zobecnit na nezávislé jevy \(A_1\), \(A_2\) až \(A_k\) s 
      pravděpodobnostmi \(p(A_1)\), \(p(A_2)\) až \(p(A_k)\). Pravděpodobnost jevu \(C = (A_1\text{ 
      a }A_2\text{ a }...\text{ a }A_K)\) pak je
      \begin{equation*}
        p(C) = p(A_1)p(A_2)\cdots p(A_k).
      \end{equation*}
 
      %--Hody kostkou trochu jinak - jev \(D\)-----------------------
      % !TeX spellcheck = cs_CZ
\begin{mdframed}[style=mdexam]
  \begin{example}\label{mai:exam056}
    \textbf{Hody kostkou trochu jinak - jev \(D\)}\newline
    Označme nyní jako jev \(A\) „Při hodu kostkou padne šestka.“ a jako jev \(B\) „Při hodu kostkou 
    padne pětka.“ Jev \(D\) nechť je definován jako \(D = (A\text{ nebo }B)\), tj. „Při hodu kostkou 
    padne šestka nebo pětka.“ Pravděpodobnosti jevů \(A\) a \(B\) jsou \(p(A) = p(B) = 1/6\). Jevy 
    \(A\) a \(B\) jsou přitom \textbf{neslučitelné} (též vylučující se ). Nemůže
    totiž padnout pětka a šestka současně. Platí
    \begin{align*}
      N(A) &= N(B) = N(D) = N = 6                                                              \\
      M(A) &= M(B) = 1                                                                         \\
      M(D) &= M(A) + M(B) = 2,                                                                 \\
      p(D) &= \dfrac{M(D)}{N(D)} = \dfrac{M(A) + M(B)}{N}                                      \\
           &= \dfrac{M(A)}{N} + \dfrac{M(B)}{N}                                                \\
           &= p(A) + p(B) = \dfrac{2}{6} = \dfrac{1}{3}.
    \end{align*}
  \end{example}
\end{mdframed}
      %--------------------------------------------------------------
      
      Je vidět, že opět směřujeme k obecnému tvrzení:
      \begin{mathlemma}{Sčítání pravděpodobností)}{lemma004}
        \textbf{(}: Pravděpodobnost jevů \(A\) nebo \(B\),
        (jev \(C\)) pro neslučitelné (vylučující se) jevy \(A\) a \(B\) rovna součtu
        pravděpodobností jevů \(A\) nebo \(B\).
        \begin{equation}\label{mai:eq053}
          p(A \text{ nebo } B)= p(A) + p(B)
        \end{equation}
        pro \(A\) a \(B\) nezávislé.
      \end{mathlemma}
      
      Opět se pokusme o přesnější definici neslučitelných jevů a o odvození vztahu
      (\ref{mai:eq053}). Označme, obdobně jako v předchozí úvaze o nezávislých jevech, množiny
      \(N_A\), \(N_B\), \(N_D\) možných výsledků, při nichž mohou nastat jevy \(A\), \(B\), \(D\).
      Předpokládejme, že \(N_A = N_B\). Pak \(N_A = N_B = N_D\), a tedy \(N(A) = N(B) = N(D) = N\).
      Jako \(M_A\), resp. \(M_B\), resp. \(M_D\) označme podmnožiny výsledků, při nichž nastane jev
      \(A\), resp. \(B\), resp. \(D\). Zřejmě \(M_D = M_A \cup M_B\). Pro počet prvků množiny
      \(M_D\) platí 
      \begin{align*}
        M(D) &= M(A) + M(B) - M(A\text{ a }B).                                    \\
        \shortintertext{Pravděpodobnost jevu \(D\) je pak}
        p(D) &= \dfrac{M(D)}{N(D)} = \dfrac{M(A) + M(B) - M(A\text{ a }B)}{N}     \\
             &= p(A) + p(B) - p(A\text{ a }B).
      \end{align*}
      
      Jevy \(A\) a \(B\) se nazývají \textbf{neslučitelné}, neboli \emph{vylučující se}, je-li 
      \(M_A \cap M_B = 0\). V takovém případě je ovšem \(M(A\text{ a }B) = 0\), a tedy
      \begin{equation*}
        p(A\text{ nebo }B) = p(A) + p(B).
      \end{equation*}
      Zobecněním na \(k\) jevů \(A_1\) až \(A_k\) po dvou neslučitelných dostáváme
      \begin{equation*}
        p(A_1\cdots\text{ nebo }\cdots A_k) = p(A_2) + p(A_2) + \cdots + p(A_k).
      \end{equation*}
      Mají-li po dvou neslučitelné jevy \(A_1\) až \(A_k\) tu vlastnost, že při daném pokusu musí 
      nastat právě jeden z nich, říkáme, že tvoří \textbf{úplný systém jevů}. Součet jejich 
      pravděpodobností je roven jedné.
      
      Jev \(\overline{A}\) se nazývá \textbf{opačný} k jevu \(A\), jestliže nastává právě tehdy, 
      když jev \(A\) nenastává. Z této definice je vidět, že jevy \(\overline{A}\) a \(A\) jsou 
      \emph{neslučitelné}. Na druhé straně je zřejmé, že jev (\(A\) nebo \(\overline{A}\)) je jevem 
      \textbf{jistým}, nastává vždy. Jeho pravděpodobnost je tedy \num{1}. Odtud
      \begin{equation}\label{mai:eq054}
        p(\overline{A}) = 1 - p(A).
      \end{equation}
      Jev \(A\) a jev \(\overline{A}\) k němu opačný tvoří úplný systém.
      
      Na závěr odstavce ještě jeden prakticky důležitý případ.
      
    \subsection{Bernoulliův pokus}\label{mai:IchapIVsecIIssecIV}
      Bernoulliův pokus spočívá v tom, že \(n\)-krát nezávisle provedeme určitý pokus, například hod
      mincí. (V terminologii teorie pravděpodobnosti nazýváme každé takové provedení opakováním
      pokusu.) Sledujeme, v kolika případech z těchto \(n\) opakování nastal daný jev (například jev
      \(A\) - padne hlava). Výsledek opakování pokusu, při kterém daný jev nastal, nazveme zdarem,
      výsledek, kdy nastal jev opačný, nezdarem. Dejme tomu, že pravděpodobnost zdaru je \(p\). (Pro
      případ padnutí hlavy na minci je \(p = 1/2\).) Pravděpodobnost nezdaru je pak \((l - p)\). (V
      případě hodů mincí je \((1 - p) = 1/2\).) Zajímáme se o to, jaká je pravděpodobnost \(P(x)\),
      že při \(n\) opakováních pokusu docílíme \(x\)-krát zdaru, \(x\) přitom můžeme předem volit
      libovolně v rozmezí \(0 \leq x \leq n\). V případě hodů mincí jistě dokážeme předem odhadnout,
      že pravděpodobnosti \(P(0)\) a \(P(n)\), tj. pravděpodobnosti toho, že nepadne hlava vůbec
      nebo že padne hlava vždy, budou při větším počtu opakování pokusu malé a budou se blížit nule
      tím více, čím větší bude \(n\). Naopak bychom se mohli domnívat, že pravděpodobnost
      \(P(n/2)\), tj. že padne hlava v polovině opakování pokusu, by měla být při velkém počtu \(n\)
      blízká \SI{100}{\percent}. Správnost tohoto našeho předběžného odhadu však posoudíme teprve
      poté, co si odvodíme obecný vzorec pro \(P(x)\). Budeme možná překvapeni. Zvolme nejprve
      pevně, při kterých konkrétních opakováních pokusu má dojít ke zdaru  (například při prvních
      \(x\)). Při ostatních pak požadujeme nezdar. Protože jevy
      \begin{gather*}
        \begin{array}{ll}
          A_1                &: \text{Při prvním opakování dojde ke zdaru.}                  \\
          A_2                &: \text{Při druhém opakování dojde ke zdaru.}                  \\
          \hdotsfor{2}                                                                       \\
          A_x                &: \text{Při \(x\)-tém opakování dojde ke zdaru.}               \\
          \overline{A}_{x+1} &: \text{Při \((x + 1)\)-tém opakování dojde k nezdaru.}        \\
          \hdotsfor{2}                                                                       \\
          \overline{A}_n     &: \text{Při posledním \(n\)-tém opakování dojde k nezdaru,}
        \end{array}
      \end{gather*}
      jsou nezávislé, je pravděpodobnost jevu
      \begin{itemize}
        \item \(B_1\): Při každém z prvních \(x\) opakování dojde ke zdaru a současně při každém z
              dalších \((n - x)\) opakování dojde k nezdaru, rovna součinu pravděpodobností
              \begin{align*}
                p(B_1) &= p(A_1)p(A_2)\cdots p(A_x)p(\overline{A}_{x+1})\cdots p(\overline{A}_n) \\
                       &= p^x (1 - p)^{n-x}.
              \end{align*}
      \end{itemize}
      Nám však jde o pravděpodobnost následujícího jevu
      \begin{itemize}
        \item \(B\): Právě při \(x\) opakováních pokusu (bez ohledu na to, kterých) dojde ke zdaru a
              současně při každém ze zbývajících opakování pokusu dojde k nezdaru.
      \end{itemize}
      
      Možností výběru \(x\) opakování, při kterých dojde ke zdaru, je \(N(x) = \binom{n}{x}\). Pokud
      bychom očíslovali jednotlivé výběry \(j = 1, 2, \cdots, N(x)\), dostaneme odpovídající jevy
      \(B_1, \cdots, B_{N(x)}\) Pravděpodobnost každého z nich je stejná a rovna pravděpodobnosti
      jevu \(B_1\), který jsme popsali před chvílí. Tyto jevy jsou po dvou neslučitelné a jev \(B\)
      znamená, že nastane právě jeden (kterýkoli) z nich. Pro jeho pravděpodobnost tedy platí, podle
      pravidla pro součet pravděpodobností po dvou neslučitelných jevů, 
      \begin{mdframed}[style=highlight]
        \begin{equation}\label{mai:eq055}
          p(B) = P(x) = \binom{n}{x}p^x (1 - p)^{n-x}.
        \end{equation}
      \end{mdframed}
      
      Zkusme nyní prověřit správnost našeho odhadu týkajícího se hodů mincí:
      \begin{align*}
        P(0) &= \binom{n}{0}
                \left(\dfrac{1}{2}\right)^0\left(\dfrac{1}{2}\right)^{n-0} 
              = \dfrac{1}{2^n}                                                 \\
        P(n) &= \binom{n}{n} 
                \left(\dfrac{1}{2}\right)^n\left(\dfrac{1}{2}\right)^{n-n} 
              = \dfrac{1}{2^n}     
      \end{align*}
      Vidíme, že náš odhad byl správný. Obě pravděpodobnosti klesají s rostoucím počtem opakování
      pokusu k nule.  Pro jediné opakování pokusu, tj. \(n = 1\), jsou obě rovny jedné polovině, a
      to bychom jistě také měli očekávat. 
      
      Pro \(n\) sudé nyní počítejme \(P(n/2)\). Položme \(n = 2m\):
      \begin{align*}
        P(m) &= \binom{2m}{m}
                \left(\dfrac{1}{2}\right)^m\left(\dfrac{1}{2}\right)^{2m-m}                       \\
             &= \dfrac{(2m)!}{m!m!}\left(\dfrac{1}{2}\right)^{2m}     
      \end{align*}
  
      {\centering
       \resizebox{1\linewidth}{!}{
        \begin{tabular}{c|rrrrr}
          \(m\)    & 1 & 2 & 3 & 5 & 10  \\ \hline
          \(P(m)\) & \num{0.500} & \num{0.375} & \num{0.313} & \num{0.246} & \num{0.176}
        \end{tabular}
       }
      \par}
     
      Tady se zdá, že nás naše intuice při odhadu pravděpodobnosti \(P(n/2)\) zklamala. Tendence
      hodnot \(P(n/2)\) je pro rostoucí \(n\) klesající. Pravděpodobnost je největší pro \(n = 2\),
      a to právě padesátiprocentní! Zkusme ještě odhad pro velká \(n\) pomocí \textbf{Stirlingova
      vzorce}. Podle něj pro velká \(n\) platí    
      \begin{equation}\label{mai:eq056}
        \boxed{n! \doteq \left(\dfrac{n}{e}\right)^n\sqrt{2\pi n}}
      \end{equation}
      Použijeme-li jej pro výpočet P(m), dostáváme
      \begin{align*}
        P(m)&\doteq \dfrac{\left(\dfrac{2m}{e}\right)^{2m}\sqrt{4\pi m}}
             {\left(\dfrac{m}{e}\right)^m\left(\dfrac{m}{e}\right)^m\left(\sqrt{2\pi m}\right)^2}
              \left(\dfrac{1}{2}\right)^{2m}                                       \\
            &= \dfrac{\sqrt{4\pi m}}{2\pi m}
         =\dfrac{1}{\sqrt{\pi m}} \longrightarrow 0
      \end{align*}
      pro velká \(m\). Kde jsme se tedy zmýlili? Ze zkušenosti víme, že budeme-li házet mincí
      mnohokrát, je prakticky jisté, že hlava skutečně padne zhruba v polovině případů! Problém
      spočívá ve slovíčku zhruba. Pravděpodobnost \(P(m)\) pro \(n = 2m\) se však týká jevu, kdy
      hlava padne přesně v polovině případů. A ta samozřejmě bude tím menší, čím větší je počet
      posuzovaných hodů mincí. Při zvyšujícím se počtu \(n\) opakování pokusu totiž roste i počet
      jednotlivých možností volby \(x\) a \(n\) a každou z nich tak „připadne“ menší
      pravděpodobnost. (Součet pravděpodobností přes všechna přípustná \(x\) musí být roven jedné.)
      Později, v odstavci \ref{mai:IchapIVsecII}, uvidíme, že jsme \textbf{nevědomky místo
      pravděpodobnosti odhadovali střední hodnotu náhodné veličiny}.
      
      Položme si ještě poslední otázku v souvislosti s Bernoulliovým pokusem: Jaká je
      pravděpodobnost, že alespoň při jednom z \(n\) opakování pokusu nastane zdar? Pokud si po
      předchozím neúspěchu s intuitivními odhady ještě trochu věříme, můžeme předpovídat, že tato
      pravděpodobnost poroste s počtem opakování pokusu \(n\) a pro velmi velká \(n\) se bude blížit
      jedné. Musíme ji ale spočítat. Někdo, kdo nečetl předchozí text příliš pečlivě, by mohl
      navrhnout jednoduchou úvahu: Pravděpodobnost zdaru při každém opakování pokusu je \(p\),
      pravděpodobnost, že nastane zdar při alespoň jednom z nich tedy musí být, podle pravidla pro
      sčítání pravděpodobností, \(np\). Úvaha je sice jednoduchá, ale zcela chybná. Vidíme to již ze
      skutečnosti, že při pevné hodnotě \(p\) a dostatečně velkém \(n\) může hodnota \(np\)
      překročit jedničku, a to nemůže žádná pravděpodobnost udělat. Kde se málo pozorný čtenář
      dopustil chyby, když chtěl sčítat pravděpodobnosti zdaru při jednotlivých opakováních?
      Neuvědomil si, že pravidlo součtu pravděpodobností jednotlivých jevů \(A_1\) až \(A_k\) při
      výpočtu pravděpodobnosti jevu (\(A_1\) nebo \(A_2\) nebo \(\cdots\) nebo \(A_k\)) může použít
      jedině pro jevy po dvou neslučitelné. Zdar při některém z opakování pokusu však nevylučuje
      možnost zdaru při jiném pokusu. Pravidlo tedy bylo použito nesprávně. Pravděpodobnost zdaru
      při alespoň jednom opakování pokusu snadno vypočteme pomocí jevu opačného. Opačný jev znamená,
      že nenastane zdar ani při jednom opakování pokusu. Jednotlivá opakování jsou nezávislá, proto
      je pravděpodobnost nezdarů při všech opakováních rovna součinu pravděpodobností při
      jednotlivých z nich, tj. \((1 - p)^n\) . Pravděpodobnost zdaru při alespoň jednom opakováni je
      pak doplňkem do jedničky, tedy \(1 - (1 - p)^n\). Je vidět, že je tím větší, čím je větší
      \(n\), a její limita pro \(n\rightarrow \infty\) je rovna jedné. A to je výsledek, který jsme
      předpověděli.

    \twocolumn[\subsection{Pravděpodobnější, než bychom čekali - podmíněná pravděpodobnost}]
      Kdysi se objevila, jako nepříliš dobrý vtip, úvaha o pravděpodobnosti bomby na palubě letadla:
      Řekněme, že pravděpodobnost, že některý z pasažérů letadla má s sebou bombu, je jedna
      tisícina. Pravděpodobnost, že dva pasažéři nezávisle na sobě budou mít bombu, je pak pouze
      jedna milióntina (\(\num{e-3}\cdot\num{e-3}= \num{e-6}\)). Vezmu-li si tedy s sebou do 
      letadla svou vlastní bombu, kterou ovšem nehodlám uvést do chodu, snížím tím pravděpodobnost 
      druhé bomby na palubě na onu jednu milióntinu. Nezabývejme se nyní tím, že již první úvaha o 
      jedné milióntině je v podstatě nesprávná, i když pro případ, že pravděpodobnost \(p\), že 
      konkrétní pasažér bude mít bombu, je velmi malá, dává správný přibližný výsledek. Klíčová 
      chyba je v úvaze, že snížení pravděpodobnosti bomby na palubě můžeme napomoci vlastní bombou 
      v zavazadle. Tato úvaha nerespektuje totiž \textbf{pojem podmíněné pravděpodobnosti}, který 
      si nyní na příkladu vyložíme.
      
      %--Jak nekoupit zmetek------------------------------------------
      % !TeX spellcheck = cs_CZ
\wikitextrule
\begin{example}\label{mai:exam058}
  \textbf{Jak nekoupit zmetek}\newline\small
    Do finále soutěže o „šmejd roku“ se dostaly dva podniky, „Hvizd, s.r.o.“ a „Svist, a.s.“, které 
    zásobují trh zábavnou pyrotechnikou. První z nich kryje požadavky trhu ze \SI{70}{\percent}, 
    druhý ze zbývajících \SI{30}{\percent}. “Zjistilo se, že \SI{83}{\percent} ze všech výrobků 
    Hvizdu je vadných (nebouchají, když je to třeba, zejména však bouchají, když se to
    nejméně očekává). V případě Svistu je zmetků pouze \SI{63}{\percent}. Porota soutěže rozhodla, 
    že cenu dostane ten z obou podniků, jehož ředitel zodpoví správně následující otázky:
    \begin{enumerate}
     \item Jaká je pravděpodobnost, že náhodně zakoupená rachejtle bude fungovat tak, jak má?
     \item Jaká je pravděpodobnost, že náhodně zakoupená rachejtle, o níž se na obalu píše, že byla 
     vyrobena podnikem Hvizd, není zmetek?
     \item Jaká je pravděpodobnost, že náhodně zakoupená rachejtle, kterou se podařilo úspěšně 
     odpálit, byla vyrobena podnikem Svist?
    \end{enumerate} 
    Postupně jednotlivé úkoly vyřešíme.
    V případě a) posuzujeme pravděpodobnost jevu \(A\): „Náhodně zakoupený výrobek je funkční.“ bez 
    dalších podmínek. Jedná se o nepodmíněnou pravděpodobnost. Dejme tomu, že na trhu je v dané 
    chvíli ke koupi \(n\) rachejtlí. Z nich \SI{70}{\percent}, tj. \(\num{0.7}n\), bylo vyrobeno v 
    Hvizdu a zbytek, \(\num{0.3}n\), ve Svistu. Víme, že \SI{17}{\percent} výrobků Hvizdu je 
    funkčních, v případě Svistu je to \SI{37}{\percent}. Na trhu je tedy v tuto chvíli
    \begin{equation*}
      m = \num{0.17} - \num{0.7}n + \num{0.37}\cdot\num{0.3}n = \num{0.23}n
    \end{equation*}
    funkčních raket. Pravděpodobnost zakoupení funkční rakety je tedy
    \begin{equation*}
      p(A) = \dfrac{m}{n} = \num{0.23}.
    \end{equation*}
    Výsledek lze snadno zobecnit. Označíme-li \(p_1\) pravděpodobnost zmetku ve firmě Hvizd a 
    \(p_2\) pravděpodobnost zmetku ve firmě Svist, je pravděpodobnost funkčního výrobku ve Hvizdu 
    \((1 - p_1)\) a ve Svistu \((1 - p_2)\). Označme \(q\) podíl Hvizdu na celkové produkci. Podíl 
    Svistu je pak \((1 - q)\). Pravděpodobnost koupě funkčního výrobku je
    \begin{equation*}
      p(A) = (1 - p_1)q + (1 - p_2)(l - q) = (1 - p_2) + q(p_2 - p_1).
    \end{equation*}
    
    V úloze b) se již jedná o \textbf{podmíněnou pravděpodobnost}. Koupíme v obchodě raketu, 
    podíváme se na obal a zjistíme, že byla vyrobena ve Hvizdu. S touto dodatečnou informací chceme 
    zjistit pravděpodobnost, že až raketu rozbalíme a odpálíme, bude skutečně fungovat. Označme 
    jako jev \(B\) „Raketa byla vyrobena ve Hvizdu.“ Naším úkolem je tedy zjistit pravděpodobnost 
    jevu \(A\) (raketa bude funkční) za podmínky, že nastal jev \(B\) (byla vyrobena ve Hvizdu). 
    Tuto pravděpodobnost značíme \(p_B(A)\). Víme, že na trhu je \(qn = \num{0.7}n\) raket 
    vyrobených ve Hvizdu. To představuje pro náš další výpočet počet případů možných. Pouze \((1 - 
    p_1) = \num{0.17}\) z nich je však funkčních, počet případů příznivých je tedy \((1 - p_1)qn = 
    \num{0.17}\cdot\num{0.7}n = \num{0.119}n\). Hledaná pravděpodobnost je
    \begin{equation*}
      p_B(A) = \dfrac{(1 - p_1)qn}{qn} = 1 - p_1 = \num{0.17}
    \end{equation*}
    Všimněme si výpočtu podrobněji. \((1 - p_1)q\) představuje pravděpodobnost jevu \((A\text{ a 
    }B)\), že náhodně zakoupená raketa bude funkční a zároveň bude vyrobena ve Hvizdu. Skutečně, na 
    trhu je v dané chvíli \(n\) raket, z nich \(qn\) bylo vyrobeno ve Hvizdu a z těchto \(qn\) 
    výrobků Hvizdu je \((1 - p_1)qn\) funkčních. Proto 
    \begin{equation*}
      p(A\text{ a }B) = \dfrac{(1 - p_1)qn}{n} = (1 - p_1)q = \num{0.119}
    \end{equation*}
    (Divíte se, že tato pravděpodobnost není součinem \(p(A)p(B)\)? Nedivte se, jevy \(A\) a \(B\) 
    nejsou totiž nezávislé!). Vidíme, že platí
    \begin{equation*}
      p(A\text{ a }B) = p(B)p_B(A),
    \end{equation*}
    neboť \(q = p(B)\). Získáváme tedy vztah pro výpočet podmíněné pravděpodobnosti:
    \adjustbox{minipage=[c][32pt][c]{406pt}}{%
      \begin{equation}\label{mai:eq057}
        p_B(A) = \dfrac{p(A\text{ a }B)}{p(B)} \qquad\text{a obdobně}\qquad
        p_A(B) = \dfrac{p(A\text{ a }B)}{p(A)}.
      \end{equation}
      }
    Vzorec (\ref{mai:eq057}) jsme získali pro konkrétní příklad. Abychom byli korektní, odvoďme jej 
    obecně. Označme \(p(A)\) a \(p(B)\) pravděpodobnost jevu \(A\) a pravděpodobnost jevu \(B\), 
    \(p_B(A)\) podmíněnou pravděpodobnost jevu \(A\) za podmínky, že nastal jev \(B\), a \(p_A(B)\) 
    podmíněnou pravděpodobnost jevu \(B\) za podmínky, že nastal jev \(A\). \(p(A\text{ a }B)\) je 
    pravděpodobnost současného nastoupení jevů \(A\) a \(B\). Dejme tomu, že v celkovém počtu \(n\) 
    opakování pokusu nastal jev \(B\) \(s\)-krát. Nechť v \(t\) případech z těch, kdy nastal jev 
    \(A\), nastal také jev \(B\). Pro pravděpodobnosti pak platí
    \begin{align*}
      p(B) &= \dfrac{s}{n}, \qquad p_B(A) = \dfrac{l}{s}, \qquad p(A\text{ a }B) = \dfrac{l}{n}  \\
      p(A\text{ a }B) &= \dfrac{sp_B(A)}{n} = \dfrac{np(B)p_B(A)}{n} = p(B)p_B(A).
    \end{align*}
    Získali jsme tak obecně první část formule (\ref{mai:eq057}). Záměnou jevů \(A\) a \(B\) 
    dostaneme její druhou část. Poznamenejme, že v případě nezávislých jevů \(A\) a \(B\) je 
    samozřejmě \(p_B(A) = p(A)\) a \(p_A(B) = p(B)\). Vzorec (\ref{mai:eq057}) tak přejde v 
    pravidlo pro součin pravděpodobností nezávislých jevů.
    
    Zbývá vyřešit soutěžní úkol c). Koupili jsme raketu a odpálili ji. Fungovala. Jaká je 
    pravděpodobnost, že když se nyní podíváme na obal, zjistíme, že šlo o výrobek Svistu? Můžeme 
    již využít vztahu (\ref{mai:eq057}). Máme totiž počítat pravděpodobnost jevu \(B\) za podmínky, 
    že nastal jev \(A\). (Poznamenejme, že je-li jev \(B\) definován jako „Náhodně koupená raketa 
    pochází z Hvizdu.“, je jev „Náhodně koupená raketa pochází ze Svistu.“ jevem opačným k \(B\).) 
    Platí
    \begin{equation*}
      p_A(\overline{B}) = \dfrac{p(A\text{ a }\overline{B})}{p(A)}
                        = \dfrac{p_{\overline{B}}(A)p(\overline{B})}{p(B)}
                        = \dfrac{(1 - p_2)(1 - q)}{(1 - p_2) + q(p_2 - p_1)}
                        = \dfrac{\num{0.37}\cdot\num{0.3}}{\num{0.37} + \num{0.7}\cdot(\num{-0.2})}
                        \simeq\num{0.483}
    \end{equation*}
    Kdo chce, může snadno dospět k výsledku přímou úvahou: Na trhu je \(n\) raket, z nich funkčních 
    je \(\num{0.17}\cdot\num{0.7}n + \num{0.37} - \num{0.3}n = \num{0.23}n\). Tato hodnota je při 
    výpočtu \(p(A\text{ a }B)\)  počtem případů možných. Počet funkčních raket vyrobených Svistem 
    je \(\num{0.37}\cdot\num{0.3}n = \num{0.111}n\). To je počet případů příznivých. Hledaná 
    pravděpodobnost je tedy podílem
    \begin{equation*}
      p(A\text{ a }\overline{B}) = \dfrac{\num{0.111}n}{\num{0.23}n}\simeq\num{0.483}.
    \end{equation*}
\normalsize
\end{example}
      %---------------------------------------------------------------

      Vzorec (\ref{mai:eq057}) jsme získali pro konkrétní příklad. Odvoďme jej nyní obecně. Označme
      \(p(A)\) a \(p(B)\) pravděpodobnost jevu \(A\) a pravděpodobnost jevu \(B\), \(p_B(A)\)
      podmíněnou pravděpodobnost jevu \(A\) za podmínky, že nastal jev \(B\), a \(p_A(B)\)
      podmíněnou pravděpodobnost jevu \(B\) za podmínky, že nastal jev \(A\). \(p(A\text{ a }B)\) je
      pravděpodobnost současného nastoupení jevů \(A\) a \(B\). Dejme tomu, že v celkovém počtu
      \(n\) opakování pokusu nastal jev \(B\) \(s\)-krát. Nechť v \(\ell\) případech z těch, kdy
      nastal jev \(A\), nastal také jev \(B\). Pro pravděpodobnosti pak platí
      \begin{align*}
        p(B)            &= \dfrac{s}{n}, \quad p_B(A) = \dfrac{\ell}{s}, 
                           \quad p(A\text{ a }B) = \dfrac{\ell}{n}                \\
        p(A\text{ a }B) &= \dfrac{sp_B(A)}{n}    = \dfrac{np(B)p_B(A)}{n} = p(B)p_B(A).
      \end{align*}
      Získali jsme tak obecně první část formule (\ref{mai:eq057}). Záměnou jevů \(A\) a \(B\) 
      dostaneme její druhou část. Poznamenejme, že v případě nezávislých jevů \(A\) a \(B\) je 
      samozřejmě \(p_B(A) = p(A)\) a \(p_A(B) = p(B)\). Vzorec (\ref{mai:eq057}) tak přejde v 
      pravidlo pro součin pravděpodobností nezávislých jevů.
            
      Nyní již snadno dokážeme přijít na chybu v úvaze o bombě v letadle, kterou jsme tento 
      odstavec uvedli. Pravděpodobnost další bomby v letadle za podmínky, že jsme tam jednu sami 
      donesli, je podmíněnou pravděpodobností. Proto je rovna podílu pravděpodobnosti, že v letadle 
      budou dvě bomby, a pravděpodobnosti, že tam bude jedna bomba, tj. \(\num{e-6}/\num{e-3} = 
      \num{e-3}\). Pocit bezpečí bychom si tedy vlastní bombou nezvýšili. Ještě abychom se báli, že 
      bouchne, zejména pokud by byla vyrobená ve Hvizdu.
      
      %--Kolika let se dožijeme?--------------------------------------
      % !TeX spellcheck = cs_CZ
\wikitextrule
\begin{example}\label{mai:exam059}
  \textbf{Kolika let se dožijeme?}\newline\small
  V rámci evidence obyvatelstva se často sledují různé údaje, které slouží k odhadům vývoje 
  mohutnosti populace. Dejme tomu, že v jihomoravském regionu zjistili, že ze statisíce dětí, které 
  se dožily pěti let, se v průměru dožije dvaceti let \num{93} tisíc a osmdesáti let \num{36} 
  tisíc. Jaká je pravděpodobnost, že vy, kteří jste se již dvaceti let dožili, se dožijete 
  osmdesátky? Označme jako jev \(A\) „Pětileté dítě se dožije osmdesáti let.“ a jako jev \(B\) 
  „Pětileté dítě se dožije dvaceti let.“ Je zřejmé, že v tomto případě platí \(p(A) = p(A\text{ a 
  }B)\) (jestliže se někdo dožil osmdesáti let, s jistotou se předtím dožil dvaceti let). My 
  posuzujeme pravděpodobnost nastoupení jevu \(A\) za podmínky, že nastal jev \(B\), tj. podmíněnou 
  pravděpodobnost \(p_B(A)\). Platí
  \begin{equation*}
    p(A)   = p(A\text{ a }B) = \num{0.36}, p(B) = \num{0.93}\Rightarrow 
    p_B(A) = \dfrac{p(A\text{ a }B)}{p(B)} = \dfrac{\num{0.36}}{\num{0.93}} \simeq \num{0.39},
  \end{equation*}
  Že ta pravděpodobnost není velká? Nezoufejte. Čísla byla fiktivní a předpovědi říkají, že již v 
  roce \num{2015} bude u nás průměrný věk žen \num{83} let, u mužů, bohužel, o něco nižší. Ještě 
  zpřesněme úvahu, která nás vede k závěru \(p(A\text{ a }B) = p(A)\). Pravděpodobnost nastoupení 
  jevu \(B\) za podmínky, že nastal jev \(A\), je v našem případě rovna jedné. Jak jsme totiž již 
  konstatovali, každý, kdo se dožil osmdesátky, se s jistotou dožil i dvacítky. Platí
  pravděpodobnost \(p_B(A)\). Platí
  \begin{equation*}
    p_A(B) = \dfrac{p(A\text{ a }B)}{p(A)} = 1 \Rightarrow p(A\text{ a }B) = p(A),
  \end{equation*}
\normalsize
\end{example}
      %---------------------------------------------------------------
      
      %--Ještě jednou bomba v letadle---------------------------------
      % !TeX spellcheck = cs_CZ
\begin{mdframed}[style=mdexam]
  \begin{example}\label{mai:exam060}
    \textbf{Ještě jednou bomba v letadle}\newline
    V úvodu odstavce jsme uvažovali o pravděpodobnosti dvou bomb v letadle jako o pravděpodobnosti
    současného nastoupení dvou nezávislých jevů s komentářem, že tato úvaha není tak docela v
    pořádku. Někdo je možná zvědavý, proč, a tak se tomuto problému budeme ještě chvíli věnovat.
    (Kdo zvědavý není, může příklad přeskočit.)
    
    Nebudeme nyní posuzovat situaci, kdy jsme do letadla přinesli bombu my sami. Zabývejme se
    přesnější odpovědí na otázku, jaká je pravděpodobnost, že v letadle budou bomby dvě, aniž bychom
    tomu sami napomáhali. Taková situace odpovídá Bernoulliovu pokusu. Samozřejmě, je třeba udělat
    jisté předpoklady, které nemusejí být zcela realistické, ale v průměru budou fungovat.
    Předpokládejme, že v letadle je \(n\) pasažérů, kteří se nijak neliší pokud jde o sklon „vzít
    bombu do letadla“. Pravděpodobnost, že daný pasažér vezme s sebou bombu, je tedy u všech stejná
    a označme ji \(p\). (To je právě ten předpoklad, který u jednotlivce není příliš realistický,
    neboť venkovská tetička jistě nemá takové nutkání vzít si spolu s husou do košíku bombu, jako
    fanatický terorista.) Hodnota \(p\) zde tedy představuje jistou „zprůměrovanou zkušenost“. Co
    přesně znamená otázka, jaká je pravděpodobnost, že v letadle je bomba? Je tím myšlena
    pravděpodobnost jevu „V letadle je alespoň jedna bomba.“ Pravděpodobnost \(P\) tohoto jevu jsme
    zadali jako jednu tisícinu (dejme tomu, že je to zase údaj odpovídající „zprůměrované zkušenosti
    “u letadel s velkým počtem cestujících). Také jsme již \(P\) počítali v závěru kapitoly
    \ref{mai:IchapIVsecIIssecIV}. Platí pro ni
    \begin{equation*}
      P = 1 - (1 - p)^n,
    \end{equation*}
    kde \(n\) je počet opakování pokusu. V našem případě nastoupení jednotlivého pasažéra do letadla
    představuje jedno opakování pokusu, takže je tento počet roven počtu pasažérů v letadle. Můžeme
    tedy určit pravděpodobnost \(p\) týkající se jednotlivého pasažéra,
    \begin{equation*}
      p = 1 - \sqrt[n]{1 - P}
    \end{equation*}
    Nyní potřebujeme znát pravděpodobnost, že v letadle jsou dvě bomby, myšleno alespoň dvě bomby.
    Označme tento jev jako \(B\). Znamená, že alespoň při dvou opakováních Bernoulliova pokusu
    nastane zdar. Jev \(\overline{B}\) k němu opačný znamená, že nastanou buď samé nezdary
    (pravděpodobnost je \((1 - p)^n\) a odpovídá hodnotě \(x = 0\) ve vzorci (\ref{mai:eq055})),
    nebo nastane právě \((n - 1)\) nezdarů a jeden zdar (pravděpodobnost je \(np(1 - p)^{n-1}\) a
    odpovídá hodnotě \(x = 1\) ve vzorci (\ref{mai:eq055})). Výsledky Bernoulliova pokusu pro různá
    \(x\) se ovšem navzájem vylučují, takže pravděpodobnost jevu \(\overline{B}\) je
    \begin{equation*}
      \boxed{p(\overline{B}) = (1 - p)^n + np(1 - p)^{n-1}}.
    \end{equation*}
    Pravděpodobnost alespoň dvou bomb v letadle (posuzovaný jev \(B\)) je pak
    \begin{equation*}
      p(B) = 1 - (1 - p)^n - np(1 - p)^{n-1}.
    \end{equation*}
    Zbývá dosadit za \(p\) pomocí známé hodnoty \(P\). Je-li \(P\) velmi malé, lze získat přibližný
    výsledek pomocí odhadů. Vzpomeneme-li si na odhady pomocí diferenciálu v odstavci
    \ref{mai:IchapVsecIII}, zjistíme, že pro hodnoty \(P\) mnohonásobně menší než \(1\) (a to je i
    náš případ) dostaneme
    \begin{equation*}
      p \simeq 1 - \left(1 - \dfrac{1}{n}P\right) = \dfrac{P}{n}.
    \end{equation*}
    Obdobně provedeme odhad pro \(p(B)\),
    \begin{align*}
      p(B) &= 1 - (1 - np) - np\left[1 - (n - 1)p\right]     \\
           &= n(n - 1)p^2\simeq \dfrac{n-1}{n}P^2\simeq P^2 = \num{e-6}.
    \end{align*}
  \end{example}
\end{mdframed}
      %---------------------------------------------------------------
      
      Nakonec ještě odvodíme obecný případ takzvané Bayesovy formule. Předpokládejme, že při
      každém opakování jistého pokusu může nastat právě jeden z \(k\) různých výsledků. Jevy \(A_1,
      A_2,\cdots, A_k\), z nichž \(j\)-tý znamená, že při pokusu byl zaznamenán \(j\)-tý výsledek, 
      jsou po dvou neslučitelné a tvoří úplný systém. Platí tedy
      \begin{equation*}
        p(A_1) + p(A_2) + \ldots  + p(A_k) = 1.
      \end{equation*}
      Označme jako \(B\) libovolný jev, který popisuje celkový výsledek pokusu. Vzhledem k 
      neslučitelnosti jevů \(A_1\) až \(A_k\) jsou neslučitelné i jevy (\(A_1\) a \(B\)) až 
      (\(A_k\) a \(B\)). Zároveň je zřejmé, že jev \(B\) lze zapsat jako
     \begin{equation*}
       B = (A_1\text{ a }B)\;\text{nebo}\; (A_2\text{ a }B) \quad\ldots
       \;\text{nebo}\; (A_k\text{ a }B),
     \end{equation*} 
      a tedy
      \begin{equation*}
        p(B) = \sum_{j=1}^{k}p(A_j\text{ a }B) = \sum_{j=1}^{k}p(A_j)\cdot p_{A_j}(B),
      \end{equation*}
      s využitím vztahu (\ref{mai:eq057}). Současně, podle téhož vztahu, platí 
      \begin{equation*}
        p(A_j\text{ a }B) = p(B)\cdot  p_B(A_j).
      \end{equation*}
      Pomocí dvou předchozích vztahů dostáváme:
      
      \begin{lemma}{Bayesova formule}{lemma005}
          \begin{equation}\label{mai:eq058}
            p_B(A_j) = \dfrac{p(A_j\text{ a }B)}{\sum_{i=1}^{k}p(A_i)\cdot p_{A_i}(B)} 
                     = \dfrac{p(A_j)\cdot p_{A_j}(B)}{\sum_{i=1}^{k}p(A_i)\cdot p_{A_i}(B)} .
          \end{equation}
      \end{lemma}
      
      Bayesova formule pro výpočet podmíněné pravděpodobnosti má řadu užitečných aplikací
      
      %--Potřebují lékaři pravděpodobnost?----------------------------
      % !TeX spellcheck = cs_CZ
\begin{mdframed}[style=mdexam]
  \begin{example}\label{mai:exam061}
    \textbf{Potřebují lékaři pravděpodobnost?}\newline
    U pacienta je podezření, že trpí právě jednou ze tří chorob \(A_1\), \(A_2\) a \(A_3\).
    Pravděpodobnosti, že pacient má danou chorobu, jsou
    \begin{equation*}
      p(A_1) = \frac{1}{2},\; p(A_2) = \frac{1}{6},\; p(A_3) = \frac{1}{3},
    \end{equation*}
    tj.
    \begin{equation*}
      p(A_1) + p(A_2) + p(A_3) = 1.
    \end{equation*}
    
    Proto je předepsán ještě doplňující test, jehož výsledek bude pozitivní s pravděpodobností
    \num{0.1} v případě diagnózy \(A_1\), s pravděpodobností \num{0.2} v případě diagnózy \(A_2\) a
    \num{0.9} v případě diagnózy \(A_3\). Doplňující test byl pozitivní. Jaké jsou pravděpodobnosti
    jednotlivých nemocí \(A_1\), \(A_2\) a \(A_3\) po provedení testu?

    Označme jako jev \(B\) to, že výsledek testu je pozitivní. V zadání úlohy jsou uvedeny tyto
    podmíněné pravděpodobnosti:
    \begin{align*}
      p_1 &= p_{A_1}(B) = \num{0.1}, \\
      p_2 &= p_{A_2}(B) = \num{0.2}, \\ 
      p_3 &= p_{A_3}(B) = \num{0.9}.
    \end{align*}
    Označili jsme si je zvláštními symboly \(p_1\), \(p_2\) a \(p_3\), neboť se na ně budeme v
    dalších částech úlohy odvolávat. Podmíněné pravděpodobnosti \(p_B(A_j)\), jejichž zjištění je
    naším úkolem, jsou dány \textbf{Bayesovou formulí} (\ref{mai:eq058}). Pro \(j= 1, 2, 3\) platí
    \begin{align*}
      p_B(A_j) &= \dfrac{p(A_j)\cdot p_{A_j}(B)}{\sum_{i=1}^{k}p(A_i)\cdot p_{A_i}(B)}  \\
               &= \dfrac{p(A_j)\cdot p_{A_j}(B)}{\frac{1}{2}\cdot\num{0.1} + 
                                                \frac{1}{6}\cdot\num{0.2} + 
                                                \frac{1}{3}\cdot\num{0.9}}              \\
               &= p(A_j)\cdot p_{A_j}(B)\cdot\frac{60}{23},                             \\
      p_B(A_1) &= \frac{60}{23} \cdot\frac{1}{2}\cdot\num{0.1}                   
                = \frac{3}{23}\simeq\num{0.130},                                        \\
      p_B(A_2) &= \frac{60}{23} \cdot\frac{1}{6}\cdot\num{0.2}                   
                = \frac{2}{23}\simeq\num{0.087},                                        \\
      p_B(A_3) &= \frac{60}{23} \cdot\frac{1}{3}\cdot\num{0.9}                   
                = \frac{18}{23}\simeq\num{0.783}.
    \end{align*}
    Všimněme si, že součet získaných podmíněných pravděpodobností je roven jedné. Překvapuje nás to?
    Nemělo by, podíváte-li se, jak by dopadl součet výrazů daných Bayesovou formulí
    (\ref{mai:eq058}) přes všechna \(j\).
    
    Je vidět, že výsledek testu velmi napomohl k určení diagnózy. Pokud lékař potřebuje ještě
    spolehlivější informace, může doplňkový test provést opakovaně. Dejme tomu, že test byl proveden
    pětkrát, ve čtyřech případech byl pozitivní a v jednom negativní. Jaké jsou nyní
    pravděpodobnosti jednotlivých diagnóz \(A_1\), \(A_2\), \(A_3\)? Tento výsledek můžeme
    interpretovat jako jev \(C\): Při pětkrát opakovaném dodatečném testu bude výsledek ve čtyřech
    případech pozitivní (zdar) a v jednom případě negativní (nezdar).
    
    Vida, opět Bernoulliův pokus. Podmíněné pravděpodobnosti \(p_{A_1}(C)\), \(p_{A_2}(C)\) a
    \(p_{A_3}(C)\) jsou dány vztahem (\ref{mai:eq055}) pro pravděpodobnost výsledku Bernoulliova
    pokusu:
    \begin{align*}
      p_{A_1}(C) &= \binom{5}{4}\cdot p_1^4\cdot(1-p_1)^1   \\
                 &= 5(\num{0.1})^4(1 - \num{0.1}) = \num{4.5e-4},                   
    \end{align*}
    \begin{align*}
      p_{A_2}(C) &= \binom{5}{4}\cdot p_2^4\cdot(1-p_2)^1   \\
                 &= 5(\num{0.2})^4(1 - \num{0.2}) = \num{6.40e-3},                 
    \end{align*}
    \begin{align*}
      p_{A_3}(C) &= \binom{5}{4}\cdot p_3^4\cdot(1-p_3)^1   \\
                 &= 5(\num{0.9})^4(1 - \num{0.9}) = \num{3.2805e-1}. 
    \end{align*}
    Bayesovu formuli nyní aplikujeme na případ jevu \(C\). Pro \(j = 1, 2, 3\) je
    \begin{gather*}
      \begin{aligned}
        p_C(A_j) &= \dfrac{p(A_j)\cdot p_{A_j}(C)}{\sum_{i=1}^{k}p(A_i)\cdot p_{A_i}(C)}  \\
                 &= \dfrac{p(A_j)\cdot p_{A_j}(C)}{\frac{1}{2}\cdot\num{4.5e-4} + 
                                                  \frac{1}{6}\cdot\num{6.40e-3} + 
                                                  \frac{1}{3}\cdot\num{0.32805}}         \\
                 &= \dfrac{p(A_j)\cdot p_{A_j}(C)}{\num{0.11064}},
      \end{aligned}
    \end{gather*}
    \begin{align*}
      p_C(A_1) &= \frac{1}{2} \cdot\frac{\num{0.00045}}{\num{0.11064}}\simeq\num{0.002},     \\
      p_C(A_2) &= \frac{1}{6} \cdot\frac{\num{0.00640}}{\num{0.11064}}\simeq\num{0.010},     \\
      p_C(A_3) &= \frac{1}{3} \cdot\frac{\num{0.32805}}{\num{0.11064}}\simeq\num{0.988}.
    \end{align*}
    Nyní je o diagnóze \(A_3\) rozhodnuto v podstatě s jistotou, přestože na začátku úlohy byla její
    pravděpodobnost pouze třetinová.
  \end{example}
\end{mdframed}
      %---------------------------------------------------------------
      
      A na závěr ještě hádanky:

      %--Pohádka o Honzovi--------------------------------------------
      % !TeX spellcheck = cs_CZ
\begin{mdframed}[style=mdexam]
  \begin{example}\label{mai:exam062}
    \textbf{Pohádka o Honzovi}\newline
    Honza se vydal ke zlému černokněžníkovi vysvobodit princeznu. Černokněžník se zachechtal a
    pravil: „Princezna je za jedním z těchto tří závěsů. Uhodneš-li, za kterým, můžeš si ji odvést.
    Ne-li, zkameníš.“ Honza tentokrát neměl na pomoc hodné mravenečky ani mazanou lišku Ryšku, a tak
    se rozhodl použít svých znalostí o pravděpodobnosti. Když zjistil, že mu nepomohou, protože
    pravděpodobnosti, že princezna je za jednotlivými závěsy, jsou stejné a rovny jedné třetině,
    zvolil závěs A. Černokněžníka napadlo, že se na Honzovy znalosti o pravděpodobnostech podívá
    lépe, a aniž závěs A odhrnul, řekl: „Mám už na zahradě kamení dost, a proto ti dám jednu
    nápovědu. Ze dvou zbývajících závěsů, B a C, odhrnu ten, za kterým princezna není.“ A odhrnul
    závěs B. Princezna tam opravdu nebyla. Černokněžník řekl: „Teď se teprve rozhodni, budeš-li
    trvat na závěsu A, nebo změníš své rozhodnutí a označíš C.“ Honza přemýšlel, až se mu z hlavy
    kouřilo, a snažil se určit pravděpodobnosti, že princezna je za závěsem A, resp. C. Napadly ho
    dvě úvahy, které vypadaly docela dobře, ale vedly k různým výsledkům:
    \begin{itemize}[noitemsep]
    \item Úvaha prvá: Za závěsem B princezna není a závěsy A a C jsou rovnocenné. Pravděpodobnost,
          že je moje vyvolená ze kterýmkoli z nich, je proto \num{0.5}. Černokněžník mi tedy nijak
          nepomohl. (Nic jiného se od něj taky čekat nedalo.)
    \item Úvaha druhá: Pravděpodobnost, že princezna je za závěsem A, který jsem předem zvolil, byla
          rovna jedné třetině. Tím, že černokněžník odkryl závěs B, se však nemohla změnit. Proto
          pravděpodobnost, že najdu princeznu za závěsem C, je nyní rovna dvěma třetinám.
    \end{itemize}
    
    Poradíme Honzovi, který ze závěsů A a C má zvolit? Vzpomeneme-li si na to, co jsme před chvílí
    zjistili o podmíněných pravděpodobnostech, určitě mu poradit můžeme: Označme jako \(A\), \(B\),
    \(C\) jevy
    \begin{itemize}
      \item Jev \(A\) : Princezna je za závěsem A.
      \item Jev \(B\) : Princezna je za závěsem B.
      \item Jev \(C\) : Princezna je za závěsem C.
    \end{itemize}
    Jejich pravděpodobnosti jsou
    \begin{equation*}
      p(A) = \dfrac{1}{3}, \qquad p(B) = \dfrac{1}{3}, \qquad p(C) = \dfrac{1}{3}.
    \end{equation*}
    Víme, že Honza vybral závěs A, černokněžník tedy může odhrnout závěs B nebo C. Označme jako
    \(B'\) a \(C'\) tyto jevy.
    \begin{itemize}
      \item Jev \(B'\) : Černokněžník odhrne závěs B.
      \item Jev \(C'\) : Černokněžník odhrne závěs C.
    \end{itemize}
    Pravděpodobnosti jevů \(B'\) i \(C'\) jsou shodné a rovny jedné polovině, tj. \(p(B') = p(C') =
    1/2\) . Podle vztahu (\ref{mai:eq057}) platí
      \begin{gather*}
        \begin{aligned}
        p(A\text{ a }B') &= p(B')\cdot p_{B'}(A) = p(A)\cdot p_{A}(B') 
                          = \dfrac{1}{3}\cdot\dfrac{1}{2} = \dfrac{1}{6},  \\
        p(A\text{ a }C') &= p(C')\cdot p_{C'}(A) = p(A)\cdot p_{A}(C')
                          = \dfrac{1}{3}\cdot\dfrac{1}{2} = \dfrac{1}{6},  \\
        \shortintertext{Obdodně je}
        p(C\text{ a }B') &= p(B')\cdot p_{B'}(C) = p(C)\cdot p_{C}(B')
                          = \dfrac{1}{3}\cdot1 = \dfrac{1}{3},              \\
        p(C\text{ a }C') &= p(C')\cdot p_{C'}(C) = p(C)\cdot p_{C}(C') = 0  \\
      \end{aligned}
    \end{gather*}
    Použijeme-li těchto výsledků, dostaneme podmíněné pravděpodobnosti, že princezna je za závěsem
    A, resp. C za podmínky, že černokněžník odhrnul B:
    \begin{align*}
      p_{B'}(A) &= \dfrac{p(A\text{ a }B')}{p_{B'}} = \dfrac{1}{6}:\dfrac{1}{2} = \dfrac{1}{3},  \\
      p_{B'}(C) &= \dfrac{p(C\text{ a }B')}{p_{B'}} = \dfrac{1}{6}:\dfrac{1}{2} = \dfrac{1}{3}    
    \end{align*}
    Správná je tedy druhá Honzova úvaha. Můžeme ještě provést její kontrolu pomocí podmíněných
    pravděpodobností: Pro jev (\(A\) a \(B'\)) nebo (\(A\) a \(C'\)), utvořený ze dvou
    neslučitelných jevů pomocí spojky „nebo“, je tedy
    \begin{equation*}
      p\left((A\text{ a }B')\text{ nebo }(A\text{ a }C')\right) = \dfrac{1}{6} + \dfrac{1}{6} 
        = \dfrac{1}{3}.
    \end{equation*}
    Poslední výsledek představuje skutečnost, že princezna je za závěsem A stále s pravděpodobností
    rovnou jedné třetině, ať se černokněžník chystá odhrnout kterýkoli ze zbývajících závěsů.
  \end{example}
\end{mdframed}
      %---------------------------------------------------------------
      
      %--Může se člověk živit sázením?--------------------------------
      % !TeX spellcheck = cs_CZ
\wikitextrule
\begin{example}\label{mai:exam063}
  \textbf{Může se člověk živit sázením?}\newline\small
  Sázením sportky nebo návštěvami kasina jistě ne! To snad každý po přečtení předchozích odstavců 
  pochopil. Je však možné docela dobře „vydělat“ sázením se s lidmi. Při odhadu pravděpodobností 
  některých jevů nás často intuice zklame a výpočtem získáme hodnoty, které bychom vůbec 
  neočekávali. Tak například odhadněte bez výpočtu, jaká je pravděpodobnost, že alespoň dva lidé ve 
  vaší třídě (čítající například \(k = 23\) studentů) mají narozeniny ve stejný den. Až tento odhad 
  učiníte, zkuste počítat: Uvažme, že rok má \(n = 365\) dní. Spočítejme nejprve pravděpodobnost 
  \(P'\), že každý ze třídy má narozeniny v jiný den. Počet možných případů odpovídá variacím s 
  opakováním \(n^k\), počet případů příznivých odpovídá variacím bez opakování 
  \(\dfrac{n!}{(n-k)!}\) Máme tedy \(P'=\dfrac{n!}{(n-k)!n^k}\). Jev, který nás zajímá (tj., že 
  alespoň dva mají narozeniny ve stejný den), je jevem opačným. Hledanou pravděpodobností bude
  \begin{equation*}
    P  = 1 - P' = 1 - \dfrac{n!}{(n - k)!n^k} \simeq \num{0.5}.
  \end{equation*}
  Vsadíte-li se na večírku s třiceti a více lidmi, že se mezi vámi najdou dva s narozeninami ve 
  stejný den, je vaše vítězství již téměř zaručeno.
\normalsize
\end{example}
      %---------------------------------------------------------------
%---------------------------------------------------------------------------------------------------