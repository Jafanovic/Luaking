% !TeX spellcheck = cs_CZ
%     An overview of high school mathematics
%---------------------------------------------------------------------------------------------------
% intro_LA.tex
%---------------------------------------------------------------------------------------------------
\setchaptertoc
\chapter{Historie matematické analýzy}\label{mai:IchapXII}

  %===============================Kapitola: Determinanty===========================================
  \section{Determinanty}
    Abychom mohli nadefinovat determinant, budeme muset vědět, jak vypočítat permutaci entice, 
    respektive znaménko permutace.
    \subsection{Permutace}
      \begin{definition}\label{permutace}
        Nechť \(\mathbf{M}\) je libovolná konečná množina. Permutací množiny \(M\) nazýváme 
        zobrazení \(\pi\) množiny \(\mathbf{M}\) na sebe.
      \end{definition}
      
      \begin{example}%(Damlová  Nagy, 1985, str. 34)
        Permutace \(\pi\) množiny \(\mathbf{M}= \lbrace a,b,c,d\rbrace\) je např. zobrazení 
        \(\pi\), definované předpisem:
        \begin{equation}\label{permutace_zadani}
          \pi\left(a\right) = c, \,
          \pi\left(b\right) = d, \,
          \pi\left(c\right) = b, \,
          \pi\left(d\right) = a,
        \end{equation}
        Místo tohoto zápisu se však používá přehlednější zápis ve tvaru matice typu \((2,4)\):
        \begin{equation}\label{LA:eq_perm_exam}
            \begin{pmatrix}
            a & b & c & d \\
            c & d & b & a
            \end{pmatrix}
        \end{equation}
        kde v prvním řádku jsou vypsány všechny prvky množiny \(\mathbf{M}\) (v libovolném pořadí) 
        a ve druhém řádku je pod každým prvkem zapsán jeho obraz v permutaci. Tutéž permutaci však 
        můžeme zapsat ve tvaru matice několika různými způsoby. Například mohou být zapsány takto:
        \begin{equation}
          \begin{array}{cc}
            \begin{pmatrix}
              b & a & c & d \\
              d & c & b & a
            \end{pmatrix},         & 
            \begin{pmatrix}
              d & c & b & a \\
              a & b & d & c
            \end{pmatrix}          \\
            \begin{pmatrix}
              d & c & a & b \\
              a & b & c & d
            \end{pmatrix},         &
            \text{apod.}
          \end{array}
        \end{equation}
      \end{example}

      Zřejmě všechny čtyři uvedené zápisy permutace rov. \ref{LA:eq_perm_exam} ve tvaru matice se 
      liší navzájem pouze pořadím sloupců. Aby bylo možné zapsat každou permutaci množiny 
      \(\mathbf{M}\) ve tvaru rov. \ref{LA:eq_perm_exam} jediným způsobem, je nutné zvolit pevné 
      pořadí prvků množiny \(\mathbf{M}\)  a v zápisu permutace uvádět prvky matice \(\mathbf{M}\)  
      v prvním řádku v tomto pořadí. Avšak známe-li toto pořadí prvků množiny \(\mathbf{M}\), je 
      pak  obvykle zbytečné jej v zápisu permutace uvádět, ale stačí uvést pouze pořadí obrazů, tj. 
      druhý řádek. Zvolíme-li např. v naší množině \(\mathbf{M}\) pevné pořadí prvků \(\lbrace 
      a,b,c,d\rbrace\), pak permutaci rov. \ref{permutace_zadani} zapíšeme jako uspořádanou 
      čtveřici \(\lbrace c,d,b,a\rbrace\).
  
      \begin{definition}\label{def_permutace_ntice}
        Když vytváříme uspořádanou \(n\)-tici navzájem různých prvků \(n\)-prv\-ko\-vé množiny 
        \(\mathbf{M}\), přiřazujeme každému prvku množiny \(\mathbf{M}\) právě jedno přirozené 
        číslo, index příslušného prvku, z množiny prvních \(n\) přirozených čísel.
        \begin{equation}\label{permutace_ntice}
          \pi = \lbrace 1, 2, 3, \ldots, n\rbrace
        \end{equation}
      \end{definition}
  
      Proto každé permutaci uspořádané \(n\)-tice prvků množiny \(\mathbf{M}\) odpovídá jednoznačně 
      permutace příslušných indexů tj. permutace množiny \ref{permutace_ntice} z definice 
      \ref{def_permutace_ntice}. Stačí se tedy omezit při vyšetřování permutací n-prvkové množin 
      na vyšetřování permutací množiny \ref{permutace_ntice}. Permutace \(\pi\) množiny 
      \ref{permutace_ntice} budeme zapisovat jako uspořádané \(n\)-tice \(\left(\pi(1), \pi(2) 
      ,\ldots, \pi(n)\right)\), kde \(\pi(i)\) je číslo z množiny \ref{permutace_ntice}, které 
      permutace \(\pi\) přiřazuje číslu \(i\).

      \begin{example}\label{ex_celk_pocet_permutaci}
        \textbf{Spočítejme celkový počet permutací množiny}. V každé uspořádané \(n\)-tici může být 
        na prvním místě kterákoli z \(n\) cifer, na druhém místě kterákoli ze zbývajících \(n-1\) 
        cifer (kromě té, která je na prvním místě), na  třetím místě každá ze zbývajících \(n-2\) 
        cifer atd. Je tedy celkový počet všech permutací \(n\)-prvkové množiny \(n(n-1)(n-2)\cdot 
        \ldots \cdot2\cdot1\). Toto číslo se zapisuje pomocí symbolu \(n!\) (čti 
        \textbf{n-faktoriál}).
      \end{example}
      
      \begin{definition}\label{def_inv_perm}\textbf{Inverze v permutaci}:
        Inverzí v permutaci \(\left(i_1,i_2,…,i_n \right)\) rozumíme každý výskyt takové dvojice 
        čísel, že větší stojí před menším, tj. vlevo od něj.
      \end{definition}  
   
  %====================== Kapitola: Polynomy ======================================================
  \section{Polynomy}
      \begin{definition}\label{def_rov_poly}\textbf{Rovnost dvou polynomů}:
        Řekneme, že dva polynomy \(f(x)=a_nx^n+a^{(n-1)}x_{(n-1)}+\ldots+a_1+a_0\) a
        \(g(x)=b_mx^m+b^{(m-1)}x_{(m-1)}+\ldots+b_1+b_0\) stupňů \(n\) a \(m\) se sobě 
        \textbf{rovnají} právě tehdy, když \(m=n\) a \(a_0=b_0\), \(a_1=b_1\), 
        \(a_{(n-1)}=b_{(m-1)}\), \(a_n=b_m\). V tomto případě také říkáme, že mnohočleny \(f(x)\) a 
        \(g(x)\) jsou \textbf{totožné}.
      \end{definition}
      \begin{lemma}\label{la:eq_eqv_poly}
        Jestliže mnohočleny \(f(x)\) a \(g(x)\) jsou dva polynomy stupně \(n\)-tého a jestliže pro 
        \(n+1\) různých reálných nebo komplexních čísel \(x\) platí \(f(x)=g(x)\), potom jsou 
        polynomy \textbf{totožné}.
      \end{lemma}
      
    \subsection{Rozklad ryze racionální funkce na parci\-ální zlomky}

      %---------------------------------------------------------------
       % !TeX spellcheck = cs_CZ
\begin{mdframed}[style=mdexam]
  \begin{example}\label{mai:exam015}
    Rozložte na parciální zlomky lomenou racionální funkci \((x):y=\frac{7x+8}{x^2+x-2}\).
    \newline\textbf{Řešení:} Nejprve vypočteme nulové body jmenovatele:
    \begin{align*} 
      x^2+px+q &=(x-u)(x-v) = x^2-(u+v)x+uv            \\
                &\rightarrow p=-(u+v),\quad q=uv
    \end{align*}
    Kořenové činitele  \(x^2+x-2\rightarrow x_1=1, x_2=-2\) zvolíme za jmenovatele parciálních
    zlomků a rozklad hledáme ve tvaru \(\frac{7x+8}{x^2+x-2}=\frac{A}{x-1}+\frac{B}{x+2}\)
    kde \(A\), \(B\) jsou neznámé konstanty. Tyto konstanty určíme tak, aby rozklad platil pro 
    každé \(x\in\mathcal{R}-\{1,-2\}\). Po jednoduché úpravě dostaneme rovnost dvou polynomů
    \(7x+8=(A+B)x+2A-B\). Podle \ref{la:eq_eqv_poly} se musí rovnat koeficienty u \(x\) a absolutní 
    členy obou stran poslední rovnice \(\Rightarrow\) dostaneme soustavu rovnic pro určení \(A\) a 
    \(B\) ve tvaru:
    \begin{align}
      % \nonumber to remove numbering (before each equation)
      7 &= A+B  \nonumber \\ 
      8 &= 2A+B \label{la:eq_parc_example}   
    \end{align}
    dostáváme \(A=5,\quad B=2\). Postup, který jsme užili, nazýváme \textbf{Metodou neurčitých 
    koeficientů}.
    
    Pro určení koeficientů \(A\), \(B\) se užívají také jiné postupy, např. dosazování
    kořenů jmenovatele, která je výhodná zejména v případech, kdy jmenovatel lomené racionální
    funkce má jednoduché kořeny. Postupujeme tak, že rov. \ref{la:eq_parc_example} násobíme
    součinem kořenových činitelů \((x-1)(x+2)=x^2+x-2\) a dostaneme rovnici 
    \(7x+8=A(x+2)+B(x-1)\) pro určení koeficientů \(A\), \(B\) dosazováním kořenů.
      \begin{align*}
        % \nonumber to remove numbering (before each equation)
        x=-2 &\rightarrow       -14+8=B(-2-1)      \rightarrow B=2\\
        x=+1 &\rightarrow  \,\,\,+7+8=A(1+2)\quad  \rightarrow A=5
      \end{align*}
  \end{example}
\end{mdframed}
      %---------------------------------------------------------------
      
  %==================== Kapitola: Vektorové prostory ===============================================
 
%} % tikzset
%---------------------------------------------------------------------------------------------------
              