% !TeX spellcheck = cs_CZ
%---------------------------------------------------------------------------------------------------
% mai2ch06.tex
%---------------------------------------------------------------------------------------------------
\setchaptertoc
\chapter{Závislosti na více parametrech aneb funkce více proměnných}\label{mai:IIchapVI}
  Skalárními funkcemi jedné reálné proměnné jsme se zabývali velmi podrobně v \ref{part:MAI} dílu.
  Umíme počítat jejich limity v bodech \emph{vlastních} i \emph{nevlastních}, umíme je derivovat i
  integrovat. Totéž dokážeme provádět i s vektorovými funkcemi jedné proměnné. Každý vektor je totiž
  dán svými složkami, takže každá vektorová funkce je zadána tolika „obyčejnými“ skalárními
  funkcemi, kolik má daný vektor složek. S vektoroVými funkcemi jedné proměnné jsme pracovali při
  zadávání trajektorií hmotných bodů a výpočtech jejich dalších charakteristik (rychlost, zrychlení,
  křivost trajektorie, apod). Tou jedinou proměnnou byl obvykle čas. Veličiny popisující objekty a
  děje v přírodě, ať již jsou tyto veličiny skaláry či vektory, však většinou závisí na více
  proměnných než jedné. Kromě času bývají typicky závislé na poloze. Vezměme třeba takové gravitační
  pole Země. Na čase sice nezávisí, zato však klesá s druhou mocninou vzdálenosti od středu Země.
  Veličina, která je charakterizuje, je buď vektorová, nebo skalární. Tou vektorovou je
  \emph{gravitační zrychlení} neboli \emph{intenzita} gravitačního pole Země, tou skalární je
  \emph{potenciál},
  \begin{equation*}
    \vec{g}(\vec{r}) = - \kappa\dfrac{M_Z}{r^2}\left(\dfrac{\vec{r}}{r}\right),\,
    V(\vec{r}) = -\kappa\dfrac{M_Z}{r}, \, r\geq R_Z
  \end{equation*}
  V předchozích Vztazích jsou \(M_Z\) a \(R_Z\) hmotnost a poloměr Země, \(\vec{r}\) je polohový
  vektor místa, v němž pole zjištujeme, vzhledem ke středu Země. Skalární i vektorové veličiny
  popisující elektrické a magnetické pole nábojů a proudů, rychlosti elementů proudící kapaliny nebo
  plynu a řada dalších fyzikálních veličin jsou nejen funkcemi času, ale také polohy bodu, v němž je
  počítáme nebo měříme. A stejně jako byly změny funkcí jedné proměnné vyjádřeny pomocí derivací,
  změny změn pomocí druhých derivací, atd., je třeba umět počítat i změny veličin závisejících na
  více proměnných. Mohou samozřejmě nastat situace, kdy se mění jen jedna z proměnných, zatímco
  ostatní zůstávají konstantní. Nejsnáze si takovou situaci představíme například tak, že měříme
  třeba elektrické pole stále ve stejném bodě prostoru, ale běží při tom čas. Pole se v daném bodě s
  časem mění. Nebo naopak v daném okamžiku sledujeme rozdílnost pole v bodech velmi blízkých danému
  bodu. Obecně se samozřejmě mění všechny proměnné a s nimi i hodnoty skalární funkce nebo složky
  vektorové funkce. Jak takové obecné změny co nejvýstižněji popsat, uvidíme právě v této kapitole.
  Setkáme se v ní s \emph{parciální derivací}, která vystihuje, jak rychle se mění hodnota funkce se
  změnou jedné z proměnných. Dále poznáme obecnější, \emph{směrovou derivaci}, která popisuje
  rychlost změny funkční hodnoty, mění-li se všechny proměnné, ale tak, že bod, který reprezentuje
  soubor jejich hodnot, se pohybuje v prostoru proměnných po libovolné přímce (nikoli jen po jedné
  souřadnicové ose, jako tomu je u parciálních derivací). A konečně zavedeme pojmy \emph{úplného
  diferenciálu} a \emph{Jacobiho zobrazení}, které popisují změnu hodnot skalární či vektorové
  funkce v lineární aproximaci, mění-li se hodnoty proměnných zcela obecně.

  Stejně jako u funkcí jedné proměnné je základem pro definici veličin popisujících rychlost změny
  funkce pojem \textbf{limity}, který úzce souvisí s pojmem \emph{okolí bodů} a obecně i s
  definičními obory funkcí. Pro případ funkcí více proměnných klade požadavek hlubšího pochopení
  pojmu limity větší nároky na soustředěnost a důkladné promýšlení různých situací, než tomu bylo u
  limit funkcí jedné proměnné. Proto se jím zabývá podstatná část poměrně rozsáhlého odstavce
  \ref{mai:IIchapVIsecII} poté, co je v odstavci \ref{mai:IIchapVIsecI} věnována značná pozornost
  různým typům definičních oborů funkcí. Čtenář, který se chce rychle propracovat k praktickým
  výpočtům a spokojí se vatím s intuitivním pochopením pojmu limity funkce více proměnných
  (založeným na dobré znalosti definice a vlastností limit funkcí jedné proměnné), může k nim v
  podstatě rovnou přejít s vědomím, že sice bude umět prakticky provádět různé standardní operace s
  funkcemi více proměnných, ale nebude si pravděpodobně vědět rady s netypickými případy. K
  důkladnému pročtení odstavců \ref{mai:IIchapVIsecI} a \ref{mai:IIchapVIsecII} se může samozřejmě
  vrátit kdykoli.

  \section{Podmonožiny euklidovských prostorů \(\mathcal{R}^n\)}\label{mai:IIchapVIsecI}
    S definičními obory funkcí jedné proměnné to bylo docela jednoduché. Byly to Většinou intervaly
    - otevřené intervaly nebo jejich sjednocení, uzavřené intervaly, popřípadě intervaly uzavřené
    jen z jedné strany. Také zde figurovala okolí bodů, ať již ryzí, z nichž byl daný bod vyloučen,
    nebo taková, která daný bod obsahovala. Nic složitého tam nebylo. Aby bylo možné funkce
    derivovat, což byla nejčastější operace, kterou jsme prováděli s cílem zjistit, jak rychle se
    funkce mění, nemohly být definiční obory „moc divoké“. Vždy bylo třeba předpokládat, že bod, v
    němž jsme funkci nebo její změnu vyšetřovali, má okolí, ve kterém je funkce přinejmenším
    definována. Pro případ funkcí více proměnných je to podobné, ale bude třeba se definičním oborům
    věnovat trochu podrobněji. Dejme tomu, že nějaká fyzikální, nebo i nefyzikální veličina bude
    záviset na \(n\) proměnných. Každá z nich může nabývat nějakých hodnot. Funkční hodnota naší
    funkce bude tak jednoznačně určena souborem \(n\) hodnot jednotlivých proměnných. Tuto
    \(n\)-tici budeme považovat za bod v prostoru \(mathcal{R}^n\). S \(n\)-ticemi jsme již
    pracovali v algebře, takže takový popis známe. V algebře jsme však s nimi prováděli algebraické
    operace, sčítání a násobení číslem. Měli jsme tedy v \(mathcal{R}^n\) zavedenou
    \textbf{algebraickou strukturu}. Pro práci s funkcemi a pro sledování jejich změn \uv{bod od bodu}
    potřebujeme v \(mathcal{R}^n\) ještě jinou strukturu. Ta je tvořena okolími, podobně jako tomu
    bylo u funkcí jedné proměnné. Taková struktura se nazývá \emph{topologická} a matematická
    disciplína, která se zabývá topologickými strukturami, se jmenuje \textbf{topologie}.

    Topologickými strukturami se nebudeme zabývat v celé obecnosti, i když je to velmi zajímavé.
    Budeme pracovat pouze se speciálním typem, takzvanou euklidovskou topologtí v \(mathcal{R}^n\).

    \subsection{Okolí bodů, otevřené a uzavřené množiny}
    S pojmem okolí bodu jsme se setkali hned v první části druhé kapitoly prvního dílu. Vzpomínáte?
    Šlo tehdy o okolí bodu \(a\) na reálné ose \(mathcal{R}\). Okolím jsme rozuměli otevřený
    interval \((a - \delta_1,a + \delta_2)\), \(\delta_1, \delta_2 > O\), ryzím okolím pak množinu
    \((a - \delta_1, a + \delta_2)\backslash{a}\), tj. sjednocení otevřených intervalů \((a -
    \delta_1, a) \cup (a, a + \delta_2)\). Sjednocení jakýchkoli otevřených intervalů jsme později
    nazývali otevřenou množinou, doplňky otevřených množin v \(mathcal{R}\) byly uzavřené množiny.
    Vybudovali jsme tak na \(mathcal{R}\) \emph{euklidovskou topologii} (dodatek F prvního dílu).
    Podobná bude situace i ve vícerozměrném případě, tedy v \(mathcal{R}^n\).

  \section{Skalární funkce více proměnných}\label{mai:IIchapVIsecII}
    V předchozím odstavci jsme si vcelku důkladně připravili pojmy týkající se vhodných definičních
    oborů funkcí více proměnných. Nyní budeme tyto funkce definovat a studovat jejich vlastnosti.
    Obdobně jako u funkcí jedné proměnné půjde o limity, spojitost a derivace. Výhoda důkladné
    přípravy pojmů v předchozím odstavci se ukáže již při budování pojmu limity, zejména v
    nevlastním bodě. Samotnému pojmu limity a s ním úzce spjatého pojmu spojitosti bude věnována
    velká pozornost. Někdo se nad tím může pozastavit: Vždyť jak často se s potřebou výpočtu limit
    setkáváme v praktických příkladech? Málo. Potřebujeme spíše derivovat, integrovat, řešit
    diferenciální rovnice. S příklady na limity se setkáváme ponejvíce v matematických sbírkách, kde
    slouží k procvičení naučené látky. Taková úvaha by byla poněkud krátkozraká. Řada vlastností
    funkcí, a vůbec možnost s nimi rozumně operovat při běžných výpočtech, je založena na
    spojitosti. A spojitost je těsně spjata s pojmem limity. Opět by mohla vzniknout námitka, že
    spojitost přece intuitivně dobře chápeme - spojitá funkce nemá „zpřetrhaný“ graf. Umíme si však
    představit graf funkce tří proměnných? Není to dost dobře možné ani u funkce dvou proměnných,
    je-li složitější, nebo chová-li se v okolí některého bodu „podezřelé“. Limitě tedy potřebujeme
    porozumět, i když její výpočet v praktických příkladech skutečně nebude na „denním pořádku“.
    Abychom porozumění podpořili, budeme se vedle teoretických úvah věnovat i řadě praktických
    příkladů.




%---------------------------------------------------------------------------------------------------