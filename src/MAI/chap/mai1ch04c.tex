% !TeX program = lualatex 
% !TeX root = luaking.tex 
% !TeX encoding = UTF-8 
% !TeX spellcheck = cs_CZ
 %---------------------------------------------------------------------------------------------------
\graphicspath{{../src/MAI/img/}}
% file mai1ch04c.tex
%---------------------------------------------------------------------------------------------------
\setchaptertoc
\chapter{Kombinatorika}\label{mai:IchapIVc}
  \textbf{Kombinatorika} se od všech matematických disciplín, v několika směrech liší. Zatímco v
  geometrii má každá přímka nekonečnou délku a každý trojúhelník nekonečně mnoho bodů, zatímco v
  algebře některé rovnice i nerovnice mají nekonečně mnoho řešení a zatímco matematická analýza
  zkoumá limity posloupností a funkcí, roste-li příslušná proměnná do nekonečna, v kombinatorice se
  s nekonečnem nesetkáme. Kombinatorika je součástí \textbf{finitní matematiky}, která studuje
  konečné soubory (množiny a uspořádané \(k\)-tice, \(k\in \mathcal{N}\)). 
  
  Další odlišností je to, že často nemáme možnost ověřit si správnost výsledku, ke kterému jsme při
  řešení kombinatorické úlohy dospěli, a jsme odkázáni jen na svůj vlastní úsudek. Proto v
  kombinatorice v míře větší než jinde platí, že „cvičení dělá mistra“. 
  
  V této kapitole je probrána část kombinatoriky, která se zabývá vytvářením skupin z daných prvků a
  určováním jejich počtu. Jde o klasickou problematiku, která byla řešena již v 17. a 18. století a
  která je spojena se jmény \emph{B. Pascala}, \emph{P. Fermata}, \emph{J Bernoulliho}, \emph{G. W.
  Leibnize}  a \emph{L. Eulera}. Dnes představuje kombinatorika rozsáhlou matematickou disciplínu,
  některé její problémy byly již vyřešeny (problém čtyř barev) mnohé další na své vyřešení čekají. 
  
  A závěrem ještě důležitá poznámka terminologická: přirozenými čísly se v této kapitole rozumějí
  čísla celá kladná tj. čísla 1, 2, 3, 4, \(\ldots\), nula se tedy mezi přirozená čísla nezahrunuje.
  \cite[s.~7]{calda2008matematika} 
    
  \twocolumn[\section{Základní kombinatorická pravidla}\label{mai:IchapIVcsecI}]
    K řešení velké části kombinatorických úloh vystačíme s dvěma jednoduchými pravidly. Je dost
    pravděpodobné, že jsme je už sami několikrát použili, aniž nám bylo známo, že se o nějaká
    kombinatorická pravidla vůbec jedná. Vysvětlíme si je na řešení následujících (velmi snadných
    příkladů).

    %--Počet všech přirozených dvojciferných čísel------------------
    % !TeX spellcheck = cs_CZ
\begin{mdframed}[style=mdexam]
  \begin{example}\label{mai:exam094}
    Pokusme se určit počet všech přirozených dvojciferných čísel, v jejichž dekadickém zápisu se
    každá číslice vyskytuje nejvýše jednou. Kredit: \cite[s.~8]{calda2008matematika} \newline
    \textbf{Řešení}:
    \begin{enumerate}[noitemsep]
      \item způsob: na místě desítek může stát libovolná z devíti číslic \(1, 2, \ldots, 9\), neboť
            zde nesmí být číslice 0. Ke každé z těchto devíti možností pro výběr číslice na místě
            desítek existuje devět možností, jak vybrat číslici pro místo jednotek: může zde totiž
            být číslice 0 a jakákoliv z osmi číslic, která je různá od číslice stojící na místě
            desítek. Počet uvažovaných dvojciferných čísel je tedy \(9\cdot9 = 81\).
      \item způsob: všechna přirozená dvojciferná čísla lze rozložit do dvou disjunktních skupin
            tak, že v první jsou dvojciferná čísla s různými a ve druhé dvojciferná čísla s týmiž
            číslicemi. Je zřejmé, že všech dvojciferných čísel je 90 a dvojciferných čísel s týmiž
            číslicemi 9 ( jde o čísla \(11, 22, \ldots, 99)\); označíme-li \(p\) hledaný počet
            dvojciferných čísel s různými číslicemi, platí \(p+9 = 90\). Odtud dostáváme, že je
            \(p=81\).      
    \end{enumerate}
  \end{example}
\end{mdframed}
    %---------------------------------------------------------------

    V prvním řešení příkladu je obsaženo kombinatorické pravidlo součinu:
    
    \begin{mdframed}[style=highlight] Počet všech uspořádaných \(k\)-tic, jejichž první člen lze
      vybrat \(n_1\) způsoby, druhý člen po výběru prvního členu \(n_2\) způsoby atd. až \(k\)-tý
      člen po výběru všech předcházejících členů \(n_k\) způsoby, je roven \(n_1\cdot n_2\cdots
      n_k\).
    \end{mdframed}

    Vskutku: uvedená dvojciferná čísla lze považovat za uspořádané dvojice s různými členy, protože
    první člen lze vybrat devíti způsoby a druhý (po výběru prvního) rovněž devíti způsoby, je počet
    těchto uspořádaných dvojic - tj. uvažovaných dvojciferných čísel - roven \(9\cdot9= 81\).

    Ve druhém řešení bylo použito kombinatorické pravidlo součtu:
    
    \begin{mdframed}[style=highlight] Jsou-li \(A_1\), \(A_2\), \(\cdots\), \(A_n\) konečné množiny,
      které mají po řadě \(p_1\), \(p_2\), \(\cdots\), \(p_n\) prvků, a jsou-li každé dvě
      disjunktní, pak - počet prvků množiny \(A_1 \cup A_2 \cup\cdots\cup A_n\), je roven \(p_1 +
      p_2 \cdots + p_n\).        
    \end{mdframed}

    Je jasné, že v uvedeném příkladu je \(A_1\) množina všech dvojciferných čísel s různými
    číslicemi a \(A_2\) množina všech dvojciferných čísel s týmiž číslicemi. 
    
    Jak je vidět, jsou obě pravidla téměř samozřejmá, což jim vsak neubírá na významu; pomocí nich
    odvodíme v dalších článcích některé kombinatorické vzorce. V následujícím příkladu uvidíme, jak
    kombinatorické pravidlo součinu může pomoci při rozhodování, kolikrát je vlastně daný objekt (v
    příkladu jde o trojúhelník) počítán.
  
    %--Počet všech trojúhelníků XYZ---------------------------------
    % !TeX spellcheck = cs_CZ
\begin{mdframed}[style=mdexam]
  \begin{example}\label{mai:exam095}
    Je dán čtverec \(ABCD\) a na jeho každé straně n vnitřních bodů. Určete počet všech trojúhelníků
    \(XYZ\), jejichž vrcholy leží v daných bodech a na různých stranách čtverce \(ABCD\). Kredit
    \cite[s.~9]{calda2008matematika}.\newline
    \textbf{Řešení}:
    \begin{itemize}[noitemsep]
      \item Vrchol \(X\) je možno zvolit v libovolném z daných bodů, takže pro něj máme \(4n\)
            způsobů výběru. Po výběru bodu \(X\) lze bod \(Y\) vybrat už jen \(3n\) způsoby, neboť
            nesmí ležet na téže straně čtverce jako bod \(X\).
      \item Po výběru bodu \(Y\) je možné bod \(Z\) vybrat pouze \(2n\) způsoby, neboť nesmí ležet
            na těch stranách čtverce \(ABCD\), na nichž leží body \(X\), \(Y\). Existuje tedy
            \(4n\cdot3n \cdot2n = 24n^3\) uspořádaných trojic utvořených z bodů \(X\), \(Y\), \(Z\).
      \item Uvědomme si však, že šest uspořádáných trojic takto sestavených určuje stejný
            trojúhelník. Tak např. každá z uspořádaných trojic \((X, Y, Z)\), \((X, Z, Y)\), \((Y,
            X, Z)\), \((Y, Z, X)\), \((Z, X, Y),\) \((Z, Y, X)\) představuje trojúhelník, a to
            \(\Delta XYZ\). 
      \item Abychom dostali počet všech trojúhelníků požadované vlastnosti, musíme získaný počet
            uspořádaných trojic dělit šesti. Trojúhelníků dané vlastnosti je tedy:
            \begin{equation*} 
              \dfrac{24n^3}{6}
            \end{equation*}
    \end{itemize}

    {\centering
    \captionsetup{type=figure} 
    \luafigure[0.8]{mai_fig069}
    \captionof{figure}{Ilustace k příkladu \ref{mai:exam095}}
    \label{mai:fig069}
    \par}

  \end{example}
\end{mdframed}
    %---------------------------------------------------------------

    V řešení následujícího \uv{turistického} příkladu jsou použita obě uvedená pravidla. Poznáme jak?
    
    %--Turistický příklad-------------------------------------------
    % !TeX spellcheck = cs_CZ
\begin{mdframed}[style=mdexam]
  \begin{example}\label{mai:exam096}
    Z místa \(A\) do místa \(B\) vedou čtyři turistické cesty, z místa \(B\) do místa \(C\) tři.
    Určete, kolika zpåsoby lze vybrat trasu z \(A\) do \(C\) a zpět tak, že z těchto sedmi cest je
    právě jedna použita dvakrát. \newline
    \textbf{Řešení}: Nejprve určíme, kolika způsoby lze vybrat trasu z \(A\) do \(C\): ke každému ze
    čtyř způsobů, jak dojít z \(A\) do \(B\), existují tři způsoby, jak dojít z \(B\) do \(C\).
    Trasu z \(A\) do \(C\) lze tedy vybrat \(4\cdot3\), tj. dvanácti způsoby (obr.
    \ref{mai:fig066a}).

    {\centering
      \captionsetup{type=figure}
      \captionsetup[subfigure]{justification=centering}
      \subcaptionbox{\label{mai:fig066a}}{\luafigure[0.5]{mai_fig066a.png}}  
      \subcaptionbox{\label{mai:fig066b}}{\luafigure[0.5]{mai_fig066b.png}} \newline
      \subcaptionbox{\label{mai:fig066c}}{\luafigure[0.5]{mai_fig066c.png}}
      \captionof{figure}{K příkladu \ref{mai:exam096} \cite[s.~11]{calda2008matematika}}
      \label{mai:fig066}
    \par}
    
    Nyní jde o to, kolika způsoby lze vybrat zpáteční trasu z \(C\) do \(A\) tak, aby v ní byla
    použita právě jedna cesta z těch, po kterých jsme už šli z \(A\) do \(C\). Máme tedy dvě
    možnosti:
    \begin{itemize}[noitemsep]
      \item Po stejné cestě se budeme vracet z \(C\) do \(B\). Potom z \(B\) do \(A\) půjdeme jinou
            cestou, než kterou jsme šli z \(A\) do \(B\). V tomto případě lze vybrat zpáteční trasu
            z \(C\) do \(A\) třemi způsoby (obr. \ref{mai:fig066b}).
      \item Z \(C\) do \(B\) půjdeme jinou cestou, než kterou jsme přišli, a z \(B\) do \(A\)
            půjdeme po stejné cestě, jako z \(A\) do \(B\). V tomto případě lze vybrat zpáteční
            trasu z \(C\) do \(A\) dvěma způsoby. (obr. \ref{mai:fig066c})
    \end{itemize}
    Protože obě uvedené možnosti se navzájem vylučují a jiné nejsou, dostáváme (podle
    kombinatorického pravidla součtu), že celkový počet tras z \(C\) do \(A\), které splňují dané
    podmínky, je roven pěti. Ke každé z dvanácti tras z \(A\) do \(C\) existuje tedy pět tras z
    \(C\) do \(A\), které splňují požadovanou podmínku. Pomocí kombinatorického pravidla součinu
    získáme výsledek úlohy: 
    
    Počet všech způsobů, kterými lze vybrat trasu z \(A\) do \(C\) a zpět tak, že z daných cest je
    právě jedna použita dvakrát, je \(12\cdot5 = 60\).

    \textbf{Podobné úlohy:} Určete počet způsobů, jimiž lze vybrat trasu
    \vspace*{-0.5\baselineskip}
    \begin{itemize}[noitemsep]
      \item z \(A\) do \(C\) a zpět: \(4\cdot3\cdot3\cdot4=144\);
      \item z \(A\) do \(C\) a zpět tak, že z těchto cest není žádná použita dvakrát:
            \(4\cdot3\cdot3\cdot2=72\);
      \item z A do C a zpět tak, že z těchto cest jsou právě dvě použity dvakrát:
            \(4\cdot3\cdot1\cdot1 = 12\).
    \end{itemize}
  \end{example}
\end{mdframed}
    %---------------------------------------------------------------

  \section{Variace}\label{mai:IchapIVcsecII}
    V kombinatorice se často setkáváme s \(k\)-člennými skupinami utvořenými z daných \(n\) prvků
    tak, že v nich \emph{záleží na pořadí} a žádný z daných prvků se v nich \emph{neopakuje}.
    Ptáme-li se třeba, kolika možnými způsoby může být mezi osm finalistů olympijského sprintu na
    \SI{100}{\m} rozdělena zlatá, stříbrná a bronzová medaile, ptáme se vlastně na to, kolika
    způsoby lze z daných osmi atletů utvořit uspořádanou trojici. Uspořádanou trojicí v tomto
    případě rozumíme trojici, v níž záleží na tom, kdo z jejích členů dostane zlatou, kdo stříbrnou
    a kdo bronzovou medaili. Takovéto skupiny se nazývají \textbf{variace}, přesněji \(k\)-členné
    variace z \(n\) prvků.

    \begin{mdframed}[style=highlight] \(k\)-členná variace bez opakování z \(n\) prvků je uspořádaná
      \(k\)-tice sestavená z těchto prvků tak, že každý se v ní vyskytuje nejvýše jednou.
    \end{mdframed}
    
    Pro ilustraci uveďme všechny tříčlenné variace ze čtyř prvků \(a\), \(b\), \(c\), \(d\) do
    tabulky \ref{mai:tab001}:
    \begin{table}[ht!]      %\ref{fyz:tab006}
      \centering
      \scalebox{0.8}{ 
      \begin{tabular}{cccc}
        (a, b, c)& (a, c, b)& (b, a, c)& (b, c, a) \\ 
        (c, a, b)& (c, b, a)& (a, b, d)& (a, d, b) \\ 
        (b, a, d)& (b, d, a)& (d, a, b)& (d, b, a) \\ 
        (a, c, d)& (a, d, c)& (c, a, d)& (c, d, a) \\
        (d, a, c)& (d, c, a)& (b, c, d)& (b, d, c) \\  
        (c, b, d)& (c, d, b)& (d, b, c)& (d, c, b)
      \end{tabular}}
      \caption{Tříčlenné variace ze čtyř prvků \(a\), \(b\), \(c\), \(d\)}
      \label{mai:tab001}
    \end{table}

    Určíme nyní počet všech \(k\)-členných variací bez opakování z \(n\) prvků; budeme jej značit
    symbolem \(V(k,n)\) a k jeho určení použijeme \emph{kombinatorické pravidlo součinu}.

    Mějme tedy dáno \(n\) navzájem různých prvků a přirozené číslo \(k\), \(k\leq n\). Pro výběr
    prvního členu uspořádané \(k\)-tice máme \(n\) možností, neboť zde může stát libovolný z daných
    \(n\) prvků; po jeho výběru máme pro výběr 2. členu už jen \(n - 1\) možností, neboť na tomto
    místě už nemůže být prvek, který jsme vybrali na místo první. Po výběru prvních dvou členů máme
    pro výběr 3. členu \(n - 2\) možností a tak dále, až pro výběr \(k\)-tého členu máme po výběru
    všech členů předcházejících právě \(n - (k - 1)\) možností. Schematicky to lze znázornit
    obrázkem \ref{mai:fig067}:

    \luagraphic[1]{mai_fig067.pdf}{Variace bez opakování - schématické znázornění 
                    \cite[s.~13]{calda2008matematika}}{mai:fig067}

    Podle kombinatorického pravidla součinu je počet všech těchto uspořádaných \(k\)-tic roven
    součinu \(n(n-1)(n-2)\cdots(n-(k-1)) = n(n-1)(n-2)\cdots(n - k + 1)\). Máme tedy výsledek

    \begin{mdframed}[style=highlight] Počet \(V(k,n)\) všech \(k\)-členných variací z \(n\) prvků je
      \begin{equation*}
        V(k,n) = n(n-1)(n-2)\cdots(n - k + 1)
      \end{equation*}
    \end{mdframed} 

    Vzorec pro \(V(k,n)\) si můžeme zapamatovat takto: \(V(k,n)\) je rovno součinu \(k\) přirozených
    čísel takových, že největší je rovno \(n\) a ke každké další je o jednu menší. V následujícím
    článku vyjádříme číslo \(V(k,n)\) pomocí tzv. \textbf{faktoriálu}. 

    Na otázku z úvodu tohto článku, kolika způsoby mohou být obsazeny stupně vítězů po olympijském
    finále v běhu na \SI{100}{\meter}, už tedy umíme odpovědět: počet těchto způsobů je \(V(3.8) =
    8\cdot7\cdot6=336\)

    %--Vlajky-------------------------------------------------------
    % !TeX spellcheck = cs_CZ
\begin{mdframed}[style=mdexam]
  \begin{example}\label{mai:exam099}
    K sestavení vlajky, která má být složena ze tří různobarevných vodorovných pruhů, jsou k
    dispozici látky barvy bílé, červené, modré, zelené a žlůté. Kredit:
    \cite[s.~14]{calda2008matematika} \newline
    \begin{enumerate}[noitemsep]
      \item Určete počet vlajek, které lze z látek těchto barev sestavit.
      \item Kolik z nich má modrý pruh?
      \item Kolik jich má modrý pruh uprostřed?
      \item Kolik jich nemá uprostřed červený pruh?  
    \end{enumerate}
    \textbf{Řesení}
    \begin{enumerate}[noitemsep]
      \item Vzhledem k tomu, že každé dva pruhy mají být různé barvy a že záleží na pořadí těchto
            pruhů, jde o tříčlenné variace z pěti prvk. Z látek daných barev lze sestavit \(V(3,5) =
            5\cdot4\cdot3=60\) různých vlajek.
      \item Vlajku s modrým pruhem dostaneme tak, že vybereme uspořádanou dvojici pruhů z látek
            barvy bílé, červené, zelené a žlůté (to lze provést \(V(2,4)=4\cdot3=12\) způsoby) a
            přidáme pruh modrý (což lze provést třemi způsoby: nahoru, doprostřed, dolů). Vlajek s
            modrým pruhem je tedy \(3\cdot V(2,4)=36\).
      \item Vlajek s modrým pruhem uprostřed je zřejmě \(V(2,4)=12\), neboť pro zařazení modrého
            pruhu už nemáme tři možnosti jako v případě předchozím, ale jedinou. 
      \item Počet vlajek, které nemají uprostřed červený pruh, je stejný jako počet vlajek, které
            nemají uprostřed modrý pruh. Tento počet je roven počtu všech vlajek zmenšenému o počet
            vlajek, které mají uprostřed modrý pruh, tj. číslu \(60-12=48\).
    \end{enumerate}


  \end{example}
\end{mdframed}
    %---------------------------------------------------------------
  
  \section{Permutace}\label{mai:IchapIVcsecIII}
    V předcházejícím článku jsme se zabývali \(k\)-člennými variacemi z \(n\) prvků, tj.
    uspořádanými \(k\)-ticemi, v nichž se každý z daných \(n\) prvků vyskytuje nejvýše jednou; pro
    přirozená čísla \(k\), \(n\) přitom platí \(k \leq n\), neboť pro \(k > n\) nelze z daných \(n\)
    prvků utvořit žádnou uspořádanou \(k\)-tici, v níž by se žádný prvek neopakoval. V tomto článku
    se budeme zabývat uspořádanými \(n\)-ticemi sestavenými z daných \(n\) prvků, tj. případem \(k =
    n\); takovéto skupiny se nazývají \textbf{permutace} (pořadí).
    \begin{mdframed}[style=highlight] Permutace z \(n\), prvků je každá n-členná variace z těchto
      prvků.
    \end{mdframed}
    Jinak řečeno:
    \begin{mdframed}[style=highlight] Permutace z \(n\) prvků je uspořádaná \(n\)-tice sestavená z
      těchto prvků tak, že každý se v ní vyskytuje právě jednou.
    \end{mdframed}

    Pro ilustraci uveďme všechny permutace ze tří prvků \(a\), \(b\), \(c\).
    \begin{align*}
      (a,b,c)\, &, (a,c,b)\,, (b,a,c)\,, \\
      (b,c,a)\, &, (c,a,b)\,, (c,b,a).
    \end{align*}        

    Poznamenejme ještě, že každá permutace z daných \(n\) prvků určuje nějakou uspořádanou
    \(n\)-tici z těchto prvků, tj. nějaké jejich pořadí. Určit počet všech možných pořadí \(n\)
    prvků neznamená nic jiného než určit všechny permutace z těchto \(n\) prvků. 
    
    Určíme nyní počet \(P(n)\) všech permutací z \(n\) prvků; dostaneme jej evidentně tak, že do
    vzorce pro počet \(k\)-členných variací z \(n\) prvků dosadíme \(k = n\):
    \begin{equation*}
      P(n) = V(n,n) = n\cdot (n-1)\cdot(n-2)\cdot\cdots\cdot2\cdot1
    \end{equation*}
    Tento výsledek upravíme zavedením symbolu \(n!\) (čteme: \(n\) faktoriál) pro součin všech
    přirozených čísel od jedné do \(n\):
    \begin{mdframed}[style=highlight] Pro každé přirozené číslo \(n\) definujeme:
      \begin{equation*}
        n! = 1\cdot2\cdot3\cdot\cdots\cdot(n-1)\cdot n
      \end{equation*}
    \end{mdframed}

    Je užitečné si uvědomit, že je \(n! = (n-1)!\cdot n = (n-2)!\cdot(n-1)\cdot n\) atd. Tyto vztahy
    se často používají při úpravách výrazů s faktoriály. Užitím faktoriálu lze tedy výsledek, který
    jsme odvodili pro počet permutací z \(n\) prvků, vyslovit takto:
    \begin{mdframed}[style=highlight] Počet \(P(n)\) všech permutací z \(n\) prvků je
      \begin{equation*}
        P(n) = n!
      \end{equation*}
    \end{mdframed}

    Vraťme se na okamžik ke vzorci pro \(V(k,n)\), který jsme odvodili v předcházejícím článku a
    vyjádříme jej pomocí faktoriálu. Dostaneme tak, že pro všechna přirozená čísla \(n, k\), \(k<n\)
    platí
    \begin{equation*}
      V(k, n) = n\cdot(n-1)\cdot\ldots\cdot(n-k+1) 
    \end{equation*}
    což lze rozepsat jako podíl dvou faktoriálů
    \begin{gather*}
      \dfrac{n\cdot(n-1)\cdot\ldots\cdot(n-k+1)
              \cdot(n-k)\cdot(n-k-1)\cdot\ldots\cdot3\cdot2\cdot1}
                  {(n-k)\cdot(n-k-1)\cdot\ldots\cdot3\cdot2\cdot1}  
    \end{gather*}                    
    konečně dostáváme vztah \(V(k, n) = \frac{n!}{(n-k)!}\)
    \vspace{2em}
    Aby tento vzorec platil i prok \(k=n\), kdy \(V(n,n) = n!\) je nutné, aby jmenovatel zlomku
    \(\frac{n!}{(n-n)!}\) byl roven jedné.
    \begin{mdframed}[style=highlight] Definujeme
      \begin{equation*}
        0! = 1
      \end{equation*}
      Dostáváme tak, že pro všechna přirozená čísla \(n, k\),\(k<n\) platí
      \begin{equation*}
        V(k,n) = \dfrac{n!}{(n-k)!}
      \end{equation*}
    \end{mdframed}
    
    %--Parlament----------------------------------------------------
      % !TeX spellcheck = cs_CZ
\wikitextrule
\begin{example}\label{mai:exam057}
  \textbf{Bernoulliův pokus}\newline\small
  Bernoulliův pokus spočívá v tom, že \(n\)-krát nezávisle provedeme určitý pokus, například hod 
  mincí. (V terminologii teorie pravděpodobnosti nazýváme každé takové provedení opakováním 
  pokusu.) Sledujeme, v kolika případech z těchto \(n\) opakování nastal daný jev (například jev 
  \(A\) — padne hlava). Výsledek opakování pokusu, při kterém daný jev nastal, nazveme zdarem, 
  výsledek, kdy nastal jev opačný, nezdarem. Dejme tomu, že pravděpodobnost zdaru je \(p\). (Pro 
  případ padnutí hlavy na minci je \(p = 1/2\).) Pravděpodobnost nezdaru je pak \((l - p)\).
  (V případě hodů mincí je \((1 — p) = 1/2\).) Zajímáme se o to, jaká je pravděpodobnost \(P(x)\), 
  že při \(n\) opakováních pokusu docílíme \(x\)-krát zdaru, \(x\) přitom můžeme předem volit 
  libovolně v rozmezí \(0 \leq x \leq n\). V případě hodů mincí jistě dokážeme předem odhadnout, 
  že pravděpodobnosti \(P(0)\) a \(P(n)\), tj. pravděpodobnosti toho, že nepadne hlava vůbec nebo 
  že padne hlava vždy, budou při větším počtu opakování pokusu malé a budou se blížit nule tím 
  více, čím větší bude \(n\). Naopak bychom se mohli domnívat, že pravděpodobnost \(P(n/2)\), 
  tj. že padne hlava v polovině opakování pokusu, by měla být při velkém počtu \(n\) blízká 
  \SI{100}{\percent}. Správnost tohoto našeho předběžného odhadu však posoudíme teprve poté, co si 
  odvodíme obecný vzorec pro \(P(x)\). Budeme možná překvapeni. Zvolme nejprve pevně, při kterých 
  konkrétních opakováních pokusu má dojít ke zdaru  (například při prvních \(x\)). Při ostatních 
  pak požadujeme nezdar. Protože jevy
  \begin{align*}
    A_1                &: \text{Při prvním opakování dojde ke zdaru.}                  \\
    A_2                &: \text{Při druhém opakování dojde ke zdaru.}                  \\
    \ldots             &: \ldots\ldots\ldots\ldots\ldots\ldots\ldots\ldots\ldots\ldots \\
    A_x                &: \text{Při \(x\)-tém opakování dojde ke zdaru.}               \\
    \overline{A}_{x+1} &: \text{Při \((x + 1)\)-tém opakování dojde k nezdaru.}        \\
    \ldots             &: \ldots\ldots\ldots\ldots\ldots\ldots\ldots\ldots\ldots\ldots \\
    \overline{A}_n     &: \text{Při posledním \(n\)-tém opakování dojde k nezdaru,}
  \end{align*}
  jsou nezávislé, je pravděpodobnost jevu
  \begin{itemize}
    \item \(B_1\): Při každém z prvních \(x\) opakování dojde ke zdaru a současně při každém z 
          dalších \((n — x)\) opakování dojde k nezdaru, rovna součinu pravděpodobností
          \begin{equation*}
            p(B_1) = p(A_1)p(A_2)\cdots p(A_x)p(\overline{A}_{x+1})\cdots p(\overline{A}_n) 
                   = p^x (1 - p)^{n-x}.
          \end{equation*}
  \end{itemize}
  Nám však jde o pravděpodobnost následujícího jevu
  \begin{itemize}
    \item \(B\): Právě při \(x\) opakováních pokusu (bez ohledu na to, kterých) dojde ke zdaru a 
          současně při každém ze zbývajících opakování pokusu dojde k nezdaru.
  \end{itemize}
  
  Možností výběru \(x\) opakování, při kterých dojde ke zdaru, je \(N(x) = \begin{pmatrix} n \\ 
  x\end{pmatrix}\). Pokud bychom očíslovali jednotlivé výběry \(j = 1, 2, \cdots, N(x)\), dostaneme 
  odpovídající jevy \(B_1, \cdots, B_{N(x)}\) Pravděpodobnost každého z nich je stejná a rovna 
  pravděpodobnosti jevu \(B_1\), který jsme popsali před chvílí. Tyto jevy jsou po dvou 
  neslučitelné a jev \(B\) znamená, že nastane právě jeden (kterýkoli) z nich. Pro jeho 
  pravděpodobnost tedy platí, podle pravidla pro součet pravděpodobností po dvou neslučitelných 
  jevů,
  \adjustbox{minipage=[c]{\textwidth}}{%
    \begin{equation}\label{mai:eq055}
      p(B) = P(x) = \begin{pmatrix} n \\ x\end{pmatrix}p^x (1 - p)^{n-x}.
    \end{equation}
    }
   
  Zkusme nyní prověřit správnost našeho odhadu týkajícího se hodů mincí:
  \begin{equation*}
    P(0) = \begin{pmatrix} n \\ 0\end{pmatrix} 
           \left(\dfrac{1}{2}\right)^0\left(\dfrac{1}{2}\right)^{n-0} 
         = \dfrac{1}{2^n}            \qquad
    P(n) = \begin{pmatrix} n \\ n\end{pmatrix} 
           \left(\dfrac{1}{2}\right)^n\left(\dfrac{1}{2}\right)^{n-n} 
         = \dfrac{1}{2^n}     
  \end{equation*}
  Vidíme, že náš odhad byl správný. Obě pravděpodobnosti klesají s rostoucím počtem opakování 
  pokusu k nule.  Pro jediné opakování pokusu, tj. \(n = 1\), jsou obě rovny jedné polovině, a to 
  bychom jistě také měli očekávat. 
  
  Pro \(n\) sudé nyní počítejme \(P(n/2)\). Položme \(n = 2m\):
  \begin{equation*}
    P(m) = \begin{pmatrix} 2m \\ m\end{pmatrix} 
           \left(\dfrac{1}{2}\right)^m\left(\dfrac{1}{2}\right)^{2m-m} 
         = \dfrac{(2m)!}{m!m!}\left(\dfrac{1}{2}\right)^{2m}     
  \end{equation*}

  \begin{table}[ht!]
    \centering
    \begin{tabular}{c|rrrrr}
      \(m\)    & 1 & 2 & 3 & 5 & 10  \\ \hline
      \(P(m)\) & \num{0.500} & \num{0.375} & \num{0.313} & \num{0.246} & \num{0.176}
    \end{tabular}
    % \caption{ }
  \end{table}
  
  Tady se zdá, že nás naše intuice při odhadu pravděpodobnosti \(P(n/2)\) zklamala. Tendence hodnot 
  \(P(n/2)\) je pro rostoucí \(n\) klesající. Pravděpodobnost je největší pro \(n = 2\), a to právě 
  padesátiprocentní! Zkusme ještě odhad pro velká \(n\) pomocí \textbf{Stirlingova vzorce}. Podle 
  něj pro velká \(n\) platí
  
  \begin{equation}\label{mai:eq056}
    n! \doteq \left(\dfrac{n}{e}\right)^n\sqrt{2\pi n}
  \end{equation}
  Použijeme-li jej pro výpočet P(m), dostáváme
  \begin{equation*}
    P(m)\doteq \dfrac{\left(\dfrac{2m}{e}\right)^{2m}\sqrt{4\pi m}}
     {\left(\dfrac{m}{e}\right)^m\left(\dfrac{m}{e}\right)^m\left(\sqrt{2\pi m}\right)^2}
     \left(\dfrac{1}{2}\right)^{2m} = \dfrac{1}{\sqrt{\pi m}} \longrightarrow 0
  \end{equation*}
  pro velká \(m\). Kde jsme se tedy zmýlili? Ze zkušenosti víme, že budeme-li házet mincí 
  mnohokrát, je prakticky jisté, že hlava skutečně padne zhruba v polovině případů! Problém spočívá 
  ve slovíčku zhruba. Pravděpodobnost \(P(m)\) pro \(n = 2m\) se však týká jevu, kdy hlava padne 
  přesně v polovině případů. A ta samozřejmě bude tím menší, čím větší je počet posuzovaných hodů 
  mincí. Při zvyšujícím se počtu \(n\) opakování pokusu totiž roste i počet jednotlivých možností 
  volby \(x\) a \(n\) a každou z nich tak „připadne“ menší pravděpodobnost. (Součet  
  pravděpodobností přes všechna přípustná \(x\) musí být roven jedné.) Později, v odstavci 
  \ref{mai:IchapIIIsecII}, uvidíme, že jsme nevědomky místo pravděpodobnosti odhadovali střední 
  hodnotu náhodné veličiny.
  
  Položme si ještě poslední otázku v souvislosti s Bernoulliovým pokusem: Jaká je pravděpodobnost, 
  že alespoň při jednom z \(n\) opakování pokusu nastane zdar? Pokud si po předchozím neúspěchu s 
  intuitivními odhady ještě trochu věříme, můžeme předpovídat, že tato pravděpodobnost poroste s 
  počtem opakování pokusu \(n\) a pro velmi velká \(n\) se bude blížit jedné. Musíme ji ale 
  spočítat. Někdo, kdo nečetl předchozí text příliš pečlivě, by mohl navrhnout jednoduchou úvahu: 
  Pravděpodobnost zdaru při každém opakování pokusu je \(p\), pravděpodobnost, že nastane zdar při 
  alespoň jednom z nich tedy musí být, podle pravidla pro sčítání pravděpodobností, \(np\).
  Úvaha je sice jednoduchá, ale zcela chybná. Vidíme to již ze skutečnosti, že při pevné hodnotě 
  \(p\) a dostatečně velkém \(n\) může hodnota \(np\) překročit jedničku, a to nemůže žádná 
  pravděpodobnost udělat. Kde se málo pozorný čtenář dopustil chyby, když chtěl sčítat 
  pravděpodobnosti zdaru při jednotlivých opakováních? Neuvědomil si, že pravidlo součtu 
  pravděpodobností jednotlivých jevů \(A_1\) až \(A_k\) při výpočtu pravděpodobnosti jevu 
  (\(A_1\) nebo \(A_2\) nebo \(\cdots\) nebo \(A_k\)) může použít jedině pro jevy po dvou 
  neslučitelné. Zdar při některém z opakování pokusu však nevylučuje možnost zdaru při jiném 
  pokusu. Pravidlo tedy bylo použito nesprávně. Pravděpodobnost zdaru při alespoň jednom opakování 
  pokusu snadno vypočteme pomocí jevu opačného. Opačný jev znamená, že nenastane zdar ani při 
  jednom opakování pokusu. Jednotlivá opakování jsou nezávislá, proto je pravděpodobnost nezdarů
  při všech opakováních rovna součinu pravděpodobností při jednotlivých z nich, tj. \((1 - p)^n\) . 
  Pravděpodobnost zdaru při alespoň jednom opakováni je pak doplňkem do jedničky, tedy \(1 - (1 - 
  p)^n\). Je vidět, že je tím větší, čím je větší \(n\), a její limita pro \(n\rightarrow \infty\) 
  je rovna jedné. A to je výsledek, který jsme předpověděli.
  \normalsize
\end{example}
    %---------------------------------------------------------------

  \section{Kombinace}\label{mai:IchapIVcsecIV}
    Až dosud jsme se zabývali skupinami vybranými z daných prvků, ve kterých záleželo na pořadí, tj.
    \(k\)-ticemi uspořádanými. Nyní nám půjde o skupiny, ve kterých na pořadí nezáleží; přitom -
    stejně jako v předchozích případech - budeme požadovat, aby v těchto skupinách byl každý z
    daných \(n\) prvků nejvýše jednou, tj. aby se v nich žádný prvek neopakoval. Chceme-li například
    vědět, kolik bude sehráno utkání ve volejbalovém turnaji, jehož se zúčastní \(n\) družstev a
    který se hraje jednokolově systémem každý s každým, pak se vlastně ptáme na počet všech
    neuspořádaných dvojic takových, že v každé se každé družstvo vyskytuje nejvýše jednou; např. pro
    čtyři družstva \(A\), \(B\), \(C\), \(D\) to jsou dvojice \(\{A, B\}\), \(\{A,C\}\), \(\{A,
    D\}\), \(\{B,C\}\), \(\{B, D\}\), \(\{C, D\}\). Proč přitom jde o neuspořádané dvojice, je
    jasné: dvojice \(\{A, B\}\) představuje totéž utkání jako dvojice \(\{B, A\}\). Skupiny tohoto
    typu se nazývají kombinace, přesněji \(k\)-členné kombinace z \(n\) prvků.

    \begin{mdframed}[style=highlight] \(k\)-členná kombinace z \(n\) prvků je neuspořádaná
      \(k\)-tice sestavená z těchto prvků tak, že každý se v ní vyskytuje nejvýše jednou.
    \end{mdframed}  

    Abychom \(k\)-členné kombinace odlišili od \(k\)-členných variací, které jsme zapisovali užitím
    závorek kulatých, budeme je zapisovat pomocí množinových závorek. Vrátíme-li se ještě k
    popisovanému volejbalovému turnaji, představují uvedené dvojice výčet všech dvoučlenných
    kombinací ze čtyř prvků \(A\), \(B\), \(C\), \(D\). Všimněme si, že jde vlastně o všechny
    dvouprvkové podmnožiny množiny \(\{A, B, C, D\}\). Je zřejmé, že pojem \(k\)-členná kombinace z
    \(n\) prvků má stejný význam jako termín \(k\)-prvková podmnožina \(n\)-prvkové množiny. Můžeme
    tedy říci:

    \begin{mdframed}[style=highlight] \(k\)-členná kombinace z \(n\) prvků je \(k\)-prvková
      podmnožina množiny těmito \(n\) prvky určené.
    \end{mdframed}  

    Výhodou definice, kterou jsme uvedli jako první, je, že pojmy variace, permutace a kombinace (a
    to nejen bez opakování, které už jsme poznali, ale i s opakováním, které nás ještě čekají) jsou
    definovány „jednotně", pomocí uspořádaných, resp. neuspořádaných \(k\)-tic. Přesto však je
    užitečné obsah \uv{druhé} definice si zapamatovat.
    
    Určíme nyní počet všech \(k\)-členných kombinací z \(n\) prvků; označíme jej \(C_k(n)\) a k jeho
    určení použijeme již odvozený výsledek pro počet \(V(k,n)\) všech \(k\)-členných variací z \(n\)
    prvků.

    Mějme tedy \(n\) prvků a utvořme z nich všechny \(k\)-členné variace; tyto variace rozdělme do
    skupin tak, aby všechny variace z téže skupiny se lišily pouze pořadím jednotlivých prvků a
    každé dvě variace z různých skupin se lišily aspoň v jednom prvku. Každá taková skupina obsahuje
    k uspořádaných \(k\)-tic, neboť \(k\) prvků lze uspořádat \(k!\) způsoby. Jestliže nám však
    nebude záležet na pořadi, splyne“ všech \(k!\) uspořádaných \(k\)-tic každé skupiny v jedinou
    \(k\)-tici neuspořádanou, tj. v jedinou \(k\)-člennou kombinaci z \(n\) prvků. To však znamená,
    že počet skupin, do nichž jsme původně všechny \(k\)-členné variace rozdělili, je roven počtu
    \(k\)-členných kombinaci z daných \(n\) prvků, a protože pak v každé skupině je \(k!\) variací,
    platí
    \begin{equation*}
      V(k,n) = k!\cdot C_k(n).
    \end{equation*}

    Celý tento postup ilustruje schéma na obr.  pro tříčlenné variace ze čtyř prvků \(a\), \(b\),
    \(c\), \(d\).

    \luagraphic[1]{mai_fig071.png}{Tvorba \(k\)-členných variací a jejich rozdělení do skupin 
                  \cite[s.~26]{calda2008matematika}}{mai:fig071}
    
    Počet \(C_k(n)\) všech \(k\)-členných kombinací z \(n\) prvků je
    \begin{equation*}
      C_k(n) = \dfrac{1}{k!}\cdot V(k,n) = \dfrac{n!}{(n-k)!k!}
    \end{equation*}

    Připomeňme si, že jsme definovali \(0! = 1\), a všimněme si, že zlomek \(\frac{n!}{(n-k)!k!}\)
    má smysl nejen pro \(n\), \(k\) přirozená, , ale i pro \(n\) přirozená a \(k=0\), a dokonce i
    pro \(n = 0\), \(k = 0\); má tedy smysl pro všechna \(n\), \(k\) celá nezáporná, \(k\leq n\).
    Pro tento zlomek se používá symbol \(\binom{n}{k}\), který se čte, \uv{n nad k} a nazývá se
    \textbf{kombinační číslo}; definuje se takto:

    \begin{mdframed}[style=highlight] Pro všechna celá nezáproná čísla \(n\), \(k\), \(k\leq n\), je
      
      \begin{equation*}
        \binom{n}{k} = \dfrac{n!}{(n-k)!k!}.
      \end{equation*}  
    \end{mdframed}

    Výsledek, který jsme odvodili pro počet \(k\)-členných kombinací z \(n\) prvků, lze tedy
    formulovat následovně:
    \begin{mdframed}[style=highlight] Počet \(C_k(n)\) všech \(k\)-členných kombinací z \(n\) prvků
      je 
      \begin{equation*}
        C_k(n) = \binom{n}{k}.
      \end{equation*}  
    \end{mdframed}     
    
    Z toho, co bylo řečeno, také vyplývá, že kombinační číslo určuje počet \(k\)-prvkových podmnožin
    \(n\)-prvkové množiny.

    Všimněme si dále, že z definice kombinačního čísla dostaneme
    \begin{align*}
      \binom{n}{n-k} &= \dfrac{n!}{(n-k)![n-(n-k)]!}             \\
                      &= \dfrac{n!}{(n-k)!k!} = \binom{n}{k}.
    \end{align*}  
    Odvodili jsme tak užitečnou větu:


  \section{Variace s opakováním}\label{mai:IchapIVcsecV}
  \section{Permutace s opakováním}\label{mai:IchapIVcsecVI}
  \section{Kombinace s opakováním}\label{mai:IchapIVcsecVII}
  \section{Vlastnosti kombinačních čísel}\label{mai:IchapIVcsecVIII}
  \section{Binomická věta}\label{mai:IchapIVcsecIX}    
  
    Uvedené vlastnosti kombinačních čísel je možno ilustrovat na následujícím schématu, které se
    nazývá \textbf{Pascalův trojúhelník} \emph{(B. PASCAL, 1623-1662, francouzský filozof, matematik
    a fyzik, jeden ze zakladatelů počtu pravděpodobnosti; z fyziky známe Pascalův zákon)}.

    \begin{align*}
      \begin{array}{c} 
        \binom{0}{0}                                                                          \\
        \binom{1}{0}\quad \binom{1}{1}                                                        \\  
        \binom{2}{0}\quad \binom{2}{1}\quad \binom{2}{2}                                      \\ 
        \binom{3}{0}\quad \binom{3}{1}\quad \binom{3}{2}\quad \binom{3}{3}                    \\ 
        \binom{4}{0}\quad \binom{4}{1}\quad \binom{4}{2}\quad \binom{4}{3}\quad  \binom{4}{4} \\ 
        \hdotsfor{1}                                                                          \\
        \binom{n}{0}\; \binom{n}{1}\; \binom{n}{2}\; \cdots\; 
        \binom{n}{n-2}\; \binom{n}{n-1}\; \binom{n}{n}                                        \\ 
      \end{array}
    \end{align*} Jestliže kombinační čísla v tomto schématu vyčíslíme, dostaneme Pascalův
    trojúhelník ve tvaru:

    Všimněme si symetrického rozmístění stejných čísel vzhledem k ose souměrnosti Pascalova
    trojúhelníku. Je to způsobeno tím, že čísla \(\binom{n}{k}\) a \(\binom{n}{n-k}\), která se sobě
    rovnají, jsou stejně vzdálená od středu každého řádku.
  
  \section{Sbírka příkladů}\label{mai:IchapIVcsecX}
    \luagraphic[0.8]{mai_fig076.png}{Sbírka příkladů}{mai:fig076}
    \subsection{Počítání s faktoriály}\label{mai:IchapIVcsecXssecI}
      %--Sbírka příkladů s faktoriály---------------------------------
      % !TeX spellcheck = cs_CZ
\begin{mdframed}[style=mdexam]
  \begin{example}\label{mai:exam105a}
    Zjednodušte výrazy \cite[s.~22]{calda2008matematika} :
    \begin{enumerate}[label=\emph{\alph*}),noitemsep]
      \item \(\dfrac{(n+1)!}{n!} - \dfrac{(2n)!}{(2n+1)!} + \dfrac{(3n-1)!}{(3n-2)!}\)
      \item \(\dfrac{(n+1)!}{(n!)^2} - \dfrac{n!}{[(n-1)!]^2}\)
    \end{enumerate}
    V obou případech zkrátíme jednotlivé zlomky a výsledný výraz upravíme:
    \begin{equation*}
      \frac{(n+1)\cancel{n!}}{\cancel{n!}} - \frac{\cancel{(2n)!}}{(2n+1)\cancel{(2n)!}} +
      \frac{(3n-1)\cancel{(3n-2)!}}{\cancel{(3n-2)!}}
    \end{equation*}
    \begin{equation*}
      (n+1) - \frac{1}{(2n+1)} + (3n-1) = \frac{8n^2 + 4n -1}{2n + 1}
    \end{equation*}
    Podobně pro druhý výraz
    \begin{equation*}
      \frac{(n+1)\cancel{n!}}{n!\cancel{n!}} + \frac{n\cancel{(n-1)!}}{(n-1)!\cancel{(n-1)!}} =
      \frac{(n+1)+n\cdot n}{n!}
    \end{equation*}
  \end{example}

  \begin{example}\label{mai:exam105b}
    Dokažte, že pro všechna přirozená čísla \(n\) platí
    \begin{equation*}
      n!(n+3)! > (n+1)!(n+2)!
    \end{equation*}
    Výraz na levé straně nerovnice upravíme:
    \begin{equation*}
      \frac{(n+1)!}{n+1}\cdot(n+3)(n+2)! = (n+1)!(n+2)!\frac{n+3}{n+1}
    \end{equation*}   
    Protože pro všechna \(n\in\naturalset\) je zlomek \(\frac{n+3}{n+1}>1\), dokazovaný vztah
    \textbf{platí}  
  \end{example}
\end{mdframed}
      %---------------------------------------------------------------   
    \subsection{Užití Binomické věty}\label{mai:IchapIVcsecXssecII}
      %--Sbírka příkladů užití Binomické věty-------------------------
      % !TeX spellcheck = cs_CZ
\begin{mdframed}[style=mdexam]
  \begin{example}\label{mai:exam106a}
    Užitím binomické věty určete a) \((a+b)^4\), b) \((a+b)^5\) pro libovolná čísla
    \(a,b\in\cmplxset\). \cite[s.~299]{polak1991matematika}\newline

    Řešení: Podle binomické formule je
    \begin{align*}
      (a+b)^4 &= a^4 + \binom{4}{1}a^3b + \binom{4}{2}a^2b^2 + \binom{4}{3}ab^3 + \\
              &+ b^4 = a^4 + 4a^3b + 6a^2b^2 + 4ab^3 + b^4
    \end{align*}
    \begin{align*}
      (a+b)^5 &= a^5 + \binom{5}{1}a^4b + \binom{5}{2}a^3b^2 +                  \\ 
              &+ \binom{5}{3}a^2b^3  +    \binom{5}{4}ab^4 + b^5 = a^5 + 5a^4b  \\
              &+ 10a^3b^2 + 10a^2b^3 + 5ab^4 + b^5
    \end{align*}
  \end{example}  
\end{mdframed}
      %---------------------------------------------------------------
%---------------------------------------------------------------------------------------------------