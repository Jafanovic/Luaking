% !TeX spellcheck = cs_CZ
\begin{mdframed}[style=mdexam]
  \begin{example}\label{TEO:exam020}
    Analyzujte obvod podle obrázku \ref{teo:fig080}. Vypočtěte proudy v jednotlivých větvích. Napětí
    zdrojů: \(U_{01} = \SI{15}{\V}\), \(U_{02} = \SI{10}{\V}\), \(U_{03} = \SI{20}{\V}\), odpory
    \(R_1 = \SI{10}{\ohm}\), \(R_2 = \SI{15}{\ohm}\), \(R_3 = \SI{30}{\ohm}\), \(R_4 =
    \SI{20}{\ohm}\), \(R_5 = \SI{10}{\ohm}\),
    \newline 
    \textbf{Řešení:}\newline V předchozích kapitolách jsme se seznámili se dvěma základními zákony
    analýzy elektrických obvodů - Kirchoffovými zákony: V obvodu na obrázku \ref{teo:fig080} mamé
    celkem 3 uzly. 
    \begin{itemize}[leftmargin=2em]
      \item Nejprve jeden z nich zvolíme jako referenční  \(B_0\) (zem s nulovým potenciálem). 
      \item Napětí v \(n-1\) zbývajících uzlech \(B_1\) a \(B_2\) vyznačíme jako uzlová napětí
            \(U_{B1}\) a \(U_{B2}\). Napětí v každém z těchto \(n-1\) uzlů je vztaženo vzhledem k
            referenčnímu uzlu. 
      \item Pro takto očíslované uzly sestavíme rovnice podle prvního Kirchhoffova zákona a jejich
            řešením stanovíme napětí označených uzlů.
            \begin{subequations}\label{teo:eq149}
              \begin{alignat}{2}
                B_1&: -I_1 + I_2 + I_3 &&=0,  \label{teo:eq149a}   \\
                B_2&: -I_1 + I_4 + I_5 &&=0   \label{teo:eq149b} 
              \end{alignat}
            \end{subequations}  
    \end{itemize}
    Uvedené proudy v soustavě rovnic \ref{teo:eq149} je možné vyjádřit pomocí známých uzlových
    napětí:
    \begin{subequations}\label{teo:eq150}
      \begin{alignat}{2}
        &B_1: - \dfrac{U_{01} - U_{B1}}{R_1} 
              + \dfrac{U_{B1} - U_{02}}{R_2} 
              + \dfrac{U_{B1} - U_{B2}}{R_3}       &&=0,  \label{teo:eq150a}   \\
        &B_2: - \dfrac{U_{B1} - U_{B2}}{R_3} 
              + \dfrac{U_{B2}}{R_4}
              + \dfrac{U_{B2} - U_{03}}{R_5}       &&=0   \label{teo:eq150b} 
      \end{alignat}
    \end{subequations}  
    {\centering
    \captionsetup{type=figure}
    \luafigure[1]{teo_fig080.pdf}
    \captionof{figure}{Ilustrace k příkladu \ref{TEO:exam020}}   
    \label{teo:fig080}
    \par}
    Rovnice lze zapsat v maticovém tvaru \(\matr{A}\vec{x} = \vec{b}\). Matice soustavy má hodnost 2
    a její koeficienty jsou:
    \begin{equation*}
      \matr{A} = 
        \left[
          \begin{matrix}
            \dfrac{1}{R_1} +\dfrac{1}{R_2} + \dfrac{1}{R_3}  &  -\dfrac{1}{R_3}        \\
                           -\dfrac{1}{R_3}  &   \dfrac{1}{R_3} + \dfrac{1}{R_4} + \dfrac{1}{R_5}
          \end{matrix}
        \right]
    \end{equation*}
    Vektor neznámých obsahuje uzlová napětí \(\vec{x} = \left[\begin{matrix} U_{B1} \\ U_{B2}
    \end{matrix}\right],\) a zatímco vektor pravých stran obsahuje pouze proudy \(\vec{b} =
    \left[\begin{matrix} \frac{U_{01}}{R_1} + \frac{U_{02}}{R_2} \\ \frac{U_{02}}{R_5}
    \end{matrix}\right]\). Pro vyřešení soustavy rovnic použijeme následující skript v Matlabu:
  \end{example}
  %---------------------------------------------------------------
  \lstinputlisting[%
  style=luaMatlabStyle,
  caption={Výpis programu \ref{TEO:exam020}.}
  ]{../src/TEO/matlab/teo_exam020.m}
  %--------------------------------------------------------------- 
\end{mdframed}