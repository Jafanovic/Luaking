% !TeX spellcheck = cs_CZ
\begin{example}
  Stanovte intenzitu magnetického pole $H=f(r)$ dlouhého dutého válcového vodiče podle obr.
  \ref{TEMP:fig_pole_duty_valec} při rovnoměrném rozložení proudu $I$ po průřezu. 
  
   {\centering
    \captionsetup{type=figure}
    \includegraphics[width=0.6\linewidth]{duf_duty_valec_H.pdf}
    \captionof{figure}{K příkladu stanovení intenzity magnetického pole dlouhého dutého válcového 
               vodiče protékaného proudem}
    \label{TEMP:fig_pole_duty_valec}
  \par}
  
  Vodič s rovnoměrně rozloženým proudem podle obr. \ref{TEMP:fig_pole_duty_valec} je rotačně
  souměrný podle své osy a tedy i jeho magnetické pole je souměrné. Silové čáry jsou soustředné
  kružnice, vektor $\vr{H}$, jenž má směr tečny ke kružnici, je po celé délce kružnice stejně
  velký. Lze tedy snadno použít integrálního tvaru 1. MR (\textbf{zákon celkového proudu})
  
  Pro body ležící vně vodiče obepíná kruhová integrační dráha (vedená po silové čáře 1) celý
  proud vodiče $I$ a platí
  \begin{equation}\label{TEMP:eq_1MR_duty_valec}
    \oint_\mathcal{C}\vr{H}d\vr{l} = H\cdot 2\pi r = I
  \end{equation}
  takže intenzita pole je
  \begin{equation}\label{TEMP:eq_H_duty_valec}
    H = \frac{I}{2\pi r}
  \end{equation}
  
  Ve stěně dutého magnetického vodiče jsou silové čáry rovněž kružnice, neboť magnetické pole
  je i zde souměrné. Tyto siločáry však obepínají jen část proudu $I'$ vodiče pro oběh siločáry
  2 platí
  \begin{equation}\label{TEMP:eq_1MR_uvnitr_valce}
    \oint_\mathcal{C}\vr{H}d\vr{l} = H\cdot 2\pi r = I' = \pi(r^2-r_1^2)J
  \end{equation}
  kde $J$ je hustota proudu ve vodiči
  \begin{equation}\label{TEMP:eq_J_duty_valec}
    J = \frac{I}{S}= \frac{I}{\pi(r_2^2-r_1^2)}
  \end{equation}
  Ve stěně vodiče je tedy intenzita pole
  \begin{equation}\label{TEMP:eq_H_uvnitr_valce}
    H = \frac{I}{2\pi r}\frac{r^2-r_1^2}{r_2^2-r_1^2}
  \end{equation}
  V dutině vodiče je intenzita rovna nule. Vzhledem k souměrnosti pole by i zde muselo platit
  $\oint_\mathcal{C}\vr{H}d\vr{l} = H\cdot 2\pi r$. Protože dráha s poloměrem $r<r_1$ neobepíná
  žádný proud, je $\oint_\mathcal{C}\vr{H}d\vr{l} = 0$ a tedy musí byt $H = 0$.
\end{example}    
