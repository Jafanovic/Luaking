% !TeX spellcheck = cs_CZ
\begin{mdframed}[style=mdexam]
  \begin{example}\label{teo:exam019}
    Mějme nabitý deskový kondenzátor \(C\) (obr. \ref{teo:fig019a}). Zvětšíme jeho kapacitu,
    například tím, že zvětšíme plochu jeho elektrod, nebo připojíme paralelně druhý stejné
    velikosti, viz obr. \ref{teo:fig019b}. Otázka zní, jak velká energie bude uložena v
    elektrostatickém poli obou kondeznátorů? Bude energie po rozdělení náboje mezi oba kondenzátory
    rovna původní energii nabitého kondenzátoru? Pokud ne, vysvětlete kam se část energie
    transformovala. 
    
    {\centering
      \captionsetup{type=figure}
      \subcaptionbox{\label{teo:fig019a}}{\luafigure[0.15]{teo_fig019a.png}}           
      \hspace{1em}
      \subcaptionbox{\label{teo:fig019b}}{\luafigure[0.35]{teo_fig019b.png}}
      \hspace{1em}
      \subcaptionbox{\label{teo:fig020}}{\luafigure[0.30]{teo_fig020.png}}
      \captionof{figure}{K příkladu \ref{teo:exam019}: a) Nabitý kondenzátor s rovnoběžnými
      rovinnými elektrodami; b) Rozložení náboje na obou kondenzátorech stejné velikosti; c)
      Rezistor \(R\) představuje ztráty, které nebyly v obvodu na obrázku \ref{teo:fig019b}
      předpokládány}
      \label{teo:fig019}
    \par}
    
    Je-li dielektrikum kondenzátoru lineární, pak pro energii elektrického pole akumulované v
    nabitém kondenzátoru platí. Podrobněji například v kapitole \ref{fyz:IIchapVsecXIX}.
    \begin{equation}
      W = \frac{1}{2}CU^2 \;\text{nebo}\; W = \frac{1}{2}\frac{Q^2}{C} \;\text{kde}\; 
      C = \frac{Q}{U}
    \end{equation}
    Předpokládejme ustálený stav po připojení druhého kondenzátoru, jak je znázorněno na obr.
    \ref{teo:fig019b}. V obvodu nepředpokládáme přítomnost odporu, který by způsobil ztrátu energie,
    vyzářené v podobě tepla. Kapacita je dvojnásobná a náboj zůstal stejný. Na každém kondenzátoru
    tedy očekáváme polovinu původního náboje. Sečteme-li energii uloženou v elektrických polích obou
    kondenzátorů dostaneme
    \begin{align*}
      W^* &= \frac{1}{2}\frac{(\frac{1}{2}Q)^2}{C} + \frac{1}{2}\frac{(\frac{1}{2}Q)^2}{C}   \\
          &= \frac{(\frac{1}{2}Q)^2}{C} =\frac{1}{4}\frac{Q^2}{C}                 
           \xrightarrow[C\rightarrow2C]{}
            \frac{1}{2}\frac{Q^2}{(2C)} = \frac{1}{2}W 
    \end{align*}
    Kupodivu, polovina energie prostě chybí a jelikož platí zákon zachování energie\footnote{viz
    partie Fyzika \ref{part:FYZI}, kapitola \ref{fyz:IchapII})}, nezbývá nic jiného než uznat, že
    elektrický obvod dle \ref{teo:fig019b}, nemodeluje fyzikální problém dost věrně. Tím jsme
    dospěli k závěru, že je nutné do obvodu dodat rezistor, tak jak je znázorněno na obrázku
    \ref{teo:fig020}.
    
    Abychom mohli určit tepelné ztráty na rezistoru dané integrálem \(\int_{0}^{\infty}
    Ri^2(t)\dd{t}\), nejdříve sestavíme jednoduchou diferenciální rovnici prvního řádu aplikací II.
    Kirchhoffova zákona, ze které odvodíme vzorec pro časovou závislost proudu \(i(t)\). 
    \begin{align*}
      \frac{Q_0 - Q}{C} - Ri(t) - \frac{Q}{C}           &= 0 \quad/\der{ }{t}             \\
      -\frac{i(t)}{C} - R\der{i(t)}{t} - \frac{i(t)}{C} &= 0                              \\
                                          \der{i(t)}{t} &= - \frac{2}{RC}i(t) \quad/\int  \\
                                                   i(t) &= I_0e^{-\frac{2}{RC}t}
    \end{align*}
    Nyní můžeme stanovit energii disipované na rezitoru \(R\)
    \begin{align*}
      W   &= \int_{0}^{\infty}Ri^2(t)\dd{t} = RI_0^2\int_{0}^{\infty}e^{-\frac{4}{RC}t}\dd{t}   \\
      \shortintertext{Do integrované funkce dosadíme novou proměnnou \(u = \frac{4}{RC}t\), \(\dd{u} 
                      = \frac{4}{RC}\dd{t}\), \(\dd{t} = \frac{RC}{4}\dd{u}\)}
          &= RI_0^2\int_{0}^{\infty}e^{-u}\frac{RC}{4}\dd{u} 
          = R^2I_0^2\frac{C}{4}\underbrace{\int_{0}^{\infty}e^{-u}\dd{u}}_1 
      \end{align*}
    Jelikož platí \(I_0 = \frac{U}{R}=\frac{Q}{CR}\) dostaneme po dosazení
    \(\cancel{R^2}\frac{Q^2}{C^2\cancel{R^2}}\frac{C}{4} = \frac{Q^2}{4C}= \frac{1}{2}W\). Nyní je
    vše v pořádku. Druhá polovina energie je disipována na rezistoru a navíc z výsledku vyplývá, že
    vůbec nezávisí na \(R\)!
  \end{example}
\end{mdframed}