% !TeX spellcheck = cs_CZ
%---------- Ponorný vařič:
\begin{example}
  Za jakou dobu uvede ponorný vodič o příkonu $600\ W$ do varu $1\ l$ vody o počáteční teplotě 
  $20°C$. Uvažujte měrnou tepelnou kapacitu vody $c = 4200\ J\cdot kg^{-1}\cdot K^{-1}$. Výměnu 
  tepla s okolím neuvažujte. \newline 
  \textbf{Řešení:}\newline Pro var vody bude zapotřebí tepla dle rovnice $Q  = m\cdot c\cdot(T_2 - 
  T_1)$. Potřebná elektrická práce je $Q_e = P\cdot t = U\cdot I\cdot t$ a tedy dobu ohřevu 
  stanovíme z rovnice:
  \begin{align*}
  P\cdot t &= m\cdot c\cdot(T_2 - T_1)               \nonumber  \\
  t &= \frac{m\cdot c}{P}\cdot(T_2 - T_1)     \nonumber  \\
  t &= \frac{1\cdot 4200}{600}\cdot(100 - 20) = 560\ s
  \end{align*}         
\end{example}