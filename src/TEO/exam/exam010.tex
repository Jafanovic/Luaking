% !TeX spellcheck = cs_CZ
%---------- Ponorný vařič:
\begin{mdframed}[style=mdexam]
  \begin{example}\label{TEO:exam010}
    Za jakou dobu uvede norný vodič o příkonu \SI{600}{\W} do varu \SI{1}{\litre} vody o počáteční
    teplotě $\SI{20}{\degreeCelsius}$. Uvažujte měrnou tepelnou kapacitu vody $c =
    \SI{4200}{\joule\per\kg\per\K}$. Výměnu tepla s okolím neuvažujte.
    \newline 
    \textbf{Řešení:}\newline Pro var vody bude zapotřebí tepla dle rovnice $Q  = m\cdot c\cdot(T_2 -
    T_1)$. Potřebná elektrická práce je $Q_e = P\cdot t = U\cdot I\cdot t$ a tedy dobu ohřevu
    stanovíme z rovnice:
    
    {\centering
    \captionsetup{type=figure}
    \luafigure[0.3]{teo_fig033.png}
    \captionof{figure}{Ilustrace k příkladu \ref{TEO:exam010}}
    \label{teo:fig033}
    \par}

    \begin{align*}
      P\cdot t &= m\cdot c\cdot(T_2 - T_1)                                               \\
             t &= \frac{m\cdot c}{P}\cdot(T_2 - T_1)                                     \\     
               &= \frac{\SI{1}{\kg}\cdot\SI{4200}{\joule\per\kg\per\K}}{\SI{600}{\W}}
                \cdot(\SI{100}{\degreeCelsius} - \SI{20}{\degreeCelsius})                \\ 
               &= \SI{560}{\s}.
    \end{align*}         
  \end{example}
\end{mdframed}