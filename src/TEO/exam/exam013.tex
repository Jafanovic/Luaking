% !TeX spellcheck = cs_CZ
\begin{mdframed}[style=mdexam]
  \begin{example}\label{TEMP:ex_koax_H}
    Stanovte intenzitu magnetického pole dlouhého přímého souosého kabelu podle obr.
    \ref{teo:fig066}. Středním vodičem (\emph{žílou}) prochází proud $I$ a týž proud
    opačného smyslu prochází vnějším vodičem (\emph{pláštěm}). Proudy jsou rovnoměrně rozloženy po
    průřezech vodičů. Nakreslete graf průběhu $H = f(r)$ \cite[s.~92]{Dufek1970},
    \cite[s.~195]{Kotlan1999}.
    
    {\centering
    \luafigure[0.45]{teo_fig066a.pdf}     \hspace{1em}
    \luafigure[0.45]{teo_fig066b.pdf}
    \captionsetup{type=figure}  
    \captionof{figure}{K příkladu stanovení intenzity magnetického pole dlouhého souosého kabelu
      protékaného proudem: a) náčrt; b) $H=f(r)$}           
    \label{teo:fig066}
    \par}
 
    \textbf{Řešení}: \newline Rovnici \ref{TEMP:eq_1MR_v_hom_p} aplikujeme na jednotlivé intervaly
    osově souměrného stacionárního magnetického pole, přičemž se prakticky jedná o superpozici dvou
    polí. V oblasti $r<r_2$ se uplatňuje pouze pole vnitřního válcového vodiče (žíly), pro $r>r_2$
    přistupuje souosé pole vnějšího trubkového vodiče.
    \begin{itemize}
      \item Pro oblast $r<r_1$ je vzhledem k 
            \begin{align*}
                % \nonumber to remove numbering (before each equation)
                \dd{I}&= \vr{J}d\vr{S} \\
                I(r)  &= \int_S dI = \int_S \vr{J}d\vr{S} = \int_S J\cos\beta dS \\
                      &= \left\lvert\begin{array}{cc}
                                \beta = 0 & H = \text{konst}   \\
                              S = \pi r^2 & dS = 2\pi r\dd{r}  \\
                              \end{array}
                        \right\rvert                                           \\
                      &= J\int_0^r 2\pi rdr = J\pi r^2
            \end{align*}
            hledané řešení 1. MR dáno 
            $$\oint_{\mathcal{c}}\vr{H}d\vr{l} = H_1 2\pi r = I(r) = J\pi r^2$$ kde celková proudová
            hustota je  $$J = \frac{I}{\pi r_1^2}$$ a tedy
            $$H_1 = \frac{I}{2\pi r_1^2}\cdot r$$
            
      \item Pro oblast $r_2>r>r_1$ řešíme v podstatě pole vně osamoceného válcového vodiče $I(r)$ a
            tedy $$H_2 = \frac{I}{2\pi r}$$
      \item Pro $r>r_3$ je magnetické pole vytvářeno celým proudem žíly $I$ a příslušnou částí
            proudu pláště $J\pi(r^2 - r_2^2)$, kde proudová hustota $$J =
            \frac{I}{\pi(r_3^2-r_2^2)}$$ má opačnou orientaci oproti proudové hustotě žíly. Pak 
            \begin{align*}
              I(r)                             &= I - I\frac{r^2-r_2^2}{r_3^2-r_2^2} \\
              \oint_{\mathcal{c}}\vr{H}d\vr{l} &= H_32\pi r = I(r)                   \\          
              H_3                              &= \frac{I}{2\pi r}\left(1 - 
              \frac{r^2 - r_2^2}{r_3^2 - r_2^2}\right) 
            \end{align*}
            Stejný výsledek dostaneme superpozicí opačně orientovaných polí
            $$H_3 = H'_3 - H''_3 = \frac{I}{2\pi r} - \frac{I}{2\pi r}\left(\frac{r^2 - r_2^2}{r_3^2
            - r_2^2}\right)$$. 
    \end{itemize}
    Průběh $H(r)$ je na obr. \ref{teo:fig066}.
  \end{example}
\end{mdframed}