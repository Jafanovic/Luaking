% !TeX spellcheck = cs_CZ
\begin{example}\label{TEO:exam007}
  Pro harmonickou veličinu, určete efektivní hodnotu, střední hodnotu, činitele tvaru, činitele
  výkyvu a činitele plnění \newline
  \textbf{Řešení:} Efektivní hodnota je:
  \begin{align}
    V &= \sqrt{\frac{1}{T}\int_0^TV_m^2\cos^2{(\omega t + \varphi)}\,dt}    \nonumber  \\
      &= \sqrt{\frac{1}{T}\int_0^TV_m^2\sin^2{(\omega t + \varphi)}\,dt} = 
         \frac{1}{\sqrt{2}}V_m \doteq 0.707 V_m   
  \end{align}
  Podrobný výpočet tohoto integrálu pomocí substituce $\omega t + \varphi=\dfrac{\alpha}{2}$ je
  poněkud zdlouhavější:
  \begin{align*}
      \omega t + \varphi=\dfrac{\alpha}{2}   
    & \rightarrow  2(\omega t + \varphi) = \alpha      \\ 
      \omega dt = \frac{1}{2}d\alpha         
    & \rightarrow dt = \frac{1}{2\omega}d\alpha
  \end{align*}
  Nesmíme zapomenout přepočítat meze $\alpha_d|_{t=0}=2\varphi$ a $\alpha_h|_{t=T} = 
  4\pi+2\varphi$ nového integrálu.
  \begin{align*}
    V^2  &= \frac{V_m}{2T\omega}\int_{\alpha_d}^{\alpha_h}\cos^2\frac{\alpha}{2}\,d\alpha    \\
         &= \frac{V_m}{4\pi}\int_{\alpha_d}^{\alpha_h}\frac{1+\cos\alpha}{2}\,d\alpha  
          = \frac{V_m}{4\pi}\left(\frac{\alpha}{2}|_{\alpha_d}^{\alpha_h}
          + \frac{1}{2}\sin\alpha|_{\alpha_d}^{\alpha_h}\right)                              \\
         &= \frac{V_m}{4\pi}\left(2\pi+\varphi-\varphi 
          + \frac{1}{2}\sin(4\pi+2\varphi)
          - \frac{1}{2}\sin(2\varphi)\right) = \frac{V_m}{2}.  
  \end{align*}  
  Při zjednodušování integrálu je užito známého goniometrického vzorce \(\cos^2\dfrac{\alpha}{2} = 
  \dfrac{1+\cos\alpha}{2}\) a faktu \(\sin(x+2k\pi)=\sin x\)
  
  Střední hodnota kladné půlvlny je 
  \begin{align*}
    V_s &= \frac{2}{T}\int\limits_{-S\frac{T}{4}-
           \frac{\varphi}{\omega}}^{\frac{T}{4}-
           \frac{\varphi}{\omega}}{V_m\cos(\omega t +\varphi)}\,dt                            \\
        &= \frac{2}{T}\int\limits_{-\frac{\varphi}{\omega}}^{-\frac{T}{2}-
           \frac{\varphi}{\omega}}{V_m\sin(\omega t +\varphi)}\,dt                            \\
        &= \frac{2}{\pi}V_m \doteq 0,637V_m
  \end{align*}
  činitele tvaru, výkyvu a plnění jsou 
  \begin{align*}
    \beta  &=\frac{V}{V_s}   =\frac{\pi}{2\sqrt{2}}=1,111, \\ 
    \gamma &=\frac{V_m}{V}   =\sqrt{2}\doteq1.414,         \\
    \alpha &=\frac{V_s}{V_m} =\frac{2}{\pi}\doteq0,637 
  \end{align*}
\end{example}