% !TeX spellcheck = cs_CZ
%---------- Driftová rychlost elektroknů ve vodiči:   
\begin{example}\label{TEO:exam008} \emph{Driftová rychlost elektronů ve vodiči:} Vodičem z 
jednomocné mědi o
  průřezu $S_0 = 1\ mm^2$ prochází elektrický proud $I = 5\ A$. Vypočtěte:
  \begin{itemize}\addtolength{\itemsep}{-0.5\baselineskip}
    \item počet volných elektronů v jednotkovém objemu Cu,
    \item úhrnný náboj volných elektronů v jednotkovém objemu,
    \item driftovou rychlost volných elektronů při proudu $I$.
  \end{itemize}
  Měď má poměrnou atomovou hmotnost $A_r = 63,54$ a hustotu\footnote{Pro hustotu budeme používat 
  alternativní značku $s$, s ohledem na kolizi značky $\rho$, jež označuje hustotu náboje.} $s = 
  8,93\cdot10^3\ kg\cdot m^{-3}$.\newline
  
  \textbf{Řešení:}
  \begin{itemize}
    \item Jeden mol mědi o molové hmotnosti $M = 0,06354\ kg\cdot mol^{-1}$ a o molovém
          objemu 
          \begin{align*}
            V_m = \frac{M}{s} 
               &= \frac{63,54\cdot10^{-3}\ kg\cdot mol^{-1}}{8,93\cdot10^3\ kg\cdot m^{-3}} \\
               &= 7,12\cdot10^{-6}\ m^3\cdot mol^{-1}
          \end{align*}
          obsahuje $N_A = 6,0221\cdot10^{23}$ jednoatomových molekul \emph{Cu} na jeden mol,
          z nichž každý má volný jeden (valenční) elektron. Tedy počet volných elektronů v
          jednotkovém objemu je 
          \begin{align*}
            n_0 &= \frac{N_A}{V_m} = \frac{sN_A}{M}                                           \\
                &= \frac{6,0221\cdot10^{23}\ mol^{-1}}{7,12\cdot10^{-6}\ m^{3}\cdot mol^{-1}} 
                = 8,46\cdot10^{28}\ m^{-3}.
          \end{align*}  
    \item Úhrnný náboj volných elektronů v jednotkovém objemu mědi je $$Q_v = -e\cdot n_0 =
    -1,36\cdot10^{10}\ Cm^{-3}.$$
    \item Velikost driftové rychlosti určíme ze vztahu $I = -en_0v_dS_0 = - Q_v v_d S_0$ tj.
    \begin{align}
    v_d &= \left|\frac{I}{Q_v S_0}\right|                             \nonumber \\ 
    &= \frac{5}{1,36\cdot10^{10}\cdot1\cdot10^{-6}}\frac{C\cdot   
      s^{-1}}{Cm^{-3}\cdot m^2}                                  \nonumber \\
    &= 3676\cdot10^{-4}\ \frac{m}{s} = 0,3676\ \frac{mm}{s}.      \nonumber 
    \end{align}
  \end{itemize}
  Z provedených výpočtů si můžeme udělat názor o mikroskopických poměrech v kovových
  vodičích: počet volných nositelů náboje - elektronů a jejich úhrný náboj v jednotkovém
  objemu je značný a proto driftová rychlost elektronů potřebná k vyvolání proudu běžné
  velikosti v drátových vodičích je nesmírně malá (doslova hlemýždí).
\end{example}  
