% !TeX spellcheck = cs_CZ
\begin{example}\label{TEO:exam016}
  Vyjádřeme pomocí \(y\)-parametrů následující přenosové parametry dvojbranu: vstupní impedance při 
  výstupu naprázdno \(Z_{vst_0}\), \(K_{I_{12_K}}\), \(Z_{12_0}\). 
  
  Řešení:
  \begin{itemize}\addtolength{\itemsep}{-0.5\baselineskip}
    \item \(Z_{vst_0}\): \(\mathbb{Y}\)-matici doplníme třetí rovnicí \(I_2=0\) (podmínka 
    definující výstup naprázdno):
    \begin{subequations}\label{ES:eq_topol10}
      \begin{align}
        I_1 &= y_{11}U_1 - y_{12}U_2  \\
        I_2 &= y_{21}U_1 - y_{22}U_2  \\
        I_2 &= 0
      \end{align}
    \end{subequations}
    Záporná znaménka v \(\mathbb{Y}\)-matici (tj. v prvních dvou rovnicích) jsou správná, protože 
    směr výstupního proudu \(I_2\) na obr. \ref{es:fig_patocka_topol08} je zvolen v souladu s 
    realitou, tj. v souladu s napětím \(U_2\). Pokud je totiž dvojbran zatížen \emph{pasivní} 
    zátěží, pracující ve \emph{spotřebičovém} režimu, proud \(I_2\) musí z výstupní svorky 
    dvojbranu vytékat, nikoli do ní vtékat\footnote{Směry obou proudů \(I_1\), \(I_2\) lze volit 
    libovolně. V literatuře se obvykle směry proudů volí tak, že oba vtékají do dvojbranu. Důvod je 
    pouze formální, pak totiž budou všechny čtyři maticové koeficienty kladně. Výsledkem je ale 
    nerealistický stav na výstupní bráně, tj. na pasivní zátěži. Proto nebudeme toto značeni 
    používat a směry všech veličin budeme systematicky volit podle obr. 
    \ref{es:fig_patocka_topol08}. Z psychologického hlediska tato volba velmi usnadní budoucí 
    analýzu transformátoru.}. Všimněme si, že jsme získali soustavu tří rovnic o čtyřech neznámých 
    veličinách \(U_1\), \(U_2\), \(I_1\), \(I_2\). Ta je pro \emph{jednotlivé} neznámé neřešitelná. 
    Je však řešitelná pro poměry dvou libovolných proměnných. Ze soustavy lze po algebraických 
    úpravách získat hledanou vstupní impedanci při výstupu naprázdno:
    \begin{equation}\label{ES:eq_topol11}
    Z_{vst_0} = \frac{U_1}{I_1}\Bigg|_{I_2=0} 
    = \frac{y_{22}}{y_{22}y_{11} - y_{12}y_{21}}.
    \end{equation}
    \item \(K_{I_{12_K}}\): \(\mathbb{Y}\)-matici doplníme třetí rovnicí \(U_1 = 0\) (podmínka 
    definující vstup nakrátko). Jedná se o přenos ve zpětném směru (\(\leftarrow\)). Při zpětném 
    přenosu tečou oba proudy na obr. \ref{es:fig_patocka_topol08} obráceným směrem, proto musí mít 
    přiřazena záporná znaménka:
    \begin{equation}\label{ES:eq_topol12}
    K_{I_{12_K}} = \frac{-I_1}{-I_2}\Bigg|_{U_1=0} 
    = \frac{I_1}{I_2}\Bigg|_{U_1=0}
    = \frac{y_{12}}{y_{22}}.
    \end{equation}          
    \item \(Z_{12_0}\): \(\mathbb{Y}\)-matici doplníme třetí rovnicí \(I_1=0\) (podmínka  
    definující vstup naprázdno). Při zpětném přenosu (\(\leftarrow\)) tečou oba proudy na obr. 
    \ref{es:fig_patocka_topol08} obráceným směrem, proto musí mít přiřazena záporná znaménka:
    \begin{equation}\label{ES:eq_topol13}
    Z_{vst_0} = \frac{U_1}{-I_2}\Bigg|_{I_1=0} 
    = \frac{y_{12}}{y_{21}y_{12} - y_{11}y_{22}}.
    \end{equation}
  \end{itemize}
  Tytéž přenosové parametry \(Z_{vst_0}\), \(K_{I_{12_K}}\), \(Z_{12_0}\) je možno vyjádřit pomocí 
  \(z\)-parametrů nebo \(h\)-parametrů. Podobným způsobem by bylo možno doplnit všechna prázdná 
  pole v tab. \ref{ES:tab_topol01}.
\end{example}
