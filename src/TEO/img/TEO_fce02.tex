%\documentclass{article}
%\usepackage{tikz}
%\usetikzlibrary{decorations.markings}
%\usetikzlibrary{intersections}
%\usetikzlibrary{calc}



% \begin{document}
  \begin{figure}[htb]  
        \newcommand{\MyPath}{%
            (18.309015,50.53919507)  .. controls
            (18.309015,50.53919507)  and (21.791754,59.15092207)  ..
            (28.507151,63.26291307)  .. controls
            (31.391329,65.02896007)  and (35.896102,67.50958007)  ..
            (39.847327,68.71973407)  .. controls
            (43.336244,69.78829507)  and (45.883,69.20748807)     ..
            (48.387485,71.78723507)  .. controls
            (51.186791,74.67066107)  and (53.141632,78.26714507)  ..
            (55.148632,82.06152507)  .. controls
            (58.500874,88.39917807)  and (60.888255,94.05049107)  ..
            (66.110428,100.96156507) .. controls
            (71.332601,107.87263907) and (75.976982,111.80666507) ..
            (83.867952,112.10204307) .. controls
            (91.758922,112.39742207) and (100.54114,97.81547307)  ..
            (107.15399,84.24536207)  .. controls
            (109.7867,78.84282207)   and (116.0911,65.30681007)   ..
            (123.19123,50.92337507)  .. controls
            (136.46247,24.03854207)  and (147.18658,6.42393707)   ..
            (159.27196,2.81113707)   .. controls
            (184.95662,-4.86702293)  and (208.53361,32.31924407)  ..
            (217.37186,52.05019707);
        }%
        \newcommand{\MyWaveB}{%
            (200.02545,22.85649707)  .. controls
            (200.02545,22.85649707)  and (209.79787,34.53519707)  ..
            (217.54727,52.29679707)  .. controls
            (221.57621,61.53106707)  and (231.76264,67.80275707)  ..
            (242.62012,69.40393707)  .. controls
            (253.4776,71.00511707)   and (260.28174,112.23421707) ..
            (282.13712,111.96387707) .. controls
            (303.99249,111.69353707) and (329.46714,1.83349707)   ..
            (362.13712,1.60669707)   .. controls
            (394.8071,1.37989707)    and (415.22326,51.87129707)  ..
            (415.22326,51.87129707);
        }%
        \newcommand{\MyWaveA}{%
            (2.2036044,23.21319707)  .. controls
            (2.2036044,23.21319707)  and (7.1896734,29.00179707)  ..
            (11.896862,37.80959707)  .. controls
            (14.463984,42.61309707)  and (16.430404,45.91309707)  ..
            (19.958399,53.79003707)  .. controls
            (23.178819,60.98023707)  and (34.001913,68.16115707)  ..
            (44.701755,69.73908707)  .. controls
            (55.401595,71.31701707)  and (62.430335,112.41311707) ..
            (84.285711,112.14277707) .. controls
            (106.14109,111.87243707) and (131.61574,2.01239707)   ..
            (164.28572,1.78559707)   .. controls
            (196.9557,1.55879707)    and (217.37186,52.05019707)  ..
            (217.37186,52.05019707);
        }%
  
    \centering
    \begin{tikzpicture}[ 
        scale=0.7,
        tangent/.style={
          decoration={
            markings,% switch on markings
            mark= at position #1 with
              {
                \coordinate (tangent point-\pgfkeysvalueof{/pgf/decoration/mark 
                             info/sequence number}) at (0pt,0pt);
                \coordinate (tangent unit vector-\pgfkeysvalueof{/pgf/decoration/mark 
                             info/sequence number}) at (1,0pt);
                \coordinate (tangent orthogonal unit vector-\pgfkeysvalueof{/pgf/decoration/mark  
                             info/sequence number}) at (0pt,1);
              }
          },
          postaction=decorate
        },
        use tangent/.style={
            shift=(tangent point-#1),
            x=(tangent unit vector-#1),
            y=(tangent orthogonal unit vector-#1)
        },
        use tangent/.default=1
    ]

        \def\xmin{-30}
        \def\xmax{450}
        \def\ymin{-80}
        \def\ymax{150}
        \def\nula{52.05019707}
        \def\phase{2.2}
        \def\myscale{0.7}

        \begin{scope}[draw=black,line join=round, miter limit=4.00,
                      line width=0.5pt, y=1pt,x=1pt, scale=\myscale]   
          \fill [fill=yellow!80] (18.309015,\nula) -- \MyPath -- (216.7025,\nula);
          \path[name path=fce1,draw=black,line join=round,even odd rule,line cap=butt,miter   
            limit=4.00, tangent=0.36695, tangent=0.7770] \MyWaveA;  
          \path[name path=fce2,draw=black,line join=round,even odd rule,line cap=butt,miter   
            limit=4.00] \MyWaveB;           
          \draw[name path=axeX,->] (\xmin,\nula) -- (\xmax,\nula)   
               node[right] {$t$} coordinate(x axis);
          \draw[name path=axeY,->] (\phase,\ymin) -- (\phase,\ymax) 
               node[left]  {$v(t)$} coordinate(y axis);     
          \draw[densely dashed, thin, use tangent=1] 
               (0,0) coordinate(vrch) -- (0,-60*\myscale) % firt part
               (0,0) -- (-70*\myscale,0)    node[left] {$V_m$};     % delka krivky: 0.3667 - 0.3668
          \draw[densely dashed, thin, use tangent=2] 
               (0,0) coordinate(dul) --  (0,50*\myscale)  % first part 
               (0,0) -- (-120*\myscale,0) node[left] {$V_{min}$}; %               0.7770 - 0.7775
          \path[name intersections={of=fce1 and axeX, name=point}]; % prusecik osy y s funkcí - posun
          \draw[dotted, name path=line1] (point-1) --+(0,-100) coordinate(Ta);  
          \draw[dotted, name path=line2] (point-2) --+(0,-100) coordinate(Tb);
          \draw[dotted, name path=line3] (point-3) --+(0,-100) coordinate(Tc);
          \draw[fill, color=black] ($ (Ta)!0.5!(Tb) $) node[above, black] {$\frac{T}{2}$};  
          \draw[fill, color=black] ($ (Tb)!0.5!(Tc) $) node[above, black] {$\frac{T}{2}$};            
          \draw[<->] (Ta) -- (Tb);  
          \draw[<->] (Tb) -- (Tc);  
          \path[name intersections={of=axeX and axeY, name=pocatek}]; 
          \node[below left] at (pocatek-1) {$0$}; 
          \draw[fill=white] (pocatek-1) circle(2pt);          
          \draw[fill=white] (vrch) circle(1pt); 
          \draw[fill=white] (dul)  circle(1pt);                  
        \end{scope} 
    \end{tikzpicture}
  \caption{Časový průběh \textbf{střídavé veličiny} $v=v(t)$, pro kterou platí, že obsahy ploch
           v jednom cyklu nad osou $t$ a pod osou $t$ jsou totožné}\label{TEO:fig_02}
\end{figure}
  
% \end{document}