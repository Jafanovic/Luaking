\documentclass[11pt]{standalone}
  \usepackage{helvet}                        % font
  \usepackage{xltxtra}                       % fontspec package
  \usepackage{tikz}  
  \usetikzlibrary{arrows}
  \usetikzlibrary{intersections}
  \usetikzlibrary{calc}

%\newpath
%\moveto(0.37880683,-0.12631738)
%\curveto(0.37880683,-0.12631738)(16.746964,-0.16331738)(27.401489,14.35218262)
%\curveto(38.204079,29.06968262)(51.661079,58.47632262)(63.387069,58.58885262)
%\curveto(75.113069,58.70138262)(81.146179,32.79738262)(97.499259,19.26928262)
%\curveto(116.61771,3.45338262)(143.05509,-0.35661738)(143.05509,-0.35661738)

\newcommand{\XYcross}{% 
   \draw[name path=axeX,->] (\xmin,\zero) -- (\xmax,\zero)   node[right] {\xlabel} coordinate(x axis);
   \draw[name path=axeY,->] (\phase,\ymin) -- (\phase,\ymax) node[left]  {\ylabel} coordinate(y axis);
   \path[name intersections={of=axeX and axeY, name=pocatek}];
   \node[below left] at (pocatek-1) {$0$};
   \draw[fill=white] (pocatek-1) circle(2pt);
}

\begin{document}
  \begin{tikzpicture}
  \def\xmin{0}
  \def\xmax{160}
  \def\ymin{-3}
  \def\ymax{80}
  \def\nula{0}
  \def\phase{0}
    
    \begin{scope}[draw=black,line join=round,miter limit=4.00,line width=0.5pt, y=1pt,x=1pt]
      \draw[->] (\xmin,\nula) -- (\xmax,\nula) node[right] {$t$} coordinate(x axis);
      \draw[->] (\phase,\ymin) -- (\phase,\ymax) node[left]  {$i(t)$} coordinate(y axis);
      \path[name path=fce,draw=black,line join=round,even odd rule,line cap=butt,miter
      limit=4.00] (0,0)             .. controls
           (0,0)   and (17,0)  .. (27,14) .. controls
           (38,29) and (52,58) .. (63,58) .. controls
           (75,58) and (81,33) .. (97,19) .. controls
           (117,3) and (143,0) .. (143,0);      
      
      \path[name path=osax] (\xmin,\nula) -- (\xmax,\nula);
      
        % Intersections
          \path[name intersections={of=osax and fce, name=point}];   % prusecik osy x s funkc�
          \draw[name path=line1] (point-1) ++ (0,-2) node[below] {$0$};
      \draw[name path=line1] (point-2) ++ (0,-2) node[below] {$t_i$}  --+(0,4);
      \node[left]  at (70,25) {$Q$};             
    \end{scope}   
  \end{tikzpicture} 
\end{document}