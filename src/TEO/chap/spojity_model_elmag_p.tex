% !TeX program = lualatex
% !TeX root = luaking.tex
% !TeX encoding = UTF-8
% !TeX spellcheck = cs_CZ
%---------------------------------------------------------------------------------------------------
% file spojity_model_elmag_p.tex
\graphicspath{{../src/TEO/img/}}
%==============================Kapitola: Spojité matematické modely jednotlivých polí ==============
\setchaptertoc
\chapter{Spojité matematické modely polí}
  \section{Elektromagnetické pole}
    \subsection{Veličiny elektromagnetického pole a jejich jednotky}
      \fbox{Elektrický proud}\label{TEMP:kap_el_proud_velicina} je znám z každodenního života,
        přesto je velmi důležité umět tento pojem vnímat jak pro označení „jevu“ (kap.
        \ref{TEMP:kap_elproud_jev}), tak jako fyzikální veličinu, která tento jev kvantitativně
        popisuje (kap. \ref{TEMP:kap_el_proud_velicina} ). Elektrický proud je \emph{skalární
        fyzikální veličina} tzn. $I$ resp. $i$, jejíž jednotkou je základní jednotka soustavy SI:
        \emph{ampér} - [A]. V této soustavě jednotek je ampér definován na základě silových
        účinků mezi dvěma vodiči, kterými prochází elektrický proud. Tato síla je magnetického
        původu, avšak magnetické pole vzniká jako důsledek pohybu elektrického náboje. Je tvořen
        uspořádaným pohybem elektrických nábojů.
        
        Připojíme-li vodič ke zdroji elektrického napětí, elektrické pole uvnitř působí elektrickou
        silou na vodivostní elektrony, vyvolává jejich pohyb a tím vytváří elektrický proud, který
        je po krátké době \emph{stacionární} (ustálený, nezávislý na čase). Jestliže vodičem projde
        náboj $\Delta Q$ resp. $dQ$ za časový interval $\Delta t$ resp. $dt$, lze definovat
        \emph{průměrný} resp. \emph{okamžitý} proud ve vodiči:
        \begin{itemize}[noitemsep]
          \item \textbf{průměrný} elektrický proud: $$I_{AV} = \frac{\Delta Q}{\Delta t}
                \quad[A],$$
          \item \textbf{okamžitý} elektrický proud (který je limitním případem proudu průměrného,
                studujeme-li množství náboje, které projde průřezem vodiče za infinitezimální
                (nekonečně krátký) časový interval): $$i = \lim_{\Delta t \rightarrow 0}\frac{\Delta
                Q}{\Delta t} = \frac{dQ}{dt} \quad[A].$$ V ustáleném stavu protéká všemi průřezy
                vodiče stejně velký proud,
          \item speciálně pohybuje-li se náboj vodičem rovnoměrně, nazýváme proud
                \textbf{stejno\-směr\-ným}, $I(t) = \text{konst}$, a platí $$ I_{DC} =
                \frac{Q}{t}\quad[A] $$
        \end{itemize}        

        Elektrický proud jako \emph{jev} charakterizuje jednu z forem fyzikálního pohybu, kterou je
        \textbf{uspořádaný pohyb elektricky nabitých částic} v látce. Přestože jakýkoliv elektrický
        proud je vždy tvořen pohybujícími se náboji, nemusí všechny pohybující se náboje vytvářet
        elektrický proud. Ve vodiči dochází ke vzniku trvalého elektrického proudu za těchto
        podmínek:
          \begin{itemize}[noitemsep]
            \item vodič se musí nacházet v trvalém elektrickém poli, což je realizováno pomocí tzv.
                  \emph{zdroje} (generátoru) elektrického napětí,
            \item ve vodiči musí být přítomny volné nosiče elektrického náboje.
          \end{itemize}
        
        Podle charakteru vnějšího elektrického pole lze rozlišit tři základní druhy proudů:
          \begin{description}[noitemsep]
            \item\textbf{stejnosměrný} proud vzniká tehdy, jestliže má intenzita elektrického pole
                   konstantní orientaci,
            \item\textbf{střídavý} proud ve vodiči vytváří vnější elektrické pole, jehož intenzita
                  periodicky mění svou orientaci na opačnou,
            \item\textbf{stacionární} stejnosměrný proud vzniká ve vodiči, je-li intenzita
                  elektrického pole konstantní co do velikosti, směru i orientace.
          \end{description}  

       Nabité částice představující volný náboj ve vodičích jsou v neustálém chaotickém tepelném
       pohybu (viz molekulová fyzika a termodynamika). Jedná se o \emph{mikroskopický pohyb}, který
       nemá za následek makroskopicky pozorovatelné přemístění náboje. Pokud ve vodiči vytvoříme
       elektrické pole, tepelný pohyb nabitých částic neustane, ale k náhodné složce rychlosti
       přibude ještě složka rychlosti ve směru vloženého pole.
       
       Při studiu elektrického proudu v kovových vodičích se zabýváme ustálenými proudy
       vodivostních elektronů, které v kovu vytváří tzv. \emph{elektronový plyn}. Tyto vodivostní
       elektrony jsou téměř volné a pohybují se v poli kladných iontů uspořádaných v krystalové
       mřížce.
        
       Experimentálně lze elektromagnetické pole prokázat silovým působením na elektricky nabité
       částice (kapitola \ref{fyz:IIchapI}). Celkovou sílu $\vec{F}$ lze rozložit na elektrickou 
       sílu $\vec{F}_e$, nezávislou na tom, zda je nabitá částice v klidu nebo v pohybu vůči 
       vztažné soustavě a na magnetickou sílu $\vec{F}_m$, působící jen na pohybující se částice. 
       Elektromagnetické pole má tedy dvě složky: \textbf{elektrické pole}, působící na náboj silou 
       $\vec{F}_e$ a \textbf{magnetické pole}, působící na pohybující se náboj silou $\vec{F}_m$  
       \cite[s.~13]{Mayer2001}. 
      
      \vspace{1em}
      \fbox{Intenzita elektrického pole $\vec{E}$} je vektorovou veličinou charakterizující
        \emph{elektrické pole}.
        Je definována jako 
        \emph{síla působící na nepohybující se jednotkový bodový náboj}:
        \begin{equation}\label{TEMP:eq_E}
          \vec{E} = \frac{\vec{F}_e}{Q} \quad\left[\frac{V}{m}\right]  
        \end{equation}        
        kde $\vec{F}_e$ je elektrická síla působící na náboj $Q$.
      
      \vspace{1em}
      \fbox{Magnetická indukce $\vec{B}$} je vektorovou veličinou charakterizující \emph{magnetické
        pole}. Je definovována vztahem
        \begin{equation}\label{TEMP:eq_B}
          \vec{F}_m = Q(\vec{v}\times\vec{B}) \quad[T]  
        \end{equation}        
        kde $\vec{F}_m$ je magnetická síla působící na náboj $Q$ pohybující se rychlostí $\vec{v}$.
        Jednotkou je \emph{tesla} $[T]$.
    
        Síla, jež působí elektromagnetické pole na pohybující se náboj se nazývá \textbf{Lorentzova
        síla}
        \begin{equation}\label{TEMP:eq_Lorentz}
          \vec{F} = \vec{F}_e + \vec{F}_m =Q(\vec{E} + \vec{v}\times\vec{B}) \quad[N]  
        \end{equation}        

%~~~~~~~~~~~~~~~~~~~~~~~~~~~~~~~~~~~~~~~~~~~~~~~~~~~~~~~~~~~~~~~~~~~~~~~~~~~~~~~~~~~~~~~~~~~~~~~~~~