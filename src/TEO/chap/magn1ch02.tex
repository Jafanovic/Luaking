% !TeX program = lualatex
% !TeX root = luaking.tex
% !TeX encoding = UTF-8
% !TeX spellcheck = cs_CZ
%---------------------------------------------------------------------------------------------------
% file magn1ch02.tex
\graphicspath{{../src/TEO/img/}}
%===========Kapitola: Topologické vlastnosti elektromagnetického pole===============================
\setchaptertoc
\chapter{Topologické vlastnosti elektromagnetického pole}\label{teo:IchapIII}
  V předchozích kapitolách byly na mnohých místech zdůrazňovány některé topologické souvislosti.
  Kapitola o topologii je úmyslně zařazena až následně, jednak aby shrnula získané poznatky a
  vtiskla jim určitý řád, jednak aby čtenář již měl předchozí konkrétní představy o některých
  abstraktních pojmech \cite[s.~40]{Patocka4}. 
  
  Topologie je matematická disciplína, patřící do vyšších pater v hierarchii matematiky. Topologie
  se zabývá prostorovými útvary, podobně jako geometrie. Na rozdíl od geometrie ji však nezajímají 
  \emph{kvantitativní} ukazatele zkoumaných geometrických útvarů, nýbrž určité vyšší obecnější
  \emph{kvalitativní} ukazatele. S nadsázkou lze říci, že je to ''geometrie, která nic neměří''.
  Topologie se dělí na dvě zdánlivě odlišné disciplíny: 
  \begin{itemize}[noitemsep]
    \item \emph{Topologie diskrétních útvarů} neboli \emph{teorie grafů} - zabývá se diskrétními
          prostorovými útvary, tj. \emph{grafy}. Graf sestává z \emph{uzlů} propojených
          \emph{hranami}. Hrany mohou být \emph{orientované} nebo \emph{neorientované}. Všechny
          diskrétní elektrické obvody jsou \emph{neorientovanými} grafy. Proto i všechny známe
          metody řešení diskrétních elektrických obvodů (metoda Kirchhoffových zákonů, metoda
          smyčkových proudů, metoda uzlových napětí) podléhají zákonitostem diskrétní topologie. 
    \item \emph{Topologie spojitých útvarů} - zabývá se spojitými prostorovými útvary, tj.
          \emph{plochami a křivkami} v prostorech libovolné dimenze. Patří sem všechny elektrické
          obvody s \emph{parametry spojitě rozprostřenými} v 3D prostoru. Takovým útvarem je např.
          krabice naplněná elektricky vodivým grafitovým práškem, do níž umístíme v libovolných
          místech dvě nebo více elektrod. Intuitivně tušíme, že v prostoru krabice budeme pracovat s
          ekvipotenciálami v podobě \emph{ploch} nebo se siločárami v podobě \emph{křivek} atd. 
  \end{itemize}
  
  Elektrické obvody se chovají poněkud jinak než obvody magnetické: 
  \begin{itemize}[noitemsep]
    \item V \emph{elektrických} obvodech je poměr mezi měrnou elektrickou vodivostí \emph{vodičů} a
          \emph{izolantů} minimálně $10^{12}$, obvykle i větší. Proto lze snadno pomocí vodiče
          obaleného izolantem docílit toho, že elektrický proud teče pouze prostorově vymezenými
          \emph{diskrétními} cestami. Pak je pochopitelné, že vhodným a \emph{absolutně přesným}
          nástrojem k analýze elektrického obvou je \emph{topologie diskrétních útvarů}.
    \item V \emph{magnetických} obvodech je poměr mezi měrnou magnetickou vodivostí (permeabilitou)
          \emph{vodičů} a \emph{izolantů} typicky $10^{3}$, což je dáno relativní permeabilitou
          feromagnetik vůči vakuu. Vakuum je tedy \emph{velmi špatný} magnetický izolant a lepší v
          přírodě bohužel neexistuje. V této situaci je obtížné realizovat ryze diskrétní magnetický
          obvod, protože železo neumíme ''obalit'' kvalitním magnetickým izolantem. U cívky se
          železným jádrem podle obr. \ref{teo:fig049} v kapitole \ref{ES:sec03} 
          jsme ukázali, že rozptylový tok vzdušných cest činí řádově 1 \% až 5 \% z toku celkového. 
          Takový obvod je sice již řešitelný metodami \emph{diskrétní} topologie, ale pouze 
          přibližně. Je to běžný inženýrský postup, který se v praxi velmi úspěšně používá. Pokud 
          však rozptylový tok nechceme nebo nemůžeme zanedbat, je nezbytné pracovat metodami 
          \emph{topologie spojitých útvarů\footnote{Všimněme si ale, že ke zjednodušeným výrazům 
          pro výpočet spřaženého toku v \emph{diskrétním} obvodu jsme dospěli pomocí integrálních 
          metod, které používá topologie \emph{spojitých} útvarů}. Berme to jako ukázku, že mezi 
          oběma topologiemi je hluboký vztah, i když není na první pohled patrný.}
  \end{itemize} 
  %----------------------------------
  % image: resistor_grid.tex label: \label{ES:fig_res_grid}
%      %\documentclass[landscape]{article}
%\usepackage[american, europeanresistors, cuteinductors, smartlabels]{circuitikz}
%\usetikzlibrary{calc}
%\ctikzset{bipoles/thickness=1}
%\ctikzset{bipoles/length=0.8cm}
%\ctikzset{bipoles/vsourceam/height/.initial=.7}
%\ctikzset{bipoles/vsourceam/width/.initial=.7}
%\tikzstyle{every node}=[font=\small]
%\tikzstyle{every path}=[line width=0.8pt,line cap=round,line join=round]

%\begin{document}
\begin{figure}[ht!]
  \centering
  \begin{tikzpicture}
      \def\ROW{1,2,3,4,5}
    \def\COL{1,2,3,4,5,6,7}
    \foreach \j in \ROW {
      \foreach \i in \COL {
          \draw (\i,\j) to [R] (\i+1,\j);
      }
      }  
      \def\ROW{2,3,4,5}
    \def\COL{1,2,3,4,5,6,7,8}  
    \foreach \j in \ROW {
      \foreach \i in \COL {
      \draw (\i,\j) to [R] (\i,\j-1);
      }
      }
    % dots
      \draw (0.5,4) to[short,o-*] (1,4);  
    \draw (0.5,2) to[short,o-*] (1,2);
    \draw (8,4) to[short,*-o] (8.5,4);  
    \draw (8,2) to[short,*-o] (8.5,2);
    
    \draw (8,3) to[short,*-] (8,3);
    \draw (1,3) to[short,*-] (1,3);
  
      \def\ROW{1,2,3,4,5}
    \def\COL{2,3,4,5,6,7}  
      \foreach \j in \ROW {
      \foreach \i in \COL {
      \draw (\i,\j) to [short,*-] (\i,\j);
      }
      }  
  \end{tikzpicture}
%  \caption{Pasivní dvojbran v podobě husté vodivostní sítě}\label{ES:fig_res_grid}
\end{figure}
%\end{document}   
  %----------------------------------
   {\centering
    \captionsetup{type=figure}
%    %\documentclass[landscape]{article}
%\usepackage[american, europeanresistors, cuteinductors, smartlabels]{circuitikz}
%\usetikzlibrary{calc}
%\ctikzset{bipoles/thickness=1}
%\ctikzset{bipoles/length=0.8cm}
%\ctikzset{bipoles/vsourceam/height/.initial=.7}
%\ctikzset{bipoles/vsourceam/width/.initial=.7}
%\tikzstyle{every node}=[font=\small]
%\tikzstyle{every path}=[line width=0.8pt,line cap=round,line join=round]

%\begin{document}
\begin{figure}[ht!]
  \centering
  \begin{tikzpicture}
      \def\ROW{1,2,3,4,5}
    \def\COL{1,2,3,4,5,6,7}
    \foreach \j in \ROW {
      \foreach \i in \COL {
          \draw (\i,\j) to [R] (\i+1,\j);
      }
      }  
      \def\ROW{2,3,4,5}
    \def\COL{1,2,3,4,5,6,7,8}  
    \foreach \j in \ROW {
      \foreach \i in \COL {
      \draw (\i,\j) to [R] (\i,\j-1);
      }
      }
    % dots
      \draw (0.5,4) to[short,o-*] (1,4);  
    \draw (0.5,2) to[short,o-*] (1,2);
    \draw (8,4) to[short,*-o] (8.5,4);  
    \draw (8,2) to[short,*-o] (8.5,2);
    
    \draw (8,3) to[short,*-] (8,3);
    \draw (1,3) to[short,*-] (1,3);
  
      \def\ROW{1,2,3,4,5}
    \def\COL{2,3,4,5,6,7}  
      \foreach \j in \ROW {
      \foreach \i in \COL {
      \draw (\i,\j) to [short,*-] (\i,\j);
      }
      }  
  \end{tikzpicture}
%  \caption{Pasivní dvojbran v podobě husté vodivostní sítě}\label{ES:fig_res_grid}
\end{figure}
%\end{document} 
    \includegraphics[width=0.7\linewidth]{resistor_grid.pdf}
    \captionof{figure}{Pasivní dvojbran v podobě husté vodivostní sítě}
    \label{ES:fig_res_grid}
    \par}       
    \vspace{1em}

  Na první pohled se zdá, že obě topologie využívají natolik odlišných matematických postupů, že
  spolu tyto disciplíny nijak nesouvisí. Opak je pravdu. Mezi oběma panuje hluboký vztah, obě
  vycházejí ze stejných základů. Vysvětlení lze hledat na obr. \ref{ES:fig_res_grid}. Je zde
  nakreslen \emph{přenosový dvojbran} se zcela obecnou vnitřní strukturou, která může mít např.
  podobu husté vodivostní\footnote{V magnetických obvodech je psychologicky výhodnější pracovat s
  magnetickými vodivostmi než s magnetickými odpory (reluktancemi). Permeabilita má totiž význam
  \emph{měrné magnetické vodivosti}} sítě, ve které mají jednotlivé vodivosti nahodile různé
  hodnoty. S ohledem na dobře známé analogie je lhostejné, zda se jedná o \emph{elektrický} nebo
  \emph{magnetický} obvod. Nakreslený obvod je zcela určitě \emph{diskrétní}, bude tedy řešen
  některou klasickou diskrétní metodu, např. metodou smyčkových proudů nebo metodou uzlových
  napětí. Výpočtem zjistíme, že z pohledu vstupní a výstupní brány má dvojbran konkrétní přenosové
  parametry (napěťový přenos naprázdno, proudový přenos nakrátko, vstupní impedanci naprázdno,
  nakrátko, atd.) Učiňme následující myšlenkový pokus: vodivostní síť budem neustále zjemňovat.
  Tj., ve směru vodorovném i svislém budeme zvyšovat počty prvků, ale tak, aby celková vodivost na
  jednotku délky zůstávala v dané oblasti \emph{konstantní}. Výsledkem zjemňování bude v limitním
  případě vznik \emph{spojité} vodivé desky (např. izolační podložka nastříkaná elektricky vodivým
  odporovým lakem). Mezi původním diskrétním obvodem a deskou zřejmě platí následující souvislosti:
  \begin{itemize}[noitemsep]
    \item Původní diskrétní vodivosti byly nahodile různé \(\longrightarrow\) deska bude
          \emph{nehomogenní, anizotropní}.
    \item Původní diskrétní vodivosti byly stejně velké ve směru \emph{x} a stejně velké (ale s
          jinou hodnotou ve směru \emph{y}) \(\longrightarrow\) deska bude \emph{homogenní
          anizotropní}.
    \item Všechny diskrétní vodivosti měly stejnou hodnotu \(\longrightarrow\) deska bude
          \emph{homogenní, izotropní}.
  \end{itemize}
  
  Intuitivně tušíme, že vytvořená \emph{spojitá} deska\footnote{Uvedený příklad se týká
  dvojrozměrné desky. Příklad lze jistě zobecnit na trojrozměrné objekty (lze si představit
  krabici naplněnou vodivým grafitovým práškem, do které zavedeme čtyři bodové elektrody).} bude mít
  všechny přenosové parametry číselně shodné s původním \emph{diskrétním} obvodem. Přitom ale u
  desky nelze tyto parametry určit klasickými diskrétními metodami (nelze určit matici obvodu). Je
  nutný přechod od diskrétních operací k operacím integrálním, tedy od topologie \emph{diskrétních
  útvarů} k topologii \emph{spojitých útvarů}. Z uvedeného myšlenkového pokusu plyne, že v
  \emph{limitním případě} velmi jemné sítě musí dát diskrétní i spojité operace stejný kvantitativní
  výsledek. Na tomto poznatku je založeno přibližné řešení spojitých prostorových polí
  \emph{metodami konečných prvků}.
  
  Především jsme ovšem chtěli ukázat, že mezi diskrétními a spojitými topologickými metodami není
  zásadního rozdílu, obě vycházejí ze stejných základů a v limitním případě spolu splývají. 
  
  \section{Topologie diskrétních útvarů}\label{teo:IchapIIIsecI}
    Cílem této kapitoly je především vysvětlit \emph{pricnip reciprocity} v pasivních elektrických
    obvodech a pomocí něho odvodit \emph{počet stupňů volnosti} elektrických obvodů. Zvláštním
    případem obvodu je \emph{pasivní přenosový dvojbran}, u kterého bude dokázáno, že má vždy
    \emph{tři stupně volnosti}. Tento poznatek má totiž mimořádný význam v teorii
    \emph{transformátorů}, který je právě typickým představitelem přenosového dvojbranu. V
    kapitolách zabývajících se transformátorem - především jeho náhradním zapojením - se budeme
    odvolávat na výsledky získané v této kapitole. 
    
    \subsection{Základní pojmy teorie grafů}\label{teo:IchapIIIsecIsubI}
      Názvosloví a základní pojmy teorie grafů lze shrnout do následujících bodů:
      \begin{itemize}[noitemsep]
        \item Základním pojmem je \emph{graf} (graf orientovaný, neorientovaný). Graf je vlastně
              „schéma“ příslušného obvodu s vynechanými obvodovými prvky.
        \item Graf sestává z \emph{uzlů} a \emph{hran}.
        \item Uzel je spojení alespoň tří hran\footnote{Spojení dvou hran je elektrický bod, 
              nikoli uzel.}.
        \item Hrana může být \emph{orientovaná} (je jí přiřazen směr), \emph{neorientovaná} (nemá
              přiřazen směr). V elektrotechnice se používají výhradně neorientované hrany - tedy i
              grafy (vlastnosti obvodových prvků R, L, C jsou nezávislé na směru proudu).
        \item \emph{Úplný strom}: nepřerušená celistvá soustava nejmenšího počtu hran, která spojuje
              všechny uzly grafu.
        \item \emph{Nezávislá hrana}: hrana nepatřící do úplného stromu.
        \item \emph{Nezávislá smyčka}: uzavřená smyčka, která musí obsahovat nezávislou hranu, tj.
              hranu nepatřící do úplného stromu.
        \item Nezávislých smyček je tolik, kolik je nezávislých hran.
      \end{itemize}
     
      Označme v grafu:
       \begin{itemize}[noitemsep]
         \item Počet uzlů:  \(q +1\)
      \item Počet hran úplného stromu:  \(q\)
      \item Počet hran (počet neznámých proudů):  \(p\)
      \item Počet nezávislých hran (nezávislých smyček):  \(n=p-q\)
      \end{itemize}
      
      U složitých obvodů je hledání \(n\) nezávislých smyček obtížné. Proto se k tomuto účelu 
      používá úplný strom, jehož nalezení je snadné. Nezávislé hrany jsou ty, které \emph{nepatří} 
      do úplného stromu. Každou nezávislou hranou pak musí procházet alespoň jedna nezávislá 
      smyčka. Všechny pojmy budou ukázány na konkrétním příkladu.
      
      Řešením obvodu se rozumí: Nalezení všech \(p\) neznámých proudů ve všech \(p\) hranách.
      Principiálně se vždy jedná o řešení soustavy \(p\) rovnic o \(p\) neznámých proudech.
      
      K řešení lze použít tři metody:
      \begin{itemize}[noitemsep]
       \item \textbf{Metoda založená na přímém použití I. a II. Kirchhoffova 
             zákona}\footnote{Gustav Robert Kirchhoff (1824-1887), německý fyzik, působil na 
             univerzitách v Heidelbergu a v Berlíně. I. a II. KZ objevil r. 1845 ještě jako 
             student. Dále se zabýval spektroskopií, tepelnou radiací černého tělesa, 
             spoluobjevitel Cesia a Rubidia. Žák F. E. Neumanna.}.
             Je nejméně efektivní, vede na nejrozsáhlejší soustavu \(p\) rovnic o \(p\) neznámých.
       \item \textbf{Metoda smyčkových proudů} (\emph{Mesh Currents Matrix Method}), Maxwellova 
             metoda. Vede na soustavu pouze \(n\) rovnic o \(n\) neznámých smyčkových 
             proudech\footnote{Ze smyčkových proudů lze skutečné proudy snadno vyřešit pomocí 
             doplňkových rovnic sestavených pomocí I. KZ.}. Vezmeme-li v úvahu nejsložitější obvod, 
             ve kterém je každá dvojice uzlů spojena hranou, pak bude:
             \begin{equation}\label{ES:eq_topol00}
               n=p-q=\frac{q(q-1)}{2},
             \end{equation}
             což je  podstatně méně rovnic než \(p\).
       \item \textbf{Metoda uzlových napětí} (\emph{Node Voltages Matrix Method}). Je to metoda 
             nejefektivnější, protože vede na nejmenší soustavu pouze \(q\) rovnic. Metoda je 
             založena na přepočtu všech napětí zdrojů na ekvivalentní zdroje proudové. Z takto 
             určených proudů a ze známých obvodových impedancí je pak možno určit napěťové úbytky 
             na každé impedanci, tedy i napětí všech \(q\) proti zvolenému uzlu referenčnímu.
      \end{itemize}
      
      % --------example: Metoda smyčkových proudů ------------
      % \label{TEO:exam015}
      % !TeX spellcheck = cs_CZ
\begin{example}\label{TEO:exam015}
  Metodou smyčkových proudů vyřešme obvod na obr. \ref{es:fig_patocka_topol02}. Řešením se rozumí 
  nalezení velikosti všech šesti proudů \(I_1\), až \(I_6\). Napětí \(U_1\), \(U_2\), \(U_3\) 
  považujme za známá. Příklad bude řešen jen obecně, nejsou zadána konkrétní čísla. Smyslem 
  příkladu je pouze sestavení \(\mathbb{Z}\)-matice obvodu a ukázka kvalitativních vlastností této 
  matice.
  
   {\centering  
    \begin{tabular}{c}
        \includegraphics[width=0.7\linewidth]{patocka_topol02a.png}  \\
        \includegraphics[width=0.7\linewidth]{patocka_topol02b.png}
    \end{tabular}
    \captionsetup{type=figure}
    \captionof{figure}{a) Řešený obvod, b) Úplný strom (silné čáry) a k němu příslušející soustava 
             třech nezávislých smyček \(A\), \(B\), \(C\) se zvolenými směry smyčkových 
             proudů. Každá ze tří nezávislých smyček \(A\), \(B\), \(C\) prochází alespoň 
             jednou nezávislou hranou (slabé čáry), která nepatří do úplného stromu. 
             \cite[s.~43]{Patocka4}} 
    \label{es:fig_patocka_topol02}
  \par}
  
  Obvod má následující parametry:
  \begin{itemize}\addtolength{\itemsep}{-0.2\baselineskip}
   \item Počet uzlů: \(q + 1 = 4\)
   \item Počet hran úplného stromu: \(q = 3\) (tři silně vytažené čáry)
   \item Počet hran (počet neznámých proudů): \(p = 6\)
   \item Počet nezávislých smyček (nezávislých hran): \(n=p-q=3\)
  \end{itemize}
  
  Pro tři nezávislé smyčky \(A\), \(B\), \(C\) lze sestavit soustavu tří rovnic o třech nově 
  zavedených neznámých fiktivních smyčkových proudech \(I_A\), \(I_B\), \(I_C\). Celková napájecí 
  napětí v každé smyčce jsou označena \(U_A\), \(U_B\), \(U_C\):
  \begin{align}\label{ES:eq_topol01}
    U_A &= U_1-U_3 = (Z_1 + Z_3)I_A +Z_5(I_A — I_C)           \nonumber\\
    U_B &= U_1     = (Z_2 + Z_4)I_B+Z6(I_B + I_C)             \nonumber\\
    U_C &= U_3     = (Z_7 + Z_8)I_C+Z_5(I_C - I_A)+Z_6(I_B+I_c)  
  \end{align}
  Přičemž zřejmě platí:
  \begin{subequations}
    \begin{align}\label{ES:eq_topol02}
     I_1 &= I_A,       \qquad\quad\; I_3 = I_A - I_C,                   \\
     I_4 &= I_B + I_C, \quad I_5 = I_C, \quad I_6 = I_C
    \end{align}
    \end{subequations}
  Pravé strany v soustavě rovnic (\ref{ES:eq_topol01}) uspořádáme podle smyčkových proudů do   
  matice ve tvaru tvaru
  \begin{equation}\label{ES:eq_topol03}
   \left(
     \begin{array}{cccc}
       Z_1+Z_3+Z_5  &              &-Z_5  \\
                    & Z_2+Z_4+Z_6  & Z_6  \\
               -Z_5 & Z_6          & Z_5+Z_6+Z_7+Z_8 
     \end{array}
   \right) 
  \end{equation}
  Další postup by byl následující: \emph{Cramerovým pravidlem} rutinně vyřešit tuto soustavu tří 
  rovnic pro tři neznámé smyčkové proudy a dodatečně dopočítat dvě rovnice (\ref{ES:eq_topol02}). 
  Zbývající čtyři identity v soustavě (\ref{ES:eq_topol02}) není nutno počítat. Tím by byl obvod 
  vyřešen. Smyslem příkladu však nebylo vlastní řešení, nýbrž konkrétní ukázka velice zajímavého 
  jevu: matice (\ref{ES:eq_topol03}) je symetrická podle hlavní diagonály. Symetrie matice je 
  důsledkem principu reciprocity, který bude diskutován v následujících kapitolách.
\end{example}  
      %-------------------------------------------------------
      
    \subsection{Princip reciprocity v pasivních lineárních 
    obvodech}\label{teo:IchapIIsecIsubII}
      Zdůrazněme, že princip reciprocity platí pouze v \emph{pasivních} obvodech, tj. v obvodech, 
      které neobsahují vnitřní (skryté) zdroje napětí nebo proudu. Příklad uvedený v předchozí 
      kapitole slouží jako názorná ukázka principu reciprocity \emph{lineárního} elektrického 
      obvodu. Matice (\ref{ES:eq_topol03}) řešeného obvodu je totiž \emph{symetrická} podle 
      \emph{hlavní diagonály}. Nejedná se o náhodu, podobnou diagonální symetrii bude vykazovat 
      každá \(\mathbb{Z}\)-matice nebo \(\mathbb{Y}\)-matice libovolného pasivního obvodu. Bude-li 
      obvod řešen např. metodou I. KZ a II. KZ, pak bude matice nejrozsáhlejší, bude mít rozměr 
      \(p\times p\), kde \(p\) je počet hran (počet neznámých proudů):
      \begin{equation}\label{ES:eq_topol04}
        \left(
          \begin{array}{ccccc}
             z_{1,1}    &  z_{1,2}   & \ldots & z_{1,p-1}   & z_{1,p}   \\
             z_{2,1}    &  z_{2,2}   & \ldots & z_{2,p-1}   & z_{2,p}   \\
             \vdots     &  \vdots    & \ddots & \vdots      &\vdots     \\
             z_{p-1,1}  &  z_{p-1,2} & \ldots & z_{p-1,p-1} & z_{p-1,p} \\
             z_{p,1}    &  z_{p,2}   & \ldots & z_{p,p-1}   & z_{p,p}   \\
          \end{array}
        \right)
     \end{equation}
      
      Tato matice bude určitě opět symetrická podle hlavní diagonály. Diagonální symetrie 
      \(\mathbb{Z}\)-matic i \(\mathbb{Y}\)-matic je nejvýznamnější a nejobecnější vlastností všech 
      pasivních obvodů. Matematicky lze tuto symetrii zapsat ve tvaru
      \begin{equation}\label{ES:eq_topol05}
        z_{ij} = z_{ji} \qquad\text{případně}\qquad y_{ij} = y_{ji}
      \end{equation}
      Obě rovnice (\ref{ES:eq_topol05}) vyjadřují princip reciprocity, který můžeme slovně 
      formulovat takto: a) Přenosová impedance naprázdno měřená podle obr. 
      \ref{es:fig_patocka_topol03a} mezi \(r\)-tou a \(y\)-tou hranou obvodu je stejná pro oba 
      směry přenosu:
      \begin{figure}[ht!]
        \centering  
        \subcaptionbox{\label{es:fig_patocka_topol03a}}{\luafigure[0.9]{patocka_topol03a.png}}   \\
        \subcaptionbox{\label{es:fig_patocka_topol03b}}{\luafigure[0.9]{patocka_topol03b.png}}
        \caption{Měření přenosových parametrů: a) \textbf{impedancí naprázdno} \(Z_{ij,0}\) a 
                 \(Z_{ji,0}\) v obou směrech mezi \(i\)-tou a \(j\)-tou hranou obvodu. Obvod je 
                 napájen zdrojem proudu a měříme napětí. Obě měření musí dát stejný výsledek. b)  
                 \textbf{admitancí nakrátko} \(Y_{ij,K}\) a \(Y_{ji,K}\) v obou směrech mezi 
                 \(i\)-tou a \(j\)-tou hranou obvodu. Obvod je napájen zdrojem napětí a měříme 
                 proud. Obě měření musí dát stejný výsledek. \cite[s.~45]{Patocka4}} 
        \label{es:fig_patocka_topol03}
      \end{figure}
      
      Princip reciprocity neplyne z žádných vyšších fyzikálních zákonů (např. ze zákona zachování 
      energie, náboje atd.). Sám je totiž samostatným základním zákonem vyjadřujícím určitou 
      topologickou kvalitu („geometrickou kvalitu“) každého elektrického obvodu.
      
      Na závěr zdůrazněme, že u \emph{lineárních} obvodů jsou všechny maticové prvky \(z_{ij}\) 
      \emph{konstantami} nezávislými na proudech \(I_1\), až \(I_p\), a podobně všechny maticové 
      prvky \(y_{ij}\). jsou \emph{konstantami} nezávislými na napětích \(U_1\), až \(U_p\). Na 
      rozdíl od nelineárních obvodů v následující kapitole.
      
    \subsection{Princip reciprocity v pasivních nelineárních obvodech}\label{teo:IchapIIIsecIsubIII}
      V této kapitole ukážeme, že princip reciprocity platí i v případě \emph{nelineárních} obvodů. 
      Musí se však jednat opět o pasivní obvody, tj. obvody, které neobsahují vnitřní zdroje.
      
      Každá \(\mathbb{Z}\)-matice nebo \(\mathbb{Y}\)-matice libovolného \emph{nelineárního} obvodu 
      je opět \emph{symetrická podle hlavní diagonály}. Ale - na rozdíl od lineárních obvodů - 
      maticové prvky \(z_{ij}\), \(y_{ij}\) nebudou konstantami, nýbrž budou \emph{mnohorozměrnými 
      funkcemi} vnějších budicích veličin. Konkrétně maticové prvky \(z_{ij}(I_1\ldots I_p)\) budou 
      funkcemi proudů \(I_1\), až \(I_p\) a maticové prvky \(y_{ij}(U_1\ldots U_p)\) budou 
      mnohorozměrnými funkcemi napětí \(U_1\), až \(U_p\). Např. \(\mathbb{Z}\)-matice bude mít 
      tvar:
        \begin{equation*}
          \left(
            \begin{array}{ccccc}
              z_{1,1}(I_1\ldots I_p)     &  z_{1,2}(I_1\ldots I_p)   & \ldots & 
              z_{1,p-1}(I_1\ldots I_p)   &  z_{1,p}(I_1\ldots I_p)                   \\
              z_{2,1}(I_1\ldots I_p)     &  z_{2,2}(I_1\ldots I_p)   & \ldots & 
              z_{2,p-1}(I_1\ldots I_p)   &  z_{2,p}(I_1\ldots I_p)                   \\
              \vdots      &  \vdots      &  \ddots     & \vdots      & \vdots        \\
              z_{p-1,1}(I_1\ldots I_p)   &  z_{p-1,2}(I_1\ldots I_p) & \ldots & 
              z_{p-1,p-1}(I_1\ldots I_p) &  z_{p-1,p}(I_1\ldots I_p)                 \\
              z_{p,1}(I_1\ldots I_p)     &  z_{p,2}(I_1\ldots I_p)   & \ldots & 
              z_{p,p-1}(I_1\ldots I_p)   &  z_{p,p}(I_1\ldots I_p)                   \\
            \end{array}
          \right)
        \end{equation*}
      
      Symetrii \(\mathbb{Z}\)-matic a \(\mathbb{Y}\)-matic podle hlavní diagonály lze matematicky 
      zapsat ve tvaru
      \begin{subequations}
        \begin{align}\label{ES:eq_topol06}
          z_{ij}(I_1\ldots I_p) &= z_{ji}(I_1\ldots I_p) \\
          \shortintertext{případně}
          y_{ij}(U_1\ldots U_p) &= y_{ji}(U_1\ldots U_p)
        \end{align}
      \end{subequations}
      
      Zdůrazněme, že klasická teorie elektrických obvodů připouští platnost principu reciprocity 
      pouze u \emph{lineárních} obvodů, např. Někdy bývá dokonce explicitně vysloveno mylné 
      tvrzení, že v \emph{nelineárních} obvodech princip neplatí. Proto nyní dokážeme, že princip 
      reciprocity je možno rozšířit i na obvody nelineární. Důkaz lze konstruovat následovně:
     
      
      \begin{enumerate}[noitemsep]
        \item Předpokládejme nelineární obvod, ale buzený signály o \emph{velmi malé} amplitudě.
        \item Bude-li se amplituda všech signálů limitně blížit nule, lze obvod 
              \emph{linearizovat} v určitém stejnosměrném pracovním bodu, tj. všechny nelineární 
              převodní funkce \(z_{ij}(I_1\ldots I_p)\) lze nahradit převodními konstantami 
              \(z_{ij}\) jejichž velikost je dána parciálními derivacemi původní funkce v onom 
              pracovním bodu (jedná se o běžný linearizační postup).
        \item Když jsme takto obvod \emph{linearizovali}, musí se pro limitně malé signály chovat 
              jako obvod \emph{lineární} - a to se všemi důsledky z toho plynoucími, tedy včetně 
              existence principu reciprocity.
        \item Nutnou a postačující podmínkou platnosti tvrzení 3) je splnění rovnice 
              (\ref{ES:eq_topol06}).
      \end{enumerate}
      Konec důkazu.
      
      Poznatek, že princip reciprocity platí i v \emph{nelineárních} elektrických soustavách, má 
      velký význam při vytváření matematických modelů nelineárních soustav: jednak bude model 
      jednodušší (menší počet převodních funkcí), jednak máme částečnou kontrolu jeho správnosti. 
      Princip reciprocity splňují nejen \emph{lineární} a \emph{nelineární} soustavy elektrické, 
      nýbrž všechny fyzikální soustavy. Jedná se o jeden z nejobecnějších fyzikálních zákonů 
      topologického charakteru\footnote{Například Einstein předpokládal platnost principu 
      reciprocity i ve své teorii gravitace. Po určitých úvahách totiž postuloval diagonální 
      symetrii Riemannova tenzoru křivosti čtyřrozměrného časoprostoru. Tento tenzor druhého řádu 
      má tvar matice o rozměru \(4\times4\). Obsahuje tedy 16 členů, 16 čtyřrozměrných 
      metrických funkcí \(g_{ij}(x, y, z, ct)\). Z postulované symetrie plyne, že pouze 10 z nich 
      je různých (\(10 = 4+3+2+1\)). Řešení gravitačního pole tedy spočívá ve vyřešení soustavy 10 
      nelineárních parciálních diferenciálních rovnic o 10 neznámých funkcích \(g_{ij}(x, y, z, 
      ct)\). Proto se jedná o velmi obtížnou úlohu.}.
      
    \subsection{Počet stupňů volnosti pasivního mnohopólu}\label{teo:IchapIIIsecIsubIV}
      Obecný mnohopól je nakreslen na obr. \ref{es:fig_patocka_topol04}. Nechť má tento obvod \(r\) 
      vnějších svorek. Pak zřejmě celkový počet všech možných dvojic svorek, tj. všech možných 
      napětí v obvodu je roven
      \begin{equation}\label{ES:eq_topol07}
        s=(r-1) + (r-2) + (r-2) + \ldots + 3 + 2 + 1 = \frac{r(r-1)}{2}.
      \end{equation}
      \begin{figure}[ht!]   %\ref{es:fig_patocka_topol04}
        \centering
        \includegraphics[width=0.6\linewidth]{patocka_topol04.png}
        \caption{Obecný \(r\)-pól \cite[s.~46]{Patocka4}.}
        \label{es:fig_patocka_topol04}
      \end{figure}
      Zřejmě však stačí připojit \((r-1)\) vnějších nezávislých napájecích napětí (napěťových 
      zdrojů) a všechna zbývající napětí mezi libovolnými dvojicemi svorek jsou pak plně 
      definována, neboť mezi napětími platí II. KZ. Podobně stačí měřit \((r-1)\) vnějších proudů, 
      zbývající \(r\)-tý proud je na nich lineárně závislý (\(r\)-pól se chová jako uzel, mezi 
      jehož proudy platí 1. KZ). Vnější chování \(r\)-pólu je tedy plně popsáno vektorem \((r-1)\) 
      napětí a vektorem \((r-1)\) proudů. Odtud plyne, že vnější vlastnosti \(r\)-pólu jsou plně 
      definovány čtvercovou \(\mathbb{Z}\)-maticí nebo \(\mathbb{Y}\)-maticí o rozměru 
      \((r-l)\times(r-1)\). Uvedeme např. \(\mathbb{Z}\)-matici:  
      \begin{equation}\label{ES:eq_topol08}
        \left(
          \begin{array}{ccccc}
            z_{1,1}    &  z_{1,2}   & \ldots & z_{1,r-2}   & z_{1,r-1}    \\
            z_{2,1}    &  z_{2,2}   & \ldots & z_{2,r-2}   & z_{2,r-1}    \\
            \vdots     &  \vdots    & \ddots & \vdots      &\vdots        \\
            z_{r-2,1}  &  z_{r-2,2} & \ldots & z_{r-2,r-2} & z_{r-2,r-1}  \\
            z_{r-1,1}  &  z_{r-1,2} & \ldots & z_{r-1,r-2} & z_{r-1,r-1}  \\
          \end{array}
        \right)
      \end{equation}
      
      Mnohopól je \emph{zvláštním případem} obecného elektrického obvodu. V každém elektrickém 
      obvodu platí princip reciprocity, proto musí platit i v případě \(r\)-pólu. Odtud plyne, že 
      jeho matice (\ref{ES:eq_topol08}) musí být symetrická podle hlavní diagonály. To znamená, že 
      z celkového počtu \((r-1)^2\) maticových prvků je různých pouze \((r-1) + (r-2) + (r-3) +... 
      + 3 + 2 + 1\). Jinak řečeno, počet stupňů volnosti s libovolného pasivního \(r\)-pólu je dán 
      rovnicí (\ref{ES:eq_topol07}). Význam počtu stupňů volnosti \(s\) je mimořádný. Lze totiž 
      dokázat následující větu.
      
      \begin{lemma}\label{ES:lem_topol01}
        Z hlediska vnějšího chování lze každý libovolně složitý \(r\)-pól nahradit \(r\)-pólem
        ekvivalentním sestaveným pouze z tolika impedancí, kolik stupňů volnosti má původní obvod, 
        tedy sestaveným z \(s = \frac{r(-1)}{2}\) impedancí. Přičemž těchto \(s\) impedancí musí 
        být zapojeno ve shodě s obr. \ref{es:fig_patocka_topol05}, tj. každá dvojice svorek musí 
        být propojena jednou impedancí.
        \begin{figure}[ht!]
          \centering  
          \subcaptionbox{\label{es:fig_patocka_topol05a}}{\luafigure[0.11]{patocka_topol05a.png}}
          \subcaptionbox{\label{es:fig_patocka_topol05b}}{\luafigure[0.18]{patocka_topol05b.png}}
          \subcaptionbox{\label{es:fig_patocka_topol05c}}{\luafigure[0.25]{patocka_topol05c.png}}
          \subcaptionbox{\label{es:fig_patocka_topol05d}}{\luafigure[0.25]{patocka_topol05d.png}}
          \caption{Příklady náhrady obecného r-pólu ekvivalentním r-pólem: a) dvojpól, b) trojpól,
                   c) čtyřpól, d) pětipól. Číslo \(s\) souhlasí s rovnicí (\ref{ES:eq_topol07}).
                   \cite[s.~47]{Patocka4}} 
          \label{es:fig_patocka_topol05}
        \end{figure}
      \end{lemma}
      Důkaz věty bude naznačen v následující kapitole.
      
    \subsection{Počet stupňů volnosti pasivního čtyřpólu}\label{teo:IchapIIIsecIsubV}
      Důkaz věty \ref{ES:lem_topol01} ukážeme na případu čtyřpólu. Systematická konstrukce důkazu 
      je znázorněna na obr. \ref{es:fig_patocka_topol06}. Původní složitý čtyřpól začneme 
      dovolenými ekvivalentními úpravami měnit tak, abychom postupně snižovali celkový počet 
      odporů. Začneme-li např. v levém horním rohu, postup bude následující:
      \begin{figure}[ht!]   %\ref{es:fig_patocka_topol06}
        \centering
        \includegraphics[width=0.95\linewidth]{patocka_topol06.png}
        \caption{Postupná náhrada složitého čtyřpólu čtyřpólem ekvivalentním, sestaveným ze šesti
                 impedancí. \cite[s.~48]{Patocka4}.}
        \label{es:fig_patocka_topol06}
      \end{figure}
      \begin{enumerate}[noitemsep]
        \item Dva černě vyznačené odpory sečteme - počet prvků klesl o jedničku.
        \item V rohu vznikl trojúhelník ze tří černě vyznačených odporů.        
        \item Vzniklý černý trojúhelník přeměníme pomocí dobře známé transfigurace na hvězdu.
        \item Po vzniku hvězdy se objeví dva odpory zapojené do série (černý a bílý). Odpory 
              sečteme - počet prvků klesl o další jedničku.
        \item Po sečtení odporů se objeví dvě černě vyznačené hvězdy ležící vedle sebe.
        \item Obě hvězdy přeměníme pomocí transfigurace na dva trojúhelníky (tato operace již není 
              nakreslena).
        \item Po vzniku trojúhelníků se objeví na sousedících stranách trojúhelníků dva odpory 
              zapojené paralelně. Odpory sečteme, čímž počet prvků klesne o další jedničku.
        \item Postupným střídáním transfigurací z hvězd na trojúhelníky a opačně, z trojúhelníků na 
              hvězdy, neustále snižujeme počet prvků o jedničku.
        \item Algoritmus lze neustále opakovat až do okamžiku, kdy nám zbude šest odporů zapojených 
              do můstku podle obrázku. Toto zápojem již nelze žádným postupem více minimalizovat.
      \end{enumerate}
      Konec důkazu.
      
      Ve zvláštním případě, jsou-li svorky \(b, d\) původního čtyřpólu vodivě spojeny (jedná se pak 
      o trojpól), výsledkem algoritmu bude \(T\)-článek sestavený pouze ze tří odporů a ten lze 
      případně přepočítat pomocí transfigurace hvězda-trojúhelník na rovnocenný 
      \(\Pi\)-článek\footnote{Lze dokázat, že transfigurace \emph{hvězda-trojúhelník} je 
      uskutečnitelná právě pouze pro \emph{trojpól}. Pro všechny ostatní r-póly, u nichž \(r > 3\), 
      realizovatelná \emph{není}. Jev není náhodný, úzce souvisí se \emph{třemi stupni volnosti}.}.
      
      V obecném případě je ale výsledkem zjednodušovacího algoritmu vždy obvod sestavený ze šesti 
      odporů podle obr. \ref{es:fig_patocka_topol07}. Obvod lze nakreslit ve tvaru křížového článku 
      nebo ve tvaru můstku zatíženého na vstupu i výstupu impedancemi. Křížový článek a zatížený 
      můstek jsou obvody topologicky totožné.
      \begin{figure}[ht!]
        \centering  
        \includegraphics[width=0.95\linewidth]{patocka_topol07.png}
        \caption{Náhrada libovolné složitého čtyřpólu čtyřpólem ekvivalentním.
                \cite[s.~48]{Patocka4}} 
        \label{es:fig_patocka_topol07}
      \end{figure}
      
    \subsection{Počet stupňů volnosti pasivního dvojbranu}\label{teo:IchapIIIsecIsubVI}
      Tuto a následující kapitoly je nutno chápat jako důležitou teoretickou připravu, bez níž není 
      možno plně pochopit teorii transformátoru uvedenou v kapitole 17.
      
      Přenosový dvojbran je zvláštním případem čtyřpólu. Čtyř\-pól podle obr. 
      \ref{es:fig_patocka_topol07} může být považován za přenosový dvojbran, pokud nás nebudou 
      zajímat podélná napětí mezi svorkami \(a-c\), \(a-d\), \(b-c\), \(b-d\). Pak se z něj stává 
      dvojbran podle obr. \ref{es:fig_patocka_topol08}, který má pouze dvě brány, z nichž jednu 
      považujeme za vstupní a druhou za výstupní.
      
      Vnější chování dvojbranu je tedy popsáno vektorem \emph{dvou napětí} a vektorem \emph{dvou 
      proudů}. Vnější vlastnosti dvojbranu jsou proto plně definovány čtvercovou 
      \(\mathbb{Z}\)-maticí nebo  \(\mathbb{Y}\)-maticí o rozměru \(2\times2\). Uvedeme např. 
      \(\mathbb{Z}\)-matici:
      \begin{equation}\label{ES:eq_topol09}
        \left(
          \begin{array}{c}
            U_1 \\ U_2   
           \end{array}
        \right)
        =
        \left(
        \begin{array}{cc}
          z_{11}    &  z_{12}   \\
          z_{21}    &  z_{22}   
        \end{array}
        \right)
        \cdot
        \left(
          \begin{array}{c}
            I_1 \\ I_2 
          \end{array}
        \right).         
      \end{equation}
      
      \begin{figure}[ht!]
        \centering  
        \includegraphics[width=0.5\linewidth]{patocka_topol08.png}
        \caption{Označení veličin vstupní a výstupní brány dvojbranu. \cite[s.~49]{Patocka4}} 
        \label{es:fig_patocka_topol08}
      \end{figure}
      
      Pasivní dvojbran musí splňovat princip reciprocity. Jeho matice tedy musí být symetrická 
      podle hlavní diagonály, proto musí platit \(z_{12} = z_{21}\). To znamená, že z celkového 
      počtu čtyř maticových prvků jsou pouze \emph{tři prvky různé}. Toto zjištění je přímým 
      důkazem následující věty:
      \begin{lemma}\label{ES:lem_topol02}
         Každý pasivní přenosový dvojbran má pouze \emph{tři stupně volnosti}. Proto může být vždy 
         nahrazen ekvivalentním \emph{trojpólem} ve tvaru \(T\)-článku (nebo \(\Pi\)-článku) 
         sestaveným ze tří impedancí. Proto jsou všechny přenosové vlastnosti dvojbranu plně 
         definovány:
         \begin{itemize}[noitemsep]
            \item buď \emph{trojicí} parametrů obvodových (u transformátoru např. \(L_1, L_2, M\)),
            \item nebo \emph{trojicí} parametrů přenosových (např. \(Z_{vst_0}\), \(Z_{vst_K}\),    
                  \(K_{U_{21,0}}\)),
            \item nebo \emph{trojicí} maticových koeficientů (např. \(z_{11}, z_{21}, z_{22}\)).
          \end{itemize}
      \end{lemma}
      
    \subsection{Popis dvojbranu pomocí matic typu Z, Y, H}\label{teo:IchapIIIsecIsubVII}
      Vnější chování každého dvojbranu lze popsat maticí o dimenzi \(2\times2\). Jedná se tedy o 
      soustavu dvou rovnic o \emph{čtyřech} vnějších veličinách \(U_1, U_2, I_1, I_2\). Má-li být 
      soustava řešitelná, je nutno některé \emph{dvě veličiny zvolit za neznámé} a zbylé dvě 
      považovat při výpočtu za \emph{známé}. Existuje šest různých dvojic vybraných ze čtveřice, 
      tomu odpovídá šest typů matic. Kromě známé \(\mathbb{Z}\)-matice a \(\mathbb{Y}\)-matice 
      existují další čtyři hybridní \(\mathbb{H}\)- matice:
      \begin{align*}
        \shortintertext{\texttt{matice }\(\mathbb{Z}\):}
          \left(
            \begin{array}{c}
              U_1 \\ U_2   
            \end{array}
          \right)
          &=
          \left(
            \begin{array}{cc}
              z_{11}    &  z_{12}   \\
              z_{21}    &  z_{22}   
            \end{array}
          \right)
          \cdot
          \left(
            \begin{array}{c}
              I_1 \\ I_2 
            \end{array}
          \right)                                                 \\
        \shortintertext{\texttt{matice }\(\mathbb{H}_K\):}
          \left(
            \begin{array}{c}
              U_1 \\ I_1   
            \end{array}
            \right)
            &=
            \left(
              \begin{array}{cc}
                h_{K_{11}}    &  h_{K12}   \\
                h_{K_{21}}    &  h_{K22}   
              \end{array}
            \right)
            \cdot
            \left(
              \begin{array}{c}
                U_2 \\ I_2 
              \end{array}
          \right)                                               \\
        \shortintertext{\texttt{matice }\(\mathbb{H}_I\): }
          \left(
            \begin{array}{c}
              U_1 \\ I_2   
            \end{array}
          \right)
          &=
          \left(
            \begin{array}{cc}
              h_{I_{11}}    &  h_{I_{12}}   \\
              h_{I_{21}}    &  h_{I_{22}}   
            \end{array}
          \right)
          \cdot
          \left(
            \begin{array}{c}
              I_1 \\ U_2 
            \end{array}
          \right)                                              \\
        \shortintertext{\texttt{matice }\(\mathbb{H}_U\):}
          \left(
            \begin{array}{c}
              I_1 \\ U_2   
            \end{array}
            \right)
            &=
            \left(
              \begin{array}{cc}
                h_{U_{11}}    &  h_{U_{12}}   \\
                h_{U_{21}}    &  h_{U_{22}}   
              \end{array}
            \right)
            \cdot
            \left(
              \begin{array}{c}
                U_1 \\ I_2 
              \end{array}
          \right)                                              \\
        \shortintertext{\texttt{matice }\(\mathbb{H}_N\):}
          \left(
            \begin{array}{c}
              U_2 \\ I_2   
            \end{array}
          \right)
          &=
          \left(
            \begin{array}{cc}
              h_{N_{11}}    &  h_{N_{12}}   \\
              h_{N_{21}}    &  h_{N_{22}}   
            \end{array}
          \right)
          \cdot
          \left(
            \begin{array}{c}
              U_1 \\ I_1 
            \end{array}
          \right)                                               \\
        \shortintertext{\texttt{matice }\(\mathbb{Y}\):}
          \left(
            \begin{array}{c}
              I_1 \\ I_2   
            \end{array}
            \right)
            &=
            \left(
              \begin{array}{cc}
                y_{11}    &  y_{12}   \\
                y_{21}    &  y_{22}
              \end{array}
            \right)
            \cdot
            \left(
              \begin{array}{c}
                U_1 \\ U_2 
              \end{array}
          \right) 
      \end{align*}
      Jednotlivým maticím odpovídají různé matematické modely dvojbranu. Mimořádný význam má 
      hybridní \(\mathbb{H}_U\)-matice vho\-dná pro modely všech dvojbranu napájených zdrojem 
      \emph{napětí} (tedy vhodná pro model transformátoru napětí) a hybridní 
      \(\mathbb{H}_I\)-matice vhodná pro modely všech dvojbranu napájených zdrojem \emph{proudu} 
      (tedy vhodná pro model transformátoru proudu). Kaskádní dopředná matice typu \(\mathbb{H}_K\) 
      a kaskádní zpětná matice typu \(\mathbb{H}_N\) se pro popis transformátoru příliš nehodí.

      
    \subsection{Přenosové parametry dvojbranu}\label{teo:IchapIIIsecIsubVIII}
      Všechny možné přenosové parametry jsou uvedeny v tab. \ref{ES:tab_topol01}. Je jich celkem 
      \num{16}, což není náhoda (\(2^4 = 16\)). Tabulka ukazuje u každého přenosového parametru 
      jeho \emph{elementární} vztah k některému maticovému parametru. Prázdná místa v tabulce by 
      bylo možno zaplnit \emph{složitými} algebraickými vztahy (do tabulky se bohužel nevejdou), 
      každý přenosový parametr se dá totiž vždy vyjádřit pomocí čtveřice parametrů \(z\) nebo \(y\) 
      nebo \(h\) (viz následující příklad). Vztahy lze snadno odvodit s využitím třetí doplňující 
      rovnice \(U_j = 0\) (\emph{brána nakrátko}) nebo \(I_j = 0\) (\emph{brána naprázdno}), kde 
      \(j = 1\) (vstupní brána) nebo \num{2} (výstupní brána). Konkrétní postup plyne z 
      následujícího příkladu.
      \begin{table*}[!t]
        \centering
        \resizebox{0.9\textwidth}{!}{
          \begin{tabular}{|l|l|l|l|l|l|l|l|}
            \hline
            &  & \multicolumn{6}{l|}{{\textbf{Vztah k maticovým parametrům} \(z, h, y\)}} \\ 
            \cline{3-8} 
            \multirow{-2}{*}{\textbf{Přenosový parametr}} & \multirow{-2}{*}{\textbf{Značka}} 
            & \(z\) & \(h_K\) & \(h_I\) & \(h_U\) & \(h_N\) & \(y\) \\ 
            \hline\hline
            vstupní impedance při výstupu naprázdno  
            & \(Z_{vst_0}\) 
                  & \(z_{11}\) & & & & &                      \\
            vstupní impedance při výstupu nakrátko  
            & \(Z_{vst_K}\) 
                  & & & \(h_{I_{11}}\) & & &                  \\
            výstupní impedance při vstupu naprázdno  
            & \(Z_{\text{výst}_0}\) 
                  & \(z_{22}\) & & & \(h_{U_{22}}\) & &       \\
            výstupní impedance při vstupu nakrátko   
            & \(Z_{\text{výst}_K}\) 
                  & & & & & &                                 \\
            přenosová impedance naprázdno (\(\rightarrow\))
            & \(Z_{21_0}\) 
                  & \(z_{21}\) & \(1/h_{K_{21}}\) & & & &     \\
            přenosová impedance naprázdno (\(\leftarrow\))
            & \(Z_{12_0}\) 
                  & \(z_{12}\) & & & & \(1/h_{N_{21}}\) &     \\
            vstupní admitance při výstupu naprázdno 
            & \(Y_{vst_0}\) 
                  & & & & \(h_{U_{11}}\)& &                   \\
            vstupní admitance při výstupu nakrátko  
            & \(Y_{vst_K}\) 
                  & & & & & & \(y_{11}\)                      \\
            výstupní admitance při vstupu naprázdno  
            & \(Y_{\text{výst}_0}\) 
                  & & & \(h_{I_{22}}\) & & &                  \\
            výstupní admitance při vstupu nakrátko  
            & \(Y_{\text{výst}_K}\) 
                  & & & & & & \(y_{22}\)                      \\
            přenosová admitance nakrátko (\(\rightarrow\))
            & \(Y_{21_K}\) 
                  & & \(1/h_{K_{11}}\) & & & & \(y_{21}\)     \\
            přenosová admitance nakrátko (\(\leftarrow\))
            & \(Y_{12_K}\) 
                  & & & & & \(1/h_{N_{12}}\) & \(y_{12}\)     \\
            napěťový přenos naprázdno (\(\rightarrow\))
            & \(K_{U_{21_0}}\) 
                  & & \(1/h_{K_{11}}\) & & \(h_{U_{21}}\) & & \\
            napěťový přenos naprázdno (\(\leftarrow\))
            & \(K_{U_{12_0}}\) 
                  & & & \(h_{I_{12}}\) & & \(1/h_{N_{11}}\) & \\
            proudový přenos nakrátko (\(\rightarrow\))
            & \(K_{I_{21_K}}\) 
                  & & \(1/h_{K_{22}}\) & \(h_{I_{21}}\) & & & \\
            proudový přenos nakrátko (\(\leftarrow\))
            & \(K_{I_{12_K}}\) 
                  & & & & \(h_{U_{12}}\) & \(1/h_{N_{22}}\) & \\
            \hline
          \end{tabular}
        }
        \caption{Přehled všech možných přenosových parametrů dvojbranu.}
        \label{ES:tab_topol01}
      \end{table*}

      % --------example: Metoda smyčkových proudů ------------
      % \label{TEO:exam016}
      % !TeX spellcheck = cs_CZ
\begin{example}\label{TEO:exam016}
  Vyjádřeme pomocí \(y\)-parametrů následující přenosové parametry dvojbranu: vstupní 
  impedance při výstupu naprázdno \(Z_{vst_0}\), \(K_{I_{12_K}}\), \(Z_{12_0}\). 
  
  Řešení:
  \begin{itemize}\addtolength{\itemsep}{-0.5\baselineskip}
    \item \(Z_{vst_0}\): \(\mathbb{Y}\)-matici doplníme třetí rovnicí \(I_2=0\) (podmínka 
    definující výstup naprázdno):
    \begin{subequations}\label{ES:eq_topol10}
      \begin{align}
        I_1 &= y_{11}U_1 - y_{12}U_2  \\
        I_2 &= y_{21}U_1 - y_{22}U_2  \\
        I_2 &= 0
      \end{align}
    \end{subequations}
    Záporná znaménka v \(\mathbb{Y}\)-matici (tj. v prvních dvou rovnicích) jsou 
    správná, protože směr výstupního proudu \(I_2\) na obr. 
    \ref{es:fig_patocka_topol08} je zvolen v souladu s realitou, tj. v souladu s 
    napětím \(U_2\). Pokud je totiž dvojbran zatížen \emph{pasivní} zátěží, pracující 
    ve \emph{spotřebičovém} režimu, proud \(I_2\) musí z výstupní svorky dvojbranu 
    vytékat, nikoli do ní vtékat\footnote{Směry obou proudů \(I_1\), \(I_2\) lze volit 
      libovolně. V literatuře se obvykle směry proudů volí tak, že oba vtékají do 
      dvojbranu. Důvod je pouze formální, pak totiž budou všechny čtyři maticové 
      koeficienty kladně. Výsledkem je ale nerealistický stav na výstupní bráně, tj. na 
      pasivní zátěži. Proto nebudeme toto značeni používat a směry všech veličin budeme 
      systematicky volit podle obr. \ref{es:fig_patocka_topol08}. Z psychologického 
      hlediska tato volba velmi usnadní budoucí analýzu transformátoru.}. Všimněme si, že 
    jsme získali soustavu tří rovnic o čtyřech neznámých veličinách \(U_1\), \(U_2\), 
    \(I_1\), \(I_2\). Ta je pro \emph{jednotlivé} neznámé neřešitelná. Je však 
    řešitelná pro poměry dvou libovolných proměnných. Ze soustavy lze po algebraických 
    úpravách získat hledanou vstupní impedanci při výstupu naprázdno:
    \begin{equation}\label{ES:eq_topol11}
    Z_{vst_0} = \frac{U_1}{I_1}\Bigg|_{I_2=0} 
    = \frac{y_{22}}{y_{22}y_{11} - y_{12}y_{21}}.
    \end{equation}
    \item \(K_{I_{12_K}}\): \(\mathbb{Y}\)-matici doplníme třetí rovnicí \(U_1 = 0\) 
    (podmínka definující vstup nakrátko). Jedná se o přenos ve zpětném směru 
    (\(\leftarrow\)). Při zpětném přenosu tečou oba proudy na obr. 
    \ref{es:fig_patocka_topol08} obráceným směrem, proto musí mít přiřazena záporná 
    znaménka:
    \begin{equation}\label{ES:eq_topol12}
    K_{I_{12_K}} = \frac{-I_1}{-I_2}\Bigg|_{U_1=0} 
    = \frac{I_1}{I_2}\Bigg|_{U_1=0}
    = \frac{y_{12}}{y_{22}}.
    \end{equation}          
    \item \(Z_{12_0}\): \(\mathbb{Y}\)-matici doplníme třetí rovnicí \(I_1=0\) (podmínka  
    definující vstup naprázdno). Při zpětném přenosu (\(\leftarrow\)) tečou oba 
    proudy na obr. \ref{es:fig_patocka_topol08} obráceným směrem, proto musí 
    mít přiřazena záporná znaménka:
    \begin{equation}\label{ES:eq_topol13}
    Z_{vst_0} = \frac{U_1}{-I_2}\Bigg|_{I_1=0} 
    = \frac{y_{12}}{y_{21}y_{12} - y_{11}y_{22}}.
    \end{equation}
  \end{itemize}
  Tytéž přenosové parametry \(Z_{vst_0}\), \(K_{I_{12_K}}\), \(Z_{12_0}\) je možno vyjádřit 
  pomocí \(z\)-parametrů nebo \(h\)-parametrů. Podobným způsobem by bylo možno doplnit 
  všechna prázdná pole v tab. \ref{ES:tab_topol01}.
\end{example}
  
      %-------------------------------------------------------

      Ze tří stupňů volnosti dvojbranu plyne, že jen některé speciálně vybrané trojice ze
      šestnácti přenosových parametrů jsou na sobě nezávislé, např. \(Z_{vst_0}\),  \(Z_{vst_K}\), 
      \(K_{U_{21_0}}\). Většina parametrů musí být navzájem lineárně závislá. Jako typické příklady 
      lineárních závislostí uveďme:
      \begin{equation*}          %\label{ES:eq_topol14}
        Z_{vst_0} = \frac{1}{Y_{vst_0}}, \quad 
        \frac{Z_{vst_0}}{Z_{vst_K}}=\frac{Z_{\text{výst}_0}}{Z_{\text{výst}_K}}, \quad
        \frac{Z_{vst_0}}{Z_{\text{výst}_0}}=\frac{Z_{\text{vst}_K}}{Z_{\text{výst}_K}}
      \end{equation*}
      Existují i mnohé další, komplikovanější. Z hlediska transformátoru jsou však velmi důležité 
      jiné dvě lineární závislosti:
      \begin{equation}\label{ES:eq_topol15}
        K_{U_{21_0}} = K_{I_{12_K}}, \qquad 
        K_{I_{21_K}} = K_{U_{12_0}}
      \end{equation}
      U transformátoru jsou tyto rovnice všeobecně známy ve smyslu tj. napěťový převod \emph{tam} 
      je stejný jako proudový převod \emph{zpět}. Ze systémového pohledu je však dobré uvědomit si, 
      že:
      \begin{itemize}[noitemsep]
        \item tuto vlastnost mají \emph{všechny} pasivní dvojbrany, např. i obyčejný odporový dělič,
        \item transformátor není žádnou výjimkou,
        \item tzv. převod transformátoru je fyzikálně totožný s pojmem \emph{napěťový přenos 
              naprázdno} nebo \emph{proudový přenos nakrátko}.
      \end{itemize}
      
    \subsection{Princip reciprocity v elektromechanických 
    soustavách}\label{teo:IchapIIsecIsubIX}
      Z hlediska teoretického je velmi zajímavé, že princip reciprocity platí i u všech 
      \emph{elektromechanických měničů} energie. Pohlížíme-li např. na stejnosměrný motor s 
      konstantním buzením jako na „přenosový dvojbran“ s elektrickými veličinami \(u(t)\), \(i(t)\) 
      na elektrické bráně a s mechanickými veličinami \(M(t)\), \(\omega(t)\) na mechanické bráně, 
      pak lze tento systém popsat např. hybridní \(\mathbb{H}\)-maticí ve tvaru
      \begin{subequations}\label{ES:eq_topol16}
        \begin{align}
          u(t) &= Ri(t) + L\der{i(t)}{t} + NBlr\omega(t)  \\
          M(t) &= NBlri(t) - J_i\der{\omega(t)}{t}        \\
          \shortintertext{neboli}
             u &= (R+pL)i + NBlr\omega  \\
             M &= NBlri - pJ_i\omega
        \end{align}
      \end{subequations}
      Je vidět, že \(\mathbb{H}\)-matice je opravdu symetrická podle hlavní diagonály a reciprocita 
      elektromechanického systému je kupodivu (!) zajištěna \emph{dualitou konstant} v rovnicích 
      indukčního zákona a zákona Lorentzovy síly:
      \begin{subequations}\label{ES:eq_topol17}
        \begin{align}
          u_i(t) &= NBlr\omega(t) = c\Phi\omega(t), \\
          \shortintertext{neboli}
          M_i(t) &= NBlr i(t) = c\Phi i(t),
        \end{align}
      \end{subequations}
      kde dolním indexem \(i\) jsou značeny \emph{vnitřní} veličiny stroje. \(c\Phi\) je tzv. 
      \emph{konstanta stroje}. Na otázku „Proč jsou převodní konstanty NBlr ve dvou různých 
      fyzikálních zákonech stejné?“ lze odpovědět jedině takto: „Protože všechny fyzikální 
      přenosové soustavy musí splňovat princip reciprocity.“ Zdůrazněme, že podobnou symetrii lze 
      nalézt bez výjimky u všech ostatních typů elektromechanických měničů, počínaje 
      elektromagnetem a konče asynchronním strojem.
  
  \section{Topologie spojitých útvarů}\label{teo:IchapIIIsecII}
    Při studiu magnetických jevů se setkáváme s diskrétním magnetickým obvodem spíše výjimečně, 
    obvykle pouze u cívek a transformátorů s feromagnetickým jádrem a navíc při vědomém zanedbání 
    vzdušných rozptylových cest. Pokud rozptylové toky nechceme nebo nemůžeme zanedbat, musíme 
    pracovat s magnetickým polem spojitě rozprostřeným v prostoru. V tom případě je vhodné 
    analyzovat problém z pohledu topologie spojitých prostorových útvarů
      
    \subsection{Základní pojmy a definice}\label{teo:IchapIIsecIIsubI}
      Základními topologickými útvary spojitě rozprostřenými v trojrozměrném\footnote{Obecněji lze 
      pracovat s topologickými útvary v libovolném \(N\)-rozměmém prostoru.} prostoru, jsou      
      plochy \(S\) a křivky \(l\). Plochy můžeme dělit na \emph{orientované} a \emph{neorientované}:
      \begin{itemize}[noitemsep]
        \item \textbf{Orientovaná plocha} - má dvě různé strany, které lze pokrýt dvěma různými 
               barvami (červená \textbf{č}, zelená \textbf{z}), aniž se barvy pomíchají nebo 
               překrývají.
        \item \textbf{Neorientovaná plocha} - má pouze jedinou stranu, lze ji pokrýt pouze jedinou 
              barvou.
      \end{itemize}
      
      Každá z obou ploch může být navíc \emph{ohraničená} nebo \emph{neohraničená}:
      \begin{itemize}[noitemsep]
        \item \textbf{Ohraničená plocha} \(S\) - má hranici v podobě uzavřené nepřerušené hraniční 
             křivky \(l\). Plocha nerozděluje 3D prostor na dvě oddělené části. Příkladem je kruh 
             umístěný v 3D prostoru.
        \item \textbf{Neohraničená plocha} \(S\) - nemá hranici, hraniční křivka \(l\) neexistuje. 
              Plocha rozděluje 3D prostor na dvě oddělené části. Příkladem je povrch koule 
              rozdělující prostor na vnitřní a vnější.
      \end{itemize}
      
      Lze vytvořit čtyři vzájemné kombinace obou uvedených vlastností, kterým odpovídají čtyři      
      základní typy ploch:
      \begin{itemize}[noitemsep]
        \item \textbf{Orientovaná ohraničená plocha} \(S_P\) „\emph{pytel}“ - příkladem jsou kruh, 
              pytel, kapsa, viz obr. \ref{es:fig_patocka_topol09}.
        \item \textbf{Orientovaná neohraničená plocha} \(S_S\) „\emph{sféra}“ - příkladem jsou 
              povrch koule, povrch krychle, nekonečná rovina, viz obr. 
              \ref{es:fig_patocka_topol10}. Tyto plochy mají dvě strany, vnější a vnitřní, nemají 
              ale žádnou hraniční křivku. Sféra vznikne vzájemným sešitím dvou pytlů po celém 
              obvodu jejich hraničních křivek. Tím hraniční křivka zmizí.      
        \item \textbf{Neorientovaná ohraničená plocha} \(S_M\) „Möbiův\footnote{August Ferdinand 
              Möbius (1790-1868). německý matematik. Působil na univerzitě v Lipsku. Zabýval se 
              projektivní geometrií a teorií čísel.} list“ - jediným zástupcem je Möbiův list, viz 
              obr. \ref{es:fig_patocka_topol11}. Jedná se o proužek papíru, jehož jeden konec 
              zkroutíme o \SI{180}{\degree} (nebo o liché násobky \SI{180}{\degree}) 
              a oba konce slepíme. Útvar má pouze jedinou stranu, ale má uzavřenou hraniční křivku.
        \item \textbf{Neorientovaná neohraničená plocha} \(S_K\) „Kleinova\footnote{Felix Klein 
              (1849-1925), německý matematik. Působil na univerzitách v Erlangenu. v Lipsku, v 
              Göttingenu. Zabýval se vztahy mezi geometrií a teorii grup.} láhev“ - jediným 
              zástupcem je Kleinova láhev, viz obr. \ref{es:fig_patocka_topol12}. Útvar vznikne 
              vzájemným sešitím levotočivého Möbiova listu s pravotočivým Möbiovým listem po celém 
              obvodu jejich hraničních křivek\footnote{Kleinova láhev může vzniknout i vzájemným 
              sešitím stejnolehle točivých Möbiových listů. Výsledný útvar však nelze 
              nakreslit tak jednoduše jako láhev na obr. \ref{es:fig_patocka_topol12}.}. Tím 
              hraniční křivka zmizí. Plocha láhve má pouze jedinou stranu.
    \end{itemize}
    \begin{figure}[ht!]
      \centering  
      \subcaptionbox{\label{es:fig_patocka_topol09a}}{\luafigure[0.15]{patocka_topol09a.png}}
      \subcaptionbox{\label{es:fig_patocka_topol09b}}{\luafigure[0.22]{patocka_topol09b.png}}
      \subcaptionbox{\label{es:fig_patocka_topol09c}}{\luafigure[0.22]{patocka_topol09c.png}}
      \subcaptionbox{\label{es:fig_patocka_topol09d}}{\luafigure[0.23]{patocka_topol09d.png}}
      \caption{Orientovaná ohraničená plocha, a) Kruh. b) Pytel, c) Pytel do sebe namotaný, d) Pytel
               sám sebe protínající. \cite[s.~53]{Patocka4}} 
      \label{es:fig_patocka_topol09}
    \end{figure}
      
    \begin{figure}[ht!]
      \centering  
      \subcaptionbox{\label{es:fig_patocka_topol10a}}{\luafigure[0.25]{patocka_topol10a.png}}
      \subcaptionbox{\label{es:fig_patocka_topol10b}}{\luafigure[0.25]{patocka_topol10b.png}}   
      \caption{Orientovaná neohraničená plocha, a) Koule. b) Nekonečná rovina \(x-y\) 
               \cite[s.~54]{Patocka4}} 
      \label{es:fig_patocka_topol10}
    \end{figure}

    \begin{figure}[ht!]
      \centering  
      \subcaptionbox{\label{es:fig_patocka_topol11a}}{\luafigure[0.27]{patocka_topol11a.png}}
      \subcaptionbox{\label{es:fig_patocka_topol11b}}{\luafigure[0.27]{patocka_topol11b.png}}
      \subcaptionbox{\label{es:fig_patocka_topol11c}}{\luafigure[0.27]{patocka_topol11c.png}}
      \caption{Neorientovaná ohraničená plocha, a) Pravotočivý Möbiův list. b) Deformovaný
               pravotočivý Möbiův list. c) Zrcadlově symetrický levotočivý Möbiův list.
               \cite[s.~54]{Patocka4}} 
      \label{es:fig_patocka_topol11}
    \end{figure}
      
    \begin{figure}[ht!]
      \centering  
      \includegraphics[width=0.85\linewidth]{patocka_topol12.png}
      \caption{Neorientovaná neohraničená plocha - Kleinova láhev. Vznikne sešitím pravotočivého
               Möbiova listu s levotočivým listem po celém obvodu jejich hraničních křivek.
               \cite[s.~54]{Patocka4}} 
      \label{es:fig_patocka_topol12}
    \end{figure}
    
    \subsection{Topologické operace}\label{teo:IchapIIsecIIsubII}
      Povolené topologické operace jsou všechny myslitelné deformace, při kterých zůstává zachována 
      kvalita plochy (typ plochy). Z toho vyplývají následující pravidla:
      \begin{itemize}[noitemsep]
        \item Plocha může v důsledku deformací protínat sama sebe libovolným způsobem (může 
              pronikat sama sebou), viz obr. \ref{es:fig_patocka_topol09}, obr. 
              \ref{es:fig_patocka_topol12}.
        \item Plocha může být v důsledku deformací libovolně navinuta na svoji vlastní hraniční  
              křivku.
        \item Uzavřená hraniční křivka nesmí být v důsledku deformací přerušena.
        \item Plocha může být ohraničena více hraničními křivkami současně, viz obr 
              \ref{es:fig_patocka_topol13}. V důsledku deformací se ale nesmí změnit počet 
              hraničních křivek, tj. nesmí se změnit počet otvorů.
     \end{itemize}
      
      Pro ilustraci je na obr \ref{es:fig_patocka_topol13} uveden sled povolených topologických 
      operací. Intuitivně se zdá, že není možný přechod z počátečního stavu do koncového, aniž 
      přitom porušíme celistvost kroužku nebo poutka. Obrázek ukazuje, že to možné je. Operace jsou 
      povolené, protože během nich nedošlo ke změnám kvality obou objektů. Poutko má stále dva 
      otvory, kroužek stále jeden otvor. Jedná se o pouhou tvarovou deformaci poutka.
    
      \begin{figure}[ht!]
        \centering  
        \includegraphics[width=0.85\linewidth]{patocka_topol13.png}
        \caption{Příklad povolených topologických operací.\cite[s.~54]{Patocka4}} 
        \label{es:fig_patocka_topol13}
      \end{figure}
    
    \subsection{Některé topologické věty}\label{teo:IchapIIsecIIsubIII}
      Na základě pravidel uvedených v předchozí kapitole lze vyslovit následující věty:
      \begin{lemma}\label{ES:lem_topol03}
        Sešitím dvou \emph{ohraničených orientovaných} ploch \(S_P\) po celém obvodu jejich 
        hraničních křivek \(l\) vznikne \emph{neohraničená orientovaná} plocha \(S_S\). Tedy 
        \(S_P\cap S_P \Rightarrow S_S\). Plocha \(S_S\) rozděluje prostor na dvě oddělené části.
      \end{lemma}  
    
      \begin{lemma}\label{ES:lem_topol04}
        Sešitím dvou \emph{ohraničených neorientovaných} ploch \(S_M\) po celém obvodu jejich 
        hraničních křivek \(l\) vznikne \emph{neohraničená neorientovaná} plocha \(S_K\).
        Tedy \(S_M\cap S_M \Rightarrow S_K\). Plocha \(S_K\) rozděluje prostor na dvě oddělené 
        části (pokud na obr. \ref{es:fig_patocka_topol12} v místě průniku hrdla stěnou zakážeme 
        vznik otvoru ve stěně).
      \end{lemma} 
      
      \begin{lemma}\label{ES:lem_topol05}
        Sešitím \emph{ohraničené orientované} plochy \(S_P\) s \emph{ohraničenou neorientovanou}
        plochou \(S_M\) po celém obvodu jejich hraničních křivek \(l\) vznikne neohraničená 
        neorientovaná plocha \(S_K\) . Tedy \(S_P\cap S_M \Rightarrow S_K\). Plocha \(S_K\) 
        rozděluje prostor na dvě oddělené části (pokud zakážeme vznik nových otvorů v 
        místech průniků).
      \end{lemma} 
      
      Ve Větách \ref{ES:lem_topol03} až \ref{ES:lem_topol05} vznikne po sešití obou původních ploch 
      \(S_P\cap S_P\), \(S_M\cap S_M\) nebo \(S_P\cap S_M\) vždy plocha \emph{neohraničená}, tedy 
      zmizí šev, tj. zmizí hraniční křivka \(l\). Představme si však situaci, že hraniční křivku 
      \(l\) necháme i po sešití úmyslně vyznačenu v podobě čáry na výsledné sešité 
      \emph{neohraničené} ploše \(S_S\) nebo \(S_K\). Odtud plyne přímý důkaz tří následujících 
      „zrcadlových" vět k větám \ref{ES:lem_topol03} až \ref{ES:lem_topol05}:
    
      \begin{lemma}\label{ES:lem_topol06}
        Každá hraniční křivka \(l\) může současně tvořit hranici dvou ohraničených orientovaných 
        ploch \(S_P\). Viz obr. \ref{es:fig_patocka_topol14}.
        \begin{figure}[ht!]
          \centering  
          \includegraphics[width=0.85\linewidth]{patocka_topol14.png}
          \caption{Hraniční křivka \(l\) tvoří současně hranici dvou \emph{ohraničených 
                   orientovaných} ploch \(S_{P1}\), \(S_{P2}\) (dvou sešitých 
                   pytlů).\cite[s.~55]{Patocka4}} 
          \label{es:fig_patocka_topol14}
        \end{figure}
      \end{lemma}
      
      \begin{lemma}\label{ES:lem_topol07}
        Každá hraniční křivka \(l\) může současně tvořit hranici dvou \emph{ohraničených 
        neorientovaných} ploch \(S_M\). Viz obr. \ref{es:fig_patocka_topol15}
        \begin{figure}[ht!]
          \centering  
          \includegraphics[width=0.85\linewidth]{patocka_topol15.png}
          \caption{Hraniční křivka \(l\) tvoří současně hranici dvou \emph{ohraničených 
                   neorientovaných} ploch \(S_{M1}\), \(S_{M2}\) (dvou sešitých Möbiových 
                   listů).\cite[s.~55]{Patocka4}} 
          \label{es:fig_patocka_topol15}
        \end{figure}
      \end{lemma}

      \begin{lemma}\label{ES:lem_topol08}
        Každá hraniční křivka \(l\) může současně tvořit hranici jedné \emph{ohraničené 
          orientované} plochy \(S_P\) a jedné \emph{ohraničené neorientované} plochy \(S_M\). Viz 
        obr. \ref{es:fig_patocka_topol16}
        \begin{figure}[ht!]
          \centering  
          \subcaptionbox{\label{es:fig_patocka_topol16a}}{\luafigure[0.45]{patocka_topol16a.png}}
          \subcaptionbox{\label{es:fig_patocka_topol16b}}{\luafigure[0.45]{patocka_topol16b.png}}
          \caption{Hraniční křivka \(l\) tvoří současně hranici jedné \emph{ohraničené orientované} 
            plochy \(S_P\) a jedné \emph{ohraničené neorientované} plochy \(S_M\). a) U 
            jednovrstvé cívky lze snadno vyznačit pouze \emph{orientovanou} plochu \(S_P\). 
            Plocha \(S_M\) Möbiova listu zde existuje, ale nelze ji nakreslit, b) 
            U dvojvrstvé cívky lze naopak snadno vyznačit \emph{neorientovanou} plochu 
            \(S_M\) Möbiova listu. Plocha \(S_P\) zde existuje, ale nelze ji 
            nakreslit.\cite[s.~56]{Patocka4}} 
          \label{es:fig_patocka_topol16}
        \end{figure}
      \end{lemma}
    
    \subsection{Aplikace topologických vět na Maxwellovy rovnice}\label{teo:IchapIIsecIIsubIV}
      Nyní na základě topologických vět můžeme upřesnit základní větu \ref{es:fig_patocka_lemma01} 
      o velikosti plošného integrálu \(\Psi = \int_S\vec{B}\cdot \dd{\vec{S}}\) z kapitoly 
      \ref{ES:sec02}:
      \begin{lemma}\label{ES:lem_topol09}
        Velikost plošného integrálu \(\Psi = \int_{S_P}\vec{B}\cdot \dd{\vec{S}}_P\) přes 
        \textbf{ohraničenou orientovanou} plochu \(S_P\), je \textbf{nezávislá} na změně tvaru 
        plochy \(S_P\) při zachování konstantního tvaru hraniční křivky \(l\).        
      \end{lemma}
      Současně lze vyslovit další důležitou větu:
      \begin{lemma}\label{ES:lem_topol10}
        Plošný integrál \(\Psi = \int_{S_M}\vec{B}\cdot \dd{\vec{S}}_M\) přes \textbf{ohraničenou 
        neorientovanou} plochu \(S_M\), je vždy \textbf{roven nule} při libovolném tvaru plochy 
        \(S_M\) i hraniční křivky \(l\).        
      \end{lemma}
      
      Z matematického pohledu lze říci, že pojem \emph{„plošný integrál přes neorientovanou 
      plochu“} nemá smysl. Z fyzikálního pohledu se s tím ale spokojit nelze. S ohledem na platnost 
      věty \ref{ES:lem_topol08} totiž každý libovolně tvarovaný vodič reprezentuje hraniční křivku 
      \(l\), která je současně hranicí \emph{orientované} plochy \(S_P\) a \emph{neorientované} 
      plochy \(S_M\) viz obr. \ref{es:fig_patocka_topol16}. Pak je velmi důležitá otázka, jaká je 
      interakce \emph{neorientované} plochy s magnetickým polem \(B\), tj. jakou měrou přispívá 
      neorientovaná plocha k výslednému spřaženému toku \(\Psi\). Odpověď je podle věty 
      \ref{ES:lem_topol10} taková, že interakce neorientované plochy s magnetickým polem je 
      principiálně nulová. Proto ze tří vět \ref{ES:lem_topol05} až \ref{ES:lem_topol08} vlastně až 
      nyní přesně vyplývá význam původně uvedené rovnice (\ref{TEO:eq083}). Tuto rovnici 
      napíšeme znovu, ale v modifikované podobě:
      \begin{align}\label{ES:eq_topol18}
        \Psi &= \int_{S_P}\vec{B}\cdot \dd{\vec{S}}_P + \int_{S_M}\vec{B}\cdot \dd{\vec{S}}_M \nonumber \\
             &= \int_{S_P}\vec{B}\cdot \dd{\vec{S}}_P + 0 
              = \int_{S}\vec{B}\cdot \dd{\vec{S}}
      \end{align}
      Je zřejmé, že přeznačením orientované ohraničené plochy Sp na S získáme původní rovnici 
      (\ref{TEO:eq083}). Všimněme si, že v kapitole \ref{ES:sec02} jsme původně vůbec 
      netušili existenci jiné plochy než orientované.
    
    \subsection{Vzduchová cívka z topologického hlediska}\label{teo:IchapIIsecIIsubV}
      \subsubsection{Vzduchová cívka šroubovicového tvaru}        
        Stejná cívka jako v kapitole \ref{ES:sec02} je navinuta z drátu tvořícího hraniční křivku 
        \(l\) orientované plochy \(S_P\). Řekli jsme, že tvar plochy odpovídá šneku v mlýnku na 
        maso nebo točitému schodišti. Tuto představu je nutno upřesnit. Především vyjdeme z věty 
        \ref{ES:lem_topol09} o nezávislosti plošného integrálu na tvaru plochy - a to při zachování 
        tvaru hraniční křivky. Výsledkem věty je možnost tvarovat plochu \emph{nekonečně mnoha} 
        způsoby. Několik z nich ukážeme.

        \begin{figure}[ht!]
          \centering  
          \subcaptionbox{\label{es:fig_patocka_topol17a}}{\luafigure[0.4]{patocka_topol17a.png}}
          \subcaptionbox{\label{es:fig_patocka_topol17b}}{\luafigure[0.4]{patocka_topol17b.png}}
          \caption{Dva příklady možných tvarů orientované ohraničené plochy \(S_P\) u šroubovicové 
                   cívky.\cite[s.~57]{Patocka4}} 
          \label{es:fig_patocka_topol17}
        \end{figure}

        Na obr. \ref{es:fig_patocka_topol17a} se celková plocha \(S_P\) skládá z plochy \(S_{PS}\) 
        točitého schodiště (tři závity) a z plochy \(S_{PR}\) obdélníkového rámu, která je napnuta 
        mezi centrální sloup schodiště a přívodní vodiče vedoucí ke svorkám. Obdélníkovou plochu 
        tvořenou přívodními vodiči totiž nelze zanedbat. Skutečnost, že se plocha schodiště a 
        plocha rámu navzájem protínají, není proti topologickým pravidlům. 
        
        Na obr. \ref{es:fig_patocka_topol17b} je celková plocha \(S_P\) tvořena „vodorovnými 
        vlákny“ napnutými levým koncem na svislý přívodní vodič a pravým koncem na obvod 
        šroubovice. Z půdorysného pohledu bude mít plocha podobu vlnícího se vějíře. Integrál 
        \(\Psi = \int_{S_P}\vec{B}\cdot \dd{\vec{S}}_P\) bude mít v obou případech a), b) vždy neomylně 
        stejnou velikost. 

        \begin{figure}[ht!]
          \centering  
          \subcaptionbox{\label{es:fig_patocka_topol18a}}{\luafigure[0.35]{patocka_topol18a.png}}
          \subcaptionbox{\label{es:fig_patocka_topol18b}}{\luafigure[0.55]{patocka_topol18b.png}}
          \caption{Třetí příklad možného tvaru orientované ohraničené plochy \(S_P\) u šroubovicové 
                   cívky. \cite[s.~58]{Patocka4}} 
          \label{es:fig_patocka_topol18}
        \end{figure}

        Na obr. \ref{es:fig_patocka_topol18} je uvedena další možná realizace plochy cívky. Má-li 
        cívka \(N\) závitů, podle obrázku \ref{es:fig_patocka_topol18a} vystřihneme z papíru \(N\) 
        kruhů. Nechť má každý kruh horní stranu červenou a dolní zelenou. Všechny kruhy nastřihneme 
        od obvodu až do středu. Oba vzniklé střihové okraje označíme 1 a 2. Kruhy klademe postupně 
        stejnolehle na sebe a slepíme vždy střihový okraj 1 dolního kruhu s okrajem 2 horního 
        kruhu. Stejně postupujeme u všech výše položených kruhů. Výsledkem bude „šroubovicová“ 
        plocha podle obrázku \ref{es:fig_patocka_topol18b}, jejíž jednotlivé vrstvy mají „kuželový“ 
        tvar, přičemž plocha protíná sama sebe \(N\)-násobně v jediném bodě - ve vrcholu „kuželů“. 
        Mezi počáteční a koncovou střihovou hranu je „nalepena“ trojúhelníkovitá svislá plocha 
        navazující na přívodní vodiče. Celková plocha je orientovaná, horní strany všech vrstev 
        budou červené, dolní zelené.
      
      \subsubsection{Vzduchová cívka ve tvaru toroidu}
        Toroidní cívka je znázorněna na obr. \ref{es:fig_patocka_topol19a}. Z topologického 
        hlediska se jedná o stejnou situaci, jakou ukazuje obr. \ref{es:fig_patocka_topol17a}. 
        Rozdíl je pouze v tom, že šroubovicová plocha \(S_{PS}\) točitého schodiště je rozprostřena 
        po celém obvodu plochy \(S_{PR}\), tj. po obvodu původně obdélníkového rámu, přičemž 
        obdélník je deformován do tvaru kružnice.

        \begin{figure}[ht!]
          \centering  
          \subcaptionbox{\label{es:fig_patocka_topol19a}}{\luafigure[0.37]{patocka_topol19a.png}}
          \subcaptionbox{\label{es:fig_patocka_topol19b}}{\luafigure[0.37]{patocka_topol19b.png}}
          \caption{Vzduchová toroidní cívka jednovrstvá. \cite[s.~58]{Patocka4}} 
          \label{es:fig_patocka_topol19}
        \end{figure}
        
        Obr. \ref{es:fig_patocka_topol20b} vznikl z obr. \ref{es:fig_patocka_topol20a} tak, že byly 
        odstraněny šroubovicové závity vlastního toroidu (např. napnutím vodiče). Důsledkem je 
        \emph{odstranění} plochy \(S_{PS}\), ale zachováni plochy \(S_{PR}\). Je zřejmé, že v obou 
        případech a) i b) vzniká uvnitř ploch \(S_{PR}\) stejně velké magnetické pole \(B\) 
        (vybuzené jedním závitem), jehož směr je určen pravidlem PPR. Přičemž obr. 
        \ref{es:fig_patocka_topol20b} lze zobecnit pro libovolný počet \(n_v\) vrstev (ale vinutých 
        stále stejným směrem), ekvivalentní kruhová cívka b) má pak nv závitů. Obvyklé tvrzení 
        říkající, že \emph{vně toroidu neexistuje magnetické pole}, je tedy na první pohled mylné.
        \begin{figure}[ht!]
          \centering  
          \subcaptionbox{\label{es:fig_patocka_topol20a}}{\luafigure[0.30]{patocka_topol20a.png}}
          \subcaptionbox{\label{es:fig_patocka_topol20b}}{\luafigure[0.30]{patocka_topol20b.png}}
          \caption{ Vzduchová cívka toroidní jednovrstvá s provlečeným vodičem. 
                    \cite[s.~58]{Patocka4}} 
          \label{es:fig_patocka_topol20}
        \end{figure}
        Aby bylo tvrzení pravdivé, je nutno učinit některou z topologických operací znázorněných na 
        obr. \ref{es:fig_patocka_topol20} nebo obr. \ref{es:fig_patocka_topol21}. V obou případech 
        došlo nikoli k odstranění plochy \(S_{PR}\), nýbrž ke \emph{zmenšení obsahu} této plochy na 
        nulu. Přičemž obr. \ref{es:fig_patocka_topol20} lze zobecnit pro lichý počet vrstev (každá 
        následující vrstva musí být vinuta v opačném směru\footnote{V opačném směru, nikoli v 
        opačném smyslu. Shodný pravotočivý (popř. levotočivý) smysl závitů ve všech vrstvách musí 
        být samozřejmě zachován.} než předchozí), Obr. \ref{es:fig_patocka_topol21} lze zobecnit 
        pro sudý počet vrstev. Kvalita uvedených topologických jevů je naprosto \emph{nezávislá} 
        na přítomnosti či nepřítomnosti feromagnetického jádra uvnitř toroidu.
        \begin{figure}[ht!]
          \centering  
          \subcaptionbox{\label{es:fig_patocka_topol21a}}{\luafigure[0.45]{patocka_topol21a.png}}
          \subcaptionbox{\label{es:fig_patocka_topol21b}}{\luafigure[0.45]{patocka_topol21b.png}}
          \caption{ Vzduchová toroidní cívka dvojvrstvá. \cite[s.~59]{Patocka4}} 
          \label{es:fig_patocka_topol21}
        \end{figure}

      \subsubsection{Dva zkroucené přívodní vodiče}
        Zvláštním případem vzduchové cívky jsou dva zkroucené přívodní vodiče podle obr. 
        \ref{es:fig_patocka_topol22}. Používají se v případech, kdy jsou vodiče obklopeny rušivým 
        magnetickým polem. Oba zkroucené vodiče tvoří hraniční křivku \(l\) orientované plochy 
        \(S\). Nechť má plocha červenou a zelenou stranu. Pak celkový spřažený tok cívky  \(\Psi = 
        \int_S\vec{B}\cdot \dd{\vec{S}}\) bude nulový za předpokladu, že délka \(l\) stoupání zkrutu je 
        podstatně menší než \emph{prostorové nehomogenity} pole. Integrální příspěvky od pole \(B\) 
        vstupujícího do červených stran plochy nechť jsou brány jako kladné. Příspěvky od zelených 
        stran jsou pak přibližně stejně velké, ale záporné. Pak platí \(\Psi = 0\), tedy 
        \(\der{\Psi}{t}\) a do přívodních vodičů se neindukuje žádné rušivé napětí\footnote{Tato 
        eliminace rušení pracuje na zcela jiném principu než stínicí plášť koaxiálního kabelu podle 
        obr. \ref{teo:fig041}.}.
        \begin{figure}[ht!]
          \centering  
          \includegraphics[width=0.8\linewidth]{patocka_topol22.png}
          \caption{Dva zkroucené vodiče v magnetickém poli \(B\). \cite[s.~59]{Patocka4}} 
          \label{es:fig_patocka_topol22}
        \end{figure}
        
      \subsubsection{Realizace maximální indukčnosti vzduchové cívky}
        Uspořádání podle obr \ref{es:fig_patocka_topol23} není optimální. Magnetické pole 
        vyšrafovaných čtverců je nulové, protože proudy podél jejich obvodu nemají stejný smysl 
        oběhu.
        \begin{figure}[ht!]
          \centering  
          \includegraphics[width=0.4\linewidth]{patocka_topol23.png}
          \caption{Vzduchová cívka ve tvaru rohožky - neoptimální uspořádání.    
                   \cite[s.~59]{Patocka4}} 
          \label{es:fig_patocka_topol23}
        \end{figure}
        Uspořádání podle obr \ref{es:fig_patocka_topol24} je výhodnější, protože všechny čtverce 
        přispívají ke tvorbě magnetického toku stejným dílem (neuvažujeme anomálie na okrajích 
        rohožky). Má-li strana každého čtverce délku \(a\), pak magnetický tok \(\Phi\) nemůže 
        sahat do  větší vzdálenosti než \emph{přibližně} \(\pm a\) na obě strany od rohožky. Tok je 
        totiž sousedními čtverci vtahován zpět. Proto ve vzdálenosti přibližně \(\pm \frac{a}{2}\) 
        od roviny rohožky leží magnetické ekvipotenciální plochy 1 a 2 vzdálené navzájem asi \(a\), 
        mezi nimiž je nulové magnetické napětí \(U_{m_{12}} = 0\).
        \begin{figure}[ht!]
          \centering  
          \includegraphics[width=0.9\linewidth]{patocka_topol24.png}
          \caption{Vzduchová cívka ve tvaru rohožky - optimální uspořádání.
                   \cite[s.~60]{Patocka4}} 
          \label{es:fig_patocka_topol24}
        \end{figure}
        
        Každý čtverec tedy produkuje tok \(\Phi\) na ploše \(a\cdot a\), při délce siločáry asi 
        \(a\). Vzhledem k tomu, že se vzestupné a sestupné toky \(\Phi\) v oblasti 
        ekvipotenciálních ploch 1 a 2 navzájem \emph{odečítají a ruší}, každý čtverec vnímá tuto 
        situaci, jako by byl \emph{magneticky izolován} od čtverců sousedních. Proto každý čtverec 
        přispívá k celkové indukčnosti \(L\) rohožky hodnotou \(L_1\), o velikosti
        \begin{equation}\label{ES:eq_topol19}
          L_1 = N_1^2\mu_0\frac{S}{l} \cong 1^2\mu_0\frac{a^2}{a} = \mu_0a.
        \end{equation} 
        Předpokládáme, že rohožka je tvořena jedním vodičem, proto je v rovnici vzato \(N_1 = 1\). 
        Obsahuje-li rohožka celkem \(N\) čtverců, pak díky chybějící magnetické vazbě mezi 
        sousedními čtverci se dílčí indukčnosti \(L_1\), prostě sečtou a výsledná indukčnost \(L\) 
        rohožky bude \(N\)-krát větší, tedy
        \begin{equation}\label{ES:eq_topol20}
          L_1 \cong N\mu_0a.
        \end{equation} 

        Výsledná indukčnost je úměrná pouze první mocnině \(N\). To je velmi nevýhodné. Kdyby totiž 
        ležely všechny čtvercové závity na sobě, pak by měly vzájemnou vazbu a výsledná indukčnost 
        by byla úměrná \(N^2\). Proto se v praxi vzduchová cívka ve tvaru rohožky nikdy nepoužívá. 
        Jedinou její nepatrnou výhodou je velmi dobré chlazení vodičů okolním vzduchem.
        
        Realizace maximální indukčnosti vzduchové cívky představuje typickou optimalizační úlohu, 
        kterou lze definovat např. následujícím způsobem: Vezměme vodič \emph{konstantní} délky 
        \(l_{Cu}\), zanedbatelné tloušťky (průřez vodiče se limitně blíží nule) a dejme mu takový 
        tvar, aby měl \emph{největší možnou} indukčnost. Je zřejmé, že závit musí mít tvar 
        kružnice\footnote{Kruh má ze všech útvarů největší plochu při daném obvodu.}. 
        Vzniká ale otázka, kolik závitů bude cívka mít?
    
        \begin{figure}[ht!]
          \centering  
          \subcaptionbox{\label{es:fig_patocka_topol25a}}{\luafigure[0.40]{patocka_topol25a.png}}
          \subcaptionbox{\label{es:fig_patocka_topol25b}}{\luafigure[0.35]{patocka_topol25b.png}}
          \caption{ Vzduchová cívka ve tvaru kružnice. Cívka má a) jeden závit, b) dva závity. 
                    Celková délka vodiče je v obou případech stejná. \cite[s.~61]{Patocka4}} 
          \label{es:fig_patocka_topol25}
        \end{figure}
        
        Vyjděme ze situace, že cívka má \emph{jeden} závit podle obr. 
        \ref{es:fig_patocka_topol25a}. Indukčnost \(L_1\), takové cívky musí vyhovovat rovnici
        \begin{equation}\label{ES:eq_topol21}
          L_1 = N_1^2\mu_0\frac{S}{l_{\text{stř}}} = 1^2\mu_0\frac{S}{l_{\text{stř}}} 
              = \mu_0\frac{S}{l_{\text{stř}}}.
        \end{equation} 
        kde \(S\) je plocha kruhového závitu (majícího obvod o délce \(l_{Cu}\)) a
        \(l_{\text{stř}}\) je střední délka magnetické siločáry. Délku \(l_{\text{stř}}\) bychom
        snadno nalezli experimentálně, změřením indukčnosti \(L_1\) a následným výpočtem z rovnice
        (\ref{ES:eq_topol21}). Bude-li mít cívka \(N\) závitů (při konstantní délce vodiče
        \(l_{Cu}\) ), pak relativní proporce cívky zůstanou zachovány, ale všechny lineární rozměry
        se zmenší \(N\)-krát a plochy \(N^2\)-krát. Rovnice (\ref{ES:eq_topol21}) přejde do tvaru
        \begin{equation}\label{ES:eq_topol22}
          L_N = N^2\mu_0\frac{S/N^2}{l_{\text{stř}}/N} = N\mu_0\frac{S}{l_{\text{stř}}} 
              = N\cdot L_1.
        \end{equation} 
        
        Je tedy výhodné aby počet závitů cívky rostl do nekonečna. Indukčnost cívky by se lineárně 
        zvyšovala s počtem závitů \(N\), rozměry cívky by se limitně blížily nule. V praxi však 
        nemůže být tloušťka vodiče nulová. Proto od okamžiku, kdy rozměry cívky začnou být řádově 
        srovnatelné s tloušťkou vodiče, přestává rovnice (\ref{ES:eq_topol22}) platit a 
        optimalizační strategie začíná být mnohem složitější. Jedná se pak o problém tzv. 
        \emph{nejlevnější cívky}, který je přesně vyřešen až v kapitole 16.
      
      \subsubsection{Cívka s feromagnetickým jádrem z topologického hlediska}
        Ve smyslu celkového objemu, aniž se zajímáme o konkrétní počet závitů, lze pohlížet na 
        cívku s jádrem jako na topologický útvar odpovídající dvěma článkům řetězu podle obr. 
        \ref{es:fig_patocka_topol26}. Bez ohledu na geometrické detaily útvar je kvantitativně 
        jednoznačně popsán dvěma čísly: průřezem jádra \(S_j\) a plochou okna \(S_o\), jak bude 
        ukázáno v kap. 11.
        \begin{figure}[ht!]
          \centering  
          \subcaptionbox{\label{es:fig_patocka_topol26a}}{\luafigure[0.25]{patocka_topol26a.png}}
          \subcaptionbox{\label{es:fig_patocka_topol26b}}{\luafigure[0.25]{patocka_topol26b.png}}
          \caption{Z topologického pohledu odpovídá cívka s feromagnetickým jádrem dvěma článkům 
                  řetězu a) volně do sebe zapojeným, b) těsně do sebe zapojeným. 
                  \cite[s.~61]{Patocka4}} 
          \label{es:fig_patocka_topol26}
        \end{figure} 
        Obrázek a) představuje neoptimální řešení. Na obrázku b) je naznačena \emph{částečná} 
        optimalizace celkového objemu železa a mědi. Další optimalizační krok může spočívat v tom, 
        že závity mědi rovnoměrně rozprostřeme po celém obvodu toroidního jádra. Okno jádra zůstane 
        zaplněno stejným způsobem, ale střední délka závitů bude \emph{kratší}. U cívek je nutno 
        zajímat se rovněž o konkrétní počty závitů. Pak oba útvary na obr. 
        \ref{es:fig_patocka_topol27} jsou topologicky naprosto \emph{rovnocenné}. V obou případech 
        se jedná o cívku se dvěma závity. 
        \begin{figure}[ht!]
          \centering  
          \subcaptionbox{\label{es:fig_patocka_topol27a}}{\luafigure[0.30]{patocka_topol27a.png}}
          \subcaptionbox{\label{es:fig_patocka_topol27b}}{\luafigure[0.25]{patocka_topol27b.png}}
          \caption{Cívka s feromagnetickým jádrem, a) Měď se ovíjí kolem železa, b) Železo se ovíjí
            kolem mědi. \cite[s.~61]{Patocka4}} 
          \label{es:fig_patocka_topol27}
        \end{figure} 
        
        Přesněji řečeno, v obou případech protíná železo (tedy tok \(\Phi\)) orientovanou plochu 
        \(S_P\) dvakrát, proto má spřažený tok v obou případech stejnou velikost \(\Psi = 
        \int_{S_P}\vec{B}\cdot \dd{\vec{S}}_P \backsimeq 2\Phi \backsimeq 2BS_{Fe}\) Z topologického 
        hlediska je tedy lhostejné, je-li jeden závit \emph{železa} ovinut dvěma závity 
        \emph{mědi}, nebo jeden závit \emph{mědi} dvěma závity \emph{železa}. Případ b) na obr 
        \ref{es:fig_patocka_topol27} se samozřejmě v praxi nepoužívá. Jak již bylo řečeno v 
        kapitole \ref{ES:sec03}, jediným důvodem je malý dosažitelný poměr mezi měrnou magnetickou 
        vodivostí (permeabilitou) feromagnetik a vakua (řádově \num{1e3}) - a naopak, veliký poměr 
        mezi měrnou elektrickou vodivostí vodičů a izolantů (minimálně \num{1e15}). Stejným 
        způsobem je nutno umět se dívat i na topologii transformátoru. Oba útvary na 
        obr. \ref{es:fig_patocka_topol28} jsou opět topologicky rovnocenné.
        \begin{figure}[ht!]
          \centering  
          \subcaptionbox{\label{es:fig_patocka_topol28a}}{\luafigure[0.30]{patocka_topol28a.png}}
          \subcaptionbox{\label{es:fig_patocka_topol28b}}{\luafigure[0.25]{patocka_topol28b.png}}
          \caption{Transformátor s převodem \(\frac{N_2}{N_1} = \frac{2}{3}\). a) Měď se ovíjí 
                   kolem železa, b) Železo se  ovíjí kolem mědi. Směr primárního proudu odpovídá 
                   pravidlu PPR, směr sekundárního proudu pravidlu PLR.
                   \cite[s.~61]{Patocka4}} 
          \label{es:fig_patocka_topol28}
        \end{figure} 
        
      \subsubsection{Neumannův vzorec pro výpočet vzájemné indukčnosti}
        V kapitole \ref{teo:IchapIIsecI} jsme pomocí metod \emph{diskrétní} topologie ukázali 
        platnost \emph{principu reciprocity} v diskrétních obvodech, zvláště pak v přenosových 
        dvojbranech.
        
        Vzduchový transformátor v obecné podobě dvou \emph{libovolně tvarovaných vodičů}, podle obr.
        \ref{es:fig_patocka_topol29}, však není možno považovat za diskrétní systém. Především nelze
        v žádném případě prohlásit za diskrétní jeho magnetický obvod\footnote{Vlastnosti
        diskrétního magnetického obvodu jsou definovány v kapitole 5.2.1.}.
        \begin{figure}[ht!]
          \centering  
          \includegraphics[width=0.65\linewidth]{patocka_topol29.png}
          \caption{Transformátor v podobě dvou libovolně tvarovaných vodičů.           
            \cite[s.~63]{Patocka4}} 
          \label{es:fig_patocka_topol29}
        \end{figure}        
        Přesto však i u transformátoru na obr.\ref{es:fig_patocka_topol29} princip reciprocity 
        neomylné platí, což se projevuje tím, že vzájemná indukčnost \(M\) činitel vazby k 
        transformátoru jsou pro oba směry přenosu stejné:
        \begin{equation}\label{ES:eq_topol23}
          M_{12} = M_{21} = M, \qquad k_{12} = k_{21} = k,
        \end{equation} 
        přičemž parametry \(L_1\), \(L_2\), \(M\), \(k\) transformátoru jsou svázány známým vztahem
        \begin{equation}\label{ES:eq_topol24}
          M = k\sqrt{L_1L_2}
        \end{equation} 
        Rovnice (\ref{ES:eq_topol23}) musí být platná vždy, protože \(\mathbb{Z}\)-matice 
        transformátoru (psaná např. v operátorovém tvaru) musí být vždy symetrická podle hlavní 
        diagonály:
        \begin{align}\label{ES:eq_topol25}
          U_1 &= (R_1+pL_1)I_1 + pM_{12}I_2  \nonumber \\
          U_2 &= pM_{21}I_1 + (R_2+pL_2)I_2  
        \end{align} 
        V	literatuře bývá jako důkaz platnosti principu reciprocity u transformátoru, tj. jako 
        důkaz platnosti rovnice (\ref{ES:eq_topol23}), někdy uváděn \textbf{Neumannův vzorec}. Je 
        to dvojný křivkový integrál pro výpočet vzájemné indukčnosti mezi dvěma smyčkami (křivkami) 
        \(l_1\), \(l_2\):
        \begin{equation}\label{ES:eq_topol26}
          M = \mu\oint\limits_{l_1}\oint\limits_{l_2}\frac{\vec{dl_1}\cdot\vec{dl_2}}{r}.
        \end{equation}
        V	čitateli zlomku se jedná o skalární součin. Geometrický význam plyne z obr. 
        \ref{es:fig_patocka_topol30}. Důkaz principu reciprocity je pak velmi jednoduchý, přímo 
        plyne z faktu, že výsledek integrace nezávisí na pořadí výpočtu křivkových integrálů. Ovšem 
        pozor, vzorec je použitelný pouze ve \emph{zvláštním} případě, jsou-li obě cívky umístěny v 
        magneticky \emph{homogenním a izotropním} prostoru s konstantní permeabilitou \(\mu\). 
        Vzorec nelze použít, vyskytují-li se v okolí cívek dva materiály s rozdílnou permeabilitou, 
        např. vzduch a feromagnetikum. Neumannův vzorec je tedy potvrzením principu reciprocity 
        pouze ve \emph{zvláštním} případě, bohužel není obecným důkazem platnosti principu. Přesto 
        jsou úvodní rovnice (\ref{ES:eq_topol23}) bez výjimky platné i v magneticky nehomogenním a 
        \emph{neizotropním} prostředí (permeabilita \(\mu\) není v prostoru konstantní, je funkcí 
        prostorových souřadnic i směru). Pak je ovšem nutno hledat důkaz platnosti principu 
        reciprocity zcela jinak. Řešení je uvedeno v následující kapitole.
        \begin{figure}[ht!]
          \centering  
          \includegraphics[width=0.65\linewidth]{patocka_topol30.png}
          \caption{K výpočtu vzájemné indukčnosti podle Neumannova vzorce. \cite[s.~64]{Patocka4}} 
          \label{es:fig_patocka_topol30}
        \end{figure} 
     
      \subsubsection{Princip reciprocity ve spojitě rozprostřených obvodech}
        Důkaz platnosti principu reciprocity pro \emph{prostorově spojitě rozprostřené obvody},
        nepopsatelné diskrétní maticí, není v literatuře znám. I když se ví, že i zde princip
        reciprocity neomylně platí. Máme na mysli např. elektrický proud tekoucí ve vodivém
        \emph{spojitém nehomogenním} 2D nebo 3D prostoru, přenosovou soustavu tvořenou vysílací a
        přijímací anténou, ale především magnetické obvody s tokem \emph{spojitě} rozloženým v 3D
        prostoru.
        
        Důkaz platnosti principu reciprocity pro tyto případy je možno konstruovat podle obr.
        \ref{es:fig_patocka_topol31}. Pro jednoduchost je důkaz řešen ve 2D prostoru, ale závěry
        jsou velmi snadno rozšiřitelné na 3D prostor. Obrázek představuje rovinnou desku nastříkanou
        elektricky vodivým odporovým lakem. Nechť je deska nastříkána \emph{nerovnoměrně}, vodivosti
        na jednotku délky budou v různých oblastech různé, 2D prostor je tedy elektricky
        \emph{nehomogenní}.
        
        Postup důkazu:
        \begin{enumerate}[noitemsep]
          \item Na desce zvolme v libovolných místech čtyři body \(A\), \(B\), \(C\), \(D\).    
                Vznikne tak přenosový dvojbran s branami \(A-B\), \(C-D\).
          
          \item Podle obrázku \ref{es:fig_patocka_topol31a} napájejme bránu \(A-B\) zdrojem 
                konstantního stejnosměrného proudu \(I_1\). Kolem bodů \(A\), \(B\) vzniknou 
                ekvipotenciální křivky.
          
          \item Z nekonečného množství všech ekvipotenciál lze určitě nalézt takové dvě křivky 
               \(a\), \(b\), které přesně procházejí protilehlými body \(C\), \(D\).
          
          \item Voltmetr připojený k \emph{libovolným} dvěma bodům, z nichž jeden leží na 
                ekvipotenciále \(a\) a druhý na ekvipotenciále \(b\), musí naměřit stále stejné 
                výstupní napětí naprázdno \(U_{2,0}\). Přenosová impedance naprázdno ve směru zleva 
                doprava má tedy reálnou hodnotu \(R_{21,0} = \frac{U_{2,0}}{I_1}\)
          
          \item Nyní celé měření obraťme: Podle obrázku \ref{es:fig_patocka_topol31b} napájejme 
                bránu \(C-D\) zdrojem konstantního stejnosměrného proudu \(I_2\) Kolem bodů \(C\), 
                \(D\) vzniknou ekvipotenciální křivky.
          
          \item Z nekonečného množství všech ekvipotenciál lze určitě nalézt takové dvě křivky 
                \(c\), \(d\), které přesně procházejí protilehlými body \(A\), \(B\).
          
          \item Voltmetr připojený k \emph{libovolným} dvěma bodům, z nichž jeden leží na
                ekvipotenciále \(c\) a druhý na ekvipotenciále \(d\), musí naměřit stále stejné
                výstupní napětí naprázdno \(U_{1,0}\). Přenosová impedance naprázdno ve směru zprava
                doleva má tedy reálnou hodnotu \(R_{12,0} = \frac{U_{1,0}}{I_2}\)
          
          \item Z topologických zákonitostí plyne, že ekvipotenciály \(a\), \(c\) se musí protnout 
                nejméně ve dvou bodech (maximálně ve čtyřech). Na obrázku 
                \ref{es:fig_patocka_topol31c} je vyznačen jeden z nich jako \(E\).
          
          \item Podobně ekvipotenciály \(b\), \(d\) se musí protnout nejméně ve dvou bodech 
                (maximálně ve čtyřech). Na obrázku \ref{es:fig_patocka_topol31c} je vyznačen jeden 
                z nich jako \(F\).
          
          \item Z topologie ekvipotenciál plyne, že průsečíky \(E\), \(F\) musí existovat.
          
          \item V bodech \(E\), \(F\) pak musí platit \(R_{21,0} = R_{12,0}\) Přenosové impedance 
                naprázdno jsou tedy v obou směrech opravdu stejně velké.
        \end{enumerate}
        Konec důkazu.
        \begin{figure}[ht!]
          \centering  
          \subcaptionbox{\label{es:fig_patocka_topol31a}}{\luafigure[0.80]{patocka_topol31a.png}} \\
          \subcaptionbox{\label{es:fig_patocka_topol31b}}{\luafigure[0.80]{patocka_topol31b.png}} \\
          \subcaptionbox{\label{es:fig_patocka_topol31c}}{\luafigure[0.80]{patocka_topol31b.png}}
          \caption{Princip reciprocity u dvojbranu s branami \(A-B\), \(C-D\) ve spojitém vodivém 
                   prostředí. \cite[s.~65]{Patocka4}} 
          \label{es:fig_patocka_topol31}
        \end{figure} 
        
        Důkaz má ryze topologický charakter založený na jednoduchých principech:
        \begin{itemize}[noitemsep]
          \item Kolem každého elektrického bodu musí vzniknout neomezené množství uzavřených 
              ekvipotenciálních křivek (libovolného neznámého tvaru).
        
          \item Alespoň jedna z nich musí procházet jiným zvoleným bodem.
        
          \item Ekvipotenciální křivky od dvou různých bodů se musí protínat alespoň ve dvou bodech 
                (počet průsečíků musí být sudé číslo). Křivky se protínají obvykle ve dvou nebo 
                čtyřech bodech.
      \end{itemize}
      
        Ve 3D prostoru by byl důkaz naprosto stejný, pouze by se pracovalo s ekvipotenciálními
        plochami vejčitého tvaru místo ekvipotenciálních křivek. Důkaz byl z důvodu větší názornosti
        konstruován pro elektrické veličiny, pro magnetické veličiny by byl zcela analogický.
%~~~~~~~~~~~~~~~~~~~~~~~~~~~~~~~~~~~~~~~~~~~~~~~~~~~~~~~~~~~~~~~~~~~~~~~~~~~~~~~~~~~~~~~~~~~~~~~~~~