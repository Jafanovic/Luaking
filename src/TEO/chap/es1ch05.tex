% !TeX program = lualatex
% !TeX root = luaking.tex
% !TeX encoding = UTF-8
% !TeX spellcheck = cs_CZ
%=================== Kapitola: Bipolární tranzistory ==============================================
\setchaptertoc
\chapter{Bipolární tranzistory}\label{es:IchapV}
  \section{Všeobecné poznatky}
    Bipolámí tranzistory představují aktivní polovodičové prvky. Jsou tvořeny dvěma přechody PN,
    tzn. třemi vrstvami z rozdílně dotovaného polovodičového materiálu. Podle pořadí vrstev
    rozlišujeme dvě skupiny bipolárních tranzistorů, a to tranzistory \textsc{NPN} a tranzistory
    \textsc{PNP}. Obě uspořádání tranzistorů jsou znázorněna na obr. 2.1, kde je zobrazeno jak
    pořadí vrstev, tak příslušná schematická značka.

    \begin{figure}[ht!]
      \centering  
      \subcaptionbox{\label{teo:fig_fig053a}}{\luafigure[0.3]{teo_fig053a.pdf}}    \hspace{1em}
      \subcaptionbox{\label{teo:fig_fig053b}}{\luafigure[0.3]{teo_fig053b.pdf}}    \newline
      \subcaptionbox{\label{teo:fig_fig053a}}{\luafigure[0.3]{teo_fig053c.pdf}}    \hspace{1em}
      \subcaptionbox{\label{teo:fig_fig053b}}{\luafigure[0.3]{teo_fig053d.pdf}}                   
      \caption{Pořadí vrstev polovodiče a schematická značka bipolárních NPN a PNP 
      tranzistorů (\cite[s.~112]{Frohn2006})}
      \label{teo:fig053}
    \end{figure}
%---------------------------------------------------------------------------------------------------