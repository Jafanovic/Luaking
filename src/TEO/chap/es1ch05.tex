% !TeX program = lualatex
% !TeX root = luaking.tex
% !TeX encoding = UTF-8
% !TeX spellcheck = cs_CZ
%=================== Kapitola: Bipolární tranzistory ==============================================
\setchaptertoc
\chapter{Bipolární tranzistory}\label{es:IchapV}
  \section{Všeobecné poznatky}\label{es:IchapVsecI}
    Tranzistor je polovodičový prvek určeny na zesilovaní nebo generovaní elektrických signálu,
    jedná se tedy o \emph{aktivní} součástku.

    Podle činnosti rozdělujeme tranzistory: 
    \begin{itemize}[noitemsep]
      \item s injekcí, které využívají majoritní i minoritní nosiče (bipolární),
      \item řízené polem, které využívají pouze majoritní nosiče náboje (unipolární).
    \end{itemize}
     
    Bipolámí tranzistory představují aktivní polovodičové prvky. Jsou tvořeny dvěma přechody PN,
    tzn. třemi vrstvami z rozdílně dotovaného polovodičového materiálu. Podle pořadí vrstev
    rozlišujeme dvě skupiny bipolárních tranzistorů, a to tranzistory \textsc{NPN} a tranzistory
    \textsc{PNP}. Obě uspořádání tranzistorů jsou znázorněna na obr. \ref{teo:fig053}, kde je
    zobrazeno jak pořadí vrstev, tak příslušná schematická značka.

    \begin{figure}[ht!]
      \centering  
      \subcaptionbox{\textsc{PNP}\label{teo:fig053a}}{\luafigure[0.3]{teo_fig053a.pdf}} \hspace{1em}
      \subcaptionbox{\textsc{NPN}\label{teo:fig053b}}{\luafigure[0.3]{teo_fig053b.pdf}} \newline
      \subcaptionbox{\label{teo:fig053c}}{\luafigure[0.3]{teo_fig053c.pdf}}             \hspace{1em}
      \subcaptionbox{\label{teo:fig053d}}{\luafigure[0.3]{teo_fig053d.pdf}}                   
      \caption{Pořadí vrstev polovodiče a schematická značka bipolárních NPN a PNP 
      tranzistorů (\cite[s.~112]{Frohn2006})}
      \label{teo:fig053}
    \end{figure}
    
    Označení \uv{tranzistor} je uměle vytvořené slovo z anglických slov \textbf{tran}sfer = přenášet
    a re\textbf{zistor} = odpor. Označení \uv{bipolámí} (bi = dva) má poukázat na skutečnost, že je
    hlavní proud určen dvěma rozdílnými druhy nositelů náboje. Ve všeobecném povědomí se však pro
    bipolární tranzistor ustálil zjednodušený pojem tranzistor. Na rozdíl od bipolárních tranzistorů
    teče u unipolárních tranzistorů (unum = jeden) hlavní proud pouze jedinou oblastí a z toho
    vyplývá, že je určen pouze jediným typem nositele náboje podle toho, jakým způsobem byl dotován
    polovodičový materiál. Unipolární tranzistory často označujeme jako tranzistory řízené
    elektrickým polem. Jejich princip bude uveden v kapitole \ref{es:IchapVI}.
    
    Pro uživatele má velký význam znalost vlastností a chování jednotlivých typů tranzistorů. Proto
    výrobci ke každému typu tranzistoru vydávají katalogové listy, které obsahují charakteristické
    hodnoty, charakteristiky a mezní hodnoty (obdobně jako u diod).
    
    Protože tranzistor představuje dvojbran, jenž má dvě vstupní a dvě výstupní elektrody, je počet
    charakteristických hodnot a charakteristik podstatně větší než u polovodičových diod. Některé
    důležité parametry a charakteristiky budou blíže objasněny v následujících odstavcích.
    
    Bipolární tranzistory využívají jako výchozího materiálu nejčastěji křemíku, pro vysoké kmitočty
    se používají tranzistory ze sloučenin typu \(A^{III}B^{V}\).
    
    Tranzistory můžeme rozdělovat podle mnoha hledisek. Z hlediska konstrukce je zásadní rozdělení
    na tranzistory pro malé signály (nízkého výkonu) a výkonové tranzistory. Takovéto rozdělení
    podle konstrukce je uvedeno na obr. \ref{teo:fig073}.
    
    Tranzistory nízkého výkonu se převážně používají pro zesilování malých střídavých signálů.
    Tranzistory mají v tomto případě pevně nastavený klidový pracovní bod a přiváděné signálové
    napětí je malé, tzn. že nesmějí být příliš vybuzeny. Další oblastí použití tranzistorů nízkého
    výkonu jsou elektronické spínače, které jsou buzeny v celém možném rozsahu charakteristik.
    
    Výkonové tranzistory jsou dimenzovány na velké proudy a velká napětí. Mají proto relativně větší
    pouzdra, díky nimž je možné rychleji odvádět větší množství vznikajícího tepla z krystalu
    polovodiče. Výkonové tranzistory nalézají uplatnění v zesilovačích velkých signálů, tj.
    výkonových a koncových stupních nebo zastávají funkci elektronických spínačů.

    \begin{mdframed}[style=mdnote]
      \small
      Krátce po skončení války v roce 1945, Bellovy laboratoře vytvořily skupinu pro výzkum fyziky
      pevných látek, pod vedením \textsc{Shockleyho}. Jejich cílem bylo najít alternativu ke křehkým
      elektronkovým zesilovačům. Jejich první pokusy byly založeny na Shockleyově myšlence, že
      vnější elektrické pole na polovodiči ovlivní jeho vodivost. Jejich experimenty však záhadně
      selhávaly.

      Skupina začala studovat atomové struktury, které se na povrchu a uvnitř látek liší. Hledané
      výsledky se začaly dostavovat v okamžiku, když začali obklopovat body dotyku mezi polovodičem
      a přívodními vodiči elektrolytem. \textsc{Hilbert Moore} postavil okruh, který jim dovolil
      snadno měnit frekvenci vstupního signálu a navrhl, aby používali glykol boritan, viskózní
      chemikálii, která se nevypařovala. Nakonec získali důkaz schopnosti zesílení signálu, když
      fyzik \textsc{Gerald Pearson}, podle návrhu Shockleyho, uvedl napětí na kapičku glykol
      boritanu umístěnou přes \textsc{P-N} přechod. V prosinci 1947 Bardeen a Brattain – pracovali
      bez Shockleyho – uspěli ve stvoření hrotového tranzistoru, který zesiloval signál.

      Další měsíc začali patentoví zástupci Bellových laboratoří pracovat na patentových
      přihláškách. Brzo objevili, že Shockleyho vliv elektrického pole na polovodič byl předpovídán
      a patentován v roce 1930 \textsc{Juliem Lilienfeldem}, který si svůj \textsc{MESFET}
      patentoval v Kanadě již v 22. října 1925. Přesto byly podány celkem čtyři žádosti o patent. Na
      žádné z těchto přihlášek se však nevyskytovalo Shockleyovo jméno. To Shockleyho rozzlobilo,
      protože práce byla založena na jeho nápadu s účinkem elektrického pole. 
      
      Ve stejnou dobu tiše pokračoval ve vlastní práci na stavbě různých druhů tranzistoru
      založených na spojení místo bodového dotyku. Předpokládal, že tento typ tranzistoru bude více
      komerčně úspěšný. Shockley pracoval na \emph{teorii elektronů a děr v polovodičích}, která
      byla nakonec vydána jako 558 stránková monografie v roce 1950. V té Shockley vypracoval
      rozhodující myšlenky týkající se pohybu elektronů a děr a diferenciální rovnici, kterou se
      řídí tok elektronů v pevných krystalech.

      Shockleyho tato práce vedla k myšlence \emph{sendvičového tranzistoru} a ke vzniku klasického
      tranzistoru. Jeho objev byl oznámen 4. června 1951 a Shockley obdržel 25. září 1951 za jeho
      objev patent. Pro výrobu tohoto tranzistoru byla vyvinuta difúzní metoda a tento tranzistor
      brzy zastínil tranzistor s bodovými kontakty. Shockley pokračoval jako vedoucí skupiny v
      Bellových laboratořích ještě dva roky.

      {\centering
        \captionsetup{type=figure}
        \luafigure[1]{teo_fig071.png}
        \captionof{figure}{\textsc{John Bardeen}, \textsc{William Shockley} a \textsc{Walter Brattain} 
          v Bell Labs, 1948. (\cite[s.~6]{KolkaBiolek2011})
        \label{teo:fig071}}
      \par}

      V roce 1951 byl zvolen členem National Academy of Sciences (NAS). Za svůj objev tranzistoru
      získal mnoho cen. Bellovy laboratoře však představovaly všechny tři vynálezce (Shockleyho,
      Bardeena a Brattaina) jako tým. To vedlo k rozkolu a Shockley později blokoval práci Bardeena
      a Brattaina na klasickém tranzistoru.

      Shockley nakonec začal řídit svoji vlastní společnost, v níž se pokoušel vytvořit nové a
      technicky obtížné zařízení (původně nazvané čtyřvrstvá dioda a nyní známé jako tyristor).
      Projekt se ale rozvíjel velmi pomalu. 

      V roce 1956 Shockley získal, spolu s Bardeenem a Brattainem, Nobelovu cenu za fyziku. Ve své
      Nobelovské přednášce plně ocenil Brattaina a Bardeena jako vynálezce tranzistoru s bodovými
      kontakty.
    \end{mdframed}
    
    Pro lepší orientaci ve velkém množství vyráběných typů tranzistorů bylo zavedeno označovací
    schéma. Evropští výrobci používají hlavně značení \uv{Pro-Electron\footnote{Pro Electron nebo
    EECA je evropský typový a registrační systém pro aktivní komponenty (jako jsou polovodiče,
    displeje z tekutých krystalů, senzorová zařízení, elektronky a katodové trubice). Společnost Pro
    Electron byla založena v roce 1966 v belgickém Bruselu. V roce 1983 byla sloučena s Evropskou
    asociací výrobců elektronických součástek (EECA) a od té doby působí jako agentura EECA.}},
    které využívá kombinace písmen a číslic.

    \luagraphic[1]{teo_fig079.jpg}{Pohled na čip \textsc{MJ1000} v TO3 pouzdře. Čip obsahuje dva 
      tranzistory v Darlingtonovo zapojení. První, méně výkonný tranzistor se zesílením cca 100 
      budí druhý výkonový tranzistor se zesílením cca 10. Výsledný zesilovací činitel je dán 
      přibližně vynásobením zesilovacích činitelů obou tranzistorů (cca 1000).}{teo:fig079}

    Aby mohl tranzistor řádně pracovat, musí mít potřebné napětí nejen mezi kolektorem a emitorem,
    ale také dostatečně velké napětí mezi bází a emitorem. Velkou roli přitom hraje teplotní
    stabilizace klidového pracovního bodu. Je potřebná z toho důvodu, že při stoupající teplotě
    okolí narůstají též proudy, které tranzistorem protékají. Následkem zvyšování teploty se krystal
    ohřívá a roste tak jeho vodivost, čímž se příslušné proudy zvětšují. Tímto způsobem stoupá
    ztrátový výkon, tranzistor se proto více ohřívá atd. Na konci tohoto koloběhu je tepelná
    destrukce tranzistoru, který se tak stává nepoužitelným.

    \begin{figure*}
      \centering
      \luafigure[1]{teo_fig073.png}
      \caption{Druhy bipolárních tranzistorů podle konstrukce. (\cite[s.~115]{Frohn2006})}
      \label{teo:fig073}
    \end{figure*}

    Tranzistory mohou pracovat ve třech různých základních zapojeních - se společným emitorem
    (\textsc{SE}), se společným kolektorem (\textsc{SC}) a se společnou bází (\textsc{SB}). Zapojení
    je pojmenováno podle toho, která z elektrod je společná vstupu a výstupu zesilovacího stupně pro
    zesilování střídavých napětí, tj. která představuje společný pól signálového napětí na vstupu a
    na výstupu daného stupně. Každé z těchto základních zapojení má své přednosti a nedostatky.
    Nejčastěji se používá zapojení \textsc{SE}. Zapojení \textsc{SC} se používá jako měnič impedance
    (velká vstupní a malá výstupní impedance). Zapojení \textsc{SB} má svůj význam hlavně ve
    vysokofrekvenční technice, jinak představuje také měnič impedance, avšak s přesně opačným
    účinkem než předchozí zapojení (malá vstupní a velká výstupní impedance).

    Při praktickém využití musejí být tranzistory, a to ať se jedná o jakékoliv zapojení, doplněny
    dalšími součástkami. Zapojení těchto součástek (rezistorů a kondenzátorů) závisí na tom, jakou
    úlohu má zapojení s tranzistorem plnit. Musíme rozlišit, zda má tranzistor pracovat ve
    stejnosměrném zesilovači, v zesilovači střídavých napětí, ve výkonovém zesilovači nebo zda má
    mít funkci spínače.

    Stejnosměrné zesilovače se používají všude tam, kde je zapotřebí zesilovat stejnosměrná napětí a
    změny napětí od velmi malých až k vysokým frekvencím. Typickým představitelem je např. zesilovač
    Y osciloskopu. Aby stejnosměrné zesilovače mohly řádně plnit svou funkci, nesmějí obsahovat
    žádné součástky, které by ovlivňovaly jejich zesílení na různých frekvencích a neumožňovaly
    galvanickou vazbu (např. kondenzátory).

    Střídavé zesilovače mohou naproti tomu zesilovat střídavá napětí o frekvenci několika \si{\Hz}
    až do několika \si{\GHz} (podle typu tranzistoru a dimenzování zapojení), avšak nemusejí
    zesilovat stejnosměrná napětí. Důležitými prvky v těchto zesilovačích jsou kondenzátory.

    Výkonové zesilovače mají spotřebiči odevzdat pokud možno co největší signálový výkon. Spotřebič
    může být představován např. reproduktorem (nízkofrekvenční výkonový zesilovač), vysílací anténou
    (vysokofrekvenční výkonový zesilovač) nebo motorem (aplikace výkonového zesilovače v regulační
    technice). Na spínače jsou kladeny zcela jiné požadavky než na stejnosměrné nebo střídavé
    zesilovače. Spínače mají co nejrychleji přecházet ze stavu \uv{zapnuto} do stavu \uv{vypnuto} a
    naopak. V tomto případě hrají velkou roli parazitní kapacity tranzistorů.


  \section{Základní princip}\label{es:IchapVsecII}
    \subsection{Princip funkce NPN a PNP tranzistorů}\label{es:IchapVsecIIssecI}
      První tranzistory byly vyráběny legováním. Tato technologie byla převzata z výroby diod. Aby
      byly vyrobeny dva přechody, byly na obě strany dotovaného základního materiálu umístěny
      pilulky cizích prvků - donorů nebo akceptorů {obr. \ref{teo:fig072}. Při výrobním procesu pak
      cizí atomy z obou stran difundovaly do výchozího materiálu. Střední vrstva byla přitom velmi
      tenká a měla výrazně menší počet volných nositelů náboje než obě vnější vrstvy. Podle
      použitých výchozích materiálů a cizích prvků vznikl tranzistor NPN nebo PNP.

      \luagraphic[0.8]{teo_fig072.png}{Výroba tranzistoru NPN legováním.
        (\cite[s.~115]{Frohn2006})}{teo:fig072}
      
      Aby tranzistor fungoval, musí být mezi bázi a emitor připojen zdroj napětí tak, aby byl spodní
      přechod PN polarizován v propustném směru. Tranzistor NPN má proto bázi kladnější než emitor,
      tranzistor PNP má bázi zápornější než emitor. Napětí mezi bází a emitorem křemíkových
      tranzistorů má velikost \(U_{BE} \approx \SI{0.7}{\V}\), tj. stejné, jako je difuzní napětí
      křemíkových diod. Na obr. \ref{teo:fig074} a \ref{teo:fig075} zakreslené proudy vyznačují směr
      toku elektronů.

      \luagraphic[1]{teo_fig074.png}{Znázornění tranzistoru NPN. \(\longrightarrow\) udáva směr 
        toku elektronů). (\cite[s.~116]{Frohn2006})}{teo:fig074}
      
      Horní přechod PN pracuje v závěrném směru. Zdroje napětí jsou z tohoto důvodu zapojeny tak,
      aby byl kolektor u NPN tranzistoru kladnější než emitor, u PNP tranzistoru naopak zápornější
      než emitor.

      Vlivem připojených napětí bude spodní přechod PN zapojen v propustném směru a horní přechod PN
      v závěrném směru. Ve střední a horní oblasti se vytvoří závěrná vrstva. Ta se rozprostírá
      téměř po celé šířce střední oblasti, která je velmi tenká a obsahuje pouze malý počet nositelů
      náboje.        

      \luagraphic[1]{teo_fig075.png}{Znázornění tranzistoru PNP.  \(\longrightarrow\) udáva směr 
        toku elektronů). (\cite[s.~116]{Frohn2006})}{teo:fig075}
      
      Protože je spodní přechod PN zapojen v propustném směru, zaplavují nositelé náboje z emitoru
      závěrnou vrstvu ve střední oblasti. Tím se tato závěrná vrstva zmenší a její odpor klesne. Tím
      mohou nositelé náboje z emitoru projít zmenšenou závěrnou vrstvou do kolektorové oblasti a
      odtud odtéct k baterii (ke zdroji).

      Jelikož nositelé náboje pocházejí ze spodní oblasti, nazývá se tato oblast emitor. Střední
      oblast představuje výchozí bod pro oba přechody PN a nazývá se proto bází. Horní oblast
      shromažďuje všechny nositele náboje, které neodtekly bází a nese označení kolektor.

      \luagraphic[1]{teo_fig076.png}{Provozní napětí a proudy tranzistoru NPN. 
        (\cite[s.~117]{Frohn2006})}{teo:fig076}
      
      Proud \(l_B\), který odtéká vývodem báze, je podstatně menší než kolektorový proud \(l_C\),
      jenž protéká zmenšenou závěrnou vrstvou. Např. při napětí \(U_{BE} \approx \SI{0.7}{\V}\)
      protéká bází proud \(I_B = \SI{1}{\mA}\) a kolektorem proud \(I_C \approx \SI{100}{\mA}\).
      Jestliže nyní nepatrné zvětšíme \(U_{BE}\), vzroste proud báze např. na \(I_B = \SI{2}{\mA}\),
      čímž k závěrné vrstvě doputuje více nositelů náboje z emitoru. Tím se závěrná vrstva ještě
      více zmenší a kolektorový proud vzroste např. na \(I_C \approx \SI{200}{\mA}\). Naopak při
      zmenšení napětí \(U_{BE}\) a tím též proudu \(l_B\) se odpor závěrné vrstvy zvětší a
      kolektorový proud \(I_C\) klesne. Zjišťujeme, že proud báze \(l_B\) a proud kolektoru \(l_C\)
      tranzistoru se v širokém rozmezí mění proporcionálně. U tranzistoru je tak možné malým proudem
      báze \(l_B\), jenž představuje vstupní proud, řídit podstatně větší kolektorový proud \(l_C\),
      který je výstupním proudem. Tato souvislost se udává formou \emph{proudového zesílení
      nakrátko} \(\beta\).

      \begin{equation*}
        \beta = \dfrac{\Delta I_C}{\Delta I_B} \qquad \text{při (\(U_{CE} = 0\)))}
      \end{equation*}
        
      \luagraphic[1]{teo_fig077.png}{Provozní napětí a proudy tranzistoru PNP. 
        (\cite[s.~117]{Frohn2006})}{teo:fig077}
      
      Tranzistory NPN a PNP se navzájem principiálně odlišují pořadím vrstev. Proto se odlišují též
      polaritou napětí \(U_{BE}\) a \(U_{CE}\). Obr. \ref {teo:fig076} a \ref {teo:fig077}
      vysvětluje souvislosti, které již byly naznačeny na obr. \ref{teo:fig074} a \ref{teo:fig075},
      tentokrát jsou ale použity schematické značky obou typů tranzistorů. Údaj napětí \(U_BE\) a
      \(U_{CE}\) je tvořen tak, že poslední písmeno udává vztažnou elektrodu, zde tedy emitor E. Při
      určování směru toku proudu vycházíme z technické orientace, jež je obvyklá (proud teče obvodem
      směrem od kladného k zápornému pólu zdroje). Všechny proudy, které tečou do tranzistoru, jsou
      kladné, vytékající proudy mají záporné znaménko.

      Aby byl tranzistor schopen činnosti, musí být vždy přechod báze-emitor v propustném směru a
      přechod báze-kolektor v závěrném směru.

      Na obr. \ref{teo:fig078a} je mezi vývod kolektoru a baterii zařazen ještě kolektorový rezistor
      \(R_C\), který plní funkci pracovního odporu. Ten omezuje proudové zesílení tranzistoru a
      přeměňuje je v napěťové zesílení. Tranzistor a kolektorový rezistor v tomto případě tvoří
      napěťový dělič pro napájecí napětí \(U_{CC} = \SI{+10}{\V}\). Při \(I_C = \SI{5}{\mA}\) a
      \(R_C = \SI{1}{\kohm}\) vzniká na kolektorovém rezistoru úbytek napětí
      \begin{equation*}
        U_{RC} = \SI{5e-3}{\mA}\cdot\SI{1}{\kohm} = \SI{5}{\V}
      \end{equation*}

      \luagraphic[1]{teo_fig078a.jpg}{apěťové zesílení tranzistoru NPN. 
        (\cite[s.~117]{Frohn2006})}{teo:fig078a}

      Napětí kolektor-emitor tranzistoru, které představuje výstupní napětí, bude 
      \begin{equation*}
        U_{CE} = U_{CC} - U_{RC} = \SI{10}{\V} - \SI{5}{\V} = \SI{5}{\V}
      \end{equation*}
      Jestliže se napětí \(U_{BE}\) vlivem střídavého napětí z připojeného generátoru v daný okamžik
      zvětší např. na \(U_{BE} = \SI{0.71}{\V}\), vzroste proud báze, který způsobí zvětšení
      kolektorového proudu, který, dejme tomu, vzroste z \(I_C = \SI{5}{\mA}\) na \(I_C =
      \SI{6.5}{\mA}\). Tento zvětšený kolektorový proud vyvolá na kolektorovém rezistoru zvětšený
      úbytek napětí
      \begin{equation*}
        U_{RC} = I_C\cdot R_C = \SI{6.5}{\mA}\cdot\SI{1}{\kohm} = \SI{6.5}{\V}
      \end{equation*}
      Tím napětí \(U_{CE}\) klesne na hodnotu \(U_{CE} = \SI{3.5}{\V}\). Při záporné půlvlně napětí
      generátoru se proud báze \(I_B\) zmenší, čímž se zmenši i proud kolektoru \(l_C\) a tím i
      úbytek napětí na kolektorovém rezistoru \(U_{RC}\). Důsledkem tohoto děje je zvětšení
      \(l_{CE}\) tranzistoru. Příslušné časové průběhy na obr. \ref{teo:fig078}. Generátor,
      připojený na bázi tranzistoru, způsobí změnu proudu báze \(\Delta I_B = \SI{20}{\micro\A}\),
      která vyvolá změnu kolektorového proudu \(\Delta I_C = \SI{3}{\mA}\).

      \begin{figure}[ht!]  %\ref{teo:fig078} 
        \centering
        \subcaptionbox{\(U_{BE}(t)\)\label{teo:fig078b}}{\luafigure[0.9]{teo_fig078b.jpg}}  \newline                                                       
        \subcaptionbox{\(I_{B}(t)\) \label{teo:fig078c}}{\luafigure[0.9]{teo_fig078c.jpg}}  \newline                  
        \subcaptionbox{\(I_{C}(t)\) \label{teo:fig078d}}{\luafigure[0.9]{teo_fig078d.jpg}}  \newline                 
        \subcaptionbox{\(U_{RC}(t)\)\label{teo:fig078e}}{\luafigure[0.9]{teo_fig078e.jpg}}  \newline                  
        \subcaptionbox{\(U_{CE}(t)\)\label{teo:fig078f}}{\luafigure[0.9]{teo_fig078f.jpg}}  \newline                   
        \caption{Napěťové zesílení tranzistoru zapojeného podle obr. \ref{teo:fig078a} 
          (\cite[s.~118]{Frohn2006})}
        \label{teo:fig078}
      \end{figure}

      Odtud pro uvedený případ zjistíme \emph{proudové zesílení} tranzistoru
      \begin{equation*}
        \beta = \dfrac{\Delta I_C}{\Delta I_B} = \dfrac{\num{3e-3}}{\num{20e-6}} = 150
      \end{equation*}
      \emph{Napěťové zesílení} \(A_u\) zapojení s tranzistorem je definováno podobně jako proudové
      zesílení.
      \begin{equation*}
        A_u = \dfrac{\Delta U_{out}}{\Delta U_{in}} = \dfrac{\Delta U_{CE}}{\Delta U_{BE}}
      \end{equation*}
      Napěťové zesílení závisí kromě proudového zesílení také na velikosti kolektorového rezistoru
      \(R_C\). V příkladu způsobí vstupní střídavé napětí \(\Delta U_{BE} = \SI{-0.02}{\V}\) změnu
      výstupního střídavého napětí \(U_{CE} = \SI{-3}{\V}\). V tomto případě je tedy napěťové
      zesílení
      \begin{equation*}
        A_u = \dfrac{\Delta U_{CE}}{\Delta U_{BE}} = \dfrac{\num{-3}}{\num{0.02}} = -150
      \end{equation*}
      Pomocí kolektorového rezistoru \(R_{C}\) se změnilo proudové zesílení na napěťové zesílení
      tranzistoru. Znaménko \uv{-} naznačuje, že kladné půlvlně vstupního napětí odpovídá záporná
      půlvlna výstupního napětí.

  \section{Základní zapojení tranzistoru}\label{es:IchapVsecIII}
%---------------------------------------------------------------------------------------------------