% !TeX spellcheck = cs_CZ
\begin{mdframed}[style=mdexam]
\begin{example}
  Generujte signál s lineárně rostoucím kmitočtem "\texttt{chirp signál}", maximální kmitočet
  $f_{max} = 20 Hz$, amplituda $A = 1$, vzorkovaný kmitočtem $f_s = 64 Hz$.

  {\centering
  \captionsetup{type=figure}
   \subcaptionbox{Spojitá forma   \label{mai:fig046a}}{\luafigure[1]{ces_fig047a.pdf}}              \newline
   \subcaptionbox{Diskrétní forma \label{mai:fig046b}}{\luafigure[1]{ces_fig047b.pdf}}
   \captionof{figure}{Chirp signál: Signál s lineárně rostoucím kmitočtem s maximální
    frekvencí 20 Hz vzorkovaný 254 Hz. Grafická reprezentace číslicových signálů bývá buď ve spojité
    formě (a) nebo v diskrétní formě (b) 
  \label{ces:fig047}}
\par}
\end{example}

  %---------------------------------------------------------------
  \lstinputlisting[%
    style=luaMatlabStyle,
    caption={\texttt{gen\_chirp\_signal.m}. Generuje chirp signál}
    ]{../src/CES/matlab/gen_chirp_signal.m}
  %--------------------------------------------------------------- 
\end{mdframed}