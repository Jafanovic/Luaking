% !TeX spellcheck = cs_CZ
\begin{mdframed}[style=mdexam]
  \begin{example}\label{cpp:exam003}
    Výstupem z tohoto programu je řetězec, dvě hodnoty integer a hodnota v rozšířené pohyblivé
    řádové čárce: 
    %---------------------------------------------------------------
      \begin{lstlisting}[style=luaCPPStyle]
        #include <iostream> 
        using namespace std;
        
        int main()
        {
          int i, j; 
          double d;

          i = 10; 
          j = 20; 
          d = 99.101;
          
          cout << "Here are some values: "; 
          cout << i; 
          cout << ' '; 
          cout << j;
          cout << ' '; 
          cout << d;

          return 0;
        }
      \end{lstlisting}
    %---------------------------------------------------------------
    Výstup z programu bude vypadat takto: 
    \begin{mdframed}[style=mdmsdos]
      Here are some values: 10 20 99.101
    \end{mdframed}

    V jediném I/O výrazu je možné provést výstup více než jedné hodnoty. 
      %---------------------------------------------------------------
      \begin{lstlisting}[style=luaCPPStyle]
        cout << i << ' ' << j << ' ' << d;
      \end{lstlisting}
    %---------------------------------------------------------------
    Povšimněme si, že mezi položky musíme explicitně vložit mezery. Když mezera chybí, neoddělená
    data se při zobrazení na obrazovce spojí.
  \end{example}
\end{mdframed}