% !TeX spellcheck = cs_CZ
{\tikzset{external/prefix={tikz/CES/}}
 \tikzset{external/figure name/.add={ch01_}{}}
%---------------------------------------------------------------------------------------------------
% file cis_sig_sys.tex
%---------------------------------------------------------------------------------------------------
\definecolor{Gray}{gray}{0.9}
%===================== Kapitola: Číslicové systémy a signály========================================
\chapter{Číslicové systémy a signály}  
\minitoc

  \section{Co je číslicový systém}
    V číslicovém systému se pracuje se signály, které mají jen konečný počet diskrétních hodnot. 
    Tím se liší od systémů analogových, u kterých jsou signály spojité, tj. mohou ve vymezeném 
    rozsahu nabývat nekonečný počet hodnot. V číslicovém systému může i čas být veličinou 
    diskrétní, tj. signály se mohou měnit jen v určitých okamžicích. Takovéto číslicové systémy
    se pak nazývají \textbf{synchronní} - na rozdíl od systémů \textbf{asynchronních}, u kterých ke 
    změnám signálů může docházet kdykoliv. Synchronní systémy jsou podstatně častější, neboť 
    existence přesně stanovených okamžiků změn signálů zavádí "pořádek" do časování signálů v 
    systému a tím usnadňuje jeho konstrukci i výrobu v podobě integrovaných obvodů. Přesné
    časování je zajištěno hodinovými (taktovacími) impulsy, což je velmi významný signál systému \cite[s.~14]{Pinker2006}.
  
    Číslicové systémy se dělí na dvě skupiny
    \begin{itemize}\addtolength{\itemsep}{-0.5\baselineskip}
      \item \textbf{kombinační systémy},
      \item \textbf{sekvenční systémy}.
    \end{itemize}
    U kombinačních systémů jsou výstupní signály závislé pouze na momentálních vstupních signálech. 
    U sekvenčních systémů jsou výstupní signály závislé nejen na momentálních vstupních signálech, 
    ale i na vstupních signálech v minulosti. Systém tedy má \emph{vnitřní paměť}.
    
    \subsection{Dvojstavové signály}
      Jak již bylo řečeno, číslicové signály mají jen konečný počet diskrétních hodnot. V naprosté 
      většině jrou to právě jen dvě hodnoty. Dvouhodnotové nebo dvoustavové signály snižují nároky 
      na výrobní tolerance. Bylo tak možné zavést výrobní postupy, které umožňují hromadnou a 
      levnou výrobu součástek. 
      
      Předpokládejme, že číslicové součástky jsou napájeny kladným napětím $+U_{CC}$. Jedna hodnota 
      bude vyjádřena nižším napětím, druhá vyšším napětím. Dvě možné hodnoty signálu označíme jako 
      '0' a '1' (v souladu se značením v \hyperref[CES:basic_bool_alg]{Boolově algebře}).
     
  \section{Kombinační logické funkce}
    Základním pojmem při úvahách o kombinačních systémech představuje pojem kombinační logická
    funkce.  \emph{Kombinační logická funkce} je pravidlo přiřazující každé kombinaci hodnot
    \texttt{0} a \texttt{1} přiřazených vstupním proměnným z definičního oboru funkce jedinou
    hodnotu výstupní proměnné. Pro daný počet vstupních proměnných je těchto funkcí konečný počet.
    Kombinační logické funkce mohou být úplně nebo neúplně určené.  \emph{Úplně určená kombinační
    logická funkce} je taková funkce, jejíž definiční obor zahrnuje všechny kombinace vstupních
    proměnných. U \emph{neúplně určené kombinační logické funkce} její definiční obor nezahrnuje
    některé tyto kombinace. Kombinací se zde rozumí kombinace hodnot  \texttt{0} a \texttt{1}
    přiřazených jednotlivým vstupním proměnným. Úplně určeným funkcím se někdy říká úplné funkce,
    funkcím neúplné určeným pak neúplné funkce.
   
  \subsection{Realizace kombinačních logických funkcí} 
    Nejčastěji se v digitální technice setkáme s těmito způsoby realizace kombinační logické funkce:
      \begin{itemize}\addtolength{\itemsep}{-0.5\baselineskip}
        \item pomocí digitálních integrovaných obvodu typu \texttt{NAND}, \texttt{NOR} (popř.   
              \texttt{AND}, \texttt{OR}) a dalších obvodů realizujících základní kombinační logické 
              funkce - např. \texttt{AND-OR-INVERT}, \texttt{EX-OR} atd.,
        \item pomocí multiplexeru a demultiplexeru,
        \item pomocí speciálních kombinačních integrovaných obvodu (převodníky kódu, generátory    
              parity, sčítačky, násobičky a podobně - sem patří i použití multiplexeru a 
              demultiplexeru),
        \item pomocí pamětí (např. \texttt{PROM} a \texttt{EEPROM}),
        \item pomocí programovatelných logických obvodu (\texttt{PLD}). 
      \end{itemize}
      
      \subsubsection{Použití multiplexerů a demultiplexerů k realizaci kombinačních logických funkcí}
   
  \subsection{Základní pravidla Booleovy algebry}\label{CES:basic_bool_alg}
    Nejdůležitější postuláty:
    \begin{align}
       \shortintertext{Univerzální vazba:}
         x + 0 = x \quad x\cdot 0 = 0 &|  x + 1 = 1 \quad x\cdot 1 = 1                         \\
       \shortintertext{Doplňek}
         x + \overline{x} = 1         &|  x\cdot\overline{x} = 0                               \\
       \shortintertext{Idempotence}
         x +x = x                     &|  x\cdot x = x      
      \label{CES:postul_Idemp}
    \end{align}
  \subsection{Zjednodušování zápisu logické funkce}
     Logická funkce vyjádřená úplnou součtovou (disjunktivní) nebo součinovou (konjuktivní) formou 
     z pravdivostní tabulky není jediným možným zápisem logické funkce. Dá se většinou nalézt 
     jednodušší algebraický zápis, z něhož můžeme předpokládat, že povede na realizaci méně 
     složitého číslicového obvodu. Který ze zápisů logické funkce povede na minimální složitost 
     obvodu závisí nejen na použitých logických členech, ale též na dalších kritériích: zpoždění, 
     spotřeba obvodu, jeho spolehlivost, potlačení hazardních stavů, atd. První metodou je 
     \emph{algebraická minimalizace}, 
     
    \subsubsection{Karnaughova metoda minimalizace pomocí mapy}  
      Jednou z možností grafického zápisu logické funkce je mapa. Nejpoužívanější je 
      \textbf{Karnaughova mapa} (čti ''karnau''). Mapa je uspořádána do čtverce či obdélníka a to 
      tak, že \emph{sousední pole} se liší vždy jen v jedné proměnné.
      \begin{table}
        \centering
        \begin{tabular}{lr}
          \begin{tabular}[t]{r|cccc|c}
             \rowcolor{Gray}{\textbf{Index}}& {$x_4$} & {$x_3$} & {$x_2$} & {$x_1$} & {$f$} \\
             \hline
             0&0&0&0&0&1\\
             1&0&0&0&1&1\\
             2&0&0&1&0&1\\
             3&0&0&1&1&0\\
             4&0&1&0&0&0\\
             5&0&1&0&1&1\\
             6&0&1&1&0&1\\
             7&0&1&1&1&0\\
             8&1&0&0&0&0\\
             9&1&0&0&1&1\\
            10&1&0&1&0&1\\
            11&1&0&1&1&0\\
            12&1&1&0&0&0\\
            13&1&1&0&1&1\\
            14&1&1&1&0&1\\
            15&1&1&1&1&0\\
          \end{tabular}
        \end{tabular}
        \caption[Pravdivostní tabulka logické funkce]{Pravdivostní tabulka logické funkce čtyř 
        proměnných, kterou se pokusíme vyjádřit také pomocí Karnaughovy mapy}
        \label{CES:tab_true1}
      \end{table} 

      Mapu chápáme jako uspořádaný zápis pravdivostní tabulky vzniklou transformací řádku tabulky 
      na jedno pole mapy \cite[s.~25]{Podlesak1994}. Tedy, každé pole mapy jednoznačně odpovídá 
      určité kombinaci všech proměnných. Nalézá-li se pole pod pruhem vyznačeným u proměnné, bude 
      tato proměnná nenegovaná. Nalézá-li se mimo pruh, bude proměnná negovaná. Tak např.
      pole označené jako x v mapě pro tři proměnné bude odpovídat kombinaci 
      $x_1\overline{x_2}\overline{x_3}$. Jak je názorně vidět, z pravdivostní tabulky funkce lze 
      snadno sestavit její mapu a naopak. Řádkům pravdivostní tabulky, ve kterých je
      funkční hodnota 1, odpovídají pole mapy s vepsanou jedničkou; obdobně to platí i pro nuly. V 
      mapě lze znázornit i neurčené stavy prázdným políčkem (nebo pomlčkou) 
      \cite[s.~219]{Wakerly1999}.
       
      \begin{figure}[hb!]
          \centering
          \karnaughmap{2}{$f(x_1,x_2):$}{{$x_2$}{$x_1$}}{1010}{}
          \karnaughmap{3}{$f(x_1,x_2,x_3):$}{{$x_3$}{$x_2$}{$x_1$}}{1x      }{}  
          \karnaughmap{4}{$f(x_1,x_2,x_3,x_4):$}{{$x_4$}{$x_3$}{$x_2$}{$x_1$}}{1010011001100110}{}        
        \caption{Příklad Karnaughovy mapy pro dvě, tři a čtyři proměnné}\label{CES:karnaugh_234}
      \end{figure}     
      
      Přiřazení kombinací hodnot vstupních proměnných (součinů) jednotlivým polím mapy se označuje 
      jako \emph{kódování}. Řádky i sloupce Karnaughovy mapy jsou kódovány \textbf{Grayovým kódem}. 
      Základní vlastností Grayova kódu je to, že sousední slova konstantí délky se liší pouze v 
      jedné proměnné. Tuto vlastnost splňuje i první a poslední kódové slovo (kód je uzavřen sám
      do sebe) viz \ref{CES:BCD_Gray_c}. Právě tato vlastnost je využita při konstrukci Karnaughovy 
      mapy - souřadnice polí jsou uspořádány tak, že u sousedních polí se liší jen v jedné 
      proměnné. Tudíž geometricky sousedící pole jsou sousední i v algebraickém smyslu (liší se v 
      jediné proměnné).  
      
     \begin{table}[ht!] 
       \centering 
       \begin{tabular}{|c|c|c|}
         \hline
         \rowcolor{CornflowerBlue}{\textbf{Číslo}}  & \textbf{Binární kód} & \textbf{Grayův kód} \\
         \rowcolor{CornflowerBlue}{ }               &     {$x_1,x_2,x_3$}  & {$x_1,x_2,x_3$}     \\
          \hline\hline  
             \cellcolor[gray]{0.9}0                 & 000                  & 000               \\
          \hline       
             \cellcolor[gray]{0.9}1                 & 001                  & 001               \\
          \hline  
             \cellcolor[gray]{0.9}2                 & 010                  & 011               \\
          \hline  
             \cellcolor[gray]{0.9}3                 & 011                  & 010               \\
          \hline  
             \cellcolor[gray]{0.9}4                 & 100                  & 110               \\
          \hline         
             \cellcolor[gray]{0.9}5                 & 101                  & 111               \\
          \hline  
             \cellcolor[gray]{0.9}6                 & 110                  & 101               \\
          \hline  
             \cellcolor[gray]{0.9}7                 & 111                  & 100               \\
          \hline
       \end{tabular}
       \caption{Binární a Grayův kód pro tři proměnné}
       \label{CES:BCD_Gray_c}    
     \end{table}       
      
     Každé pole s hodnotou 1 odpovídá \textbf{mintermu} z pravdivostní tabulky. Sousední pole tedy 
     odpovídají mintermům lišícím se jen jednou proměnnou, a ty lze spojovat do 
     \textbf{implikantů}. Sousední jsou i pole na okrajích mapy, neboť i ta se liší
     jen v jedné proměnné (konce řádek, konce sloupců a rohy mapy). Spojováním výrazů sousedních 
     políček provádíme minimalizaci, která díky jasnému geometrickému postupu vyhýbá 
     problematickému hledání těchto součtů nebo součinů. 
     
     Spojování polí se vyznačí \textbf{smyčkou}. Pole po dvojicích sousední lze spojovat do větších 
     smyček, ty opět do větších atd. Každá smyčka tedy musí mít stranu dlouhou právě $2^k$ polí, 
     kde $k$  je celé kladné číslo. Smyčky zahrnují 2 pole, 4 pole, 8 polí, atd. Každá smyčka v 
     mapě odpovídá implikantu funkce. Princip minimalizace spočívá v pokrytí všech     
     jedniček\footnote{nul pro součinovou formu} (a libovolných neurčených stavů) soustavou smyček 
     pro součtovou formu, přičemž:
     \begin{itemize}\addtolength{\itemsep}{-0.5\baselineskip}
       \item smyčky musí být co možná největší,
       \item smyček musí být co nejmenší počet.       
     \end{itemize}     
     Tento princip je ilustrován na následující mapě funkce čtyř proměnných. Jako příklad vezmeme 
     funkci definovanou pravdivostní tabulkou \ref{CES:tab_true1}. Odpovídající Karnaughova mapa 
     pro čtyři proměnné je na obrázku \ref{CES:karnaugh_234}. Základní součtový tvar této funkce je 
     dán rovnici:     
     \begin{align}
        f(x_1, x_2, x_3, x_4) 
          &= \overline{x_1x_2x_3x_4} + \overline{x_1}x_2\overline{x_3x_4} +           \nonumber \\
          &+ x_1\overline{x_2}x_3\overline{x_4} + \overline{x_1}x_2x_3\overline{x_4}+ \nonumber \\ 
          &+ x_1\overline{x_2x_3}x_4 + \overline{x_1}x_2\overline{x_3}x_4 +           \nonumber \\ 
          &+ x_1\overline{x_2}x_3x_4 + \overline{x_1}x_2x_3x_4.                          
     \end{align}
     V mapě můžeme vytvořit celkem čtyři smyčky, kterými spojíme sousední políčka. Všimněme si, že 
     některé smyčky se částečně překrývají. To však nevadí, protože k logické funkci můžeme na 
     základě postulátu \ref{CES:postul_Idemp} (\emph{idempotence})
     přidat tentýž součin několikrát. 
     
      \begin{figure}[hb!] 
          \centering
          \renewcommand{\kvcontentsize}{\Large}
          \renewcommand{\kvindexsize}{\normalsize}
          \kvunitlength=15mm
          \karnaughmap{4}{$f(x_1,x_2,x_3,x_4):$}{{$x_4$}{$x_3$}{$x_2$}{$x_1$}}{1010011001100110}%
            {% The Karnaugh map has its pivot point at the lower left point and a unitlength.
              \thinlines
              \put(0,2){\oval(1.9,1.9)[r]}   
              \put(-0.2,1.05){\line(1,0){0.2}} % horizontal line
              \put(-0.2,2.95){\line(1,0){0.2}}
              \put(4,2){\oval(1.9,1.9)[l]}    
              \put(4.0,1.05){\line(1,0){0.2}}  % horizontal line
              \put(4.0,2.95){\line(1,0){0.2}}              
              \put(1.5,0.5){\oval(0.9,0.9)[l]}
              \put(2.5,0.5){\oval(0.9,0.9)[r]}
              \put(1.5,0.95){\line(1,0){1}}
              \put(1.5,0.05){\line(1,0){1}}
              \put(2.5,3.5){\oval(0.9,0.9)[b]}
              \put(2.05,3.5){\line(0,1){0.7}} 
              \put(2.95,3.5){\line(0,1){0.7}}               
              \put(2.5,0.5){\oval(0.9,0.9)[t]}
              \put(2.05,-0.2){\line(0,1){0.7}}
              \put(2.95,-0.2){\line(0,1){0.7}}               
              \put(0.5,2.5){\oval(0.9,0.9)[b]}
              \put(0.5,3.5){\oval(0.9,0.9)[t]}
              \put(0.05,2.5){\line(0,1){1}}
              \put(0.95,2.5){\line(0,1){1}}  
              \put(0.2,3.5){\line(-1,1){0.5}}\put(-0.5,4){$1$}  
              \put(3.7,2.6){\line(2,1){0.7}}\put(4.5,2.9){$2$}  
              \put(1.8,0.8){\line(2,1){2.6}}\put(4.5,2.1){$3$}
              \put(2.8,3.6){\line(2,1){1.6}}\put(4.5,4.4){$4$}                     
             }  
        \caption{Minimalizace pomocí Karnaughovy mapy. Zakreslené smyčky byly vytvořeny tak, aby 
                 každá zahrnovala co největší počet polí s vepsanou 
                 jedničkou.}\label{CES:karnaugh_minim1}
      \end{figure}     
     Smyčka ze dvou políček označená na mapě \ref{CES:karnaugh_minim1} jako č. 1, může být vyjádřena:  
     \begin{equation}
       \overline{x_1x_2x_3x_4} + \overline{x_1}x_2\overline{x_3x_4} = \overline{x_1x_3x_4}(\overline{x_2} + x_2) =
       \overline{x_1x_3x_4}
     \end{equation}
     Smyčka ze čtyř polí na pravé a levé straně (označená č. 2):
     \begin{align}
       \overline{x_1}x_2\overline{x_3x_4} + 
       \overline{x_1}x_2\overline{x_3}x_4 + 
       \overline{x_1}x_2x_3\overline{x_4} +
       \overline{x_1}x_2x_3x_4                                &=               \\ \nonumber
       \overline{x_1}x_2\overline{x_3}(\overline{x_4}+x_4) +  
       \overline{x_1}x_2x_3(\overline{x_4}+x_4)               &=               \\ \nonumber
       \overline{x_1}x_2(\overline{x_3}+x_3)                  &= 
       \overline{x_1}x_2 
     \end{align}
     Smyčka ze dvou polí v poslední řádce (označená č. 3):
     \begin{equation}
       x_1\overline{x_2x_3}x_4 + x_1\overline{x_2}x_3x_4 = 
     \end{equation}     
     \begin{figure}[ht!]
         \centering
         \karnaughmap{5}{$f(x_1,x_2,x_3,x_4,x_5):$}{{$x_5$}{$x_4$}{$x_3$}{$x_2$}{$x_1$}}%
            {01001001111000110101110011000000}{}  
        \caption{Příklad Karnaughovy mapy pro pět proměnných}\label{CES:karnaugh_5}
      \end{figure}

     \begin{figure}[hb!]
         \centering                 
         \karnaughmap{6}{$f(x_1,x_2,x_3,x_4,x_5,x_6):$}{{$x_6$}{$x_5$}{$x_4$}{$x_3$}{$x_2$}{$x_1$}}%
          {{0}{1}{1}{0}{0}{1}{0}{0}%
           {0}{0}{1}{0}{0}{0}{0}{0}%
           {1}{1}{1}{1}{0}{1}{0}{0}%
           {0}{1}{0}{0}{1}{0}{0}{1}%
           {0}{0}{0}{1}{1}{1}{0}{0}%
           {1}{0}{0}{1}{0}{0}{1}{1}%
           {1}{1}{1}{0}{0}{0}{1}{1}%
           {0}{0}{0}{0}{1}{1}{1}{0}}{}%                 
        \caption{Příklad Karnaughovy mapy pro šest proměnných}\label{CES:karnaugh_6}
      \end{figure}

} % tikzset
%---------------------------------------------------------------------------------------------------
\printbibliography[title={Seznam literatury}, heading=subbibliography]
\addcontentsline{toc}{section}{Seznam literatury}