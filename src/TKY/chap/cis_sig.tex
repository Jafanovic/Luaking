% !TeX spellcheck = cs_CZ
%{\tikzset{external/prefix={tikz/TKY/}}
% \tikzset{external/figure name/.add={ch02_}{}}
%============ Kapitola: Číslicové signály - posloupnosti ===========================================
\setchaptertoc
\chapter{Číslicové signály - posloupnosti}
  Číslicové signály (matematicky posloupnosti čísel) \cite{Sovka2002} jsou v literatuře
  oz\-na\-čo\-vá\-ny symboly $x_n, x(n)$, nebo $x[nT]$, kde $n$ je celé číslo a označuje pořadí
  prvku v posloupnosti\footnote{Takto zavedené označení je nejednoznačné, neboť nerozlišuje mezi
  celou posloupností a jejím jediným prvkem. Posloupnost by měla být správně označena např. symbolem
  $\{x[n]\}$, zatímco symbol $x[n]$ by měl být vyhrazen pro její jeden prvek. Nicméně uvedené
  značení je všeobecně používáno.} Poslední uvedený symbol $x[nT]$ zdůrazňuje souvislost číslicového
  signálu se signálem spojitým v čase(analogovým signálem), ze kterého vznikl vzorkováním a
  kvantováním. Symbol $T$ označuje použitý \emph{vzorkovací krok}. Jeho převrácená hodnota je rovna
  \emph{vzorkovací frekvenci} $f_s=\frac{1}{T}$.

  \section{Základní typy posloupností}
    \begin{itemize}
      \item \textbf{Jednotkový impuls}
            \begin{equation}\label{SAS:eq_jednotkovy_imp}
              \delta[n]=
              \begin{cases} 
                 1, &  n = 0, \\
                 0, &  n \neq 0,
              \end{cases}
            \end{equation}
      \item \textbf{Jednotkový skok}
            \begin{equation}\label{SAS:eq_jednotkovy_skok}
              u[n]=
              \begin{cases} 
                 1, &  n \geq 0, \\
                 0, &  n < 0,
              \end{cases}
            \end{equation}
      \item \textbf{Reálná exponenciální posloupnost}
            \begin{equation}\label{SAS:eq_exp}
              x[n] = A\alpha^n, n\geq0,
            \end{equation}
      \item \textbf{Chirp signál}
            \begin{equation}\label{SAS:eq_chirp}
              x[n] = sin\left(\frac{\pi f_{max}n^2}{(N-1)f_s}\right),
            \end{equation}
            kde $f_{max}$ je maximální požadovaný kmitočet, který musí být menší než polovina
            vzorkovacího kmitočtu $f_{max}<\frac{f_s}{2}$ a $N$ je celkový počet vzorků.
      \item \textbf{Pseudonáhodná posloupnost} je posloupnost, která nahrazuje ideální bílý šum.
            Tuto posloupnost lze generovat různými algoritmy, které zaručují velmi dlouhou 
            periodicitu generované posloupnosti. Má-li tato posloupnost aproximovat bílý šum, musí 
            co nejlépe splňovat požadavek nekorelovanosti sousedních vzorků (tedy konstantní 
            spektrální výkonové hustoty) a nulové střední hodnoty. Často je požadován i jednotkový 
            rozptyl.
            \begin{figure}[ht!]
             \centering
             \includegraphics[width=0.8\linewidth]{randn_posloupnost.pdf}
             \caption[Příklad pseudonáhodné posloupnosti]{Příklad pseudonáhodné posloupnosti
                      generované pomocí funkce \texttt{randn(1, 300)} v MATLABu}
             \label{SAS:fig_randn}
         \end{figure}
    \end{itemize}
  
  \section{Generování jednoduchých signálů a jejich zobrazení v MATLABu}
    %---------------------------------------------------------------
    % !TeX spellcheck = cs_CZ
\begin{example}
  Generujte signál s lineárně rostoucím kmitočtem "\texttt{chirp signál}", maximální kmitočet
  $f_{max} = 20 Hz$, amplituda $A = 1$, vzorkovaný kmitočtem $f_s = 64 Hz$.

    {\centering
     \begin{tabular}{c}
         \includegraphics[width=0.8\linewidth]{Chirp_signal_plot.pdf}  \\
         \includegraphics[width=0.8\linewidth]{Chirp_signal_stem.pdf} 
     \end{tabular}  
     \captionof{figure}{Chirp signál: Signál s lineárně rostoucím kmitočtem s maximální
              frekvencí 20 Hz vzorkovaný 254 Hz. Grafická reprezentace číslicových signálů bývá
              buď ve spojité formě (a) nebo v diskrétní formě (b) 
     \label{SAS:fig_chirp_sig}}
  \par}
  
  M-file:
  %---------------------------------------------------------------
  \lstinputlisting{../src/TKY/matlab/gen_chirp_signal.m}
  \begin{lstlisting}[caption=\texttt{gen\_chirp\_signal.m}. Generuje chirp signál]
  \end{lstlisting}
  %---------------------------------------------------------------
\end{example}
    %---------------------------------------------------------------

  \section{Základní operace s posloupnosti}
    V dalším textu budeme používat tři základní lineární operace zobrazené na \ref{tky:fig_007}:
    \begin{itemize}
      \item \texttt{součin} signálu $x[n]$ a reálné konstanty $b$:
            $$w[n]=bx[n], n = 0,1,2, \ldots$$ Tato operace je v praxi realizována násobičkou a je
            zdrojem numerických chyb, tedy kvantizačního šumu, který produkují číslicová zařízení.
      \item \texttt{součet} signálu $x[n]$ a signálu $y[n]$:
            $$v[n]=x[n]+y[n], n = 0,1,2, \ldots$$ Tuto operaci provádí sčítačka. Při neošetření může
            tato operace generovat hrubé chyby.
      \item \texttt{zpoždění} signálu $x[n]$ o $k$ vzorkovacích kroků:  
            $$y[n]=x[n-k], n = 0,1,2, \ldots, n = 1,2, \ldots, M $$  Hodnoty $x[-k], k = 1, 2,
            \ldots, M$ se nazývají \emph{počáteční podmínky}. V digitálních implementací provádíme
            operaci zpoždění paměťového registru pro každou jednotku požadovaného zpoždění $z^{-1}$.
    \end{itemize}

    \begin{figure}[ht!]
      \centering
      \subcaptionbox{\label{tky:fig_007a}}{\luafigure[0.45]{tky_fig007a.pdf}}  
      \subcaptionbox{\label{tky:fig_007b}}{\luafigure[0.45]{tky_fig007b.pdf}}  \newline
      \subcaptionbox{\label{tky:fig_007c}}{\luafigure[0.45]{tky_fig007c.pdf}}
      \caption[Základní operace]{Symboly základních operací \cite[s.~7]{Sovka2002}} 
      \label{tky:fig_007}
    \end{figure}

%} %tikzset
%~~~~~~~~~~~~~~~~~~~~~~~~~~~~~~~~~~~~~~~~~~~~~~~~~~~~~~~~~~~~~~~~~~~~~~~~~~~~~~~~~~~~~~~~~~~~~~~~~~