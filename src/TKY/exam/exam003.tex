% !TeX spellcheck = cs_CZ Lineární obvody a systémy - Jan Bičák  - strana 10 Popis spojitých systémů
%===================================================================================================
\begin{tkyexam}{Diferenciální rovnice pro napětí \(u_2(t)\) analogového obvodu na obr.
  \ref{tky:fig008} z příkladu \ref{tky:exam002} má při nulových počátečních podmínkách tvar
  \begin{equation*}
    LC\frac{d^2u_2(t)}{dt^2}+RC\frac{du_2(t)}{dt}+u_2(t)=u_1(t)
  \end{equation*}
  Proveďme transformaci této diferenciální rovnice na diferenční. Vzorkovací periodu volme \(T =
  \qty{2}{\us}\).}{exam003}

  \noindent\textbf{Řešení:}
  V diferenciální rovnici provedeme náhradu první a druhé derivace odpovídajícími difrencemi s
  využitím výrazů \ref{tky:eq048} až \ref{tky:eq051}. Spojité funkce \(u_2(t)\) a \(u_1(t)\)
  nahradíme jejich funkčními hodnotami podle vzoru
  \begin{equation*}
    \left.u(t)\right\rvert_{t=nT}  \rightarrow u[n]
  \end{equation*}
  a dostaneme následující diferenční rovnici
  \begin{align*}
      LC\dfrac{u_2[n] - 2u_2[n-1]    + u_2[n-2]}{T^2} &+      \\
    + RC\dfrac{u_2[n] - u_2[n-1]}{T}                  &+ u_2[n]=u_1[n]. 
  \end{align*}
  Odtud jednoduchou úpravou dospějeme k diferenční rovnici
  \begin{equation*}
    a_0u_2[n] + a_1u_2[n-1] + a_2u_2[n-2]=u_1[n].
  \end{equation*}
  kde
  \begin{equation*}
    a_0 = \frac{LC}{T^2} + \frac{RC}{T} + 1, \, a_1 = -2\frac{LC}{T^2} - \frac{RC}{T}, \,  
    a_2 = \frac{LC}{T^2}
  \end{equation*}
  Nakonec dosadíme numerické hodnoty a dostaneme
  \begin{align}
    \num{75.579367}u_2[n] &- \num{137.90481}u_2[n-1]                       \nonumber \\
                          &+ \num{63.325442}u_2[n-2] = u_1[n].             \label{tky:eq061}
  \end{align}
\end{tkyexam}