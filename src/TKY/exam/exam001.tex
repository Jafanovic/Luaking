% !TeX spellcheck = cs_CZ 
%===================================================================================================
  \begin{tkyexam}{Uvažujme LTI systém, popsaný svou impulzní odezvou: \(h[n] = \{\num{0.45},\:
    \num{1.72},\:\num{0.62}\}\). Na jeho vstupu působí posloupnost \(x[n] = \{2,\:\num{0.95}\}\).
    Úkolem  je vypočítat vzorky výstupní posloupnosti \(y[n]\).}{exam001}

    \noindent\textbf{Řešení:} Podle (\ref{tky:eq025}) bude délka výstupní posloupnosti \(N_y = 2 + 3
    - 1= 4\) s pořadovými indexy 0 až 3. Pro její výpočet rozepíšeme vztah (\ref{tky:eq024}):
    \begin{align*}
      y[0] &= \sum_{k=0}^0 x[0]h[0] = \num{2}\cdot\num{0.45} = \num{0.9}                   \\
      y[1] &= \sum_{k=0}^1 x[k]h[1-k] = x[0]h[1] + x[1]h[0] =                              \\
           &=  \num{2}\cdot\num{1.72} + \num{0.95}\cdot\num{0.45} = \num{3.8675}           \\
      y[2] &= \sum_{k=0}^2 x[k]h[2-k] =                                                    \\
           &= x[0]h[2] + x[1]h[1] + x[2]h[0] =                                             \\
           &= \num{2}\cdot\num{0.62} + \num{0.95}\cdot\num{1.72} + \num{0}\cdot\num{0.45}  
            = \num{2.8740}                                                                 \\
      y[3] &= \sum_{k=0}^3 x[k]h[3-k] =                                                    \\
           &= x[0]h[3] + x[1]h[2] + x[2]h[1] +  x[3]h[0] =                                 \\
           &= \num{2}\cdot\num{0} + \num{0.95}\cdot\num{0.62} + \num{0}\cdot\num{1.72}
            + \num{0}\cdot\num{0.45} =                                                     \\
           &= \num{0.589} 
    \end{align*} 
    Pro výstupní posloupnost tedy platí:
    \begin{equation*}
      y[n] = \{\num{0.9},\:\num{3.8675},\:\num{2.8740},\:\num{0.589}\}
    \end{equation*}
    Situaci názorně shrnuje obr. \ref{tky:fig011}

    {\centering
      \captionsetup{type=figure}
      \luafigure[1]{tky_fig011.pdf}
      \captionof{figure}{Odezva lineárního číslicového systému na vstupní posloupnost. 
                \cite{Zaplatilek2013}}
      \label{tky:fig011}
    \par}

    V systému \textsc{MATLAB} je vestavěna vnitřní funkce \lstinline[style=luaMatlabText]!conv! pro
    snadný výpočet lineární diskrétní konvoluce, jak dokumentuje výpis \ref{tky:lst001}. Vyzkoušejme
    si, že obrátíme-li pořadí proměnných v příkazu \lstinline[style=luaMatlabText]!conv!, vyjde
    shodný výsledek (komutace symbolů).

  \begin{lstlisting}[style=luaMatlabText,gobble=4, label={tky:lst001}]
    h = [0.45, 1.72, 0.62];
    x = [2,  0.95];
    y = conv(x,h)
  \end{lstlisting}

\end{tkyexam}