% !TeX spellcheck = cs_CZ Lineární obvody a systémy - Jan Bičák  - strana 10 Popis spojitých systémů
%===================================================================================================
\begin{tkyexam}{Lineární spojitý systém je dán zapojením dle obrázku}{exam002}
  Určete
  \begin{enumerate}[leftmargin=12pt,noitemsep]
    \item diferenciální rovnici pro odezvu $u_2(t)$, je-li na vstupu buzen napětím $u_1(t)$,
    \item přenos napětí $H(p)=\dfrac{U_2(p)}{U_1(p)}$,
    \item impulsní odezvu $h(t)$.
  \end{enumerate}
  
  {\centering
   \captionsetup{type=figure}
   \luafigure[1]{tky_fig008.pdf}
   \captionof{figure}{Zapojení obvodu RLC.}
   \label{tky:fig008}
  \par}

  \noindent\textbf{Řešení:} Pro zapojení dle obrázku \ref{tky:fig008} získáme metodou uzlových
  napětí integrodiferenciální rovnice pro uzly \texttt{A} a \texttt{B}:
  \begin{gather*}
    \begin{align*}
      \shortintertext{uzel A:}
      \frac{u_3(t)-u_1(t)}{R}+\frac{1}{L}\int_0^t{[u_3(t)-u_2(t)]}\dd{\tau}+i_L(0_+) &= 0  \\
      \shortintertext{uzel B:}
      \frac{1}{L}\int_0^t[(u_2(t)-u_3(t))]\dd{\tau}+C\der{u_2}{t}-i_L(0_+)           &= 0
    \end{align*}
  \end{gather*}
  ve kterých \(i_L(0_+)\) je počáteční podmínka pro proud induktoru. Derivováním a eliminací
  $u_3(t)$ z původních rovnic dostaneme pro odezvu $u_2(t)$ diferenciální rovnici II. řádu. Začneme
  derivováním rovnice v uzlu \texttt{B} tj. \(\frac{d}{dt}(B)\):
  \begin{align*}
    u_2(t)-u_3(t)+LC\frac{d^2u_2(t)}{dt^2} &=0 \Rightarrow   \\
    u_2(t)+LC\frac{d^2u_2(t)}{dt^2}        &=u_3(t)
  \end{align*}
  Nyní můžeme z rovnice pro uzel \texttt{A} odstranit napětí \(u_3(t)\):
  \begin{align*}
    \frac{u_2(t)+LC\dfrac{d^2u_2(t)}{dt^2}-u_1(t)}{R}             &+    \\
    \frac{1}{L}\int_0^t{(LC\frac{d^2u_2(t)}{dt^2})}\dd{\tau}+i_L(0_+) &=  0 \\
    \shortintertext{}
    u_2(t)+LC\frac{d^2u_2(t)}{dt^2}-u_1(t)                        &+    \\
    RC\left[\frac{du_2(t)}{dt}\right]_0^t+Ri_L(0_+)               &=  0
  \end{align*}
  Při nulových počátečních podmínkách: $\left.\frac{du_2(t)}{dt}\right\rvert_{t=0}=0$,
  $i_L(0_+)=0$ dostaneme:
  \begin{equation*}
    \boxed{LC\frac{d^2u_2(t)}{dt^2}+RC\frac{du_2(t)}{dt}+u_2(t)=u_1(t)}
  \end{equation*}
  V Laplaceově transformaci platí:
  \begin{align*}
    \mathcal{L}\left[\frac{du_2(t)}{dt}\right]     &= pU_2(p)-u_2(0) \\
    \mathcal{L}\left[\frac{d^2u_2(t)}{dt^2}\right] &= p^2U_2(p)-pu_2(0)-\dot{u}_2(0),
  \end{align*}
  kde \(\dot{u}_2(0)=\left.\frac{du_2(t)}{dt}\right\rvert_{t=0}\). Při nulových počátečních
  podmínkách \(u_2(0) = 0\), \(\dot{u}_2(0) = 0\) a užitím Laplaceovy transformace přejde
  diferenciální rovnice na algebraickou rovnici:
  \begin{equation*}
    p^2LCU_2(p)+pRCU_2(p)+U_2(p)=U_1(p)
  \end{equation*}
  Odtud vyplývá \textbf{přenosová funkce} $H(p)=\frac{U_2(p)}{U_1(p)}$
  \begin{align}
    H(p) &=\dfrac{1}{p^2LC+pRC+1}                                   \nonumber \\
          &=\dfrac{1}{LC}\frac{1}{p^2+p\dfrac{R}{L}+\dfrac{1}{LC}}
          =\frac{Q(p)}{N(p)}                                        \label{tky:eq020}
  \end{align}
  K nalezení \textbf{impulsní odezvy} nejprve určíme póly přenosové funkce řešením rovnice
  $N(p)=0$
  \begin{align}
    p_{\infty_{12}} 
      &=\dfrac{\dfrac{R}{L}\pm\sqrt{\left(\dfrac{R}{L}\right)^2-\dfrac{4}{LC}}}{2}   \nonumber \\
      &=\frac{R}{2L}\pm\sqrt{\left(\frac{R}{2L}\right)^2-\frac{1}{LC}}          \label{tky:eq021}
  \end{align}
  a přenosovou funkci pak upravíme do tvaru
  \begin{equation*}
    H(p)=\frac{K}{(p-p_{\infty_1})\cdot(p-p_{\infty_2})}, \quad K=\frac{1}{LC}
  \end{equation*}
  kde \(K\) je násobná konstanta a \(p_{\infty_{1,2}}\) jsou její póly. 
  \begin{itemize}[leftmargin=12pt,noitemsep]
    \item Uvažujeme-li jednoduché póly a bude-li $R>2\sqrt{\frac{L}{C}}$ , potom z  rov.
          \ref{tky:eq021} vyplývají dva reálné různé póly. Přenosovou funkci tedy můžeme
          zapsat obecným tvarem:
          \begin{equation*}
            H(p)=\frac{K}{(p+a_1)\cdot(p+a_2)}=\frac{k_1}{p+a_1}+\frac{k_2}{p+a_2}
          \end{equation*}
          kde $p_{\infty_1}=-a_1,\, p_{\infty_2}=-a_2$, Rezidua  \(k_1\), \(k_2\) určíme z rov.
          \ref{tky:eq006}. 
          \begin{equation*}
            k_1=\frac{K}{a_2-a_1}, \quad k_2=\frac{K}{a_1-a_2}.
          \end{equation*}
          Impulsní odezvu pak vypočteme užitím rov. \ref{tky:eq023}.
          \begin{align*}
            h(t)&=\mathcal{L}^{-1}[H(p)]               \\
                &=\frac{K}{a_2-a_1}e^{-a_1t}+\frac{K}{a_1-a_2}e^{-a_2t}
          \end{align*}
    \item Když bude $R<2\sqrt{\frac{L}{C}}$, obdržíme dvojici komplexně sdružených pólů a
          přenosovou funkci může obecně zapsat takto:
          \begin{align*}
            H(p)&=\frac{K}{(p+a_1)\cdot(p+a_2)}            \\
                &=\frac{k_1}{p+a-\jmath b}+\frac{k_2}{p+a+\jmath b}
          \end{align*}
          kde $p_{\infty_1}=-a+\jmath b$, $p_{\infty_2}=-a-\jmath b$. Rezidua v pólech jsou dány
          výrazy $k_1=-\frac{\jmath K}{2b}$, $k_2=\frac{\jmath K}{2b}$. Impulzní odezvu opět
          určíme užitím rov. \ref{tky:eq023}.
          \begin{equation*}
            h(t) = \frac{Ke^{-at}}{b}
                    \left[\dfrac{\jmath\cdot\left(-e^{\jmath bt}+e^{-\jmath bt}\right)}{2}\right]
          \end{equation*}
          Výraz v závorce upravíme
          \begin{equation*}
            \left[\jmath\cdot\left(\underline{-\cos(bt)}-\jmath\sin(bt) + 
                                    \underline{\cos(bt)} -\jmath\sin(bt)
                              \right)
            \right]                                                     
          \end{equation*}
          konečně dostáváme hledaný vzah
          \begin{equation*}              
            h(t) = \frac{K}{b}e^{-at}\sin(bt)                                   
          \end{equation*}
  \end{itemize}
  
  Na obr. \ref{tky:fig009} je uvedena impulsní charakteristika uvaožovaného obvodu odpovídající
  hodnotám stavebních prvků: \(R=\qty{1}{\kohm}\), \(L=\qty{11.5}{\milli\henry}\),
  \(C=\qty{22.5}{\nano\farad}\). Výpis m-file \texttt{SAS\_exam\_02\_symb\_Hp\_solve.m} ukazuje
  symbolický způsob řešení operátorových obvodových rovnic pomocí \texttt{MATLABu}. Jde o filtr typu
  \textbf{dolní propust}, jehož přenosová funkce má tvar:
  $$H(p)= \frac{3.9506\cdot10^9}{p^2+8.8889\cdot10^4p+3.9506\cdot10^9}.$$

  Impulzní charakteristiku obdržíme dosazením do vztahu \ref{tky:eq005}
  \begin{align*}
    h(t)&=\frac{K}{b}e^{-at}\sin(bt)                                      \\
        &=8.8890\cdot10^4e^{-4.4444\cdot10^4t}\sin(4.4444\cdot10^4t).
  \end{align*}
      
    {\centering
    \captionsetup{type=figure}
    \luafigure[1]{tky_fig009.pdf}
    \captionof{figure}{Impulzní charakteristika}
    \label{tky:fig009}
    \par}
  
  Z hlediska analýzy obvodů v kmitočtové oblasti je výhodné sestavovat obvodové rovnice (metodami
  uzlových napětí a smyčkových proudů) přímo v operátorovém tvaru. Kirchhoffovy zákony pro uzavřenou
  smyčku a proudu do uzlu pak mají tvar $$\sum_{k=1}^{n}U_k(p) = 0, \qquad \sum_{k=1}^{n}I_k(p) =
  0.$$ Metodou uzlových napětí pro zapojení na obr. \ref{tky:fig008} obdržíme rovnice
  \begin{align}
    \frac{U_3(p)-U_1(p)}{R}+\frac{U_3(p)-U_2(p)}{pL} &=  0 \\
    pCU_2(p) + \frac{U_2(p)-U_3(p)}{pL}              &=  0 
  \end{align}
  Na rozdíl od \ref{tky:eq019} jde o algebraické rovnice, ze kterých eliminací uzlového napětí
  $U_3(p)$ vyplývá přenosová funkce \ref{tky:eq020} $$H(p) = \frac{U_2(p)}{U_1(p)} =
  \frac{1}{LC}\frac{1}{p^2+p\frac{R}{L} + \frac{1}{LC}}$$
  
  {\centering
    \captionsetup{type=figure}
    \luafigure[1]{tky_fig010.pdf}
    \captionof{figure}{Modulová, fázová charakteristika a skupinové zpoždění filtru}
    \label{tky:fig010}
    \par}    
  
  Dosazením za $p=\jmath\omega$ lze z přenosové funkce vyjádřit modulovou charakteristiku
  $H(\jmath\omega)$ a fázovou charakteristiku $\Phi(\omega)= \texttt{arg} H(j\omega)$. Skupinové
  zpoždění vyplývá ze vztahu \ref{tky:eq042}. Modulová, fázová charakteristika a skupinové zpoždění
  jsou na obr. \ref{tky:fig010}.
  
  Filtr má maximálně plochou modulovou charakteristiku přenosu. Mezní kmitočet propustného pásma je
  $f_p = 10 kHz$, při kterém je $\abs{H(j\omega_p)}= 0.707$. Tato hodnota odpovídá poklesu modulové
  charakteristiky o $3 dB$.
    
  %---------------------------------------------------------------
  \lstinputlisting[%
  style=luaMatlabStyle,
  caption={Výpis programu TKY\_exam\_03\_Hp.m k příkladu \ref{tky:exam002}.}
  ]{../src/TKY/matlab/SAS_exam_03_Hp.m}
  %--------------------------------------------------------------- 
  %---------------------------------------------------------------
  \lstinputlisting[%
  style=luaMatlabStyle,
  caption={Výpis programu TKY\_exam\_03\_Hp.m k příkladu \ref{tky:exam002}.}
  ]{../src/TKY/matlab/SAS_exam_02_symb_Hp_solve.m}
  %--------------------------------------------------------------- 
\end{tkyexam}