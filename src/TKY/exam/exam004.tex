% !TeX spellcheck = cs_CZ Lineární obvody a systémy - Jan Bičák  - strana 10 Popis spojitých systémů
%===================================================================================================
\begin{mdframed}[style=mdexam]
  \begin{example}\label{tky:exam004}
    Na diferenční rovnici \ref{tky:eq061} vypočítanou v příkladu \ref{tky:exam003} aplikujeme
    \(\mathcal{Z}\)-transformaci a vyjádříme přenosovou funkci diskrétního systému ve tvaru
    \begin{equation*}
      H(z) = \dfrac{U_2(z)}{U_1(z)}.
    \end{equation*}
    \noindent\textbf{Řešení:}
    Na diferenční rovnici \ref{tky:eq061} aplikujeme \(\mathcal{Z}\)-transformaci a dostaneme
    algebraickou rovnici
    \begin{align}
      \num{75.579367}U_2(z) &- \num{137.90481}z^{-1}U_2(z)                       \nonumber \\
                            &+ \num{63.325442}z^{-2}U_2(z) = U_1(z).             \label{tky:eq062}
    \end{align}
    Z rovnice \ref{tky:eq062} můžeme vyjádřit přenosovou funkci \(H(z)\) ve tvaru racionálně lomené
    funkce
    \begin{equation}\label{tky:eq063}
      H(z) = \dfrac{1}{\num{75.579367}-\num{137.90481}z^{-1}+\num{63.325442}z^{-2}}.
    \end{equation}
    
    {\centering
      \captionsetup{type=figure}
      \luafigure[1]{tky_fig013.pdf}
      \captionof{figure}{Impulsní charakteristika přenosové funkce \ref{tky:eq063}}
      \label{tky:fig013}
    \par}
    Obrázek \ref{tky:fig013} ukazuje impulsní odezvu diskrétního systému popsaného přenosovou funkcí
    \ref{tky:eq063} pro \(t\in\langle0,\num{3e-4}\rangle\) vykreslené pomocí matlabovské funkce
    \lstinline[style=luaMatlabText]!stem(x)!. Vzorkovací periodu zvolme \(T = \SI{2}{\us}\)

    Program využívá funkce \lstinline[style=luaMatlabText]!impz(a,b)!, která vrací impulsní odezvu
    přenosové funkce s koeficienty čitatele \(b\) a jmenovatelem \(a\). Funkce je součástí balíku 
    \emph{Signal Processing Toolbox}    
  \end{example} 

  %---------------------------------------------------------------
  \lstinputlisting[%
  style=luaMatlabStyle,
  caption={Výpis programu tky012.m k příkladu \ref{tky:exam004}.}
  ]{../src/TKY/matlab/tky012.m}
  %--------------------------------------------------------------- 
\end{mdframed}