% !TeX spellcheck = cs_CZ Lineární obvody a systémy - Jan Bičák  - strana 10 Popis spojitých systémů
%===================================================================================================
\begin{mdframed}[style=mdexam]
  \begin{example}\label{tky:exam004}
    V

    {\centering
      \captionsetup{type=figure}
      \luafigure[1]{tky_fig013.pdf}
      \captionof{figure}{Zapojení obvodu RLC.}
      \label{tky:fig008}
    \par}
    \noindent\textbf{Řešení:} Pro zapojení dle obrázku \ref{tky:fig008} získáme metodou uzlových
    napětí integrodiferenciální rovnice pro uzly \texttt{A} a \texttt{B}:
    \begin{gather*}
      \begin{align*}
        \shortintertext{uzel A:}
        \frac{u_3(t)-u_1(t)}{R}+\frac{1}{L}\int_0^t{[u_3(t)-u_2(t)]}\dd{\tau}+i_L(0_+) &= 0  \\
        \shortintertext{uzel B:}
        \frac{1}{L}\int_0^t[(u_2(t)-u_3(t))]\dd{\tau}+C\der{u_2}{t}-i_L(0_+)           &= 0
      \end{align*}
    \end{gather*}
    
    V Matlabu vypočítáme impulsní charakteristiku 
    
  \end{example} 

  %---------------------------------------------------------------
  \lstinputlisting[%
  style=luaMatlabStyle,
  caption={Výpis programu tky012.m k příkladu \ref{tky:exam004}.}
  ]{../src/TKY/matlab/tky012.m}
  %--------------------------------------------------------------- 
\end{mdframed}