% !TeX spellcheck = de_DE
%================ Kapitola 2: Unsere Familie ==============================================
\chapter{Lektion 2: Unsere Familie}\label{NJ:chap_N1_L2}

    \textbf{Unsere Familie ist ziemlich groß}. Darf ich sie vorstellen? Mein Vater ist 53 
    Jahre 
    alt, er arbeitet als Techniker. Seine Arbeit ist interessant, er kommt aber oft spät nach 
    Hause. Er wandert gern und bastelt auch viel.
    
    \textbf{Meine Mutter ist 49}, sie ist von Beruf Verkäuferin. Ihre Arbeit ist schwer und zu 
    Hause hat sie auch viel zu tun. Sie ist sehr fleißig, sie schafft immer alles. Und ihr Hobby? 
    Sie näht und strickt gern und geht auch schwimmen. Ich habe zwei Geschwister - einen Bruder und 
    eine Schwester. Meine Schwester Jana ist schon 24.  Sie ist verheiratet und wohnt jetzt in 
    Brno. Sie bekommt bald ein Kind. Dann werde ich Tante und Martin Onkel.
    
    Wer ist Martin? Das ist doch mein Bruder. Er ist Schüler, er besucht eine Fachschule.   
    Natürlich ist er noch ledig, aber er hat schon eine Freundin. Sie ist ganz nett. Martin ist ein 
    bisschen zu dick1, aber er treibt aktiv Sport: er spielt Fußball und Tennis, manchmal auch 
    Basketball und Volleyball. Die Schule findet er langweilig. Er lernt zwar leicht, aber er ist 
    faul.
    
    Also, jetzt kennt ihr schon meine Familie. Nein, einen Augenblick, mich kennt ihr doch noch 
    nicht! Ich heiße Petra, bin 21 Jahre alt, klein, schlank, blond und studiere Psychologie.    
    Ich habe einen Freund, aber wir heiraten noch nicht. Wir haben nämlich keine Zeit. Ich glaube 
    auch, ich bin noch zu jung. Was meint ihr?

  \section*{Slovní zásoba}

    \begin{widetext}
     \centering
%    \begin{table}[ht!] % L2_Wortschatz01.jpg, L2_Wortschatz02.jpg 
      \begin{tabular}{llll}  
        \hline
          e Familie          & rodina                & groß            & velký                 \\
          r Vater            & otec                  & stricken        & plést                 \\
          s Jahr             & rok                   & schwimmen       & plavat                \\
          alt                & starý, staře          & die Geschwister & sourozenci            \\
          arbeiten (als)     & pracovat (jako)       & r Bruder        & bratr                 \\
          e Arbeit           & práce e               & Schwester       & sestra                \\
          interessant        & zajímavý, -ě          & schon           & už, již               \\
          oft                & často                 & verheiratet     & ženatý, vdaná         \\
          spät               & pozdě, pozdní         & jetzt           & teď, nyní             \\
          s Haus             & dům                   & bald            & brzy                  \\
          zu Hause           & doma                  & s Kind          & dítě                  \\
          nach Hause         & domů                  & werden          & stát se (čím)         \\
          wandern            & chodit na             & e Freundin      & přítelkyně,           \\ 
                             & pěší výlety           &                 & kamarádka             \\
          gern               & rád, -a, -o           & e Tante         & teta                  \\
          bastel             & kutit r               & Onkel           & strýc, strýček        \\
          viel               & mnoho                 & erst            & teprve (časově)       \\
          e Mutter           & matka r               & Schüler         & žák                   \\
          r Beruf            & povolání,             & e Fachschule    & střední odborná       \\
                             & zaměstnání            &                 & škola                 \\
          e Verkäuferin      & prodavačka            & ledig           & svobodný (neženatý)   \\
          schwer             & těžký, těžce          & bekommen        & dostat                \\
          dann               & potom, pak            & ein bisschen    & trochu                \\
          tun, ich tue       & činit, konat, dělat   & zu              & příliš                \\
          fleißig            & pilný, -ě             & dick            & tlustý, -ě            \\
          immer              & vždy, stále           & r Sport         & sport                 \\
          schaffen           & stihnout, dokázat     & Sport treiben   & sportovat             \\
          alles              & všechno               & spielen         & hrát (si)             \\
          nähen              & šít                   & ganz            & docela, celý          \\
          nett               & milý, -e              & manchmal        & někdy, občas          \\
          finden             & najít                 & vorstellen      & představit            \\
                             & shledávat (jakým)     & klein           & malý                  \\
          \hline 
          langweilig         & nudný, -ě             & schlank         & 
          štíhlý                \\
          lernen             & učit se               & heiraten (4.p.) & ženit se, vdávat 
          se,  \\
                             &                       &                 & brát se 
                             (si)          \\
          zwar               & sice                  & nämlich         & 
          totiž                 \\
          leicht             & lehký, lehce          & e Zeit          & čas, 
          doba             \\
          faul               & líny, -e              & glauben         & domnívat se, 
          věřit    \\
          also               & tedy, tak             & jung            & mladý, 
          -ě             \\
          r Augenblick       & okamžik               & meinen          & mínit, 
          myslet         \\
        \hline
       \end{tabular}
%       \caption*{ }
%    \end{table}
     \end{widetext}
    
    \begin{table}[ht!] % L2_Wortschatz03.jpg
      \begin{tabular}{llll} 
        \hline 
        r Sohn            & syn        & e Tochter     & dcera        \\
        r Großvater       & dědeček    & e Großmutter  & babička      \\
        r Opa             & dědeček    & e Oma         & babička      \\
        r Schwager        & švagr      & e Schwägerin  & švagrová     \\
        r Neffe           & Synovec    & e Nichte      & neteř        \\
        r Cousin          & bratranec  & e Cousine     & sestřenice   \\
        r Schwiegervater  & tchán & e Schwiegermutter  & tchyně       \\
        die Eltern (jen mn.č.)    & rodiče  & die Schwiegereltern  & tchán a tchyně  \\
        \hline
      \end{tabular}
      \caption*{ }
    \end{table}
    \newpage
    \subsection*{Časování slovesa „haben" (mít)}
      \begin{table}[ht!]   % L2_Grammatik01.jpg
        \centering
        \begin{tabular}{llllll}
          \hline
           \multicolumn{3}{c}{\texttt{j. č.}} &
           \multicolumn{3}{c}{\texttt{mn. č.}}                                 \\  
          \hline
            ich         & \textbf{habe} & mám & wir  & \textbf{haben} & máme  \\
            du          & \textbf{hast} & máš & ihr  & \textbf{habt}  & máte  \\
            er, sie, es & \textbf{hat}  & má  & sie  & \textbf{haben} & mají  \\
                        &               &     & Sie  & \textbf{haben} & máte  \\
          \hline
        \end{tabular}
        \caption*{ }
      \end{table}
  
    \subsection*{Přivlasťnovací zájména}
      \begin{table}[ht!]   % L2_Grammatik02.jpg
        \centering
        \begin{tabular}{llll}
          \hline
          \multicolumn{2}{c}{\texttt{j. č.}} &
          \multicolumn{2}{c}{\texttt{mn. č.}}                       \\  
          \hline
            mein, meine, mein   & můj,  má,  mé    & unser, unsere, unser  & náš, naše  \\
            dein, deine, dein   & tvůj, tvá, tvé   & euer,  euere,  eure   & váš, vaše  \\
            sein, seine, sein   & jeho             & ihr,   ihre,   ihr    & jejich     \\
            ihr,  ihre,  ihr    & její             & Ihr,   Ihre,   Ihr    & Váš, Vaše  \\
            sein, seine, sein   & jeho             &                       &            \\
          \hline
        \end{tabular}
        \caption*{ }
      \end{table}
    
      \begin{itemize}\addtolength{\itemsep}{-0.5\baselineskip} % \label{NJ:fig_L2_Grammatik03}
        \item Kennen Sie meinen (unseren) Vater? \emph{Znáte mého (našeho) otce?}
        \item Kennen Sie meine (unsere) Mutter? \emph{Znáte mou (naši) matku?} 
        \item Kennen Sie mein (unser) Kind? \emph{Znáte me (naše) dítě?}
      \end{itemize}
      Přivlastňovací zájmena se skloňují v jednotném čísle jako neurčity člen. Použijeme-li 
      přivlastňovací zájmeno, člen vynecháme. Pokud u tvaru „\emph{euer}" následuje další 
      samohláska, vypouští se druhé „e": „\emph{eure}"
  
    \subsection*{Zápor (nein, nicht, kein)}  % L2_Grammatik04.jpg
      \begin{table}[ht!]  
        \begin{tabular}{ll} 
          Nein, wir kommen nicht. & Ne, nepřijdeme. \\
          Nein, er hat kein Kind. & Ne, nemá dítě.
        \end{tabular}
        \caption*{ }
      \end{table}
      Německá záporná věta může mít \textbf{jen jeden zápor}. Záporka „\emph{nein}", která je 
      opakem „\emph{ja}", stojí vždy před větou, není její součástí a je vždy oddělena čárkou.
      
      \begin{table}[ht!]   
        \begin{tabular}{ll} 
          Es ist nicht schwer.        & Není to těžké.        \\ 
          Petr ist nicht aus Hamburg. & Petr není z Hamburku. \\
          Er kennt die Stadt nicht.   & Nezná to město.       \\
          Begleitest du mich nicht?   & Nedoprovodíš mě?
        \end{tabular}
        \caption*{ }
      \end{table}
      \textbf{Slovesný zápor} se tvoří slovem „\emph{nicht}", které \emph{stojí vždy} za určitým 
      slovesem, někdy až na konci věty, často před popiraným slovem.
  
      \begin{table}[ht!]   
        \begin{tabular}{ll} 
          Hast du einen Bruder?   & Máš bratra?  \\
          Ich habe keinen Bruder. & Nemám bratra.\\
          Hat er ein Hobby?       & Má nějakého koníčka?\\
          Er hat kein Hobby.      & Nemá žádného koníčka.\\
          Habt ihr Zeit?          & Máte čas?\\
          Wir haben keine Zeit.   & Nemáme čas\\
        \end{tabular}
        \caption*{ }
      \end{table}
      % L2_Grammatik05.jpg
      Záporným zájmenem „\textbf{kein}, \textbf{keine}, \textbf{kein}" se vyjadřuje zápor u 
      podstatného jména „Kein" (žádný) se skloňuje v jedn. čísle jako neurčitý člen. Užívá 
      se ho zpravidla tam kde v kladném případě stálo podst. jméno s neurčitými členem nebo 
      bez členu.
    
    \subsection*{Základní číslovky}
      \begin{table}[ht!]   % L2_Grammatik06.jpg
        \centering
        \begin{tabular}{llllll}
          \hline
            0  & null     & 14 & vierzehn       & 70        & siebzig          \\
            1  & eins     & 15 & fünfzehn       & 80        & achtzig          \\
            2  & zwei     & 16 & sechzehn       & 90        & neunzig          \\
            3  & drei     & 17 & siebzehn       & 100       & (ein)hundert     \\
            4  & vier     & 18 & achtzehn       & 101       & (ein)hunderteins \\
            5  & fünf     & 19 & neunzehn       & 200       & zweihundert      \\ 
            6  & sechs    & 20 & zwanzig        & 300       & dreihundert      \\
            7  & sieben   & 21 & einundzwanzig  & 1000      & (ein)tausend     \\ 
            8  & acht     & 22 & zweiundzwanzig & 1001      & (ein)tausendeins \\
            9  & neun     & 23 & dreiundzwanzig & 2000      & zweitausend      \\
            10 & zehn     & 30 & dreißig        & 3000      & dreitausend      \\
            11 & elf      & 40 & vierzig        & 10 000    & zehntausend      \\
            12 & zwölf    & 50 & fünfzig        & 100 000   & hunderttausend   \\
            13 & dreizehn & 60 & sechzig        & 1 000 000 & eine Million     \\
          \hline
        \end{tabular}
        \caption*{ }
      \end{table}

    \subsection*{Přítomný čas místo budoucího}
      V němčině se často používá tvaru přítomného času i pro vyjádření budoucnosti
      \begin{table}[ht!]   % L2_Grammatik07.jpg
        \centering
        \begin{tabular}{ll} 
          Er kommt bald.      & Přijde brzy.           \\
          Was machst du noch? & Co budeš ještě dělat?  \\
          Dann bin ich Onkel. & Potom budu strýčkem.   \\
        \end{tabular}
        \caption*{ }
      \end{table}
      
    \subsection*{Bezespojkové věty}
      \begin{table}[ht!]   % L2_Grammatik08.jpg
        \centering
        \begin{tabular}{ll} 
          Sie meinen, ich bin noch zu jung. & \emph{Myslí si, že jsem ještě přílis mladý(á).} \\
          Ich glaube, er kommt bald.        & \emph{Myslím, že přijde brzy.}  \\
          Er sagt, er studiert Geschichte.  & \emph{Říká, ze studuje historii.}  \\
        \end{tabular}
        \caption*{ }
       \end{table} 
       Po slovesech „sagen, meinen, glauben" apod. je v němčině možné přiřadit další větu beze 
       spojky. V tom případě se používá slovosled samostatné oznamovací věty. Uvozující věta však 
       musí byt kladná.


  \begin{itemize}\addtolength{\itemsep}{-0.5\baselineskip} % L2_Redensart01.jpg
    \item Darf ich sie vorstellen?         Mohu (smím) ji představit?
    \item Wie alt ist er?                  Kolik mu je (let)?
    \item Er ist 50 (Jahre alt).           Je mu 50 (let).
    \item Sie ist (von Beruf) Verkäuferin. Je (povoláním) prodavačka.
    \item Ich habe viel zu tun.            Mám mnoho práce.
    \item Sie bekommt ein Kind.            Čeká / bude mít dítě.
    \item Ich werde Tante.                 Budu / stanu se tetou.
    \item Er ist Schüler.                  Chodí do školy.
    \item Er hat eine Freundin.            Má přítelkyni / dívku.  
    \item Er findet es langweilig.         Zdá se mu to nudné.
    \item Einen Augenblick!                Okamžik! 
  \end{itemize}