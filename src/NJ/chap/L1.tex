% !TeX spellcheck = cs_CZ
%================ Kapitola 1: Sie sind aber neugierig! =============================================
% \let\cleardoublepage\clearpage  % remove blank pages coming between two chapters 
\chapter{Lektion 1: Sie sind aber neugierig!}\label{NJ:chap_N1_L1}

\section*{Slovní zásoba}
  Ve slovníčku se užívají zkrácené tvary členu: \textbf{der} \(\rightarrow\) r, \textbf{die} 
  \(\rightarrow\) e, \textbf{das} \(\rightarrow\) s.  
  
  \begin{widetext}
%  \begin{table}[ht!]   % L1_Wortschatz01.jpg
    \centering
    \begin{tabular}{llll}
      \hline
      aber        & ale               & was               & co                                \\
      neugierig   & zvědavý, -ě       & machen            & dělat                             \\
      wie         & jak, jako, jaký   & hier              & zde, tady                         \\
      hei{\ss}en  & jmenovat se       & fahren            & jet                               \\
      bitten      & prosit            & e Universit{\"a}t & univerzita                        \\
      und         & a                 & auch              & také, i                           \\
      woher       & odkud             & e Straßenbahn     & tramvaj                           \\
      kommen      & přijet, přicházet & begleiten         & doprovodit, doprovázet            \\
      aus         & z                 & dich              & tě, tebe                          \\
      (s) M{\"u}nchen  & Mnichov      & nat{\"u}rlich     & přirozeně, -ě,                    \\
                  &                   &                   & samozřejmě                        \\
      freuen      & (po)těšit         & noch              & jestě                             \\
      mich        & mě, mne           & r Freund          & přítel                            \\
      (s) Prag    & Praha             & studieren         & studovat                          \\
      nein        & ne, nikoli        & eigentlich        & vlastně                           \\
      wo          & kde               & e Geschichte      & dějiny, dějepis,                  \\
                  &                   &                   & historie, příběh                  \\
      liegen      & ležet; být        & e Schule          & škola                             \\
      weit        & daleko            & besuchen          & navštívit, navštěvovat            \\
      von         & od, z, o          & r Name            & jméno                             \\
      von hier    & odtud             & wohnen            & bydlet                            \\
      ja          & ano               & in                & v, do, za (časově)                \\
      ziemlich    & dost, značný, -ě  & (s) Graz          & Štýrskě Hradec                    \\
      gehen       & jít, chodit       & sagen             & říci, říkat                       \\
      wer         & kdo               & mein, meine, mein & můj, má, mé‚                      \\
      e Frau      & paní, slečna      & e Stadt           & město                             \\
      dort        & tam               & kennen            & znát                              \\
      doch        & přece             & sehr              & velmi                             \\
      danken      & (po)děkovat       & sch{\"o}n         & pěkný, -ě                         \\
      gut         & dobrý, dobře      & s M{\"a}dchen     & dívka, děvče \\
      \hline
    \end{tabular}
%    \caption*{ }
%  \end{table}
  \end{widetext}
  
  \subsection*{Vazby}
    \begin{table}[ht!]   % L1_Redensart01.jpg
      \begin{tabular}{ll}
        bitte                      & prosím                     \\
        Woher kommen Sie?          & Odkud jste?                \\
        Das stimmt.                & To je pravda. To souhlasí  \\
        Verzeihung!                & Promiňte!                  \\
        (Es) freut mich!           & Těší mě!                   \\
        Wo liegt es?               & Kde to je?                 \\
        danke                      & děkuji                     \\
        Wie gehťs? (Wie geht es?)  & Jak se daří?               \\
        Ich fahre zur Universität. & Jedu na univerzitu.        \\
        Ich gehe zur Straßenbahn   & Jdu na tramvaj.            \\
        Darf ich?                  & Smím? / Mohu?              \\
        Er geht zur Schule.        & Chodí/ jde do školy. 
      \end{tabular}
      \caption*{ }
    \end{table}


\section*{Gramatika}
  \subsection*{Osobní zájména v 1. pádě}
    \begin{table}[ht!]   % L1_Grammatik01.jpg
      \begin{tabular}{lllll}
        \hline
              & \multicolumn{2}{l}{\texttt{j. č.}}
              & \multicolumn{2}{l}{\texttt{mn. č.}}     \\  
        \hline
        1.os. & ich & já  & wir & my          \\
        2.os. & du  & ty  & ihr & vy          \\
        3.os. & er  & on  & sie & oni         \\
              & sie & ona & Sie & vy (vykání) \\
              & es  & ono &     &             \\
        \hline
      \end{tabular}
      \caption*{ }
    \end{table}
    Při \emph{vykání} jedné nebo více osobám se používá \emph{3. osoby množného čísla}, tedy 
    zájmena \emph{Sie}, které se píše \emph{vždy} s velkým písmenem.
          
  \subsection*{Časování pravidelných sloves v přítomném čase}
    \begin{table}[ht!]   % L1_Grammatik01.jpg
      \begin{tabular}{llll}
        \hline
        \multicolumn{4}{c}{\texttt{machen \(\rightarrow\) dělat}}     \\
        \hline
        ich mache & dělám & wir machen & děláme  \\
        du machst & děláš & ihr macht  & děláte  \\
        er macht  & dělá  & sie machen & dělají  \\
        sie macht & dělá  & Sie machen & děláte  \\
        es macht  & dělá  &            &         \\
        \hline
      \end{tabular}
      \caption*{ }
    \end{table}

    \begin{table}[ht!]   % L1_Grammatik01.jpg
      \begin{tabular}{llll}
        \hline
        \multicolumn{4}{c}{\texttt{begleiten \(\rightarrow\) doprovázet}}     \\
        \hline   
        ich begleite    & doprovázím & wir begleiten  & doprovázíme \\
        du begleitest   & doprovázíš & ihr begleitet  & doprovázíte \\
        er begleitet    & doprovází  & sie begleiten  & doprovázejí \\
        sie begleitet   & doprovází  & Sie begleiten  & doprovázíte \\
        es begleitet    & doprovází  &                &             \\ 
        \hline
      \end{tabular}
      \caption*{ }
    \end{table}
    
    Infinitiv slovesa končí převážně na \emph{-en}, jen v málo případech na \emph{-n}. Slovesný 
    kmen získáme odtržením infinitivní koncovky. Končí-li kmen na \emph{-t} (begleit-) nebo 
    \emph{-d}, vsouvá se pro snazší výslovnost mezi kmen a koncovku hláska \emph{-e-}, tj. v 2. a 
    3. osobě jednotného čísla a ve 2. osobě množného čísla: du begleitest, er begleitet, ihr 
    begleitet.

    U sloves, kde končí kmen na sykavku (hei{\ss}-), se přidává v 2. osobě jednotného čísla 
    pouze koncovka \emph{-t}. 2. a 3. osoba jednotného čísla pak mají shodný tvar: du hei{\ss}t, er 
    hei{\ss}t.
    
    Časujte ve všech osobách následující slovesa: \textbf{bitten} - prosit, \textbf{hei{\ss}en} - jmenovat se:  
    \begin{table}[ht!]   % 
      \begin{tabular}{llll}
        \hline
        \multicolumn{2}{c}{\texttt{bitten \(\rightarrow\) prosit}} & 
        \multicolumn{2}{c}{\texttt{hei{\ss}en \(\rightarrow\) jmenovat se}}     \\
        \hline   
        ich bitte         & wir bitten      & ich hei{\ss}e        & wir hei{\ss}en     \\
        du bittest        & ihr bittet      & du hei{\ss}t         & ihr hei{\ss}t      \\
        er/sie/es bittet  & sie/Sie bitten  & er/sie/es hei{\ss}t  & sie/Sie hei{\ss}en \\
        \hline
      \end{tabular}
      \caption*{ }
    \end{table}

  \subsection*{Časování slovesa „sein" (být)}
    \begin{table}[ht!]   % L1_Grammatik02.jpg
      \begin{tabular}{llll}
        \hline
          \multicolumn{2}{l}{\texttt{j. č.}}  & 
          \multicolumn{2}{l}{\texttt{mn. č.}}  \\  
        \hline
         ich bin  & jsem & wir sind & jsme          \\
         du  bist & jsi  & ihr seid & jste          \\
         er  ist  & je   & sie sind & jsou          \\
         sie ist  & je   & Sie sind & jste (vykání) \\
         es  ist  & je   &     &                    \\
        \hline
      \end{tabular}
      \caption*{ }
    \end{table}
    
    Ich \underline{bin} Gabi. Er \underline{ist} Rolf. Wir \underline{sind} in Prag. Sie 
    \underline{sind} in Berlin. Du \underline{bist} Heike. Ihr \underline{seid} in Brno. Sie 
    \underline{sind} hier. 


  \subsection*{Člen a podstatné jméno v 1. pádě}
    \begin{table}[ht!]   % L1_Grammatik03.jpg
      \begin{tabular}{ll}
        Dort ist eine Frau.          & Tam je nějaká paní. \\
        Die Frau heißt Jutta Klein.  & Ta paní se jmenuje Jutta Kleinová.   \\
      \end{tabular}
      \caption*{ }
    \end{table}
    \begin{itemize}\addtolength{\itemsep}{-0.5\baselineskip}
      \item \textbf{Člen určitý} označuje věci známé nebo již zmíněné; do češtiny jej lze někdy 
            přeložit ukazovacím zájmenem „ten, ta, to".
      \item \textbf{Člen neurčitý} označuje věci neznámé, dosud nezmíněné; do češtiny jej lze 
            někdy přeložit neurčitým zájmenem „nějaký", případně číslovkou „jeden".
    \end{itemize}

    Podstatná jména se v němčině používají zpravidla se členem a píší se vždy velkým písmenem. 
    \textbf{Rod podstatných jmen} bývá často od češtiny odlišný: die Stadt - město, der Name - 
    jméno, das M{\"a}dchen - dívka atd.
    \begin{itemize}\addtolength{\itemsep}{-0.5\baselineskip} % L1_Grammatik04
      \item Ich heiße Marek. Sind Sie aus Prag?
      \item Das ist Frau Sommer. Ich studiere Medizin.
    \end{itemize}
    U podstatných jmen se někdy člen vynechává. Je to např. před osobními jmény, tituly, názvy 
    měst, u oborů studia apod. a v některých ustálených vazbách. Na tato podstatná jména je 
    průběžně upozorňováno jednak v gramatice, jednak u vazeb.

  \subsection*{Pořádek slov ve větě}  %L1_Grammatik05.jpg
    \begin{enumerate}
       \item \textbf{Slovosled v oznamovací větě:}
             \begin{table}[ht!]   
               \hspace*{2em}
               \begin{tabular}{llll}
                 Er           & studiert & in Prag.        & Studuje v Praze.                  \\
                 Eva aus Köln & studiert & auch in Prag.   & Eva z Kolína studuje také v Praze.\\
                 In Prag      & studiert & sie Geschichte. & V Praze studuje historii.         \\
               \end{tabular}
               \caption*{ }
             \end{table}
             
            Sloveso stojí v německé oznamovací větě vždy jako \textbf{2. větný člen}. Oznamovací 
            věta může mít \emph{přímý} nebo \emph{nepřímý} pořádek slov:
            \begin{itemize}
             \item \textbf{Přímý pořádek}:\newline
                   \emph{Podmět - sloveso - ostatní větné členy}
             \item \textbf{Nepřímý pořádek}:\newline
                   \emph{Zdůrazněný větný člen - sloveso - podmět - ostatní větné členy} 
          \end{itemize}
      \item \textbf{Slovosled v tázací větě:}
        \begin{table}[ht!] 
          \hspace*{2em} 
          \begin{tabular}{llll}
            Was machst du hier?  & Co tady děláš? \\
            Sind Sie aus Prag?   & Jste z Prahy?  \\
          \end{tabular}
          \caption*{ }
        \end{table}
         
         V \textbf{doplňovací otázce} (otázka začínající tázacím zájmenem) stojí sloveso 
         rovněž jako 2. větný člen. Ve \textbf{zjišťovací otázce} (otázka, na kterou je možno 
         odpovědět ano nebo ne) stojí sloveso na začátku věty, pak následuje podmět a teprve za ním 
         ostatní větné členy.
    \end{enumerate}
  
  \newpage
  \subsection*{Cvičení}
    \begin{example}\textbf{Fill in the correct ending:}\newline
      Ich geh\textbf{e} in die Schule. - Er komm\textbf{t} aus Spanien. - Ihr lern\textbf{t} 
      Deutsch. - Wir sprech\textbf{en} Englisch. - Du schreib\textbf{st} einen Brief.
    \end{example}

    \begin{example}\textbf{Fill in the correct personal pronoun:}\newline
     \textbf{Ich} gehe nach Hause. - \textbf{Ihr} seid aus Italien. - \textbf{Sie} (die Frau) 
     hei{\ss}t Sabine. - \textbf{Du} wohnst in Berlin. - Meine Mutter und ich, \textbf{wir} lernen 
     Deutsch.     
    \end{example}
    
    \begin{example}\textbf{Fill in the correct verbform:}\newline
      Wir \textbf{sind} (sein) in Wien. - Ihr arbeitet (arbeiten) bei einer gro{\ss}en Firma. - 
      Herr M{\"u}ller, Sie \textbf{kommen} (kommen) aber sp{\"a}t. - Mein Bruder \textbf{lebt} 
      (leben) in M{\"u}nchen. - Ich \textbf{schreibe} (schreiben) ein Buch. 
l    \end{example}

    \begin{example}\textbf{Přeložte:}\newline
      Co tady děláš? \emph{Was machst du hier?} - Jak se jmenuješ? \emph{Wie hei{\ss}t du?} - Kde 
      vlastně bydlíte? \emph{Wo wohnen Sie eigentlich.} - Bydlíme tady. \emph{Wir wohnen hier}. - 
      Kde je paní Sommerová? \emph{Wo ist frau Sommerova?} - (Ona) je přece tady. \emph{Sie ist 
      doch hier} - Tady jste! \emph{Hier sind Sie ja!} - Co říkáte? \emph{Was sagen Sie?} - 
      Přijdete také? \emph{Kommen Sie auch?} - Jste velmi zvědavý \emph{Sie sind sehr neugierig}. - 
      To je pravda. \emph{Das stimmt} - Jdeš na tramvaj? Doprovodím tě. Smím? \emph{Gehst du zur 
      Straßenbahn? Ich begleite dich. Darf ich?} - Kdo je to? Můj přítel z Prahy.\emph{Wer ist das? 
      Mein Freund aus Prag.} -  Kde je (leží) Praha? Znají mě. Studují také historii. \emph{Wo 
      liegt Prag? Kennen Sie mich? Sie studieren auch Geschichte.}
    \end{example}